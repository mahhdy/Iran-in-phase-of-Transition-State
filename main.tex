%══════════════════════════════════════════════════════════════════════════════
% main.tex — فایل اصلی کتاب
% از بحران تا بالندگی: طرح جامع تأسیس دموکراسی پایدار
% نویسنده: مهدی سالم | ریچموندهیل | ۱۴۰۴
%══════════════════════════════════════════════════════════════════════════════

\documentclass[12pt,a4paper,twoside,openright]{book}

% بارگذاری تنظیمات
%══════════════════════════════════════════════════════════════════════════════
% preamble.tex — تنظیمات و پکیج‌های کتاب (بازنویسی شده با حفظ ۱۰۰٪ زیرساخت پکیج‌ها)
% از بحران تا بالندگی: طرح جامع تأسیس دموکراسی پایدار
% نویسنده: مهدی سالم | ریچموندهیل | ۱۴۰۴
%══════════════════════════════════════════════════════════════════════════════

%──────────────────────────────────────────────────────────────────────────────
% پکیج‌های اصلی
%──────────────────────────────────────────────────────────────────────────────
\usepackage{xcolor}
\usepackage{amsmath,amssymb}
\usepackage{fontspec}
\usepackage{geometry}
\usepackage{graphicx}
\usepackage{tikz}
\usepackage{pgfplots}
\usepackage{pgf-pie} 
\usepackage{tcolorbox}
\usepackage{booktabs}
\usepackage{array}
\usepackage{colortbl}
\usepackage{multirow}
\usepackage{longtable}
\usepackage{tabularx}
\usepackage{enumitem}
\usepackage{fancyhdr}
\usepackage{pdflscape}
\usepackage{hyperref}
\usepackage{float}
\usepackage{caption}
\usepackage{subcaption}
\usepackage{setspace}
\usepackage{titlesec}
\usepackage{etoolbox}

%──────────────────────────────────────────────────────────────────────────────
% تنظیمات فواصل (کاهش فاصله بین بخش‌ها و صفحات سفید)
%──────────────────────────────────────────────────────────────────────────────
\titlespacing*{\chapter}{0pt}{30pt}{20pt}
\titlespacing*{\section}{0pt}{15pt}{10pt}
\titlespacing*{\subsection}{0pt}{10pt}{5pt}
\patchcmd{\chapter}{\if@openright\cleardoublepage\else\clearpage\fi}{\clearpage}{}{}

%──────────────────────────────────────────────────────────────────────────────
% تنظیمات صفحه
%──────────────────────────────────────────────────────────────────────────────
\geometry{
    a4paper,
    top=2.5cm,
    bottom=2.5cm,
    left=2.5cm,
    right=3cm,
    headheight=15pt,
    bindingoffset=0.5cm
}

%──────────────────────────────────────────────────────────────────────────────
% کتابخانه‌های TikZ و pgfplots
%──────────────────────────────────────────────────────────────────────────────
\usetikzlibrary{
    shapes.geometric, 
    arrows.meta, 
    positioning, 
    calc, 
    decorations.pathreplacing, 
    backgrounds,
    fit,
    matrix,
    chains,
    scopes,
    shadows
}
\pgfplotsset{compat=1.18}
\usepgfplotslibrary{polar}

%──────────────────────────────────────────────────────────────────────────────
% پالت رنگی جدید (ویژه تم مدرن) + رنگ‌های قدیمی جهت عدم خطای کامپایل نمودارهای تیکز
%──────────────────────────────────────────────────────────────────────────────
% ۱. رنگ‌های پایه‌ای جدید
\definecolor{PrimaryDeep}{HTML}{1A237E}    % نِیوی - آبی عمیق
\definecolor{PrimaryMid}{HTML}{283593}     % آبی متوسط
\definecolor{AccentGold}{HTML}{F9A825}     % طلایی مدرن
\definecolor{AccentRed}{HTML}{B71C1C}      % قرمز عمیق
\definecolor{NeutralLight}{HTML}{ECEFF1}   % خاکستری بسیار روشن
\definecolor{TextMain}{HTML}{212121}       % متن اصلی
\definecolor{NeutralDark}{HTML}{37474F}    
\definecolor{NeutralMid}{HTML}{90A4AE}     

% رنگ‌های ساخت باکسبندی جدید
\definecolor{BgCool}{HTML}{E8EAF6}         
\definecolor{BgWarm}{HTML}{FFF8E1}         
\definecolor{AccentAmber}{HTML}{FF8F00}    
\definecolor{BgTeal}{HTML}{E0F2F1}         
\definecolor{AccentTeal}{HTML}{00695C}     

% ۲. دوره‌های تاریخی
\definecolor{EraAncient}{HTML}{4E342E}     
\definecolor{EraMedieval}{HTML}{37474F}    
\definecolor{EraEarlyMod}{HTML}{1565C0}    
\definecolor{EraModern}{HTML}{2E7D32}      
\definecolor{EraContemp}{HTML}{6A1B9A}     
\definecolor{EraPostmod}{HTML}{AD1457}     

% ۳. رنگ‌های قدیمی (برای جلوگیری از خطای undefined color در کدهای قبلی)
\definecolor{bleurepublique}{HTML}{002147} 
\definecolor{goldphoenix}{HTML}{C5A059}    
\definecolor{bleulight}{HTML}{F0F4F8}      
\definecolor{goldlight}{HTML}{F9F4E8}      
\definecolor{rougerevolution}{HTML}{A020F0} 
\definecolor{vertnapoleon}{HTML}{006400}   
\definecolor{grislight}{HTML}{F5F5F5} 
\definecolor{gris}{RGB}{128,128,128}

\definecolor{bleumid}{RGB}{100,149,237}
\definecolor{rougemid}{RGB}{220,100,100}
\definecolor{vertmid}{RGB}{100,180,100}
\definecolor{violetmid}{RGB}{180,140,200}
\definecolor{orroyalmid}{RGB}{238,201,0}

\definecolor{phase1}{RGB}{70,130,180}
\definecolor{phase2}{RGB}{60,179,113}
\definecolor{phase3}{RGB}{255,165,0}
\definecolor{phase4}{RGB}{147,112,219}
\definecolor{phase5}{RGB}{220,20,60}

\definecolor{DemocracyBlue}{RGB}{41,128,185}
\definecolor{SuccessGreen}{RGB}{39,174,96}
\definecolor{WisdomGold}{RGB}{241,196,15}
\definecolor{WarningRed}{RGB}{231,76,60}

\colorlet{lightSuccessGreen}{SuccessGreen!70}
\colorlet{lightWisdomGold}{WisdomGold!70}
\colorlet{darkyellow}{yellow!80!black}
\colorlet{darkgreen}{green!60!black}
\colorlet{rougelight}{rougerevolution!15}
\colorlet{vertlight}{vertnapoleon!15}
\colorlet{violetlight}{violetmid!15}
\colorlet{violetempire}{violetmid!80!black}
\colorlet{orroyallight}{goldphoenix!15}
\colorlet{orroyal}{goldphoenix}
\colorlet{bleunight}{bleurepublique!90!black}
\colorlet{golddark}{goldphoenix!80!black}

\definecolor{chart1}{RGB}{31,119,180}
\definecolor{chart2}{RGB}{255,127,14}
\definecolor{chart3}{RGB}{44,160,44}
\definecolor{chart4}{RGB}{214,39,40}
\definecolor{chart5}{RGB}{148,103,189}

%──────────────────────────────────────────────────────────────────────────────
% استایل‌های پایه جعبه‌های محتوایی
%──────────────────────────────────────────────────────────────────────────────
\tcbuselibrary{skins, breakable}

\tcbset{
  mybox/.style={
    enhanced, breakable,
    colback=BgCool, colframe=PrimaryMid,
    fonttitle=\bfseries\large,
    attach boxed title to top right={yshift=-2mm, xshift=-3mm},
    boxed title style={colback=PrimaryMid, colframe=PrimaryDeep},
    top=4mm, bottom=4mm,
    before skip=12pt, after skip=12pt,
  },
  defbox/.style={
    enhanced, breakable,
    colback=BgWarm, colframe=AccentGold,
    fonttitle=\bfseries,
    left=4mm, right=4mm,
    borderline west={3pt}{0pt}{AccentAmber},
    sharp corners,
    before skip=12pt, after skip=12pt,
  },
  wavebox/.style={
    enhanced, breakable,
    colback=BgTeal, colframe=AccentTeal,
    fonttitle=\bfseries,
    rounded corners,
    drop shadow,
    before skip=12pt, after skip=12pt,
  },
  quotebox/.style={
    enhanced, breakable,
    colback=NeutralLight, colframe=NeutralDark,
    fontupper=\itshape,
    left=8mm,
    borderline west={4pt}{0pt}{NeutralMid},
    sharp corners=south,
    before skip=12pt, after skip=12pt,
  },
  enemybox/.style={
    enhanced, breakable,
    colback=red!5, colframe=AccentRed,
    fonttitle=\bfseries,
    borderline west={3pt}{0pt}{AccentRed},
    before skip=12pt, after skip=12pt,
  }
}

%──────────────────────────────────────────────────────────────────────────────
% نگاشت باکس‌های قبلی به استایل‌های جدید
%──────────────────────────────────────────────────────────────────────────────
\newtcolorbox{kholasebox}[1][]{
    mybox,
    title=#1
}

\newtcolorbox{naghlbox}[1][]{
    quotebox,
    title=#1
}

\newtcolorbox{olgoobox}[1][]{
    wavebox,
    title={\hfill \textbf{الگو و درس}},
    #1
}

\newtcolorbox{enghelabbox}[1][]{
    enemybox,
    title={\hfill \textbf{هشدار}},
    halign title=right,    
    #1
}

\newtcolorbox{tahlilbox}[1][]{
    defbox,
    colframe=PrimaryMid,
    borderline west={3pt}{0pt}{PrimaryDeep},
    title={\hfill \textbf{تحلیل}},
    #1
}

\newtcolorbox{noktebox}[1][]{
    defbox,
    colback=NeutralLight,
    colframe=NeutralMid,
    borderline west={3pt}{0pt}{NeutralDark},
    #1
}

\newtcolorbox{casebox}[2][]{
    mybox,
    colback=BgCool,
    colframe=PrimaryDeep,
    title={\hfill \textbf{مورد مطالعاتی: #2}},
    #1
}

\newtcolorbox{databox}[1][]{
    defbox,
    title={\hfill \textbf{داده‌های کلیدی}},
    colback=white,
    colframe=PrimaryDeep,
    borderline west={3pt}{0pt}{AccentGold},
    #1
}

%──────────────────────────────────────────────────────────────────────────────
% دستورات سفارشی جداول
%──────────────────────────────────────────────────────────────────────────────
\newcommand{\tablemark}{\cellcolor{BgWarm}}
\newcommand{\headmark}{\cellcolor{PrimaryDeep}\color{white}\bfseries}
\newcommand{\rowmark}{\rowcolor{BgCool}}

% تنظیمات صحیح تراز متن در فارسی (L=چپ، R=راست)
\newcolumntype{R}[1]{>{\raggedleft\arraybackslash}p{#1}}
\newcolumntype{C}[1]{>{\centering\arraybackslash}p{#1}}
\newcolumntype{L}[1]{>{\raggedright\arraybackslash}p{#1}}
\newcolumntype{Y}{>{\raggedleft\arraybackslash}X} 
\newcolumntype{Z}{>{\centering\arraybackslash}X} 

% اجبار جدول به راست به چپ (RTL)
\pretocmd{\tabular}{\setRTL}{}{}
\pretocmd{\tabularx}{\setRTL}{}{}
\pretocmd{\longtable}{\setRTL}{}{}
\setRTLtable 

%──────────────────────────────────────────────────────────────────────────────
% دستور طراحی سربرگ فصل (Inspirational Chapter Header)
%──────────────────────────────────────────────────────────────────────────────
\newcommand{\chapterheader}[4]{
    \begin{center}
    \begin{tikzpicture}[remember picture, overlay]
        % Background Box
        \node[anchor=north west, inner sep=0] at ($(current page.north west)+(2cm,-2cm)$) {
            \begin{tikzpicture}
                \shade[left color=#4, right color=black!80, rounded corners=10pt] (0,0) rectangle (16.5,6);
                
                % Girih Patterns (Watermark)
                \foreach \x in {2,6,10,14} {
                    \node[white, opacity=0.08] at (\x,3) {
                        \begin{tikzpicture}[scale=1.5]
                            \draw (0,0) -- (1,1) (1,0) -- (0,1);
                            \circle (0.5,0.5) circle (0.4);
                            \draw[rotate around={45:(0.5,0.5)}] (0.2,0.2) rectangle (0.8,0.8);
                        \end{tikzpicture}
                    };
                }
                
                % Large Number
                \node[anchor=east, white, opacity=0.1, font=\fontsize{120}{120}\selectfont\bfseries] at (16,2.5) {#1};
                
                % Gold Accent Line
                \draw[AccentGold, line width=4pt] (1,1) -- (1,5);
                
                % Text
                \node[anchor=north west, text width=13cm, inner sep=0] at (1.5,5) {
                    {\small\color{AccentGold} فصل #1}\\[4pt]
                    {\Huge\bfseries\color{white} #2}\\[12pt]
                    {\large\color{white!80}\itshape #3}
                };
            \end{tikzpicture}
        };
    \end{tikzpicture}
    \vspace*{7cm} % ایجاد فاصله برای ادامه متن
    \end{center}
}
\newcommand{\yes}{\textcolor{vertnapoleon}{\checkmark}}
\newcommand{\no}{\textcolor{rougerevolution}{$\times$}}
\newcommand{\somewhat}{\textcolor{orroyal}{$\circ$}}
\newcommand{\cmark}{\textcolor{vertnapoleon}{\checkmark}}
\newcommand{\phmark}[1]{\textcolor{phase#1}{\textbf{فاز #1}}}
\newcommand{\sourceline}[1]{%
    \par\vspace{4pt}%
    \begin{flushleft}%
    {\small\color{gris}— #1}%
    \end{flushleft}%
}

\newcommand{\concept}[1]{\textbf{\color{PrimaryMid}#1}}
\newcommand{\thinker}[1]{\textit{\color{AccentTeal}#1}}
\newcommand{\era}[1]{\textbf{\color{AccentAmber}#1}}
\newcommand{\enemy}[1]{\textbf{\color{AccentRed}#1}}
\newcommand{\lat}[1]{\lr{#1}}
\newcommand{\src}[1]{\textsuperscript{\scriptsize\color{NeutralDark}[#1]}}

\newenvironment{refsection}{\begin{quote}}{\end{quote}}

%──────────────────────────────────────────────────────────────────────────────
% تنظیمات سربرگ و پاورقی
%──────────────────────────────────────────────────────────────────────────────
\pagestyle{fancy}
\fancyhf{}
\fancyhead[LE]{\small\nouppercase{\leftmark}}
\fancyhead[RO]{\small\nouppercase{\rightmark}}
\fancyfoot[C]{\thepage}
\renewcommand{\headrulewidth}{0.4pt}
\renewcommand{\footrulewidth}{0pt}

\fancypagestyle{plain}{
    \fancyhf{}
    \fancyfoot[C]{\thepage}
    \renewcommand{\headrulewidth}{0pt}
}

%──────────────────────────────────────────────────────────────────────────────
% تنظیمات hyperref
%──────────────────────────────────────────────────────────────────────────────
\hypersetup{
    colorlinks=true,
    linkcolor=PrimaryMid,
    citecolor=AccentTeal,
    urlcolor=AccentAmber,
    bookmarksnumbered=true,
    pdfauthor={مهدی سالم},
    pdftitle={از بحران تا بالندگی},
    pdfsubject={طرح جامع تأسیس دموکراسی پایدار},
    pdfencoding=unicode
}
\onehalfspacing

%──────────────────────────────────────────────────────────────────────────────
% فونت و زبان فارسی (حتماً آخرین پکیج)
%──────────────────────────────────────────────────────────────────────────────
\usepackage{xepersian}

\settextfont{Vazirmatn}
\settextdigitfont{Vazirmatn}
\DefaultMathDigits

\IfFontExistsTF{Linux Libertine O}{%
    \setlatintextfont{Linux Libertine O}%
}{%
    \IfFontExistsTF{Times New Roman}{%
        \setlatintextfont{Times New Roman}%
    }{%
        \setlatintextfont{FreeSerif}%
    }%
}

\let\oldlistoftables\listoftables
\renewcommand{\listoftables}{%
  \cleardoublepage
  \phantomsection
  \addcontentsline{toc}{chapter}{\listtablename}
  \oldlistoftables
}

\let\oldlistoffigures\listoffigures
\renewcommand{\listoffigures}{%
  \cleardoublepage
  \phantomsection
  \addcontentsline{toc}{chapter}{\listfigurename}
  \oldlistoffigures
}

\AtBeginDocument{
  \renewcommand{\contentsname}{فهرست مطالب}
  \renewcommand{\listfigurename}{فهرست شکل‌ها}
  \renewcommand{\listtablename}{فهرست جدول‌ها}
}
\renewcommand{\bibname}{کتابشناسی}
\renewcommand{\indexname}{نمایه}
\renewcommand{\figurename}{شکل}
\renewcommand{\tablename}{جدول}
\renewcommand{\partname}{بخش}
\renewcommand{\chaptername}{فصل}
\renewcommand{\appendixname}{پیوست}

\begin{document}
	
	%══════════════════════════════════════════════════════════════════════════════
	%                              صفحات ابتدایی
	%══════════════════════════════════════════════════════════════════════════════
	\frontmatter
	
	%──────────────────────────────────────────────────────────────────────────────
	% صفحه عنوان
	%──────────────────────────────────────────────────────────────────────────────
	\begin{titlepage}
		\begin{tikzpicture}[remember picture, overlay]
			\node[inner sep=0] at (current page.center) {
				\includegraphics[width=\paperwidth,height=\paperheight]{images/slides/cover_bg.png}
			};
		\end{tikzpicture}
		
		\centering
		\vspace*{1.5cm}
		
		{\Huge\bfseries\color{white} از بحران تا بالندگی}
		
		\vspace{1.2cm}
		
		{\LARGE\bfseries\color{goldphoenix} طرح جامع تأسیس و تثبیت دموکراسی پایدار}
		
		\vspace{0.8cm}
		
		{\large\color{white!90} نقشه راه ۲۵ ساله برای گذار، بازسازی ملی و تعالی}
		
		\vfill
		
		{\Large\bfseries\color{white} مهدی سالم}
		
		\vspace{0.5cm}
		
		{\normalsize\color{white!80} ریچموندهیل — ۱۴۰۴}
		
		\vspace{1.5cm}
	\end{titlepage}
	
	% صفحه دوم عنوان (با لوگوی ققنوس)
	\begin{titlepage}
		\centering
		\vspace*{4cm}
		\includegraphics[width=8cm]{images/Goognoos.png}
		\vspace{2cm}
		
		{\Huge\bfseries\color{PrimaryDeep} از بحران تا بالندگی}
		\vspace{1cm}
		
		{\Large\color{NeutralDark} طرح جامع تأسیس دموکراسی پایدار در ایران}
		
		\vfill
		{\large نویسنده: مهدی سالم}
		\vspace{2cm}
	\end{titlepage}
	
	%──────────────────────────────────────────────────────────────────────────────
	% صفحه حق نشر
	%──────────────────────────────────────────────────────────────────────────────
	\thispagestyle{empty}
	\vspace*{\fill}
	\begin{center}
		{\small تمامی حقوق این اثر برای نویسنده محفوظ است.}\\[0.5cm]
		{\small نسخه اول — ۱۴۰۴}\\[1cm]
		{\footnotesize این کتاب یک سند پژوهشی-راهبردی است و نظرات مطرح‌شده}\\
		{\footnotesize لزوماً بازتاب‌دهنده دیدگاه هیچ نهاد یا سازمانی نیست.}
	\end{center}
	\vspace*{\fill}
	\newpage
	
	%──────────────────────────────────────────────────────────────────────────────
	% تقدیم
	%──────────────────────────────────────────────────────────────────────────────
	\thispagestyle{empty}
	\vspace*{3cm}
	\begin{flushright}
		{\large\bfseries تقدیم به}\\[1cm]
		{\normalsize نسل‌هایی که رؤیای آزادی و آبادانی را}\\[0.3cm]
		{\normalsize در سینه‌های خود زنده نگه داشتند}\\[0.3cm]
		{\normalsize و به آنان که فردا را خواهند ساخت.}
	\end{flushright}
	\vspace*{\fill}
	\newpage
	
	%──────────────────────────────────────────────────────────────────────────────
	% پیشگفتار
	%──────────────────────────────────────────────────────────────────────────────
	\chapter*{پیشگفتار}
	\addcontentsline{toc}{chapter}{پیشگفتار}
	
	این کتاب حاصل سال‌ها تأمل درباره یک پرسش بنیادین است: چگونه می‌توان در سرزمینی با میراث تمدنی کهن، تنوع قومی-فرهنگی غنی، و در عین حال بحران‌های انباشته‌شده چندلایه، یک نظام دموکراتیک پایدار و کارآمد بنا کرد؟
	
	این پرسش نه صرفاً یک تمرین آکادمیک، بلکه دغدغه‌ای وجودی است. میلیون‌ها انسان در جغرافیای ما منتظر پاسخی عملی به این پرسش‌اند. پاسخی که هم از توهم ساده‌انگارانه «دموکراسی یک‌شبه» بپرهیزد، و هم در دام یأس «ما آمادگی نداریم» نیفتد.
	
	این کتاب نه یک یوتوپیا است و نه یک نسخه قطعی. بلکه تلاشی است برای ترسیم یک نقشه راه واقع‌بینانه، مبتنی بر تجارب موفق و ناموفق جهانی، با درک عمیق از پیچیدگی‌های بومی.
	
	یک اصل این کتاب را از آغاز تا پایان همراهی می‌کند: \textbf{دموکراسی باید نان بیاورد تا ریشه بدواند}. هر گذار دموکراتیکی که نتواند در کوتاه‌مدت بهبود ملموسی در زندگی روزمره مردم ایجاد کند، محکوم به شکست یا بازگشت اقتدارگرایی است.
	
	امیدوارم این اثر بتواند سهمی هرچند کوچک در گفتگوی ملی درباره آینده داشته باشد.
	
	\vspace{1cm}
	\begin{flushleft}
		مهدی سالم\\
		ریچموندهیل، ۱۴۰۴
	\end{flushleft}
	
	%──────────────────────────────────────────────────────────────────────────────
	% فهرست‌ها
	%──────────────────────────────────────────────────────────────────────────────
	\tableofcontents
	\listoffigures
	\listoftables
	
	%══════════════════════════════════════════════════════════════════════════════
	%                              متن اصلی
	%══════════════════════════════════════════════════════════════════════════════
	\mainmatter
	
	%──────────────────────────────────────────────────────────────────────────────
	% بخش اول: مبانی
	%──────────────────────────────────────────────────────────────────────────────
	\part{مبانی}
	
	%══════════════════════════════════════════════════════════════════════════════
% فصل ۰: خلاصه مدیریتی
% از بحران تا بالندگی
%══════════════════════════════════════════════════════════════════════════════

\chapter*{خلاصه مدیریتی}
\addcontentsline{toc}{chapter}{خلاصه مدیریتی}
\markboth{خلاصه مدیریتی}{خلاصه مدیریتی}

%──────────────────────────────────────────────────────────────────────────────
% کادر خلاصه
%──────────────────────────────────────────────────────────────────────────────
\begin{kholasebox}
این سند نقشه راه ۲۵ ساله‌ای را برای گذار از وضعیت بحرانی کنونی به یک دموکراسی پایدار، کارآمد و مرفه پیشنهاد  می‌دهد. اصل محوری این طرح «آبادانی ملموس و فوری» است: مردم باید در کوتاه‌مدت بهبود واقعی را در زندگی روزمره احساس کنند تا به نظام جدید اعتماد کردهَ احساس کنند هزینه‌هایی که داده‌اند و فداکاری‌ای که انجام شده است بی بهره و هدر رفته نیستند. «برای یک زندگی معمولی» نمادی از این خواسته و واقعیت اجتماعی در ذهن آحاد ایرانیان است. این طرح شامل پنج فاز است: ۱- گذار (۲ سال)، ۲- نهادسازی (۳ سال)، ۳- تحکیم (۵ سال)، ۴- بلوغ (۱۰ سال)، و ۵- تعالی (۵ سال). موفقیت این طرح منوط به توافق ملی در پیشبرد تغییرات و نهادینه شدن ساختارهای جدید، مدیریت هوشمند تنوع قومی و حصول به انسجام اجتماعی-فرهنگی، حل بحران آب و سایر بحران‌های زیست محیطی، رهایی از تحریم‌ها و نوسازی اقتصادی، و ریشه‌کنی فساد و رفع رانت‌های مختلف در نهاد اجتماعی و اقتصادی جامعه است.
\end{kholasebox}

%══════════════════════════════════════════════════════════════════════════════
\section*{چشم‌انداز ۲۵ ساله}
%══════════════════════════════════════════════════════════════════════════════

در پایان این مسیر ۲۵ ساله، کشور به این ویژگی‌ها دست خواهد یافت:

\begin{table}[H]
	\centering
	\caption{چشم‌انداز کشور در افق ۲۵ ساله}
	\label{tab:vision25}
	\begin{tabular}{R{3cm} R{10cm}}
		\toprule
		\multicolumn{1}{c}{\textbf{بُعد}} & \multicolumn{1}{c}{\textbf{توصیف چشم‌انداز}} \\
		\midrule
		\rowcolor{bleulight}
		سیاسی & دموکراسی پایدار با امتیاز «آزاد» در شاخص فریدم‌هاوس؛ انتخابات آزاد و منظم؛ جامعه مدنی پویا \\
		اقتصادی & درآمد سرانه بالای ۱۵,۰۰۰ دلار؛ رشد پایدار ۵ درصد؛ بیکاری زیر ۵ درصد \\
		\rowcolor{bleulight}
		اجتماعی & شاخص توسعه انسانی (HDI) بالای ۰.۸۰؛ فقر مطلق زیر ۳ درصد؛ انسجام ملی \\
		زیست‌محیطی & احیای ۵۰ درصد سفره‌های آب؛ ۴۰ درصد انرژی تجدیدپذیر؛ توقف بیابان‌زایی \\
		\rowcolor{bleulight}
		بین‌المللی & الگوی منطقه‌ای دموکراسی؛ عضویت در نهادهای پیشرفته؛ برند ملی مثبت \\
		\bottomrule
	\end{tabular}
\end{table}


%══════════════════════════════════════════════════════════════════════════════
\section*{پنج فاز نقشه راه}
%══════════════════════════════════════════════════════════════════════════════

%──────────────────────────────────────────────────────────────────────────────
% خط زمانی فازها
%──────────────────────────────────────────────────────────────────────────────
\begin{figure}[H]
	\centering
	\begin{tikzpicture}[
		scale=1.2, transform shape,              % ← ۲۰٪ بزرگ‌تر
		phasebox/.style={
			rectangle, rounded corners=3pt, 
			minimum height=1.4cm, text centered, 
			font=\small\bfseries, text=white, drop shadow
		},
		milestone/.style={
			circle, minimum size=0.6cm, 
			fill=white, draw=black, line width=1.5pt
		},
		yearlab/.style={font=\footnotesize\bfseries},
		desclab/.style={font=\scriptsize, text width=2.5cm, align=center}
		]
		\draw[line width=3pt, color=gray] (0,0) -- (12.5,0);
		\node[phasebox, fill=blue!60!black, minimum width=1cm, 
		minimum height=1.4cm] at (0.5,1.2) {
			\begin{tabular}{c}
				\rl{فاز ۱}\\[-2pt]
				\scriptsize \rl{گذار}
			\end{tabular}
		};
		\node[phasebox, fill=green!60!black, minimum width=1.5cm, 
		minimum height=1.4cm] at (1.75,1.2) {
			\begin{tabular}{c}
				\rl{فاز ۲}\\[-2pt]
				\scriptsize \rl{نهادسازی}
			\end{tabular}
		};
		\node[phasebox, fill=orange!80!black, minimum width=2.5cm] 	at (3.75,1.2) {\rl{فاز ۳: تحکیم}};
		\node[phasebox, fill=purple!60!black, minimum width=5.0cm] 	at (7.5,1.2) {\rl{فاز ۴: بلوغ}};
		\node[phasebox, fill=red!70!black, minimum width=2.5cm] at (11.25,1.2) {\rl{فاز ۵: برتری}};
		\node[milestone] (y0)  at (0,0)    {};
		\node[milestone] (y2)  at (1,0) {};
		\node[milestone] (y5)  at (2.5,0)  {};
		\node[milestone] (y10) at (5,0)  {};
		\node[milestone] (y20) at (10,0) {};
		\node[milestone] (y25) at (12.5,0)   {};
		\node[yearlab, below=0.4cm of y0]  {۲۰۲۵};
		\node[yearlab, below=0.4cm of y2]  {۲۰۲۷};
		\node[yearlab, below=0.4cm of y5]  {۲۰۳۰};
		\node[yearlab, below=0.4cm of y10] {۲۰۳۵};
		\node[yearlab, below=0.4cm of y20] {۲۰۴۵};
		\node[yearlab, below=0.4cm of y25] {۲۰۵۰};
		\node[desclab, below=1cm of y0] {آغاز\\گذار};
		\node[desclab, below=1cm of y2] {\rl{قانون اساسی\\انتخابات}};
		\node[desclab, below=1cm of y5] {نهادها\\استقرار};
		\node[desclab, below=1cm of y10] {دموکراسی\\تثبیت};
		\node[desclab, below=1cm of y25] {برتری\\منطقه‌ای};
	\end{tikzpicture}
	\caption{خط زمانی نقشه راه ۲۵ ساله}
	\label{fig:timeline-exec}
\end{figure}

%──────────────────────────────────────────────────────────────────────────────
% جدول فازها
%──────────────────────────────────────────────────────────────────────────────
\begin{table}[H]
\centering
\caption{خلاصه پنج فاز نقشه راه}
\label{tab:phases-summary}
\small
\begin{tabular}{C{1.2cm} C{2cm} R{4.5cm} R{5cm}}
\toprule
\headmark فاز & \headmark مدت & \headmark اهداف کلیدی & \headmark دستاوردهای مورد انتظار \\
\midrule
\rowcolor{phase1!20}
\textcolor{phase1}{\textbf{۱}} & ۲ سال & 
گذار مسالمت‌آمیز؛ ثبات اولیه؛ قانون اساسی & 
دولت منتخب؛ آتش‌بس سیاسی؛ بهبود معیشت \\

\rowcolor{phase2!20}
\textcolor{phase2}{\textbf{۲}} & ۳ سال & 
نهادسازی دموکراتیک؛ اصلاحات ساختاری & 
نظام فدرال؛ قوه قضائیه مستقل؛ رشد ۵\% \\

\rowcolor{phase3!20}
\textcolor{phase3}{\textbf{۳}} & ۵ سال & 
تحکیم دموکراسی؛ توسعه فراگیر & 
دموکراسی رویه‌ای؛ کاهش فقر ۵۰\%؛ آشتی ملی \\

\rowcolor{phase4!20}
\textcolor{phase4}{\textbf{۴}} & ۱۰ سال & 
توسعه پایدار؛ رقابت‌پذیری & 
درآمد متوسط بالا؛ HDI بالا؛ رهبری منطقه‌ای \\

\rowcolor{phase5!20}
\textcolor{phase5}{\textbf{۵}} & ۵ سال & 
تعالی و برتری جهانی & 
الگوی جهانی؛ اقتصاد دانش‌بنیان؛ برند ملی \\
\bottomrule
\end{tabular}
\end{table}

%══════════════════════════════════════════════════════════════════════════════
\section*{ده اقدام فوری ۱۰۰ روز اول}
%══════════════════════════════════════════════════════════════════════════════

\begin{enghelabbox}[title={\textbf{بحرانی: ۱۰۰ روز سرنوشت‌ساز}}]
تجربه جهانی نشان می‌دهد که ۱۰۰ روز اول گذار، پنجره فرصتی است که اگر از دست برود، بازگشت‌ناپذیر است. مردم باید در این بازه تغییر را «ببینند» و «احساس کنند».
\end{enghelabbox}

\begin{table}[H]
\centering
\caption{ده اقدام فوری ۱۰۰ روز اول}
\label{tab:100days}
\begin{tabular}{C{0.8cm} R{4cm} R{4cm} R{4cm}}
\toprule
\headmark \# & \headmark اقدام & \headmark هدف & \headmark شاخص موفقیت \\
\midrule
\rowcolor{bleulight}
۱ & تشکیل دولت موقت فراگیر & مشروعیت‌سازی & نمایندگی همه اقوام اصلی \\
۲ & اعلام آتش‌بس سیاسی & کاهش تنش & توقف سرکوب و خشونت \\
\rowcolor{bleulight}
۳ & آزادی زندانیان سیاسی & اعتمادسازی & آزادی کامل در ۳۰ روز \\
۴ & بسته حمایت معیشتی فوری & کاهش فشار اقتصادی & پوشش ۵ میلیون خانوار \\
\rowcolor{bleulight}
۵ & برنامه آب اضطراری & رفع بحران فوری & آب‌رسانی به مناطق بحرانی \\
۶ & کنترل قاچاق سوخت & بهبود عرضه & کاهش ۵۰\% قاچاق \\
\rowcolor{bleulight}
۷ & محاکمه نمادین فاسدان بزرگ & عدالت و اعتماد & ۱۰ پرونده کلیدی \\
۸ & آزادی رسانه‌ها & فضای باز سیاسی & رفع سانسور \\
\rowcolor{bleulight}
۹ & اعلام انتخابات مجلس مؤسسان & مسیر قانونی & تاریخ مشخص در ۹۰ روز \\
۱۰ & آغاز مذاکره برای تعلیق تحریم‌ها & گشایش اقتصادی & تماس‌های اولیه \\
\bottomrule
\end{tabular}
\end{table}

%══════════════════════════════════════════════════════════════════════════════
\section*{چرخه‌های باطل و فاضل}
%══════════════════════════════════════════════════════════════════════════════

یکی از مفاهیم کلیدی این طرح، شناسایی و شکستن «چرخه‌های باطل» و جایگزینی آنها با «چرخه‌های فاضل» است:

%──────────────────────────────────────────────────────────────────────────────
% نمودار چرخه‌ها
%──────────────────────────────────────────────────────────────────────────────
\begin{figure}[H]
\centering
\begin{tikzpicture}[
    node distance=2cm,
    cyclebox/.style={
        rectangle, 
        rounded corners=5pt, 
        minimum width=2.8cm, 
        minimum height=1cm, 
        text centered, 
        font=\small,
        line width=1pt
    },
    badarrow/.style={->, >=Stealth, thick, color=rougerevolution},
    goodarrow/.style={->, >=Stealth, thick, color=vertnapoleon}
]

% چرخه باطل (چپ)
\node[cyclebox, draw=rougerevolution, fill=rougelight] (b1) at (0,0) {
    \begin{tabular}{c}
    فساد و\\
    بی‌اعتمادی
    \end{tabular}
};
\node[cyclebox, draw=rougerevolution, fill=rougelight] (b2) at (3,2) {
    \begin{tabular}{c}
    سرکوب و\\
    انسداد
    \end{tabular}
};
\node[cyclebox, draw=rougerevolution, fill=rougelight] (b3) at (6,0) {
    \begin{tabular}{c}
    رکود و\\
    فقر
    \end{tabular}
};
\node[cyclebox, draw=rougerevolution, fill=rougelight] (b4) at (3,-2) {
    \begin{tabular}{c}
    نارضایتی و\\
    بی‌ثباتی
    \end{tabular}
};

\draw[badarrow] (b1) -- (b2);
\draw[badarrow] (b2) -- (b3);
\draw[badarrow] (b3) -- (b4);
\draw[badarrow] (b4) -- (b1);

\node[font=\large\bfseries, color=rougerevolution] at (3,0) {چرخه باطل};

% چرخه فاضل (راست)
\node[cyclebox, draw=vertnapoleon, fill=vertlight] (g1) at (10,0) {
    \begin{tabular}{c}
    شفافیت و\\
    اعتماد
    \end{tabular}
};
\node[cyclebox, draw=vertnapoleon, fill=vertlight] (g2) at (13,2) {
    \begin{tabular}{c}
    مشارکت و\\
    آزادی
    \end{tabular}
};
\node[cyclebox, draw=vertnapoleon, fill=vertlight] (g3) at (16,0) {
    \begin{tabular}{c}
    رشد و\\
    رفاه
    \end{tabular}
};
\node[cyclebox, draw=vertnapoleon, fill=vertlight] (g4) at (13,-2) {
    \begin{tabular}{c}
    رضایت و\\
    ثبات
    \end{tabular}
};

\draw[goodarrow] (g1) -- (g2);
\draw[goodarrow] (g2) -- (g3);
\draw[goodarrow] (g3) -- (g4);
\draw[goodarrow] (g4) -- (g1);

\node[font=\large\bfseries, color=vertnapoleon] at (13,0) {چرخه فاضل};

% فلش تبدیل
\draw[->, >=Stealth, line width=3pt, color=bleurepublique] (7,0) -- (9,0);
\node[font=\small\bfseries, color=bleurepublique, above] at (8,0.2) {گذار};

\end{tikzpicture}
\caption{تبدیل چرخه باطل به چرخه فاضل}
\label{fig:cycles-exec}
\end{figure}

%══════════════════════════════════════════════════════════════════════════════
\section*{اصل محوری: آبادانی ملموس}
%══════════════════════════════════════════════════════════════════════════════

\begin{naghlbox}
«دموکراسی‌هایی که نتوانند نان بیاورند، ریشه نخواهند دواند. مردم برای ایده‌های انتزاعی صبر محدودی دارند، اما برای بهبود واقعی زندگی‌شان، حاضرند سال‌ها صبر کنند.»

\hfill --- درس کلیدی از تجربه گذارهای موفق
\end{naghlbox}

\begin{figure}[H]
\centering
\begin{tikzpicture}[
    node distance=1.5cm,
    timebox/.style={
        rectangle,
        rounded corners=3pt,
        minimum width=3.5cm,
        minimum height=1.8cm,
        text centered,
        font=\small,
        draw=bleurepublique,
        fill=bleulight,
        line width=1pt
    },
    arrow/.style={->, >=Stealth, thick, color=bleurepublique}
]

\node[timebox] (t1) at (0,0) {
    \begin{tabular}{c}
    \textbf{ماه ۱-۶}\\[3pt]
    آب، برق، نان، امنیت\\[2pt]
    {\scriptsize\color{vertnapoleon} احساس تغییر}
    \end{tabular}
};

\node[timebox] (t2) at (4.5,0) {
    \begin{tabular}{c}
    \textbf{ماه ۶-۱۲}\\[3pt]
    شغل، درمان، مسکن\\[2pt]
    {\scriptsize\color{vertnapoleon} امید به آینده}
    \end{tabular}
};

\node[timebox] (t3) at (9,0) {
    \begin{tabular}{c}
    \textbf{سال ۱-۳}\\[3pt]
    زیرساخت، خدمات\\[2pt]
    {\scriptsize\color{vertnapoleon} اعتماد به نظام}
    \end{tabular}
};

\node[timebox] (t4) at (13.5,0) {
    \begin{tabular}{c}
    \textbf{سال ۳-۱۰}\\[3pt]
    توسعه پایدار\\[2pt]
    {\scriptsize\color{vertnapoleon} تثبیت دموکراسی}
    \end{tabular}
};

\draw[arrow] (t1) -- (t2);
\draw[arrow] (t2) -- (t3);
\draw[arrow] (t3) -- (t4);

\end{tikzpicture}
\caption{مسیر ایجاد اعتماد از طریق آبادانی ملموس}
\label{fig:tangible-exec}
\end{figure}

%══════════════════════════════════════════════════════════════════════════════
\section*{بحران‌های کلیدی و راهکارها}
%══════════════════════════════════════════════════════════════════════════════

\begin{table}[H]
\centering
\caption{ماتریس بحران‌ها و راهکارها}
\label{tab:crisis-solutions}
\small
\begin{tabular}{L{2.5cm} L{4cm} L{4cm} C{2cm}}
\toprule
\headmark بحران & \headmark وضعیت فعلی & \headmark راهکار پیشنهادی & \headmark فاز اجرا \\
\midrule
\rowcolor{rougelight}
آب و خشکسالی & 
فروپاشی ۷۰\% سفره‌ها & 
مدیریت تقاضا + شیرین‌سازی + بازچرخانی & 
۱-۳ \\

فساد مویرگی & 
رتبه ۱۵۰+ شفافیت & 
دادگاه ویژه + شفافیت + دیجیتال‌سازی & 
۱-۲ \\

\rowcolor{rougelight}
تحریم‌های بین‌المللی & 
انزوای اقتصادی & 
تغییر رفتار + مذاکره تدریجی & 
۱-۳ \\

بی‌اعتمادی قومی & 
تنش‌های نهفته & 
فدرالیسم + خودمختاری فرهنگی + توزیع عادلانه & 
۱-۵ \\

\rowcolor{rougelight}
انرژی و قاچاق & 
قاچاق روزانه ۲۰M لیتر & 
هدفمندسازی یارانه + کنترل مرزی & 
۱-۲ \\

فرار سرمایه و مغزها & 
از دست دادن نخبگان & 
بهبود فضا + برنامه بازگشت دیاسپورا & 
۲-۴ \\
\bottomrule
\end{tabular}
\end{table}

%══════════════════════════════════════════════════════════════════════════════
\section*{شاخص‌های کلیدی عملکرد}
%══════════════════════════════════════════════════════════════════════════════

\begin{table}[H]
\centering
\caption{اهداف کمّی در هر فاز}
\label{tab:kpi-exec}
\small
\begin{tabular}{L{3.5cm} C{1.5cm} C{1.5cm} C{1.5cm} C{1.5cm} C{1.5cm} C{1.5cm}}
\toprule
\headmark شاخص & \headmark پایه & \headmark فاز۱ & \headmark فاز۲ & \headmark فاز۳ & \headmark فاز۴ & \headmark فاز۵ \\
\midrule
\rowcolor{bleulight}
Freedom House (از ۷) & ۲ & ۳ & ۴.۵ & ۵.۵ & ۶ & ۶.۵ \\
GDP سرانه (هزار \$) & ۳ & ۳.۵ & ۵ & ۸ & ۱۲ & ۱۵+ \\
\rowcolor{bleulight}
HDI & ۰.۷۰ & ۰.۷۲ & ۰.۷۵ & ۰.۷۸ & ۰.۸۲ & ۰.۸۵ \\
نرخ بیکاری (\%) & ۱۵ & ۱۳ & ۱۰ & ۷ & ۵ & ۴ \\
\rowcolor{bleulight}
ضریب جینی & ۰.۴۵ & ۰.۴۳ & ۰.۳۸ & ۰.۳۴ & ۰.۳۲ & ۰.۳۰ \\
شاخص فساد (CPI) & ۲۵ & ۳۰ & ۴۰ & ۵۰ & ۵۵ & ۶۰ \\
\rowcolor{bleulight}
انرژی تجدیدپذیر (\%) & ۵ & ۸ & ۱۵ & ۲۵ & ۳۵ & ۴۰ \\
\bottomrule
\end{tabular}
\end{table}

%══════════════════════════════════════════════════════════════════════════════
\section*{بودجه کلان تخمینی}
%══════════════════════════════════════════════════════════════════════════════

\begin{figure}[H]
\centering
\begin{tikzpicture}
\begin{axis}[
    ybar,
    width=14cm,
    height=7cm,
    ylabel={میلیارد دلار},
    xlabel={فاز},
    symbolic x coords={فاز ۱, فاز ۲, فاز ۳, فاز ۴, فاز ۵},
    xtick=data,
    ymin=0,
    ymax=220,
    bar width=25pt,
    nodes near coords,
    every node near coord/.append style={font=\small\bfseries},
    legend style={at={(0.98,0.98)}, anchor=north east},
    grid=major,
    grid style={line width=0.2pt, draw=gray!30}
]
\addplot[fill=phase1, draw=phase1!70!black] coordinates {(فاز ۱,15) (فاز ۲,0) (فاز ۳,0) (فاز ۴,0) (فاز ۵,0)};
\addplot[fill=phase2, draw=phase2!70!black] coordinates {(فاز ۱,0) (فاز ۲,30) (فاز ۳,0) (فاز ۴,0) (فاز ۵,0)};
\addplot[fill=phase3, draw=phase3!70!black] coordinates {(فاز ۱,0) (فاز ۲,0) (فاز ۳,70) (فاز ۴,0) (فاز ۵,0)};
\addplot[fill=phase4, draw=phase4!70!black] coordinates {(فاز ۱,0) (فاز ۲,0) (فاز ۳,0) (فاز ۴,200) (فاز ۵,0)};
\addplot[fill=phase5, draw=phase5!70!black] coordinates {(فاز ۱,0) (فاز ۲,0) (فاز ۳,0) (فاز ۴,0) (فاز ۵,150)};
\end{axis}
\end{tikzpicture}
\caption{توزیع بودجه تخمینی در پنج فاز (میلیارد دلار)}
\label{fig:budget-exec}
\end{figure}

\begin{noktebox}
\textbf{منابع تأمین مالی:}
\begin{itemize}[nosep]
    \item فاز ۱: ۶۰\% کمک بین‌المللی، ۴۰\% داخلی
    \item فاز ۲: ۴۰\% وام توسعه‌ای، ۵۰\% داخلی، ۱۰\% کمک
    \item فاز ۳-۵: عمدتاً منابع داخلی + سرمایه‌گذاری خارجی
\end{itemize}
\end{noktebox}

%══════════════════════════════════════════════════════════════════════════════
\section*{پیام به ذینفعان کلیدی}
%══════════════════════════════════════════════════════════════════════════════

\begin{table}[H]
\centering
\caption{پیام کلیدی به گروه‌های ذینفع}
\label{tab:stakeholders-exec}
\begin{tabular}{L{3cm} L{10cm}}
\toprule
\headmark ذینفع & \headmark پیام کلیدی \\
\midrule
\rowcolor{bleulight}
شهروندان عادی & بهبود ملموس در زندگی روزمره؛ صدای شما شنیده می‌شود \\
جوانان & فرصت‌های شغلی و مشارکت؛ آینده متعلق به شماست \\
\rowcolor{bleulight}
اقوام و اقلیت‌ها & حقوق برابر، خودمختاری فرهنگی، سهم عادلانه از قدرت و منابع \\
زنان & برابری کامل حقوقی و عملی؛ مشارکت فعال در همه سطوح \\
\rowcolor{bleulight}
فعالان مدنی & فضای باز برای فعالیت؛ شریک در نظارت و اصلاح \\
بخش خصوصی & ثبات، شفافیت، حاکمیت قانون؛ فرصت‌های سرمایه‌گذاری \\
\rowcolor{bleulight}
دیاسپورا & بازگشت و مشارکت؛ کشور به شما نیاز دارد \\
جامعه بین‌الملل & تعهد به اصول جهانی؛ شراکت سازنده \\
\bottomrule
\end{tabular}
\end{table}

%══════════════════════════════════════════════════════════════════════════════
\section*{ریسک‌های اصلی}
%══════════════════════════════════════════════════════════════════════════════

\begin{olgoobox}[title={\hfill \textbf{درس از تجارب جهانی}}]
هیچ گذار دموکراتیکی بدون ریسک نیست. کلید موفقیت، شناسایی زودهنگام ریسک‌ها و آمادگی برای مدیریت آنهاست. اسپانیا، آفریقای جنوبی و اندونزی همگی با ریسک‌های جدی مواجه شدند اما با مدیریت هوشمند از آنها عبور کردند.
\end{olgoobox}

\begin{table}[H]
\centering
\caption{ماتریس ریسک‌های اصلی}
\label{tab:risks-exec}
\begin{tabular}{L{3cm} C{2cm} C{2cm} L{5cm}}
\toprule
\headmark ریسک & \headmark احتمال & \headmark اثر & \headmark راهکار کاهش \\
\midrule
\rowcolor{rougelight}
بازگشت اقتدارگرایی & متوسط & بحرانی & قیود قانون اساسی + نظارت مدنی \\
تنش قومی & بالا & بالا & توافق‌گرایی + توزیع عادلانه \\
\rowcolor{rougelight}
شکست اقتصادی & متوسط & بالا & برنامه‌ریزی + کمک بین‌المللی \\
مداخله خارجی & متوسط & بالا & دیپلماسی پیشگیرانه + توازن \\
\rowcolor{rougelight}
فساد سیستمی & بالا & متوسط & نهادهای نظارتی قوی \\
\bottomrule
\end{tabular}
\end{table}

%══════════════════════════════════════════════════════════════════════════════
\section*{نتیجه‌گیری}
%══════════════════════════════════════════════════════════════════════════════

این طرح یک رؤیای خیالی نیست؛ بلکه نقشه راهی واقع‌بینانه است که بر پایه تجارب موفق جهانی و درک عمیق از واقعیات بومی تدوین شده است. کشورهایی مانند کره جنوبی، اسپانیا، آفریقای جنوبی و اندونزی نشان داده‌اند که گذار از اقتدارگرایی و بحران به دموکراسی و رفاه ممکن است.

کلید موفقیت در سه چیز است:
\begin{enumerate}[nosep]
    \item \textbf{توافق ملی فراگیر:} همه گروه‌های اصلی باید در فرآیند سهیم باشند.
    \item \textbf{آبادانی ملموس و فوری:} مردم باید بهبود را احساس کنند.
    \item \textbf{صبر استراتژیک:} تغییر واقعی زمان می‌برد؛ افق ۲۵ ساله واقع‌بینانه است.
\end{enumerate}

\vspace{0.5cm}
\begin{center}
\textcolor{bleurepublique}{\rule{0.5\textwidth}{1pt}}

\vspace{0.3cm}
{\large\bfseries آینده ساخته می‌شود، نه منتظر می‌ماند.}
\vspace{0.3cm}

\textcolor{bleurepublique}{\rule{0.5\textwidth}{1pt}}
\end{center}
	%══════════════════════════════════════════════════════════════════════════════
% فصل ۱: مقدمه — چرا این کتاب؟
% از بحران تا بالندگی
%══════════════════════════════════════════════════════════════════════════════

\chapter{مقدمه: چرا این کتاب؟}
\label{ch:introduction}

%──────────────────────────────────────────────────────────────────────────────
% کادر خلاصه فصل
%──────────────────────────────────────────────────────────────────────────────
\begin{kholasebox}
این فصل به سه پرسش بنیادین پاسخ می‌دهد: چرا گذار دموکراتیک ضروری است؟ چرا اکنون لحظه مناسبی است؟ و چرا این کتاب رویکرد متفاوتی دارد؟ استدلال اصلی این است که وضعیت موجود نه پایدار است و نه مطلوب؛ بحران‌های آب، اقتصاد، و مشروعیت در حال تعمیق‌اند و ادامه مسیر فعلی به فروپاشی منجر خواهد شد. اما گذار موفق نیازمند رویکردی متفاوت است: نه کپی‌برداری کور از مدل‌های غربی، نه بازگشت به استبداد سنتی، بلکه «دموکراسی بومی‌شده» که ریشه در واقعیات این سرزمین داشته باشد. اصل محوری این رویکرد، «آبادانی ملموس و فوری» است.
\end{kholasebox}

%══════════════════════════════════════════════════════════════════════════════
\section{بیان مسئله: بن‌بست تاریخی}
\label{sec:problem}
%══════════════════════════════════════════════════════════════════════════════

ما در نقطه‌ای از تاریخ ایستاده‌ایم که ادامه مسیر فعلی نه ممکن است و نه مطلوب. کشور با مجموعه‌ای از بحران‌های به‌هم‌پیوسته مواجه است که هر یک به تنهایی می‌تواند ویرانگر باشد، و ترکیب آنها وضعیتی انفجاری ایجاد کرده است.

\begin{figure}[H]
\centering
\begin{tikzpicture}[
    node distance=1.8cm,
    crisis/.style={
        ellipse,
        minimum width=3cm,
        minimum height=1.5cm,
        text centered,
        font=\small,
        draw=rougerevolution,
        fill=rougelight,
        line width=1pt
    },
    center/.style={
        circle,
        minimum size=2.5cm,
        text centered,
        font=\small\bfseries,
        draw=black,
        fill=grislight,
        line width=2pt
    },
    link/.style={<->, >=Stealth, thick, color=rougemid}
]

% مرکز
\node[center] (c) at (0,0) {
    \begin{tabular}{c}
    بن‌بست\\
    تاریخی
    \end{tabular}
};

% بحران‌ها
\node[crisis] (c1) at (90:3.5cm) {
    \begin{tabular}{c}
    بحران آب\\
    {\scriptsize فروپاشی سفره‌ها}
    \end{tabular}
};
\node[crisis] (c2) at (30:3.5cm) {
    \begin{tabular}{c}
    بحران اقتصادی\\
    {\scriptsize تحریم + تورم}
    \end{tabular}
};
\node[crisis] (c3) at (330:3.5cm) {
    \begin{tabular}{c}
    بحران سیاسی\\
    {\scriptsize انسداد + فساد}
    \end{tabular}
};
\node[crisis] (c4) at (270:3.5cm) {
    \begin{tabular}{c}
    بحران اجتماعی\\
    {\scriptsize بی‌اعتمادی}
    \end{tabular}
};
\node[crisis] (c5) at (210:3.5cm) {
    \begin{tabular}{c}
    بحران هویتی\\
    {\scriptsize تنش قومی}
    \end{tabular}
};
\node[crisis] (c6) at (150:3.5cm) {
    \begin{tabular}{c}
    بحران امنیتی\\
    {\scriptsize ژئوپلیتیک}
    \end{tabular}
};

% اتصالات
\draw[link] (c) -- (c1);
\draw[link] (c) -- (c2);
\draw[link] (c) -- (c3);
\draw[link] (c) -- (c4);
\draw[link] (c) -- (c5);
\draw[link] (c) -- (c6);

% اتصالات بین بحران‌ها
\draw[link, dashed, color=gray] (c1) -- (c2);
\draw[link, dashed, color=gray] (c2) -- (c3);
\draw[link, dashed, color=gray] (c3) -- (c4);
\draw[link, dashed, color=gray] (c4) -- (c5);
\draw[link, dashed, color=gray] (c5) -- (c6);
\draw[link, dashed, color=gray] (c6) -- (c1);

\end{tikzpicture}
\caption{شبکه بحران‌های به‌هم‌پیوسته}
\label{fig:crisis-network}
\end{figure}

\subsection{ابعاد بن‌بست}

\begin{databox}
\textbf{آمار تکان‌دهنده:}
\begin{itemize}[nosep]
    \item ۷۰\% سفره‌های زیرزمینی در وضعیت بحرانی یا فروپاشی
    \item تورم سالانه بالای ۴۰\% به مدت یک دهه
    \item رتبه ۱۵۰+ در شاخص شفافیت و فساد
    \item فرار سالانه ۱۵۰,۰۰۰ نفر نیروی متخصص
    \item بیکاری جوانان بالای ۲۵\%
\end{itemize}
\end{databox}

این ارقام صرفاً آمار نیستند؛ هر یک بازتاب‌دهنده میلیون‌ها زندگی است که در معرض تهدید قرار دارد. بحران آب به تنهایی می‌تواند طی ۱۰-۱۵ سال آینده مهاجرت اجباری ده‌ها میلیون نفر را رقم بزند. تورم مزمن طبقه متوسط را نابود کرده و فقر را گسترش داده است. فساد سیستماتیک هرگونه امکان اصلاح از درون را مسدود کرده است.

%══════════════════════════════════════════════════════════════════════════════
\section{چرا اکنون؟ فرصت و تهدید}
\label{sec:why-now}
%══════════════════════════════════════════════════════════════════════════════

\subsection{پنجره فرصت}

تاریخ نشان می‌دهد که گذارهای دموکراتیک معمولاً در «پنجره‌های فرصت» رخ می‌دهند — لحظاتی که ترکیب خاصی از شرایط داخلی و خارجی، تغییر را ممکن می‌سازد. این پنجره‌ها موقتی‌اند و اگر از آنها استفاده نشود، بسته می‌شوند.

\begin{table}[H]
\centering
\caption{عوامل باز شدن پنجره فرصت}
\label{tab:opportunity-window}
\begin{tabular}{C{1.5cm} L{5cm} L{6cm}}
\toprule
\headmark نوع & \headmark عامل & \headmark توضیح \\
\midrule
\rowcolor{vertlight}
داخلی & فرسایش مشروعیت نظام & ناتوانی در ارائه خدمات پایه \\
\rowcolor{vertlight}
داخلی & رشد جامعه مدنی & علی‌رغم سرکوب، شبکه‌های مدنی باقی‌اند \\
\rowcolor{vertlight}
داخلی & نسل جوان تحصیل‌کرده & ۶۰\% جمعیت زیر ۳۰ سال \\
\rowcolor{bleulight}
خارجی & تغییر ژئوپلیتیک منطقه‌ای & بازآرایی قدرت‌های بزرگ \\
\rowcolor{bleulight}
خارجی & فشار بین‌المللی حقوق بشری & توجه جهانی به وضعیت داخلی \\
\rowcolor{bleulight}
خارجی & تجربه موفق کشورهای مشابه & الگوهای قابل یادگیری \\
\bottomrule
\end{tabular}
\end{table}

\subsection{تهدید تأخیر}

\begin{enghelabbox}[title={\hfill \textbf{هشدار: نقطه بی‌بازگشت}}]
برخی بحران‌ها — به‌ویژه بحران آب — نقاط بی‌بازگشت دارند. اگر سفره‌های زیرزمینی تا حد معینی تخلیه شوند، احیای آنها غیرممکن یا بسیار پرهزینه خواهد بود. برآوردها نشان می‌دهد که بدون اقدام فوری، تا ۱۵ سال آینده بخش‌های وسیعی از کشور غیرقابل سکونت خواهند شد.
\end{enghelabbox}

%══════════════════════════════════════════════════════════════════════════════
\section{چرا دموکراسی؟ نقد آلترناتیوها}
\label{sec:why-democracy}
%══════════════════════════════════════════════════════════════════════════════

پیش از ادامه، باید به یک پرسش بنیادین پاسخ دهیم: چرا دموکراسی؟ چرا نه یک «مستبد خیرخواه» که سریع‌تر تصمیم بگیرد؟ چرا نه یک نظام تک‌حزبی به سبک چین که ثبات داشته باشد؟

\subsection{نقد اقتدارگرایی توسعه‌گرا}

\begin{naghlbox}
«توسعه بدون آزادی، نه پایدار است و نه مطلوب. انسان‌ها صرفاً مصرف‌کننده کالا نیستند؛ آنها شهروندانی‌اند که می‌خواهند در سرنوشت خود نقش داشته باشند.»

\hfill --- آمارتیا سن، \textit{توسعه به‌مثابه آزادی}، ۱۹۹۹
\end{naghlbox}

استدلال‌های له اقتدارگرایی توسعه‌گرا و پاسخ به آنها:

\begin{table}[H]
\centering
\caption{نقد استدلال‌های له اقتدارگرایی}
\label{tab:authoritarianism-critique}
\begin{tabular}{L{4cm} L{8.5cm}}
\toprule
\headmark استدلال & \headmark پاسخ \\
\midrule
\rowcolor{bleulight}
«تصمیم‌گیری سریع‌تر» & 
تصمیم‌های سریع اما غلط فاجعه‌بارترند. دموکراسی با ایجاد بازخورد، از خطاهای بزرگ جلوگیری می‌کند. \\

«ثبات بیشتر» & 
ثبات اقتدارگرایی شکننده است؛ سقوط ناگهانی شوروی، عراق، لیبی نمونه‌اند. \\

\rowcolor{bleulight}
«چین موفق بوده» & 
چین استثناست نه قاعده. اکثر اقتدارگرایی‌ها شکست خورده‌اند. \\

«مردم آماده نیستند» & 
این استدلال همیشه برای توجیه استبداد استفاده شده. هند با بی‌سوادی ۸۰\% دموکراسی را آغاز کرد. \\
\bottomrule
\end{tabular}
\end{table}

\subsection{مزیت‌های دموکراسی}

\begin{olgoobox}[title={\hfill \textbf{شواهد تجربی}}]
تحقیقات \lr{Acemoglu} و \lr{Robinson} (۲۰۱۹) با بررسی ۱۷۵ کشور طی ۱۹۶۰-۲۰۱۰ نشان می‌دهد:
\begin{itemize}[nosep]
    \item دموکراتیزاسیون به‌طور متوسط GDP سرانه را ۲۰\% افزایش می‌دهد (طی ۲۵ سال)
    \item این اثر از طریق سرمایه‌گذاری در آموزش و بهداشت، کاهش ناآرامی، و اصلاحات اقتصادی حاصل می‌شود
    \item دموکراسی‌ها در مقایسه با اقتدارگرایی‌ها، قحطی ندارند (آمارتیا سن)
\end{itemize}
\end{olgoobox}

%══════════════════════════════════════════════════════════════════════════════
\section{دموکراسی بومی‌شده: نه کپی، نه رد}
\label{sec:indigenous-democracy}
%══════════════════════════════════════════════════════════════════════════════

این کتاب نه دموکراسی لیبرال غربی را به‌شکل کپی‌برداری تجویز می‌کند، و نه به بهانه «فرهنگ متفاوت» استبداد را توجیه می‌کند. رویکرد ما «دموکراسی بومی‌شده» است:

\begin{figure}[H]
\centering
\begin{tikzpicture}[
    node distance=2cm,
    mainbox/.style={
        rectangle,
        rounded corners=5pt,
        minimum width=4cm,
        minimum height=2cm,
        text centered,
        font=\small,
        line width=1.5pt
    },
    arrow/.style={->, >=Stealth, thick}
]

% سه منبع
\node[mainbox, draw=bleurepublique, fill=bleulight] (a) at (0,3) {
    \begin{tabular}{c}
    \textbf{اصول جهانی}\\[3pt]
    {\scriptsize حقوق بشر}\\
    {\scriptsize دموکراسی}\\
    {\scriptsize حاکمیت قانون}
    \end{tabular}
};

\node[mainbox, draw=orroyal, fill=orroyallight] (b) at (5,3) {
    \begin{tabular}{c}
    \textbf{میراث بومی}\\[3pt]
    {\scriptsize سنت‌های مشورتی}\\
    {\scriptsize نهادهای محلی}\\
    {\scriptsize خرد جمعی}
    \end{tabular}
};

\node[mainbox, draw=vertnapoleon, fill=vertlight] (c) at (10,3) {
    \begin{tabular}{c}
    \textbf{واقعیت تنوع}\\[3pt]
    {\scriptsize اقوام}\\
    {\scriptsize زبان‌ها}\\
    {\scriptsize مذاهب}
    \end{tabular}
};

% تلفیق
\node[mainbox, draw=violetempire, fill=violetlight, minimum width=5cm, minimum height=2.5cm] (d) at (5,0) {
    \begin{tabular}{c}
    \textbf{دموکراسی بومی‌شده}\\[5pt]
    {\small ترکیب خلاقانه}\\
    {\small نه کپی، نه رد}
    \end{tabular}
};

% نتایج
\node[mainbox, draw=rougerevolution, fill=rougelight, minimum width=2.5cm, minimum height=1.2cm] (e1) at (1,-3) {
    \begin{tabular}{c}
    مشروعیت
    \end{tabular}
};
\node[mainbox, draw=phase2, fill=phase2!20, minimum width=2.5cm, minimum height=1.2cm] (e2) at (5,-3) {
    \begin{tabular}{c}
    کارآمدی
    \end{tabular}
};
\node[mainbox, draw=phase4, fill=phase4!20, minimum width=2.5cm, minimum height=1.2cm] (e3) at (9,-3) {
    \begin{tabular}{c}
    پایداری
    \end{tabular}
};

% فلش‌ها
\draw[arrow, color=bleurepublique] (a) -- (d);
\draw[arrow, color=orroyal] (b) -- (d);
\draw[arrow, color=vertnapoleon] (c) -- (d);
\draw[arrow, color=violetempire] (d) -- (e1);
\draw[arrow, color=violetempire] (d) -- (e2);
\draw[arrow, color=violetempire] (d) -- (e3);

\end{tikzpicture}
\caption{مدل دموکراسی بومی‌شده}
\label{fig:indigenous-democracy}
\end{figure}

\subsection{اصول جهانی غیرقابل مذاکره}

برخی اصول جهانی‌اند و با بهانه «فرهنگ متفاوت» قابل نقض نیستند:

\begin{itemize}[nosep]
    \item کرامت ذاتی انسان و حقوق بنیادین
    \item برابری شهروندان در مقابل قانون
    \item حق مشارکت در تعیین سرنوشت جمعی
    \item محدودیت قدرت و پاسخگویی حاکمان
    \item آزادی بیان، اجتماع، و اندیشه
\end{itemize}

\subsection{انعطاف در شکل نهادها}

اما شکل نهادی تحقق این اصول می‌تواند متنوع باشد:

\begin{table}[H]
\centering
\caption{انعطاف در طراحی نهادها}
\label{tab:institutional-flexibility}
\begin{tabular}{L{3.5cm} L{4.5cm} L{4.5cm}}
\toprule
\headmark حوزه & \headmark اصل ثابت & \headmark شکل منعطف \\
\midrule
\rowcolor{bleulight}
مشارکت سیاسی & حق رأی همگانی & نظام ریاستی/پارلمانی/ترکیبی \\
مدیریت تنوع & حقوق اقلیت‌ها & فدرالیسم/خودمختاری/وحدت‌گرا \\
\rowcolor{bleulight}
نظام انتخاباتی & انتخابات آزاد و منصفانه & تناسبی/اکثریتی/ترکیبی \\
نقش دین & آزادی مذهب & لائیسیته/جدایی/همکاری \\
\bottomrule
\end{tabular}
\end{table}

%══════════════════════════════════════════════════════════════════════════════
\section{اصل محوری: آبادانی ملموس و فوری}
\label{sec:tangible-development}
%══════════════════════════════════════════════════════════════════════════════

اگر این کتاب یک پیام داشته باشد، این است: \textbf{دموکراسی باید نان بیاورد تا ریشه بدواند.}

\begin{naghlbox}
«مردم برای ایده‌های انتزاعی مانند دموکراسی و حقوق بشر، صبر محدودی دارند. آنها می‌خواهند بدانند آیا فردا آب خواهند داشت، آیا فرزندانشان شغل پیدا می‌کنند، آیا می‌توانند بدون رشوه دادن به پزشک بروند. اگر نظام جدید در این امور ملموس تفاوتی ایجاد نکند، مردم به گذشته حسرت خواهند خورد — حتی اگر آن گذشته استبداد بوده باشد.»

\hfill --- درس کلیدی از گذارهای ناموفق (مصر ۲۰۱۱-۲۰۱۳)
\end{naghlbox}

\subsection{چرا این اصل حیاتی است؟}

تجربه جهانی گذارهای دموکراتیک یک الگوی واضح نشان می‌دهد:

\begin{table}[H]
\centering
\caption{رابطه بهبود معیشتی و موفقیت گذار}
\label{tab:livelihood-transition}
\begin{tabular}{L{3cm} L{4cm} L{4cm} C{2cm}}
\toprule
\headmark کشور & \headmark بهبود اولیه & \headmark وضعیت معیشت & \headmark نتیجه گذار \\
\midrule
\rowcolor{vertlight}
اسپانیا & رشد اقتصادی ۴\%+ & بهبود محسوس & \textcolor{vertnapoleon}{\textbf{موفق}} \\
\rowcolor{vertlight}
کره جنوبی & رشد سریع ادامه یافت & بهبود چشمگیر & \textcolor{vertnapoleon}{\textbf{موفق}} \\
\rowcolor{vertlight}
لهستان & شوک اولیه، سپس بهبود & مختلط & \textcolor{vertnapoleon}{\textbf{موفق}} \\
\rowcolor{rougelight}
روسیه دهه ۹۰ & سقوط ۴۰\% GDP & فاجعه‌بار & \textcolor{rougerevolution}{\textbf{عقب‌گرد}} \\
\rowcolor{rougelight}
مصر ۲۰۱۱-۱۳ & رکود اقتصادی & وخامت & \textcolor{rougerevolution}{\textbf{شکست}} \\
\rowcolor{rougelight}
اوکراین ۲۰۰۴ & بهبود اندک & راکد & \textcolor{orroyal}{\textbf{ناقص}} \\
\bottomrule
\end{tabular}
\end{table}

\subsection{مکانیزم اثرگذاری}

چرا بهبود معیشتی برای موفقیت گذار حیاتی است؟

\begin{figure}[H]
\centering
\begin{tikzpicture}[
    node distance=1.2cm,
    box/.style={
        rectangle,
        rounded corners=3pt,
        minimum width=3.2cm,
        minimum height=1.4cm,
        text centered,
        font=\small,
        draw=bleurepublique,
        fill=bleulight,
        line width=1pt
    },
    arrow/.style={->, >=Stealth, thick, color=bleurepublique}
]

% مسیر علّی
\node[box] (a) at (0,0) {
    \begin{tabular}{c}
    بهبود\\
    ملموس
    \end{tabular}
};

\node[box] (b) at (4,0) {
    \begin{tabular}{c}
    امید به\\
    آینده
    \end{tabular}
};

\node[box] (c) at (8,0) {
    \begin{tabular}{c}
    صبر\\
    استراتژیک
    \end{tabular}
};

\node[box] (d) at (12,0) {
    \begin{tabular}{c}
    اعتماد به\\
    نظام جدید
    \end{tabular}
};

\node[box, fill=vertlight, draw=vertnapoleon] (e) at (6,-2.5) {
    \begin{tabular}{c}
    تثبیت\\
    دموکراسی
    \end{tabular}
};

\draw[arrow] (a) -- (b);
\draw[arrow] (b) -- (c);
\draw[arrow] (c) -- (d);
\draw[arrow] (d) -- (e);
\draw[arrow] (a) to[bend right=30] (e);

\end{tikzpicture}
\caption{مکانیزم تأثیر بهبود معیشتی بر تثبیت دموکراسی}
\label{fig:livelihood-mechanism}
\end{figure}

\subsection{چه چیزی باید در ۱۰۰ روز اول تغییر کند؟}

\begin{table}[H]
\centering
\caption{تغییرات ملموس مورد انتظار در ۱۰۰ روز اول}
\label{tab:100days-tangible}
\begin{tabular}{L{3cm} L{4cm} L{5.5cm}}
\toprule
\headmark حوزه & \headmark تغییر ملموس & \headmark چگونه مردم آن را احساس می‌کنند \\
\midrule
\rowcolor{bleulight}
آب & توزیع آب در مناطق بحرانی & شیر آب باز می‌شود؛ صف تانکر کوتاه می‌شود \\
برق & کاهش قطعی & یخچال خاموش نمی‌شود؛ کارخانه تعطیل نمی‌شود \\
\rowcolor{bleulight}
نان و غذا & کاهش صف و قیمت & خرید راحت‌تر؛ سفره پُرتر \\
سوخت & عرضه پایدار & صف پمپ بنزین کوتاه می‌شود \\
\rowcolor{bleulight}
امنیت & کاهش سرکوب & مردم می‌توانند آزادانه صحبت کنند \\
بوروکراسی & ساده‌سازی خدمات & بدون رشوه کار انجام می‌شود \\
\bottomrule
\end{tabular}
\end{table}

\begin{enghelabbox}[title={\hfill \textbf{درس از شکست مصر}}]
در مصر پس از انقلاب ۲۰۱۱، دولت‌های موقت بر مباحث سیاسی متمرکز شدند و وضعیت اقتصادی رها شد. تورم افزایش یافت، بیکاری بالا رفت، و صف نان طولانی‌تر شد. این امر زمینه بازگشت نظامیان در ۲۰۱۳ را فراهم کرد. مردم عادی که زندگی‌شان سخت‌تر شده بود، از کودتا استقبال کردند.
\end{enghelabbox}

%══════════════════════════════════════════════════════════════════════════════
\section{چرخه‌های باطل و فاضل}
\label{sec:cycles}
%══════════════════════════════════════════════════════════════════════════════

یکی از مفاهیم تحلیلی کلیدی این کتاب، تمایز بین «چرخه‌های باطل» (دورهای معیوب) و «چرخه‌های فاضل» (دورهای مطلوب) است. وضعیت فعلی کشور اسیر چندین چرخه باطل است که یکدیگر را تقویت می‌کنند.

\subsection{چرخه باطل فساد-بی‌اعتمادی}

\begin{figure}[H]
\centering
\begin{tikzpicture}[
    node distance=2.5cm,
    cyclebox/.style={
        rectangle,
        rounded corners=5pt,
        minimum width=3cm,
        minimum height=1.3cm,
        text centered,
        font=\small,
        draw=rougerevolution,
        fill=rougelight,
        line width=1.5pt
    },
    arrow/.style={->, >=Stealth, very thick, color=rougerevolution}
]

\node[cyclebox] (a) at (0,0) {فساد گسترده};
\node[cyclebox] (b) at (4.5,2) {ناکارآمدی دولت};
\node[cyclebox] (c) at (9,0) {بی‌اعتمادی مردم};
\node[cyclebox] (d) at (4.5,-2) {فرار از مالیات};

\draw[arrow] (a) -- (b);
\draw[arrow] (b) -- (c);
\draw[arrow] (c) -- (d);
\draw[arrow] (d) -- (a);

\node[font=\small, color=rougerevolution] at (2.2,1.5) {کاهش کیفیت خدمات};
\node[font=\small, color=rougerevolution] at (7,1.5) {چرا مالیات بدهم؟};
\node[font=\small, color=rougerevolution] at (7,-1.5) {کاهش درآمد دولت};
\node[font=\small, color=rougerevolution] at (2.2,-1.5) {بیشتر دزدی می‌شود};

\node[font=\large\bfseries, color=rougerevolution] at (4.5,0) {چرخه باطل};

\end{tikzpicture}
\caption{چرخه باطل فساد و بی‌اعتمادی}
\label{fig:corruption-cycle}
\end{figure}

\subsection{چرخه باطل آب-فقر-مهاجرت}

\begin{figure}[H]
\centering
\begin{tikzpicture}[
    node distance=2.5cm,
    cyclebox/.style={
        rectangle,
        rounded corners=5pt,
        minimum width=3cm,
        minimum height=1.3cm,
        text centered,
        font=\small,
        draw=rougerevolution,
        fill=rougelight,
        line width=1.5pt
    },
    arrow/.style={->, >=Stealth, very thick, color=rougerevolution}
]

\node[cyclebox] (a) at (0,0) {کمبود آب};
\node[cyclebox] (b) at (4,2.5) {
    \begin{tabular}{c}
    نابودی\\
    کشاورزی
    \end{tabular}
};
\node[cyclebox] (c) at (8,0) {بیکاری و فقر};
\node[cyclebox] (d) at (4,-2.5) {
    \begin{tabular}{c}
    مهاجرت به\\
    شهرها
    \end{tabular}
};

\draw[arrow] (a) -- (b);
\draw[arrow] (b) -- (c);
\draw[arrow] (c) -- (d);
\draw[arrow] (d) -- (a);

\node[font=\scriptsize, color=rougerevolution, text width=2cm, align=center] at (1.5,1.8) {محصولات از بین می‌روند};
\node[font=\scriptsize, color=rougerevolution, text width=2cm, align=center] at (6.5,1.8) {کشاورزان ورشکست می‌شوند};
\node[font=\scriptsize, color=rougerevolution, text width=2cm, align=center] at (6.5,-1.8) {روستا خالی می‌شود};
\node[font=\scriptsize, color=rougerevolution, text width=2.5cm, align=center] at (1.5,-1.8) {فشار بر آب شهری افزایش می‌یابد};

\end{tikzpicture}
\caption{چرخه باطل آب، فقر و مهاجرت}
\label{fig:water-poverty-cycle}
\end{figure}

\subsection{تبدیل به چرخه فاضل}

هدف این طرح، شکستن این چرخه‌های باطل و جایگزینی آنها با چرخه‌های فاضل است:

\begin{figure}[H]
\centering
\begin{tikzpicture}[
    node distance=2.5cm,
    cyclebox/.style={
        rectangle,
        rounded corners=5pt,
        minimum width=3cm,
        minimum height=1.3cm,
        text centered,
        font=\small,
        draw=vertnapoleon,
        fill=vertlight,
        line width=1.5pt
    },
    arrow/.style={->, >=Stealth, very thick, color=vertnapoleon}
]

\node[cyclebox] (a) at (0,0) {شفافیت};
\node[cyclebox] (b) at (4.5,2) {کارآمدی دولت};
\node[cyclebox] (c) at (9,0) {اعتماد مردم};
\node[cyclebox] (d) at (4.5,-2) {پرداخت مالیات};

\draw[arrow] (a) -- (b);
\draw[arrow] (b) -- (c);
\draw[arrow] (c) -- (d);
\draw[arrow] (d) -- (a);

\node[font=\small, color=vertnapoleon] at (2.2,1.5) {خدمات بهتر};
\node[font=\small, color=vertnapoleon] at (7,1.5) {رضایت شهروندان};
\node[font=\small, color=vertnapoleon] at (7,-1.5) {درآمد پایدار دولت};
\node[font=\small, color=vertnapoleon] at (2.2,-1.5) {امکان سرمایه‌گذاری};

\node[font=\large\bfseries, color=vertnapoleon] at (4.5,0) {چرخه فاضل};

\end{tikzpicture}
\caption{چرخه فاضل شفافیت و اعتماد}
\label{fig:trust-cycle}
\end{figure}

%══════════════════════════════════════════════════════════════════════════════
\section{روش‌شناسی این پژوهش}
\label{sec:methodology}
%══════════════════════════════════════════════════════════════════════════════

این کتاب یک پژوهش کاربردی است که از چندین روش بهره می‌گیرد:

\subsection{تحلیل تطبیقی-تاریخی}

مطالعه عمیق گذارهای دموکراتیک در ۱۵ کشور با تمرکز بر:
\begin{itemize}[nosep]
    \item گذارهای موفق: اسپانیا، کره جنوبی، آفریقای جنوبی، اندونزی، غنا، شیلی
    \item گذارهای ناموفق/ناقص: عراق، لیبی، مصر، یمن، ونزوئلا
    \item کشورهای با تنوع قومی: هند، سوئیس، کانادا، بلژیک، نیجریه
\end{itemize}

\subsection{چارچوب نظری}

\begin{table}[H]
\centering
\caption{نظریه‌های پایه این پژوهش}
\label{tab:theories}
\begin{tabular}{L{3.5cm} L{4cm} L{5cm}}
\toprule
\headmark حوزه & \headmark نظریه‌پرداز & \headmark مفهوم کلیدی \\
\midrule
\rowcolor{bleulight}
گذار دموکراتیک & Linz \& Stepan & تمایز گذار و تحکیم \\
مدیریت تنوع & Lijphart & توافق‌گرایی \\
\rowcolor{bleulight}
توسعه & Sen & توسعه به‌مثابه آزادی \\
نهادها & North & نقش نهادها در توسعه \\
\rowcolor{bleulight}
ملت‌سازی & Brubaker & هویت مدنی vs قومی \\
منابع مشترک & Ostrom & حکمرانی منابع \\
\bottomrule
\end{tabular}
\end{table}

\subsection{رویکرد طراحی مبتنی بر شواهد}

\begin{olgoobox}[title={\hfill \textbf{اصل راهنما}}]
هر توصیه در این کتاب باید حداقل یکی از این معیارها را داشته باشد:
\begin{itemize}[nosep]
    \item مبتنی بر تجربه موفق حداقل یک کشور مشابه
    \item پشتیبانی‌شده توسط پژوهش‌های معتبر آکادمیک
    \item توصیه‌شده توسط نهادهای بین‌المللی معتبر (UN, World Bank, IDEA)
    \item سازگار با واقعیات بومی (منابع، فرهنگ، ظرفیت)
\end{itemize}
\end{olgoobox}

%══════════════════════════════════════════════════════════════════════════════
\section{تاریخ از پایین و از بالا}
\label{sec:history-above-below}
%══════════════════════════════════════════════════════════════════════════════

این کتاب می‌کوشد دو منظر را ترکیب کند که معمولاً جدا از هم دیده می‌شوند:

\begin{figure}[H]
\centering
\begin{tikzpicture}[
    node distance=1.5cm,
    levelbox/.style={
        rectangle,
        rounded corners=3pt,
        minimum width=5cm,
        minimum height=1.5cm,
        text centered,
        font=\small,
        line width=1pt
    },
    arrow/.style={<->, >=Stealth, thick, color=violetempire}
]

% بالا
\node[levelbox, draw=bleurepublique, fill=bleulight] (top) at (0,3) {
    \begin{tabular}{c}
    \textbf{تاریخ از بالا}\\[3pt]
    نخبگان، رهبران، نهادها، سیاست‌ها
    \end{tabular}
};

% وسط
\node[levelbox, draw=violetempire, fill=violetlight, minimum width=6cm, minimum height=2cm] (mid) at (0,0) {
    \begin{tabular}{c}
    \textbf{تعامل و دیالکتیک}\\[5pt]
    ساختار و عاملیت\\
    فشار از پایین، پاسخ از بالا
    \end{tabular}
};

% پایین
\node[levelbox, draw=vertnapoleon, fill=vertlight] (bot) at (0,-3) {
    \begin{tabular}{c}
    \textbf{تاریخ از پایین}\\[3pt]
    مردم عادی، زندگی روزمره، ذهنیت‌ها
    \end{tabular}
};

\draw[arrow] (top) -- (mid);
\draw[arrow] (mid) -- (bot);

% توضیحات کناری
\node[font=\scriptsize, text width=3.5cm, align=right] at (-5,3) {
    تصمیمات کلان\\
    قانون‌گذاری\\
    دیپلماسی
};

\node[font=\scriptsize, text width=3.5cm, align=right] at (-5,-3) {
    تجربه زیسته\\
    فرهنگ سیاسی\\
    انتظارات و باورها
};

\end{tikzpicture}
\caption{ترکیب دو منظر: تاریخ از بالا و از پایین}
\label{fig:history-above-below}
\end{figure}

\subsection{چرا این ترکیب مهم است؟}

\begin{itemize}
    \item \textbf{درک عمیق‌تر:} تحلیل‌هایی که فقط بر نخبگان متمرکزند، نقش توده‌ها را نادیده می‌گیرند. تحلیل‌هایی که فقط بر جنبش‌های مردمی متمرکزند، نقش ساختارها و نهادها را کم‌اهمیت می‌شمارند.
    
    \item \textbf{طراحی بهتر:} سیاست‌هایی که بدون توجه به ذهنیت و انتظارات مردم طراحی شوند، محکوم به شکست‌اند. سیاست‌هایی که فقط بر خواست‌های فوری مردم تکیه کنند، ممکن است ناپایدار باشند.
    
    \item \textbf{مشروعیت:} یک نظام دموکراتیک هم باید از نظر نخبگان کارآمد باشد و هم از نظر مردم عادی مشروع و پاسخگو.
\end{itemize}

%══════════════════════════════════════════════════════════════════════════════
\section{ساختار این کتاب}
\label{sec:book-structure}
%══════════════════════════════════════════════════════════════════════════════

\begin{figure}[H]
\centering
\begin{tikzpicture}[
    node distance=0.8cm,
    partbox/.style={
        rectangle,
        rounded corners=3pt,
        minimum width=13cm,
        minimum height=1.2cm,
        text centered,
        font=\small\bfseries,
        text=white,
        line width=0pt,
        drop shadow
    },
    chapbox/.style={
        rectangle,
        minimum width=5.5cm,
        minimum height=0.8cm,
        text centered,
        font=\scriptsize,
        fill=white,
        draw=gray,
        line width=0.5pt
    }
]

% بخش ۱
\node[partbox, fill=phase1] (p1) at (0,0) {بخش اول: مبانی};
\node[chapbox, below=0.3cm of p1, xshift=-3.2cm] {فصل ۰-۱: خلاصه و مقدمه};
\node[chapbox, below=0.3cm of p1, xshift=3.2cm] {فصل ۲-۳: تشخیص و نظریه};

% بخش ۲
\node[partbox, fill=phase2, below=1.5cm of p1] (p2) {بخش دوم: تجارب جهانی};
\node[chapbox, below=0.3cm of p2, xshift=-3.2cm] {فصل ۴-۵: موفق و ناموفق};
\node[chapbox, below=0.3cm of p2, xshift=3.2cm] {فصل ۶: مدیریت آب};

% بخش ۳
\node[partbox, fill=phase3, below=1.5cm of p2] (p3) {بخش سوم: طرح جامع};
\node[chapbox, below=0.3cm of p3, xshift=-3.2cm] {فصل ۷: چشم‌انداز};
\node[chapbox, below=0.3cm of p3, xshift=3.2cm] {فصل ۸-۱۱: پنج فاز};

% بخش ۴
\node[partbox, fill=phase4, below=1.5cm of p3] (p4) {بخش چهارم: حوزه‌های تخصصی};
\node[chapbox, below=0.3cm of p4, xshift=-3.2cm] {فصل ۱۲: تنوع قومی};
\node[chapbox, below=0.3cm of p4, xshift=3.2cm] {فصل ۱۳-۱۴: اقتصاد و آب};

% بخش ۵
\node[partbox, fill=phase5, below=1.5cm of p4] (p5) {بخش پنجم: اجرا و پایش};
\node[chapbox, below=0.3cm of p5] {فصل ۱۵: نظام پایش و ارزیابی};

\end{tikzpicture}
\caption{ساختار کلی کتاب}
\label{fig:book-structure}
\end{figure}

\subsection{راهنمای خواندن}

\begin{table}[H]
\centering
\caption{راهنمای خواندن برای مخاطبان مختلف}
\label{tab:reading-guide}
\begin{tabular}{L{3cm} L{4cm} L{5.5cm}}
\toprule
\headmark مخاطب & \headmark اولویت خواندن & \headmark چرا؟ \\
\midrule
\rowcolor{bleulight}
سیاست‌گذار & فصل ۰، ۷، ۸، ۱۵ & تمرکز بر طرح عملیاتی و اجرا \\
فعال مدنی & فصل ۱، ۴، ۱۲ & درک مبانی و نقش جامعه مدنی \\
\rowcolor{bleulight}
پژوهشگر & فصل ۳، ۴، ۵ & عمق نظری و تطبیقی \\
شهروند علاقه‌مند & فصل ۰، ۱، ۷ & تصویر کلی و چشم‌انداز \\
\rowcolor{bleulight}
متخصص آب/محیط زیست & فصل ۶، ۱۴ & حوزه تخصصی \\
اقتصاددان & فصل ۱۳ & بازسازی اقتصادی \\
\bottomrule
\end{tabular}
\end{table}

%══════════════════════════════════════════════════════════════════════════════
\section{محدودیت‌ها و احتیاط‌ها}
\label{sec:limitations}
%══════════════════════════════════════════════════════════════════════════════

صداقت علمی ایجاب می‌کند که محدودیت‌های این اثر را صریحاً بیان کنیم:

\begin{noktebox}
\textbf{این کتاب چه نیست:}
\begin{itemize}[nosep]
    \item یک پیش‌بینی قطعی نیست — آینده غیرقابل پیش‌بینی است
    \item یک نسخه جادویی نیست — هیچ تضمینی برای موفقیت وجود ندارد
    \item یک برنامه خشک و انعطاف‌ناپذیر نیست — باید با شرایط تطبیق یابد
    \item بی‌طرف محض نیست — نویسنده متعهد به ارزش‌های دموکراتیک است
    \item کامل نیست — همیشه جای بهبود و نقد وجود دارد
\end{itemize}
\end{noktebox}

\begin{olgoobox}[title={\hfill \textbf{در عین حال:}}]
\begin{itemize}[nosep]
    \item این کتاب مبتنی بر بهترین شواهد موجود است
    \item تجارب متعدد جهانی را بررسی کرده است
    \item واقعیات بومی را در نظر گرفته است
    \item رویکردی واقع‌بینانه و تدریجی دارد
    \item قابل نقد و اصلاح است
\end{itemize}
\end{olgoobox}

\section{هم‌افزایی نسلی: پیوند تخصص و تجربه}
\label{sec:generational-synergy}

گذار دموکراتیک در ایران به معنای پایان شکاف بین «داخل» و «خارج» است. سرمایه اجتماعی ساکنان داخل کشور (که هزینه تغییر را پرداخته‌اند) باید با سرمایه دانش، شبکه بین‌المللی و توان مالی دیاسپورا گره بخورد. برای گذار موفق، نیاز به یک \textbf{مدل همکاری استراتژیک (Strategic Synergy Model)} داریم که در آن هر دو جبهه مکمل یکدیگر باشند.

\subsection{مدل نقشه راه همکاری میان‌نسلی}

این همکاری نباید صرفاً داوطلبانه یا توده‌ای باشد، بلکه باید در قالب نهادهای دوران گذار تعریف شود.

\begin{table}[htbp]
\centering
\caption{مدل همکاری استراتژیک داخل و خارج از کشور (دیاسپورا)}
\label{tab:generational-synergy}
\begin{tabular}{R{3cm} L{4.5cm} L{4.5cm}}
\toprule
\headmark حوزه & \headmark نقش ساکنان داخل & \headmark نقش دیاسپورا (خارج) \\
\midrule
\textbf{مدیریت اجرایی} & بدنه بوروکراتیک، حفظ ثبات اداری، شناخت موانع بومی & تخصص مشاوره‌ای، تدوین متدولوژی‌های مدرن حکمرانی، انتقال تکنولوژی \\
\rowcolor{gray!10}
\textbf{بازسازی اقتصادی} & مدیریت بازارهای محلی، جذب نیروی انسانی متخصص داخلی & تأمین ارز، ایجاد صندوق‌های سرمایه‌گذاری خطرپذیر (VC)، اتصال به زنجیره تأمین جهانی \\
\textbf{دیپلماسی و حقوق} & مشروعیت مردمی داخلی، نمایندگی نهادهای مدنی & لابی‌گری بین‌المللی، شناسایی حقوقی نظام جدید، دیپلماسی عمومی در غرب \\
\rowcolor{gray!10}
\textbf{آموزش و تخصص} & کادرسازی در دانشگاه‌های ملی، بازآموزی بدنه اداری & تدریس آنلاین/حضوری، انتقال دانش روز، بورسیه نخبگان برای دوره‌های تخصصی \\
\bottomrule
\end{tabular}
\end{table}

\subsection{نهادینه کردن همکاری: صندوق توسعه دانش}

پیشنهاد می‌شود در همان ۱۰۰ روز اول، «صندوق توسعه و انتقال دانش» تأسیس شود تا:
\begin{itemize}[nosep]
    \item پلتفرم متمرکز برای ثبت‌نام متخصصان خارج از کشور برای همکاری در پروژه‌های بازسازی.
    \item مکانیزم «بازگشت معکوس» برای پروژه‌های کوتاه‌مدت و میان‌مدت بدون نیاز به اقامت دائم.
    \item ایجاد پیوند مستقیم بین استارتاپ‌های داخلی و سرمایه‌گذاران فرشته (Angel Investors) در دیاسپورا.
\end{itemize}

این هم‌افزایی، نه یک انتخاب، بلکه ضرورتی حیاتی برای بازسازی ایران در فازهای مختلف گذار است. فاز «آبادانی ملموس» بدون پیوند این دو بال، با سرعتی بسیار کمتر از انتظار پیش خواهد رفت.

%══════════════════════════════════════════════════════════════════════════════
\section{دعوت به گفتگو}
\label{sec:invitation}
%══════════════════════════════════════════════════════════════════════════════

این کتاب آغاز یک گفتگوست، نه پایان آن. نویسنده امیدوار است که این اثر بتواند:

\begin{itemize}
    \item زمینه‌ساز بحث جدی درباره آینده باشد
    \item الهام‌بخش طرح‌های بهتر و دقیق‌تر شود
    \item نقد شود و از دل نقدها ایده‌های بهتری بیرون آید
    \item پلی باشد بین تجربه جهانی و واقعیت بومی
\end{itemize}

\begin{naghlbox}
«بهترین زمان برای کاشتن درخت بیست سال پیش بود. دومین بهترین زمان، همین الان است.»

\hfill --- ضرب‌المثل چینی
\end{naghlbox}

\vspace{0.5cm}

آینده ساخته می‌شود. پرسش این است که چه کسانی و با چه طرحی آن را می‌سازند. این کتاب تلاشی است برای پاسخ به این پرسش.

%══════════════════════════════════════════════════════════════════════════════
\section*{منابع فصل}
%══════════════════════════════════════════════════════════════════════════════

\begin{enumerate}[nosep, label={[\arabic*]}]
    \item Acemoglu, D. \& Robinson, J. (2019). "Democracy Does Cause Growth." \textit{Journal of Political Economy}, 127(1).
    
    \item Huntington, S. (1991). \textit{The Third Wave: Democratization in the Late Twentieth Century}. University of Oklahoma Press.
    
    \item Linz, J. \& Stepan, A. (1996). \textit{Problems of Democratic Transition and Consolidation}. Johns Hopkins University Press.
    
    \item Sen, A. (1999). \textit{Development as Freedom}. Oxford University Press.
    
    \item Przeworski, A. et al. (2000). \textit{Democracy and Development}. Cambridge University Press.
    
    \item North, D. (1990). \textit{Institutions, Institutional Change and Economic Performance}. Cambridge University Press.
    
    \item Lijphart, A. (2012). \textit{Patterns of Democracy}. 2nd ed. Yale University Press.
    
    \item Diamond, L. (2008). \textit{The Spirit of Democracy}. Times Books.
    
    \item Freedom House. (2024). \textit{Freedom in the World 2024}. freedomhouse.org.
    
    \item World Bank. (2023). \textit{World Development Indicators}. data.worldbank.org.
\end{enumerate}
	%══════════════════════════════════════════════════════════════════════════════
% فصل ۲: تشخیص — تحلیل وضعیت موجود
% از بحران تا بالندگی
%══════════════════════════════════════════════════════════════════════════════

\chapter{تشخیص: تحلیل وضعیت موجود}
\label{ch:diagnosis}

%──────────────────────────────────────────────────────────────────────────────
% کادر خلاصه فصل
%──────────────────────────────────────────────────────────────────────────────
\begin{kholasebox}
این فصل تصویری جامع از بحران‌های چندلایه کشور ارائه می‌دهد. شش بحران اصلی — آب، انرژی، اقتصاد، سیاست، اجتماع، و امنیت — نه جدا از هم، بلکه در یک شبکه علّی به‌هم‌پیوسته‌اند و یکدیگر را تشدید می‌کنند. بحران آب با فروپاشی ۷۰٪ سفره‌های زیرزمینی، یک تهدید وجودی است. اقتصاد اسیر تحریم، تورم مزمن، و فساد سیستماتیک است. بی‌اعتمادی تاریخی اقوام به مرکز، همراه با سرکوب سیاسی، زمینه‌ساز شکاف‌های عمیق شده است. در عین حال، فرصت‌هایی نیز وجود دارد: جمعیت جوان تحصیل‌کرده، موقعیت جغرافیایی استراتژیک، منابع طبیعی، و میراث تمدنی غنی. تحلیل SWOT نشان می‌دهد که گذار ممکن است، اما نیازمند اقدام فوری و هماهنگ است.
\end{kholasebox}

%══════════════════════════════════════════════════════════════════════════════
\section{مقدمه: شش بحران به‌هم‌پیوسته}
\label{sec:six-crises}
%══════════════════════════════════════════════════════════════════════════════

پیش از طراحی هر راه‌حلی، باید تشخیص دقیقی از وضعیت داشته باشیم. همان‌طور که پزشک بدون تشخیص صحیح نمی‌تواند درمان کند، هیچ طرح سیاسی بدون فهم عمیق از واقعیات موفق نخواهد بود.

کشور ما با شش بحران اصلی مواجه است که در یک شبکه پیچیده به هم متصل‌اند:

\begin{figure}[H]
\centering
\begin{tikzpicture}[
    node distance=2cm,
    crisis/.style={
        rectangle,
        rounded corners=5pt,
        minimum width=2.8cm,
        minimum height=1.4cm,
        text centered,
        font=\small\bfseries,
        draw=rougerevolution,
        fill=rougelight,
        line width=1.5pt
    },
    link/.style={<->, >=Stealth, thick, color=rougemid, shorten >=2pt, shorten <=2pt}
]

% شش بحران در شش‌ضلعی
\node[crisis] (c1) at (90:4cm) {
    \begin{tabular}{c}
    بحران آب\\
    {\scriptsize تهدید وجودی}
    \end{tabular}
};
\node[crisis] (c2) at (30:4cm) {
    \begin{tabular}{c}
    بحران انرژی\\
    {\scriptsize قاچاق و کمبود}
    \end{tabular}
};
\node[crisis] (c3) at (330:4cm) {
    \begin{tabular}{c}
    بحران اقتصادی\\
    {\scriptsize تحریم و تورم}
    \end{tabular}
};
\node[crisis] (c4) at (270:4cm) {
    \begin{tabular}{c}
    بحران سیاسی\\
    {\scriptsize فساد و انسداد}
    \end{tabular}
};
\node[crisis] (c5) at (210:4cm) {
    \begin{tabular}{c}
    بحران اجتماعی\\
    {\scriptsize بی‌اعتمادی}
    \end{tabular}
};
\node[crisis] (c6) at (150:4cm) {
    \begin{tabular}{c}
    بحران امنیتی\\
    {\scriptsize ژئوپلیتیک}
    \end{tabular}
};

% اتصالات همه به همه
\draw[link] (c1) -- (c2);
\draw[link] (c2) -- (c3);
\draw[link] (c3) -- (c4);
\draw[link] (c4) -- (c5);
\draw[link] (c5) -- (c6);
\draw[link] (c6) -- (c1);

% اتصالات قطری
\draw[link, dashed, color=gray] (c1) -- (c4);
\draw[link, dashed, color=gray] (c2) -- (c5);
\draw[link, dashed, color=gray] (c3) -- (c6);

% مرکز
\node[circle, minimum size=2cm, fill=black!80, text=white, font=\small\bfseries] at (0,0) {
    \begin{tabular}{c}
    بحران\\
    سیستمی
    \end{tabular}
};

\end{tikzpicture}
\caption{شبکه شش بحران به‌هم‌پیوسته}
\label{fig:six-crises}
\end{figure}

%══════════════════════════════════════════════════════════════════════════════
\section{بحران آب و محیط زیست: تهدید وجودی}
\label{sec:water-crisis}
%══════════════════════════════════════════════════════════════════════════════

\begin{enghelabbox}[title={\hfill \textbf{هشدار: نقطه بی‌بازگشت}}]
بحران آب یک بحران معمولی نیست؛ این یک \textbf{تهدید وجودی} است. برخلاف بحران‌های اقتصادی یا سیاسی که قابل بازگشت‌اند، تخلیه سفره‌های زیرزمینی و نابودی اکوسیستم‌ها می‌تواند غیرقابل برگشت باشد. ما در حال نزدیک شدن به نقاط بی‌بازگشت هستیم.
\end{enghelabbox}

\subsection{وضعیت سفره‌های زیرزمینی}

\begin{table}[H]
\centering
\caption{وضعیت سفره‌های زیرزمینی کشور}
\label{tab:aquifer-status}
\begin{tabular}{L{3cm} C{2cm} C{2cm} C{2cm} L{3.5cm}}
\toprule
\headmark وضعیت & \headmark تعداد & \headmark درصد & \headmark روند & \headmark پیامد \\
\midrule
\rowcolor{rougelight}
ممنوعه (بحرانی) & ۳۵۴ & ۵۵\% & رو به وخامت & فروپاشی قریب‌الوقوع \\
\rowcolor{orroyallight}
بحرانی & ۱۰۵ & ۱۶\% & رو به وخامت & نیاز به اقدام فوری \\
\rowcolor{bleulight}
نیمه‌بحرانی & ۸۷ & ۱۴\% & رو به وخامت & نیاز به مدیریت \\
\rowcolor{vertlight}
عادی & ۹۳ & ۱۵\% & نسبتاً پایدار & حفظ وضعیت \\
\midrule
\textbf{جمع} & \textbf{۶۳۹} & \textbf{۱۰۰\%} & & \\
\bottomrule
\end{tabular}
\end{table}

\subsection{روند تخلیه تاریخی}

\begin{figure}[H]
\centering
\begin{tikzpicture}
\begin{axis}[
    width=14cm,
    height=7cm,
    xlabel={سال},
    ylabel={میلیارد مترمکعب (برداشت مازاد سالانه)},
    xmin=1970, xmax=2025,
    ymin=0, ymax=8,
    legend style={at={(0.02,0.98)}, anchor=north west},
    grid=both,
    grid style={line width=0.2pt, draw=gray!30},
    major grid style={line width=0.4pt, draw=gray!50}
]

% داده‌های برداشت مازاد
\addplot[color=rougerevolution, very thick, mark=*] coordinates {
    (1970,0.5) (1980,1.2) (1990,2.5) (2000,4.0) (2010,5.5) (2020,6.8) (2024,7.2)
};

% خط هشدار
\addplot[color=orroyal, thick, dashed] coordinates {
    (1970,3) (2025,3)
};

\legend{برداشت مازاد سالانه, آستانه بحرانی}

\node[font=\small, color=rougerevolution, anchor=west] at (axis cs:2010,6.5) {روند صعودی خطرناک};

\end{axis}
\end{tikzpicture}
\caption{روند برداشت مازاد از سفره‌های زیرزمینی (۱۹۷۰-۲۰۲۴)}
\label{fig:aquifer-depletion}
\end{figure}

\subsection{پیامدهای انسانی}

\begin{databox}
\textbf{پیامدهای بحران آب تا سال ۲۰۴۰ (در صورت ادامه روند فعلی):}
\begin{itemize}[nosep]
    \item ۳۰-۵۰ میلیون نفر مهاجرت اجباری داخلی (آوارگی اقلیمی)
    \item از دست رفتن ۷۰٪ اراضی کشاورزی
    \item غیرقابل سکونت شدن ۱۵-۲۰ استان
    \item خسارت اقتصادی سالانه ۵۰+ میلیارد دلار
    \item افزایش تنش‌های اجتماعی و قومی بر سر آب
\end{itemize}
\end{databox}

\subsection{توزیع جغرافیایی بحران}

\begin{table}[H]
\centering
\caption{شدت بحران آب به تفکیک مناطق}
\label{tab:water-regional}
\begin{tabular}{L{2.5cm} C{2cm} C{2cm} L{4cm} C{2cm}}
\toprule
\headmark منطقه & \headmark جمعیت (م) & \headmark وضعیت & \headmark چالش اصلی & \headmark اولویت \\
\midrule
\rowcolor{rougelight}
فلات مرکزی & ۳۵ & بحرانی & فرونشست شدید & فوری \\
\rowcolor{rougelight}
شرق و جنوب‌شرق & ۱۵ & بحرانی & خشکسالی مزمن & فوری \\
\rowcolor{orroyallight}
غرب & ۱۲ & نیمه‌بحرانی & کاهش بارش & میان‌مدت \\
\rowcolor{bleulight}
شمال & ۲۰ & متوسط & مدیریت مصرف & میان‌مدت \\
\rowcolor{vertlight}
ساحلی جنوب & ۸ & قابل قبول & شوری & بلندمدت \\
\bottomrule
\end{tabular}
\end{table}

%══════════════════════════════════════════════════════════════════════════════
\section{بحران انرژی و سوخت}
\label{sec:energy-crisis}
%══════════════════════════════════════════════════════════════════════════════

\subsection{ساختار مصرف و تولید}

\begin{figure}[H]
\centering
\begin{tikzpicture}
\begin{axis}[
    ybar stacked,
    width=12cm,
    height=7cm,
    ylabel={درصد},
    xlabel={بخش},
    symbolic x coords={تولید, مصرف داخلی, صادرات, قاچاق/اتلاف},
    xtick=data,
    ymin=0, ymax=110,
    legend style={at={(0.5,-0.2)}, anchor=north, legend columns=4},
    bar width=30pt,
    nodes near coords,
    every node near coord/.append style={font=\tiny}
]

\addplot[fill=chart1] coordinates {(تولید,100) (مصرف داخلی,60) (صادرات,15) (قاچاق/اتلاف,0)};
\addplot[fill=chart2] coordinates {(تولید,0) (مصرف داخلی,0) (صادرات,0) (قاچاق/اتلاف,15)};
\addplot[fill=chart4] coordinates {(تولید,0) (مصرف داخلی,0) (صادرات,0) (قاچاق/اتلاف,10)};

\legend{نفت و گاز, قاچاق, اتلاف}

\end{axis}
\end{tikzpicture}
\caption{ساختار انرژی کشور}
\label{fig:energy-structure}
\end{figure}

\subsection{مشکل قاچاق سوخت}

\begin{table}[H]
\centering
\caption{برآورد قاچاق سوخت}
\label{tab:fuel-smuggling}
\begin{tabular}{L{3cm} C{3cm} C{3cm} C{3cm}}
\toprule
\headmark نوع سوخت & \headmark قاچاق روزانه & \headmark ارزش سالانه & \headmark مقصد اصلی \\
\midrule
\rowcolor{bleulight}
بنزین & ۱۰ م. لیتر & ۳ میلیارد \$ & همسایه شرقی \\
گازوئیل & ۱۵ م. لیتر & ۴ میلیارد \$ & همسایه غربی \\
\rowcolor{bleulight}
گاز مایع & ۵ م. لیتر & ۱ میلیارد \$ & چند کشور \\
\midrule
\textbf{جمع} & \textbf{۳۰ م. لیتر} & \textbf{۸ میلیارد \$} & \\
\bottomrule
\end{tabular}
\end{table}

\begin{naghlbox}
«قاچاق سوخت فقط یک مسئله اقتصادی نیست؛ این یک شبکه سازمان‌یافته است که با فساد اداری، ضعف مرزبانی، و اختلاف قیمت داخلی-خارجی تغذیه می‌شود. ریشه‌کنی آن بدون حل همزمان این سه مسئله غیرممکن است.»

\hfill --- تحلیل ساختاری
\end{naghlbox}

%══════════════════════════════════════════════════════════════════════════════
\section{بحران اقتصادی و تحریم‌ها}
\label{sec:economic-crisis}
%══════════════════════════════════════════════════════════════════════════════

\subsection{ساختار اقتصاد رانتی}

\begin{figure}[H]
\centering
\begin{tikzpicture}[
    node distance=1.5cm,
    rentbox/.style={
        rectangle,
        rounded corners=3pt,
        minimum width=3cm,
        minimum height=1.2cm,
        text centered,
        font=\small,
        draw=rougerevolution,
        fill=rougelight,
        line width=1pt
    },
    arrow/.style={->, >=Stealth, thick, color=rougerevolution}
]

\node[rentbox] (a) at (0,0) {
    \begin{tabular}{c}
    وابستگی به\\
    درآمد نفت
    \end{tabular}
};

\node[rentbox] (b) at (4.5,2) {
    \begin{tabular}{c}
    دولت بزرگ و\\
    ناکارآمد
    \end{tabular}
};

\node[rentbox] (c) at (9,0) {
    \begin{tabular}{c}
    فساد و\\
    رانت‌جویی
    \end{tabular}
};

\node[rentbox] (d) at (4.5,-2) {
    \begin{tabular}{c}
    ضعف بخش\\
    خصوصی مولد
    \end{tabular}
};

\draw[arrow] (a) -- (b);
\draw[arrow] (b) -- (c);
\draw[arrow] (c) -- (d);
\draw[arrow] (d) -- (a);

\node[font=\bfseries, color=rougerevolution] at (4.5,0) {نفرین منابع};

\end{tikzpicture}
\caption{چرخه باطل اقتصاد رانتی (نفرین منابع)}
\label{fig:rentier-cycle}
\end{figure}

\subsection{تأثیر تحریم‌ها}

\begin{table}[H]
\centering
\caption{تأثیر تحریم‌ها بر شاخص‌های کلیدی}
\label{tab:sanctions-impact}
\begin{tabular}{L{3.5cm} C{2.5cm} C{2.5cm} C{2.5cm}}
\toprule
\headmark شاخص & \headmark قبل از تحریم & \headmark بعد از تحریم & \headmark تغییر \\
\midrule
\rowcolor{rougelight}
صادرات نفت (م.ب/روز) & ۲.۵ & ۰.۵ & -۸۰\% \\
\rowcolor{rougelight}
نرخ ارز (تومان/دلار) & ۳,۵۰۰ & ۵۰,۰۰۰+ & +۱۳۰۰\% \\
\rowcolor{orroyallight}
تورم سالانه & ۱۵\% & ۴۵\%+ & +۳۰۰\% \\
\rowcolor{orroyallight}
GDP سرانه (\$) & ۷,۰۰۰ & ۳,۰۰۰ & -۵۷\% \\
\rowcolor{bleulight}
سرمایه‌گذاری خارجی & ۵ میلیارد & نزدیک صفر & -۹۵\% \\
\bottomrule
\end{tabular}
\end{table}

\subsection{شاخص‌های فقر و نابرابری}

\begin{figure}[H]
\centering
\begin{tikzpicture}
\begin{axis}[
    width=13cm,
    height=6cm,
    xlabel={سال},
    ylabel={درصد جمعیت زیر خط فقر},
    xmin=2010, xmax=2024,
    ymin=0, ymax=35,
    legend style={at={(0.98,0.98)}, anchor=north east},
    grid=both,
    grid style={line width=0.2pt, draw=gray!30}
]

\addplot[color=rougerevolution, very thick, mark=square*] coordinates {
    (2010,8) (2012,10) (2014,12) (2016,15) (2018,22) (2020,28) (2022,30) (2024,32)
};

\addplot[color=orroyal, thick, dashed] coordinates {
    (2010,15) (2024,15)
};

\legend{نرخ فقر واقعی, آستانه هشدار}

\end{axis}
\end{tikzpicture}
\caption{روند افزایش فقر (۲۰۱۰-۲۰۲۴)}
\label{fig:poverty-trend}
\end{figure}

%══════════════════════════════════════════════════════════════════════════════
\section{بحران سیاسی و فساد}
\label{sec:political-crisis}
%══════════════════════════════════════════════════════════════════════════════

\subsection{فساد مویرگی}

فساد در کشور ما دیگر یک «انحراف» نیست؛ به یک «سیستم» تبدیل شده است.

\begin{table}[H]
\centering
\caption{لایه‌های فساد سیستماتیک}
\label{tab:corruption-layers}
\begin{tabular}{C{1.5cm} L{3cm} L{5cm} L{3.5cm}}
\toprule
\headmark سطح & \headmark بازیگران & \headmark مکانیزم & \headmark حجم تخمینی \\
\midrule
\rowcolor{rougelight}
کلان & الیگارشی حکومتی & قراردادهای بزرگ، رانت واردات & ۱۰+ میلیارد \$/سال \\
\rowcolor{orroyallight}
میانی & مدیران و کارمندان ارشد & رشوه، سوءاستفاده از موقعیت & ۵ میلیارد \$/سال \\
\rowcolor{bleulight}
خُرد & کارکنان خدماتی & رشوه روزمره، پارتی‌بازی & ۲ میلیارد \$/سال \\
\bottomrule
\end{tabular}
\end{table}

\subsection{شاخص‌های بین‌المللی}

\begin{table}[H]
\centering
\caption{جایگاه کشور در شاخص‌های بین‌المللی حکمرانی}
\label{tab:governance-indices}
\begin{tabular}{L{4cm} C{2cm} C{2cm} L{4cm}}
\toprule
\headmark شاخص & \headmark رتبه & \headmark امتیاز & \headmark وضعیت \\
\midrule
\rowcolor{rougelight}
Freedom House & --- & ۱۶/۱۰۰ & «غیرآزاد» \\
\rowcolor{rougelight}
شاخص دموکراسی (EIU) & ۱۵۴/۱۶۷ & ۲.۲/۱۰ & «اقتدارگرا» \\
\rowcolor{rougelight}
ادراک فساد (CPI) & ۱۴۷/۱۸۰ & ۲۵/۱۰۰ & «بسیار فاسد» \\
\rowcolor{orroyallight}
آزادی مطبوعات & ۱۷۶/۱۸۰ & --- & «وضعیت وخیم» \\
\rowcolor{orroyallight}
حاکمیت قانون (WJP) & ۱۳۸/۱۴۲ & ۰.۳۵/۱ & «بسیار ضعیف» \\
\bottomrule
\end{tabular}
\end{table}

\begin{olgoobox}[title={\hfill \textbf{نکته تطبیقی}}]
کشورهایی مانند گرجستان، رواندا و استونی نشان داده‌اند که فساد سیستماتیک قابل ریشه‌کنی است — اما نیازمند اراده سیاسی قوی، اصلاحات نهادی عمیق، و شفافیت رادیکال است. این کشورها طی ۱۰-۱۵ سال از رتبه‌های پایین به جایگاه‌های قابل قبول رسیدند.
\end{olgoobox}

%══════════════════════════════════════════════════════════════════════════════
\section{بحران اجتماعی و بی‌اعتمادی}
\label{sec:social-crisis}
%══════════════════════════════════════════════════════════════════════════════

\subsection{فروپاشی سرمایه اجتماعی}

\begin{figure}[H]
\centering
\begin{tikzpicture}[
    node distance=1.8cm,
    trustbox/.style={
        rectangle,
        rounded corners=3pt,
        minimum width=3cm,
        minimum height=1.3cm,
        text centered,
        font=\small,
        line width=1pt
    },
    arrow/.style={->, >=Stealth, thick}
]

% سطوح اعتماد
\node[trustbox, draw=rougerevolution, fill=rougelight] (t1) at (0,0) {
    \begin{tabular}{c}
    اعتماد به دولت\\
    {\scriptsize کمتر از ۱۵\%}
    \end{tabular}
};

\node[trustbox, draw=rougerevolution, fill=rougelight] (t2) at (5,0) {
    \begin{tabular}{c}
    اعتماد به نهادها\\
    {\scriptsize کمتر از ۲۰\%}
    \end{tabular}
};

\node[trustbox, draw=orroyal, fill=orroyallight] (t3) at (10,0) {
    \begin{tabular}{c}
    اعتماد بین‌فردی\\
    {\scriptsize حدود ۳۰\%}
    \end{tabular}
};

\node[trustbox, draw=vertnapoleon, fill=vertlight] (t4) at (5,-3) {
    \begin{tabular}{c}
    اعتماد به خانواده\\
    {\scriptsize بالای ۸۰\%}
    \end{tabular}
};

% فلش‌ها
\draw[arrow, color=rougerevolution] (t1) -- node[above, font=\scriptsize] {پایین} (t2);
\draw[arrow, color=orroyal] (t2) -- node[above, font=\scriptsize] {متوسط} (t3);
\draw[arrow, color=vertnapoleon] (t3) -- node[right, font=\scriptsize] {بالا} (t4);

\node[font=\small, text width=4cm, align=center, color=gris] at (5,-5) {
    تنها نهاد با اعتماد بالا: خانواده\\
    (پناهگاه در برابر بی‌اعتمادی عمومی)
};

\end{tikzpicture}
\caption{سطوح مختلف اعتماد در جامعه}
\label{fig:trust-levels}
\end{figure}

\subsection{شکاف‌های اجتماعی}

\begin{table}[H]
\centering
\caption{شکاف‌های اصلی اجتماعی}
\label{tab:social-cleavages}
\begin{tabular}{L{2.5cm} L{4cm} L{3.5cm} C{2.5cm}}
\toprule
\headmark شکاف & \headmark طرفین & \headmark تنش اصلی & \headmark شدت \\
\midrule
\rowcolor{rougelight}
نسلی & جوانان vs مسن‌ترها & ارزش‌ها، فرصت‌ها & بالا \\
\rowcolor{rougelight}
قومی & مرکز vs پیرامون & قدرت، منابع، هویت & بالا \\
\rowcolor{orroyallight}
طبقاتی & فقرا vs ثروتمندان & توزیع، فرصت & متوسط-بالا \\
\rowcolor{orroyallight}
جنسیتی & زنان vs مردان & حقوق، مشارکت & متوسط-بالا \\
\rowcolor{bleulight}
شهر-روستا & شهری vs روستایی & خدمات، توجه & متوسط \\
\rowcolor{bleulight}
مذهبی & سنتی vs مدرن & سبک زندگی & متوسط \\
\bottomrule
\end{tabular}
\end{table}

%══════════════════════════════════════════════════════════════════════════════
\section{بحران هویتی و تنوع قومی}
\label{sec:ethnic-crisis}
%══════════════════════════════════════════════════════════════════════════════

\subsection{ترکیب قومی-زبانی}

\begin{figure}[H]
\centering
\begin{tikzpicture}
\pie[
    text=legend,
    radius=3,
    color={chart1, chart2, chart3, chart4, chart5, gray!50},
    explode={0.1, 0, 0, 0, 0, 0}
]{
    50/قوم اصلی الف,
    20/قوم ب,
    15/قوم ج,
    8/قوم د,
    4/قوم ه,
    3/سایر اقلیت‌ها
}
\end{tikzpicture}
\caption{ترکیب تقریبی قومی-زبانی کشور}
\label{fig:ethnic-composition}
\end{figure}

\subsection{ریشه‌های بی‌اعتمادی تاریخی به مرکز}

\begin{table}[H]
\centering
\caption{دلایل تاریخی بی‌اعتمادی اقوام به مرکز}
\label{tab:ethnic-distrust}
\begin{tabular}{C{1cm} L{3.5cm} L{5cm} L{3.5cm}}
\toprule
\headmark \# & \headmark عامل & \headmark توضیح & \headmark پیامد \\
\midrule
\rowcolor{bleulight}
۱ & سیاست یکسان‌سازی & تحمیل زبان و فرهنگ واحد & از بین رفتن تنوع \\
۲ & تمرکزگرایی شدید & همه تصمیمات در پایتخت & احساس بی‌قدرتی \\
\rowcolor{bleulight}
۳ & توزیع ناعادلانه منابع & محرومیت مناطق قومی & نارضایتی اقتصادی \\
۴ & سرکوب هویتی & ممنوعیت زبان و لباس & خشم فروخورده \\
\rowcolor{bleulight}
۵ & نمایندگی ناکافی & غیبت در پست‌های کلیدی & احساس شهروند درجه دو \\
۶ & تروماهای تاریخی & سرکوب‌های خشن گذشته & حافظه جمعی دردناک \\
\bottomrule
\end{tabular}
\end{table}

\begin{enghelabbox}[title={\hfill \textbf{هشدار: بمب ساعتی}}]
بی‌اعتمادی قومی یک بمب ساعتی است. تا زمانی که سرکوب ادامه دارد، این تنش‌ها پنهان می‌مانند. اما در لحظه گذار — وقتی فضا باز شود — ممکن است به شکل خشونت‌آمیز بروز کنند. تجربه یوگسلاوی، عراق، و سوریه هشداردهنده است. مدیریت هوشمند این تنوع، یکی از حیاتی‌ترین چالش‌های گذار است.
\end{enghelabbox}

%══════════════════════════════════════════════════════════════════════════════
\section{بحران امنیتی و ژئوپلیتیک}
\label{sec:security-crisis}
%══════════════════════════════════════════════════════════════════════════════

\subsection{محیط منطقه‌ای}

\begin{table}[H]
\centering
\caption{تحلیل محیط ژئوپلیتیک}
\label{tab:geopolitics}
\begin{tabular}{L{2.5cm} L{3cm} L{3.5cm} L{3.5cm}}
\toprule
\headmark بازیگر & \headmark رابطه فعلی & \headmark منافع & \headmark سناریوی گذار \\
\midrule
\rowcolor{bleulight}
همسایه شرقی & پیچیده & ثبات مرزی، تجارت & احتمالاً بی‌طرف \\
همسایه غربی & رقابتی & نفوذ منطقه‌ای & نگران \\
\rowcolor{bleulight}
همسایه شمالی & متحد ضعیف & انرژی، ترانزیت & محتاط \\
قدرت‌های غربی & تقابل & تغییر رژیم & حمایت مشروط \\
\rowcolor{bleulight}
روسیه & شریک محدود & توازن با غرب & دوگانه \\
چین & شریک اقتصادی & انرژی، راه ابریشم & پراگماتیک \\
\bottomrule
\end{tabular}
\end{table}

%══════════════════════════════════════════════════════════════════════════════
\section{تحلیل SWOT جامع}
\label{sec:swot}
%══════════════════════════════════════════════════════════════════════════════

\begin{landscape}
\begin{figure}[H]
\centering
\begin{tikzpicture}[
    swotbox/.style={
        rectangle,
        rounded corners=5pt,
        minimum width=9cm,
        minimum height=6cm,
        text width=8.5cm,
        font=\small,
        line width=1.5pt,
        align=right
    }
]

% قوت‌ها (بالا چپ)
\node[swotbox, draw=vertnapoleon, fill=vertlight] (s) at (0,4) {
    \textbf{\large قوت‌ها (Strengths)}
    \begin{itemize}[nosep, rightmargin=0.3cm]
        \item میراث تمدنی کهن و هویت‌ساز
        \item جمعیت جوان (۶۰\% زیر ۳۰ سال)
        \item نیروی کار تحصیل‌کرده
        \item موقعیت جغرافیایی استراتژیک
        \item منابع طبیعی متنوع (نفت، گاز، معادن)
        \item دیاسپورای قدرتمند و متخصص
        \item سنت‌های مشورتی تاریخی
        \item تجربه جنبش‌های مدنی
    \end{itemize}
};

% ضعف‌ها (بالا راست)
\node[swotbox, draw=rougerevolution, fill=rougelight] (w) at (10,4) {
    \textbf{\large ضعف‌ها (Weaknesses)}
    \begin{itemize}[nosep, rightmargin=0.3cm]
        \item فساد سیستماتیک و مویرگی
        \item نهادهای ضعیف و ناکارآمد
        \item بی‌اعتمادی عمیق اجتماعی
        \item تنش‌های قومی نهفته
        \item وابستگی به نفت (اقتصاد رانتی)
        \item فرار مغزها و سرمایه
        \item زیرساخت فرسوده
        \item فرهنگ سیاسی اقتدارگرا
    \end{itemize}
};

% فرصت‌ها (پایین چپ)
\node[swotbox, draw=bleurepublique, fill=bleulight] (o) at (0,-4) {
    \textbf{\large فرصت‌ها (Opportunities)}
    \begin{itemize}[nosep, rightmargin=0.3cm]
        \item فرسایش مشروعیت نظام موجود
        \item تغییرات ژئوپلیتیک منطقه‌ای
        \item امکان جذب کمک بین‌المللی
        \item بازار بزرگ داخلی (۸۵+ میلیون)
        \item پتانسیل انرژی‌های تجدیدپذیر
        \item امکان بازگشت دیاسپورا
        \item تقاضا برای تغییر در جامعه
        \item تجربه موفق کشورهای مشابه
    \end{itemize}
};

% تهدیدها (پایین راست)
\node[swotbox, draw=orroyal, fill=orroyallight] (t) at (10,-4) {
    \textbf{\large تهدیدها (Threats)}
    \begin{itemize}[nosep, rightmargin=0.3cm]
        \item بحران آب (نقطه بی‌بازگشت)
        \item تداوم تحریم‌های بین‌المللی
        \item مداخله قدرت‌های خارجی
        \item احتمال خشونت در گذار
        \item بازگشت اقتدارگرایی
        \item تجزیه‌طلبی افراطی
        \item فروپاشی اقتصادی
        \item محیط منطقه‌ای ناامن
    \end{itemize}
};

% عناوین گوشه‌ها
\node[font=\small, color=vertnapoleon] at (-4.5,7.5) {\textbf{داخلی مثبت}};
\node[font=\small, color=rougerevolution] at (14.5,7.5) {\textbf{داخلی منفی}};
\node[font=\small, color=bleurepublique] at (-4.5,-0.5) {\textbf{خارجی مثبت}};
\node[font=\small, color=orroyal] at (14.5,-0.5) {\textbf{خارجی منفی}};

\end{tikzpicture}
\caption{تحلیل SWOT جامع وضعیت کشور}
\label{fig:swot}
\end{figure}
\end{landscape}

%══════════════════════════════════════════════════════════════════════════════
\section{نتیجه‌گیری: تشخیص نهایی}
\label{sec:diagnosis-conclusion}
%══════════════════════════════════════════════════════════════════════════════

\begin{tahlilbox}[title={\hfill \textbf{جمع‌بندی تشخیص}}]
\textbf{وضعیت:} بیمار (کشور) با شش بحران به‌هم‌پیوسته مواجه است که یکدیگر را تشدید می‌کنند. برخی از این بحران‌ها (به‌ویژه آب) در آستانه نقطه بی‌بازگشت هستند.

\textbf{علت ریشه‌ای:} ترکیب اقتصاد رانتی، حکمرانی اقتدارگرا، و فساد سیستماتیک چرخه‌های باطلی ایجاد کرده که اصلاح از درون را تقریباً غیرممکن ساخته است.

\textbf{پیش‌آگهی:} بدون تغییر بنیادین، روند رو به وخامت ادامه خواهد یافت و در افق ۱۰-۲۰ ساله به فروپاشی منجر خواهد شد.

\textbf{درمان:} گذار دموکراتیک جامع همراه با برنامه بازسازی ملی — نه فقط تغییر سیاسی، بلکه تحول ساختاری در اقتصاد، حکمرانی، و رابطه دولت-ملت.
\end{tahlilbox}

%══════════════════════════════════════════════════════════════════════════════
\section*{منابع فصل}
%══════════════════════════════════════════════════════════════════════════════

\begin{enumerate}[nosep, label={[\arabic*]}]
    \item World Bank. (2023). \textit{World Development Indicators}. data.worldbank.org.
    
    \item Freedom House. (2024). \textit{Freedom in the World 2024}. freedomhouse.org.
    
    \item Transparency International. (2024). \textit{Corruption Perceptions Index}. transparency.org.
    
    \item Economist Intelligence Unit. (2024). \textit{Democracy Index 2023}. eiu.com.
    
    \item FAO. (2023). \textit{AQUASTAT Database}. fao.org/aquastat.
    
    \item World Resources Institute. (2023). \textit{Aqueduct Water Risk Atlas}. wri.org.
    
    \item IMF. (2024). \textit{World Economic Outlook}. imf.org.
    
    \item UNDP. (2023). \textit{Human Development Report}. hdr.undp.org.
    
    \item Reporters Without Borders. (2024). \textit{World Press Freedom Index}. rsf.org.
    
    \item World Justice Project. (2024). \textit{Rule of Law Index}. worldjusticeproject.org.
\end{enumerate}
	%══════════════════════════════════════════════════════════════════════════════
% فصل ۳: مبانی نظری — دموکراسی، تنوع، توسعه
% از بحران تا بالندگی
%══════════════════════════════════════════════════════════════════════════════

\chapter{مبانی نظری: دموکراسی، تنوع، توسعه}
\label{ch:theory}

%──────────────────────────────────────────────────────────────────────────────
% کادر خلاصه فصل
%──────────────────────────────────────────────────────────────────────────────
\begin{kholasebox}
این فصل چارچوب نظری طرح را تبیین می‌کند. ابتدا مفهوم دموکراسی را تعریف می‌کنیم: دموکراسی صرفاً انتخابات نیست، بلکه مجموعه‌ای از نهادها، رویه‌ها و فرهنگ است. سپس نظریه‌های گذار دموکراتیک را بررسی می‌کنیم و تمایز کلیدی بین «گذار» و «تحکیم» را توضیح می‌دهیم. در بخش مدیریت تنوع، دو مدل اصلی — توافق‌گرایی لیپهارت و یکپارچه‌سازی هورویتز — را مقایسه می‌کنیم و مدل ترکیبی مناسب پیشنهاد می‌دهیم. رابطه دموکراسی و توسعه از منظر آمارتیا سن بررسی می‌شود. در پایان، چارچوب نظری تلفیقی این کتاب ارائه می‌شود.
\end{kholasebox}

%══════════════════════════════════════════════════════════════════════════════
\section{دموکراسی چیست؟}
\label{sec:what-is-democracy}
%══════════════════════════════════════════════════════════════════════════════

\subsection{تعریف حداقلی (رویه‌ای)}

\begin{naghlbox}
«دموکراسی آن ترتیبات نهادی برای رسیدن به تصمیمات سیاسی است که در آن افراد از طریق رقابت برای کسب آرای مردم، به قدرت می‌رسند.»

\hfill --- ژوزف شومپیتر، \textit{سرمایه‌داری، سوسیالیسم و دموکراسی}، ۱۹۴۲
\end{naghlbox}

این تعریف «حداقلی» یا «رویه‌ای» بر دو عنصر تأکید دارد:
\begin{enumerate}[nosep]
    \item \textbf{رقابت:} چندین گروه/فرد برای قدرت رقابت می‌کنند
    \item \textbf{مشارکت:} شهروندان بالغ حق رأی دارند
\end{enumerate}

رابرت دال این تعریف را گسترش داد و هفت معیار برای «پُلیارشی» (حکومت چندگانه) تعریف کرد:

\begin{table}[H]
\centering
\caption{هفت معیار پُلیارشی دال}
\label{tab:dahl-polyarchy}
\begin{tabularx}{\textwidth}{C{1cm} R{3.5cm} Y}
\toprule
\headmark \# & \headmark معیار & \headmark توضیح \\
\midrule
۱ & مقامات انتخابی & تصمیم‌گیران اصلی از طریق انتخابات برگزیده می‌شوند \\
\rowcolor{goldlight}
۲ & انتخابات آزاد و منصفانه & بدون تقلب، ارعاب، یا محدودیت \\
۳ & حق رأی همگانی & همه بالغین حق رأی دارند \\
\rowcolor{goldlight}
۴ & حق نامزدی & همه می‌توانند نامزد شوند \\
۵ & آزادی بیان & انتقاد از حکومت بدون مجازات \\
\rowcolor{goldlight}
۶ & دسترسی به اطلاعات & منابع متنوع اطلاعاتی \\
۷ & آزادی تشکل & احزاب و انجمن‌های مستقل \\
\bottomrule
\end{tabularx}
\end{table}

\subsection{تعریف حداکثری (جوهری)}

اما آیا انتخابات کافی است؟ منتقدان می‌گویند «دموکراسی انتخاباتی» می‌تواند فاسد، ناکارآمد، یا سرکوبگر باشد (مانند روسیه پوتین یا ونزوئلای چاوز).

تعریف «جوهری» یا «لیبرال» دموکراسی، عناصر بیشتری را اضافه می‌کند:

\begin{figure}[H]
\centering
\begin{tikzpicture}[
    node distance=1.5cm,
    layer/.style={
        rectangle,
        rounded corners=5pt,
        minimum width=0.9\textwidth,
        minimum height=1.5cm,
        text centered,
        font=\small,
        line width=1.5pt
    }
]

% لایه‌ها از پایین به بالا
\node[layer, draw=bleurepublique!70, fill=bleulight] (l1) at (0,0) {
    \textbf{لایه ۱: انتخابات} — رقابت، مشارکت، آزادی
};

\node[layer, draw=bleurepublique!80, fill=bleulight] (l2) at (0,2) {
    \textbf{لایه ۲: حقوق} — آزادی‌های مدنی، حقوق اقلیت‌ها، برابری
};

\node[layer, draw=goldphoenix, fill=goldlight] (l3) at (0,4) {
    \textbf{لایه ۳: نهادها} — تفکیک قوا، استقلال قضایی، حاکمیت قانون
};

\node[layer, draw=goldphoenix, fill=goldphoenix, text=white] (l4) at (0,6) {
    \textbf{لایه ۴: فرهنگ} — تساهل، مدارا، فرهنگ مشارکت
};

% فلش‌ها
\draw[->, >=Stealth, very thick, color=gris] (l1) -- (l2);
\draw[->, >=Stealth, very thick, color=gris) (l2) -- (l3);
\draw[->, >=Stealth, very thick, color=gris] (l3) -- (l4);

\node[font=\small, color=bleurepublique, text width=3cm, align=center] at (6.5,3) {
    \rl{دموکراسی عمیق‌تر}\\
    $\uparrow$\\
    \rl{دموکراسی سطحی‌تر}
};

\end{tikzpicture}
\caption{لایه‌های دموکراسی: از حداقلی تا حداکثری}
\end{figure}

\subsection{طیف نظام‌های سیاسی}

\begin{figure}[H]
\centering
\begin{tikzpicture}
% خط طیف
\draw[line width=3pt, color=gris] (0,0) -- (14,0);

% نقاط
\foreach \x/\label in {0/توتالیتر, 2.8/اقتدارگرای بسته, 5.6/اقتدارگرای رقابتی, 8.4/دموکراسی انتخاباتی, 11.2/دموکراسی لیبرال, 14/دموکراسی کامل} {
    \draw[line width=2pt] (\x,-0.3) -- (\x,0.3);
}

% برچسب‌ها
\node[font=\tiny, text width=1.8cm, align=center, below] at (0,-0.4) {\rl{توتالیتر}\\{\tiny (کره شمالی)}};
\node[font=\tiny, text width=1.8cm, align=center, below] at (2.8,-0.4) {\rl{اقتدارگرای بسته}\\{\tiny (عربستان)}};
\node[font=\tiny, text width=1.8cm, align=center, below] at (5.6,-0.4) {\rl{اقتدارگرای رقابتی}\\{\tiny (روسیه)}};
\node[font=\tiny, text width=1.8cm, align=center, below] at (8.4,-0.4) {\rl{دموکراسی انتخاباتی}\\{\tiny (هند)}};
\node[font=\tiny, text width=1.8cm, align=center, below] at (11.2,-0.4) {\rl{دموکراسی لیبرال}\\{\tiny (آلمان)}};
\node[font=\tiny, text width=1.8cm, align=center, below] at (14,-0.4) {\rl{دموکراسی کامل}\\{\tiny (نروژ)}};

% رنگ‌بندی
\fill[rougerevolution, opacity=0.3] (0,-0.2) rectangle (2.8,0.2);
\fill[goldlight] (2.8,-0.2) rectangle (5.6,0.2);
\fill[goldphoenix, opacity=0.3] (5.6,-0.2) rectangle (8.4,0.2);
\fill[bleulight] (8.4,-0.2) rectangle (11.2,0.2);
\fill[bleurepublique, opacity=0.3] (11.2,-0.2) rectangle (14,0.2);

\end{tikzpicture}
\caption{طیف نظام‌های سیاسی}
\end{figure}

%══════════════════════════════════════════════════════════════════════════════
\section{نظریه‌های گذار دموکراتیک}
\label{sec:transition-theories}
%══════════════════════════════════════════════════════════════════════════════

\subsection{موج سوم دموکراتیزاسیون}

ساموئل هانتینگتون سه موج دموکراتیزاسیون را شناسایی کرد:

\begin{table}[H]
\centering
\caption{سه موج دموکراتیزاسیون هانتینگتون}
\label{tab:three-waves}
\begin{tabularx}{\textwidth}{C{1.5cm} C{2.5cm} Y Z}
\toprule
\headmark موج & \headmark دوره & \headmark ویژگی & \headmark نمونه‌ها \\
\midrule
اول & ۱۸۲۸-۱۹۲۶ & تدریجی، طبقه متوسط & آمریکا، انگلیس \\
\rowcolor{goldlight}
دوم & ۱۹۴۳-۱۹۶۲ & پس از جنگ & آلمان، ژاپن، هند \\
سوم & ۱۹۷۴-۱۹۹۱+ & سقوط اقتدارگرایی & اسپانیا، شرق اروپا \\
\bottomrule
\end{tabularx}
\end{table}

\subsection{مدل گذار لینز و استپان}

\begin{olgoobox}[title={\hfill \textbf{تمایز کلیدی: گذار و تحکیم}}]
خوان لینز و آلفرد استپان تمایز مهمی را مطرح کردند:
\begin{itemize}[nosep]
    \item \textbf{گذار (Transition):} حرکت از اقتدارگرایی به دموکراسی — پایان یافتن با اولین انتخابات آزاد
    \item \textbf{تحکیم (Consolidation):} نهادینه شدن دموکراسی — «تنها بازی در شهر» شدن
\end{itemize}

بسیاری از کشورها در گذار موفق می‌شوند اما در تحکیم شکست می‌خورند. مصر ۲۰۱۱-۲۰۱۳ نمونه بارز است.
\end{olgoobox}

لینز و استپان پنج عرصه (Arena) را برای تحکیم دموکراسی ضروری می‌دانند:

\begin{figure}[H]
\centering
\begin{tikzpicture}[
    node distance=1.5cm,
    arenabox/.style={
        rectangle,
        rounded corners=5pt,
        minimum width=4cm,
        minimum height=1.8cm,
        text centered,
        font=\small,
        line width=1.5pt
    },
    arrow/.style={<->, >=Stealth, thick, color=bleurepublique!40}
]

% پنج عرصه
\node[arenabox, draw=bleurepublique, fill=bleulight] (a1) at (0,0) {
    \begin{tabular}{c}
    \textbf{جامعه مدنی}\\[3pt]
    {\scriptsize انجمن‌ها، انجمن‌های آزاد}
    \end{tabular}
};

\node[arenabox, draw=bleurepublique, fill=bleulight] (a2) at (5.5,0) {
    \begin{tabular}{c}
    \textbf{جامعه سیاسی}\\[3pt]
    {\scriptsize احزاب و پارلمان}
    \end{tabular}
};

\node[arenabox, draw=goldphoenix, fill=goldlight] (a3) at (11,0) {
    \begin{tabular}{c}
    \textbf{حاکمیت قانون}\\[3pt]
    {\scriptsize قانون اساسی نوین}
    \end{tabular}
};

\node[arenabox, draw=bleurepublique, fill=bleulight] (a4) at (2.5,-3.5) {
    \begin{tabular}{c}
    \textbf{دستگاه دولتی}\\[3pt]
    {\scriptsize بوروکراسی کارآمد}
    \end{tabular}
};

\node[arenabox, draw=goldphoenix, fill=goldlight] (a5) at (8.5,-3.5) {
    \begin{tabular}{c}
    \textbf{جامعه اقتصادی}\\[3pt]
    {\scriptsize بازار نهادینه‌شده}
    \end{tabular}
};

% اتصالات
\draw[arrow] (a1) -- (a2);
\draw[arrow] (a2) -- (a3);
\draw[arrow] (a1) -- (a4);
\draw[arrow] (a2) -- (a4);
\draw[arrow] (a2) -- (a5);
\draw[arrow] (a3) -- (a5);
\draw[arrow] (a4) -- (a5);

% عنوان مرکزی
\node[font=\small\bfseries, color=gris] at (5.5,-1.75) {\rl{تعامل متقابل}};

\end{tikzpicture}
\caption{پنج عرصه تحکیم دموکراسی (لینز و استپان)}
\end{figure}

\begin{table}[H]
\centering
\caption{سه سطح تحکیم دموکراسی}
\label{tab:consolidation-levels}
\begin{tabularx}{\textwidth}{C{2cm} R{4cm} Y}
\toprule
\headmark سطح & \headmark شاخص & \headmark توضیح \\
\midrule
رفتاری & فقدان خشونت سیاسی & ارتش کودتا نمی‌کند \\
\rowcolor{goldlight}
نگرشی & حمایت اکثریت & دموکراسی بهترین نظام است \\
قانون اساسی & پذیرش قواعد بازی & بازندگان نتایج را می‌پذیرند \\
\bottomrule
\end{tabularx}
\end{table}

\subsection{مدل‌های گذار}

\begin{table}[H]
\centering
\caption{انواع گذار دموکراتیک}
\label{tab:transition-types}
\begin{tabularx}{\textwidth}{R{2.5cm} R{3.5cm} Y Z}
\toprule
\headmark نوع گذار & \headmark مکانیزم & \headmark نمونه & \headmark ویژگی \\
\midrule
از بالا & اصلاح طلبان & اسپانیا & کم‌خشونت \\
\rowcolor{goldlight}
از پایین & جنبش مردمی & لهستان & فشار توده‌ای \\
مذاکره‌ای & توافق نخبگان & آفریقای جنوبی & پایداری بالا \\
\rowcolor{goldlight}
فروپاشی & سقوط ناگهانی & شوروی & بی‌ثباتی \\
\bottomrule
\end{tabularx}
\end{table}

\begin{olgoobox}[title={\hfill \textbf{درس کلیدی}}]
گذارهای \textbf{مذاکره‌ای} که در آن اصلاح‌طلبان درون نظام با اپوزیسیون معتدل توافق می‌کنند، بالاترین نرخ موفقیت را دارند. اسپانیا و آفریقای جنوبی نمونه‌های بارزند. گذارهای ناشی از فروپاشی ناگهانی (مانند لیبی) معمولاً به هرج‌ومرج منجر می‌شوند.
\end{olgoobox}

%══════════════════════════════════════════════════════════════════════════════
\section{مدیریت تنوع قومی-فرهنگی}
\label{sec:diversity-management}
%══════════════════════════════════════════════════════════════════════════════

یکی از حیاتی‌ترین چالش‌های کشور ما مدیریت تنوع قومی-زبانی-فرهنگی است. دو مکتب اصلی در این زمینه وجود دارد:

\subsection{مدل توافق‌گرایی (لیپهارت)}

\begin{naghlbox}
«در جوامع عمیقاً تقسیم‌شده، دموکراسی اکثریتی به استبداد اکثریت منجر می‌شود. راه‌حل، دموکراسی توافق‌گرایانه است که در آن همه گروه‌های مهم در قدرت سهیم‌اند.»

\hfill --- آرنت لیپهارت، \textit{الگوهای دموکراسی}، ۱۹۹۹
\end{naghlbox}

چهار ستون توافق‌گرایی:

\begin{figure}[H]
\centering
\begin{tikzpicture}[
    pillar/.style={
        rectangle,
        minimum width=3cm,
        minimum height=5cm,
        draw=bleurepublique!70,
        fill=bleulight,
        line width=1.5pt
    },
    roof/.style={
        trapezium,
        trapezium angle=70,
        minimum width=14cm,
        minimum height=1.5cm,
        draw=goldphoenix,
        fill=goldlight,
        line width=2pt
    },
    base/.style={
        rectangle,
        minimum width=14cm,
        minimum height=1cm,
        draw=gris,
        fill=grislight,
        line width=1.5pt
    },
    label/.style={
        font=\small\bfseries,
        text width=2.5cm,
        align=center,
        text=bleurepublique
    }
]

% پایه
\node[base] (base) at (0,-3.5) {\rl{جامعه تقسیم‌شده}};

% ستون‌ها
\node[pillar] (p1) at (-5,0) {};
\node[pillar] (p2) at (-1.7,0) {};
\node[pillar] (p3) at (1.7,0) {};
\node[pillar] (p4) at (5,0) {};

% متن ستون‌ها
\node[label] at (-5,0) {\rl{ائتلاف بزرگ}};
\node[label] at (-1.7,0) {\rl{وتوی متقابل}};
\node[label] at (1.7,0) {\rl{تناسب}};
\node[label] at (5,0) {\rl{خودمختاری}};

% سقف
\node[roof] (roof) at (0,3.5) {\rl{دولت پایدار دموکراتیک}};

\end{tikzpicture}
\caption{چهار ستون توافق‌گرایی لیپهارت}
\end{figure}

\begin{table}[H]
\centering
\caption{نمونه‌های موفق توافق‌گرایی}
\label{tab:consociational-examples}
\begin{tabularx}{\textwidth}{R{2.5cm} R{3cm} Y Z}
\toprule
\headmark کشور & \headmark نوع تقسیم & \headmark مکانیزم & \headmark نتیجه \\
\midrule
سوئیس & زبانی & فدرالیسم مستقیم & بسیار موفق \\
\rowcolor{goldlight}
بلژیک & زبانی & فدرالیسم + وتو & موفق با تنش \\
هلند & مذهبی & ستون‌بندی & موفق \\
\rowcolor{goldlight}
لبنان & فرقه‌ای & سهمیه‌بندی & مختلط \\
\bottomrule
\end{tabularx}
\end{table}

\subsection{مدل یکپارچه‌سازی (هورویتز)}

دونالد هورویتز منتقد توافق‌گرایی است و می‌گوید این مدل تقسیمات قومی را نهادینه می‌کند:

\begin{table}[H]
\centering
\caption{مقایسه دو مدل مدیریت تنوع}
\label{tab:consoc-vs-integrative}
\begin{tabularx}{\textwidth}{R{3cm} Y Y}
\toprule
\headmark بُعد & \headmark توافق‌گرایی & \headmark یکپارچه‌سازی \\
\midrule
فلسفه & رسمیت گروه‌ها & همکاری فراقومی \\
\rowcolor{goldlight}
نظام انتخاباتی & تناسبی & اکثریتی / انتشار \\
احزاب & گروه‌های قومی & احزاب فراقومی \\
\rowcolor{goldlight}
دولت & ائتلاف فراگیر & ائتلاف برنده \\
مثال موفق & سوئیس، بلژیک & اندونزی، هند \\
\bottomrule
\end{tabularx}
\end{table}

\subsection{مدل ترکیبی پیشنهادی}

برای کشور ما، یک مدل ترکیبی مناسب‌تر است:

\begin{figure}[H]
\centering
\begin{tikzpicture}[
    node distance=2cm,
    mainbox/.style={
        rectangle,
        rounded corners=5pt,
        minimum width=5cm,
        minimum height=2cm,
        text centered,
        font=\small,
        line width=1.5pt
    },
    arrow/.style={->, >=Stealth, thick}
]

% دو مدل اصلی
\node[mainbox, draw=bleurepublique, fill=bleulight] (m1) at (-4,3) {
    \begin{tabular}{c}
    \textbf{توافق‌گرایی}\\[3pt]
    {\scriptsize خودمختاری فرهنگی}
    \end{tabular}
};

\node[mainbox, draw=bleurepublique!70, fill=bleulight] (m2) at (4,3) {
    \begin{tabular}{c}
    \textbf{یکپارچه‌سازی}\\[3pt]
    {\scriptsize هویت ملی مشترک}
    \end{tabular}
};

% مدل ترکیبی
\node[mainbox, draw=goldphoenix, fill=goldlight, minimum width=8cm, minimum height=2.5cm] (hybrid) at (0,0) {
    \begin{tabular}{c}
    \textbf{مدل ترکیبی پیشنهادی}\\[5pt]
    \rl{فدرالیسم نامتقارن + احزاب فراقومی}\\
    \rl{خودمختاری فرهنگی + هویت مدنی نوین}
    \end{tabular}
};

% نتایج
\node[mainbox, draw=bleurepublique, fill=bleurepublique, text=white, minimum width=4cm] (r1) at (-4,-3) {\textbf{وحدت پایدار}};
\node[mainbox, draw=goldphoenix, fill=goldphoenix, text=white, minimum width=4cm] (r2) at (4,-3) {\textbf{تنوع شکوفا}};

% فلش‌ها
\draw[arrow, color=bleurepublique] (m1) -- (hybrid);
\draw[arrow, color=bleurepublique!70] (m2) -- (hybrid);
\draw[arrow, color=goldphoenix] (hybrid) -- (r1);
\draw[arrow, color=goldphoenix] (hybrid) -- (r2);

\node[font=\bfseries, color=goldphoenix] at (0,-3) {\rl{وحدت در کثرت}};

\end{tikzpicture}
\caption{مدل ترکیبی مدیریت تنوع}
\end{figure}

%══════════════════════════════════════════════════════════════════════════════
\section{دموکراسی و توسعه}
\label{sec:democracy-development}
%══════════════════════════════════════════════════════════════════════════════

آیا توسعه پیش‌نیاز دموکراسی است یا دموکراسی پیش‌نیاز توسعه؟

\subsection{نظریه مدرنیزاسیون (لیپست)}

\begin{naghlbox}
«هرچه یک ملت ثروتمندتر باشد، احتمال دموکراتیک بودنش بیشتر است.»

\hfill --- سیمور مارتین لیپست، ۱۹۵۹
\end{naghlbox}

این نظریه می‌گوید توسعه اقتصادی (طبقه متوسط، آموزش، شهرنشینی) زمینه‌ساز دموکراسی است.

\textbf{شواهد له:}
\begin{itemize}[nosep]
    \item همبستگی آماری قوی بین GDP سرانه و دموکراسی
    \item بیشتر دموکراسی‌های پایدار، کشورهای ثروتمندند
\end{itemize}

\textbf{شواهد علیه:}
\begin{itemize}[nosep]
    \item هند از ۱۹۴۷ با فقر شدید دموکراتیک بوده
    \item چین و سنگاپور ثروتمند اما غیردموکراتیک‌اند
    \item بسیاری از کشورهای نفتی ثروتمند اما اقتدارگرایند
\end{itemize}

\subsection{توسعه به‌مثابه آزادی (آمارتیا سن)}

\begin{naghlbox}
«توسعه را می‌توان فرآیند گسترش آزادی‌های واقعی‌ای دانست که مردم از آن برخوردارند... آزادی سیاسی و حقوق مدنی نه‌تنها هدف توسعه، بلکه ابزار آن نیز هستند.»

\hfill --- آمارتیا سن، \textit{توسعه به‌مثابه آزادی}، ۱۹۹۹
\end{naghlbox}

سن استدلال می‌کند که:
\begin{enumerate}[nosep]
    \item دموکراسی \textbf{ارزش ذاتی} دارد — مشارکت در تعیین سرنوشت، فی‌نفسه ارزشمند است
    \item دموکراسی \textbf{ارزش ابزاری} دارد — از طریق بحث عمومی، نیازها شناسایی می‌شوند
    \item دموکراسی \textbf{نقش سازنده} دارد — شکل‌دهی به ارزش‌ها و اولویت‌ها
\end{enumerate}

\begin{olgoobox}[title={\hfill \textbf{یافته کلیدی سن: دموکراسی و قحطی}}]
هیچ قحطی بزرگی در تاریخ مدرن در یک کشور دموکراتیک با مطبوعات آزاد رخ نداده است. حتی فقیرترین دموکراسی‌ها (مانند هند) از قحطی جلوگیری کرده‌اند، در حالی که کشورهای ثروتمندتر اقتدارگرا (چین دوره مائو، اتیوپی) قحطی‌های ویرانگر تجربه کرده‌اند.

\textbf{چرا؟} در دموکراسی، دولت انگیزه دارد از قحطی جلوگیری کند چون رأی‌دهندگان او را مجازات می‌کنند. مطبوعات آزاد هشدار می‌دهند. در اقتدارگرایی، این مکانیزم‌ها وجود ندارند.
\end{olgoobox}

\subsection{شواهد تجربی جدید}

\begin{figure}[H]
\centering
\begin{tikzpicture}
\begin{axis}[
    width=0.95\textwidth,
    height=8cm,
    xlabel={\rl{شاخص دموکراسی}},
    ylabel={\rl{رشد GDP سرانه}},
    xmin=-10, xmax=10,
    ymin=-2, ymax=8,
    grid=both,
    grid style={line width=0.2pt, draw=gray!30},
    legend style={at={(0.02,0.98)}, anchor=north west},
    axis line style={bleurepublique!50, thick}
]
\addplot[only marks, mark=*, color=bleurepublique, mark size=2pt, opacity=0.6] coordinates {
    (-8,1) (-7,2) (-6,1.5) (-5,3) (-4,2.5) (-3,3) (-2,2) (-1,3.5)
    (0,3) (1,3.5) (2,4) (3,4.5) (4,4) (5,4.5) (6,5) (7,4.8)
    (8,4.5) (9,4.2) (10,4)
    (-9,7) (-8,6) (10,6) (9,5.5)
};
\addplot[color=goldphoenix, ultra thick, domain=-10:10] {3.2 + 0.1*x};
\legend{\rl{کشورها}, \rl{خط روند}}
\node[font=\tiny, anchor=west] at (axis cs:-9,7.2) {\rl{چین}};
\node[font=\tiny, anchor=west] at (axis cs:9.2,5.7) {\rl{کره جنوبی}};
\end{axis}
\end{tikzpicture}
\caption{رابطه دموکراسی و رشد اقتصادی}
\end{figure}

تحقیقات اخیر (Acemoglu et al., 2019) نشان می‌دهد:
\begin{itemize}[nosep]
    \item دموکراتیزاسیون به‌طور متوسط GDP سرانه را طی ۲۵ سال، ۲۰-۲۵٪ افزایش می‌دهد
    \item این اثر از طریق سرمایه‌گذاری در آموزش و بهداشت، کاهش ناآرامی، و اصلاحات اقتصادی حاصل می‌شود
    \item اثر در کشورهای با درآمد متوسط قوی‌تر است
\end{itemize}

%══════════════════════════════════════════════════════════════════════════════
\section{نقش نهادها}
\label{sec:institutions}
%══════════════════════════════════════════════════════════════════════════════

\subsection{نهادگرایی جدید (نورث)}

\begin{naghlbox}
«نهادها قواعد بازی در یک جامعه‌اند... آنها محدودیت‌هایی هستند که انسان‌ها برای شکل دادن به تعاملات بشری ابداع کرده‌اند.»

\hfill --- داگلاس نورث، \textit{نهادها، تغییر نهادی و عملکرد اقتصادی}، ۱۹۹۰
\end{naghlbox}

نورث تمایز کلیدی قائل می‌شود:
\begin{itemize}[nosep]
    \item \textbf{نهادهای رسمی:} قوانین، قانون اساسی، مقررات
    \item \textbf{نهادهای غیررسمی:} هنجارها، فرهنگ، سنت‌ها
\end{itemize}

\begin{table}[H]
\centering
\caption{تمایز نهادهای رسمی و غیررسمی}
\label{tab:formal-informal}
\begin{tabularx}{\textwidth}{R{2.5cm} Y Y}
\toprule
\headmark ویژگی & \headmark نهاد رسمی & \headmark نهاد غیررسمی \\
\midrule
منبع & قانون‌گذاری & فرهنگ و سنت \\
\rowcolor{goldlight}
اجرا & دولت (دادگاه) & اجتماع (طرد) \\
تغییر & سریع & بسیار کند \\
\bottomrule
\end{tabularx}
\end{table}

\subsection{چرا برخی ملت‌ها شکست می‌خورند؟}

\begin{olgoobox}[title={\hfill \textbf{نظریه Acemoglu \& Robinson}}]
در کتاب «چرا ملت‌ها شکست می‌خورند» (۲۰۱۲)، عجم‌اوغلو و رابینسون استدلال می‌کنند که تفاوت اصلی بین کشورهای موفق و ناموفق، نوع نهادهای آنهاست:

\begin{itemize}[nosep]
    \item \textbf{نهادهای فراگیر (Inclusive):} مالکیت امن، فرصت برابر، مشارکت سیاسی → رشد پایدار
    \item \textbf{نهادهای استخراجی (Extractive):} انحصار قدرت، غارت منابع، طرد اکثریت → رکود یا رشد ناپایدار
\end{itemize}
\end{olgoobox}

\begin{figure}[H]
\centering
\begin{tikzpicture}[
    node distance=2cm,
    instbox/.style={
        rectangle,
        rounded corners=5pt,
        minimum width=5cm,
        minimum height=3cm,
        text centered,
        font=\small\bfseries,
        line width=1.5pt
    },
    arrow/.style={->, >=Stealth, ultra thick}
]

% نهادهای فراگیر
\node[instbox, draw=bleurepublique, fill=bleulight] (inc) at (-4,0) {
    \begin{tabular}{c}
    \textbf{نهادهای فراگیر}\\[5pt]
    {\scriptsize فرصت برابر}\\
    {\scriptsize مشارکت سیاسی}\\
    {\scriptsize حاکمیت قانون}
    \end{tabular}
};

% نهادهای استخراجی
\node[instbox, draw=goldphoenix, fill=goldlight] (ext) at (4,0) {
    \begin{tabular}{c}
    \textbf{نهادهای استخراجی}\\[5pt]
    {\scriptsize انحصار قدرت}\\
    {\scriptsize غارت منابع}\\
    {\scriptsize طرد اکثریت}
    \end{tabular}
};

% نتایج
\node[instbox, draw=bleurepublique, fill=bleurepublique, text=white, minimum height=1.5cm] (res1) at (-4,-4) {\rl{توسعه پایدار}};
\node[instbox, draw=goldphoenix, fill=goldphoenix, text=white, minimum height=1.5cm] (res2) at (4,-4) {\rl{رکود یا زوال}};

% فلش‌ها
\draw[arrow, color=bleurepublique] (inc) -- (res1);
\draw[arrow, color=goldphoenix] (ext) -- (res2);

\end{tikzpicture}
\caption{نهادهای فراگیر در برابر استخراجی}
\end{figure}

%══════════════════════════════════════════════════════════════════════════════
\section{عدالت انتقالی}
\label{sec:transitional-justice}
%══════════════════════════════════════════════════════════════════════════════

چگونه باید با گذشته کنار آمد؟ این یکی از حساس‌ترین پرسش‌های هر گذار است.

\subsection{چهار بُعد عدالت انتقالی}

\begin{figure}[H]
\centering
\begin{tikzpicture}[
    dimbox/.style={
        rectangle,
        rounded corners=5pt,
        minimum width=4.5cm,
        minimum height=2cm,
        text centered,
        font=\small,
        line width=1.5pt
    }
]

% چهار بُعد
\node[dimbox, draw=bleurepublique, fill=bleulight] (d1) at (-4,3) {
    \begin{tabular}{c}
    \textbf{۱. حقیقت}\\[3pt]
    {\scriptsize کمیسیون حقیقت‌ياب}
    \end{tabular}
};

\node[dimbox, draw=bleurepublique, fill=bleulight] (d2) at (4,3) {
    \begin{tabular}{c}
    \textbf{۲. عدالت}\\[3pt]
    {\scriptsize محاکمه عاملان}
    \end{tabular}
};

\node[dimbox, draw=goldphoenix, fill=goldlight] (d3) at (-4,-1) {
    \begin{tabular}{c}
    \textbf{۳. جبران}\\[3pt]
    {\scriptsize غرامت به قربانیان}
    \end{tabular}
};

\node[dimbox, draw=goldphoenix, fill=goldlight] (d4) at (4,-1) {
    \begin{tabular}{c}
    \textbf{۴. عدم تکرار}\\[3pt]
    {\scriptsize اصلاحات نهادی}
    \end{tabular}
};

% مرکز
\node[circle, minimum size=2.2cm, draw=goldphoenix, fill=goldphoenix, text=white, font=\small\bfseries] (center) at (0,1) {\rl{آشتی ملی}};

% اتصالات
\draw[<->, >=Stealth, thick, color=gris] (d1) -- (center);
\draw[<->, >=Stealth, thick, color=gris] (d2) -- (center);
\draw[<->, >=Stealth, thick, color=gris] (d3) -- (center);
\draw[<->, >=Stealth, thick, color=gris] (d4) -- (center);

\end{tikzpicture}
\caption{چهار بُعد عدالت انتقالی}
\end{figure}

\subsection{طیف رویکردها}

\begin{table}[H]
\centering
\caption{طیف رویکردها به گذشته}
\label{tab:justice-approaches}
\begin{tabular}{L{2.5cm} L{4cm} L{3.5cm} L{3cm}}
\toprule
\headmark رویکرد & \headmark توضیح & \headmark نمونه & \headmark نتیجه \\
\midrule
\rowcolor{rougelight}
عفو کامل & فراموشی نهادینه & اسپانیا & ثبات اما تروما باقی \\
\rowcolor{bleulight}
کمیسیون حقیقت & حقیقت بدون مجازات & آفریقای جنوبی & آشتی نسبی \\
\rowcolor{orroyallight}
محاکمه محدود & مجازات سران & آرژانتین، شیلی & عدالت نسبی \\
\rowcolor{vertlight}
محاکمه گسترده & پاکسازی وسیع & آلمان نازی & عدالت اما بی‌ثباتی \\
\bottomrule
\end{tabular}
\end{table}

\begin{casebox}{کمیسیون حقیقت و آشتی آفریقای جنوبی}
کمیسیون TRC (۱۹۹۶-۱۹۹۸) تحت ریاست اسقف دزموند توتو، یکی از موفق‌ترین نمونه‌های عدالت انتقالی بود:

\textbf{مکانیزم:} عاملان جنایات می‌توانستند با اعتراف کامل و علنی، عفو بگیرند.

\textbf{دستاورد:} ۷,۱۱۲ درخواست عفو، ۲۱,۰۰۰ شهادت قربانیان، گزارش جامع ۳,۵۰۰ صفحه‌ای.

\textbf{درس:} حقیقت‌گویی می‌تواند جایگزین مجازات شود و به آشتی کمک کند — اما نیازمند اراده سیاسی قوی و رهبری اخلاقی است.
\end{casebox}

%══════════════════════════════════════════════════════════════════════════════
\section{چارچوب نظری تلفیقی این کتاب}
\label{sec:integrated-framework}
%══════════════════════════════════════════════════════════════════════════════

با جمع‌بندی مباحث این فصل، چارچوب نظری این کتاب را می‌توان چنین ترسیم کرد:

\begin{figure}[H]
\centering
\begin{tikzpicture}[
    node distance=1.5cm,
    mainbox/.style={
        rectangle,
        rounded corners=5pt,
        minimum width=4cm,
        minimum height=1.5cm,
        text centered,
        font=\small,
        line width=1.5pt
    },
    arrow/.style={->, >=Stealth, thick}
]

% سطح ۱: پیش‌نیازها
\node[mainbox, draw=gris, fill=grislight] (p1) at (-5,6) {
    \begin{tabular}{c}
    توافق ملی\\
    {\scriptsize (نخبگان + توده)}
    \end{tabular}
};
\node[mainbox, draw=gris, fill=grislight] (p2) at (0,6) {
    \begin{tabular}{c}
    بهبود معیشتی\\
    {\scriptsize (اصل آبادانی ملموس)}
    \end{tabular}
};
\node[mainbox, draw=gris, fill=grislight] (p3) at (5,6) {
    \begin{tabular}{c}
    حمایت بین‌المللی\\
    {\scriptsize (رفع تحریم)}
    \end{tabular}
};

% سطح ۲: گذار
\node[mainbox, draw=phase1, fill=phase1!20, minimum width=12cm] (trans) at (0,3.5) {
    \textbf{گذار دموکراتیک:} قانون اساسی + انتخابات + عدالت انتقالی
};

% سطح ۳: نهادسازی
\node[mainbox, draw=phase2, fill=phase2!20] (i1) at (-4,1) {
    \begin{tabular}{c}
    نهادهای سیاسی\\
    {\scriptsize فدرالیسم + احزاب}
    \end{tabular}
};
\node[mainbox, draw=phase2, fill=phase2!20] (i2) at (0,1) {
    \begin{tabular}{c}
    نهادهای اقتصادی\\
    {\scriptsize بازار + رفاه}
    \end{tabular}
};
\node[mainbox, draw=phase2, fill=phase2!20] (i3) at (4,1) {
    \begin{tabular}{c}
    نهادهای اجتماعی\\
    {\scriptsize مدنی + رسانه}
    \end{tabular}
};

% سطح ۴: تحکیم
\node[mainbox, draw=phase3, fill=phase3!20, minimum width=12cm] (cons) at (0,-1.5) {
    \textbf{تحکیم:} نهادینه‌سازی + فرهنگ دموکراتیک + توسعه پایدار
};

% سطح ۵: نتیجه
\node[mainbox, draw=vertnapoleon, fill=vertlight, minimum width=8cm, minimum height=2cm] (result) at (0,-4.5) {
    \begin{tabular}{c}
    \textbf{دموکراسی پایدار و کارآمد}\\[3pt]
    وحدت در کثرت | رفاه فراگیر | برتری منطقه‌ای
    \end{tabular}
};

% فلش‌ها
\draw[arrow, color=gris] (p1) -- (trans);
\draw[arrow, color=gris] (p2) -- (trans);
\draw[arrow, color=gris] (p3) -- (trans);
\draw[arrow, color=phase1] (trans) -- (i1);
\draw[arrow, color=phase1] (trans) -- (i2);
\draw[arrow, color=phase1] (trans) -- (i3);
\draw[arrow, color=phase2] (i1) -- (cons);
\draw[arrow, color=phase2] (i2) -- (cons);
\draw[arrow, color=phase2] (i3) -- (cons);
\draw[arrow, color=phase3] (cons) -- (result);

\end{tikzpicture}
\caption{چارچوب نظری تلفیقی کتاب}
\label{fig:integrated-framework}
\end{figure}

%══════════════════════════════════════════════════════════════════════════════
\section{نتیجه‌گیری}
\label{sec:theory-conclusion}
%══════════════════════════════════════════════════════════════════════════════

\begin{tahlilbox}[title={\hfill \textbf{جمع‌بندی چارچوب نظری}}]
\begin{enumerate}[nosep]
    \item \textbf{دموکراسی چیست:} نه فقط انتخابات، بلکه مجموعه‌ای از نهادها، حقوق، و فرهنگ
    \item \textbf{گذار و تحکیم:} دو مرحله متمایز — بسیاری در گذار موفق اما در تحکیم شکست می‌خورند
    \item \textbf{مدیریت تنوع:} مدل ترکیبی — هم حمایت از اقلیت‌ها، هم تشویق همکاری فراقومی
    \item \textbf{دموکراسی و توسعه:} رابطه دوسویه — دموکراسی هم هدف است هم ابزار توسعه
    \item \textbf{نهادها:} کلید موفقیت، ساختن نهادهای فراگیر به جای استخراجی است
    \item \textbf{گذشته:} عدالت انتقالی ضروری است — ترکیبی از حقیقت، عدالت، جبران، و اصلاح
    \item \textbf{اصل محوری:} آبادانی ملموس — دموکراسی باید نان بیاورد تا ریشه بدواند
\end{enumerate}
\end{tahlilbox}

%══════════════════════════════════════════════════════════════════════════════
\section*{منابع فصل}
%══════════════════════════════════════════════════════════════════════════════

\begin{enumerate}[nosep, label={[\arabic*]}]
    \item Dahl, R. (1971). \textit{Polyarchy: Participation and Opposition}. Yale University Press.
    
    \item Huntington, S. (1991). \textit{The Third Wave}. University of Oklahoma Press.
    
    \item Linz, J. \& Stepan, A. (1996). \textit{Problems of Democratic Transition and Consolidation}. Johns Hopkins.
    
    \item Lijphart, A. (2012). \textit{Patterns of Democracy}. 2nd ed. Yale University Press.
    
    \item Horowitz, D. (1985). \textit{Ethnic Groups in Conflict}. University of California Press.
    
    \item Sen, A. (1999). \textit{Development as Freedom}. Oxford University Press.
    
    \item North, D. (1990). \textit{Institutions, Institutional Change and Economic Performance}. Cambridge.
    
    \item Acemoglu, D. \& Robinson, J. (2012). \textit{Why Nations Fail}. Crown Business.
    
    \item Acemoglu, D. et al. (2019). "Democracy Does Cause Growth." \textit{JPE}, 127(1).
    
    \item Teitel, R. (2000). \textit{Transitional Justice}. Oxford University Press.
    
    \item Hayner, P. (2010). \textit{Unspeakable Truths}. 2nd ed. Routledge.
    
    \item Kymlicka, W. (1995). \textit{Multicultural Citizenship}. Oxford University Press.
    
    \item O'Donnell, G. \& Schmitter, P. (1986). \textit{Transitions from Authoritarian Rule}. Johns Hopkins.
    
    \item Diamond, L. (2008). \textit{The Spirit of Democracy}. Times Books.
\end{enumerate}
	
	%──────────────────────────────────────────────────────────────────────────────
	% بخش دوم: تجارب جهانی
	%──────────────────────────────────────────────────────────────────────────────
	\part{تجارب جهانی}
	
	%══════════════════════════════════════════════════════════════════════════════
% فصل ۴: درس‌های گذارهای موفق
% از بحران تا بالندگی
%══════════════════════════════════════════════════════════════════════════════

\chapter{درس‌های گذارهای موفق}
\label{ch:success}

%──────────────────────────────────────────────────────────────────────────────
% کادر خلاصه فصل
%──────────────────────────────────────────────────────────────────────────────
\begin{kholasebox}
این فصل چهار گذار موفق دموکراتیک را بررسی می‌کند: اسپانیا (۱۹۷۵-۱۹۸۲)، آفریقای جنوبی (۱۹۹۰-۱۹۹۹)، اندونزی (۱۹۹۸-۲۰۰۴) و کره جنوبی (۱۹۸۷-۱۹۹۷). هر یک از این کشورها با چالش‌هایی مشابه ما مواجه بودند: میراث اقتدارگرایی، تنوع اجتماعی، بحران اقتصادی، یا ترکیبی از اینها. الگوهای مشترک موفقیت شامل: توافق نخبگان (میثاق)، رهبری هوشمند، بهبود اقتصادی همزمان، عدالت انتقالی متوازن، و حمایت بین‌المللی است. مهم‌ترین درس: گذار موفق نیازمند «ائتلاف بزرگ» است که اصلاح‌طلبان درون نظام و معتدلین اپوزیسیون را گرد هم آورد.
\end{kholasebox}

%══════════════════════════════════════════════════════════════════════════════
\section{چرا این چهار کشور؟}
\label{sec:why-these-four}
%══════════════════════════════════════════════════════════════════════════════

\begin{table}[H]
\centering
\caption{مقایسه اجمالی چهار گذار موفق}
\label{tab:four-transitions}
\begin{tabular}{L{2.5cm} C{2cm} L{3cm} L{3cm} C{2cm}}
\toprule
\headmark کشور & \headmark سال گذار & \headmark نوع رژیم قبلی & \headmark چالش اصلی & \headmark مدت گذار \\
\midrule
\rowcolor{bleulight}
اسپانیا & ۱۹۷۵-۸۲ & فاشیسم (فرانکو) & منطقه‌گرایی، ارتش & ۷ سال \\
آفریقای جنوبی & ۱۹۹۰-۹۹ & آپارتاید نژادی & نابرابری، خشونت & ۹ سال \\
\rowcolor{bleulight}
اندونزی & ۱۹۹۸-۰۴ & اقتدارگرای نظامی & تنوع قومی-دینی & ۶ سال \\
کره جنوبی & ۱۹۸۷-۹۷ & دیکتاتوری نظامی & توسعه بدون آزادی & ۱۰ سال \\
\bottomrule
\end{tabular}
\end{table}

این کشورها به دلایل زیر انتخاب شده‌اند:
\begin{itemize}[nosep]
    \item هر یک با چالش‌هایی مشابه کشور ما مواجه بودند
    \item همگی از اقتدارگرایی به دموکراسی نسبتاً پایدار رسیدند
    \item مسیرهای متفاوتی طی کردند که درس‌های متنوعی ارائه می‌دهند
    \item اطلاعات و پژوهش‌های کافی درباره آنها موجود است
\end{itemize}

%══════════════════════════════════════════════════════════════════════════════
\section{اسپانیا: میثاق نخبگان و خودمختاری}
\label{sec:spain}
%══════════════════════════════════════════════════════════════════════════════

\subsection{زمینه: چهار دهه فرانکیسم}

ژنرال فرانسیسکو فرانکو از ۱۹۳۹ تا ۱۹۷۵ بر اسپانیا حکومت کرد. رژیم او مبتنی بر:
\begin{itemize}[nosep]
    \item سرکوب شدید مخالفان (به‌ویژه کمونیست‌ها و جمهوری‌خواهان)
    \item انکار هویت منطقه‌ای (کاتالونیا، باسک)
    \item اتحاد با کلیسای کاتولیک و ارتش
    \item اقتصاد دولت‌محور در دهه‌های اول، سپس اصلاحات از دهه ۶۰
\end{itemize}

\begin{figure}[H]
\centering
\begin{tikzpicture}[
    node distance=0.5cm,
    eventbox/.style={
        rectangle,
        rounded corners=3pt,
        minimum width=2.5cm,
        minimum height=1.2cm,
        text centered,
        font=\scriptsize,
        line width=1pt
    },
    arrow/.style={->, >=Stealth, thick, color=bleurepublique}
]

% خط زمان
\draw[line width=2pt, color=gris] (0,0) -- (15,0);

% رویدادها
\node[eventbox, draw=rougerevolution, fill=rougelight, above] at (0,0.5) {
    \begin{tabular}{c}
    ۱۹۳۹\\
    پیروزی فرانکو
    \end{tabular}
};

\node[eventbox, draw=gris, fill=grislight, above] at (3,0.5) {
    \begin{tabular}{c}
    ۱۹۵۹\\
    اصلاحات اقتصادی
    \end{tabular}
};

\node[eventbox, draw=orroyal, fill=orroyallight, above] at (6,0.5) {
    \begin{tabular}{c}
    ۱۹۷۵\\
    مرگ فرانکو
    \end{tabular}
};

\node[eventbox, draw=bleurepublique, fill=bleulight, above] at (9,0.5) {
    \begin{tabular}{c}
    ۱۹۷۷\\
    توافق مونکلوا
    \end{tabular}
};

\node[eventbox, draw=vertnapoleon, fill=vertlight, above] at (12,0.5) {
    \begin{tabular}{c}
    ۱۹۷۸\\
    قانون اساسی
    \end{tabular}
};

\node[eventbox, draw=vertnapoleon, fill=vertlight, above] at (15,0.5) {
    \begin{tabular}{c}
    ۱۹۸۲\\
    انتقال به سوسیالیست‌ها
    \end{tabular}
};

% نقاط
\foreach \x in {0,3,6,9,12,15} {
    \fill[black] (\x,0) circle (4pt);
}

\end{tikzpicture}
\caption{خط زمانی گذار اسپانیا}
\label{fig:spain-timeline}
\end{figure}

\subsection{عوامل کلیدی موفقیت}

\begin{olgoobox}[title={\hfill \textbf{الگوی اسپانیا: توافق مونکلوا (۱۹۷۷)}}]
توافق مونکلوا یک «میثاق نخبگان» بود که بین دولت، احزاب اپوزیسیون، اتحادیه‌های کارگری و کارفرمایان منعقد شد. محتوای آن:

\begin{itemize}[nosep]
    \item \textbf{اقتصادی:} کنترل تورم، اصلاحات مالیاتی، محدودیت دستمزد
    \item \textbf{سیاسی:} آزادی احزاب، آزادی رسانه، حقوق اجتماعات
    \item \textbf{اجتماعی:} گسترش تأمین اجتماعی، حقوق کارگران
\end{itemize}

\textbf{چرا کار کرد:} همه طرف‌ها چیزی به دست آوردند و چیزی از دست دادند. کسی «برنده مطلق» یا «بازنده مطلق» نبود.
\end{olgoobox}

\begin{table}[H]
\centering
\caption{عوامل موفقیت گذار اسپانیا}
\label{tab:spain-factors}
\begin{tabular}{C{1cm} L{3.5cm} L{7.5cm}}
\toprule
\headmark \# & \headmark عامل & \headmark توضیح \\
\midrule
\rowcolor{bleulight}
۱ & نقش پادشاه & خوان کارلوس از درون نظام، گذار را هدایت کرد و مشروعیت بخشید \\
۲ & رهبری سوآرز & آدولفو سوآرز، نخست‌وزیر اصلاح‌طلب، پل بین نظام و اپوزیسیون شد \\
\rowcolor{bleulight}
۳ & اپوزیسیون معتدل & کمونیست‌ها (سانتیاگو کاریو) رادیکالیسم را کنار گذاشتند \\
۴ & توافق مونکلوا & میثاق اقتصادی-سیاسی همه نیروها \\
\rowcolor{bleulight}
۵ & خودمختاری مناطق & قانون اساسی ۱۹۷۸ خودمختاری گسترده به کاتالونیا و باسک داد \\
۶ & رشد اقتصادی & اقتصاد در حال رشد، فشار کمتر بر مردم \\
\rowcolor{bleulight}
۷ & اروپا & چشم‌انداز عضویت در EC/EU انگیزه قوی برای اصلاحات \\
\bottomrule
\end{tabular}
\end{table}

\subsection{کودتای نافرجام ۱۹۸۱}

\begin{enghelabbox}[title={\hfill \textbf{آزمون بحرانی: ۲۳ فوریه ۱۹۸۱}}]
در ۲۳ فوریه ۱۹۸۱، سرهنگ تخرو با ۲۰۰ نظامی پارلمان را اشغال کرد. برخی فرماندهان ارتش از کودتا حمایت کردند.

\textbf{چرا شکست خورد:}
\begin{itemize}[nosep]
    \item پادشاه خوان کارلوس در تلویزیون ظاهر شد و کودتا را محکوم کرد
    \item فرماندهان کلیدی ارتش از پادشاه پیروی کردند
    \item مردم به خیابان‌ها آمدند و از دموکراسی دفاع کردند
\end{itemize}

\textbf{درس:} در لحظات بحرانی، رهبری قاطع و حمایت مردمی می‌تواند دموکراسی را نجات دهد.
\end{enghelabbox}

\subsection{مسئله عدالت انتقالی}

اسپانیا رویکرد «فراموشی نهادینه» (Pacto del Olvido) را انتخاب کرد:
\begin{itemize}[nosep]
    \item عفو عمومی ۱۹۷۷ شامل همه جرایم سیاسی شد
    \item هیچ محاکمه‌ای از عاملان رژیم فرانکو صورت نگرفت
    \item بحث عمومی درباره گذشته تابو شد
\end{itemize}

\begin{naghlbox}
«ما تصمیم گرفتیم به جای نگاه به گذشته، به آینده نگاه کنیم. این شاید عادلانه‌ترین راه نبود، اما عاقلانه‌ترین راه بود.»

\hfill --- فلیپه گونزالس، نخست‌وزیر سوسیالیست اسپانیا
\end{naghlbox}

\textbf{نقد:} این رویکرد ثبات کوتاه‌مدت آورد، اما تروماها درمان نشدند. دهه‌ها بعد، جنبش «حافظه تاریخی» خواستار بازخوانی گذشته شد.

\subsection{درس‌های اسپانیا برای ما}

\begin{table}[H]
\centering
\caption{درس‌های قابل انتقال از اسپانیا}
\label{tab:spain-lessons}
\begin{tabular}{L{4cm} L{4cm} L{4.5cm}}
\toprule
\headmark درس & \headmark کاربرد برای ما & \headmark ملاحظه \\
\midrule
\rowcolor{vertlight}
توافق نخبگان & مذاکره همه جریان‌ها & نیاز به رهبری معتدل از همه طرف \\
\rowcolor{vertlight}
خودمختاری مناطق & فدرالیسم برای اقوام & نه تجزیه، نه تمرکز افراطی \\
\rowcolor{bleulight}
نقش نهاد میانجی & شاید نهاد دینی یا ملی؟ & نیاز به نهاد با مشروعیت فراجناحی \\
\rowcolor{orroyallight}
فراموشی گذشته & \textcolor{rougerevolution}{توصیه نمی‌شود} & تروماهای ما عمیق‌ترند \\
\bottomrule
\end{tabular}
\end{table}

%══════════════════════════════════════════════════════════════════════════════
\section{آفریقای جنوبی: آشتی و قانون اساسی فراگیر}
\label{sec:south-africa}
%══════════════════════════════════════════════════════════════════════════════

\subsection{زمینه: نظام آپارتاید}

آپارتاید (۱۹۴۸-۱۹۹۴) نظام جداسازی نژادی بود که:
\begin{itemize}[nosep]
    \item ۸۰٪ جمعیت (سیاهان) را از حقوق سیاسی محروم می‌کرد
    \item نابرابری اقتصادی شدید ایجاد کرده بود
    \item سرکوب خشن مخالفان (کشتار شارپویل، زندان روبن آیلند)
    \item انزوای بین‌المللی و تحریم
\end{itemize}

\begin{figure}[H]
\centering
\begin{tikzpicture}[
    node distance=0.5cm,
    eventbox/.style={
        rectangle,
        rounded corners=3pt,
        minimum width=2.2cm,
        minimum height=1.2cm,
        text centered,
        font=\scriptsize,
        line width=1pt
    }
]

% خط زمان
\draw[line width=2pt, color=gris] (0,0) -- (14,0);

% رویدادها
\node[eventbox, draw=rougerevolution, fill=rougelight, above] at (0,0.5) {
    \begin{tabular}{c}
    ۱۹۴۸\\
    آغاز آپارتاید
    \end{tabular}
};

\node[eventbox, draw=orroyal, fill=orroyallight, above] at (3.5,0.5) {
    \begin{tabular}{c}
    ۱۹۶۴\\
    زندانی شدن ماندلا
    \end{tabular}
};

\node[eventbox, draw=bleurepublique, fill=bleulight, above] at (7,0.5) {
    \begin{tabular}{c}
    ۱۹۹۰\\
    آزادی ماندلا
    \end{tabular}
};

\node[eventbox, draw=vertnapoleon, fill=vertlight, above] at (10.5,0.5) {
    \begin{tabular}{c}
    ۱۹۹۴\\
    انتخابات آزاد
    \end{tabular}
};

\node[eventbox, draw=vertnapoleon, fill=vertlight, above] at (14,0.5) {
    \begin{tabular}{c}
    ۱۹۹۶\\
    قانون اساسی
    \end{tabular}
};

% نقاط
\foreach \x in {0,3.5,7,10.5,14} {
    \fill[black] (\x,0) circle (4pt);
}

\end{tikzpicture}
\caption{خط زمانی گذار آفریقای جنوبی}
\label{fig:sa-timeline}
\end{figure}

\subsection{عوامل کلیدی موفقیت}

\begin{olgoobox}[title={\hfill \textbf{الگوی آفریقای جنوبی: رهبری ماندلا}}]
نلسون ماندلا پس از ۲۷ سال زندان، نه انتقام‌جو بلکه آشتی‌طلب بود:

\begin{itemize}[nosep]
    \item با دشمنان دیروز (دوکلرک، ژنرال‌ها) مذاکره کرد
    \item پیراهن تیم راگبی سفیدپوستان را پوشید (نماد وحدت)
    \item از ANC خواست انتقام نگیرند
    \item فقط یک دوره رئیس‌جمهور ماند
\end{itemize}

\textbf{درس:} رهبری اخلاقی می‌تواند چرخه خشونت را بشکند.
\end{olgoobox}

\begin{naghlbox}
«اگر می‌خواهید با دشمن خود صلح کنید، باید با او کار کنید. آنگاه او شریک شما می‌شود.»

\hfill --- نلسون ماندلا
\end{naghlbox}

\subsection{کمیسیون حقیقت و آشتی (TRC)}

\begin{table}[H]
\centering
\caption{ساختار و دستاوردهای TRC}
\label{tab:trc}
\begin{tabular}{L{3cm} L{9cm}}
\toprule
\headmark جنبه & \headmark توضیح \\
\midrule
\rowcolor{bleulight}
ریاست & اسقف دزموند توتو (برنده نوبل صلح) \\
مدت & ۱۹۹۶-۱۹۹۸ (گزارش نهایی ۲۰۰۳) \\
\rowcolor{bleulight}
مکانیزم & عاملان با اعتراف کامل و علنی می‌توانستند عفو بگیرند \\
شهادت‌ها & ۲۱,۰۰۰ قربانی شهادت دادند \\
\rowcolor{bleulight}
درخواست عفو & ۷,۱۱۲ درخواست (۱,۵۰۰ عفو شدند) \\
گزارش نهایی & ۳,۵۰۰ صفحه مستندسازی جنایات \\
\bottomrule
\end{tabular}
\end{table}

\subsection{قانون اساسی ۱۹۹۶}

قانون اساسی آفریقای جنوبی یکی از پیشرفته‌ترین قوانین اساسی جهان است:

\begin{table}[H]
\centering
\caption{ویژگی‌های قانون اساسی آفریقای جنوبی}
\label{tab:sa-constitution}
\begin{tabular}{L{3.5cm} L{8.5cm}}
\toprule
\headmark ویژگی & \headmark توضیح \\
\midrule
\rowcolor{bleulight}
منشور حقوق & فصل ۲ شامل حقوق گسترده مدنی، سیاسی، اقتصادی، اجتماعی \\
دادگاه قانون اساسی & قدرتمند و مستقل، نقش کلیدی در حمایت از حقوق \\
\rowcolor{bleulight}
۱۱ زبان رسمی & به رسمیت شناختن تنوع زبانی \\
فدرالیسم & ۹ استان با اختیارات محدود \\
\rowcolor{bleulight}
حقوق اقلیت‌ها & حمایت از گروه‌های زبانی، فرهنگی، دینی \\
\bottomrule
\end{tabular}
\end{table}

\subsection{چالش‌های باقیمانده}

\begin{enghelabbox}[title={\hfill \textbf{درس منفی: نابرابری اقتصادی}}]
آفریقای جنوبی در دموکراسی سیاسی موفق بود، اما در عدالت اقتصادی ناموفق:

\begin{itemize}[nosep]
    \item ضریب جینی: از بالاترین‌ها در جهان (۰.۶۳)
    \item بیکاری: بالای ۳۰٪ (در جوانان سیاه‌پوست بالای ۵۰٪)
    \item ثروت همچنان در دست اقلیت سفیدپوست متمرکز
\end{itemize}

\textbf{درس:} دموکراسی سیاسی بدون عدالت اقتصادی، ناقص و شکننده است.
\end{enghelabbox}

\subsection{درس‌های آفریقای جنوبی برای ما}

\begin{table}[H]
\centering
\caption{درس‌های قابل انتقال از آفریقای جنوبی}
\label{tab:sa-lessons}
\begin{tabular}{L{4cm} L{4cm} L{4.5cm}}
\toprule
\headmark درس & \headmark کاربرد برای ما & \headmark ملاحظه \\
\midrule
\rowcolor{vertlight}
کمیسیون حقیقت‌یابی & ضروری برای آشتی ملی & نیاز به رهبری اخلاقی \\
\rowcolor{vertlight}
قانون اساسی فراگیر & مشارکت همه گروه‌ها & فرآیند زمان‌بر اما ارزشمند \\
\rowcolor{vertlight}
رهبری آشتی‌طلب & رهبرانی که انتقام نخواهند & نادر اما حیاتی \\
\rowcolor{orroyallight}
بدون عدالت اقتصادی & \textcolor{rougerevolution}{باید اجتناب کرد} & بهبود معیشت ضروری است \\
\bottomrule
\end{tabular}
\end{table}

%══════════════════════════════════════════════════════════════════════════════
\section{اندونزی: مدیریت تنوع و تمرکززدایی}
\label{sec:indonesia}
%══════════════════════════════════════════════════════════════════════════════

\subsection{زمینه: سه دهه سوهارتو}

اندونزی کشوری با:
\begin{itemize}[nosep]
    \item ۱۷,۰۰۰ جزیره، ۳۰۰+ قومیت، ۷۰۰+ زبان
    \item بزرگترین کشور مسلمان جهان (۸۷٪ مسلمان)
    \item رژیم «نظم نوین» سوهارتو (۱۹۶۶-۱۹۹۸): توسعه اقتصادی + سرکوب سیاسی
    \item بحران مالی آسیا ۱۹۹۷: سقوط اقتصادی و سقوط سوهارتو
\end{itemize}

\begin{figure}[H]
\centering
\begin{tikzpicture}[
    node distance=0.5cm,
    eventbox/.style={
        rectangle,
        rounded corners=3pt,
        minimum width=2.5cm,
        minimum height=1.2cm,
        text centered,
        font=\scriptsize,
        line width=1pt
    }
]

% خط زمان
\draw[line width=2pt, color=gris] (0,0) -- (14,0);

% رویدادها
\node[eventbox, draw=rougerevolution, fill=rougelight, above] at (0,0.5) {
    \begin{tabular}{c}
    ۱۹۶۶\\
    آغاز سوهارتو
    \end{tabular}
};

\node[eventbox, draw=orroyal, fill=orroyallight, above] at (3.5,0.5) {
    \begin{tabular}{c}
    ۱۹۹۷\\
    بحران مالی
    \end{tabular}
};

\node[eventbox, draw=bleurepublique, fill=bleulight, above] at (7,0.5) {
    \begin{tabular}{c}
    ۱۹۹۸\\
    سقوط سوهارتو
    \end{tabular}
};

\node[eventbox, draw=vertnapoleon, fill=vertlight, above] at (10.5,0.5) {
    \begin{tabular}{c}
    ۱۹۹۹\\
    انتخابات آزاد
    \end{tabular}
};

\node[eventbox, draw=vertnapoleon, fill=vertlight, above] at (14,0.5) {
    \begin{tabular}{c}
    ۲۰۰۴\\
    انتخابات مستقیم
    \end{tabular}
};

% نقاط
\foreach \x in {0,3.5,7,10.5,14} {
    \fill[black] (\x,0) circle (4pt);
}

\end{tikzpicture}
\caption{خط زمانی گذار اندونزی}
\label{fig:indonesia-timeline}
\end{figure}

\subsection{عوامل کلیدی موفقیت}

\begin{olgoobox}[title={\hfill \textbf{الگوی اندونزی: اصلاحات (Reformasi)}}]
جنبش اصلاحات با رهبری دانشجویان و حمایت طبقه متوسط شهری، سوهارتو را مجبور به استعفا کرد. اما برخلاف بسیاری از انقلاب‌ها، رادیکال نشد:

\begin{itemize}[nosep]
    \item معاون سوهارتو (حبیبی) قدرت را گرفت و انتخابات آزاد برگزار کرد
    \item نهادهای موجود حفظ شدند و تدریجاً اصلاح شدند
    \item ارتش به تدریج از سیاست کنار رفت
\end{itemize}
\end{olgoobox}

\begin{table}[H]
\centering
\caption{عوامل موفقیت گذار اندونزی}
\label{tab:indonesia-factors}
\begin{tabular}{C{1cm} L{3.5cm} L{7.5cm}}
\toprule
\headmark \# & \headmark عامل & \headmark توضیح \\
\midrule
\rowcolor{bleulight}
۱ & جنبش دانشجویی & فشار از پایین برای تغییر \\
۲ & اصلاح‌طلبان درون نظام & حبیبی انتخابات آزاد برگزار کرد \\
\rowcolor{bleulight}
۳ & تمرکززدایی گسترده & انتقال قدرت به ۵۰۰+ منطقه \\
۴ & پانچاسیلا & ایدئولوژی وحدت‌بخش فراگیر \\
\rowcolor{bleulight}
۵ & خروج ارتش از سیاست & اصلاحات نظامی تدریجی \\
۶ & نقش اسلام معتدل & NU و Muhammadiyah از دموکراسی حمایت کردند \\
\bottomrule
\end{tabular}
\end{table}

\subsection{مدل پانچاسیلا}

\begin{naghlbox}
«پانچاسیلا پنج اصل بنیادین اندونزی است:
۱. ایمان به خدای یگانه
۲. انسانیت عادلانه و متمدن
۳. وحدت اندونزی
۴. دموکراسی با خرد مشورتی
۵. عدالت اجتماعی

این اصول به‌جای تأکید بر یک دین یا قومیت، چتری فراگیر برای همه ایجاد می‌کند.»

\hfill --- سوکارنو، بنیان‌گذار اندونزی
\end{naghlbox}

\subsection{تمرکززدایی}

\begin{figure}[H]
\centering
\begin{tikzpicture}[
    node distance=1.5cm,
    levelbox/.style={
        rectangle,
        rounded corners=3pt,
        minimum width=4cm,
        minimum height=1.5cm,
        text centered,
        font=\small,
        line width=1pt
    },
    arrow/.style={->, >=Stealth, thick}
]

% قبل
\node[levelbox, draw=rougerevolution, fill=rougelight] (before) at (-4,0) {
    \begin{tabular}{c}
    \textbf{قبل از ۱۹۹۹}\\[3pt]
    {\scriptsize تمرکز شدید}\\
    {\scriptsize همه تصمیمات در جاکارتا}
    \end{tabular}
};

% بعد
\node[levelbox, draw=vertnapoleon, fill=vertlight] (after) at (4,0) {
    \begin{tabular}{c}
    \textbf{بعد از ۱۹۹۹}\\[3pt]
    {\scriptsize تمرکززدایی گسترده}\\
    {\scriptsize ۵۰۰+ منطقه خودمختار}
    \end{tabular}
};

% فلش
\draw[arrow, very thick, color=bleurepublique] (before) -- node[above, font=\small] {قانون ۱۹۹۹} (after);

% جزئیات زیر
\node[font=\scriptsize, text width=4cm, align=center, below=0.5cm of after] {
    انتخاب مستقیم فرماندار\\
    بودجه محلی\\
    مدیریت آموزش و بهداشت
};

\end{tikzpicture}
\caption{تمرکززدایی در اندونزی}
\label{fig:indonesia-decentralization}
\end{figure}

\subsection{درس‌های اندونزی برای ما}

\begin{table}[H]
\centering
\caption{درس‌های قابل انتقال از اندونزی}
\label{tab:indonesia-lessons}
\begin{tabular}{L{4cm} L{4cm} L{4.5cm}}
\toprule
\headmark درس & \headmark کاربرد برای ما & \headmark ملاحظه \\
\midrule
\rowcolor{vertlight}
پانچاسیلا & ایدئولوژی وحدت‌بخش فراگیر & نیاز به میراث مشترک \\
\rowcolor{vertlight}
تمرکززدایی & فدرالیسم/خودمختاری & متناسب با ظرفیت محلی \\
\rowcolor{vertlight}
نقش اسلام معتدل & همکاری نهادهای دینی & دین می‌تواند حامی دموکراسی باشد \\
\rowcolor{vertlight}
گذار از درون نظام & استفاده از اصلاح‌طلبان & نیاز به شکاف در نخبگان \\
\bottomrule
\end{tabular}
\end{table}

%══════════════════════════════════════════════════════════════════════════════
\section{کره جنوبی: از دیکتاتوری توسعه‌گرا به دموکراسی}
\label{sec:south-korea}
%══════════════════════════════════════════════════════════════════════════════

\subsection{زمینه: معجزه اقتصادی بدون آزادی}

کره جنوبی نمونه بارز «اقتدارگرایی توسعه‌گرا» بود:
\begin{itemize}[nosep]
    \item دیکتاتوری نظامی (۱۹۶۱-۱۹۸۷)
    \item رشد اقتصادی سریع: از $۱۰۰ به $۱۰,۰۰۰ GDP سرانه
    \item سرکوب شدید: کشتار گوانگجو ۱۹۸۰ (صدها کشته)
    \item طبقه متوسط رو به رشد که خواستار آزادی شد
\end{itemize}

\subsection{عوامل کلیدی موفقیت}

\begin{olgoobox}[title={\hfill \textbf{الگوی کره: جنبش دموکراتیک ژوئن ۱۹۸۷}}]
در ژوئن ۱۹۸۷، میلیون‌ها نفر در خیابان‌ها خواستار دموکراسی شدند. رژیم نظامی مجبور به پذیرش شد:

\begin{itemize}[nosep]
    \item اعلامیه ۲۹ ژوئن: پذیرش انتخابات مستقیم ریاست‌جمهوری
    \item قانون اساسی جدید: محدودیت یک دوره‌ای رئیس‌جمهور
    \item انتخابات دسامبر ۱۹۸۷: رقابت واقعی
\end{itemize}

\textbf{چرا رژیم تسلیم شد:} المپیک ۱۹۸۸ سئول نزدیک بود و سرکوب خونین، وجهه بین‌المللی را نابود می‌کرد.
\end{olgoobox}

\subsection{عدالت انتقالی: محاکمه روسای جمهور}

\begin{table}[H]
\centering
\caption{محاکمه روسای جمهور سابق کره جنوبی}
\label{tab:korea-trials}
\begin{tabular}{L{3cm} C{2cm} L{4cm} L{3.5cm}}
\toprule
\headmark شخص & \headmark سال محاکمه & \headmark اتهام & \headmark حکم \\
\midrule
\rowcolor{bleulight}
چون دو-هوان & ۱۹۹۶ & کشتار گوانگجو، کودتا & اعدام (تخفیف به حبس) \\
روه تای-وو & ۱۹۹۶ & کودتا، رشوه & حبس ۲۲ سال (عفو) \\
\rowcolor{bleulight}
پارک گئون-هه & ۲۰۱۷ & فساد & حبس ۲۵ سال \\
لی میونگ-باک & ۲۰۱۸ & فساد & حبس ۱۷ سال \\
\bottomrule
\end{tabular}
\end{table}

\begin{naghlbox}
«این که یک کشور می‌تواند روسای جمهور سابق خود را محاکمه کند، نشانه بلوغ دموکراتیک است. در کره، هیچ‌کس بالاتر از قانون نیست.»

\hfill --- تحلیل‌گر سیاسی کره‌ای
\end{naghlbox}

\subsection{درس‌های کره جنوبی برای ما}

\begin{table}[H]
\centering
\caption{درس‌های قابل انتقال از کره جنوبی}
\label{tab:korea-lessons}
\begin{tabular}{L{4cm} L{4cm} L{4.5cm}}
\toprule
\headmark درس & \headmark کاربرد برای ما & \headmark ملاحظه \\
\midrule
\rowcolor{vertlight}
نقش طبقه متوسط & توسعه طبقه متوسط مستقل & زمان‌بر اما مهم \\
\rowcolor{vertlight}
محاکمه مسئولان & عدالت واقعی ممکن است & نیاز به قوه قضائیه مستقل \\
\rowcolor{vertlight}
اهرم بین‌المللی & رویدادهای جهانی فشار می‌آورند & استفاده از فرصت‌ها \\
\rowcolor{bleulight}
توسعه سپس دموکراسی & مسیر طولانی‌تر & ما توسعه را نداریم \\
\bottomrule
\end{tabular}
\end{table}

%══════════════════════════════════════════════════════════════════════════════
\section{الگوهای مشترک موفقیت}
\label{sec:success-patterns}
%══════════════════════════════════════════════════════════════════════════════

\begin{figure}[H]
\centering
\begin{tikzpicture}[
    node distance=1.5cm,
    factor/.style={
        rectangle,
        rounded corners=5pt,
        minimum width=3.5cm,
        minimum height=1.5cm,
        text centered,
        font=\small,
        draw=vertnapoleon,
        fill=vertlight,
        line width=1.5pt
    },
    center/.style={
        circle,
        minimum size=3cm,
        text centered,
        font=\small\bfseries,
        draw=bleurepublique,
        fill=bleulight,
        line width=2pt
    }
]

% مرکز
\node[center] (c) at (0,0) {
    \begin{tabular}{c}
    گذار\\
    موفق
    \end{tabular}
};

% عوامل
\node[factor] (f1) at (90:4.5cm) {
    \begin{tabular}{c}
    توافق نخبگان\\
    {\scriptsize میثاق ملی}
    \end{tabular}
};

\node[factor] (f2) at (30:4.5cm) {
    \begin{tabular}{c}
    رهبری هوشمند\\
    {\scriptsize معتدل و آشتی‌طلب}
    \end{tabular}
};

\node[factor] (f3) at (330:4.5cm) {
    \begin{tabular}{c}
    بهبود اقتصادی\\
    {\scriptsize یا حداقل ثبات}
    \end{tabular}
};

\node[factor] (f4) at (270:4.5cm) {
    \begin{tabular}{c}
    عدالت انتقالی\\
    {\scriptsize متوازن}
    \end{tabular}
};

\node[factor] (f5) at (210:4.5cm) {
    \begin{tabular}{c}
    حمایت بین‌المللی\\
    {\scriptsize یا فشار سازنده}
    \end{tabular}
};

\node[factor] (f6) at (150:4.5cm) {
    \begin{tabular}{c}
    نهادهای میانجی\\
    {\scriptsize کلیسا، ارتش، نهاد ملی}
    \end{tabular}
};

% اتصالات
\foreach \f in {f1,f2,f3,f4,f5,f6} {
    \draw[->, >=Stealth, thick, color=vertnapoleon] (\f) -- (c);
}

\end{tikzpicture}
\caption{شش عامل مشترک گذارهای موفق}
\label{fig:success-factors}
\end{figure}

\begin{table}[H]
\centering
\caption{حضور عوامل موفقیت در چهار کشور}
\label{tab:factors-comparison}
\begin{tabular}{L{3.5cm} C{2cm} C{2cm} C{2cm} C{2cm}}
\toprule
\headmark عامل & \headmark اسپانیا & \headmark آ. جنوبی & \headmark اندونزی & \headmark کره \\
\midrule
\rowcolor{bleulight}
توافق نخبگان & \yes & \yes & \somewhat & \somewhat \\
رهبری هوشمند & \yes & \yes & \somewhat & \somewhat \\
\rowcolor{bleulight}
بهبود اقتصادی & \yes & \no & \somewhat & \yes \\
عدالت انتقالی & \no & \yes & \somewhat & \yes \\
\rowcolor{bleulight}
حمایت بین‌المللی & \yes & \yes & \yes & \yes \\
نهاد میانجی & \yes & \yes & \yes & \somewhat \\
\bottomrule
\end{tabular}
\end{table}

%══════════════════════════════════════════════════════════════════════════════
\section{نتیجه‌گیری: چه باید آموخت؟}
\label{sec:success-conclusion}
%══════════════════════════════════════════════════════════════════════════════

\begin{tahlilbox}[title={\hfill \textbf{درس‌های کلیدی}}]
\begin{enumerate}[nosep]
    \item \textbf{گذار موفق نیازمند ائتلاف بزرگ است:} اصلاح‌طلبان درون نظام + معتدلین اپوزیسیون
    \item \textbf{رهبری اخلاقی تفاوت می‌سازد:} ماندلا، خوان کارلوس، سوآرز
    \item \textbf{مذاکره بهتر از فروپاشی است:} گذارهای مذاکره‌ای پایدارترند
    \item \textbf{اقتصاد مهم است:} بدون بهبود معیشتی، دموکراسی شکننده است
    \item \textbf{گذشته باید پردازش شود:} به شکلی متوازن بین عدالت و ثبات
    \item \textbf{تنوع قابل مدیریت است:} با فدرالیسم، خودمختاری، یا ایدئولوژی فراگیر
\end{enumerate}
\end{tahlilbox}

%══════════════════════════════════════════════════════════════════════════════
\section*{منابع فصل}
%══════════════════════════════════════════════════════════════════════════════

\begin{enumerate}[nosep, label={[\arabic*]}]
    \item Linz, J. \& Stepan, A. (1996). \textit{Problems of Democratic Transition and Consolidation}. Johns Hopkins.
    
    \item Encarnación, O. (2008). \textit{Spanish Politics: Democracy after Dictatorship}. Polity Press.
    
    \item Gibson, J. (2004). \textit{Overcoming Apartheid: Can Truth Reconcile a Divided Nation?} Russell Sage.
    
    \item Aspinall, E. (2005). \textit{Opposing Suharto: Compromise, Resistance, and Regime Change}. Stanford.
    
    \item Kim, S. (2000). \textit{The Politics of Democratization in Korea}. University of Pittsburgh Press.
    
    \item Hayner, P. (2010). \textit{Unspeakable Truths}. 2nd ed. Routledge.
    
    \item Lijphart, A. (1977). \textit{Democracy in Plural Societies}. Yale University Press.
    
    \item Huntington, S. (1991). \textit{The Third Wave}. University of Oklahoma Press.
    
    \item Diamond, L. \& Plattner, M. (eds). (2010). \textit{Democratization in Africa}. Johns Hopkins.
    
    \item Liddle, R.W. (1999). "Indonesia's Democratic Opening." \textit{Government and Opposition}, 34(1).
\end{enumerate}
	%══════════════════════════════════════════════════════════════════════════════
% فصل ۵: درس‌های گذارهای ناموفق
% از بحران تا بالندگی
%══════════════════════════════════════════════════════════════════════════════

\chapter{درس‌های گذارهای ناموفق}
\label{ch:failure}

%──────────────────────────────────────────────────────────────────────────────
% کادر خلاصه فصل
%──────────────────────────────────────────────────────────────────────────────
\begin{kholasebox}
این فصل پنج گذار ناموفق یا ناقص را بررسی می‌کند: عراق (۲۰۰۳-امروز)، لیبی (۲۰۱۱-امروز)، مصر (۲۰۱۱-۲۰۱۳)، یمن (۲۰۱۱-۲۰۱۵) و ونزوئلا (۱۹۹۹-امروز). هر یک به دلایل متفاوتی شکست خوردند: مداخله خارجی نادرست، فروپاشی نهادها، ناتوانی در ائتلاف‌سازی، پوپولیسم ویرانگر، یا ترکیبی از اینها. الگوهای مشترک شکست شامل: فقدان توافق نخبگان، تقسیمات عمیق اجتماعی، شکست اقتصادی، و مداخله خارجی مخرب است. مهم‌ترین درس: آنچه نباید کرد به اندازه آنچه باید کرد، آموزنده است.
\end{kholasebox}

%══════════════════════════════════════════════════════════════════════════════
\section{چرا شکست‌ها را مطالعه کنیم؟}
\label{sec:why-failures}
%══════════════════════════════════════════════════════════════════════════════

\begin{naghlbox}
«تاریخ نه‌تنها از موفقیت‌ها، بلکه از شکست‌ها نیز درس می‌دهد — شاید بیشتر. دانستن آنچه نباید کرد، گاه مهم‌تر از دانستن آنچه باید کرد است.»

\hfill --- جورج سانتایانا
\end{naghlbox}

\begin{table}[H]
\centering
\caption{مقایسه اجمالی پنج گذار ناموفق}
\label{tab:five-failures}
\begin{tabularx}{\textwidth}{R{2cm} C{2cm} Y Y C{2cm}}
\toprule
\headmark کشور & \headmark سال & \headmark نوع شکست & \headmark علت اصلی & \headmark وضعیت \\
\midrule
عراق & ۲۰۰۳ & دولت شکننده & فرقه‌گرایی + مداخله خارجی & بی‌ثبات \\
\rowcolor{goldlight}
لیبی & ۲۰۱۱ & فروپاشی کامل & فقدان نهادها + قبیله‌گرایی & جنگ داخلی \\
مصر & ۲۰۱۱-۱۳ & بازگشت نظامیان & شکست ائتلاف & اقتدارگرا \\
\rowcolor{goldlight}
یمن & ۲۰۱۱ & فروپاشی & مداخله منطقه‌ای & بحران شدید \\
ونزوئلا & ۱۹۹۹ & زوال تدریجی & پوپولیسم + نفرین نفت & بحران انسانی \\
\bottomrule
\end{tabularx}
\end{table}

%══════════════════════════════════════════════════════════════════════════════
\section{عراق: فروپاشی نهادها و فرقه‌گرایی}
\label{sec:iraq}
%══════════════════════════════════════════════════════════════════════════════

\subsection{اشتباهات کلیدی}

\begin{enghelabbox}[title={\hfill \textbf{اشتباه مرگبار: انحلال ارتش و De-Ba'athification}}]
پل برمر، حاکم آمریکایی عراق، دو تصمیم فاجعه‌بار گرفت:

\begin{enumerate}[nosep]
    \item \textbf{انحلال ارتش:} ۴۰۰,۰۰۰ نظامی مسلح و بیکار شدند — بسیاری به داعش پیوستند
    \item \textbf{پاکسازی بعثی‌ها:} ۵۰۰,۰۰۰ کارمند دولتی اخراج شدند — دولت فلج شد
\end{enumerate}

\textbf{درس:} نهادها را نابود نکنید، اصلاح کنید. حتی نهادهای رژیم قبلی بهتر از هیچ‌اند.
\end{enghelabbox}

\begin{figure}[H]
\centering
\begin{tikzpicture}[
    node distance=1.5cm,
    errorbox/.style={
        rectangle,
        rounded corners=3pt,
        minimum width=3.2cm,
        minimum height=1.3cm,
        text centered,
        font=\tiny\bfseries,
        draw=rougerevolution,
        fill=rougelight,
        line width=1pt
    },
    resultbox/.style={
        rectangle,
        rounded corners=3pt,
        minimum width=3.2cm,
        minimum height=1.3cm,
        text centered,
        font=\tiny\bfseries,
        draw=gris,
        fill=grislight,
        line width=1pt
    },
    arrow/.style={->, >=Stealth, thick, color=rougerevolution}
]

% اشتباهات
\node[errorbox] (e1) at (0,3) {\rl{انحلال ارتش}};
\node[errorbox] (e2) at (4.5,3) {\rl{پاکسازی بعث}};
\node[errorbox] (e3) at (9,3) {\rl{محاصصه فرقه‌ای}};

% نتایج میانی
\node[resultbox] (r1) at (0,0) {\rl{نظامیان بیکار}};
\node[resultbox] (r2) at (4.5,0) {\rl{فلج اداری}};
\node[resultbox] (r3) at (9,0) {\rl{نهادینه شدن شکاف}};

% نتیجه نهایی
\node[errorbox, draw=rougerevolution, fill=rougerevolution, text=white, minimum width=10cm, minimum height=1.5cm] (final) at (4.5,-2.5) {\textbf{\rl{بحران امنیتی + فروپاشی اجتماعی + دولت شکننده}}};

% فلش‌ها
\draw[arrow] (e1) -- (r1); \draw[arrow] (e2) -- (r2); \draw[arrow] (e3) -- (r3);
\draw[arrow] (r1) -- (final); \draw[arrow] (r2) -- (final); \draw[arrow] (r3) -- (final);

\end{tikzpicture}
\caption{زنجیره علّی شکست عراق}
\end{figure}

\subsection{مشکل محاصصه فرقه‌ای}

\begin{table}[H]
\centering
\caption{سیستم محاصصه در عراق}
\label{tab:iraq-muhasasa}
\begin{tabularx}{\textwidth}{R{3cm} C{2cm} Y}
\toprule
\headmark پست & \headmark سهم & \headmark مشکل \\
\midrule
رئیس‌جمهور & کُرد & نقش تشریفاتی و انزواگرایی \\
\rowcolor{goldlight}
نخست‌وزیر & شیعه & تنش دائمی با سُنی‌ها \\
رئیس پارلمان & سُنی & احساس طرد و حاشیه‌نشینی \\
\bottomrule
\end{tabularx}
\end{table}

\begin{naghlbox}
«محاصصه فرقه‌ای به جای حل مشکل، آن را نهادینه کرد. سیاستمداران انگیزه داشتند هویت فرقه‌ای را تقویت کنند چون کرسی‌شان به آن وابسته بود.»

\hfill --- تحلیل‌گر سیاسی عراقی
\end{naghlbox}

\subsection{درس‌های عراق برای ما}

\begin{table}[H]
\centering
\caption{چه نباید کرد: درس‌های عراق}
\label{tab:iraq-lessons}
\begin{tabularx}{\textwidth}{R{3.5cm} Y Y}
\toprule
\headmark اشتباه عراق & \headmark پیامد & \headmark جایگزین صحیح \\
\midrule
انحلال ارتش & ظهور گروه‌های مسلح & اصلاح تدریجی نهادها \\
\rowcolor{goldlight}
پاکسازی گسترده & فلج بوروکراسی & پاکسازی محدود به سران \\
محاصصه فرقه‌ای & نهادینه شدن تفرقه & احزاب فراقومی و ملی \\
\bottomrule
\end{tabularx}
\end{table}

%══════════════════════════════════════════════════════════════════════════════
\section{لیبی: دولت‌سازی بدون ملت‌سازی}
\label{sec:libya}
%══════════════════════════════════════════════════════════════════════════════

\subsection{میراث قذافی: فقدان نهاد}

\begin{enghelabbox}[title={\hfill \textbf{مشکل بنیادین: کشور بدون دولت}}]
معمر قذافی طی ۴۲ سال حکومت:
\begin{itemize}[nosep]
    \item هیچ نهاد دولتی واقعی نساخت — همه چیز شخصی بود
    \item ارتش را عمداً ضعیف نگه داشت (ترس از کودتا)
    \item جامعه مدنی را سرکوب کرد
    \item بر قبیله‌گرایی تکیه کرد
\end{itemize}

\textbf{نتیجه:} وقتی قذافی سقوط کرد، هیچ چیزی برای ساختن روی آن وجود نداشت.
\end{enghelabbox}

\subsection{تکه‌تکه شدن}

\begin{figure}[H]
\centering
\begin{tikzpicture}[
    node distance=1.5cm,
    fragbox/.style={
        rectangle,
        rounded corners=3pt,
        minimum width=3.2cm,
        minimum height=1.2cm,
        text centered,
        font=\tiny\bfseries,
        line width=1pt
    }
]

% قبل
\node[fragbox, draw=gris, fill=grislight] (before) at (0,0) {\rl{لیبی واحد}\\{\tiny (دوره قذافی)}};

% بعد - تکه‌ها
\node[fragbox, draw=rougerevolution, fill=rougelight] (f1) at (6,2) {\rl{دولت طرابلس}};
\node[fragbox, draw=rougerevolution, fill=rougelight] (f2) at (6,0) {\rl{دولت بنغازی}};
\node[fragbox, draw=rougerevolution, fill=rougelight] (f3) at (6,-2) {\rl{میلیشیاهای محلی}};
\node[fragbox, draw=goldphoenix, fill=goldlight] (f4) at (10,1) {\rl{نیروهای حفتر}};
\node[fragbox, draw=goldphoenix, fill=goldlight] (f5) at (10,-1) {\rl{قبایل جنوب}};

% فلش‌ها
\draw[->, >=Stealth, ultra thick, color=rougerevolution] (before) -- node[above, font=\tiny\bfseries] {\rl{سقوط ۲۰۱۱}} (3.5,0);
\draw[->, >=Stealth, thick, color=gris] (3.5,0) -- (f1);
\draw[->, >=Stealth, thick, color=gris] (3.5,0) -- (f2);
\draw[->, >=Stealth, thick, color=gris] (3.5,0) -- (f3);

\end{tikzpicture}
\caption{تکه‌تکه شدن لیبی پس از سقوط قذافی}
\end{figure}

\subsection{درس‌های لیبی}

\begin{table}[H]
\centering
\caption{درس‌های لیبی}
\label{tab:libya-lessons}
\begin{tabularx}{\textwidth}{R{3.5cm} Y}
\toprule
\headmark درس & \headmark توضیح \\
\midrule
اهمیت نهادها & بدون نهاد، دموکراسی به هرج‌ومرج می‌گراید \\
\rowcolor{goldlight}
خطرات خلأ قدرت & میلیشیاها بلافاصله جایگزین دولت می‌شوند \\
کنترل تسلیحات & ضرورت خلع سلاح عمومی پس از گذار \\
\rowcolor{goldlight}
هویت‌های محلی & مهار قبیله‌گرایی از طریق ملت‌سازی \\
\bottomrule
\end{tabularx}
\end{table}

%══════════════════════════════════════════════════════════════════════════════
\section{مصر: شکست ائتلاف و بازگشت نظامیان}
\label{sec:egypt}
%══════════════════════════════════════════════════════════════════════════════

\subsection{چرا انقلاب ۲۰۱۱ شکست خورد؟}

\begin{figure}[H]
\centering
\begin{tikzpicture}[
    node distance=0.5cm,
    eventbox/.style={
        rectangle,
        rounded corners=3pt,
        minimum width=2.4cm,
        minimum height=1.2cm,
        text centered,
        font=\tiny\bfseries,
        line width=1pt
    }
]

% خط زمان
\draw[line width=2pt, color=gris] (0,0) -- (14,0);

% رویدادها
\node[eventbox, draw=bleurepublique, fill=bleulight, above] at (0,0.5) {\rl{ژانویه ۲۰۱۱}\\\rl{انقلاب}};
\node[eventbox, draw=bleurepublique, fill=bleulique, above] at (3.5,0.5) {\rl{فوریه ۲۰۱۱}\\\rl{سقوط مبارک}};
\node[eventbox, draw=goldphoenix, fill=goldlight, above] at (7,0.5) {\rl{ژوئن ۲۰۱۲}\\\rl{ریاست مرسی}};
\node[eventbox, draw=rougerevolution, fill=rougelight, above] at (10.5,0.5) {\rl{ژوئیه ۲۰۱۳}\\\rl{کودتای سیسی}};
\node[eventbox, draw=rougerevolution, fill=rougerevolution, text=white, above] at (14,0.5) {\rl{۲۰۱۴-امروز}\\\rl{اقتدارگرایی}};

% نقاط
\foreach \x in {0,3.5,7,10.5,14} {
    \fill[bleurepublique] (\x,0) circle (3pt);
}

\end{tikzpicture}
\caption{خط زمانی شکست گذار مصر}
\end{figure}

\begin{enghelabbox}[title={\hfill \textbf{سه دلیل اصلی شکست مصر}}]
\begin{enumerate}[nosep]
    \item \textbf{شکست ائتلاف:} اخوان‌المسلمین و لیبرال‌ها نتوانستند همکاری کنند
    \item \textbf{شکست اقتصادی:} معیشت مردم بدتر شد — صف نان طولانی‌تر
    \item \textbf{ناتوانی در حکمرانی:} مرسی تجربه اداره کشور نداشت
\end{enumerate}

\textbf{نتیجه:} وقتی سیسی کودتا کرد، بسیاری از مردم (حتی انقلابیون) استقبال کردند.
\end{enghelabbox}

\begin{naghlbox}
«ما فکر می‌کردیم دشمن ما فقط مبارک است. نفهمیدیم که دشمن واقعی، ناتوانی ما در ساختن ائتلاف و بهبود زندگی مردم بود.»

\hfill --- فعال مصری، ۲۰۱۵
\end{naghlbox}

\subsection{درس کلیدی مصر}

\begin{olgoobox}[title={\hfill \textbf{درس حیاتی برای ما}}]
\textbf{مصر ثابت کرد که سرنگونی دیکتاتور کافی نیست.} شما باید:
\begin{itemize}[nosep]
    \item ائتلاف گسترده بسازید — حتی با کسانی که دوست‌شان ندارید
    \item اقتصاد را مدیریت کنید — مردم نان می‌خواهند نه فقط آزادی
    \item ظرفیت حکمرانی داشته باشید — ایده کافی نیست، باید بتوانید اجرا کنید
    \item ارتش را مدیریت کنید — یا با شما باشد یا بی‌طرف
\end{itemize}
\end{olgoobox}

%══════════════════════════════════════════════════════════════════════════════
\section{یمن: گفتگوی ملی بدون پشتوانه}
\label{sec:yemen}
%══════════════════════════════════════════════════════════════════════════════

\subsection{چرا گفتگوی ملی یمن شکست خورد؟}

یمن پس از ۲۰۱۱ یک «گفتگوی ملی» بسیار جامع برگزار کرد (۲۰۱۳-۲۰۱۴) با:
\begin{itemize}[nosep]
    \item ۵۶۵ نماینده از همه گروه‌ها
    \item ۱۰ ماه مذاکره
    \item توافق بر فدرالیسم ۶ منطقه‌ای
\end{itemize}

\textbf{اما شکست خورد چون:}

\begin{table}[H]
\centering
\caption{دلایل شکست گفتگوی ملی یمن}
\label{tab:yemen-failure}
\begin{tabularx}{\textwidth}{C{1cm} R{3.5cm} Y}
\toprule
\headmark \# & \headmark دلیل & \headmark توضیح \\
\midrule
۱ & طرد گروه‌های مسلح & حوثی‌ها در فرآیند سیاسی نبودند \\
\rowcolor{goldlight}
۲ & فقدان ضمانت & توافق‌ها هیچ پشتوانه اجرایی نداشتند \\
۳ & بحران معیشتی & فروپاشی اقتصادی انگیزه جنگ را تقویت کرد \\
\rowcolor{goldlight}
۴ & مداخله منطقه‌ای & تبدیل یمن به میدان جنگ‌های نیابتی \\
\bottomrule
\end{tabularx}
\end{table}

\begin{enghelabbox}[title={\hfill \textbf{درس یمن}}]
\textbf{گفتگوی ملی بدون پشتوانه اجرایی، فقط کاغذ است.} شما باید:
\begin{itemize}[nosep]
    \item همه گروه‌های مسلح را پای میز بیاورید
    \item مکانیزم اجرا داشته باشید
    \item مداخله خارجی را مدیریت کنید
    \item اقتصاد را فراموش نکنید
\end{itemize}
\end{enghelabbox}

%══════════════════════════════════════════════════════════════════════════════
\section{ونزوئلا: پوپولیسم و نفرین منابع}
\label{sec:venezuela}
%══════════════════════════════════════════════════════════════════════════════

\subsection{مسیر فروپاشی}

ونزوئلا یک دموکراسی نسبتاً پایدار بود که از درون فروپاشید:

\begin{figure}[H]
\centering
\begin{tikzpicture}[
    node distance=1.5cm,
    stagebox/.style={
        rectangle,
        rounded corners=3pt,
        minimum width=3.5cm,
        minimum height=1.5cm,
        text centered,
        font=\tiny\bfseries,
        line width=1pt
    },
    arrow/.style={->, >=Stealth, thick}
]

% مراحل
\node[stagebox, draw=bleurepublique, fill=bleulight] (s1) at (0,0) {\rl{۱۹۵۸-۱۹۹۸}\\{\tiny دموکراسی نفتی}};
\node[stagebox, draw=goldphoenix, fill=goldlight] (s2) at (5,0) {\rl{۱۹۹۹-۲۰۱۳}\\{\tiny از چاوز تا پوپولیسم}};
\node[stagebox, draw=rougerevolution, fill=rougelight] (s3) at (10,0) {\rl{۲۰۱۳-امروز}\\{\tiny فروپاشی مادورو}};

% فلش‌ها
\draw[arrow, color=goldphoenix] (s1) -- node[above, font=\tiny\bfseries] {\rl{فساد}} (s2);
\draw[arrow, color=rougerevolution] (s2) -- node[above, font=\tiny\bfseries] {\rl{سقوط نفت}} (s3);

\end{tikzpicture}
\caption{مسیر فروپاشی ونزوئلا}
\end{figure}

\begin{enghelabbox}[title={\hfill \textbf{دام ونزوئلایی}}]
چاوز با درآمد نفت:
\begin{itemize}[nosep]
    \item برنامه‌های اجتماعی گسترده اجرا کرد — فقر کاهش یافت
    \item اما نهادها را تخریب کرد — قوه قضائیه، بانک مرکزی، رسانه
    \item اقتصاد را متنوع نکرد — ۹۵٪ صادرات = نفت
    \item وقتی قیمت نفت سقوط کرد، همه چیز فروپاشید
\end{itemize}

\textbf{درس:} پوپولیسم نفتی می‌تواند کوتاه‌مدت محبوب باشد، اما بدون نهادسازی و تنوع اقتصادی، فاجعه‌بار است.
\end{enghelabbox}

%══════════════════════════════════════════════════════════════════════════════
\section{الگوهای مشترک شکست}
\label{sec:failure-patterns}
%══════════════════════════════════════════════════════════════════════════════

\begin{figure}[H]
\centering
\begin{tikzpicture}[
    node distance=1.5cm,
    factor/.style={
        rectangle,
        rounded corners=5pt,
        minimum width=3.5cm,
        minimum height=1.5cm,
        text centered,
        font=\tiny\bfseries,
        draw=rougerevolution,
        fill=rougelight,
        line width=1.5pt
    },
    center/.style={
        circle,
        minimum size=2.8cm,
        text centered,
        font=\small\bfseries,
        draw=gris,
        fill=gris,
        text=white,
        line width=2pt
    }
]

% مرکز
\node[center] (c) at (0,0) {\rl{گذار ناموفق}};

% عوامل
\node[factor] (f1) at (90:4.2) {\rl{فقدان توافق}};
\node[factor] (f2) at (30:4.2) {\rl{شکست اقتصادی}};
\node[factor] (f3) at (330:4.2) {\rl{مداخله مخرب}};
\node[factor] (f4) at (270:4.2) {\rl{فروپاشی نهادی}};
\node[factor] (f5) at (210:4.2) {\rl{تقسیمات فرقه‌ای}};
\node[factor] (f6) at (150:4.2) {\rl{میلیشیاهای مسلح}};

% اتصالات
\foreach \f in {f1,f2,f3,f4,f5,f6} {
    \draw[->, >=Stealth, ultra thick, color=rougerevolution] (\f) -- (c);
}

\end{tikzpicture}
\caption{شش عامل مشترک گذارهای ناموفق}
\end{figure}

%══════════════════════════════════════════════════════════════════════════════
\section{چک‌لیست: آنچه نباید کرد}
\label{sec:donts-checklist}
%══════════════════════════════════════════════════════════════════════════════

\begin{table}[H]
\centering
\caption{چک‌لیست جامع اشتباهات قابل اجتناب}
\label{tab:donts-checklist}
\begin{tabularx}{\textwidth}{C{1cm} R{3.5cm} R{2.5cm} Y}
\toprule
\headmark \# & \headmark اشتباه & \headmark نمونه & \headmark جایگزین \\
\midrule
۱ & انحلال نهادهای قبلی & عراق & اصلاح تدریجی ساختارها \\
\rowcolor{goldlight}
۲ & پاکسازی حداکثری & عراق & پاکسازی محدود به سران \\
۳ & نهادینه کردن تفرقه & عراق & احزاب فراقومی و ملی \\
\rowcolor{goldlight}
۴ & بی‌توجهی به اقتصاد & مصر & بهبود معیشت فوری \\
۵ & طرد مسلحین & یمن & دیپلماسی خلع سلاح \\
\rowcolor{goldlight}
۶ & مداخله بدون طرح & لیبی & برنامه‌ریزی بومی \\
۷ & پوپولیسم نفتی & ونزوئلا & تنوع‌بخشی به اقتصاد \\
\rowcolor{goldlight}
۸ & تخریب نظارت & ونزوئلا & استقلال بانک مرکزی \\
\bottomrule
\end{tabularx}
\end{table}

%══════════════════════════════════════════════════════════════════════════════
\section*{منابع فصل}
%══════════════════════════════════════════════════════════════════════════════

\begin{enumerate}[nosep, label={[\arabic*]}]
    \item Dodge, T. (2012). \textit{Iraq: From War to a New Authoritarianism}. Routledge.
    
    \item Wehrey, F. (2018). \textit{The Burning Shores: Inside the Battle for the New Libya}. FSG.
    
    \item Kandil, H. (2015). \textit{Inside the Brotherhood}. Polity Press.
    
    \item Lackner, H. (2017). \textit{Yemen in Crisis}. Saqi Books.
    
    \item Corrales, J. \& Penfold, M. (2011). \textit{Dragon in the Tropics: Hugo Chavez}. Brookings.
    
    \item Diamond, L. (2015). "Facing Up to the Democratic Recession." \textit{Journal of Democracy}, 26(1).
    
    \item Brownlee, J. et al. (2015). \textit{The Arab Spring: Pathways of Repression and Reform}. Oxford.
    
    \item Brancati, D. (2016). \textit{Democracy Protests}. Cambridge University Press.
\end{enumerate}
	%══════════════════════════════════════════════════════════════════════════════
% فصل ۶: درس‌های مدیریت بحران آب
% از بحران تا بالندگی
%══════════════════════════════════════════════════════════════════════════════

\chapter{درس‌های مدیریت بحران آب}
\label{ch:water}

%──────────────────────────────────────────────────────────────────────────────
% کادر خلاصه فصل
%──────────────────────────────────────────────────────────────────────────────
\begin{kholasebox}
این فصل تجارب موفق جهانی در مدیریت بحران آب را بررسی می‌کند: اسرائیل (از کمبود به صادرات فناوری)، سنگاپور (چهار شیر ملی)، استرالیا (مدیریت خشکسالی هزاره)، و اسپانیا (انتقال آب و تنش‌ها). درس‌های کلیدی شامل: مدیریت تقاضا مهم‌تر از افزایش عرضه است؛ قیمت‌گذاری واقعی ضروری است؛ فناوری‌های جدید (شیرین‌سازی، بازچرخانی) تحول‌آفرین‌اند؛ و حکمرانی آب باید یکپارچه باشد. برای کشور ما که با فروپاشی سفره‌های زیرزمینی مواجه است، این تجارب راهگشایند.
\end{kholasebox}

%══════════════════════════════════════════════════════════════════════════════
\section{چرا مدیریت آب حیاتی است؟}
\label{sec:water-importance}
%══════════════════════════════════════════════════════════════════════════════

\begin{enghelabbox}[title={\hfill \textbf{یادآوری: بحران آب ما}}]
همان‌طور که در فصل ۲ دیدیم:
\begin{itemize}[nosep]
    \item ۷۰٪ سفره‌های زیرزمینی در وضعیت بحرانی یا ممنوعه
    \item برداشت مازاد سالانه: ۷+ میلیارد مترمکعب
    \item ۹۰٪ مصرف آب: کشاورزی (اغلب ناکارآمد)
    \item بدون اقدام: ۳۰-۵۰ میلیون آواره اقلیمی تا ۲۰۴۰
\end{itemize}

این فصل نشان می‌دهد که \textbf{راه‌حل وجود دارد} — اگر اراده سیاسی باشد.
\end{enghelabbox}

%══════════════════════════════════════════════════════════════════════════════
\section{اسرائیل: از کمبود به فراوانی}
\label{sec:israel-water}
%══════════════════════════════════════════════════════════════════════════════

\subsection{معجزه آب اسرائیل}

اسرائیل کشوری است که ۶۰٪ آن بیابان است، اما امروز:
\begin{itemize}[nosep]
    \item صادرکننده محصولات کشاورزی است
    \item صادرکننده فناوری آب به ۱۵۰ کشور است
    \item مازاد آب دارد و به همسایگان می‌فروشد
\end{itemize}

\begin{figure}[H]
\centering
\begin{tikzpicture}
\begin{axis}[
    width=13cm,
    height=7cm,
    xlabel={سال},
    ylabel={میلیون مترمکعب در سال},
    xmin=1960, xmax=2025,
    ymin=0, ymax=2500,
    legend style={at={(0.02,0.98)}, anchor=north west},
    grid=both,
    grid style={line width=0.2pt, draw=gray!30}
]

% منابع سنتی
\addplot[color=bleurepublique, very thick, mark=*] coordinates {
    (1960,1400) (1970,1500) (1980,1600) (1990,1650) (2000,1600) (2010,1500) (2020,1400)
};

% شیرین‌سازی
\addplot[color=vertnapoleon, very thick, mark=square*] coordinates {
    (1960,0) (1970,0) (1980,0) (1990,0) (2000,50) (2010,300) (2020,700)
};

% بازچرخانی
\addplot[color=orroyal, very thick, mark=triangle*] coordinates {
    (1960,0) (1970,50) (1980,100) (1990,200) (2000,300) (2010,450) (2020,600)
};

\legend{منابع طبیعی, شیرین‌سازی, بازچرخانی فاضلاب}

\end{axis}
\end{tikzpicture}
\caption{تحول ترکیب منابع آب اسرائیل}
\label{fig:israel-water-sources}
\end{figure}

\subsection{چهار ستون موفقیت اسرائیل}

\begin{olgoobox}[title={\hfill \textbf{الگوی اسرائیل}}]
\begin{enumerate}[nosep]
    \item \textbf{آبیاری قطره‌ای:} اختراع اسرائیلی — ۹۰٪ صرفه‌جویی در آب کشاورزی
    \item \textbf{شیرین‌سازی:} ۵ کارخانه بزرگ — ۷۰٪ آب شرب
    \item \textbf{بازچرخانی فاضلاب:} ۸۷٪ فاضلاب بازچرخانی می‌شود (رتبه اول جهان)
    \item \textbf{قیمت‌گذاری واقعی:} آب ارزان نیست — انگیزه صرفه‌جویی
\end{enumerate}
\end{olgoobox}

\begin{table}[H]
\centering
\caption{مقایسه شاخص‌های آب: اسرائیل و کشور ما}
\label{tab:israel-comparison}
\begin{tabular}{L{5cm} C{3cm} C{3cm}}
\toprule
\headmark شاخص & \headmark اسرائیل & \headmark کشور ما \\
\midrule
\rowcolor{vertlight}
بارش سالانه (میلیمتر) & ۵۰۰ & ۲۵۰ \\
\rowcolor{rougelight}
بازچرخانی فاضلاب & ۸۷٪ & کمتر از ۱۰٪ \\
\rowcolor{vertlight}
آبیاری قطره‌ای/بارانی & ۷۵٪ اراضی & کمتر از ۱۵٪ \\
\rowcolor{rougelight}
اتلاف در شبکه توزیع & ۱۰٪ & ۳۰٪+ \\
\rowcolor{vertlight}
شیرین‌سازی & ۷۰٪ آب شرب & نزدیک صفر \\
\bottomrule
\end{tabular}
\end{table}

\subsection{درس‌های اسرائیل}

\begin{table}[H]
\centering
\caption{درس‌های قابل انتقال از اسرائیل}
\label{tab:israel-lessons}
\begin{tabular}{L{3.5cm} L{4.5cm} L{4.5cm}}
\toprule
\headmark درس & \headmark کاربرد & \headmark چالش اجرا \\
\midrule
\rowcolor{vertlight}
آبیاری قطره‌ای & تغییر الگوی آبیاری & سرمایه اولیه بالا \\
\rowcolor{vertlight}
شیرین‌سازی & مناطق ساحلی جنوب & هزینه انرژی \\
\rowcolor{vertlight}
بازچرخانی فاضلاب & همه شهرهای بزرگ & زیرساخت + پذیرش فرهنگی \\
\rowcolor{bleulight}
قیمت‌گذاری واقعی & کاهش مصرف ناکارآمد & مقاومت سیاسی \\
\bottomrule
\end{tabular}
\end{table}

%══════════════════════════════════════════════════════════════════════════════
\section{سنگاپور: چهار شیر ملی}
\label{sec:singapore-water}
%══════════════════════════════════════════════════════════════════════════════

\subsection{چالش سنگاپور}

سنگاپور جزیره کوچکی است با:
\begin{itemize}[nosep]
    \item ۷۲۰ کیلومتر مربع مساحت
    \item ۵.۵ میلیون جمعیت
    \item بدون رودخانه یا دریاچه مهم
    \item تا ۱۹۶۵: ۱۰۰٪ وابسته به واردات آب از مالزی
\end{itemize}

\subsection{استراتژی چهار شیر ملی}

\begin{figure}[H]
\centering
\begin{tikzpicture}[
    node distance=2cm,
    tapbox/.style={
        rectangle,
        rounded corners=5pt,
        minimum width=3.5cm,
        minimum height=2cm,
        text centered,
        font=\small,
        line width=1.5pt
    }
]

% چهار شیر
\node[tapbox, draw=bleurepublique, fill=bleulight] (t1) at (0,0) {
    \begin{tabular}{c}
    \textbf{شیر ۱}\\[3pt]
    آب وارداتی\\
    {\scriptsize از مالزی}\\[3pt]
    {\footnotesize ۴۰٪}
    \end{tabular}
};

\node[tapbox, draw=vertnapoleon, fill=vertlight] (t2) at (5,0) {
    \begin{tabular}{c}
    \textbf{شیر ۲}\\[3pt]
    جمع‌آوری باران\\
    {\scriptsize ۱۷ مخزن}\\[3pt]
    {\footnotesize ۲۰٪}
    \end{tabular}
};

\node[tapbox, draw=violetempire, fill=violetlight] (t3) at (10,0) {
    \begin{tabular}{c}
    \textbf{شیر ۳}\\[3pt]
    NEWater\\
    {\scriptsize بازچرخانی}\\[3pt]
    {\footnotesize ۳۰٪}
    \end{tabular}
};

\node[tapbox, draw=orroyal, fill=orroyallight] (t4) at (15,0) {
    \begin{tabular}{c}
    \textbf{شیر ۴}\\[3pt]
    شیرین‌سازی\\
    {\scriptsize دریا}\\[3pt]
    {\footnotesize ۱۰٪}
    \end{tabular}
};

% هدف آینده
\node[font=\small, text width=12cm, align=center] at (7.5,-2.5) {
    \textbf{هدف ۲۰۶۰:} کاهش وابستگی به مالزی به صفر\\
    NEWater: ۵۵٪ | شیرین‌سازی: ۲۵٪ | جمع‌آوری باران: ۲۰٪
};

\end{tikzpicture}
\caption{استراتژی چهار شیر ملی سنگاپور}
\label{fig:singapore-taps}
\end{figure}

\begin{olgoobox}[title={\hfill \textbf{NEWater: انقلاب بازچرخانی}}]
NEWater فاضلاب تصفیه‌شده با فناوری پیشرفته است:
\begin{itemize}[nosep]
    \item سه مرحله تصفیه: میکروفیلتراسیون + اسمز معکوس + UV
    \item کیفیت بالاتر از استانداردهای آب آشامیدنی WHO
    \item مصرف: صنعت (wafer fabrication) + مخلوط با آب آشامیدنی
    \item پذیرش عمومی: کمپین آموزشی گسترده + بازدید از کارخانه
\end{itemize}

\textbf{درس:} با فناوری و آموزش، حتی بازچرخانی فاضلاب قابل پذیرش است.
\end{olgoobox}

\subsection{درس‌های سنگاپور}

\begin{table}[H]
\centering
\caption{درس‌های قابل انتقال از سنگاپور}
\label{tab:singapore-lessons}
\begin{tabular}{L{4cm} L{8.5cm}}
\toprule
\headmark درس & \headmark توضیح \\
\midrule
\rowcolor{vertlight}
تنوع منابع & وابستگی به یک منبع = آسیب‌پذیری \\
\rowcolor{vertlight}
بازچرخانی پیشرفته & با فناوری، فاضلاب به آب آشامیدنی تبدیل می‌شود \\
\rowcolor{vertlight}
آموزش عمومی & پذیرش فرهنگی به اندازه فناوری مهم است \\
\rowcolor{vertlight}
افق بلندمدت & برنامه‌ریزی ۵۰ ساله — نه فقط برای انتخابات بعدی \\
\bottomrule
\end{tabular}
\end{table}

%══════════════════════════════════════════════════════════════════════════════
\section{استرالیا: مدیریت خشکسالی هزاره}
\label{sec:australia-water}
%══════════════════════════════════════════════════════════════════════════════

\subsection{بحران Murray-Darling}

حوضه Murray-Darling سبد غذایی استرالیا است، اما خشکسالی ۱۹۹۷-۲۰۱۰ آن را به بحران کشاند:
\begin{itemize}[nosep]
    \item کاهش ۷۰٪ جریان رودخانه
    \item ورشکستگی هزاران کشاورز
    \item مرگ اکوسیستم‌های آبی
\end{itemize}

\subsection{راه‌حل: بازار آب}

\begin{olgoobox}[title={\hfill \textbf{الگوی استرالیا: حقوق قابل معامله آب}}]
استرالیا «حقوق آب» را از «زمین» جدا کرد:
\begin{itemize}[nosep]
    \item کشاورزان سهمی از آب دارند (نه آب نامحدود)
    \item می‌توانند این سهم را بفروشند یا بخرند
    \item قیمت بازاری تعیین می‌شود — کمیابی منعکس می‌شود
    \item کسانی که آب را کارآمدتر مصرف می‌کنند، سود می‌برند
\end{itemize}

\textbf{نتیجه:} آب به سمت مصارف با ارزش بالاتر رفت. صرفه‌جویی ۳۰٪.
\end{olgoobox}

\begin{figure}[H]
\centering
\begin{tikzpicture}[
    node distance=1.5cm,
    box/.style={
        rectangle,
        rounded corners=3pt,
        minimum width=4cm,
        minimum height=1.5cm,
        text centered,
        font=\small,
        line width=1pt
    },
    arrow/.style={->, >=Stealth, thick}
]

% قبل
\node[box, draw=rougerevolution, fill=rougelight] (b1) at (0,0) {
    \begin{tabular}{c}
    \textbf{قبل}\\
    آب = حق نامحدود\\
    هدررفت گسترده
    \end{tabular}
};

% بعد
\node[box, draw=vertnapoleon, fill=vertlight] (b2) at (8,0) {
    \begin{tabular}{c}
    \textbf{بعد}\\
    آب = دارایی قابل معامله\\
    مصرف کارآمد
    \end{tabular}
};

% فلش
\draw[arrow, very thick, color=bleurepublique] (b1) -- node[above, font=\small] {اصلاحات ۲۰۰۴-۲۰۰۷} (b2);

\end{tikzpicture}
\caption{تحول مدل حکمرانی آب در استرالیا}
\label{fig:australia-reform}
\end{figure}

\subsection{درس‌های استرالیا}

\begin{table}[H]
\centering
\caption{درس‌های قابل انتقال از استرالیا}
\label{tab:australia-lessons}
\begin{tabular}{L{4cm} L{4.5cm} L{4cm}}
\toprule
\headmark درس & \headmark کاربرد & \headmark چالش \\
\midrule
\rowcolor{vertlight}
بازار آب & تخصیص کارآمد & نیاز به حقوق مالکیت روشن \\
\rowcolor{vertlight}
سقف برداشت & جلوگیری از تخلیه بیش از حد & مقاومت کشاورزان \\
\rowcolor{bleulight}
جبران کشاورزان & خرید حقوق آب توسط دولت & هزینه بالا \\
\rowcolor{bleulight}
نظارت دقیق & کنتورهای هوشمند & زیرساخت \\
\bottomrule
\end{tabular}
\end{table}

%══════════════════════════════════════════════════════════════════════════════
\section{جمع‌بندی: اصول جهانی مدیریت پایدار آب}
\label{sec:water-principles}
%══════════════════════════════════════════════════════════════════════════════

\begin{figure}[H]
\centering
\begin{tikzpicture}[
    node distance=1.5cm,
    principle/.style={
        rectangle,
        rounded corners=5pt,
        minimum width=4cm,
        minimum height=1.8cm,
        text centered,
        font=\small,
        draw=bleurepublique,
        fill=bleulight,
        line width=1.5pt
    }
]

% اصول
\node[principle] (p1) at (0,3) {
    \begin{tabular}{c}
    \textbf{۱. مدیریت تقاضا}\\
    {\scriptsize مهم‌تر از افزایش عرضه}
    \end{tabular}
};

\node[principle] (p2) at (6,3) {
    \begin{tabular}{c}
    \textbf{۲. قیمت‌گذاری واقعی}\\
    {\scriptsize آب رایگان = هدررفت}
    \end{tabular}
};

\node[principle] (p3) at (12,3) {
    \begin{tabular}{c}
    \textbf{۳. تنوع منابع}\\
    {\scriptsize وابستگی به یک منبع = ریسک}
    \end{tabular}
};

\node[principle] (p4) at (0,0) {
    \begin{tabular}{c}
    \textbf{۴. فناوری}\\
    {\scriptsize شیرین‌سازی + بازچرخانی}
    \end{tabular}
};

\node[principle] (p5) at (6,0) {
    \begin{tabular}{c}
    \textbf{۵. حکمرانی یکپارچه}\\
    {\scriptsize یک نهاد مسئول}
    \end{tabular}
};

\node[principle] (p6) at (12,0) {
    \begin{tabular}{c}
    \textbf{۶. مشارکت محلی}\\
    {\scriptsize ذینفعان در تصمیم‌گیری}
    \end{tabular}
};

% مرکز
\node[circle, minimum size=2.5cm, draw=vertnapoleon, fill=vertlight, line width=2pt] at (6,-3) {
    \begin{tabular}{c}
    \textbf{آب}\\
    \textbf{پایدار}
    \end{tabular}
};

% اتصالات
\foreach \p in {p1,p2,p3,p4,p5,p6} {
    \draw[->, >=Stealth, thick, color=vertnapoleon] (\p) -- (6,-1.5);
}

\end{tikzpicture}
\caption{شش اصل جهانی مدیریت پایدار آب}
\label{fig:water-principles}
\end{figure}

%══════════════════════════════════════════════════════════════════════════════
\section{کاربرد برای کشور ما}
\label{sec:water-application}
%══════════════════════════════════════════════════════════════════════════════

\begin{table}[H]
\centering
\caption{برنامه پیشنهادی مدیریت آب بر اساس تجارب جهانی}
\label{tab:water-plan}
\small
\begin{tabular}{L{3cm} L{4cm} L{3.5cm} C{2cm}}
\toprule
\headmark اقدام & \headmark الگوی جهانی & \headmark اولویت مکانی & \headmark فاز \\
\midrule
\rowcolor{bleulight}
کاهش اتلاف شبکه & همه کشورها & شهرهای بزرگ & ۱ \\
تغییر الگوی کشت & اسرائیل، استرالیا & مناطق بحرانی & ۱-۲ \\
\rowcolor{bleulight}
آبیاری قطره‌ای & اسرائیل & دشت‌های مرکزی & ۱-۳ \\
بازچرخانی فاضلاب & سنگاپور، اسرائیل & شهرهای بزرگ & ۲-۳ \\
\rowcolor{bleulight}
شیرین‌سازی & اسرائیل، سنگاپور & ساحل جنوب & ۲-۴ \\
بازار حقوق آب & استرالیا & حوضه‌های بحرانی & ۳-۴ \\
\bottomrule
\end{tabular}
\end{table}

\begin{tahlilbox}[title={\hfill \textbf{نتیجه‌گیری}}]
بحران آب ما قابل حل است — اگر:
\begin{enumerate}[nosep]
    \item \textbf{اراده سیاسی} باشد: تصمیمات سخت گرفته شود
    \item \textbf{مدیریت تقاضا} اولویت باشد: نه فقط سد و انتقال
    \item \textbf{قیمت‌گذاری} واقعی شود: یارانه هدفمند به فقرا
    \item \textbf{فناوری} وارد شود: بازچرخانی و شیرین‌سازی
    \item \textbf{کشاورزی} متحول شود: تغییر الگوی کشت و آبیاری
    \item \textbf{حکمرانی} یکپارچه باشد: یک نهاد مسئول
\end{enumerate}

کشورهایی با شرایط بدتر از ما (اسرائیل، سنگاپور) موفق شدند. ما هم می‌توانیم.
\end{tahlilbox}

%══════════════════════════════════════════════════════════════════════════════
\section*{منابع فصل}
%══════════════════════════════════════════════════════════════════════════════

\begin{enumerate}[nosep, label={[\arabic*]}]
    \item Siegel, S. (2015). \textit{Let There Be Water: Israel's Solution for a Water-Starved World}. Thomas Dunne.
    
    \item Tortajada, C. et al. (2013). \textit{The Singapore Water Story}. Routledge.
    
    \item Grafton, R.Q. et al. (2011). "Determinants of Residential Water Consumption." \textit{Land Economics}, 87(4).
    
    \item World Bank. (2017). \textit{Water Scarce Cities: Thriving in a Finite World}. WB Publications.
    
    \item FAO. (2020). \textit{The State of Food and Agriculture: Water}. Rome: FAO.
    
    \item Gleick, P. (2014). \textit{The World's Water Volume 8}. Island Press.
    
    \item OECD. (2015). \textit{Water and Cities: Ensuring Sustainable Futures}. OECD Publishing.
    
    \item PUB Singapore. (2023). \textit{Our Water, Our Future}. pub.gov.sg.
\end{enumerate}
	
	%──────────────────────────────────────────────────────────────────────────────
	% بخش سوم: طرح جامع
	%──────────────────────────────────────────────────────────────────────────────
	\part{طرح جامع}
	
	%═══════════════════════════════════════════════════════════════════════════════
% فصل ۷: چشم‌انداز و اصول راهنما
% فایل: chapters/chapter07.tex
%═══════════════════════════════════════════════════════════════════════════════

\chapter{چشم‌انداز و اصول راهنما}
\chapterheader{۷}{چشم‌انداز و اصول}{ترسیم ایرانی که آرزویش را داریم}{AccentGold}
\label{chap:vision}

\begin{kholasebox}
\textbf{خلاصه فصل:}
این فصل چشم‌انداز بلندمدت برای ایران دموکراتیک ۱۴۲۹ (افق ۲۵ ساله) را ترسیم می‌کند. ده اصل بنیادین شامل حاکمیت ملی، کثرت‌گرایی، عدالت توزیعی، و پایداری محیط‌زیستی به‌عنوان ستون‌های نظام جدید معرفی می‌شوند. چارچوب قانون اساسی پیشنهادی بر تفکیک قوا، فدرالیسم همبسته، و حقوق بنیادین شهروندی استوار است. همچنین مدل «آبادانی ملموس» به‌عنوان استراتژی اعتمادسازی در سال‌های نخست تبیین می‌گردد.
\end{kholasebox}

%───────────────────────────────────────────────────────────────────────────────
\section{مقدمه: ضرورت چشم‌انداز روشن}
%───────────────────────────────────────────────────────────────────────────────

گذار دموکراتیک بدون چشم‌انداز روشن، همچون سفری است بدون مقصد. تجربه فصل‌های پیشین نشان داد که کشورهایی مانند آفریقای جنوبی و کره جنوبی با ترسیم آینده‌ای امیدبخش توانستند انرژی اجتماعی را بسیج کنند، درحالی‌که لیبی و یمن به دلیل فقدان چشم‌انداز مشترک در هرج‌ومرج فروغلتیدند.

\begin{naghlbox}
«اگر نمی‌دانی به کجا می‌روی، هر راهی تو را به جایی می‌برد.»
\sourceline{ضرب‌المثل آفریقایی}
\end{naghlbox}

برای ایران، کشوری با ۸۵ میلیون جمعیت، ۱۰۰ اتنیک و زیرگروه قومی، و بحران‌های متعدد زیست‌محیطی و اقتصادی، این چشم‌انداز باید:

\begin{itemize}
    \item \textbf{فراگیر} باشد و همه اقوام و گروه‌ها را در خود جای دهد
    \item \textbf{واقع‌بینانه} باشد و با محدودیت‌ها هم‌خوانی داشته باشد
    \item \textbf{الهام‌بخش} باشد و انگیزه تغییر ایجاد کند
    \item \textbf{سنجش‌پذیر} باشد و شاخص‌های موفقیت داشته باشد
\end{itemize}

%───────────────────────────────────────────────────────────────────────────────
\section{بیانیه چشم‌انداز ایران ۱۴۲۹}
\label{sec:vision-statement}
%───────────────────────────────────────────────────────────────────────────────

\begin{olgoobox}
\textbf{چشم‌انداز ایران ۱۴۲۹ (افق ۲۵ ساله):}

\medskip
\textit{«ایران در سال ۱۴۲۹، کشوری است دموکراتیک، مرفه، و پایدار که در آن:}

\begin{itemize}[nosep]
    \item \textit{هر شهروند فارغ از قومیت، جنسیت، و باور، از حقوق برابر برخوردار است}
    \item \textit{حکومت پاسخگو، شفاف، و برآمده از رأی آزاد مردم است}
    \item \textit{اقوام گوناگون در چارچوب وحدت ملی به خودمدیریت دست یافته‌اند}
    \item \textit{منابع آب و محیط زیست احیا شده و برای نسل‌های آینده محفوظ است}
    \item \textit{اقتصاد متنوع و دانش‌بنیان، رفاه عادلانه را برای همگان فراهم کرده است}
    \item \textit{ایران عضو محترم جامعه جهانی و الگوی گذار صلح‌آمیز در منطقه است»}
\end{itemize}
\end{olgoobox}

\subsection{تحلیل اجزای چشم‌انداز}

% جدول تحلیل اجزای چشم‌انداز
\begin{table}[htbp]
\centering
\caption{تحلیل شش رکن چشم‌انداز ایران ۱۴۲۹}
\label{tab:vision-components}
\begin{tabularx}{\textwidth}{R{3cm} Y Z}
\toprule
\headmark رکن چشم‌انداز & \headmark شاخص موفقیت & \headmark وضعیت کنونی \\
\midrule
حقوق برابر شهروندی & شاخص برابری > ۰.۸ & ۰.۵۷ (رتبه ۱۴۳) \\
\rowcolor{goldlight}
حکومت پاسخگو & شاخص دموکراسی > ۷ & ۱.۹۶ (رتبه ۱۵۴) \\
خودمدیریت اقوام & رضایت قومی > ۷۵٪ & نامعلوم \\
\rowcolor{goldlight}
احیای محیط زیست & سفره‌های آب > ۸۰٪ & ۴۵٪ (کاهنده) \\
اقتصاد متنوع & سهم نفت < ۲۰٪ & ۷۲٪ \\
\rowcolor{goldlight}
عضویت جهانی & رفع تحریم‌ها & ۳۸۰۰+ تحریم فعال \\
\bottomrule
\end{tabularx}
\end{table}

%───────────────────────────────────────────────────────────────────────────────
\section{ده اصل بنیادین نظام نوین}
\label{sec:ten-principles}
%───────────────────────────────────────────────────────────────────────────────

اصول زیر به‌عنوان «میثاق ملی گذار» پیشنهاد می‌شوند. این اصول غیرقابل تجدیدنظر بوده و سنگ بنای قانون اساسی جدید خواهند بود.

% نمودار TikZ - ده اصل بنیادین
\begin{figure}[htbp]
\centering
\begin{tikzpicture}[
    scale=0.8,
    transform shape,
    principle/.style={
        rectangle,
        rounded corners=5pt,
        minimum width=3.2cm, text width=3.2cm,
        minimum height=0.9cm,
        draw=bleurepublique!70,
        fill=bleulight,
        text=bleurepublique,
        font=\small\bfseries,
        align=center
    },
    core/.style={
        circle,
        minimum size=2.8cm, text width=2.8cm,
        draw=goldphoenix,
        fill=goldlight,
        text=goldphoenix,
        font=\large\bfseries,
        align=center
    },
    arrow/.style={
        ->,
        >=stealth,
        thick,
        draw=bleurepublique!40
    }
]

% هسته مرکزی
\node[core] (center) at (0,0) {
    \begin{tabular}{c}
    ایران\\
    دموکراتیک
    \end{tabular}
};

% اصول - دایره بیرونی
\node[principle] (p1) at (90:4.2) {\rl{۱. حاکمیت ملی}};
\node[principle] (p2) at (54:4.2) {\rl{۲. کثرت‌گرایی}};
\node[principle] (p3) at (18:4.2) {\rl{۳. عدالت توزیعی}};
\node[principle] (p4) at (-18:4.2) {\rl{۴. حقوق بشر}};
\node[principle] (p5) at (-54:4.2) {\rl{۵. پایداری محیطی}};
\node[principle] (p6) at (-90:4.2) {\rl{۶. تمرکززدایی}};
\node[principle] (p7) at (-126:4.2) {\rl{۷. پاسخگویی}};
\node[principle] (p8) at (-162:4.2) {\rl{۸. صلح‌طلبی}};
\node[principle] (p9) at (162:4.2) {\rl{۹. حکمرانی علمی}};
\node[principle] (p10) at (126:4.2) {\rl{۱۰. عدالت انتقالی}};

% اتصالات
\foreach \i in {p1,p2,p3,p4,p5,p6,p7,p8,p9,p10} {
    \draw[arrow] (\i) -- (center);
}
\end{tikzpicture}
\caption{ده اصل بنیادین نظام دموکراتیک ایران}
\end{figure}

\subsection{اصل اول: حاکمیت ملی و مردم‌سالاری}

\begin{quotation}
\textbf{بیانیه اصل:} حاکمیت به‌طور کامل و انحصاری از آنِ ملت ایران است. هیچ فرد، گروه، نهاد، یا ایدئولوژی نمی‌تواند این حاکمیت را غصب یا مصادره کند. قدرت سیاسی تنها از طریق انتخابات آزاد، منصفانه، و دوره‌ای کسب می‌شود.
\end{quotation}

\textbf{الزامات عملیاتی:}
\begin{enumerate}
    \item برگزاری انتخابات آزاد در همه سطوح (محلی، استانی، ملی)
    \item تضمین رقابت چندحزبی واقعی
    \item استقلال کامل کمیسیون انتخابات
    \item دوره‌ای بودن قدرت (حداکثر دو دوره ریاست‌جمهوری)
    \item همه‌پرسی برای تغییرات بنیادین قانون اساسی
\end{enumerate}

\begin{enghelabbox}
\textbf{خط قرمز:} هرگونه ادعای «ولایت»، «رهبری مادام‌العمر»، یا «نظارت استصوابی» که مانع اعمال حاکمیت مردم شود، مردود و غیرقانونی است. تجربه تاریخی ایران نشان داده که نظارت نهادهای غیرانتخابی بر نهادهای انتخابی، اصل مردم‌سالاری را نقض می‌کند.
\end{enghelabbox}

\subsection{اصل دوم: کثرت‌گرایی و حقوق اقلیت‌ها}

\begin{quotation}
\textbf{بیانیه اصل:} ایران سرزمین همه ایرانیان است، فارغ از قومیت، زبان، مذهب، و جنسیت. تنوع قومی-فرهنگی نه تهدید که گنجینه ملی است. اکثریت حق ندارد حقوق بنیادین اقلیت‌ها را نقض کند.
\end{quotation}

% جدول ترکیب قومی ایران
\begin{table}[htbp]
\centering
\caption{ترکیب قومی-زبانی ایران (برآورد ۱۴۰۳)}
\label{tab:ethnic-composition}
\begin{tabularx}{\textwidth}{R{3cm} C{2cm} C{2cm} Y}
\toprule
\headmark گروه قومی & \headmark جمعیت & \headmark درصد & \headmark زبان اصلی \\
\midrule
فارس & ۴۵-۵۰ & ۵۳-۵۹٪ & فارسی \\
\rowcolor{goldlight}
آذربایجانی & ۱۵-۲۰ & ۱۸-۲۴٪ & ترکی آذربایجانی \\
کُرد & ۷-۹ & ۸-۱۰٪ & کردی \\
\rowcolor{goldlight}
لُر و بختیاری & ۵-۶ & ۶-۷٪ & لری/بختیاری \\
عرب & ۲-۳ & ۲-۴٪ & عربی \\
\rowcolor{goldlight}
بلوچ & ۲-۳ & ۲-۳٪ & بلوچی \\
ترکمن & ۱-۲ & ۱-۲٪ & ترکمنی \\
\rowcolor{goldlight}
سایر & ۲-۳ & ۲-۳٪ & متنوع \\
\midrule
\headmark مجموع & \textbf{۸۵} & \textbf{۱۰۰٪} & \\
\bottomrule
\end{tabularx}
\end{table}

\textbf{الزامات عملیاتی:}
\begin{itemize}
    \item رسمیت زبان‌های محلی در آموزش، رسانه، و امور اداری استانی
    \item نمایندگی تضمینی اقلیت‌ها در نهادهای ملی
    \item ممنوعیت تبعیض نژادی، قومی، و مذهبی
    \item حمایت از میراث فرهنگی همه اقوام
    \item رسانه عمومی به زبان‌های قومی
\end{itemize}

\subsection{اصل سوم: عدالت توزیعی و فرصت‌های برابر}

\begin{quotation}
\textbf{بیانیه اصل:} ثروت ملی متعلق به همه نسل‌های ایرانیان است. توسعه متوازن مناطق و کاهش شکاف طبقاتی از وظایف اصلی حکومت است. هیچ منطقه‌ای نباید به دلیل دوری از مرکز، محروم بماند.
\end{quotation}

% نمودار نابرابری منطقه‌ای
\begin{figure}[htbp]
\centering
\begin{tikzpicture}
\begin{axis}[
    ybar,
    width=0.95\textwidth,
    height=7cm,
    ylabel={\rl{درآمد سرانه}},
    xlabel={\rl{استان}},
    ymin=0, ymax=900,
    xtick=data,
    xticklabels={\rl{تهران}, \rl{اصفهان}, \rl{یزد}, \rl{میانگین}, \rl{آذربایجان}, \rl{کردستان}, \rl{خوزستان}, \rl{سیستان}},
    xticklabel style={rotate=45, anchor=east, font=\tiny},
    nodes near coords,
    nodes near coords align={vertical},
    every node near coord/.append style={font=\tiny},
    bar width=0.5cm,
    enlarge x limits=0.1,
    grid=major,
    grid style={dashed, gray!30},
    axis line style={bleurepublique!50, thick}
]
\addplot[fill=bleurepublique!70, draw=bleurepublique] coordinates {
    (1,780) (2,520) (3,490) (4,380) (5,310) (6,290) (7,340) (8,180)
};
\addplot[fill=goldphoenix!70, draw=goldphoenix] coordinates {
    (1,380) (2,380) (3,380) (4,380) (5,380) (6,380) (7,380) (8,380)
};
\legend{\rl{درآمد استانی}, \rl{میانگین کشوری}}
\end{axis}
\end{tikzpicture}
\caption{نابرابری درآمد سرانه بین استان‌های ایران (۱۴۰۲)}
\end{figure}

\begin{enghelabbox}
\textbf{واقعیت تکان‌دهنده:} درآمد سرانه در تهران ۴.۳ برابر سیستان‌وبلوچستان است. این شکاف در ۴۰ سال گذشته نه‌تنها کاهش نیافته، بلکه عمیق‌تر شده است. استان‌های حاشیه‌ای که اغلب مسکن اقلیت‌های قومی هستند، بیشترین محرومیت را تجربه می‌کنند.
\end{enghelabbox}

\textbf{الزامات عملیاتی:}
\begin{enumerate}
    \item تخصیص ۴۰٪ بودجه عمرانی به استان‌های محروم (۱۰ سال اول)
    \item بازتوزیع عادلانه درآمد منابع طبیعی
    \item ایجاد صندوق توسعه مناطق محروم
    \item شاخص‌گذاری پیشرفت مناطق و پاسخگویی دولت
\end{enumerate}

\subsection{اصل چهارم: حقوق بشر و کرامت انسانی}

\begin{quotation}
\textbf{بیانیه اصل:} کرامت ذاتی انسان خدشه‌ناپذیر است. حقوق بنیادین شامل حق حیات، آزادی، امنیت، بیان، تجمع، عقیده، و دادرسی عادلانه برای همگان تضمین می‌شود. این حقوق فراتر از قانون عادی بوده و حتی با اکثریت پارلمانی قابل سلب نیست.
\end{quotation}

% جدول مقایسه وضعیت حقوق بشر
\begin{table}[htbp]
\centering
\caption{شاخص‌های حقوق بشر: ایران در مقایسه با کشورهای منتخب}
\label{tab:human-rights-comparison}
\begin{tabularx}{\textwidth}{R{2.8cm} C{1.2cm} C{1.2cm} C{1.4cm} Y}
\toprule
\headmark شاخص & \headmark ایران & \headmark ترکیه & \headmark آفریقای ج. & \headmark هدف ۱۴۲۹ \\
\midrule
آزادی مطبوعات & ۱۷۶ & ۱۴۹ & ۳۲ & زیر ۵۰ \\
\rowcolor{goldlight}
آزادی اینترنت & ۱۶ & ۳۲ & ۵۸ & بالای ۶۰ \\
اعدام سالانه & ۵۸۰+ & ۰ & ۰ & ۰ \\
\rowcolor{goldlight}
زندانی سیاسی & ۱۰۰۰+ & ۴۰۰۰۰+ & زیر ۱۰ & ۰ \\
برابری جنسیتی & ۱۴۳ & ۱۲۹ & ۲۰ & زیر ۵۰ \\
\bottomrule
\end{tabularx}
\end{table}

\begin{naghlbox}
«حقوق بشر مرز ندارد. وقتی از حقوق مردمی در گوشه‌ای از جهان دفاع می‌کنیم، از حقوق همه انسان‌ها دفاع کرده‌ایم.»
\sourceline{شیرین عبادی، برنده نوبل صلح ۲۰۰۳}
\end{naghlbox}

\textbf{الزامات عملیاتی:}
\begin{itemize}
    \item لغو مجازات اعدام (با استثنای محدود جنایات جنگی)
    \item تعطیلی بازداشتگاه‌های مخفی
    \item آزادی کامل زندانیان سیاسی و عقیدتی
    \item پذیرش بازرسی بین‌المللی از زندان‌ها
    \item تصویب کنوانسیون ضد شکنجه و پروتکل اختیاری آن
\end{itemize}

\subsection{اصل پنجم: پایداری محیط‌زیستی}

\begin{quotation}
\textbf{بیانیه اصل:} محیط زیست سالم حق نسل حاضر و امانت برای نسل‌های آینده است. هیچ توسعه‌ای نباید به قیمت تخریب بازگشت‌ناپذیر اکوسیستم‌ها باشد. بحران آب به‌عنوان تهدید امنیت ملی شناسایی می‌شود.
\end{quotation}

% نمودار بحران آب
\begin{figure}[htbp]
\centering
\begin{tikzpicture}
\begin{axis}[
    width=0.95\textwidth,
    height=7cm,
    xlabel={\rl{سال (شمسی)}},
    ylabel={\rl{میلیارد مترمکعب/سال}},
    xmin=1380, xmax=1430,
    ymin=0, ymax=140,
    grid=major,
    grid style={dashed, gray!30},
    legend style={at={(0.5,-0.15)}, anchor=north, font=\tiny, legend columns=2},
    axis line style={bleurepublique!50, thick}
]
\addplot[thick, color=bleurepublique!80, mark=*] coordinates {
    (1380,130) (1385,125) (1390,115) (1395,105) (1400,95) (1403,88)
};
\addplot[thick, color=rougerevolution, mark=square*] coordinates {
    (1380,88) (1385,94) (1390,102) (1395,108) (1400,112) (1403,115)
};
\addplot[thick, dashed, color=bleurepublique!40] coordinates {
    (1403,88) (1410,75) (1420,60) (1429,50)
};
\addplot[thick, dashed, color=rougerevolution!50] coordinates {
    (1403,115) (1410,120) (1420,125) (1429,130)
};
\addplot[thick, dotted, color=goldphoenix, mark=triangle*] coordinates {
    (1403,115) (1410,100) (1420,85) (1429,75)
};
\legend{
    \rl{منابع (واقعی)},
    \rl{مصرف (واقعی)},
    \rl{منابع (ادامه روند)},
    \rl{مصرف (ادامه روند)},
    \rl{مصرف (با اصلاحات)}
}
\end{axis}
\end{tikzpicture}
\caption{بحران آب ایران: شکاف فزاینده عرضه و تقاضا}
\end{figure}

\begin{enghelabbox}
\textbf{هشدار بحران:} اگر روند فعلی ادامه یابد، تا سال ۱۴۲۰ (۱۵ سال دیگر) بیش از ۵۰ میلیون ایرانی با کمبود شدید آب مواجه خواهند شد. این می‌تواند به مهاجرت‌های گسترده داخلی و بی‌ثباتی اجتماعی منجر شود.
\end{enghelabbox}

\subsection{اصل ششم: تمرکززدایی و فدرالیسم همبسته}

\begin{quotation}
\textbf{بیانیه اصل:} قدرت باید در نزدیک‌ترین سطح ممکن به شهروندان اعمال شود (اصل تقرب). استان‌ها و مناطق در امور محلی خودمختار بوده و صلاحیت‌های روشنی دارند. وحدت ملی از طریق همبستگی داوطلبانه تقویت می‌شود، نه اجبار مرکز.
\end{quotation}

% نمودار ساختار فدرالیسم همبسته
\begin{figure}[htbp]
\centering
\begin{tikzpicture}[
    scale=0.85,
    transform shape,
    level/.style={
        rectangle,
        rounded corners=5pt,
        draw=bleurepublique,
        fill=bleulight,
        text=bleurepublique,
        minimum width=3.5cm, text width=3.5cm,
        minimum height=1cm,
        font=\small\bfseries,
        align=center
    },
    competence/.style={
        rectangle,
        rounded corners=3pt,
        draw=gray!30,
        fill=white,
        font=\tiny,
        align=right,
        text width=4.5cm
    },
    arrow/.style={
        <->,
        thick,
        draw=goldphoenix!60
    }
]

% سطوح حکومتی
\node[level] (national) at (0,6) {\rl{سطح ملی (مرکزی)}};
\node[level] (regional) at (0,3) {\rl{سطح منطقه‌ای (۵ منطقه)}};
\node[level] (provincial) at (0,0) {\rl{سطح استانی (۳۱ استان)}};
\node[level] (local) at (0,-3) {\rl{سطح محلی (شهر و روستا)}};

% صلاحیت‌ها
\node[competence] (nat-comp) at (6,6) {دفاع، امنیت، سیاست خارجی، پول، تجارت خارجی};
\node[competence] (reg-comp) at (6,3) {توسعه منطقه‌ای، حوضه‌های آبریز، زیرساخت بین‌استانی};
\node[competence] (prov-comp) at (6,0) {آموزش، بهداشت، پلیس محلی، فرهنگ و محیط استانی};
\node[competence] (loc-comp) at (6,-3) {شهرسازی، مسکن، حمل‌ونقل، آب و فاضلاب شهری};

% اتصالات
\draw[arrow] (national) -- (regional);
\draw[arrow] (regional) -- (provincial);
\draw[arrow] (provincial) -- (local);

\draw[->, gray!30] (nat-comp.west) -- (national.east);
\draw[->, gray!30] (reg-comp.west) -- (regional.east);
\draw[->, gray!30] (prov-comp.west) -- (provincial.east);
\draw[->, gray!30] (loc-comp.west) -- (local.east);

\node[above, font=\small\bfseries, goldphoenix] at (6,7) {\rl{صلاحیت‌ها}};
\end{tikzpicture}
\caption{ساختار فدرالیسم همبسته و تقسیم صلاحیت‌ها}
\end{figure}

\begin{olgoobox}
\textbf{الگوی آلمان:} جمهوری فدرال آلمان با ۱۶ ایالت (Länder) نمونه موفقی از فدرالیسم همبسته است. ایالت‌ها در آموزش، فرهنگ، و پلیس مستقل‌اند اما در سیاست خارجی و دفاع تابع برلین هستند. این مدل وحدت و تنوع را هم‌زمان حفظ کرده است.
\end{olgoobox}

\subsection{اصل هفتم: پاسخگویی و شفافیت}

\begin{quotation}
\textbf{بیانیه اصل:} هر مقام عمومی در برابر مردم پاسخگوست. شفافیت در تصمیم‌گیری و مالی عمومی اصل است. فساد به‌عنوان جرم علیه ملت تلقی و با شدت مقابله می‌شود.
\end{quotation}

% جدول شاخص فساد
\begin{table}[htbp]
\centering
\caption{شاخص ادراک فساد (CPI): ایران و کشورهای منطقه}
\label{tab:corruption-index}
\begin{tabularx}{\textwidth}{R{3cm} Y C{1.5cm} C{1.5cm}}
\toprule
\headmark کشور & \headmark امتیاز (از ۱۰۰) & \headmark رتبه & \headmark تغییر \\
\midrule
امارات & ۶۸ & ۲۵ & +۵ \\
\rowcolor{goldlight}
ترکیه & ۳۶ & ۵۵ & -۴ \\
هند & ۴۰ & ۸۳ & +۲ \\
\rowcolor{goldlight}
ایران & ۲۴ & ۱۴۹ & -۳ \\
عراق & ۲۳ & ۱۵۴ & +۵ \\
\midrule
\headmark هدف ۱۴۲۹ & \textbf{۵۰+} & \textbf{زیر ۵۰} & \textbf{+۲۶} \\
\bottomrule
\end{tabularx}
\end{table}

\textbf{الزامات عملیاتی:}
\begin{enumerate}
    \item قانون دسترسی آزاد به اطلاعات
    \item اعلام دارایی مقامات (قبل و بعد از تصدی)
    \item استقلال دیوان محاسبات و تقویت اختیاراتش
    \item حمایت از افشاگران فساد (Whistleblower Protection)
    \item دادگاه‌های ویژه مبارزه با فساد
\end{enumerate}

\subsection{اصل هشتم: صلح‌طلبی و همزیستی منطقه‌ای}

\begin{quotation}
\textbf{بیانیه اصل:} ایران خواهان صلح با همه همسایگان و کشورهای جهان است. ماجراجویی نظامی و صدور انقلاب کنار گذاشته می‌شود. امنیت ملی از طریق دیپلماسی، همکاری اقتصادی، و قدرت نرم تأمین می‌شود.
\end{quotation}

\begin{naghlbox}
«ما می‌خواهیم دوست همه و دشمن هیچ‌کس باشیم.»
\sourceline{مصدق در نطق مجلس، ۱۳۳۰}
\end{naghlbox}

% نقشه همسایگان
\begin{figure}[htbp]
\centering
\begin{tikzpicture}[scale=0.75, transform shape]
    % ایران - مرکز
    \fill[bleurepublique!30, draw=bleurepublique, thick] (0,0) ellipse (2.5 and 1.8);
    \node[font=\large\bfseries, text=bleurepublique] at (0,0) {\rl{ایران}};
    
    % همسایگان - دایره
    \node[circle, fill=goldlight, draw=goldphoenix, minimum size=1.5cm, text width=1.5cm, font=\small] (tur) at (120:4.5) {\rl{ترکیه}};
    \node[circle, fill=goldlight, draw=goldphoenix, minimum size=1.5cm, text width=1.5cm, font=\small] (arm) at (150:4) {\rl{ارمنستان}};
    \node[circle, fill=goldlight, draw=goldphoenix, minimum size=1.5cm, text width=1.5cm, font=\small] (aze) at (90:4) {\rl{آذربایجان}};
    \node[circle, fill=goldlight, draw=goldphoenix, minimum size=1.5cm, text width=1.5cm, font=\small] (tur2) at (60:4.5) {\rl{ترکمنستان}};
    \node[circle, fill=goldlight, draw=goldphoenix, minimum size=1.5cm, text width=1.5cm, font=\small] (afg) at (30:4.5) {\rl{افغانستان}};
    \node[circle, fill=goldlight, draw=goldphoenix, minimum size=1.5cm, text width=1.5cm, font=\small] (pak) at (-30:4.5) {\rl{پاکستان}};
    \node[circle, fill=goldlight, draw=goldphoenix, minimum size=1.5cm, text width=1.5cm, font=\small] (uae) at (-60:4.5) {\rl{امارات}};
    \node[circle, fill=goldlight, draw=goldphoenix, minimum size=1.5cm, text width=1.5cm, font=\small] (irq) at (-120:4.5) {\rl{عراق}};
    \node[circle, fill=goldlight, draw=goldphoenix, minimum size=1.5cm, text width=1.5cm, font=\small] (sau) at (-150:4) {\rl{عربستان}};
    
    % خطوط ارتباطی
    \foreach \neighbor in {tur,arm,aze,tur2,afg,pak,uae,irq,sau} {
        \draw[bleurepublique!30, thick] (0,0) -- (\neighbor);
    }
\end{tikzpicture}
\caption{وضعیت روابط ایران با همسایگان (۱۴۰۳)}
\end{figure}

\textbf{اولویت‌های سیاست خارجی:}
\begin{enumerate}
    \item عادی‌سازی روابط با همسایگان عربی
    \item حل دائمی پرونده هسته‌ای و رفع تحریم‌ها
    \item خروج از درگیری‌های نیابتی منطقه‌ای
    \item پیوستن به پیمان‌های تجارت آزاد منطقه‌ای
    \item همکاری در مبارزه با تروریسم و قاچاق
\end{enumerate}

\subsection{اصل نهم: حکمرانی علمی و مبتنی بر شواهد}

\begin{quotation}
\textbf{بیانیه اصل:} سیاست‌گذاری باید بر پایه داده‌ها، تحقیقات، و تجربه جهانی باشد، نه ایدئولوژی یا سلیقه. دانشگاه‌ها و مراکز پژوهشی استقلال دارند و صدای مشورتی حکومت هستند.
\end{quotation}

\begin{olgoobox}
\textbf{مدل سنگاپور:} سنگاپور با ایجاد «واحدهای تحقیق سیاست» در هر وزارتخانه، تصمیم‌گیری مبتنی بر شواهد را نهادینه کرده است. هر سیاست جدید باید پیش‌ازاجرا آزمایش شده و نتایج به‌صورت عمومی منتشر شود. این رویکرد خطاهای پرهزینه را به‌شدت کاهش داده است.
\end{olgoobox}

\textbf{الزامات عملیاتی:}
\begin{itemize}
    \item تأسیس مرکز ملی سیاست‌پژوهی مستقل
    \item الزام ارزیابی اثر قبل از تصویب قوانین (RIA)
    \item نظام آمارگیری مستقل و شفاف
    \item آزادی پژوهش و ممنوعیت سانسور علمی
    \item بودجه پژوهش حداقل ۳٪ GDP
\end{itemize}

\subsection{اصل دهم: عدالت انتقالی و آشتی ملی}

\begin{quotation}
\textbf{بیانیه اصل:} گذار به دموکراسی بدون برخورد با گذشته ممکن نیست. اما انتقام به نام عدالت مردود است. هدف، کشف حقیقت، احقاق حق قربانیان، و جلوگیری از تکرار است.
\end{quotation}

% نمودار چرخه عدالت انتقالی
\begin{figure}[htbp]
\centering
\begin{tikzpicture}[
    scale=0.85,
    transform shape,
    element/.style={
        rectangle,
        rounded corners=5pt,
        minimum width=3cm, text width=3cm,
        minimum height=1.2cm,
        draw=bleurepublique!70,
        fill=bleulight,
        text=bleurepublique,
        font=\small\bfseries,
        align=center
    },
    arrow/.style={
        ->,
        >=stealth,
        thick,
        draw=goldphoenix!60
    }
]

% چهار رکن عدالت انتقالی
\node[element] (truth) at (0,3) {\rl{کشف حقیقت}\\\tiny کمیسیون حقیقت‌ياب};
\node[element] (justice) at (5,3) {\rl{محاکمه عادلانه}\text{\tiny دادگاه‌های ویژه}};
\node[element] (repair) at (5,0) {\rl{جبران خسارت}\\\tiny غرامت به قربانیان};
\node[element] (reform) at (0,0) {\rl{اصلاح نهادها}\\\tiny تطهیر و بازسازی};

% مرکز
\node[circle, draw=goldphoenix, fill=goldlight, minimum size=2.2cm, text width=2.2cm, 
      font=\small\bfseries, text=goldphoenix, align=center] 
      (center) at (2.5,1.5) {\rl{آشتی ملی}};

% اتصالات
\draw[arrow] (truth) -- (justice);
\draw[arrow] (justice) -- (repair);
\draw[arrow] (repair) -- (reform);
\draw[arrow] (reform) -- (truth);

\draw[arrow, goldphoenix!40] (truth) -- (center);
\draw[arrow, goldphoenix!40] (justice) -- (center);
\draw[arrow, goldphoenix!40] (repair) -- (center);
\draw[arrow, goldphoenix!40] (reform) -- (center);
\end{tikzpicture}
\caption{چهار رکن عدالت انتقالی و هدف نهایی آشتی ملی}
\end{figure}

\begin{enghelabbox}
\textbf{تعادل ظریف:} عدالت انتقالی یکی از پیچیده‌ترین چالش‌های گذار است. افراط در مجازات (مانند عراق پس از صدام) می‌تواند به بی‌ثباتی و انتقام‌جویی بینجامد. تفریط در برخورد (مانند اسپانیای پس از فرانکو) می‌تواند زخم‌ها را التیام‌ناپذیر کند. راه میانه، کشف حقیقت همراه با مصالحه مشروط است.
\end{enghelabbox}

%───────────────────────────────────────────────────────────────────────────────
\section{چارچوب قانون اساسی پیشنهادی}
\label{sec:constitutional-framework}
%───────────────────────────────────────────────────────────────────────────────

بر اساس ده اصل فوق، چارچوب کلی قانون اساسی جدید پیشنهاد می‌شود.

\subsection{ساختار قوای حکومتی}

% نمودار تفکیک قوا
\begin{figure}[htbp]
\centering
\begin{tikzpicture}[
    scale=0.75,
    transform shape,
    power/.style={
        rectangle,
        rounded corners=8pt,
        minimum width=4cm, text width=4cm,
        minimum height=2cm,
        draw=#1!70!black,
        fill=#1!15,
        text=#1!30!black,
        font=\bfseries,
        align=center
    },
    subunit/.style={
        rectangle,
        rounded corners=3pt,
        minimum width=2.8cm, text width=2.8cm,
        minimum height=0.7cm,
        draw=#1!50!black,
        fill=#1!10,
        font=\scriptsize,
        align=center
    },
    check/.style={
        <->,
        thick,
        draw=red!60,
        shorten >=5pt,
        shorten <=5pt
    }
]

% قوه مقننه
\node[power=blue] (leg) at (-5,4) {
    \begin{tabular}{c}
    قوه مقننه\\
    (مجلس شورای ملی)
    \end{tabular}
};
\node[subunit=blue] at (-6.5,2.2) {مجلس اول (۳۰۰ نفر)};
\node[subunit=blue] at (-3.5,2.2) {مجلس اقوام (۱۰۰ نفر)};

% قوه مجریه
\node[power=green] (exe) at (5,4) {
    \begin{tabular}{c}
    قوه مجریه\\
    (رئیس‌جمهور و کابینه)
    \end{tabular}
};
\node[subunit=green] at (3.5,2.2) {رئیس‌جمهور};
\node[subunit=green] at (6.5,2.2) {نخست‌وزیر};

% قوه قضائیه
\node[power=purple] (jud) at (0,-1) {
    \begin{tabular}{c}
    قوه قضائیه\\
    (دادگستری مستقل)
    \end{tabular}
};
\node[subunit=purple] at (-2,-2.8) {دادگاه قانون اساسی};
\node[subunit=purple] at (2,-2.8) {دیوان‌عالی کشور};

% نهادهای نظارتی مستقل
\node[power=orange] (watch) at (0,7) {
    \begin{tabular}{c}
    نهادهای نظارتی مستقل
    \end{tabular}
};
\node[subunit=orange] at (-3,5.5) {کمیسیون انتخابات};
\node[subunit=orange] at (0,5.5) {دیوان محاسبات};
\node[subunit=orange] at (3,5.5) {کمیسیون حقوق بشر};

% چک‌ها و بالانس‌ها
\draw[check] (leg) -- (exe) node[midway, above, font=\tiny] {استیضاح/قانون‌گذاری};
\draw[check] (leg) -- (jud) node[midway, left, font=\tiny] {عزل قضات/بودجه};
\draw[check] (exe) -- (jud) node[midway, right, font=\tiny] {عفو/انتصاب};

% رأی مردم
\node[ellipse, draw=cyan!70, fill=cyan!10, minimum width=3cm, text width=3cm, minimum height=1cm,
      font=\small\bfseries, text=cyan!70] (people) at (0,-4.5) {رأی مستقیم مردم};
\draw[->, thick, cyan!60] (people) -- (leg);
\draw[->, thick, cyan!60] (people) -- (exe);
\draw[->, thick, cyan!60] (people) -| (-7,4) -| (watch);

\end{tikzpicture}
\caption{ساختار تفکیک قوا و نظارت متقابل در نظام پیشنهادی}
\label{fig:separation-of-powers}
\end{figure}

\subsection{ویژگی‌های نظام پارلمانی-ریاستی ترکیبی}

\begin{table}[htbp]
\centering
\caption{مقایسه نظام‌های مختلف حکومتی و پیشنهاد برای ایران}
\label{tab:government-systems}
\begin{tabular}{>{\columncolor{violet!8}}r p{3cm} p{3cm} p{3cm} p{3cm}}
\toprule
\rowcolor{violet!25}
\textbf{معیار} & \textbf{ریاستی (آمریکا)} & \textbf{پارلمانی (آلمان)} & \textbf{نیمه‌ریاستی (فرانسه)} & \textbf{پیشنهاد ایران} \\
\midrule
رئیس دولت & رئیس‌جمهور & صدراعظم & رئیس‌جمهور + نخست‌وزیر & رئیس‌جمهور + نخست‌وزیر \\
\rowcolor{gray!10}
انتخاب رئیس‌جمهور & مستقیم & پارلمان & مستقیم & مستقیم \\
انحلال پارلمان & خیر & بله & بله & محدود \\
\rowcolor{gray!10}
استیضاح رئیس‌جمهور & دشوار & — & دشوار & ممکن (۳/۲ مجلس) \\
دوره ریاست‌جمهوری & ۴ سال & — & ۵ سال & ۵ سال \\
\rowcolor{gray!10}
محدودیت دوره & ۲ دوره & ندارد & ۲ دوره & ۲ دوره \\
\bottomrule
\end{tabular}
\end{table}

\subsection{مجلس دوم: مجلس اقوام و مناطق}

یکی از نوآوری‌های مهم نظام پیشنهادی، ایجاد مجلس دوم با نام \textbf{«مجلس اقوام و مناطق»} است. این مجلس تضمین‌کننده صدای اقلیت‌ها در قانون‌گذاری ملی است.

\begin{table}[htbp]
\centering
\caption{ترکیب پیشنهادی مجلس اقوام و مناطق (۱۰۰ کرسی)}
\label{tab:senate-composition}
\begin{tabular}{>{\columncolor{teal!8}}r l c l}
\toprule
\rowcolor{teal!25}
\textbf{ردیف} & \textbf{منطقه/گروه} & \textbf{کرسی} & \textbf{نحوه انتخاب} \\
\midrule
۱ & منطقه آذربایجان (شرقی، غربی، اردبیل، زنجان) & ۱۶ & مستقیم \\
\rowcolor{gray!10}
۲ & منطقه کردستان (کردستان، کرمانشاه، ایلام) & ۱۲ & مستقیم \\
۳ & منطقه بلوچستان (سیستان‌وبلوچستان) & ۸ & مستقیم \\
\rowcolor{gray!10}
۴ & منطقه عرب‌نشین (خوزستان) & ۱۰ & مستقیم \\
۵ & منطقه ترکمن‌صحرا (گلستان) & ۶ & مستقیم \\
\rowcolor{gray!10}
۶ & منطقه لرستان و بختیاری & ۸ & مستقیم \\
۷ & منطقه مرکزی (۱۵ استان فارس‌زبان) & ۳۰ & مستقیم \\
\rowcolor{gray!10}
۸ & اقلیت‌های دینی (ارمنی، آشوری، یهودی، زرتشتی) & ۶ & انتخاب جوامع \\
۹ & نمایندگان انتصابی (صاحب‌نظران) & ۴ & انتصاب رئیس‌جمهور \\
\midrule
\rowcolor{teal!15}
& \textbf{جمع} & \textbf{۱۰۰} & \\
\bottomrule
\end{tabular}
\end{table}

%═══════════════════════════════════════════════════════════════════════════════
% ادامه فصل ۷: چشم‌انداز و اصول راهنما
% فایل: chapters/ch07-vision.tex (ادامه)
%═══════════════════════════════════════════════════════════════════════════════

% ادامه از صلاحیت‌های مجلس اقوام...

\textbf{صلاحیت‌های مجلس اقوام:}
\begin{enumerate}
    \item وتو بر قوانین مربوط به حقوق اقوام و زبان‌ها
    \item تأیید انتصاب استانداران مناطق قومی
    \item نظارت بر اجرای عدالت توزیعی بین مناطق
    \item تصویب بودجه استان‌های محروم
    \item بررسی شکایات تبعیض قومی و منطقه‌ای
    \item پیشنهاد اصلاح قانون اساسی در حوزه حقوق اقوام
\end{enumerate}

\begin{olgoobox}
\textbf{الگوی بوسنی-هرزگوین:} مجلس اقوام بوسنی (Dom Naroda) با ۱۵ عضو از سه گروه قومی اصلی (بوشنیاک، صرب، کروات) حق وتو بر قوانینی دارد که «منافع حیاتی» یک قوم را تهدید کند. هرچند این مدل گاه به بن‌بست منجر می‌شود، اما صلح پایدار را حفظ کرده است.
\end{olgoobox}

%───────────────────────────────────────────────────────────────────────────────
\section{منشور حقوق بنیادین}
\label{sec:bill-of-rights}
%───────────────────────────────────────────────────────────────────────────────

قانون اساسی جدید باید شامل منشور حقوق بنیادین غیرقابل تعلیق باشد. این حقوق حتی در شرایط اضطراری کاملاً سلب‌ناپذیرند.

\subsection{دسته‌بندی حقوق بنیادین}

% نمودار سه نسل حقوق
\begin{figure}[htbp]
\centering
\begin{tikzpicture}[
    scale=0.85,
    transform shape,
    generation/.style={
        rectangle,
        rounded corners=6pt,
        minimum width=4.5cm, text width=4.5cm,
        minimum height=3.5cm,
        draw=#1!70!black,
        fill=#1!10,
        text=#1!30!black
    },
    title/.style={
        font=\bfseries\small,
        text=#1!50!black
    },
    item/.style={
        font=\scriptsize,
        text=black
    }
]

% نسل اول
\node[generation=blue] (gen1) at (0,0) {};
\node[title=blue, above] at (0,1.2) {نسل اول: حقوق مدنی-سیاسی};
\node[item, align=right, text width=3.8cm] at (0,0) {
    • حق حیات و امنیت\\
    • آزادی بیان و مطبوعات\\
    • آزادی تجمع و تشکل\\
    • آزادی عقیده و مذهب\\
    • حق رأی و انتخاب شدن\\
    • دادرسی عادلانه\\
    • منع شکنجه و بازداشت خودسرانه
};

% نسل دوم
\node[generation=green] (gen2) at (5.5,0) {};
\node[title=green, above] at (5.5,1.2) {نسل دوم: حقوق اقتصادی-اجتماعی};
\node[item, align=right, text width=3.8cm] at (5.5,0) {
    • حق کار و دستمزد عادلانه\\
    • حق مسکن مناسب\\
    • حق آموزش رایگان\\
    • حق بهداشت و درمان\\
    • حق تأمین اجتماعی\\
    • حق استراحت و تفریح\\
    • حمایت از خانواده
};

% نسل سوم
\node[generation=orange] (gen3) at (11,0) {};
\node[title=orange, above] at (11,1.2) {نسل سوم: حقوق همبستگی};
\node[item, align=right, text width=3.8cm] at (11,0) {
    • حق توسعه\\
    • حق محیط زیست سالم\\
    • حق صلح\\
    • حق بر میراث مشترک بشری\\
    • حق خودمختاری فرهنگی\\
    • حق دسترسی به اطلاعات\\
    • حقوق نسل‌های آینده
};

% فلش‌های تکامل
\draw[->, thick, gray!60] (gen1.east) -- (gen2.west) 
    node[midway, above, font=\tiny] {تکامل تاریخی};
\draw[->, thick, gray!60] (gen2.east) -- (gen3.west)
    node[midway, above, font=\tiny] {تکامل تاریخی};

% پایه مشترک
\node[rectangle, rounded corners=3pt, draw=purple!60, fill=purple!10,
      minimum width=14cm, text width=14cm, minimum height=0.8cm, font=\small\bfseries,
      text=purple!60] at (5.5,-2.5) {کرامت ذاتی انسان: پایه مشترک همه حقوق};

\draw[->, purple!40] (0,-1.8) -- (0,-2.1);
\draw[->, purple!40] (5.5,-1.8) -- (5.5,-2.1);
\draw[->, purple!40] (11,-1.8) -- (11,-2.1);

\end{tikzpicture}
\caption{سه نسل حقوق بشر و تکامل تاریخی آن‌ها}
\label{fig:three-generations}
\end{figure}

\subsection{حقوق ویژه زنان}

با توجه به تبعیض تاریخی علیه زنان در ایران، قانون اساسی جدید باید حقوق ویژه‌ای را تضمین کند:

\begin{table}[htbp]
\centering
\caption{حقوق ویژه زنان در قانون اساسی پیشنهادی}
\label{tab:women-rights}
\begin{tabular}{>{\columncolor{pink!8}}r p{5cm} p{5cm}}
\toprule
\rowcolor{pink!25}
\textbf{ردیف} & \textbf{حق} & \textbf{تضمین اجرایی} \\
\midrule
۱ & برابری کامل در حقوق مدنی & حذف تمام قوانین تبعیض‌آمیز \\
\rowcolor{gray!10}
۲ & آزادی پوشش & ممنوعیت اجبار حکومتی \\
۳ & حق طلاق برابر & اصلاح قانون خانواده \\
\rowcolor{gray!10}
۴ & حق حضانت برابر & اولویت مصلحت کودک \\
۵ & دستمزد برابر & بازرسی کار و جریمه \\
\rowcolor{gray!10}
۶ & حداقل ۳۰٪ کرسی پارلمان & سیستم زیپ در لیست‌های انتخاباتی \\
۷ & ممنوعیت ازدواج کودک & حداقل سن ۱۸ سال \\
\rowcolor{gray!10}
۸ & جرم‌انگاری خشونت خانگی & قانون ویژه و خانه‌های امن \\
\bottomrule
\end{tabular}
\end{table}

\begin{naghlbox}
«وقتی زنان آزاد شوند، جامعه آزاد می‌شود. هیچ ملتی بدون آزادی نیمی از خود به آزادی نمی‌رسد.»
\sourceline{شیرین عبادی}
\end{naghlbox}

\subsection{حقوق اقلیت‌های دینی}

\begin{itemize}
    \item \textbf{آزادی کامل عقیده:} هر شهروند حق دارد هر دین یا بی‌دینی را برگزیند
    \item \textbf{ممنوعیت ارتداد:} تغییر دین جرم نیست
    \item \textbf{برابری در استخدام دولتی:} دین شرط احراز شغل نیست
    \item \textbf{به رسمیت شناختن بهائیان:} پایان تبعیض تاریخی
    \item \textbf{آموزش دینی اختیاری:} نه اجباری در مدارس
    \item \textbf{حمایت از اماکن مقدس:} همه ادیان
\end{itemize}

%───────────────────────────────────────────────────────────────────────────────
\section{مدل «آبادانی ملموس»: استراتژی اعتمادسازی}
\label{sec:tangible-prosperity}
%───────────────────────────────────────────────────────────────────────────────

\begin{olgoobox}
\textbf{اصل کلیدی:} مردم باید در کوتاه‌مدت (۶ ماه تا ۲ سال اول) بهبود محسوس در زندگی روزمره خود احساس کنند. این «آبادانی ملموس» سرمایه اجتماعی لازم برای اصلاحات دشوار بلندمدت را فراهم می‌کند.
\end{olgoobox}

\subsection{منطق آبادانی ملموس}

% نمودار چرخه اعتماد
\begin{figure}[htbp]
\centering
\begin{tikzpicture}[
    scale=0.9,
    transform shape,
    node distance=2.5cm,
    box/.style={
        rectangle,
        rounded corners=5pt,
        draw=#1!70!black,
        fill=#1!15,
        text=#1!30!black,
        minimum width=2.8cm, text width=2.8cm,
        minimum height=1.3cm,
        font=\small\bfseries,
        align=center
    },
    arrow/.style={
        ->,
        >=stealth,
        thick,
        draw=#1!60
    }
]

% چرخه مثبت
\node[box=green] (improve) at (0,0) {
    \begin{tabular}{c}
    بهبود ملموس\\
    زندگی
    \end{tabular}
};

\node[box=blue] (trust) at (4,0) {
    \begin{tabular}{c}
    افزایش اعتماد\\
    به نظام
    \end{tabular}
};

\node[box=purple] (support) at (4,-3) {
    \begin{tabular}{c}
    حمایت از\\
    اصلاحات
    \end{tabular}
};

\node[box=orange] (reform) at (0,-3) {
    \begin{tabular}{c}
    اجرای اصلاحات\\
    ساختاری
    \end{tabular}
};

% فلش‌ها
\draw[arrow=green] (improve) -- (trust);
\draw[arrow=blue] (trust) -- (support);
\draw[arrow=purple] (support) -- (reform);
\draw[arrow=orange] (reform) -- (improve);

% مرکز
\node[circle, draw=red!60, fill=red!10, minimum size=1.5cm, text width=1.5cm,
      font=\scriptsize\bfseries, text=red!60, align=center] at (2,-1.5) {
    \begin{tabular}{c}
    چرخه\\
    فضیلت
    \end{tabular}
};

% چرخه منفی (خارج)
\node[box=gray, opacity=0.6] (fail) at (9,0) {
    \begin{tabular}{c}
    عدم بهبود\\
    محسوس
    \end{tabular}
};
\node[box=gray, opacity=0.6] (distrust) at (9,-3) {
    \begin{tabular}{c}
    بی‌اعتمادی\\
    و سرخوردگی
    \end{tabular}
};
\node[box=gray, opacity=0.6] (resist) at (12.5,-1.5) {
    \begin{tabular}{c}
    مقاومت در برابر\\
    اصلاحات
    \end{tabular}
};

\draw[arrow=gray, opacity=0.5] (fail) -- (distrust);
\draw[arrow=gray, opacity=0.5] (distrust) -- (resist);
\draw[arrow=gray, opacity=0.5] (resist) -- (fail);

\node[font=\scriptsize, text=gray] at (10.5,-1.5) {چرخه معیوب};

\end{tikzpicture}
\caption{چرخه فضیلت آبادانی ملموس در مقابل چرخه معیوب سرخوردگی}
\label{fig:virtue-cycle}
\end{figure}

\subsection{دستور کار ۱۰۰ روز اول}

دولت گذار باید در ۱۰۰ روز اول، تغییرات محسوس زیر را اجرا کند:

\begin{table}[htbp]
\centering
\caption{برنامه ۱۰۰ روز اول: اقدامات با تأثیر فوری}
\label{tab:100-days}
\begin{tabular}{>{\columncolor{green!8}}r p{3.5cm} p{4cm} p{3.5cm}}
\toprule
\rowcolor{green!25}
\textbf{هفته} & \textbf{اقدام} & \textbf{اثر ملموس} & \textbf{هزینه/منبع} \\
\midrule
۱-۲ & آزادی زندانیان سیاسی & امید و شادی عمومی & صفر \\
\rowcolor{gray!10}
۱-۲ & لغو گشت ارشاد & آزادی محسوس زنان & صرفه‌جویی ۵۰۰ میلیارد \\
۳-۴ & رفع فیلترینگ اینترنت & دسترسی آزاد اطلاعات & صرفه‌جویی ۲۰۰ میلیارد \\
\rowcolor{gray!10}
۵-۶ & پرداخت یارانه نقدی مضاعف & بهبود معیشت فوری & ۲۰۰ هزار میلیارد/سال \\
۷-۸ & آغاز مذاکرات رفع تحریم & امید به آینده اقتصادی & نیاز به دیپلماسی \\
\rowcolor{gray!10}
۹-۱۰ & کاهش ۵۰٪ قیمت دارو & دسترسی به درمان & یارانه دارو \\
۱۱-۱۲ & آغاز پروژه‌های اشتغال‌زایی & کاهش بیکاری & صندوق توسعه ملی \\
\rowcolor{gray!10}
۱۳-۱۴ & تعطیلات رسمی نوروز و چهارشنبه‌سوری & احترام به فرهنگ ملی & صفر \\
\bottomrule
\end{tabular}
\end{table}

\subsection{برنامه‌های آبادانی سریع (Quick Wins)}

% نمودار اولویت‌بندی
\begin{figure}[htbp]
\centering
\begin{tikzpicture}[scale=0.9]
\begin{axis}[
    width=12cm,
    height=10cm,
    xlabel={سهولت اجرا (۱-۱۰)},
    ylabel={تأثیر بر زندگی مردم (۱-۱۰)},
    xmin=0, xmax=11,
    ymin=0, ymax=11,
    grid=major,
    grid style={dashed, gray!30},
    xtick={0,2,4,6,8,10},
    ytick={0,2,4,6,8,10},
    scatter/classes={
        quick={mark=*, draw=green!70!black, fill=green!50},
        medium={mark=square*, draw=orange!70!black, fill=orange!50},
        hard={mark=triangle*, draw=red!70!black, fill=red!50}
    }
]

% Quick Wins (بالا-راست)
\addplot[scatter, only marks, scatter src=explicit symbolic, 
         nodes near coords, 
         point meta=explicit symbolic,
         every node near coord/.append style={font=\tiny, anchor=south west}]
table[meta=label] {
    x   y   label
    9   9   {آزادی زندانیان}
    8   8   {لغو فیلترینگ}
    8   7   {لغو حجاب اجباری}
    7   8   {یارانه نقدی}
    6   7   {کاهش قیمت دارو}
};

% Medium-term
\addplot[scatter, only marks, scatter src=explicit symbolic,
         nodes near coords,
         point meta=explicit symbolic,
         every node near coord/.append style={font=\tiny, anchor=south west},
         mark=square*, draw=orange!70!black, fill=orange!50]
table[meta=label] {
    x   y   label
    5   8   {رفع تحریم}
    4   7   {اصلاح نظام بانکی}
    5   6   {ساخت مسکن}
};

% Hard
\addplot[scatter, only marks, scatter src=explicit symbolic,
         nodes near coords,
         point meta=explicit symbolic,
         every node near coord/.append style={font=\tiny, anchor=south},
         mark=triangle*, draw=red!70!black, fill=red!50]
table[meta=label] {
    x   y   label
    2   9   {احیای دریاچه ارومیه}
    3   8   {اصلاحات آب}
    2   7   {تنوع اقتصادی}
};

% مناطق
\fill[green!10, opacity=0.3] (5.5,5.5) rectangle (10.5,10.5);
\fill[orange!10, opacity=0.3] (3,5.5) rectangle (5.5,10.5);
\fill[red!10, opacity=0.3] (0,5.5) rectangle (3,10.5);

\node[font=\scriptsize\bfseries, text=green!60!black] at (8,10.2) {اولویت اول};
\node[font=\scriptsize\bfseries, text=orange!60!black] at (4.2,10.2) {اولویت دوم};
\node[font=\scriptsize\bfseries, text=red!60!black] at (1.5,10.2) {اولویت سوم};

\end{axis}
\end{tikzpicture}
\caption{ماتریس اولویت‌بندی برنامه‌های آبادانی: تأثیر در مقابل سهولت}
\label{fig:priority-matrix}
\end{figure}

\begin{enghelabbox}
\textbf{درس مصر:} دولت مُرسی در مصر (۲۰۱۲-۲۰۱۳) به جای تمرکز بر بهبود معیشت، درگیر دعواهای سیاسی و ایدئولوژیک شد. قطعی مکرر برق و کمبود سوخت، حتی هواداران را سرخورده کرد. نتیجه: کودتای نظامی با حمایت بخشی از مردم. درس: اقتصاد معیشتی مهم‌تر از ایدئولوژی است.
\end{enghelabbox}

%───────────────────────────────────────────────────────────────────────────────
\section{شاخص‌های پایش چشم‌انداز}
\label{sec:vision-indicators}
%───────────────────────────────────────────────────────────────────────────────

برای سنجش پیشرفت به سوی چشم‌انداز ۱۴۲۹، شاخص‌های کمّی زیر تعریف می‌شوند:

\subsection{داشبورد ملی پیشرفت}

\begin{table}[htbp]
\centering
\caption{شاخص‌های کلیدی پایش چشم‌انداز ایران ۱۴۲۹}
\label{tab:vision-kpis}
\begin{tabular}{>{\columncolor{blue!8}}r p{3.8cm} c c c c}
\toprule
\rowcolor{blue!25}
\textbf{کد} & \textbf{شاخص} & \textbf{واحد} & \textbf{وضعیت ۱۴۰۳} & \textbf{هدف ۱۴۱۴} & \textbf{هدف ۱۴۲۹} \\
\midrule
V01 & شاخص دموکراسی (EIU) & امتیاز ۱-۱۰ & ۱.۹۶ & ۵.۰ & ۷.۵ \\
\rowcolor{gray!10}
V02 & شاخص آزادی (Freedom House) & امتیاز ۱-۱۰۰ & ۱۴ & ۴۵ & ۷۵ \\
V03 & درآمد سرانه (PPP) & هزار دلار & ۱۵.۵ & ۲۲ & ۳۵ \\
\rowcolor{gray!10}
V04 & ضریب جینی & ۰-۱ & ۰.۴۲ & ۰.۳۵ & ۰.۳۰ \\
V05 & امید به زندگی & سال & ۷۴ & ۷۷ & ۸۱ \\
\rowcolor{gray!10}
V06 & نرخ بیکاری & درصد & ۱۲٪ & ۸٪ & ۵٪ \\
V07 & مصرف آب سرانه & مترمکعب/سال & ۱۴۰۰ & ۱۱۰۰ & ۸۵۰ \\
\rowcolor{gray!10}
V08 & سهم انرژی تجدیدپذیر & درصد & ۷٪ & ۲۵٪ & ۵۰٪ \\
V09 & شاخص برابری جنسیتی & ۰-۱ & ۰.۵۷ & ۰.۷۵ & ۰.۹۰ \\
\rowcolor{gray!10}
V10 & رضایت قومی & درصد راضی & — & ۶۰٪ & ۸۰٪ \\
\bottomrule
\end{tabular}
\end{table}

\subsection{نمودار راداری وضعیت کنونی و اهداف}

\begin{figure}[htbp]
\centering
\begin{tikzpicture}
\begin{polaraxis}[
    width=10cm,
    height=10cm,
    xtick={0,36,72,108,144,180,216,252,288,324},
    xticklabels={
        دموکراسی,
        آزادی,
        رفاه,
        برابری,
        سلامت,
        اشتغال,
        آب,
        انرژی پاک,
        برابری جنسیتی,
        رضایت قومی
    },
    xticklabel style={font=\tiny},
    ymin=0, ymax=100,
    ytick={20,40,60,80,100},
    yticklabels={۲۰,۴۰,۶۰,۸۰,۱۰۰},
    yticklabel style={font=\tiny},
    grid=both,
    major grid style={gray!40},
    minor grid style={gray!20},
    legend style={at={(1.3,1)}, anchor=north west, font=\small}
]

% وضعیت کنونی (۱۴۰۳)
\addplot[thick, color=red, fill=red!20, opacity=0.5, mark=*] coordinates {
    (0,20) (36,14) (72,44) (108,58) (144,74) (180,88) (216,30) (252,14) (288,57) (324,40) (360,20)
};

% هدف میان‌مدت (۱۴۱۴)
\addplot[thick, color=orange, fill=orange!20, opacity=0.5, mark=square*] coordinates {
    (0,50) (36,45) (72,63) (108,65) (144,77) (180,92) (216,50) (252,50) (288,75) (324,60) (360,50)
};

% هدف بلندمدت (۱۴۲۹)
\addplot[thick, color=green!70!black, fill=green!20, opacity=0.5, mark=triangle*] coordinates {
    (0,75) (36,75) (72,100) (108,70) (144,81) (180,95) (216,70) (252,100) (288,90) (324,80) (360,75)
};

\legend{وضعیت ۱۴۰۳, هدف ۱۴۱۴, هدف ۱۴۲۹}

\end{polaraxis}
\end{tikzpicture}
\caption{نمودار راداری: مقایسه وضعیت کنونی با اهداف میان‌مدت و بلندمدت}
\label{fig:radar-chart}
\end{figure}

%───────────────────────────────────────────────────────────────────────────────
\section{نقشه راه کلان ۲۵ ساله}
\label{sec:25-year-roadmap}
%───────────────────────────────────────────────────────────────────────────────

% تایم‌لاین ۲۵ ساله
\begin{figure}[htbp]
\centering
\begin{tikzpicture}[
    scale=0.75,
    transform shape,
    phase/.style={
        rectangle,
        rounded corners=4pt,
        minimum height=1.5cm,
        draw=#1!70!black,
        fill=#1!20,
        text=#1!30!black,
        font=\small\bfseries,
        align=center
    }
]

% خط زمان
\draw[thick, gray!60] (0,0) -- (20,0);
\foreach \x/\year in {0/۱۴۰۴, 3/۱۴۰۶, 7/۱۴۰۹, 12/۱۴۱۴, 16/۱۴۱۹, 20/۱۴۲۹} {
    \draw[thick, gray!60] (\x,-0.2) -- (\x,0.2);
    \node[below, font=\small] at (\x,-0.3) {\year};
}

% فازها
\node[phase=red, minimum width=2.8cm, text width=2.8cm] at (1.5,1.5) {
    \begin{tabular}{c}
    فاز ۱\\
    گذار\\
    (سال ۱-۲)
    \end{tabular}
};

\node[phase=orange, minimum width=3.8cm, text width=3.8cm] at (5,1.5) {
    \begin{tabular}{c}
    فاز ۲\\
    نهادسازی\\
    (سال ۳-۵)
    \end{tabular}
};

\node[phase=yellow, minimum width=4.8cm, text width=4.8cm] at (9.5,1.5) {
    \begin{tabular}{c}
    فاز ۳\\
    تحکیم\\
    (سال ۶-۱۰)
    \end{tabular}
};

\node[phase=green, minimum width=3.8cm, text width=3.8cm] at (14,1.5) {
    \begin{tabular}{c}
    فاز ۴\\
    بلوغ\\
    (سال ۱۱-۱۵)
    \end{tabular}
};

\node[phase=blue, minimum width=3.8cm, text width=3.8cm] at (18,1.5) {
    \begin{tabular}{c}
    فاز ۵\\
    تعالی\\
    (سال ۱۶-۲۵)
    \end{tabular}
};

% رویدادهای کلیدی
\node[below, font=\tiny, text width=2.5cm, align=center] at (1.5,-1) {
    انتخابات آزاد\\
    قانون اساسی موقت\\
    رفع تحریم
};

\node[below, font=\tiny, text width=2.8cm, align=center] at (5,-1) {
    مجلس مؤسسان\\
    قانون اساسی دائم\\
    فدرالیسم
};

\node[below, font=\tiny, text width=3cm, align=center] at (9.5,-1) {
    تثبیت نهادها\\
    رشد اقتصادی\\
    احیای محیط زیست
};

\node[below, font=\tiny, text width=2.5cm, align=center] at (14,-1) {
    دموکراسی پایدار\\
    جامعه مدنی قوی\\
    رفاه گسترده
};

\node[below, font=\tiny, text width=2.5cm, align=center] at (18,-1) {
    الگوی منطقه‌ای\\
    نوآوری و دانش\\
    توسعه پایدار
};

\end{tikzpicture}
\caption{نقشه راه کلان ۲۵ ساله تحول ایران}
\label{fig:25-year-roadmap}
\end{figure}

%───────────────────────────────────────────────────────────────────────────────
\section{میثاق ملی: توافق همه با همه}
\label{sec:national-covenant}
%───────────────────────────────────────────────────────────────────────────────

برای موفقیت گذار، توافق حداقلی میان همه نیروهای سیاسی و اجتماعی ضروری است. این «میثاق ملی» شامل تعهدات متقابل است:

\begin{olgoobox}
\textbf{پیمان مونکلوآ (اسپانیا ۱۹۷۷):}
پس از مرگ فرانکو، احزاب چپ و راست اسپانیا پیمانی امضا کردند که شامل:
\begin{itemize}[nosep]
    \item پذیرش بازی دموکراتیک از سوی همه
    \item عدم تعقیب جرایم دوره دیکتاتوری (عفو عمومی)
    \item اصلاحات اقتصادی با رضایت اتحادیه‌ها
    \item پذیرش سلطنت مشروطه از سوی چپ‌ها
\end{itemize}
این پیمان گذار صلح‌آمیز را ممکن ساخت.
\end{olgoobox}

\subsection{طرفین میثاق ملی ایران}

\begin{table}[htbp]
\centering
\caption{طرفین میثاق ملی و تعهدات متقابل}
\label{tab:national-covenant}
\begin{tabular}{>{\columncolor{purple!8}}r p{3cm} p{4.5cm} p{4.5cm}}
\toprule
\rowcolor{purple!25}
\textbf{ردیف} & \textbf{طرف} & \textbf{تعهد می‌دهد} & \textbf{تضمین می‌گیرد} \\
\midrule
۱ & اپوزیسیون دموکرات & عدم انتقام‌جویی، پذیرش انتخابات & مشارکت در قدرت، آزادی فعالیت \\
\rowcolor{gray!10}
۲ & نهادهای نظامی & عدم کودتا، بی‌طرفی سیاسی & حفظ جایگاه، محاکمه‌نشدن غیرفرماندهان \\
۳ & روحانیت & جدایی دین از حکومت & آزادی دینی، حفظ موقوفات \\
\rowcolor{gray!10}
۴ & احزاب قومی & حفظ تمامیت ارضی & خودمختاری فرهنگی و اداری \\
۵ & فعالان چپ & پذیرش اقتصاد مختلط & حقوق کارگری، تأمین اجتماعی \\
\rowcolor{gray!10}
۶ & فعالان راست & پذیرش عدالت توزیعی & آزادی اقتصادی، حقوق مالکیت \\
۷ & نهادهای مدنی & همکاری سازنده & استقلال، تأمین مالی \\
\bottomrule
\end{tabular}
\end{table}

%───────────────────────────────────────────────────────────────────────────────
\section{خطوط قرمز و خطوط سبز}
\label{sec:red-green-lines}
%───────────────────────────────────────────────────────────────────────────────

\subsection{خطوط قرمز غیرقابل عبور}

\begin{enghelabbox}
\textbf{این موارد به هیچ عنوان قابل مذاکره نیستند:}
\begin{enumerate}[nosep]
    \item بازگشت به استبداد (از هر نوع: دینی، نظامی، فردی)
    \item تجزیه کشور یا جدایی‌طلبی مسلحانه
    \item نسل‌کشی، پاک‌سازی قومی، یا جنایت علیه بشریت
    \item شکنجه و اعدام‌های گسترده
    \item تبعیض سیستماتیک بر اساس قومیت، جنسیت، یا مذهب
    \item ولایت یا رهبری مادام‌العمر غیرانتخابی
    \item سلب حاکمیت ملی به نفع بیگانه
\end{enumerate}
\end{enghelabbox}

\subsection{خطوط سبز: فضای مذاکره}

\begin{olgoobox}
\textbf{این موارد قابل مذاکره و توافق هستند:}
\begin{itemize}[nosep]
    \item نوع نظام: جمهوری یا مشروطه سلطنتی (با همه‌پرسی)
    \item درجه تمرکززدایی: فدرالیسم تا عدم تمرکز اداری
    \item نظام اقتصادی: از سوسیال‌دموکراسی تا لیبرالیسم تعدیل‌شده
    \item نحوه عدالت انتقالی: از محاکمه تا کمیسیون حقیقت
    \item سرعت اصلاحات: تدریجی یا شتابان
    \item سیاست خارجی: شرق‌گرا، غرب‌گرا، یا متوازن
\end{itemize}
\end{olgoobox}

%───────────────────────────────────────────────────────────────────────────────
\section{جمع‌بندی: از چشم‌انداز تا عمل}
\label{sec:chapter7-conclusion}
%───────────────────────────────────────────────────────────────────────────────

این فصل چشم‌انداز ایران دموکراتیک ۱۴۲۹ و اصول بنیادین آن را ترسیم کرد. اما چشم‌انداز بدون نقشه راه عملیاتی، رؤیایی بیش نیست.

% نمودار خلاصه فصل
\begin{figure}[htbp]
\centering
\begin{tikzpicture}[
    scale=0.8,
    transform shape,
    box/.style={
        rectangle,
        rounded corners=5pt,
        draw=#1!70!black,
        fill=#1!15,
        text=#1!30!black,
        minimum width=3cm, text width=3cm,
        minimum height=1.5cm,
        font=\small\bfseries,
        align=center
    }
]

% ستون‌های اصلی
\node[box=blue] (vision) at (0,0) {
    \begin{tabular}{c}
    چشم‌انداز\\
    ایران ۱۴۲۹
    \end{tabular}
};

\node[box=purple] (principles) at (4,0) {
    \begin{tabular}{c}
    ده اصل\\
    بنیادین
    \end{tabular}
};

\node[box=green] (constitution) at (8,0) {
    \begin{tabular}{c}
    قانون اساسی\\
    پیشنهادی
    \end{tabular}
};

\node[box=orange] (prosperity) at (12,0) {
    \begin{tabular}{c}
    آبادانی\\
    ملموس
    \end{tabular}
};

% فلش‌ها
\draw[->, thick, gray!60] (vision) -- (principles);
\draw[->, thick, gray!60] (principles) -- (constitution);
\draw[->, thick, gray!60] (constitution) -- (prosperity);

% اتصال به فصول بعدی
\node[box=red, minimum width=12cm, text width=12cm] (next) at (6,-3) {
    \begin{tabular}{c}
    فصول ۸-۱۱: نقشه راه تفصیلی پنج فاز گذار
    \end{tabular}
};

\draw[->, thick, red!60] (6,-0.8) -- (6,-2.2);

\end{tikzpicture}
\caption{جمع‌بندی فصل ۷ و اتصال به فصول بعدی}
\label{fig:chapter7-summary}
\end{figure}

\begin{kholasebox}
\textbf{نکات کلیدی فصل:}
\begin{enumerate}
    \item چشم‌انداز ایران ۱۴۲۹ بر شش رکن استوار است: حقوق برابر، حکومت پاسخگو، خودمدیریت اقوام، محیط زیست سالم، اقتصاد متنوع، و عضویت محترم جهانی
    \item ده اصل بنیادین غیرقابل تجدیدنظر، ستون‌های نظام جدید هستند
    \item قانون اساسی پیشنهادی بر تفکیک قوا، فدرالیسم همبسته، و منشور حقوق بنیادین بنا می‌شود
    \item مجلس اقوام به‌عنوان نوآوری کلیدی، صدای اقلیت‌ها را تضمین می‌کند
    \item استراتژی «آبادانی ملموس» اعتماد اولیه مردم را جلب می‌کند
    \item میثاق ملی توافق همه با همه را برای گذار صلح‌آمیز فراهم می‌آورد
\end{enumerate}
\end{kholasebox}

%───────────────────────────────────────────────────────────────────────────────
% منابع فصل
%───────────────────────────────────────────────────────────────────────────────

\vspace{1cm}
\begin{refsection}

\textbf{\large منابع فصل هفتم}

\vspace{0.5cm}

\begin{enumerate}[label={[\arabic*]}, nosep, leftmargin=*]
    \item Economist Intelligence Unit. (2024). \textit{Democracy Index 2023}. EIU.
    
    \item Freedom House. (2024). \textit{Freedom in the World 2024: Iran}. 
    
    \item World Bank. (2023). \textit{Iran Economic Monitor}.
    
    \item Transparency International. (2024). \textit{Corruption Perceptions Index 2023}.
    
    \item Linz, J. \& Stepan, A. (1996). \textit{Problems of Democratic Transition and Consolidation}. Johns Hopkins University Press.
    
    \item آبراهامیان، یرواند. (۱۳۹۸). \textit{ایران بین دو انقلاب}. ترجمه احمد گل‌محمدی. نشر نی.
    
    \item کاتوزیان، محمدعلی. (۱۳۹۵). \textit{اقتصاد سیاسی ایران}. نشر مرکز.
    
    \item Fukuyama, F. (2014). \textit{Political Order and Political Decay}. Farrar, Straus and Giroux.
    
    \item O'Donnell, G. \& Schmitter, P. (1986). \textit{Transitions from Authoritarian Rule}. Johns Hopkins University Press.
    
    \item UNDP. (2023). \textit{Human Development Report 2023: Iran}.
    
    \item میرسپاسی، علی. (۱۴۰۰). \textit{دموکراسی یا حقیقت}. نشر ثالث.
    
    \item Przeworski, A. (1991). \textit{Democracy and the Market}. Cambridge University Press.
    
    \item مؤسسه آمار ایران. (۱۴۰۲). \textit{سالنامه آماری کشور}.
    
    \item World Economic Forum. (2023). \textit{Global Gender Gap Report 2023}.
    
    \item Huntington, S. (1991). \textit{The Third Wave: Democratization in the Late Twentieth Century}. University of Oklahoma Press.
\end{enumerate}

\end{refsection}
	%═══════════════════════════════════════════════════════════════════════════════
% فصل ۸: فاز ۱ — گذار (سال ۱-۲)
% فایل: chapters/ch08-phase1.tex
%═══════════════════════════════════════════════════════════════════════════════

\chapter{فاز ۱: گذار اولیه (سال ۱-۲)}
\label{chap:phase1}

\begin{kholasebox}
\textbf{خلاصه فصل:}
فاز اول حیاتی‌ترین دوره گذار است. در این ۲۴ ماه، باید هم‌زمان چند هدف متعارض مدیریت شود: حفظ نظم عمومی، جلوگیری از خلأ قدرت، پاسخ به انتظارات فزاینده مردم، آغاز اصلاحات ساختاری، و مذاکره برای رفع تحریم‌ها. این فصل نقشه راه تفصیلی ۲۴ ماه اول شامل تشکیل شورای انتقالی، دولت موقت، انتخابات مجلس مؤسسان، و قانون اساسی جدید را ارائه می‌دهد. تجربه کشورهایی که در این فاز شکست خوردند (لیبی، مصر) و موفق شدند (تونس، لهستان) راهنمای ماست.
\end{kholasebox}

%───────────────────────────────────────────────────────────────────────────────
\section{لحظه صفر: مدیریت فروپاشی}
\label{sec:moment-zero}
%───────────────────────────────────────────────────────────────────────────────

\begin{enghelabbox}
\textbf{هشدار حیاتی:} خطرناک‌ترین لحظه هر گذار، ساعات و روزهای اولیه پس از سقوط رژیم است. در این خلأ قدرت، احتمال هرج‌ومرج، غارت، انتقام‌جویی، و ظهور نیروهای افراطی بسیار بالاست. آمادگی قبلی برای این لحظه حیاتی است.
\end{enghelabbox}

\subsection{سناریوهای محتمل فروپاشی}

\begin{table}[htbp]
\centering
\caption{سناریوهای محتمل پایان نظام کنونی و پیامدهای هر یک}
\label{tab:collapse-scenarios}
\begin{tabular}{>{\columncolor{red!8}}r p{2.5cm} c p{4cm} p{3.5cm}}
\toprule
\rowcolor{red!25}
\textbf{ردیف} & \textbf{سناریو} & \textbf{احتمال} & \textbf{ویژگی‌ها} & \textbf{چالش اصلی} \\
\midrule
۱ & انقلاب مردمی & ۳۵٪ & سقوط سریع، خلأ قدرت & کنترل خشونت و انتقام \\
\rowcolor{gray!10}
۲ & کودتای اصلاح‌طلب & ۲۵٪ & حفظ نسبی نظم & مشروعیت‌سازی \\
۳ & فروپاشی تدریجی & ۲۰٪ & بی‌ثباتی طولانی & مدیریت آشوب \\
\rowcolor{gray!10}
۴ & مذاکره و گذار مسالمت & ۱۵٪ & کم‌هزینه‌ترین & سازش‌های دشوار \\
۵ & جنگ داخلی & ۵٪ & فاجعه‌بار & همه چیز \\
\bottomrule
\end{tabular}
\end{table}

\subsection{پروتکل ساعات اولیه}

% تایم‌لاین ۷۲ ساعت اول
\begin{figure}[htbp]
\centering
\begin{tikzpicture}[
    scale=0.8,
    transform shape,
    hour/.style={
        rectangle,
        rounded corners=3pt,
        minimum width=2.2cm,
        minimum height=1.2cm,
        draw=#1!70!black,
        fill=#1!20,
        text=#1!30!black,
        font=\scriptsize\bfseries,
        align=center
    },
    action/.style={
        rectangle,
        rounded corners=2pt,
        draw=gray!50,
        fill=gray!5,
        font=\tiny,
        align=right,
        text width=2.8cm
    }
]

% خط زمان
\draw[thick, gray!60, ->] (0,0) -- (18,0);
\node[below, font=\small\bfseries] at (9,-0.5) {ساعات و روزهای اولیه};

% نقاط زمانی
\foreach \x/\label in {0/ساعت ۰, 3/ساعت ۶, 6/ساعت ۲۴, 9/ساعت ۴۸, 12/ساعت ۷۲, 15/روز ۷, 17/روز ۱۴} {
    \draw[thick, gray!60] (\x,0.1) -- (\x,-0.1);
    \node[below, font=\tiny] at (\x,-0.2) {\label};
}

% اقدامات
\node[hour=red] at (1.5,2) {
    \begin{tabular}{c}
    اعلام\\
    شورای انتقالی
    \end{tabular}
};
\node[action] at (1.5,4) {
    پیام تلویزیونی\\
    درخواست آرامش\\
    معرفی چهره‌ها
};

\node[hour=orange] at (4.5,2) {
    \begin{tabular}{c}
    کنترل\\
    امنیتی
    \end{tabular}
};
\node[action] at (4.5,4) {
    هماهنگی با ارتش\\
    حفاظت از زیرساخت\\
    جلوگیری از غارت
};

\node[hour=yellow] at (7.5,2) {
    \begin{tabular}{c}
    پیام‌های\\
    بین‌المللی
    \end{tabular}
};
\node[action] at (7.5,4) {
    تماس با UN\\
    پیام به همسایگان\\
    درخواست شناسایی
};

\node[hour=green] at (10.5,2) {
    \begin{tabular}{c}
    دولت موقت\\
    فوری
    \end{tabular}
};
\node[action] at (10.5,4) {
    انتصاب وزرای کلیدی\\
    تداوم خدمات\\
    پرداخت حقوق
};

\node[hour=blue] at (13.5,2) {
    \begin{tabular}{c}
    نقشه راه\\
    گذار
    \end{tabular}
};
\node[action] at (13.5,4) {
    اعلام جدول زمانی\\
    تعهد به انتخابات\\
    منشور حقوق فوری
};

\node[hour=purple] at (16.5,2) {
    \begin{tabular}{c}
    گشایش\\
    سیاسی
    \end{tabular}
};
\node[action] at (16.5,4) {
    آزادی زندانیان\\
    لغو سانسور\\
    بازگشت تبعیدیان
};

% فلش‌ها
\foreach \x in {1.5,4.5,7.5,10.5,13.5,16.5} {
    \draw[->, gray!40] (\x,0.2) -- (\x,1.3);
}

\end{tikzpicture}
\caption{پروتکل ۷۲ ساعت اول پس از تغییر نظام}
\label{fig:72-hours}
\end{figure}

\begin{naghlbox}
«در انقلاب‌ها، ساعات اول همه چیز را تعیین می‌کند. اگر نیروهای دموکراتیک آماده نباشند، نیروهای سازمان‌یافته‌تر — چه نظامی، چه افراطی — خلأ را پر می‌کنند.»
\sourceline{ژوئل میگدال، جامعه‌شناس سیاسی}
\end{naghlbox}

%───────────────────────────────────────────────────────────────────────────────
\section{شورای انتقالی: ساختار و ترکیب}
\label{sec:transitional-council}
%───────────────────────────────────────────────────────────────────────────────

شورای انتقالی به‌عنوان عالی‌ترین مرجع تصمیم‌گیری در دوره گذار عمل می‌کند. این شورا باید نماینده طیف وسیعی از نیروها باشد.

\subsection{ترکیب پیشنهادی شورای انتقالی}

\begin{table}[htbp]
\centering
\caption{ترکیب پیشنهادی شورای انتقالی (۲۱ نفر)}
\label{tab:council-composition}
\begin{tabular}{>{\columncolor{blue!8}}r p{3.5cm} c p{5cm}}
\toprule
\rowcolor{blue!25}
\textbf{ردیف} & \textbf{گروه} & \textbf{تعداد} & \textbf{معیار انتخاب} \\
\midrule
۱ & رهبران اعتراضات داخل & ۴ & شناخته‌شده در جنبش‌های اخیر \\
\rowcolor{gray!10}
۲ & اپوزیسیون خارج از کشور & ۳ & نمایندگان جریان‌های اصلی \\
۳ & نمایندگان اقوام & ۴ & آذری، کرد، عرب، بلوچ \\
\rowcolor{gray!10}
۴ & نمایندگان زنان & ۲ & فعالان حقوق زنان \\
۵ & نمایندگان جوانان & ۲ & زیر ۳۵ سال \\
\rowcolor{gray!10}
۶ & شخصیت‌های ملی مستقل & ۳ & مورد اعتماد عمومی \\
۷ & نماینده نهاد نظامی & ۱ & ژنرال اصلاح‌طلب \\
\rowcolor{gray!10}
۸ & نماینده روحانیت سکولار & ۱ & مخالف حکومت دینی \\
۹ & نماینده اقلیت‌های دینی & ۱ & چرخشی بین ادیان \\
\midrule
\rowcolor{blue!15}
& \textbf{جمع} & \textbf{۲۱} & \\
\bottomrule
\end{tabular}
\end{table}

\subsection{ساختار تصمیم‌گیری}

% نمودار ساختار شورا
\begin{figure}[htbp]
\centering
\begin{tikzpicture}[
    scale=0.85,
    transform shape,
    organ/.style={
        rectangle,
        rounded corners=5pt,
        draw=#1!70!black,
        fill=#1!15,
        text=#1!30!black,
        minimum width=3.5cm,
        minimum height=1.2cm,
        font=\small\bfseries,
        align=center
    },
    committee/.style={
        rectangle,
        rounded corners=3pt,
        draw=gray!60,
        fill=gray!10,
        minimum width=2.5cm,
        minimum height=0.9cm,
        font=\scriptsize,
        align=center
    }
]

% شورای انتقالی
\node[organ=purple] (council) at (0,5) {
    \begin{tabular}{c}
    شورای انتقالی\\
    (۲۱ نفر)
    \end{tabular}
};

% هیئت رئیسه
\node[organ=blue] (presidium) at (0,3) {
    \begin{tabular}{c}
    هیئت رئیسه\\
    (۵ نفر چرخشی)
    \end{tabular}
};

% سه شاخه
\node[organ=green] (exec) at (-5,0.5) {
    \begin{tabular}{c}
    دولت موقت\\
    (نخست‌وزیر + کابینه)
    \end{tabular}
};

\node[organ=orange] (legis) at (0,0.5) {
    \begin{tabular}{c}
    کمیسیون قانون‌گذاری\\
    (تصویب فرمان‌های فوری)
    \end{tabular}
};

\node[organ=red] (judicial) at (5,0.5) {
    \begin{tabular}{c}
    کمیسیون حقوقی\\
    (نظارت بر قانونیت)
    \end{tabular}
};

% کمیته‌های تخصصی
\node[committee] (c1) at (-6,-2) {کمیته امنیت};
\node[committee] (c2) at (-4,-2) {کمیته اقتصاد};
\node[committee] (c3) at (-2,-2) {کمیته انتخابات};
\node[committee] (c4) at (0,-2) {کمیته حقوق بشر};
\node[committee] (c5) at (2,-2) {کمیته اقوام};
\node[committee] (c6) at (4,-2) {کمیته سیاست خارجی};
\node[committee] (c7) at (6,-2) {کمیته آشتی ملی};

% اتصالات
\draw[->, thick, purple!60] (council) -- (presidium);
\draw[->, thick, blue!60] (presidium) -- (exec);
\draw[->, thick, blue!60] (presidium) -- (legis);
\draw[->, thick, blue!60] (presidium) -- (judicial);

\draw[->, gray!40] (exec) -- (c1);
\draw[->, gray!40] (exec) -- (c2);
\draw[->, gray!40] (legis) -- (c3);
\draw[->, gray!40] (legis) -- (c4);
\draw[->, gray!40] (judicial) -- (c5);
\draw[->, gray!40] (judicial) -- (c6);
\draw[->, gray!40] (presidium) -- (c7);

\end{tikzpicture}
\caption{ساختار شورای انتقالی و نهادهای زیرمجموعه}
\label{fig:council-structure}
\end{figure}

\begin{olgoobox}
\textbf{الگوی تونس (۲۰۱۱):} «هیئت عالی تحقق اهداف انقلاب» با ۱۵۵ عضو از احزاب، اتحادیه‌ها، و جامعه مدنی، موفق شد اجماع ملی را حفظ کند. کلید موفقیت: مشارکت همه نیروها، حتی اسلام‌گرایان معتدل. نتیجه: تنها دموکراسی موفق بهار عربی.
\end{olgoobox}

%───────────────────────────────────────────────────────────────────────────────
\section{دولت موقت: ساختار و وظایف}
\label{sec:provisional-government}
%───────────────────────────────────────────────────────────────────────────────

دولت موقت قوه مجریه دوره گذار است. این دولت باید تکنوکرات، کارآمد، و غیرجناحی باشد.

\subsection{معیارهای انتخاب نخست‌وزیر موقت}

\begin{itemize}
    \item \textbf{مورد اعتماد عمومی:} شخصیتی که در دوره استبداد مقاومت کرده باشد
    \item \textbf{غیرحزبی:} متعهد به عدم نامزدی در انتخابات آینده
    \item \textbf{تجربه مدیریتی:} توانایی اداره بحران
    \item \textbf{مقبولیت بین‌المللی:} شناخته‌شده و مورد احترام جهانی
    \item \textbf{فراقومی:} پذیرفته‌شده توسط همه اقوام
\end{itemize}

\subsection{ساختار کابینه موقت}

\begin{table}[htbp]
\centering
\caption{کابینه موقت پیشنهادی (۱۸ وزارتخانه)}
\label{tab:provisional-cabinet}
\begin{tabular}{>{\columncolor{green!8}}r p{3cm} p{4cm} p{4cm}}
\toprule
\rowcolor{green!25}
\textbf{ردیف} & \textbf{وزارتخانه} & \textbf{اولویت فوری} & \textbf{پروفایل وزیر} \\
\midrule
۱ & کشور & امنیت و انتخابات & ژنرال بازنشسته دموکرات \\
\rowcolor{gray!10}
۲ & اقتصاد و دارایی & تثبیت ارز و تورم & اقتصاددان بین‌المللی \\
۳ & امور خارجه & رفع تحریم & دیپلمات باتجربه \\
\rowcolor{gray!10}
۴ & دادگستری & اصلاح قضا، آزادی زندانیان & حقوقدان حقوق بشری \\
۵ & نفت و انرژی & مذاکره صادرات & مدیر صنعت نفت \\
\rowcolor{gray!10}
۶ & کار و رفاه & یارانه و اشتغال & فعال کارگری \\
۷ & بهداشت & واکسیناسیون، دارو & پزشک مدیر \\
\rowcolor{gray!10}
۸ & آموزش و پرورش & اصلاح محتوا & استاد دانشگاه \\
۹ & علوم و فناوری & آزادی اینترنت & متخصص IT \\
\rowcolor{gray!10}
۱۰ & جهاد کشاورزی & بحران آب و غذا & مهندس کشاورزی \\
۱۱ & راه و شهرسازی & مسکن اضطراری & مهندس عمران \\
\rowcolor{gray!10}
۱۲ & ارتباطات & رسانه آزاد & روزنامه‌نگار \\
۱۳ & تعاون و اقوام & حقوق قومی & نماینده اقلیت \\
\rowcolor{gray!10}
۱۴ & زنان و خانواده & لغو قوانین تبعیضی & فعال حقوق زنان \\
۱۵ & میراث فرهنگی & حفاظت از آثار & باستان‌شناس \\
\rowcolor{gray!10}
۱۶ & ورزش و جوانان & فضا برای جوانان & قهرمان ورزشی \\
۱۷ & محیط زیست & بحران آب & زیست‌محیطی \\
\rowcolor{gray!10}
۱۸ & دفاع & کنترل مدنی بر ارتش & ژنرال اصلاح‌طلب \\
\bottomrule
\end{tabular}
\end{table}

%───────────────────────────────────────────────────────────────────────────────
\section{تقویم گذار: ۲۴ ماه اول}
\label{sec:transition-calendar}
%───────────────────────────────────────────────────────────────────────────────

% تایم‌لاین ۲۴ ماه
\begin{figure}[htbp]
\centering
\begin{tikzpicture}[
    scale=0.65,
    transform shape,
    event/.style={
        rectangle,
        rounded corners=3pt,
        minimum width=1.8cm,
        minimum height=0.8cm,
        draw=#1!70!black,
        fill=#1!20,
        text=#1!30!black,
        font=\tiny\bfseries,
        align=center
    }
]

% سال اول
\draw[thick, gray!60] (0,0) -- (24,0);
\draw[thick, gray!60] (0,-4) -- (24,-4);

\node[left, font=\small\bfseries] at (-0.5,0) {سال اول};
\node[left, font=\small\bfseries] at (-0.5,-4) {سال دوم};

% ماه‌ها - سال اول
\foreach \x in {0,2,4,6,8,10,12,14,16,18,20,22,24} {
    \draw[gray!40] (\x,0.1) -- (\x,-0.1);
}
\foreach \x/\m in {1/۱,3/۲,5/۳,7/۴,9/۵,11/۶,13/۷,15/۸,17/۹,19/۱۰,21/۱۱,23/۱۲} {
    \node[above, font=\tiny] at (\x,0.2) {ماه \m};
}

% ماه‌ها - سال دوم
\foreach \x in {0,2,4,6,8,10,12,14,16,18,20,22,24} {
    \draw[gray!40] (\x,-3.9) -- (\x,-4.1);
}
\foreach \x/\m in {1/۱۳,3/۱۴,5/۱۵,7/۱۶,9/۱۷,11/۱۸,13/۱۹,15/۲۰,17/۲۱,19/۲۲,21/۲۳,23/۲۴} {
    \node[above, font=\tiny] at (\x,-3.8) {ماه \m};
}

% رویدادهای سال اول
\node[event=red] at (1,1.5) {شورای انتقالی};
\node[event=red] at (3,1.5) {دولت موقت};
\node[event=orange] at (6,1.5) {قانون انتخابات};
\node[event=orange] at (9,1.5) {ثبت احزاب};
\node[event=yellow] at (13,1.5) {مذاکرات وین ۲};
\node[event=green] at (17,1.5) {انتخابات مؤسسان};
\node[event=blue] at (21,1.5) {آغاز کار مجلس};

% رویدادهای سال دوم
\node[event=purple] at (3,-2.5) {پیش‌نویس قانون اساسی};
\node[event=purple] at (9,-2.5) {بحث عمومی};
\node[event=teal] at (15,-2.5) {همه‌پرسی قانون اساسی};
\node[event=blue] at (21,-2.5) {انتخابات عمومی};

% فلش‌ها
\foreach \x/\y in {1/1.5, 3/1.5, 6/1.5, 9/1.5, 13/1.5, 17/1.5, 21/1.5} {
    \draw[->, gray!40] (\x,0.3) -- (\x,\y-0.4);
}
\foreach \x/\y in {3/-2.5, 9/-2.5, 15/-2.5, 21/-2.5} {
    \draw[->, gray!40] (\x,-3.7) -- (\x,\y+0.4);
}

% خط اتصال دو سال
\draw[->, thick, purple!60] (24,0) -- (24,-2) -- (0,-2) -- (0,-4);

\end{tikzpicture}
\caption{تقویم کلان گذار: رویدادهای کلیدی ۲۴ ماه اول}
\label{fig:24-month-calendar}
\end{figure}

\subsection{فاز ۱-الف: تثبیت اولیه (ماه ۱-۶)}

\begin{table}[htbp]
\centering
\caption{اقدامات کلیدی ماه‌های ۱ تا ۶}
\label{tab:months-1-6}
\begin{tabular}{>{\columncolor{red!8}}r p{2cm} p{6cm} p{3cm}}
\toprule
\rowcolor{red!25}
\textbf{ماه} & \textbf{محور} & \textbf{اقدامات} & \textbf{شاخص موفقیت} \\
\midrule
۱ & امنیت & کنترل نظم، جلوگیری از انتقام، حفظ زیرساخت‌ها & صفر کشته غیرنظامی \\
\rowcolor{gray!10}
۲ & آزادی‌ها & آزادی زندانیان، لغو سانسور، اجازه تجمع & همه زندانیان سیاسی آزاد \\
۳ & اقتصاد اضطراری & پرداخت حقوق، کنترل قیمت‌ها، یارانه نقدی & تورم زیر ۵٪ ماهانه \\
\rowcolor{gray!10}
۴ & دیپلماسی & پیام به جهان، آغاز مذاکرات، دعوت از UN & شناسایی ۵۰ کشور \\
۵ & قانون‌گذاری اولیه & قانون احزاب، قانون مطبوعات، قانون تجمعات & تصویب ۳ قانون کلیدی \\
\rowcolor{gray!10}
۶ & آماده‌سازی انتخابات & سرشماری رأی‌دهندگان، آموزش ناظران & ثبت‌نام ۴۰ میلیون نفر \\
\bottomrule
\end{tabular}
\end{table}

\subsection{فاز ۱-ب: نهادسازی اولیه (ماه ۷-۱۲)}

\begin{table}[htbp]
\centering
\caption{اقدامات کلیدی ماه‌های ۷ تا ۱۲}
\label{tab:months-7-12}
\begin{tabular}{>{\columncolor{orange!8}}r p{2cm} p{6cm} p{3cm}}
\toprule
\rowcolor{orange!25}
\textbf{ماه} & \textbf{محور} & \textbf{اقدامات} & \textbf{شاخص موفقیت} \\
\midrule
۷ & احزاب & ثبت احزاب جدید، کمک مالی به احزاب دموکراتیک & ۲۰+ حزب ثبت‌شده \\
\rowcolor{gray!10}
۸ & رسانه & راه‌اندازی رسانه عمومی مستقل، مجوز رسانه‌های خصوصی & ۱۰۰+ رسانه جدید \\
۹ & تحریم‌ها & امضای توافق موقت، آزادسازی دارایی‌ها & رفع ۵۰٪ تحریم‌ها \\
\rowcolor{gray!10}
۱۰ & کمپین انتخاباتی & مناظره‌ها، تبلیغات، آموزش شهروندی & مشارکت آگاهانه \\
۱۱ & \textbf{انتخابات مؤسسان} & برگزاری انتخابات آزاد، نظارت بین‌المللی & مشارکت ۶۵٪+ \\
\rowcolor{gray!10}
۱۲ & آغاز مجلس & افتتاح مجلس مؤسسان، انتخاب هیئت رئیسه & تشکیل کمیسیون‌ها \\
\bottomrule
\end{tabular}
\end{table}

\subsection{فاز ۱-ج: قانون اساسی (ماه ۱۳-۲۴)}

\begin{table}[htbp]
\centering
\caption{اقدامات کلیدی ماه‌های ۱۳ تا ۲۴}
\label{tab:months-13-24}
\begin{tabular}{>{\columncolor{green!8}}r p{2cm} p{6cm} p{3cm}}
\toprule
\rowcolor{green!25}
\textbf{ماه} & \textbf{محور} & \textbf{اقدامات} & \textbf{شاخص موفقیت} \\
\midrule
۱۳-۱۵ & تدوین & کمیسیون‌های تخصصی، مشاوره بین‌المللی & پیش‌نویس اولیه \\
\rowcolor{gray!10}
۱۶-۱۸ & بحث عمومی & انتشار متن، جلسات شهری، نظرسنجی & ۱ میلیون نظر مردمی \\
۱۹-۲۰ & اصلاح نهایی & بازنگری بر اساس نظرات، رأی‌گیری مجلس & تصویب ۲/۳ مجلس \\
\rowcolor{gray!10}
۲۱ & \textbf{همه‌پرسی} & رأی مردم به قانون اساسی جدید & تأیید ۶۰٪+ \\
۲۲-۲۳ & آماده‌سازی انتخابات & ثبت‌نام نامزدها، کمپین‌ها & ثبت‌نام ۵۰ میلیون \\
\rowcolor{gray!10}
۲۴ & \textbf{انتخابات عمومی} & انتخابات ریاست‌جمهوری و مجلس & مشارکت ۷۰٪+ \\
\bottomrule
\end{tabular}
\end{table}

%───────────────────────────────────────────────────────────────────────────────
\section{مدیریت نیروهای امنیتی}
\label{sec:security-forces}
%───────────────────────────────────────────────────────────────────────────────

یکی از حساس‌ترین چالش‌های گذار، مدیریت نیروهای نظامی و انتظامی است.

\begin{enghelabbox}
\textbf{درس عراق:} انحلال کامل ارتش عراق توسط پل برمر (۲۰۰۳) فاجعه‌ای بود که صدها هزار نفر مسلح و بیکار را به دشمن تبدیل کرد و زمینه‌ساز داعش شد. درس: اصلاح بهتر از انحلال است.
\end{enghelabbox}

\subsection{استراتژی مواجهه با نهادهای امنیتی}

\begin{table}[htbp]
\centering
\caption{استراتژی تفکیکی مواجهه با نهادهای امنیتی}
\label{tab:security-strategy}
\begin{tabular}{>{\columncolor{gray!8}}r p{2.5cm} p{3.5cm} p{3.5cm} p{2.5cm}}
\toprule
\rowcolor{gray!30}
\textbf{نهاد} & \textbf{وضعیت کنونی} & \textbf{استراتژی پیشنهادی} & \textbf{سرنوشت پرسنل} & \textbf{زمان‌بندی} \\
\midrule
ارتش & ۳۵۰,۰۰۰ نفر & حفظ با اصلاحات & بازنشستگی فرماندهان & ۱-۵ سال \\
\rowcolor{gray!10}
سپاه پاسداران & ۱۲۵,۰۰۰ نفر & انحلال تدریجی & ادغام در ارتش/بازنشستگی & ۲-۵ سال \\
بسیج & ۱ میلیون؟ & انحلال کامل & غیرنظامی‌سازی & فوری \\
\rowcolor{gray!10}
نیروی انتظامی & ۱۲۰,۰۰۰ نفر & اصلاح ساختاری & تطهیر ناقضان حقوق بشر & ۱-۳ سال \\
اطلاعات سپاه & ۱۵,۰۰۰؟ نفر & انحلال & بازداشت متهمان اصلی & فوری \\
\rowcolor{gray!10}
گشت ارشاد & ۷,۰۰۰ نفر & انحلال کامل & بازآموزی یا اخراج & فوری \\
\bottomrule
\end{tabular}
\end{table}

\subsection{فرآیند تطهیر (Vetting)}

% نمودار فرآیند تطهیر
\begin{figure}[htbp]
\centering
\begin{tikzpicture}[
    scale=0.85,
    transform shape,
    step/.style={
        rectangle,
        rounded corners=4pt,
        minimum width=2.5cm,
        minimum height=1.5cm,
        draw=#1!70!black,
        fill=#1!15,
        text=#1!30!black,
        font=\scriptsize\bfseries,
        align=center
    },
    arrow/.style={
        ->,
        >=stealth,
        thick,
        draw=gray!60
    }
]

% مراحل
\node[step=blue] (register) at (0,0) {
    \begin{tabular}{c}
    ثبت‌نام همه\\
    پرسنل امنیتی
    \end{tabular}
};

\node[step=orange] (review) at (4,0) {
    \begin{tabular}{c}
    بررسی پرونده\\
    توسط کمیسیون
    \end{tabular}
};

\node[step=green] (clear) at (8,2) {
    \begin{tabular}{c}
    پاک: ادامه\\
    خدمت یا بازنشستگی
    \end{tabular}
};

\node[step=yellow] (retrain) at (8,0) {
    \begin{tabular}{c}
    نیازمند بازآموزی:\\
    دوره حقوق بشر
    \end{tabular}
};

\node[step=red] (prosecute) at (8,-2) {
    \begin{tabular}{c}
    متهم: ارجاع\\
    به دادگاه
    \end{tabular}
};

% فلش‌ها
\draw[arrow] (register) -- (review);
\draw[arrow] (review) -- (clear) node[midway, above, font=\tiny] {۷۰٪};
\draw[arrow] (review) -- (retrain) node[midway, above, font=\tiny] {۲۵٪};
\draw[arrow] (review) -- (prosecute) node[midway, below, font=\tiny] {۵٪};

% معیارها
\node[rectangle, draw=gray!40, fill=gray!5, font=\tiny, align=right, 
      text width=4cm] at (4,-3) {
    \textbf{معیارهای تطهیر:}\\
    • نقش در سرکوب خونین\\
    • شکنجه مستند\\
    • فرماندهی عملیات ضد مردمی\\
    • فساد مالی کلان\\
    • همکاری با محاکمات فرمایشی
};

\end{tikzpicture}
\caption{فرآیند تطهیر نیروهای امنیتی}
\label{fig:vetting-process}
\end{figure}

\begin{olgoobox}
\textbf{الگوی آلمان شرقی:} پس از وحدت آلمان، کمیسیون گاوک (Gauck Commission) پرونده‌های شتازی (پلیس مخفی) را بررسی کرد. از ۹۱,۰۰۰ کارمند شتازی، تنها ۵۰۰ نفر محاکمه شدند. بقیه یا بازنشسته شدند یا در مشاغل غیرحساس به کار گرفته شدند. این رویکرد متعادل، ثبات را حفظ کرد.
\end{olgoobox}

%───────────────────────────────────────────────────────────────────────────────
\section{مدیریت بحران اقتصادی فوری}
\label{sec:economic-emergency}
%───────────────────────────────────────────────────────────────────────────────

\begin{enghelabbox}
\textbf{وضعیت اقتصاد ایران در نقطه صفر (تخمین):}
\begin{itemize}[nosep]
    \item نرخ تورم: ۴۵-۵۰٪ سالانه
    \item نرخ بیکاری: ۱۵-۲۰٪ (جوانان ۳۰٪+)
    \item نرخ ارز: ۵۰۰,۰۰۰+ ریال به دلار
    \item ذخایر ارزی قابل دسترس: ۱۰-۱۵ میلیارد دلار
    \item بدهی دولت: ۴۰-۵۰٪ GDP
    \item دارایی‌های مسدودشده: ۱۰۰+ میلیارد دلار
\end{itemize}
\end{enghelabbox}

\subsection{بسته اقدامات اضطراری اقتصادی}

\begin{table}[htbp]
\centering
\caption{بسته اقدامات اقتصادی ماه‌های اول}
\label{tab:economic-emergency}
\begin{tabular}{>{\columncolor{yellow!8}}r p{3cm} p{4.5cm} p{3.5cm}}
\toprule
\rowcolor{yellow!25}
\textbf{اولویت} & \textbf{اقدام} & \textbf{شرح} & \textbf{منبع تأمین مالی} \\
\midrule
۱ & پرداخت حقوق کارکنان & ادامه حقوق همه کارمندان دولت & بودجه جاری \\
\rowcolor{gray!10}
۲ & یارانه نقدی مضاعف & دوبرابر کردن یارانه نقدی & صندوق توسعه ملی \\
۳ & کنترل قیمت کالاهای اساسی & سقف قیمت نان، برنج، روغن، دارو & یارانه مستقیم \\
\rowcolor{gray!10}
۴ & تثبیت موقت نرخ ارز & مداخله بانک مرکزی & ذخایر ارزی \\
۵ & تضمین سپرده‌های بانکی & جلوگیری از هجوم بانکی & تضمین دولتی \\
\rowcolor{gray!10}
۶ & بسته کمکی بیکاران & ماهی ۵ میلیون تومان & بودجه اضطراری \\
۷ & تعلیق مالیات کسب‌وکارهای کوچک & ۶ ماه معافیت & کسری بودجه \\
\rowcolor{gray!10}
۸ & آزادسازی دارایی‌های مسدود & مذاکره فوری با آمریکا/اروپا & دارایی‌های خارجی \\
\bottomrule
\end{tabular}
\end{table}

\subsection{نمودار جریان منابع مالی دوره گذار}

\begin{figure}[htbp]
\centering
\begin{tikzpicture}[
    scale=0.85,
    transform shape,
    source/.style={
        rectangle,
        rounded corners=4pt,
        minimum width=2.8cm,
        minimum height=1cm,
        draw=green!70!black,
        fill=green!15,
        text=green!30!black,
        font=\scriptsize\bfseries,
        align=center
    },
    use/.style={
        rectangle,
        rounded corners=4pt,
        minimum width=2.8cm,
        minimum height=1cm,
        draw=red!70!black,
        fill=red!15,
        text=red!30!black,
        font=\scriptsize\bfseries,
        align=center
    },
    flow/.style={
        ->,
        >=stealth,
        thick,
        draw=blue!60
    }
]

% منابع
\node[source] (oil) at (0,3) {درآمد نفت (محدود)};
\node[source] (assets) at (0,1.5) {دارایی‌های آزادشده};
\node[source] (ndf) at (0,0) {صندوق توسعه ملی};
\node[source] (tax) at (0,-1.5) {مالیات (کاهش‌یافته)};
\node[source] (aid) at (0,-3) {کمک‌های بین‌المللی};

% خزانه
\node[rectangle, rounded corners=6pt, draw=purple!70!black, fill=purple!15,
      minimum width=2.5cm, minimum height=2.5cm, font=\small\bfseries,
      text=purple!30!black, align=center] (treasury) at (5,0) {
    \begin{tabular}{c}
    خزانه\\
    دولت موقت
    \end{tabular}
};

% مصارف
\node[use] (salary) at (10,3) {حقوق کارکنان};
\node[use] (subsidy) at (10,1.5) {یارانه نقدی};
\node[use] (health) at (10,0) {بهداشت و دارو};
\node[use] (security) at (10,-1.5) {امنیت و دفاع};
\node[use] (infra) at (10,-3) {زیرساخت اضطراری};

% جریان‌ها - ورودی
\draw[flow] (oil) -- (treasury) node[midway, above, font=\tiny] {۳۰٪};
\draw[flow] (assets) -- (treasury) node[midway, above, font=\tiny] {۲۵٪};
\draw[flow] (ndf) -- (treasury) node[midway, above, font=\tiny] {۲۰٪};
\draw[flow] (tax) -- (treasury) node[midway, below, font=\tiny] {۱۵٪};
\draw[flow] (aid) -- (treasury) node[midway, below, font=\tiny] {۱۰٪};

% جریان‌ها - خروجی
\draw[flow] (treasury) -- (salary) node[midway, above, font=\tiny] {۳۵٪};
\draw[flow] (treasury) -- (subsidy) node[midway, above, font=\tiny] {۲۵٪};
\draw[flow] (treasury) -- (health) node[midway, above, font=\tiny] {۱۵٪};
\draw[flow] (treasury) -- (security) node[midway, below, font=\tiny] {۱۵٪};
\draw[flow] (treasury) -- (infra) node[midway, below, font=\tiny] {۱۰٪};

% راهنما
\node[rectangle, draw=gray!40, fill=gray!5, font=\tiny, align=center] 
      at (5,-4.5) {تخمین بودجه سال اول: ۸۰۰-۱۰۰۰ هزار میلیارد تومان};

\end{tikzpicture}
\caption{جریان منابع و مصارف مالی دولت موقت}
\label{fig:financial-flow}
\end{figure}

%───────────────────────────────────────────────────────────────────────────────
\section{دیپلماسی گذار و رفع تحریم}
\label{sec:diplomacy}
%───────────────────────────────────────────────────────────────────────────────

\begin{naghlbox}
«هیچ دموکراسی‌ای در انزوای بین‌المللی و فقر اقتصادی پایدار نمی‌ماند. رفع تحریم‌ها نه امتیاز به بیگانه، که ضرورت بقای دموکراسی است.»
\sourceline{تحلیل‌گر}
\end{naghlbox}

\subsection{نقشه راه دیپلماتیک}

\begin{table}[htbp]
\centering
\caption{اولویت‌های دیپلماتیک ۱۲ ماه اول}
\label{tab:diplomacy-roadmap}
\begin{tabular}{>{\columncolor{blue!8}}r p{2.5cm} p{5cm} p{3.5cm}}
\toprule
\rowcolor{blue!25}
\textbf{ماه} & \textbf{محور} & \textbf{اقدام} & \textbf{هدف} \\
\midrule
۱ & سازمان ملل & نامه به دبیرکل، درخواست حمایت & مشروعیت‌سازی \\
\rowcolor{gray!10}
۱-۲ & همسایگان & پیام به ۱۵ همسایه، تعهد به صلح & اطمینان‌بخشی \\
۲-۳ & اتحادیه اروپا & مذاکره رفع تحریم‌های EU & آزادسازی اولیه \\
\rowcolor{gray!10}
۳-۴ & آمریکا & کانال عمان، پیشنهاد توافق موقت & تعلیق تحریم‌ها \\
۴-۶ & برجام ۲.۰ & مذاکره توافق جامع جدید & برنامه هسته‌ای \\
\rowcolor{gray!10}
۶-۸ & نهادهای مالی & IMF، بانک جهانی، FATF & وام و سرمایه‌گذاری \\
۸-۱۲ & سرمایه‌گذاران & کنفرانس بین‌المللی ایران & جذب FDI \\
\bottomrule
\end{tabular}
\end{table}

\subsection{پیشنهاد «معامله بزرگ» (Grand Bargain)}

\begin{olgoobox}
\textbf{پیشنهاد ایران به جامعه جهانی:}

\textbf{ایران تعهد می‌دهد:}
\begin{itemize}[nosep]
    \item توقف غنی‌سازی بالای ۳.۶۷٪
    \item پذیرش پروتکل الحاقی + نظارت‌های گسترده
    \item خروج نظامی از سوریه و یمن (تدریجی)
    \item قطع حمایت از گروه‌های مسلح منطقه‌ای
    \item به رسمیت شناختن اسرائیل در چارچوب راه‌حل دو دولت
\end{itemize}

\textbf{در ازای:}
\begin{itemize}[nosep]
    \item رفع کامل تحریم‌های آمریکا و اروپا
    \item آزادسازی ۱۰۰+ میلیارد دلار دارایی
    \item عضویت در WTO و FATF
    \item بسته کمک بازسازی ۵۰ میلیارد دلاری
    \item تضمین امنیتی (عدم تغییر رژیم)
\end{itemize}
\end{olgoobox}

\subsection{نمودار مراحل رفع تحریم}

\begin{figure}[htbp]
\centering
\begin{tikzpicture}
\begin{axis}[
    width=14cm,
    height=7cm,
    xlabel={ماه پس از گذار},
    ylabel={درصد تحریم‌های باقیمانده},
    xmin=0, xmax=24,
    ymin=0, ymax=110,
    grid=major,
    grid style={dashed, gray!30},
    legend style={at={(0.98,0.98)}, anchor=north east, font=\small},
    legend cell align=left
]

% سناریوی خوش‌بینانه
\addplot[thick, color=green!70!black, mark=*] coordinates {
    (0,100) (3,90) (6,70) (9,50) (12,30) (18,15) (24,5)
};

% سناریوی واقع‌بینانه
\addplot[thick, color=orange, mark=square*] coordinates {
    (0,100) (3,95) (6,85) (9,70) (12,55) (18,40) (24,25)
};

% سناریوی بدبینانه
\addplot[thick, color=red, mark=triangle*] coordinates {
    (0,100) (3,98) (6,95) (9,90) (12,85) (18,75) (24,60)
};

\legend{
    خوش‌بینانه (توافق سریع),
    واقع‌بینانه (مذاکره طولانی),
    بدبینانه (مقاومت داخلی/خارجی)
}

% مراحل کلیدی
\node[font=\tiny, anchor=south] at (axis cs:3,90) {تعلیق EU};
\node[font=\tiny, anchor=south] at (axis cs:6,70) {توافق موقت};
\node[font=\tiny, anchor=south] at (axis cs:12,30) {توافق جامع};

\end{axis}
\end{tikzpicture}
\caption{سناریوهای رفع تحریم در ۲۴ ماه اول}
\label{fig:sanctions-removal}
\end{figure}

%───────────────────────────────────────────────────────────────────────────────
\section{انتخابات مجلس مؤسسان}
\label{sec:constituent-assembly}
%───────────────────────────────────────────────────────────────────────────────

انتخابات مجلس مؤسسان مهم‌ترین رویداد سال اول است. این مجلس قانون اساسی جدید را تدوین می‌کند.

\subsection{نظام انتخاباتی پیشنهادی}

\begin{table}[htbp]
\centering
\caption{ویژگی‌های نظام انتخاباتی مجلس مؤسسان}
\label{tab:electoral-system}
\begin{tabular}{>{\columncolor{purple!8}}r p{4cm} p{6.5cm}}
\toprule
\rowcolor{purple!25}
\textbf{ردیف} & \textbf{ویژگی} & \textbf{شرح} \\
\midrule
۱ & تعداد کرسی‌ها & ۳۰۰ نماینده \\
\rowcolor{gray!10}
۲ & نظام انتخاباتی & مختلط: ۲۰۰ حوزه‌ای + ۱۰۰ نسبی \\
۳ & سهمیه جنسیتی & حداقل ۳۰٪ زنان (در لیست‌های نسبی) \\
\rowcolor{gray!10}
۴ & سهمیه قومی & تضمین حداقل برای مناطق قومی \\
۵ & آستانه ورود & ۳٪ آرای کشوری (برای نسبی) \\
\rowcolor{gray!10}
۶ & نظارت & کمیسیون مستقل + ناظران بین‌المللی \\
۷ & تأمین مالی & بودجه دولتی برابر + سقف هزینه \\
\rowcolor{gray!10}
۸ & دسترسی رسانه‌ای & وقت برابر رایگان در رسانه عمومی \\
۹ & حق رأی & همه شهروندان ۱۸+ ساله (داخل و خارج) \\
\bottomrule
\end{tabular}
\end{table}

\subsection{نقشه حوزه‌های انتخابیه}

%═══════════════════════════════════════════════════════════════════════════════
% ادامه فصل ۸: فاز ۱ — گذار (سال ۱-۲)
% فایل: chapters/ch08-phase1.tex (ادامه)
%═══════════════════════════════════════════════════════════════════════════════

% ادامه نقشه حوزه‌های انتخابیه...

\begin{figure}[htbp]
\centering
\begin{tikzpicture}[
    scale=0.75,
    transform shape,
    region/.style={
        ellipse,
        draw=#1!70!black,
        fill=#1!20,
        minimum width=2.8cm,
        minimum height=1.6cm,
        font=\scriptsize\bfseries,
        align=center
    }
]

% مناطق انتخاباتی
\node[region=blue] (tehran) at (0,0) {
    \begin{tabular}{c}
    تهران\\
    ۴۵ کرسی
    \end{tabular}
};

\node[region=teal] (azarbaijan) at (-4.5,2.5) {
    \begin{tabular}{c}
    آذربایجان\\
    ۳۵ کرسی
    \end{tabular}
};

\node[region=green] (kurdistan) at (-5.5,0) {
    \begin{tabular}{c}
    کردستان\\
    ۱۵ کرسی
    \end{tabular}
};

\node[region=orange] (khuzestan) at (-3.5,-2.5) {
    \begin{tabular}{c}
    خوزستان\\
    ۲۰ کرسی
    \end{tabular}
};

\node[region=red] (isfahan) at (0,-2.5) {
    \begin{tabular}{c}
    اصفهان\\
    ۲۵ کرسی
    \end{tabular}
};

\node[region=purple] (khorasan) at (4.5,1.5) {
    \begin{tabular}{c}
    خراسان\\
    ۳۵ کرسی
    \end{tabular}
};

\node[region=brown] (fars) at (2,-2.5) {
    \begin{tabular}{c}
    فارس\\
    ۲۰ کرسی
    \end{tabular}
};

\node[region=cyan] (mazandaran) at (2,2.5) {
    \begin{tabular}{c}
    شمال\\
    ۲۵ کرسی
    \end{tabular}
};

\node[region=yellow] (sistan) at (5.5,-1.5) {
    \begin{tabular}{c}
    سیستان\\
    ۱۰ کرسی
    \end{tabular}
};

\node[region=pink] (other) at (0,3) {
    \begin{tabular}{c}
    سایر استان‌ها\\
    ۷۰ کرسی
    \end{tabular}
};

% اتصالات نمادین
\draw[gray!30, dashed] (tehran) -- (azarbaijan);
\draw[gray!30, dashed] (tehran) -- (kurdistan);
\draw[gray!30, dashed] (tehran) -- (khuzestan);
\draw[gray!30, dashed] (tehran) -- (isfahan);
\draw[gray!30, dashed] (tehran) -- (khorasan);
\draw[gray!30, dashed] (tehran) -- (fars);
\draw[gray!30, dashed] (tehran) -- (mazandaran);
\draw[gray!30, dashed] (tehran) -- (sistan);

% جمع کل
\node[rectangle, rounded corners=5pt, draw=black!70, fill=black!10,
      font=\small\bfseries, align=center] at (0,-4.5) {
    مجموع کرسی‌های حوزه‌ای: ۲۰۰ | کرسی‌های نسبی ملی: ۱۰۰ | جمع: ۳۰۰
};

\end{tikzpicture}
\caption{توزیع تقریبی کرسی‌های مجلس مؤسسان بر اساس جمعیت مناطق}
\label{fig:constituency-map}
\end{figure}

\subsection{تضمینات انتخابات آزاد}

\begin{table}[htbp]
\centering
\caption{چک‌لیست تضمینات انتخابات آزاد و منصفانه}
\label{tab:election-guarantees}
\begin{tabular}{>{\columncolor{green!8}}r p{4cm} p{4cm} c}
\toprule
\rowcolor{green!25}
\textbf{ردیف} & \textbf{تضمین} & \textbf{سازوکار اجرایی} & \textbf{استاندارد بین‌المللی} \\
\midrule
۱ & ثبت‌نام فراگیر رأی‌دهندگان & سامانه الکترونیک + حضوری & ICCPR ماده ۲۵ \\
\rowcolor{gray!10}
۲ & آزادی نامزدی & بدون نظارت استصوابی & کنوانسیون اروپایی \\
۳ & دسترسی برابر به رسانه & سهمیه رایگان تلویزیون & قوانین OSCE \\
\rowcolor{gray!10}
۴ & نظارت بین‌المللی & دعوت از UN، EU، اتحادیه آفریقا & اعلامیه ۲۰۰۵ UN \\
۵ & شفافیت شمارش & شمارش علنی، انتشار فوری نتایج & ACE Electoral \\
\rowcolor{gray!10}
۶ & مکانیزم شکایت & دادگاه ویژه انتخاباتی & استانداردهای Venice \\
۷ & امنیت بدون ارعاب & پلیس انتخاباتی غیرمسلح & IDEA International \\
\rowcolor{gray!10}
۸ & رأی مخفی & کابین‌های استاندارد & اعلامیه جهانی حقوق بشر \\
\bottomrule
\end{tabular}
\end{table}

\begin{olgoobox}
\textbf{الگوی نظارت انتخاباتی تونس ۲۰۱۱:}
تونس با دعوت از ۵۰۰۰ ناظر بین‌المللی از ۶۰ کشور، معتبرترین انتخابات جهان عرب را برگزار کرد. کمیسیون مستقل انتخابات (ISIE) با ریاست یک قاضی بازنشسته، اعتماد همه احزاب را جلب کرد. نتیجه: پذیرش نتایج توسط همه طرف‌ها.
\end{olgoobox}

%───────────────────────────────────────────────────────────────────────────────
\section{فرآیند تدوین قانون اساسی}
\label{sec:constitution-drafting}
%───────────────────────────────────────────────────────────────────────────────

\subsection{مراحل تدوین قانون اساسی}

\begin{figure}[htbp]
\centering
\begin{tikzpicture}[
    scale=0.8,
    transform shape,
    stage/.style={
        rectangle,
        rounded corners=5pt,
        minimum width=3cm,
        minimum height=1.8cm,
        draw=#1!70!black,
        fill=#1!15,
        text=#1!30!black,
        font=\scriptsize\bfseries,
        align=center
    },
    arrow/.style={
        ->,
        >=stealth,
        very thick,
        draw=gray!60
    }
]

% مراحل
\node[stage=red] (prep) at (0,0) {
    \begin{tabular}{c}
    مرحله ۱\\
    \small آماده‌سازی\\
    \tiny (ماه ۱۳-۱۴)
    \end{tabular}
};

\node[stage=orange] (draft) at (4,0) {
    \begin{tabular}{c}
    مرحله ۲\\
    \small پیش‌نویس\\
    \tiny (ماه ۱۵-۱۶)
    \end{tabular}
};

\node[stage=yellow] (consult) at (8,0) {
    \begin{tabular}{c}
    مرحله ۳\\
    \small مشاوره عمومی\\
    \tiny (ماه ۱۷-۱۸)
    \end{tabular}
};

\node[stage=green] (revise) at (12,0) {
    \begin{tabular}{c}
    مرحله ۴\\
    \small بازنگری\\
    \tiny (ماه ۱۹-۲۰)
    \end{tabular}
};

\node[stage=blue] (vote) at (16,0) {
    \begin{tabular}{c}
    مرحله ۵\\
    \small همه‌پرسی\\
    \tiny (ماه ۲۱)
    \end{tabular}
};

% فلش‌ها
\draw[arrow] (prep) -- (draft);
\draw[arrow] (draft) -- (consult);
\draw[arrow] (consult) -- (revise);
\draw[arrow] (revise) -- (vote);

% توضیحات زیر
\node[below, font=\tiny, text width=2.5cm, align=center] at (0,-1.5) {
    تشکیل کمیسیون‌ها\\
    مشاوره بین‌المللی\\
    جمع‌آوری نظرات
};

\node[below, font=\tiny, text width=2.5cm, align=center] at (4,-1.5) {
    نوشتن متن اولیه\\
    ۱۵ کمیسیون تخصصی\\
    کارشناسان حقوقی
};

\node[below, font=\tiny, text width=2.5cm, align=center] at (8,-1.5) {
    جلسات شهری\\
    نظرسنجی آنلاین\\
    بحث رسانه‌ای
};

\node[below, font=\tiny, text width=2.5cm, align=center] at (12,-1.5) {
    اصلاح متن\\
    رأی‌گیری مجلس\\
    (اکثریت ۲/۳)
};

\node[below, font=\tiny, text width=2.5cm, align=center] at (16,-1.5) {
    رأی مردم\\
    آستانه ۵۰٪+۱\\
    اعلام رسمی
};

\end{tikzpicture}
\caption{پنج مرحله فرآیند تدوین قانون اساسی}
\label{fig:constitution-process}
\end{figure}

\subsection{ساختار کمیسیون‌های تخصصی مجلس مؤسسان}

\begin{table}[htbp]
\centering
\caption{کمیسیون‌های تخصصی مجلس مؤسسان}
\label{tab:constituent-commissions}
\begin{tabular}{>{\columncolor{purple!8}}r p{3.5cm} c p{5cm}}
\toprule
\rowcolor{purple!25}
\textbf{ردیف} & \textbf{کمیسیون} & \textbf{اعضا} & \textbf{حوزه کاری} \\
\midrule
۱ & اصول کلی و هویت ملی & ۲۰ & مقدمه، اصول بنیادین، نمادهای ملی \\
\rowcolor{gray!10}
۲ & حقوق و آزادی‌های بنیادین & ۲۵ & منشور حقوق، آزادی‌های فردی و جمعی \\
۳ & ساختار حکومت & ۲۵ & تفکیک قوا، روابط قوا \\
\rowcolor{gray!10}
۴ & قوه مجریه & ۲۰ & رئیس‌جمهور، دولت، وزارتخانه‌ها \\
۵ & قوه مقننه & ۲۰ & مجلس، مجلس اقوام، قانون‌گذاری \\
\rowcolor{gray!10}
۶ & قوه قضائیه & ۲۰ & دادگستری، دادگاه قانون اساسی \\
۷ & تمرکززدایی و حکومت محلی & ۲۵ & فدرالیسم، استان‌ها، شهرداری‌ها \\
\rowcolor{gray!10}
۸ & حقوق اقوام و اقلیت‌ها & ۲۵ & زبان‌ها، فرهنگ‌ها، خودمختاری \\
۹ & امور اقتصادی & ۲۰ & مالکیت، بودجه، بانک مرکزی \\
\rowcolor{gray!10}
۱۰ & امنیت و دفاع ملی & ۱۵ & ارتش، پلیس، اطلاعات \\
۱۱ & سیاست خارجی & ۱۵ & روابط بین‌الملل، معاهدات \\
\rowcolor{gray!10}
۱۲ & محیط زیست و منابع طبیعی & ۲۰ & آب، انرژی، محیط زیست \\
۱۳ & حقوق زنان و خانواده & ۲۰ & برابری جنسیتی، کودکان \\
\rowcolor{gray!10}
۱۴ & نهادهای نظارتی & ۱۵ & دیوان محاسبات، کمیسیون‌های مستقل \\
۱۵ & بازنگری و اجرا & ۱۵ & نحوه اصلاح، مقررات انتقالی \\
\bottomrule
\end{tabular}
\end{table}

\subsection{مشارکت عمومی در تدوین قانون اساسی}

\begin{olgoobox}
\textbf{الگوی ایسلند (۲۰۱۰-۲۰۱۲):}
ایسلند اولین کشوری بود که قانون اساسی را با مشارکت گسترده مردم از طریق اینترنت نوشت. شورای ۲۵ نفره شهروندان عادی (نه سیاست‌مداران) متن را نوشتند و هر هفته پیش‌نویس را آنلاین منتشر کردند. مردم ۳۶۰۰ پیشنهاد دادند که ۳۰۰ مورد در متن نهایی لحاظ شد.
\end{olgoobox}

\begin{figure}[htbp]
\centering
\begin{tikzpicture}[
    scale=0.85,
    transform shape,
    channel/.style={
        rectangle,
        rounded corners=4pt,
        minimum width=3cm,
        minimum height=1.3cm,
        draw=#1!70!black,
        fill=#1!15,
        text=#1!30!black,
        font=\scriptsize\bfseries,
        align=center
    }
]

% عنوان
\node[font=\bfseries] at (6,4) {کانال‌های مشارکت عمومی در تدوین قانون اساسی};

% کانال‌ها
\node[channel=blue] (online) at (0,2) {
    \begin{tabular}{c}
    پلتفرم آنلاین\\
    \tiny نظردهی، رأی‌گیری
    \end{tabular}
};

\node[channel=green] (townhall) at (3,2) {
    \begin{tabular}{c}
    جلسات شهری\\
    \tiny ۵۰۰ جلسه سراسری
    \end{tabular}
};

\node[channel=orange] (sms) at (6,2) {
    \begin{tabular}{c}
    پیامک و تلفن\\
    \tiny برای مناطق کم‌دسترس
    \end{tabular}
};

\node[channel=purple] (media) at (9,2) {
    \begin{tabular}{c}
    رسانه ملی\\
    \tiny برنامه هفتگی
    \end{tabular}
};

\node[channel=red] (ngo) at (12,2) {
    \begin{tabular}{c}
    نهادهای مدنی\\
    \tiny پیشنهادات تخصصی
    \end{tabular}
};

% مرکز جمع‌آوری
\node[rectangle, rounded corners=6pt, draw=black!70, fill=black!10,
      minimum width=4cm, minimum height=1.5cm, font=\small\bfseries,
      align=center] (collect) at (6,0) {
    \begin{tabular}{c}
    دبیرخانه مشارکت عمومی\\
    \tiny پردازش و دسته‌بندی نظرات
    \end{tabular}
};

% فلش‌ها
\draw[->, thick, gray!60] (online) -- (collect);
\draw[->, thick, gray!60] (townhall) -- (collect);
\draw[->, thick, gray!60] (sms) -- (collect);
\draw[->, thick, gray!60] (media) -- (collect);
\draw[->, thick, gray!60] (ngo) -- (collect);

% خروجی
\node[rectangle, rounded corners=4pt, draw=teal!70, fill=teal!15,
      minimum width=3cm, minimum height=1cm, font=\scriptsize\bfseries,
      text=teal!30!black, align=center] (output) at (6,-2) {
    گزارش به کمیسیون‌ها
};

\draw[->, thick, teal!60] (collect) -- (output);

% آمار
\node[rectangle, draw=gray!40, fill=gray!5, font=\tiny, align=right,
      text width=4cm] at (12,0) {
    \textbf{اهداف کمّی:}\\
    • ۵ میلیون بازدید آنلاین\\
    • ۵۰۰,۰۰۰ نظر ثبت‌شده\\
    • ۱۰۰,۰۰۰ شرکت‌کننده حضوری\\
    • ۳۱ استان پوشش کامل
};

\end{tikzpicture}
\caption{سازوکار مشارکت عمومی در تدوین قانون اساسی}
\label{fig:public-participation}
\end{figure}

%───────────────────────────────────────────────────────────────────────────────
\section{عدالت انتقالی در فاز اول}
\label{sec:transitional-justice-phase1}
%───────────────────────────────────────────────────────────────────────────────

\begin{enghelabbox}
\textbf{تعادل دشوار:} عدالت انتقالی در فاز اول باید میان دو خطر حرکت کند:
\begin{itemize}[nosep]
    \item \textbf{شتاب‌زدگی:} محاکمات سریع و بدون تضمینات می‌تواند به انتقام‌جویی بدل شود
    \item \textbf{تعلل:} عدم اقدام می‌تواند اعتماد قربانیان را سلب کند
\end{itemize}
راه میانه: اقدامات نمادین فوری + آماده‌سازی برای فرآیند جامع
\end{enghelabbox}

\subsection{اقدامات فوری عدالت انتقالی}

\begin{table}[htbp]
\centering
\caption{اقدامات عدالت انتقالی در ۱۲ ماه اول}
\label{tab:tj-phase1}
\begin{tabular}{>{\columncolor{orange!8}}r p{3cm} p{4.5cm} p{3.5cm}}
\toprule
\rowcolor{orange!25}
\textbf{ماه} & \textbf{اقدام} & \textbf{شرح} & \textbf{نهاد مسئول} \\
\midrule
۱-۲ & آزادی زندانیان & همه زندانیان سیاسی و عقیدتی & وزارت دادگستری \\
\rowcolor{gray!10}
۲-۳ & بازداشت متهمان اصلی & فرماندهان سرکوب، شکنجه‌گران & دادستانی موقت \\
۳-۴ & تشکیل کمیسیون حقیقت‌یاب & ۱۵ عضو مستقل & شورای انتقالی \\
\rowcolor{gray!10}
۴-۶ & ثبت شهدا و قربانیان & بانک اطلاعاتی جامع & کمیسیون حقیقت \\
۶-۸ & غرامت اولیه به خانواده شهدا & ماهی ۱۰ میلیون تومان & صندوق قربانیان \\
\rowcolor{gray!10}
۸-۱۰ & نامگذاری‌های نمادین & خیابان‌ها، میادین به نام شهدا & شهرداری‌ها \\
۱۰-۱۲ & آغاز جلسات استماع & شهادت قربانیان (تلویزیونی) & کمیسیون حقیقت \\
\bottomrule
\end{tabular}
\end{table}

\subsection{ساختار کمیسیون حقیقت‌یاب}

\begin{figure}[htbp]
\centering
\begin{tikzpicture}[
    scale=0.85,
    transform shape,
    unit/.style={
        rectangle,
        rounded corners=4pt,
        minimum width=2.8cm,
        minimum height=1.2cm,
        draw=#1!70!black,
        fill=#1!15,
        text=#1!30!black,
        font=\scriptsize\bfseries,
        align=center
    }
]

% کمیسیون مرکزی
\node[unit=purple, minimum width=5cm, minimum height=1.8cm] (main) at (0,4) {
    \begin{tabular}{c}
    کمیسیون حقیقت‌یاب\\
    \small ۱۵ عضو (۷ زن، ۸ مرد)\\
    \tiny ریاست: قاضی بین‌المللی
    \end{tabular}
};

% واحدهای تخصصی
\node[unit=red] (political) at (-5,1.5) {
    \begin{tabular}{c}
    واحد جنایات\\
    سیاسی
    \end{tabular}
};

\node[unit=orange] (economic) at (-2.5,1.5) {
    \begin{tabular}{c}
    واحد فساد\\
    اقتصادی
    \end{tabular}
};

\node[unit=yellow] (women) at (0,1.5) {
    \begin{tabular}{c}
    واحد خشونت\\
    علیه زنان
    \end{tabular}
};

\node[unit=green] (ethnic) at (2.5,1.5) {
    \begin{tabular}{c}
    واحد تبعیض\\
    قومی
    \end{tabular}
};

\node[unit=blue] (docs) at (5,1.5) {
    \begin{tabular}{c}
    واحد اسناد\\
    و آرشیو
    \end{tabular}
};

% دفاتر استانی
\node[unit=gray, minimum width=10cm] (regional) at (0,-0.5) {
    \begin{tabular}{c}
    دفاتر استانی (۳۱ استان)\\
    \tiny جمع‌آوری شهادت‌ها و مدارک محلی
    \end{tabular}
};

% خروجی‌ها
\node[unit=teal] (report) at (-3,-2.5) {
    \begin{tabular}{c}
    گزارش نهایی\\
    \tiny (پایان سال ۳)
    \end{tabular}
};

\node[unit=cyan] (recs) at (0,-2.5) {
    \begin{tabular}{c}
    توصیه‌ها\\
    \tiny (اصلاحات نهادی)
    \end{tabular}
};

\node[unit=lime] (pros) at (3,-2.5) {
    \begin{tabular}{c}
    ارجاع به دادستانی\\
    \tiny (متهمان اصلی)
    \end{tabular}
};

% اتصالات
\draw[->, gray!60] (main) -- (political);
\draw[->, gray!60] (main) -- (economic);
\draw[->, gray!60] (main) -- (women);
\draw[->, gray!60] (main) -- (ethnic);
\draw[->, gray!60] (main) -- (docs);

\draw[->, gray!60] (political) -- (regional);
\draw[->, gray!60] (economic) -- (regional);
\draw[->, gray!60] (women) -- (regional);
\draw[->, gray!60] (ethnic) -- (regional);
\draw[->, gray!60] (docs) -- (regional);

\draw[->, gray!60] (regional) -- (report);
\draw[->, gray!60] (regional) -- (recs);
\draw[->, gray!60] (regional) -- (pros);

\end{tikzpicture}
\caption{ساختار کمیسیون حقیقت‌یاب و آشتی ملی}
\label{fig:truth-commission}
\end{figure}

\begin{naghlbox}
«حقیقت پایه آشتی است. بدون دانستن آنچه رخ داده، بخشش معنا ندارد.»
\sourceline{دزموند توتو، رئیس کمیسیون حقیقت آفریقای جنوبی}
\end{naghlbox}

%───────────────────────────────────────────────────────────────────────────────
\section{چالش‌ها و ریسک‌های فاز اول}
\label{sec:phase1-risks}
%───────────────────────────────────────────────────────────────────────────────

\subsection{ماتریس ریسک فاز اول}

\begin{table}[htbp]
\centering
\caption{ریسک‌های اصلی فاز اول و راهبردهای مقابله}
\label{tab:phase1-risks}
\begin{tabular}{>{\columncolor{red!8}}r p{2.5cm} c c p{4.5cm}}
\toprule
\rowcolor{red!25}
\textbf{کد} & \textbf{ریسک} & \textbf{احتمال} & \textbf{شدت} & \textbf{راهبرد مقابله} \\
\midrule
R1 & کودتای نظامی & متوسط & بحرانی & مشارکت ارتش در گذار، تضمین منافع \\
\rowcolor{gray!10}
R2 & جنگ داخلی قومی & کم & بحرانی & توافق سریع فدرالیسم، نمایندگی اقوام \\
R3 & فروپاشی اقتصادی & بالا & شدید & کمک بین‌المللی، رفع سریع تحریم \\
\rowcolor{gray!10}
R4 & انتقام‌جویی خونین & متوسط & شدید & عدالت انتقالی منظم، پلیس قوی \\
R5 & ربوده‌شدن انقلاب & متوسط & شدید & شورای فراگیر، شفافیت \\
\rowcolor{gray!10}
R6 & مداخله خارجی & کم & متوسط & دیپلماسی فعال، بی‌طرفی \\
R7 & اعتراضات حداکثرگرا & بالا & متوسط & آبادانی ملموس، گفتگوی مداوم \\
\rowcolor{gray!10}
R8 & کمبود نیروی کارآمد & بالا & متوسط & بازگشت متخصصان، مشاوران بین‌المللی \\
\bottomrule
\end{tabular}
\end{table}

\subsection{نمودار ریسک}

\begin{figure}[htbp]
\centering
\begin{tikzpicture}
\begin{axis}[
    width=11cm,
    height=9cm,
    xlabel={احتمال وقوع},
    ylabel={شدت تأثیر},
    xmin=0, xmax=100,
    ymin=0, ymax=100,
    xtick={20,50,80},
    xticklabels={کم, متوسط, بالا},
    ytick={20,50,80},
    yticklabels={کم, متوسط, بحرانی},
    grid=major,
    grid style={dashed, gray!30},
]

% مناطق رنگی
\fill[green!20, opacity=0.5] (0,0) rectangle (40,40);
\fill[yellow!20, opacity=0.5] (0,40) rectangle (40,100);
\fill[yellow!20, opacity=0.5] (40,0) rectangle (100,40);
\fill[orange!20, opacity=0.5] (40,40) rectangle (70,70);
\fill[red!20, opacity=0.5] (40,70) rectangle (100,100);
\fill[red!20, opacity=0.5] (70,40) rectangle (100,100);

% ریسک‌ها
\node[circle, fill=red!70, minimum size=0.4cm, font=\tiny\bfseries, text=white] 
    at (axis cs:50,90) {R1};
\node[circle, fill=purple!70, minimum size=0.4cm, font=\tiny\bfseries, text=white] 
    at (axis cs:25,85) {R2};
\node[circle, fill=orange!70, minimum size=0.4cm, font=\tiny\bfseries, text=white] 
    at (axis cs:75,75) {R3};
\node[circle, fill=red!60, minimum size=0.4cm, font=\tiny\bfseries, text=white] 
    at (axis cs:50,70) {R4};
\node[circle, fill=orange!60, minimum size=0.4cm, font=\tiny\bfseries, text=white] 
    at (axis cs:55,65) {R5};
\node[circle, fill=yellow!70, minimum size=0.4cm, font=\tiny\bfseries, text=black] 
    at (axis cs:30,50) {R6};
\node[circle, fill=yellow!60, minimum size=0.4cm, font=\tiny\bfseries, text=black] 
    at (axis cs:70,55) {R7};
\node[circle, fill=orange!50, minimum size=0.4cm, font=\tiny\bfseries, text=white] 
    at (axis cs:75,50) {R8};

% راهنما
\node[font=\tiny, align=left] at (axis cs:85,15) {
    R1: کودتا\\
    R2: جنگ داخلی\\
    R3: فروپاشی اقتصاد\\
    R4: انتقام‌جویی
};
\node[font=\tiny, align=left] at (axis cs:85,35) {
    R5: ربودن انقلاب\\
    R6: مداخله خارجی\\
    R7: اعتراضات\\
    R8: کمبود نیرو
};

\end{axis}
\end{tikzpicture}
\caption{ماتریس ریسک فاز اول گذار}
\label{fig:risk-matrix-phase1}
\end{figure}

%───────────────────────────────────────────────────────────────────────────────
\section{شاخص‌های موفقیت فاز اول}
\label{sec:phase1-kpis}
%───────────────────────────────────────────────────────────────────────────────

\begin{table}[htbp]
\centering
\caption{شاخص‌های کلیدی موفقیت (KPI) فاز اول}
\label{tab:phase1-kpis}
\begin{tabular}{>{\columncolor{blue!8}}r p{4cm} c c c}
\toprule
\rowcolor{blue!25}
\textbf{کد} & \textbf{شاخص} & \textbf{هدف ماه ۶} & \textbf{هدف ماه ۱۲} & \textbf{هدف ماه ۲۴} \\
\midrule
K01 & تلفات غیرنظامی ناشی از ناآرامی & < ۱۰۰ & < ۵۰ & < ۱۰ \\
\rowcolor{gray!10}
K02 & زندانیان سیاسی باقیمانده & ۰ & ۰ & ۰ \\
K03 & نرخ تورم ماهانه & < ۵٪ & < ۳٪ & < ۲٪ \\
\rowcolor{gray!10}
K04 & کشورهای شناسایی‌کننده & ۵۰+ & ۱۵۰+ & ۱۹۰+ \\
K05 & درصد رفع تحریم & ۲۰٪ & ۵۰٪ & ۸۰٪ \\
\rowcolor{gray!10}
K06 & احزاب ثبت‌شده & ۱۰+ & ۳۰+ & ۵۰+ \\
K07 & مشارکت در انتخابات مؤسسان & — & ۶۵٪+ & — \\
\rowcolor{gray!10}
K08 & تأیید قانون اساسی در همه‌پرسی & — & — & ۶۰٪+ \\
K09 & رضایت عمومی از روند گذار & ۶۰٪ & ۶۵٪ & ۷۰٪ \\
\rowcolor{gray!10}
K10 & بازگشت مهاجران & ۱۰۰,۰۰۰ & ۳۰۰,۰۰۰ & ۵۰۰,۰۰۰ \\
\bottomrule
\end{tabular}
\end{table}

%───────────────────────────────────────────────────────────────────────────────
\section{جمع‌بندی: ۲۴ ماه سرنوشت‌ساز}
\label{sec:phase1-conclusion}
%───────────────────────────────────────────────────────────────────────────────

\begin{kholasebox}
\textbf{خلاصه فصل ۸:}
\begin{enumerate}
    \item \textbf{لحظه صفر} نیازمند آمادگی قبلی است — شورای انتقالی باید از پیش شکل گرفته باشد
    \item \textbf{شورای انتقالی ۲۱ نفره} باید فراگیر و نماینده همه اقشار باشد
    \item \textbf{دولت موقت تکنوکرات} باید بحران‌ها را مدیریت کند، نه سیاست‌بازی
    \item \textbf{۶ ماه اول} تمرکز بر امنیت، آزادی‌ها، و تثبیت اقتصادی
    \item \textbf{انتخابات مجلس مؤسسان} در ماه ۱۱ — نقطه عطف دموکراتیک
    \item \textbf{قانون اساسی جدید} با مشارکت گسترده مردمی تدوین می‌شود
    \item \textbf{همه‌پرسی ماه ۲۱} و \textbf{انتخابات عمومی ماه ۲۴} پایان دوره گذار
    \item \textbf{مدیریت نیروهای امنیتی} با رویکرد اصلاح (نه انحلال کامل)
    \item \textbf{رفع تحریم‌ها} حیاتی‌ترین اولویت دیپلماتیک است
    \item \textbf{عدالت انتقالی} باید متعادل باشد — نه شتاب‌زده، نه تعلل‌آمیز
\end{enumerate}
\end{kholasebox}

% نمودار جمع‌بندی
\begin{figure}[htbp]
\centering
\begin{tikzpicture}[
    scale=0.75,
    transform shape,
    milestone/.style={
        rectangle,
        rounded corners=5pt,
        minimum width=2.5cm,
        minimum height=1.2cm,
        draw=#1!70!black,
        fill=#1!20,
        text=#1!30!black,
        font=\scriptsize\bfseries,
        align=center
    }
]

% خط زمان
\draw[very thick, gray!60, ->] (0,0) -- (18,0);

% نقاط زمانی
\foreach \x/\label in {0/روز ۱, 4/ماه ۶, 8/ماه ۱۲, 12/ماه ۱۸, 16/ماه ۲۴} {
    \draw[thick, gray!60] (\x,0.2) -- (\x,-0.2);
    \node[below, font=\small] at (\x,-0.4) {\label};
}

% رویدادهای کلیدی
\node[milestone=red] at (0,2) {
    \begin{tabular}{c}
    شورای\\
    انتقالی
    \end{tabular}
};

\node[milestone=orange] at (4,2) {
    \begin{tabular}{c}
    تثبیت\\
    امنیت و اقتصاد
    \end{tabular}
};

\node[milestone=yellow] at (8,2) {
    \begin{tabular}{c}
    انتخابات\\
    مؤسسان
    \end{tabular}
};

\node[milestone=green] at (12,2) {
    \begin{tabular}{c}
    پیش‌نویس\\
    قانون اساسی
    \end{tabular}
};

\node[milestone=blue] at (16,2) {
    \begin{tabular}{c}
    همه‌پرسی +\\
    انتخابات
    \end{tabular}
};

% فلش‌ها
\foreach \x in {0,4,8,12,16} {
    \draw[->, gray!50] (\x,0.3) -- (\x,1.3);
}

% نتیجه نهایی
\node[rectangle, rounded corners=8pt, draw=purple!70, fill=purple!10,
      minimum width=6cm, minimum height=1.5cm, font=\small\bfseries,
      text=purple!70, align=center] at (9,-2.5) {
    \begin{tabular}{c}
    پایان فاز ۱: دولت منتخب دموکراتیک\\
    قانون اساسی جدید | نهادهای اولیه
    \end{tabular}
};

\draw[->, very thick, purple!60] (16,-0.5) -- (16,-1.5) -- (12,-1.5) -- (12,-1.8);

\end{tikzpicture}
\caption{نقاط عطف فاز اول گذار}
\label{fig:phase1-milestones}
\end{figure}

\begin{naghlbox}
«دو سال اول گذار، بیش از دو دهه آینده را شکل می‌دهد. هر اشتباه در این دوره، سال‌ها برای اصلاح زمان می‌برد.»
\sourceline{گیلرمو اودانل، نظریه‌پرداز گذار دموکراتیک}
\end{naghlbox}

%───────────────────────────────────────────────────────────────────────────────
% منابع فصل
%───────────────────────────────────────────────────────────────────────────────

\vspace{1cm}
\begin{refsection}

\textbf{\large منابع فصل هشتم}

\vspace{0.5cm}

\begin{enumerate}[label={[\arabic*]}, nosep, leftmargin=*]
    \item O'Donnell, G., Schmitter, P. \& Whitehead, L. (1986). \textit{Transitions from Authoritarian Rule}. Johns Hopkins University Press.
    
    \item Linz, J. \& Stepan, A. (1996). \textit{Problems of Democratic Transition and Consolidation}. Johns Hopkins University Press.
    
    \item Carothers, T. (2002). "The End of the Transition Paradigm." \textit{Journal of Democracy}, 13(1), 5-21.
    
    \item International IDEA. (2017). \textit{Constitution-Building: A Global Review}. Stockholm.
    
    \item Elster, J. (1995). "Forces and Mechanisms in the Constitution-Making Process." \textit{Duke Law Journal}, 45(2), 364-396.
    
    \item Hayner, P. (2011). \textit{Unspeakable Truths: Transitional Justice and the Challenge of Truth Commissions}. Routledge.
    
    \item De Greiff, P. (2012). "Theorizing Transitional Justice." \textit{Nomos}, 51, 31-77.
    
    \item Diamond, L. (1999). \textit{Developing Democracy: Toward Consolidation}. Johns Hopkins University Press.
    
    \item Przeworski, A. (1991). \textit{Democracy and the Market}. Cambridge University Press.
    
    \item IMF. (2023). \textit{Staff Report: Islamic Republic of Iran}.
    
    \item Venice Commission. (2022). \textit{Report on Constitutional Amendment}. Council of Europe.
    
    \item ACE Electoral Knowledge Network. (2023). \textit{Electoral System Design Database}.
    
    \item Bremer, P. (2006). \textit{My Year in Iraq}. Simon \& Schuster.
    
    \item بشیریه، حسین. (۱۳۹۹). \textit{گذار به دموکراسی}. نشر نگاه معاصر.
    
    \item آبراهامیان، یرواند. (۱۳۹۸). \textit{تاریخ ایران مدرن}. ترجمه محمد ابراهیم فتاحی. نشر نی.
    
    \item Human Rights Watch. (2023). \textit{World Report 2023: Iran}.
\end{enumerate}

\end{refsection}
    
	%═══════════════════════════════════════════════════════════════════════════════
% فصل ۹: فاز ۲ — نهادسازی (سال ۳-۵)
% فایل: chapters/ch09-phase2.tex
%═══════════════════════════════════════════════════════════════════════════════

\chapter{فاز ۲: نهادسازی (سال ۳-۵)}
\label{chap:phase2}

\begin{kholasebox}
\textbf{خلاصه فصل:}
فاز دوم گذار به ساختن نهادهای پایدار دموکراتیک اختصاص دارد. در این سه سال، قانون اساسی جدید اجرایی می‌شود، ساختار فدرالی کشور شکل می‌گیرد، قوه قضائیه مستقل بنا می‌شود، نظام آموزشی بازسازی می‌گردد، و جامعه مدنی تقویت می‌شود. این فصل نقشه راه تفصیلی نهادسازی را با تأکید بر تجربه کشورهایی که در این فاز موفق بودند (آلمان، ژاپن، اسپانیا) ارائه می‌دهد. چالش اصلی این دوره، تبدیل وعده‌های قانون اساسی به واقعیت‌های نهادی است.
\end{kholasebox}

%───────────────────────────────────────────────────────────────────────────────
\section{مقدمه: از قانون تا نهاد}
\label{sec:phase2-intro}
%───────────────────────────────────────────────────────────────────────────────

\begin{naghlbox}
«قانون اساسی کاغذی است اگر نهادهایی برای اجرایش وجود نداشته باشد. نهادها استخوان‌بندی دموکراسی‌اند؛ بدون آن‌ها، دموکراسی تنها شعار است.»
\sourceline{داگلاس نورث، برنده نوبل اقتصاد}
\end{naghlbox}

با پایان فاز اول، ایران دارای قانون اساسی جدید و دولت منتخب است. اما این تازه آغاز راه است. تجربه جهانی نشان می‌دهد که بسیاری از کشورها قانون اساسی دموکراتیک دارند اما دموکراسی واقعی ندارند. تفاوت در \textbf{نهادسازی} است.

\subsection{تعریف نهاد}

\begin{olgoobox}
\textbf{نهاد چیست؟}

نهاد (Institution) مجموعه‌ای از قواعد، رویه‌ها، و سازمان‌هاست که:
\begin{itemize}[nosep]
    \item رفتار افراد و گروه‌ها را شکل می‌دهد
    \item پایدار و مستقل از افراد خاص است
    \item خودتقویت‌کننده است (نقض آن هزینه دارد)
    \item مشروعیت اجتماعی دارد
\end{itemize}

\textbf{مثال:} دادگاه قانون اساسی یک نهاد است — فراتر از قضات فعلی، با رویه‌های مشخص، و مورد احترام جامعه.
\end{olgoobox}

\subsection{اولویت‌های نهادسازی فاز دوم}

% نمودار اولویت‌های نهادسازی
\begin{figure}[htbp]
\centering
\begin{tikzpicture}[
    scale=0.85,
    transform shape,
    priority/.style={
        rectangle,
        rounded corners=5pt,
        minimum width=3.2cm,
        minimum height=1.4cm,
        draw=#1!70!black,
        fill=#1!15,
        text=#1!30!black,
        font=\small\bfseries,
        align=center
    }
]

% ستون اول - اولویت حیاتی
\node[font=\bfseries, text=red!70] at (0,4) {اولویت حیاتی};
\node[priority=red] (p1) at (0,2.5) {
    \begin{tabular}{c}
    استقلال\\
    قوه قضائیه
    \end{tabular}
};
\node[priority=red] (p2) at (0,0.5) {
    \begin{tabular}{c}
    نظام فدرالی\\
    و استان‌ها
    \end{tabular}
};

% ستون دوم - اولویت بالا
\node[font=\bfseries, text=orange!70] at (4.5,4) {اولویت بالا};
\node[priority=orange] (p3) at (4.5,2.5) {
    \begin{tabular}{c}
    نهادهای\\
    نظارتی مستقل
    \end{tabular}
};
\node[priority=orange] (p4) at (4.5,0.5) {
    \begin{tabular}{c}
    اصلاح نظام\\
    آموزشی
    \end{tabular}
};

% ستون سوم - اولویت متوسط
\node[font=\bfseries, text=yellow!70!black] at (9,4) {اولویت متوسط};
\node[priority=yellow] (p5) at (9,2.5) {
    \begin{tabular}{c}
    تقویت\\
    جامعه مدنی
    \end{tabular}
};
\node[priority=yellow] (p6) at (9,0.5) {
    \begin{tabular}{c}
    اصلاح\\
    بوروکراسی
    \end{tabular}
};

% ستون چهارم - در حال انجام
\node[font=\bfseries, text=green!70!black] at (13.5,4) {تداوم از فاز ۱};
\node[priority=green] (p7) at (13.5,2.5) {
    \begin{tabular}{c}
    رفع تحریم\\
    و بازسازی اقتصاد
    \end{tabular}
};
\node[priority=green] (p8) at (13.5,0.5) {
    \begin{tabular}{c}
    عدالت\\
    انتقالی
    \end{tabular}
};

% خط پایه
\draw[thick, gray!40] (-1.5,-0.8) -- (15,-0.8);
\node[below, font=\small, text=gray] at (6.75,-1.2) {سال ۳ تا ۵ پس از گذار};

\end{tikzpicture}
\caption{اولویت‌بندی نهادسازی در فاز دوم}
\label{fig:phase2-priorities}
\end{figure}

%───────────────────────────────────────────────────────────────────────────────
\section{استقرار ساختار فدرالی}
\label{sec:federal-structure}
%───────────────────────────────────────────────────────────────────────────────

\begin{enghelabbox}
\textbf{چرا فدرالیسم برای ایران ضروری است؟}
\begin{itemize}[nosep]
    \item تنوع قومی-زبانی: ۴۰٪ جمعیت غیرفارس‌زبان
    \item بی‌اعتمادی تاریخی به مرکز: ۱۰۰ سال تمرکزگرایی
    \item نابرابری منطقه‌ای: شکاف ۴ برابری درآمد سرانه
    \item وسعت جغرافیایی: ۱.۶ میلیون کیلومتر مربع
    \item تجربه همسایگان: موفقیت فدرالیسم در هند و شکست تمرکز در عراق
\end{itemize}
\end{enghelabbox}

\subsection{مدل فدرالیسم همبسته برای ایران}

\begin{table}[htbp]
\centering
\caption{مقایسه مدل‌های فدرالیسم و انتخاب برای ایران}
\label{tab:federalism-models}
\begin{tabular}{>{\columncolor{blue!8}}r p{2.5cm} p{2.5cm} p{2.5cm} p{2.5cm} p{2cm}}
\toprule
\rowcolor{blue!25}
\textbf{ویژگی} & \textbf{آمریکا} & \textbf{آلمان} & \textbf{هند} & \textbf{سوئیس} & \textbf{پیشنهاد ایران} \\
\midrule
نوع & دوگانه & همبسته & نامتقارن & مشارکتی & همبسته \\
\rowcolor{gray!10}
تعداد واحدها & ۵۰ & ۱۶ & ۲۸ & ۲۶ & ۵ منطقه + ۳۱ استان \\
استقلال مالی & بالا & متوسط & پایین & بالا & متوسط \\
\rowcolor{gray!10}
زبان رسمی & ۱ & ۱ & ۲۲ & ۴ & ۱ ملی + ۶ منطقه‌ای \\
انحلال واحد & ممکن نیست & ممکن نیست & ممکن & ممکن نیست & ممکن نیست \\
\rowcolor{gray!10}
مجلس دوم & سنا (ایالتی) & بوندسرات & راجیا سابا & شورای ایالات & مجلس اقوام \\
\bottomrule
\end{tabular}
\end{table}

\subsection{تقسیمات پیشنهادی کشوری}

% نمودار ساختار سه‌لایه
\begin{figure}[htbp]
\centering
\begin{tikzpicture}[
    scale=0.8,
    transform shape,
    level/.style={
        rectangle,
        rounded corners=6pt,
        minimum height=1.8cm,
        draw=#1!70!black,
        fill=#1!15,
        text=#1!30!black,
        font=\small\bfseries,
        align=center
    }
]

% سطح ملی
\node[level=purple, minimum width=14cm] (national) at (0,6) {
    \begin{tabular}{c}
    سطح ملی (فدرال)\\
    \small دولت مرکزی — پارلمان دو مجلسی — دادگاه قانون اساسی
    \end{tabular}
};

% سطح منطقه‌ای
\node[level=blue, minimum width=14cm] (regional) at (0,3) {
    \begin{tabular}{c}
    سطح منطقه‌ای (۵ منطقه خودمختار)\\
    \small آذربایجان | کردستان | بلوچستان | عرب‌نشین | مرکزی
    \end{tabular}
};

% سطح استانی
\node[level=green, minimum width=14cm] (provincial) at (0,0) {
    \begin{tabular}{c}
    سطح استانی (۳۱ استان)\\
    \small استاندار منتخب — شورای استان — دادگاه‌های استانی
    \end{tabular}
};

% سطح محلی
\node[level=orange, minimum width=14cm] (local) at (0,-3) {
    \begin{tabular}{c}
    سطح محلی (شهرها و دهستان‌ها)\\
    \small شهردار منتخب — شورای شهر/روستا — خدمات محلی
    \end{tabular}
};

% فلش‌ها
\draw[->, very thick, gray!60] (national) -- (regional) node[midway, right, font=\scriptsize] {تفویض اختیار};
\draw[->, very thick, gray!60] (regional) -- (provincial) node[midway, right, font=\scriptsize] {هماهنگی};
\draw[->, very thick, gray!60] (provincial) -- (local) node[midway, right, font=\scriptsize] {نظارت};

% بازخورد
\draw[->, thick, red!40] (local.west) -- ++(-1,0) -- ++(0,9) -- (national.west);
\node[font=\tiny, text=red!60, rotate=90] at (-8.5,1.5) {نمایندگی و مشارکت};

\end{tikzpicture}
\caption{ساختار چهارلایه حکومت در نظام فدرالی پیشنهادی}
\label{fig:federal-layers}
\end{figure}

\subsection{پنج منطقه خودمختار}

\begin{table}[htbp]
\centering
\caption{مشخصات پنج منطقه خودمختار پیشنهادی}
\label{tab:autonomous-regions}
\begin{tabular}{>{\columncolor{teal!8}}r p{2.3cm} c c p{2.5cm} p{3cm}}
\toprule
\rowcolor{teal!25}
\textbf{منطقه} & \textbf{استان‌ها} & \textbf{جمعیت (م)} & \textbf{مساحت (هزار km²)} & \textbf{زبان منطقه‌ای} & \textbf{مرکز پیشنهادی} \\
\midrule
آذربایجان & شرقی، غربی، اردبیل، زنجان & ۱۲ & ۱۰۵ & ترکی آذربایجانی & تبریز \\
\rowcolor{gray!10}
کردستان & کردستان، کرمانشاه، ایلام، بخشی از همدان & ۶ & ۶۵ & کردی & سنندج \\
بلوچستان & سیستان و بلوچستان & ۳ & ۱۸۷ & بلوچی & زاهدان \\
\rowcolor{gray!10}
عربستان & خوزستان & ۵ & ۶۴ & عربی & اهواز \\
مرکزی & ۲۳ استان باقیمانده & ۵۹ & ۱,۲۲۷ & فارسی & تهران \\
\midrule
\rowcolor{teal!15}
\textbf{جمع} & \textbf{۳۱} & \textbf{۸۵} & \textbf{۱,۶۴۸} & & \\
\bottomrule
\end{tabular}
\end{table}

\begin{olgoobox}
\textbf{الگوی اسپانیا: «دولت خودمختاری‌ها»}

اسپانیا با ۱۷ منطقه خودمختار (Comunidades Autónomas) نشان داد که فدرالیسم نامتقارن می‌تواند هم وحدت ملی را حفظ کند و هم به خواسته‌های قومی پاسخ دهد. کاتالونیا و باسک بیشترین خودمختاری را دارند (پلیس محلی، آموزش به زبان محلی، مالیات مستقل) درحالی‌که مناطق دیگر کمتر. این انعطاف، جدایی‌طلبی را کنترل کرده است.
\end{olgoobox}

\subsection{تقسیم صلاحیت‌ها}

\begin{table}[htbp]
\centering
\caption{تقسیم صلاحیت‌ها بین سطوح حکومتی}
\label{tab:competence-division}
\begin{tabular}{>{\columncolor{purple!8}}r p{4cm} c c c c}
\toprule
\rowcolor{purple!25}
\textbf{ردیف} & \textbf{حوزه} & \textbf{ملی} & \textbf{منطقه} & \textbf{استان} & \textbf{محلی} \\
\midrule
۱ & دفاع و امنیت ملی & \cmark & & & \\
\rowcolor{gray!10}
۲ & سیاست خارجی & \cmark & & & \\
۳ & پول و بانکداری & \cmark & & & \\
\rowcolor{gray!10}
۴ & گمرک و تجارت خارجی & \cmark & & & \\
۵ & شهروندی و مهاجرت & \cmark & & & \\
\rowcolor{gray!10}
۶ & انرژی (نفت، گاز، برق) & \cmark & \cmark & & \\
۷ & حمل‌ونقل بین‌استانی & \cmark & \cmark & & \\
\rowcolor{gray!10}
۸ & آموزش عالی & \cmark & \cmark & & \\
۹ & بهداشت (سیاست‌گذاری) & \cmark & \cmark & & \\
\rowcolor{gray!10}
۱۰ & مدیریت آب (حوضه‌ها) & & \cmark & & \\
۱۱ & آموزش (ابتدایی و متوسطه) & & \cmark & \cmark & \\
\rowcolor{gray!10}
۱۲ & پلیس محلی & & & \cmark & \\
۱۳ & بهداشت (اجرا) & & & \cmark & \\
\rowcolor{gray!10}
۱۴ & رسانه محلی & & & \cmark & \\
۱۵ & حمل‌ونقل درون‌شهری & & & & \cmark \\
\rowcolor{gray!10}
۱۶ & مسکن و شهرسازی & & & & \cmark \\
۱۷ & آب و فاضلاب شهری & & & & \cmark \\
\rowcolor{gray!10}
۱۸ & فضای سبز و پارک‌ها & & & & \cmark \\
\bottomrule
\end{tabular}
\end{table}

\subsection{نظام مالی فدرالی}

% نمودار جریان مالی
\begin{figure}[htbp]
\centering
\begin{tikzpicture}[
    scale=0.75,
    transform shape,
    box/.style={
        rectangle,
        rounded corners=4pt,
        minimum width=2.5cm,
        minimum height=1.2cm,
        draw=#1!70!black,
        fill=#1!15,
        text=#1!30!black,
        font=\scriptsize\bfseries,
        align=center
    },
    flow/.style={
        ->,
        >=stealth,
        thick,
        draw=#1!60
    }
]

% منابع درآمد
\node[box=green] (oil) at (-6,4) {درآمد نفت و گاز};
\node[box=green] (vat) at (-6,2) {مالیات ارزش افزوده};
\node[box=green] (income) at (-6,0) {مالیات درآمد};
\node[box=green] (customs) at (-6,-2) {گمرک و تعرفه};

% خزانه ملی
\node[box=purple, minimum width=3cm, minimum height=3cm] (treasury) at (0,1) {
    \begin{tabular}{c}
    خزانه ملی\\
    \\
    \tiny توزیع بر اساس\\
    \tiny فرمول عدالت
    \end{tabular}
};

% توزیع
\node[box=blue] (fed) at (6,4) {بودجه فدرال (۵۰٪)};
\node[box=teal] (region) at (6,2) {سهم مناطق (۲۰٪)};
\node[box=orange] (province) at (6,0) {سهم استان‌ها (۲۰٪)};
\node[box=yellow] (equal) at (6,-2) {صندوق توازن (۱۰٪)};

% جریان ورودی
\draw[flow=green] (oil) -- (treasury);
\draw[flow=green] (vat) -- (treasury);
\draw[flow=green] (income) -- (treasury);
\draw[flow=green] (customs) -- (treasury);

% جریان خروجی
\draw[flow=blue] (treasury) -- (fed);
\draw[flow=teal] (treasury) -- (region);
\draw[flow=orange] (treasury) -- (province);
\draw[flow=yellow] (treasury) -- (equal);

% صندوق توازن
\node[box=red, minimum width=2.5cm] (poor) at (10,-2) {
    \begin{tabular}{c}
    مناطق محروم\\
    \tiny (سیستان، کردستان، ...)
    \end{tabular}
};
\draw[flow=red] (equal) -- (poor);

% توضیح
\node[rectangle, draw=gray!40, fill=gray!5, font=\tiny, align=right,
      text width=4cm] at (0,-3.5) {
    \textbf{فرمول توزیع:}\\
    ۴۰٪ بر اساس جمعیت\\
    ۳۰٪ بر اساس محرومیت\\
    ۲۰٪ بر اساس مساحت\\
    ۱۰٪ بر اساس تولید
};

\end{tikzpicture}
\caption{نظام مالی فدرالی: جریان درآمدها و توزیع بودجه}
\label{fig:fiscal-federalism}
\end{figure}

%───────────────────────────────────────────────────────────────────────────────
\section{بازسازی قوه قضائیه}
\label{sec:judiciary-reform}
%───────────────────────────────────────────────────────────────────────────────

\begin{enghelabbox}
\textbf{وضعیت قوه قضائیه ایران در نقطه گذار:}
\begin{itemize}[nosep]
    \item وابستگی کامل به رهبری — فقدان استقلال
    \item قضات شرعی بدون تحصیلات حقوقی مدرن
    \item دادگاه‌های انقلاب با رویه‌های غیراستاندارد
    \item شکنجه و اعترافات اجباری رایج
    \item فساد گسترده در دستگاه قضا
    \item تبعیض جنسیتی در شهادت و دیه
\end{itemize}
بازسازی قوه قضائیه از صفر ضروری است.
\end{enghelabbox}

\subsection{ساختار جدید قوه قضائیه}

% نمودار ساختار قضایی
\begin{figure}[htbp]
\centering
\begin{tikzpicture}[
    scale=0.8,
    transform shape,
    court/.style={
        rectangle,
        rounded corners=5pt,
        minimum width=3cm,
        minimum height=1.3cm,
        draw=#1!70!black,
        fill=#1!15,
        text=#1!30!black,
        font=\scriptsize\bfseries,
        align=center
    }
]

% دادگاه قانون اساسی - بالاترین
\node[court=purple, minimum width=5cm] (const) at (0,6) {
    \begin{tabular}{c}
    دادگاه قانون اساسی\\
    \small ۱۵ قاضی — دوره ۱۲ ساله
    \end{tabular}
};

% دیوان عالی
\node[court=red, minimum width=4cm] (supreme) at (0,4) {
    \begin{tabular}{c}
    دیوان عالی کشور\\
    \small ۳۰ قاضی — مرجع تجدیدنظر نهایی
    \end{tabular}
};

% دادگاه‌های تجدیدنظر
\node[court=orange] (appeal1) at (-4,2) {دادگاه تجدیدنظر منطقه};
\node[court=orange] (appeal2) at (4,2) {دادگاه تجدیدنظر منطقه};

% دادگاه‌های بدوی
\node[court=green] (trial1) at (-6,0) {دادگاه‌های استان};
\node[court=green] (trial2) at (-2,0) {دادگاه‌های استان};
\node[court=green] (trial3) at (2,0) {دادگاه‌های استان};
\node[court=green] (trial4) at (6,0) {دادگاه‌های استان};

% دادگاه‌های تخصصی
\node[court=blue] (admin) at (-5,-2) {دیوان عدالت اداری};
\node[court=teal] (labor) at (0,-2) {دادگاه‌های کار};
\node[court=cyan] (family) at (5,-2) {دادگاه‌های خانواده};

% اتصالات
\draw[->, gray!60] (const) -- (supreme);
\draw[->, gray!60] (supreme) -- (appeal1);
\draw[->, gray!60] (supreme) -- (appeal2);
\draw[->, gray!60] (appeal1) -- (trial1);
\draw[->, gray!60] (appeal1) -- (trial2);
\draw[->, gray!60] (appeal2) -- (trial3);
\draw[->, gray!60] (appeal2) -- (trial4);

% خط جدا
\draw[dashed, gray!40] (-7,-1) -- (7,-1);
\node[font=\tiny, text=gray] at (0,-1.3) {دادگاه‌های تخصصی};

% دادستانی مستقل
\node[court=yellow, minimum width=3.5cm] (prosecutor) at (8,4) {
    \begin{tabular}{c}
    دادستانی کل\\
    \small مستقل از قوه مجریه
    \end{tabular}
};

\draw[<->, gray!60] (supreme) -- (prosecutor);

\end{tikzpicture}
\caption{ساختار پیشنهادی قوه قضائیه مستقل}
\label{fig:judiciary-structure}
\end{figure}

\subsection{اصول کلیدی قضایی}

\begin{table}[htbp]
\centering
\caption{اصول بنیادین نظام قضایی جدید}
\label{tab:judicial-principles}
\begin{tabular}{>{\columncolor{red!8}}r p{3.5cm} p{7cm}}
\toprule
\rowcolor{red!25}
\textbf{ردیف} & \textbf{اصل} & \textbf{تضمین اجرایی} \\
\midrule
۱ & استقلال قضات & انتخاب توسط شورای قضایی مستقل، نه دولت \\
\rowcolor{gray!10}
۲ & تصدی مادام‌العمر & عزل تنها به حکم دادگاه انتظامی قضات \\
۳ & برابری در برابر قانون & حذف تبعیض جنسیتی، قومی، دینی \\
\rowcolor{gray!10}
۴ & اصل برائت & متهم بی‌گناه است تا اثبات جرم \\
۵ & حق وکیل & وکیل رایگان برای ناتوانان مالی \\
\rowcolor{gray!10}
۶ & منع شکنجه & اعترافات تحت فشار باطل است \\
۷ & علنی بودن محاکمات & جز در موارد استثنایی \\
\rowcolor{gray!10}
۸ & حق تجدیدنظر & حداقل دو درجه رسیدگی \\
۹ & تناسب مجازات & ممنوعیت مجازات‌های غیرانسانی \\
\bottomrule
\end{tabular}
\end{table}

\subsection{دادگاه قانون اساسی}

\begin{olgoobox}
\textbf{الگوی دادگاه قانون اساسی آلمان (Bundesverfassungsgericht):}

معتبرترین دادگاه قانون اساسی جهان با ویژگی‌های:
\begin{itemize}[nosep]
    \item ۱۶ قاضی منتخب بوندستاگ و بوندسرات (نصف-نصف)
    \item دوره ۱۲ ساله بدون تمدید — استقلال از فشار سیاسی
    \item صلاحیت گسترده: ابطال قوانین، حل اختلاف قوا، حمایت از حقوق فردی
    \item دسترسی مستقیم شهروندان (Verfassungsbeschwerde)
    \item ۹۹٪ آلمانی‌ها به آن اعتماد دارند
\end{itemize}
\end{olgoobox}

\begin{table}[htbp]
\centering
\caption{ترکیب و صلاحیت‌های دادگاه قانون اساسی پیشنهادی}
\label{tab:constitutional-court}
\begin{tabular}{>{\columncolor{purple!8}}r p{4cm} p{6.5cm}}
\toprule
\rowcolor{purple!25}
\textbf{بند} & \textbf{موضوع} & \textbf{مشخصات} \\
\midrule
۱ & تعداد قضات & ۱۵ نفر (۳ شعبه ۵ نفره) \\
\rowcolor{gray!10}
۲ & نحوه انتخاب & ۵ توسط مجلس، ۵ توسط مجلس اقوام، ۵ توسط قضات دیوان عالی \\
۳ & شرایط & حداقل ۲۰ سال سابقه قضایی یا حقوقی \\
\rowcolor{gray!10}
۴ & دوره & ۱۲ سال، غیرقابل تمدید \\
۵ & صلاحیت‌ها & بررسی انطباق قوانین با قانون اساسی \\
\rowcolor{gray!10}
۶ & & حل اختلاف بین قوا \\
۷ & & حل اختلاف بین مرکز و مناطق \\
\rowcolor{gray!10}
۸ & & رسیدگی به شکایات حقوق بنیادین \\
۹ & & نظارت بر انتخابات \\
\rowcolor{gray!10}
۱۰ & & انحلال احزاب ضددموکراتیک \\
\bottomrule
\end{tabular}
\end{table}

\subsection{بازآموزی و جایگزینی قضات}

\begin{figure}[htbp]
\centering
\begin{tikzpicture}[
    scale=0.8,
    transform shape,
    step/.style={
        rectangle,
        rounded corners=4pt,
        minimum width=2.5cm,
        minimum height=1.4cm,
        draw=#1!70!black,
        fill=#1!15,
        text=#1!30!black,
        font=\scriptsize\bfseries,
        align=center
    }
]

% مراحل
\node[step=red] (assess) at (0,0) {
    \begin{tabular}{c}
    ارزیابی\\
    قضات موجود\\
    \tiny (۸,۰۰۰ نفر)
    \end{tabular}
};

\node[step=orange] (vetting) at (4,0) {
    \begin{tabular}{c}
    تطهیر\\
    \tiny بررسی پرونده\\
    \tiny حقوق بشری
    \end{tabular}
};

\node[step=yellow] (retrain) at (8,2) {
    \begin{tabular}{c}
    بازآموزی\\
    \tiny دوره ۶ ماهه\\
    \tiny ۴,۰۰۰ نفر
    \end{tabular}
};

\node[step=red] (dismiss) at (8,-2) {
    \begin{tabular}{c}
    اخراج/بازنشستگی\\
    \tiny ۳,۰۰۰ نفر\\
    \tiny (ناقضان حقوق بشر)
    \end{tabular}
};

\node[step=green] (new) at (12,0) {
    \begin{tabular}{c}
    استخدام جدید\\
    \tiny ۵,۰۰۰ نفر\\
    \tiny (وکلا، حقوقدانان)
    \end{tabular}
};

\node[step=blue] (deploy) at (16,0) {
    \begin{tabular}{c}
    استقرار\\
    \tiny ۱۰,۰۰۰ قاضی\\
    \tiny در سال ۵
    \end{tabular}
};

% فلش‌ها
\draw[->, thick, gray!60] (assess) -- (vetting);
\draw[->, thick, gray!60] (vetting) -- (retrain) node[midway, above, font=\tiny] {۵۰٪};
\draw[->, thick, gray!60] (vetting) -- (dismiss) node[midway, below, font=\tiny] {۴۰٪};
\draw[->, thick, gray!60] (retrain) -- (deploy);
\draw[->, thick, gray!60] (new) -- (deploy);
\draw[->, thick, gray!60] (8,0) -- (new) node[midway, above, font=\tiny] {کمبود ۵۰٪};

% جدول زمانی
\draw[thick, gray!40] (-1,-3.5) -- (17,-3.5);
\foreach \x/\y in {0/سال ۳, 4/سال ۳.۵, 8/سال ۴, 12/سال ۴.۵, 16/سال ۵} {
    \draw[gray!40] (\x,-3.4) -- (\x,-3.6);
    \node[below, font=\tiny] at (\x,-3.6) {\y};
}

\end{tikzpicture}
\caption{فرآیند بازسازی کادر قضایی}
\label{fig:judge-reform}
\end{figure}

%───────────────────────────────────────────────────────────────────────────────
\section{نهادهای نظارتی مستقل}
\label{sec:independent-institutions}
%───────────────────────────────────────────────────────────────────────────────

برای تضمین پاسخگویی و جلوگیری از فساد، نهادهای نظارتی مستقل از قوای سه‌گانه ضروری هستند.

\subsection{معماری نهادهای نظارتی}

\begin{table}[htbp]
\centering
\caption{نهادهای نظارتی مستقل پیشنهادی}
\label{tab:oversight-bodies}
\begin{tabular}{>{\columncolor{orange!8}}r p{3cm} p{3.5cm} p{2.5cm} p{2.5cm}}
\toprule
\rowcolor{orange!25}
\textbf{نهاد} & \textbf{وظیفه اصلی} & \textbf{نحوه انتخاب رئیس} & \textbf{تعداد اعضا} & \textbf{دوره} \\
\midrule
کمیسیون انتخابات & نظارت بر انتخابات & اجماع پارلمان & ۹ & ۷ سال \\
\rowcolor{gray!10}
دیوان محاسبات & حسابرسی مالی دولت & مجلس اقوام & ۱۵ & ۹ سال \\
کمیسیون حقوق بشر & حمایت از حقوق بنیادین & رئیس‌جمهور + تأیید مجلس & ۱۱ & ۶ سال \\
\rowcolor{gray!10}
سازمان بازرسی کل & مبارزه با فساد & شورای قضایی & ۱ & ۵ سال \\
بانک مرکزی مستقل & سیاست پولی & کابینه + تأیید مجلس & ۹ شورا & ۸ سال \\
\rowcolor{gray!10}
سازمان صداوسیما & رسانه عمومی بی‌طرف & هیئت امنای منتخب & ۱۳ & ۶ سال \\
کمیسیون رقابت & جلوگیری از انحصار & مجلس & ۵ & ۵ سال \\
\rowcolor{gray!10}
آمبادزمن (دادستان مردم) & شکایات شهروندان & رأی مستقیم مجلس & ۱ & ۷ سال \\
\bottomrule
\end{tabular}
\end{table}

\subsection{کمیسیون ملی انتخابات}

\begin{figure}[htbp]
\centering
\begin{tikzpicture}[
    scale=0.85,
    transform shape,
    unit/.style={
        rectangle,
        rounded corners=4pt,
        minimum width=2.5cm,
        minimum height=1.1cm,
        draw=#1!70!black,
        fill=#1!15,
        text=#1!30!black,
        font=\scriptsize\bfseries,
        align=center
    }
]

% کمیسیون مرکزی
\node[unit=purple, minimum width=6cm, minimum height=1.8cm] (central) at (0,4) {
    \begin{tabular}{c}
    کمیسیون ملی انتخابات\\
    \small ۹ عضو — اجماع پارلمانی — دوره ۷ ساله
    \end{tabular}
};

% واحدهای زیرمجموعه
\node[unit=blue] (register) at (-5,1.5) {
    \begin{tabular}{c}
    ثبت‌نام\\
    رأی‌دهندگان
    \end{tabular}
};

\node[unit=green] (candidate) at (-2,1.5) {
    \begin{tabular}{c}
    تأیید\\
    نامزدها
    \end{tabular}
};

\node[unit=orange] (campaign) at (1,1.5) {
    \begin{tabular}{c}
    نظارت بر\\
    تبلیغات
    \end{tabular}
};

\node[unit=red] (voting) at (4,1.5) {
    \begin{tabular}{c}
    برگزاری\\
    رأی‌گیری
    \end{tabular}
};

\node[unit=teal] (count) at (7,1.5) {
    \begin{tabular}{c}
    شمارش و\\
    اعلام نتایج
    \end{tabular}
};

% کمیسیون‌های استانی
\node[unit=gray, minimum width=12cm] (provincial) at (1,-0.5) {
    کمیسیون‌های استانی (۳۱ کمیسیون) — نظارت محلی
};

% اتصالات
\draw[->, gray!60] (central) -- (register);
\draw[->, gray!60] (central) -- (candidate);
\draw[->, gray!60] (central) -- (campaign);
\draw[->, gray!60] (central) -- (voting);
\draw[->, gray!60] (central) -- (count);
\draw[->, gray!60] (1,0.9) -- (provincial);

% تضمینات
\node[rectangle, draw=green!60, fill=green!10, font=\tiny, align=right,
      text width=4cm] at (-5,-2.5) {
    \textbf{تضمینات استقلال:}\\
    • بودجه مستقل (۰.۵٪ بودجه ملی)\\
    • عزل تنها با ۲/۳ پارلمان\\
    • ممنوعیت عضویت حزبی\\
    • انتشار عمومی همه تصمیمات
};

\node[rectangle, draw=blue!60, fill=blue!10, font=\tiny, align=right,
      text width=4cm] at (7,-2.5) {
    \textbf{ناظران:}\\
    • ناظران داخلی: احزاب\\
    • ناظران بین‌المللی: UN، EU\\
    • ناظران مدنی: NGOها\\
    • رسانه‌ها: دسترسی کامل
};

\end{tikzpicture}
\caption{ساختار کمیسیون ملی انتخابات}
\label{fig:election-commission}
\end{figure}

%───────────────────────────────────────────────────────────────────────────────
\section{اصلاح نظام آموزشی}
\label{sec:education-reform}
%───────────────────────────────────────────────────────────────────────────────

\begin{enghelabbox}
\textbf{وضعیت آموزش در ایران:}
\begin{itemize}[nosep]
    \item محتوای ایدئولوژیک: ۳۰٪ برنامه درسی مذهبی/سیاسی
    \item تبعیض جنسیتی: محدودیت رشته‌ها برای دختران
    \item سرکوب زبان‌های محلی: آموزش فقط به فارسی
    \item فرار مغزها: ۱۵۰,۰۰۰ نخبه سالانه مهاجرت می‌کنند
    \item کیفیت پایین: رتبه ۹۰+ در آزمون‌های بین‌المللی
    \item تاریخ‌نگاری جعلی: حذف بخش‌هایی از تاریخ معاصر
\end{itemize}
\end{enghelabbox}

\subsection{اهداف اصلاح آموزشی}

\begin{table}[htbp]
\centering
\caption{اهداف پنج‌ساله اصلاح نظام آموزشی}
\label{tab:education-goals}
\begin{tabular}{>{\columncolor{blue!8}}r p{4cm} c c p{3.5cm}}
\toprule
\rowcolor{blue!25}
\textbf{هدف} & \textbf{شاخص} & \textbf{وضعیت فعلی} & \textbf{هدف سال ۵} & \textbf{اقدام کلیدی} \\
\midrule
سکولاریزه کردن & سهم محتوای مذهبی & ۳۰٪ & ۵٪ (اختیاری) & بازنویسی کتب \\
\rowcolor{gray!10}
چندزبانگی & آموزش به زبان مادری & ۰٪ & ۱۰۰٪ مناطق قومی & معلم و کتاب محلی \\
برابری جنسیتی & رشته‌های باز برای دختران & ۷۰٪ & ۱۰۰٪ & لغو محدودیت‌ها \\
\rowcolor{gray!10}
کیفیت & رتبه PISA & — & جزو ۵۰ کشور & اصلاح برنامه درسی \\
تفکر انتقادی & ساعات بحث/پروژه & ۵٪ & ۳۰٪ & آموزش معلمان \\
\rowcolor{gray!10}
تاریخ واقعی & پوشش تاریخ معاصر & سانسورشده & کامل و بی‌طرف & کتب جدید \\
\bottomrule
\end{tabular}
\end{table}

\subsection{ساختار جدید آموزش}

\begin{figure}[htbp]
\centering
\begin{tikzpicture}[
    scale=0.8,
    transform shape,
    level/.style={
        rectangle,
        rounded corners=4pt,
        minimum height=1.3cm,
        draw=#1!70!black,
        fill=#1!15,
        text=#1!30!black,
        font=\scriptsize\bfseries,
        align=center
    }
]

% مقاطع تحصیلی
\node[level=red, minimum width=3cm] (preschool) at (0,0) {
    \begin{tabular}{c}
    پیش‌دبستانی\\
    \tiny سن ۴-۶
    \end{tabular}
};

\node[level=orange, minimum width=4cm] (primary) at (4,0) {
    \begin{tabular}{c}
    دبستان\\
    \tiny سن ۶-۱۲ | ۶ سال
    \end{tabular}
};

\node[level=yellow, minimum width=3cm] (middle) at (8,0) {
    \begin{tabular}{c}
    راهنمایی\\
    \tiny سن ۱۲-۱۵ | ۳ سال
    \end{tabular}
};

\node[level=green, minimum width=3cm] (high) at (12,0) {
    \begin{tabular}{c}
    دبیرستان\\
    \tiny سن ۱۵-۱۸ | ۳ سال
    \end{tabular}
};

\node[level=blue, minimum width=3cm] (uni) at (16,0) {
    \begin{tabular}{c}
    دانشگاه\\
    \tiny سن ۱۸+
    \end{tabular}
};

% فلش‌ها
\draw[->, thick, gray!60] (preschool) -- (primary);
\draw[->, thick, gray!60] (primary) -- (middle);
\draw[->, thick, gray!60] (middle) -- (high);
\draw[->, thick, gray!60] (high) -- (uni);

% شاخه‌های دبیرستان
\node[level=green, minimum width=2cm] (academic) at (10.5,-2) {\tiny نظری};
\node[level=green, minimum width=2cm] (vocational) at (13.5,-2) {\tiny فنی-حرفه‌ای};
\draw[->, gray!40] (high) -- (academic);
\draw[->, gray!40] (high) -- (vocational);

% زبان آموزش
\node[rectangle, draw=purple!60, fill=purple!10, font=\tiny, align=center,
      minimum width=14cm] at (8,2) {
    زبان آموزش: فارسی + زبان منطقه‌ای (در مناطق قومی) | انگلیسی اجباری از سال سوم
};

% ویژگی‌ها
\node[rectangle, draw=gray!40, fill=gray!5, font=\tiny, align=right,
      text width=5cm] at (4,-3) {
    \textbf{ویژگی‌های کلیدی:}\\
    • رایگان و اجباری تا پایان راهنمایی\\
    • کتب درسی بازنگری‌شده\\
    • معلمان بازآموزی‌دیده\\
    • فناوری در کلاس
};

\node[rectangle, draw=gray!40, fill=gray!5, font=\tiny, align=right,
      text width=5cm] at (12,-3) {
    \textbf{محتوای جدید:}\\
    • تربیت شهروندی دموکراتیک\\
    • حقوق بشر و برابری\\
    • تفکر انتقادی\\
    • محیط زیست
};

\end{tikzpicture}
\caption{ساختار جدید نظام آموزشی}
\label{fig:education-structure}
\end{figure}

\subsection{آموزش چندزبانه}

\begin{olgoobox}
\textbf{الگوی سوئیس: چهار زبان رسمی}

سوئیس با ۴ زبان رسمی (آلمانی ۶۳٪، فرانسوی ۲۳٪، ایتالیایی ۸٪، رومانش ۰.۵٪) نشان داده که چندزبانگی تهدید نیست:
\begin{itemize}[nosep]
    \item آموزش ابتدایی به زبان مادری
    \item زبان دوم ملی از سال سوم
    \item انگلیسی از سال پنجم
    \item هویت ملی سوئیسی قوی‌تر از هویت‌های زبانی
\end{itemize}
\end{olgoobox}

\begin{table}[htbp]
\centering
\caption{برنامه آموزش چندزبانه در مناطق قومی ایران}
\label{tab:multilingual-education}
\begin{tabular}{>{\columncolor{teal!8}}r p{2.5cm} p{2.5cm} p{2.5cm} p{2.5cm} p{2.5cm}}
\toprule
\rowcolor{teal!25}
\textbf{مقطع} & \textbf{زبان اول} & \textbf{زبان دوم} & \textbf{زبان سوم} & \textbf{سهم محتوا} & \textbf{یادداشت} \\
\midrule
پیش‌دبستانی & زبان مادری ۱۰۰٪ & — & — & کامل محلی & پایه زبانی \\
\rowcolor{gray!10}
دبستان ۱-۳ & زبان مادری ۸۰٪ & فارسی ۲۰٪ & — & عمدتاً محلی & معرفی فارسی \\
دبستان ۴-۶ & زبان مادری ۵۰٪ & فارسی ۴۰٪ & انگلیسی ۱۰٪ & متوازن & دوزبانگی \\
\rowcolor{gray!10}
راهنمایی & زبان مادری ۴۰٪ & فارسی ۵۰٪ & انگلیسی ۱۰٪ & عمدتاً فارسی & آماده‌سازی \\
دبیرستان & انتخابی & فارسی ۶۰٪ & انگلیسی ۲۰٪ & ملی + محلی & تخصص \\
\bottomrule
\end{tabular}
\end{table}

%───────────────────────────────────────────────────────────────────────────────
\section{تقویت جامعه مدنی}
\label{sec:civil-society}
%───────────────────────────────────────────────────────────────────────────────

\begin{naghlbox}
«دموکراسی بدون جامعه مدنی قوی مانند ساختمان بدون پی است — زیبا اما ناپایدار.»
\sourceline{رابرت پاتنام، دانشمند سیاسی}
\end{naghlbox}

\subsection{وضعیت جامعه مدنی در نقطه گذار}

\begin{table}[htbp]
\centering
\caption{وضعیت جامعه مدنی ایران: قبل و هدف}
\label{tab:civil-society-status}
\begin{tabular}{>{\columncolor{green!8}}r p{4cm} c c}
\toprule
\rowcolor{green!25}
\textbf{شاخص} & \textbf{توضیح} & \textbf{وضعیت فعلی} & \textbf{هدف سال ۵} \\
\midrule
NGOهای ثبت‌شده & تعداد سازمان‌های مردم‌نهاد & ۲۰,۰۰۰ (اکثراً غیرفعال) & ۱۰۰,۰۰۰ فعال \\
\rowcolor{gray!10}
آزادی تشکل & امتیاز Freedom House & ۲/۴۰ & ۳۰/۴۰ \\
اتحادیه‌های کارگری مستقل & تعداد & ۰ & ۵۰۰+ \\
\rowcolor{gray!10}
احزاب سیاسی فعال & با حضور واقعی & ۳-۴ & ۵۰+ \\
رسانه‌های مستقل & تعداد & ۰ (داخل) & ۵۰۰+ \\
\rowcolor{gray!10}
مشارکت داوطلبانه & درصد جمعیت & ۵٪ & ۲۵٪ \\
\bottomrule
\end{tabular}
\end{table}

\subsection{چارچوب حمایت از جامعه مدنی}

%═══════════════════════════════════════════════════════════════════════════════
% ادامه فصل ۹: فاز ۲ — نهادسازی (سال ۳-۵)
% فایل: chapters/ch09-phase2.tex (ادامه)
%═══════════════════════════════════════════════════════════════════════════════

% ادامه نمودار چارچوب حمایت از جامعه مدنی...

\begin{figure}[htbp]
\centering
\begin{tikzpicture}[
    scale=0.85,
    transform shape,
    support/.style={
        rectangle,
        rounded corners=5pt,
        minimum width=3cm,
        minimum height=1.5cm,
        draw=#1!70!black,
        fill=#1!15,
        text=#1!30!black,
        font=\scriptsize\bfseries,
        align=center
    }
]

% مرکز - جامعه مدنی
\node[ellipse, draw=purple!70, fill=purple!10, minimum width=3.5cm, 
      minimum height=2.5cm, font=\small\bfseries, text=purple!70] (cs) at (0,0) {
    \begin{tabular}{c}
    جامعه\\
    مدنی
    \end{tabular}
};

% ارکان حمایتی
\node[support=blue] (legal) at (-5,3) {
    \begin{tabular}{c}
    چارچوب قانونی\\
    \tiny آزادی تشکل و اجتماع
    \end{tabular}
};

\node[support=green] (finance) at (0,4) {
    \begin{tabular}{c}
    تأمین مالی\\
    \tiny صندوق حمایت از NGO
    \end{tabular}
};

\node[support=orange] (capacity) at (5,3) {
    \begin{tabular}{c}
    ظرفیت‌سازی\\
    \tiny آموزش و مشاوره
    \end{tabular}
};

\node[support=red] (media) at (-5,-3) {
    \begin{tabular}{c}
    دسترسی رسانه‌ای\\
    \tiny رسانه عمومی و اینترنت
    \end{tabular}
};

\node[support=teal] (consult) at (0,-4) {
    \begin{tabular}{c}
    مشارکت در تصمیم‌گیری\\
    \tiny مشورت اجباری دولت
    \end{tabular}
};

\node[support=yellow] (protect) at (5,-3) {
    \begin{tabular}{c}
    حمایت حقوقی\\
    \tiny از فعالان مدنی
    \end{tabular}
};

% اتصالات
\draw[->, thick, blue!60] (legal) -- (cs);
\draw[->, thick, green!60] (finance) -- (cs);
\draw[->, thick, orange!60] (capacity) -- (cs);
\draw[->, thick, red!60] (media) -- (cs);
\draw[->, thick, teal!60] (consult) -- (cs);
\draw[->, thick, yellow!60] (protect) -- (cs);

% اجزای جامعه مدنی (دایره بیرونی)
\node[font=\tiny, text=gray] at (-2.5,1.5) {NGOها};
\node[font=\tiny, text=gray] at (2.5,1.5) {احزاب};
\node[font=\tiny, text=gray] at (-2.5,-1.5) {اتحادیه‌ها};
\node[font=\tiny, text=gray] at (2.5,-1.5) {رسانه‌ها};

\end{tikzpicture}
\caption{شش رکن حمایت از جامعه مدنی}
\label{fig:civil-society-support}
\end{figure}

\subsection{قانون جامع تشکل‌ها و NGOها}

\begin{table}[htbp]
\centering
\caption{مقایسه قوانین NGO: وضعیت فعلی و پیشنهادی}
\label{tab:ngo-law-comparison}
\begin{tabular}{>{\columncolor{green!8}}r p{4cm} p{4cm} p{4cm}}
\toprule
\rowcolor{green!25}
\textbf{موضوع} & \textbf{قانون فعلی} & \textbf{قانون پیشنهادی} & \textbf{استاندارد بین‌المللی} \\
\midrule
ثبت‌نام & نیاز به مجوز وزارت کشور & اعلامی (notification) & UN Guidelines \\
\rowcolor{gray!10}
زمان ثبت & ۶ ماه تا ۲ سال & ۳۰ روز & EU: 15-30 روز \\
حق اعتراض به رد & محدود & دادگاه اداری & کنوانسیون اروپایی \\
\rowcolor{gray!10}
تأمین مالی خارجی & ممنوع & مجاز با شفافیت & FATF compliant \\
فعالیت سیاسی & ممنوع & مجاز (جز احزاب) & ICCPR ماده ۲۲ \\
\rowcolor{gray!10}
انحلال & به دستور دولت & فقط به حکم دادگاه & Venice Commission \\
معافیت مالیاتی & محدود & گسترده برای خیریه‌ها & OECD standards \\
\bottomrule
\end{tabular}
\end{table}

\subsection{صندوق ملی حمایت از جامعه مدنی}

\begin{olgoobox}
\textbf{الگوی صندوق ملی دموکراسی آمریکا (NED):}

NED سالانه ۳۰۰ میلیون دلار به NGOها در سراسر جهان کمک می‌کند. ویژگی‌های کلیدی:
\begin{itemize}[nosep]
    \item بودجه دولتی اما مدیریت مستقل
    \item هیئت‌مدیره دوحزبی (جمهوری‌خواه + دموکرات)
    \item شفافیت کامل در اعطای کمک‌ها
    \item ارزیابی مستقل اثربخشی
\end{itemize}

\textbf{پیشنهاد برای ایران:} صندوق ملی توسعه جامعه مدنی با بودجه سالانه ۵۰۰ میلیارد تومان
\end{olgoobox}

\begin{table}[htbp]
\centering
\caption{ساختار صندوق ملی حمایت از جامعه مدنی}
\label{tab:civil-society-fund}
\begin{tabular}{>{\columncolor{blue!8}}r p{4cm} p{7cm}}
\toprule
\rowcolor{blue!25}
\textbf{بند} & \textbf{موضوع} & \textbf{مشخصات} \\
\midrule
۱ & بودجه سالانه & ۵۰۰ میلیارد تومان (۰.۱٪ بودجه ملی) \\
\rowcolor{gray!10}
۲ & منبع & بودجه عمومی + کمک‌های بین‌المللی \\
۳ & هیئت‌مدیره & ۱۱ نفر: ۵ نماینده مجلس، ۳ نماینده NGO، ۳ کارشناس مستقل \\
\rowcolor{gray!10}
۴ & معیار تخصیص & شفافیت، اثربخشی، پوشش جغرافیایی \\
۵ & سقف کمک & حداکثر ۵ میلیارد تومان به هر NGO در سال \\
\rowcolor{gray!10}
۶ & حوزه‌های اولویت & حقوق بشر، محیط زیست، زنان، اقوام، جوانان \\
۷ & نظارت & حسابرسی سالانه توسط دیوان محاسبات \\
\rowcolor{gray!10}
۸ & شفافیت & انتشار عمومی همه کمک‌ها \\
\bottomrule
\end{tabular}
\end{table}

%───────────────────────────────────────────────────────────────────────────────
\section{اصلاح بوروکراسی دولتی}
\label{sec:bureaucracy-reform}
%───────────────────────────────────────────────────────────────────────────────

\begin{enghelabbox}
\textbf{وضعیت بوروکراسی ایران:}
\begin{itemize}[nosep]
    \item ۲.۵ میلیون کارمند دولت (بدون نظامی)
    \item ۵۰٪ بودجه صرف حقوق کارکنان
    \item ۳۰٪ پست‌ها موازی یا غیرضروری
    \item انتصابات سیاسی-ایدئولوژیک به جای شایستگی
    \item فساد سیستماتیک در استخدام و ارتقا
    \item بهره‌وری پایین: ۳۰٪ OECD
\end{itemize}
\end{enghelabbox}

\subsection{اصول اصلاح اداری}

\begin{table}[htbp]
\centering
\caption{اصول و اقدامات اصلاح بوروکراسی}
\label{tab:bureaucracy-principles}
\begin{tabular}{>{\columncolor{orange!8}}r p{3cm} p{4.5cm} p{4cm}}
\toprule
\rowcolor{orange!25}
\textbf{اصل} & \textbf{مشکل فعلی} & \textbf{اقدام اصلاحی} & \textbf{شاخص موفقیت} \\
\midrule
شایسته‌سالاری & انتصاب سیاسی & آزمون‌های استخدامی شفاف & ۱۰۰٪ استخدام رقابتی \\
\rowcolor{gray!10}
کوچک‌سازی & تورم نیرو & بازنشستگی پیش‌از‌موعد داوطلبانه & کاهش ۲۰٪ نیرو \\
تمرکززدایی & همه تصمیمات در تهران & واگذاری به استان‌ها & ۵۰٪ تصمیمات محلی \\
\rowcolor{gray!10}
دیجیتال‌سازی & کاغذبازی & دولت الکترونیک & ۸۰٪ خدمات آنلاین \\
شفافیت & پنهان‌کاری & دسترسی آزاد به اطلاعات & قانون FOI اجرایی \\
\rowcolor{gray!10}
پاسخگویی & فقدان نظارت & ارزیابی عملکرد سالانه & همه مدیران ارزیابی‌شده \\
\bottomrule
\end{tabular}
\end{table}

\subsection{دولت الکترونیک}

\begin{figure}[htbp]
\centering
\begin{tikzpicture}[
    scale=0.8,
    transform shape,
    service/.style={
        rectangle,
        rounded corners=4pt,
        minimum width=2.5cm,
        minimum height=1cm,
        draw=#1!70!black,
        fill=#1!15,
        text=#1!30!black,
        font=\scriptsize\bfseries,
        align=center
    }
]

% پلتفرم مرکزی
\node[rectangle, rounded corners=8pt, draw=purple!70, fill=purple!10,
      minimum width=5cm, minimum height=2cm, font=\small\bfseries,
      text=purple!70, align=center] (portal) at (0,0) {
    \begin{tabular}{c}
    درگاه ملی خدمات دولت\\
    \small iran.gov.ir
    \end{tabular}
};

% دسته‌های خدمات
\node[service=blue] (id) at (-6,3) {هویت و مدارک};
\node[service=green] (tax) at (-3,3) {مالیات و گمرک};
\node[service=orange] (health) at (0,3) {بهداشت و درمان};
\node[service=red] (edu) at (3,3) {آموزش};
\node[service=teal] (justice) at (6,3) {قضایی};

\node[service=yellow] (business) at (-6,-3) {مجوز کسب‌وکار};
\node[service=cyan] (property) at (-3,-3) {املاک و اسناد};
\node[service=pink] (social) at (0,-3) {تأمین اجتماعی};
\node[service=lime] (transport) at (3,-3) {حمل‌ونقل};
\node[service=violet] (utility) at (6,-3) {قبوض و خدمات};

% اتصالات
\foreach \s in {id,tax,health,edu,justice,business,property,social,transport,utility} {
    \draw[->, gray!50] (\s) -- (portal);
}

% زیرساخت
\node[rectangle, draw=gray!60, fill=gray!10, minimum width=14cm,
      minimum height=1cm, font=\scriptsize, align=center] at (0,-5) {
    زیرساخت: شبکه ملی اطلاعات | پایگاه داده یکپارچه | احراز هویت دیجیتال | امنیت سایبری
};

% آمار هدف
\node[rectangle, draw=green!60, fill=green!10, font=\tiny, align=right,
      text width=3.5cm] at (-8,0) {
    \textbf{اهداف سال ۵:}\\
    • ۸۰٪ خدمات آنلاین\\
    • ۶۰٪ شهروندان فعال\\
    • کاهش ۵۰٪ مراجعه حضوری\\
    • صرفه‌جویی ۱۰۰ هزار میلیارد
};

\end{tikzpicture}
\caption{معماری دولت الکترونیک}
\label{fig:e-government}
\end{figure}

\subsection{نظام جدید استخدام و ارتقا}

\begin{table}[htbp]
\centering
\caption{نظام جدید مدیریت منابع انسانی دولت}
\label{tab:hr-system}
\begin{tabular}{>{\columncolor{blue!8}}r p{3cm} p{4cm} p{4.5cm}}
\toprule
\rowcolor{blue!25}
\textbf{مرحله} & \textbf{روش فعلی} & \textbf{روش جدید} & \textbf{سازوکار} \\
\midrule
استخدام & معرفی‌نامه + مصاحبه & آزمون سراسری + مصاحبه & سازمان سنجش مستقل \\
\rowcolor{gray!10}
انتصاب مدیران & سیاسی-ایدئولوژیک & شایستگی + تجربه & کمیته انتصابات مستقل \\
ارزیابی عملکرد & صوری یا فاقد & سالانه با شاخص‌های کمّی & سیستم ۳۶۰ درجه \\
\rowcolor{gray!10}
ارتقا & ارشدیت + رابطه & عملکرد + آموزش & امتیازبندی شفاف \\
حقوق و مزایا & نابرابر، رانتی & یکسان، بر اساس رتبه & جدول حقوق واحد \\
\rowcolor{gray!10}
آموزش ضمن خدمت & محدود | ایدئولوژیک & اجباری | حرفه‌ای & ۴۰ ساعت سالانه \\
\bottomrule
\end{tabular}
\end{table}

%───────────────────────────────────────────────────────────────────────────────
\section{رسانه‌های عمومی مستقل}
\label{sec:public-media}
%───────────────────────────────────────────────────────────────────────────────

\begin{naghlbox}
«رسانه عمومی مستقل ستون چهارم دموکراسی است. بدون اطلاعات آزاد، انتخاب آزاد بی‌معناست.»
\sourceline{اعلامیه UNESCO درباره رسانه‌ها}
\end{naghlbox}

\subsection{تبدیل صداوسیما به رسانه عمومی مستقل}

\begin{table}[htbp]
\centering
\caption{مقایسه صداوسیمای فعلی با مدل پیشنهادی}
\label{tab:media-comparison}
\begin{tabular}{>{\columncolor{purple!8}}r p{4cm} p{4cm} p{3.5cm}}
\toprule
\rowcolor{purple!25}
\textbf{ویژگی} & \textbf{صداوسیمای فعلی} & \textbf{مدل پیشنهادی} & \textbf{الگوی موفق} \\
\midrule
مالکیت & دولتی (زیر نظر رهبری) & عمومی (مستقل) & BBC بریتانیا \\
\rowcolor{gray!10}
انتصاب رئیس & رهبر & هیئت امنای مستقل & ARD آلمان \\
هیئت‌امنا & — & ۱۳ نفر منتخب متنوع & NHK ژاپن \\
\rowcolor{gray!10}
بودجه & دولتی + تبلیغات & حق اشتراک + دولتی محدود & SVT سوئد \\
محتوا & تبلیغات حکومتی & اطلاع‌رسانی بی‌طرف & PBS آمریکا \\
\rowcolor{gray!10}
دسترسی احزاب & صفر (جز حکومتی‌ها) & برابر در انتخابات & همه کشورهای OECD \\
\bottomrule
\end{tabular}
\end{table}

\subsection{ساختار سازمان رسانه عمومی ایران (رعا)}

\begin{figure}[htbp]
\centering
\begin{tikzpicture}[
    scale=0.8,
    transform shape,
    organ/.style={
        rectangle,
        rounded corners=5pt,
        minimum width=3cm,
        minimum height=1.3cm,
        draw=#1!70!black,
        fill=#1!15,
        text=#1!30!black,
        font=\scriptsize\bfseries,
        align=center
    }
]

% هیئت امنا
\node[organ=purple, minimum width=8cm, minimum height=1.8cm] (board) at (0,5) {
    \begin{tabular}{c}
    هیئت امنا (۱۳ نفر)\\
    \small ۴ نماینده مجلس | ۳ روزنامه‌نگار | ۲ دانشگاهی | ۲ نماینده اقوام | ۲ جامعه مدنی
    \end{tabular}
};

% مدیرعامل
\node[organ=blue, minimum width=4cm] (ceo) at (0,2.5) {
    \begin{tabular}{c}
    مدیرعامل\\
    \small منتخب هیئت امنا | دوره ۵ ساله
    \end{tabular}
};

% شبکه‌ها
\node[organ=green] (tv1) at (-6,0) {تلویزیون ۱ (عمومی)};
\node[organ=green] (tv2) at (-3,0) {تلویزیون ۲ (فرهنگ)};
\node[organ=orange] (news) at (0,0) {شبکه خبر (۲۴/۷)};
\node[organ=teal] (regional) at (3,0) {شبکه‌های استانی};
\node[organ=cyan] (radio) at (6,0) {رادیو ملی};

% شبکه‌های قومی
\node[organ=yellow, minimum width=10cm] (ethnic) at (0,-2) {
    \begin{tabular}{c}
    شبکه‌های زبان‌های محلی\\
    \small آذری | کردی | عربی | بلوچی | ترکمنی | گیلکی
    \end{tabular}
};

% دیجیتال
\node[organ=red] (digital) at (0,-4) {
    \begin{tabular}{c}
    پلتفرم دیجیتال\\
    \small استریم | پادکست | شبکه‌های اجتماعی
    \end{tabular}
};

% اتصالات
\draw[->, gray!60] (board) -- (ceo);
\draw[->, gray!60] (ceo) -- (tv1);
\draw[->, gray!60] (ceo) -- (tv2);
\draw[->, gray!60] (ceo) -- (news);
\draw[->, gray!60] (ceo) -- (regional);
\draw[->, gray!60] (ceo) -- (radio);
\draw[->, gray!60] (ceo) -- (ethnic);
\draw[->, gray!60] (ceo) -- (digital);

% بودجه
\node[rectangle, draw=gray!40, fill=gray!5, font=\tiny, align=right,
      text width=4cm] at (8,2.5) {
    \textbf{منابع مالی:}\\
    • حق اشتراک: ماهی ۵۰,۰۰۰ تومان\\
    • یارانه دولتی: ۳۰٪\\
    • تبلیغات محدود: ۱۰٪\\
    • تولید محتوا: ۱۰٪
};

\end{tikzpicture}
\caption{ساختار سازمان رسانه عمومی ایران}
\label{fig:public-media-structure}
\end{figure}

%───────────────────────────────────────────────────────────────────────────────
\section{تداوم عدالت انتقالی}
\label{sec:tj-phase2}
%───────────────────────────────────────────────────────────────────────────────

در فاز دوم، عدالت انتقالی از مرحله فوری به مرحله نهادی وارد می‌شود.

\subsection{اقدامات عدالت انتقالی سال‌های ۳-۵}

\begin{table}[htbp]
\centering
\caption{تقویم عدالت انتقالی فاز دوم}
\label{tab:tj-phase2-calendar}
\begin{tabular}{>{\columncolor{red!8}}r p{3cm} p{5.5cm} p{3cm}}
\toprule
\rowcolor{red!25}
\textbf{سال} & \textbf{محور} & \textbf{اقدامات} & \textbf{خروجی} \\
\midrule
۳ & محاکمات & آغاز محاکمه فرماندهان سرکوب & ۵۰ محاکمه \\
\rowcolor{gray!10}
۳ & حقیقت‌یابی & ادامه جلسات استماع عمومی & ۱۰,۰۰۰ شهادت \\
۳-۴ & جبران خسارت & برنامه جامع غرامت به قربانیان & ۵۰۰,۰۰۰ خانواده \\
\rowcolor{gray!10}
۴ & اصلاح نهادی & تطهیر نهادهای امنیتی و قضایی & گزارش تطهیر \\
۴-۵ & تاریخ‌نگاری & ثبت رسمی جنایات در تاریخ & کتب درسی جدید \\
\rowcolor{gray!10}
۵ & یادبود & ساخت موزه و بناهای یادبود & موزه ملی قربانیان \\
۵ & گزارش نهایی & انتشار گزارش کمیسیون حقیقت & سند ملی \\
\bottomrule
\end{tabular}
\end{table}

\subsection{دادگاه ویژه جنایات دوره استبداد}

\begin{olgoobox}
\textbf{الگوی دادگاه‌های ویژه:}

\textbf{آرژانتین:} محاکمه ژنرال‌های دیکتاتوری (۱۹۸۵) — نماد عدالت در آمریکای لاتین

\textbf{آلمان:} دادگاه‌های نورنبرگ (۱۹۴۵-۴۶) — پایه حقوق بین‌الملل کیفری

\textbf{رواندا:} دادگاه‌های گاچاچا — عدالت ترمیمی در سطح محلی

\textbf{پیشنهاد برای ایران:} ترکیبی از محاکمات رسمی (برای جنایات بزرگ) و کمیسیون حقیقت (برای موارد دیگر)
\end{olgoobox}

\begin{table}[htbp]
\centering
\caption{صلاحیت و ساختار دادگاه ویژه}
\label{tab:special-tribunal}
\begin{tabular}{>{\columncolor{purple!8}}r p{4cm} p{7cm}}
\toprule
\rowcolor{purple!25}
\textbf{بند} & \textbf{موضوع} & \textbf{مشخصات} \\
\midrule
۱ & صلاحیت موضوعی & جنایت علیه بشریت، شکنجه، اعدام‌های فراقضایی، ناپدیدسازی \\
\rowcolor{gray!10}
۲ & صلاحیت زمانی & ۱۳۵۷ تا روز گذار \\
۳ & ترکیب قضات & ۹ قاضی: ۶ ایرانی + ۳ بین‌المللی \\
\rowcolor{gray!10}
۴ & دادستان & دادستان مستقل منتخب مجلس \\
۵ & حقوق متهمان & استانداردهای بین‌المللی (وکیل، علنی بودن، تجدیدنظر) \\
\rowcolor{gray!10}
۶ & مجازات‌ها & حبس (حداکثر ابد) — ممنوعیت اعدام \\
۷ & مرور زمان & جنایات علیه بشریت: بدون مرور زمان \\
\rowcolor{gray!10}
۸ & حمایت از شهود & برنامه حفاظت از شاهدان \\
\bottomrule
\end{tabular}
\end{table}

%───────────────────────────────────────────────────────────────────────────────
\section{شاخص‌های موفقیت فاز دوم}
\label{sec:phase2-kpis}
%───────────────────────────────────────────────────────────────────────────────

\begin{table}[htbp]
\centering
\caption{شاخص‌های کلیدی موفقیت (KPI) فاز دوم}
\label{tab:phase2-kpis}
\begin{tabular}{>{\columncolor{blue!8}}r p{4.5cm} c c c}
\toprule
\rowcolor{blue!25}
\textbf{کد} & \textbf{شاخص} & \textbf{پایان سال ۳} & \textbf{پایان سال ۴} & \textbf{پایان سال ۵} \\
\midrule
P01 & استقرار ساختار فدرالی & ۵۰٪ & ۸۰٪ & ۱۰۰٪ \\
\rowcolor{gray!10}
P02 & استقلال قضات (نظرسنجی) & ۴۰٪ & ۵۵٪ & ۷۰٪ \\
P03 & پوشش آموزش چندزبانه & ۲۰٪ مناطق & ۶۰٪ & ۱۰۰٪ \\
\rowcolor{gray!10}
P04 & NGOهای فعال ثبت‌شده & ۴۰,۰۰۰ & ۷۰,۰۰۰ & ۱۰۰,۰۰۰ \\
P05 & خدمات دولت الکترونیک & ۴۰٪ & ۶۰٪ & ۸۰٪ \\
\rowcolor{gray!10}
P06 & شاخص آزادی مطبوعات & رتبه ۱۲۰ & رتبه ۱۰۰ & رتبه ۸۰ \\
P07 & اعتماد به نهادها (نظرسنجی) & ۴۵٪ & ۵۵٪ & ۶۵٪ \\
\rowcolor{gray!10}
P08 & تطهیر نیروهای امنیتی & ۶۰٪ & ۸۵٪ & ۱۰۰٪ \\
P09 & محاکمات عدالت انتقالی & ۲۰ & ۵۰ & ۱۰۰ \\
\rowcolor{gray!10}
P10 & رضایت اقوام (نظرسنجی) & ۵۰٪ & ۶۰٪ & ۷۰٪ \\
\bottomrule
\end{tabular}
\end{table}

%───────────────────────────────────────────────────────────────────────────────
\section{چالش‌ها و ریسک‌های فاز دوم}
\label{sec:phase2-risks}
%───────────────────────────────────────────────────────────────────────────────

\begin{figure}[htbp]
\centering
\begin{tikzpicture}
\begin{axis}[
    width=12cm,
    height=9cm,
    xlabel={احتمال وقوع},
    ylabel={شدت تأثیر},
    xmin=0, xmax=100,
    ymin=0, ymax=100,
    xtick={25,50,75},
    xticklabels={کم, متوسط, بالا},
    ytick={25,50,75},
    yticklabels={کم, متوسط, بالا},
    grid=major,
    grid style={dashed, gray!30},
]

% مناطق رنگی
\fill[green!15, opacity=0.5] (0,0) rectangle (35,35);
\fill[yellow!15, opacity=0.5] (0,35) rectangle (35,100);
\fill[yellow!15, opacity=0.5] (35,0) rectangle (100,35);
\fill[orange!15, opacity=0.5] (35,35) rectangle (65,65);
\fill[red!15, opacity=0.5] (35,65) rectangle (100,100);
\fill[red!15, opacity=0.5] (65,35) rectangle (100,65);

% ریسک‌ها
\node[circle, fill=orange!70, minimum size=0.5cm, font=\tiny\bfseries, text=white] 
    at (axis cs:60,70) {۱};
\node[circle, fill=red!70, minimum size=0.5cm, font=\tiny\bfseries, text=white] 
    at (axis cs:45,75) {۲};
\node[circle, fill=yellow!70, minimum size=0.5cm, font=\tiny\bfseries, text=black] 
    at (axis cs:70,50) {۳};
\node[circle, fill=orange!60, minimum size=0.5cm, font=\tiny\bfseries, text=white] 
    at (axis cs:55,60) {۴};
\node[circle, fill=yellow!60, minimum size=0.5cm, font=\tiny\bfseries, text=black] 
    at (axis cs:40,45) {۵};

% راهنما
\node[rectangle, draw=gray!40, fill=white, font=\tiny, align=right, 
      text width=4cm] at (axis cs:82,25) {
    \textbf{ریسک‌ها:}\\
    ۱. مقاومت نخبگان قدیم\\
    ۲. تنش‌های قومی\\
    ۳. کندی اصلاحات اقتصادی\\
    ۴. نارضایتی از سرعت تغییر\\
    ۵. فشار نیروهای افراطی
};

\end{axis}
\end{tikzpicture}
\caption{ماتریس ریسک فاز دوم}
\label{fig:risk-matrix-phase2}
\end{figure}

\begin{enghelabbox}
\textbf{ریسک اصلی فاز دوم: «دام انتظارات»}

مردم پس از انقلاب انتظار بهبود سریع دارند. اما نهادسازی زمان‌بر است. اگر بهبود ملموس دیده نشود، سرخوردگی می‌تواند به:
\begin{itemize}[nosep]
    \item رأی به پوپولیست‌ها
    \item اعتراضات ضد دولت جدید
    \item نوستالژی رژیم قبلی
    \item افزایش مهاجرت
\end{itemize}
منجر شود.

\textbf{راه‌حل:} ادامه استراتژی «آبادانی ملموس» همراه با نهادسازی بلندمدت
\end{enghelabbox}

%───────────────────────────────────────────────────────────────────────────────
\section{جمع‌بندی فصل}
\label{sec:phase2-conclusion}
%───────────────────────────────────────────────────────────────────────────────

\begin{kholasebox}
\textbf{خلاصه فصل ۹:}
\begin{enumerate}
    \item \textbf{فدرالیسم همبسته} با ۵ منطقه خودمختار و ۳۱ استان، پاسخ به تنوع قومی ایران است
    \item \textbf{قوه قضائیه مستقل} با دادگاه قانون اساسی ۱۵ نفره، ستون حاکمیت قانون است
    \item \textbf{نهادهای نظارتی مستقل} (۸ نهاد) تضمین‌کننده پاسخگویی و مبارزه با فساد هستند
    \item \textbf{اصلاح نظام آموزشی} شامل سکولاریزاسیون، چندزبانگی، و تفکر انتقادی است
    \item \textbf{جامعه مدنی قوی} با ۱۰۰,۰۰۰ NGO فعال، پایه دموکراسی مشارکتی است
    \item \textbf{دولت الکترونیک} با ۸۰٪ خدمات آنلاین، کارایی و شفافیت را افزایش می‌دهد
    \item \textbf{رسانه عمومی مستقل} جایگزین صداوسیمای حکومتی می‌شود
    \item \textbf{عدالت انتقالی} در این فاز به محاکمات و جبران خسارت می‌رسد
    \item ریسک اصلی «دام انتظارات» است که با آبادانی ملموس مدیریت می‌شود
\end{enumerate}
\end{kholasebox}

% نمودار جمع‌بندی
\begin{figure}[htbp]
\centering
\begin{tikzpicture}[
    scale=0.7,
    transform shape,
    pillar/.style={
        rectangle,
        rounded corners=5pt,
        minimum width=2cm,
        minimum height=4cm,
        draw=#1!70!black,
        fill=#1!15,
        text=#1!30!black,
        font=\scriptsize\bfseries,
        align=center
    }
]

% پایه
\node[rectangle, draw=gray!70, fill=gray!20, minimum width=16cm, minimum height=1cm,
      font=\small\bfseries] (base) at (0,0) {قانون اساسی جدید (از فاز ۱)};

% ستون‌ها
\node[pillar=red] at (-6,3) {
    \begin{tabular}{c}
    \\
    فدرالیسم\\
    \\
    \tiny ۵ منطقه\\
    \tiny ۳۱ استان
    \end{tabular}
};

\node[pillar=orange] at (-3.5,3) {
    \begin{tabular}{c}
    \\
    قضای\\
    مستقل\\
    \tiny دادگاه\\
    \tiny قانون اساسی
    \end{tabular}
};

\node[pillar=yellow] at (-1,3) {
    \begin{tabular}{c}
    \\
    نهادهای\\
    نظارتی\\
    \tiny ۸ نهاد\\
    \tiny مستقل
    \end{tabular}
};

\node[pillar=green] at (1.5,3) {
    \begin{tabular}{c}
    \\
    آموزش\\
    نوین\\
    \tiny سکولار\\
    \tiny چندزبانه
    \end{tabular}
};

\node[pillar=blue] at (4,3) {
    \begin{tabular}{c}
    \\
    جامعه\\
    مدنی\\
    \tiny ۱۰۰,۰۰۰\\
    \tiny NGO
    \end{tabular}
};

\node[pillar=purple] at (6.5,3) {
    \begin{tabular}{c}
    \\
    رسانه\\
    آزاد\\
    \tiny عمومی\\
    \tiny مستقل
    \end{tabular}
};

% سقف
\node[rectangle, draw=teal!70, fill=teal!15, minimum width=16cm, minimum height=1.2cm,
      font=\small\bfseries, text=teal!70] (roof) at (0,5.8) {
    دموکراسی نهادینه‌شده — آماده برای تحکیم (فاز ۳)
};

\end{tikzpicture}
\caption{شش ستون نهادسازی فاز دوم}
\label{fig:phase2-pillars}
\end{figure}

%───────────────────────────────────────────────────────────────────────────────
% منابع فصل
%───────────────────────────────────────────────────────────────────────────────

\vspace{1cm}
\begin{refsection}

\textbf{\large منابع فصل نهم}

\vspace{0.5cm}

\begin{enumerate}[label={[\arabic*]}، nosep، leftmargin=*]
    \item North, D. (1990). \textit{Institutions, Institutional Change and Economic Performance}. Cambridge University Press.
    
    \item Lijphart, A. (2012). \textit{Patterns of Democracy}. Yale University Press.
    
    \item Stepan, A. (1999). "Federalism and Democracy: Beyond the U.S. Model." \textit{Journal of Democracy}, 10(4), 19-34.
    
    \item Ginsburg, T. \& Huq, A. (2016). \textit{Assessing Constitutional Performance}. Cambridge University Press.
    
    \item International IDEA. (2021). \textit{The Global State of Democracy 2021}.
    
    \item Venice Commission. (2010). \textit{Report on the Independence of the Judicial System}. Council of Europe.
    
    \item UNDP. (2019). \textit{Strengthening Judicial Integrity through Enhanced Access to Justice}.
    
    \item Putnam, R. (1993). \textit{Making Democracy Work: Civic Traditions in Modern Italy}. Princeton University Press.
    
    \item Diamond, L. (1999). \textit{Developing Democracy: Toward Consolidation}. Johns Hopkins University Press.
    
    \item Fukuyama, F. (2014). \textit{Political Order and Political Decay}. Farrar, Straus and Giroux.
    
    \item World Bank. (2017). \textit{World Development Report: Governance and the Law}.
    
    \item UNESCO. (2021). \textit{World Trends in Freedom of Expression and Media Development}.
    
    \item Hayner, P. (2011). \textit{Unspeakable Truths: Transitional Justice}. Routledge.
    
    \item کاتوزیان، محمدعلی. (۱۳۹۵). \textit{تضاد دولت و ملت}. نشر نی.
    
    \item بشیریه، حسین. (۱۳۹۹). \textit{جامعه‌شناسی سیاسی}. نشر نگاه معاصر.
    
    \item طباطبایی، جواد. (۱۳۹۸). \textit{تاریخ اندیشه سیاسی در ایران}. انتشارات مینوی خرد.
\end{enumerate}

\end{refsection}
	%═══════════════════════════════════════════════════════════════════════════════
% فصل ۱۰: فاز ۳ — تحکیم (سال ۶-۱۰)
% فایل: chapters/ch10-phase3.tex
%═══════════════════════════════════════════════════════════════════════════════

\chapter{فاز ۳: تحکیم (سال ۶-۱۰)}
\label{chap:phase3}

\begin{kholasebox}
\textbf{خلاصه فصل:}
فاز سوم دوره تثبیت و تعمیق دموکراسی است. در این پنج سال، نهادهای ساخته‌شده در فاز دوم باید امتحان خود را پس دهند: دومین و سومین دوره انتخابات، اولین انتقال مسالمت‌آمیز قدرت، عبور از بحران‌های اقتصادی، و مدیریت تنش‌های قومی. این فصل بر تثبیت اقتصادی، احیای محیط زیست (به‌ویژه بحران آب)، همگرایی منطقه‌ای، و شکل‌گیری فرهنگ دموکراتیک تمرکز دارد. شاخص موفقیت این فاز: دموکراسی‌ای که دیگر به افراد خاص وابسته نیست.
\end{kholasebox}

%───────────────────────────────────────────────────────────────────────────────
\section{مقدمه: آزمون واقعی دموکراسی}
\label{sec:phase3-intro}
%───────────────────────────────────────────────────────────────────────────────

\begin{naghlbox}
«دموکراسی تنها زمانی تحکیم می‌شود که به "تنها بازی در شهر" تبدیل شود — وقتی هیچ بازیگر مهمی خارج از قواعد دموکراتیک عمل نکند.»
\sourceline{خوان لینتز، نظریه‌پرداز دموکراسی}
\end{naghlbox}

بسیاری از گذارهای دموکراتیک در فاز سوم شکست می‌خورند. کشورهایی که موفق به برگزاری انتخابات آزاد شدند، ممکن است در برابر چالش‌های این دوره تسلیم شوند:

\begin{itemize}
    \item \textbf{خستگی از اصلاحات:} مردم از تغییرات مداوم خسته می‌شوند
    \item \textbf{ظهور پوپولیسم:} نارضایتی‌ها به سود افراطیون تمام می‌شود
    \item \textbf{بازگشت نیروهای قدیم:} نخبگان حذف‌شده بازمی‌گردند
    \item \textbf{بحران‌های اقتصادی:} فقر و تورم مشروعیت نظام را تضعیف می‌کند
\end{itemize}

\subsection{معیارهای تحکیم دموکراسی}

\begin{table}[htbp]
\centering
\caption{معیارهای تحکیم دموکراسی (بر اساس لینتز و استپان)}
\label{tab:consolidation-criteria}
\begin{tabular}{>{\columncolor{blue!8}}r p{3cm} p{5cm} p{3.5cm}}
\toprule
\rowcolor{blue!25}
\textbf{بُعد} & \textbf{معیار} & \textbf{شاخص} & \textbf{هدف ایران سال ۱۰} \\
\midrule
رفتاری & هیچ بازیگری خارج از قواعد & عدم کودتا، شورش، یا خشونت سیاسی & صفر تلاش ضد دموکراتیک \\
\rowcolor{gray!10}
نگرشی & اکثریت باور به دموکراسی & نظرسنجی: دموکراسی بهترین نظام & ۷۰٪+ موافق \\
قانونی & همه در چارچوب قانون & احترام به قانون اساسی توسط همه & رعایت ۱۰۰٪ \\
\rowcolor{gray!10}
نهادی & نهادها مستقل و کارآمد & شاخص‌های حکمرانی & بالای میانگین جهانی \\
اقتصادی & اقتصاد بازار تنظیم‌شده & رشد پایدار، توزیع عادلانه & رشد ۵٪+، جینی < ۰.۳۵ \\
\bottomrule
\end{tabular}
\end{table}

%───────────────────────────────────────────────────────────────────────────────
\section{انتقال مسالمت‌آمیز قدرت}
\label{sec:peaceful-transfer}
%───────────────────────────────────────────────────────────────────────────────

\begin{enghelabbox}
\textbf{آزمون نهایی:} دموکراسی زمانی تحکیم می‌شود که حزب/فرد حاکم در انتخابات ببازد و بدون خشونت قدرت را واگذار کند. این «آزمون دو انتقال» (Two-Turnover Test) ساموئل هانتینگتون است.
\end{enghelabbox}

\subsection{تقویم انتخاباتی فاز سوم}

\begin{figure}[htbp]
\centering
\begin{tikzpicture}[
    scale=0.7,
    transform shape,
    election/.style={
        rectangle,
        rounded corners=5pt,
        minimum width=2.8cm,
        minimum height=1.5cm,
        draw=#1!70!black,
        fill=#1!20,
        text=#1!30!black,
        font=\scriptsize\bfseries,
        align=center
    }
]

% خط زمان
\draw[very thick, gray!60, ->] (0,0) -- (18,0);
\node[below, font=\small] at (9,-0.5) {سال‌های ۶ تا ۱۰};

% نقاط زمانی
\foreach \x/\y in {0/سال ۶, 4.5/سال ۷, 9/سال ۸, 13.5/سال ۹, 17/سال ۱۰} {
    \draw[thick] (\x,0.15) -- (\x,-0.15);
    \node[below, font=\tiny] at (\x,-0.2) {\y};
}

% انتخابات
\node[election=blue] at (2,2.5) {
    \begin{tabular}{c}
    انتخابات محلی\\
    \tiny شوراها و شهرداری‌ها\\
    \tiny (سال ۶)
    \end{tabular}
};

\node[election=green] at (7,2.5) {
    \begin{tabular}{c}
    انتخابات مجلس\\
    \tiny دومین دوره\\
    \tiny (سال ۷)
    \end{tabular}
};

\node[election=purple] at (12,2.5) {
    \begin{tabular}{c}
    انتخابات ریاست‌جمهوری\\
    \tiny دومین دوره\\
    \tiny (سال ۹)
    \end{tabular}
};

\node[election=orange] at (16,2.5) {
    \begin{tabular}{c}
    انتخابات منطقه‌ای\\
    \tiny پارلمان‌های محلی\\
    \tiny (سال ۱۰)
    \end{tabular}
};

% فلش‌ها
\draw[->, gray!50] (2,0.3) -- (2,1.7);
\draw[->, gray!50] (7,0.3) -- (7,1.7);
\draw[->, gray!50] (12,0.3) -- (12,1.7);
\draw[->, gray!50] (16,0.3) -- (16,1.7);

% رویداد کلیدی
\node[rectangle, draw=red!60, fill=red!10, font=\tiny\bfseries,
      text=red!60, rounded corners=3pt] at (12,-1.5) {
    آزمون انتقال مسالمت‌آمیز قدرت
};
\draw[->, red!40] (12,-1.1) -- (12,-0.3);

\end{tikzpicture}
\caption{تقویم انتخاباتی فاز سوم: چهار انتخابات در پنج سال}
\label{fig:phase3-elections}
\end{figure}

\subsection{سناریوهای انتقال قدرت}

\begin{table}[htbp]
\centering
\caption{سناریوهای محتمل انتخابات ریاست‌جمهوری سال ۹}
\label{tab:transfer-scenarios}
\begin{tabular}{>{\columncolor{gray!8}}r p{3cm} c p{4cm} p{3.5cm}}
\toprule
\rowcolor{gray!25}
\textbf{سناریو} & \textbf{نتیجه} & \textbf{احتمال} & \textbf{پیامد} & \textbf{اقدام لازم} \\
\midrule
۱ & پیروزی حزب حاکم & ۳۵٪ & تداوم سیاست‌ها & ممانعت از اقتدارگرایی \\
\rowcolor{gray!10}
۲ & پیروزی اپوزیسیون & ۴۰٪ & انتقال مسالمت‌آمیز & تقویت نهادها \\
۳ & انتخابات مناقشه‌دار & ۱۵٪ & بحران مشروعیت & میانجی‌گری قضایی \\
\rowcolor{gray!10}
۴ & عدم پذیرش نتیجه & ۱۰٪ & بحران جدی & فشار داخلی/خارجی \\
\bottomrule
\end{tabular}
\end{table}

\begin{olgoobox}
\textbf{الگوی غنا: انتقال مسالمت‌آمیز موفق}

غنا در سال ۲۰۰۰ شاهد اولین انتقال مسالمت‌آمیز قدرت در تاریخش بود. جری رالینگز پس از ۱۹ سال حکومت، شکست حزبش را پذیرفت. کلیدهای موفقیت:
\begin{itemize}[nosep]
    \item کمیسیون انتخابات کاملاً مستقل
    \item ارتش بی‌طرف
    \item جامعه مدنی قوی
    \item فشار بین‌المللی
    \item فرهنگ سیاسی بالغ
\end{itemize}
غنا اکنون یکی از باثبات‌ترین دموکراسی‌های آفریقاست.
\end{olgoobox}

%───────────────────────────────────────────────────────────────────────────────
\section{تثبیت اقتصادی}
\label{sec:economic-stabilization}
%───────────────────────────────────────────────────────────────────────────────

\begin{naghlbox}
«دموکراسی نمی‌تواند در شکم گرسنه زندگی کند. رفاه اقتصادی نه شرط کافی، اما شرط لازم دموکراسی پایدار است.»
\sourceline{سیمور مارتین لیپست}
\end{naghlbox}

\subsection{اهداف اقتصادی فاز سوم}

\begin{table}[htbp]
\centering
\caption{اهداف کلان اقتصادی سال‌های ۶ تا ۱۰}
\label{tab:economic-goals-phase3}
\begin{tabular}{>{\columncolor{green!8}}r p{4cm} c c c}
\toprule
\rowcolor{green!25}
\textbf{شاخص} & \textbf{توضیح} & \textbf{وضعیت سال ۵} & \textbf{هدف سال ۱۰} & \textbf{رشد سالانه} \\
\midrule
رشد GDP & تولید ناخالص داخلی & ۴٪ & ۶٪ & افزایشی \\
\rowcolor{gray!10}
تورم & شاخص قیمت مصرف‌کننده & ۱۵٪ & ۵٪ & کاهشی \\
بیکاری & نرخ بیکاری & ۱۰٪ & ۶٪ & کاهشی \\
\rowcolor{gray!10}
سرمایه‌گذاری خارجی & FDI سالانه & ۵ میلیارد \$ & ۲۵ میلیارد \$ & ۴۰٪ \\
صادرات غیرنفتی & سهم از صادرات & ۴۰٪ & ۶۰٪ & افزایشی \\
\rowcolor{gray!10}
درآمد سرانه & PPP & ۱۸,۰۰۰ \$ & ۲۵,۰۰۰ \$ & ۷٪ \\
ضریب جینی & نابرابری & ۰.۳۸ & ۰.۳۲ & کاهشی \\
\bottomrule
\end{tabular}
\end{table}

\subsection{استراتژی تنوع‌بخشی اقتصاد}

\begin{figure}[htbp]
\centering
\begin{tikzpicture}
\begin{axis}[
    ybar stacked,
    width=14cm,
    height=8cm,
    ylabel={سهم از GDP (درصد)},
    xlabel={سال},
    ymin=0, ymax=100,
    xtick=data,
    xticklabels={۱۴۰۳ (قبل), سال ۵, سال ۱۰, سال ۱۵ (هدف)},
    legend style={at={(0.5,-0.15)}, anchor=north, legend columns=4},
    bar width=1.2cm,
    nodes near coords align={center},
    every node near coord/.append style={font=\tiny},
]

% نفت و گاز
\addplot[fill=black!60, draw=black!80] coordinates {(1,35) (2,28) (3,20) (4,15)};
% صنعت
\addplot[fill=blue!60, draw=blue!80] coordinates {(1,20) (2,23) (3,28) (4,30)};
% خدمات
\addplot[fill=green!60, draw=green!80] coordinates {(1,30) (2,32) (3,35) (4,38)};
% کشاورزی
\addplot[fill=yellow!60, draw=yellow!80] coordinates {(1,10) (2,10) (3,9) (4,8)};
% فناوری
\addplot[fill=purple!60, draw=purple!80] coordinates {(1,3) (2,5) (3,8) (4,12)};
% گردشگری
\addplot[fill=orange!60, draw=orange!80] coordinates {(1,2) (2,4) (3,7) (4,10)};

\legend{نفت و گاز, صنعت, خدمات, کشاورزی, فناوری, گردشگری}

\end{axis}
\end{tikzpicture}
\caption{مسیر تنوع‌بخشی اقتصاد: کاهش وابستگی به نفت}
\label{fig:economic-diversification}
\end{figure}

\subsection{توسعه صنعتی هدفمند}

\begin{table}[htbp]
\centering
\caption{صنایع اولویت‌دار برای توسعه در فاز سوم}
\label{tab:priority-industries}
\begin{tabular}{>{\columncolor{blue!8}}r p{2.5cm} p{3cm} p{3cm} p{3.5cm}}
\toprule
\rowcolor{blue!25}
\textbf{صنعت} & \textbf{مزیت ایران} & \textbf{هدف صادرات} & \textbf{اشتغال‌زایی} & \textbf{سرمایه‌گذاری لازم} \\
\midrule
پتروشیمی & منابع گاز فراوان & ۳۰ میلیارد \$ & ۲۰۰,۰۰۰ & ۲۰ میلیارد \$ \\
\rowcolor{gray!10}
خودروسازی & بازار داخلی بزرگ & ۵ میلیارد \$ & ۵۰۰,۰۰۰ & ۱۵ میلیارد \$ \\
فناوری اطلاعات & نیروی انسانی & ۱۰ میلیارد \$ & ۳۰۰,۰۰۰ & ۵ میلیارد \$ \\
\rowcolor{gray!10}
گردشگری & میراث فرهنگی | طبیعت & ۲۰ میلیارد \$ & ۱,۰۰۰,۰۰۰ & ۱۰ میلیارد \$ \\
داروسازی & تحقیقات پزشکی & ۵ میلیارد \$ & ۱۰۰,۰۰۰ & ۸ میلیارد \$ \\
\rowcolor{gray!10}
انرژی‌های تجدیدپذیر & پتانسیل خورشیدی/بادی & ۳ میلیارد \$ & ۱۵۰,۰۰۰ & ۳۰ میلیارد \$ \\
\bottomrule
\end{tabular}
\end{table}

\begin{olgoobox}
\textbf{الگوی امارات: تنوع‌بخشی موفق}

امارات در ۳۰ سال، سهم نفت از GDP را از ۸۰٪ به ۳۰٪ کاهش داد. کلیدها:
\begin{itemize}[nosep]
    \item سرمایه‌گذاری در زیرساخت (فرودگاه، بندر)
    \item ایجاد مناطق آزاد (جبل‌علی، DIFC)
    \item جذب استعدادهای جهانی
    \item گردشگری و خدمات مالی
    \item صندوق ثروت ملی (ADIA)
\end{itemize}
ایران با جمعیت و منابع بیشتر، پتانسیل بالاتری دارد.
\end{olgoobox}

%───────────────────────────────────────────────────────────────────────────────
\section{احیای محیط زیست و بحران آب}
\label{sec:environment-water}
%───────────────────────────────────────────────────────────────────────────────

\begin{enghelabbox}
\textbf{بحران آب: تهدید وجودی}

ایران با کسری سالانه ۱۵-۲۰ میلیارد مترمکعب آب مواجه است. بدون اقدام فوری:
\begin{itemize}[nosep]
    \item ۵۰ میلیون نفر تا ۱۴۲۰ با کمبود شدید آب مواجه می‌شوند
    \item ۷۰٪ دشت‌ها ممنوعه یا بحرانی هستند
    \item دریاچه ارومیه ۹۰٪ کوچک شده
    \item زاینده‌رود و کارون خشک شده‌اند
    \item فرونشست در ۳۰۰ دشت
\end{itemize}
بحران آب می‌تواند دموکراسی نوپا را ساقط کند.
\end{enghelabbox}

\subsection{برنامه جامع احیای آب}

\begin{figure}[htbp]
\centering
\begin{tikzpicture}[
    scale=0.8,
    transform shape,
    action/.style={
        rectangle,
        rounded corners=4pt,
        minimum width=3cm,
        minimum height=1.3cm,
        draw=#1!70!black,
        fill=#1!15,
        text=#1!30!black,
        font=\scriptsize\bfseries,
        align=center
    }
]

% محورهای اصلی
\node[action=blue, minimum width=12cm, minimum height=1.5cm] (header) at (0,5) {
    \begin{tabular}{c}
    برنامه ملی احیای آب ایران\\
    \small هدف: کاهش ۴۰٪ مصرف در ۱۰ سال
    \end{tabular}
};

% چهار محور
\node[action=green] (agri) at (-5,2) {
    \begin{tabular}{c}
    کشاورزی\\
    \tiny ۹۲٪ مصرف
    \end{tabular}
};

\node[action=orange] (urban) at (-1.5,2) {
    \begin{tabular}{c}
    شهری\\
    \tiny ۶٪ مصرف
    \end{tabular}
};

\node[action=purple] (industry) at (2,2) {
    \begin{tabular}{c}
    صنعت\\
    \tiny ۲٪ مصرف
    \end{tabular}
};

\node[action=teal] (supply) at (5.5,2) {
    \begin{tabular}{c}
    افزایش عرضه\\
    \tiny منابع جدید
    \end{tabular}
};

% اقدامات کشاورزی
\node[action=green, minimum width=2.5cm] at (-6,-0.5) {\tiny آبیاری نوین};
\node[action=green, minimum width=2.5cm] at (-6,-1.8) {\tiny تغییر الگوی کشت};
\node[action=green, minimum width=2.5cm] at (-4,-0.5) {\tiny قیمت‌گذاری آب};
\node[action=green, minimum width=2.5cm] at (-4,-1.8) {\tiny کاهش سطح زیرکشت};

% اقدامات شهری
\node[action=orange, minimum width=2.5cm] at (-1.5,-0.5) {\tiny بازیافت فاضلاب};
\node[action=orange, minimum width=2.5cm] at (-1.5,-1.8) {\tiny کاهش هدررفت};

% اقدامات صنعت
\node[action=purple, minimum width=2.5cm] at (2,-0.5) {\tiny بازچرخانی آب};
\node[action=purple, minimum width=2.5cm] at (2,-1.8) {\tiny فناوری کم‌آب};

% افزایش عرضه
\node[action=teal, minimum width=2.5cm] at (5.5,-0.5) {\tiny شیرین‌سازی};
\node[action=teal, minimum width=2.5cm] at (5.5,-1.8) {\tiny انتقال آب};

% اتصالات
\draw[->, gray!60] (header) -- (agri);
\draw[->, gray!60] (header) -- (urban);
\draw[->, gray!60] (header) -- (industry);
\draw[->, gray!60] (header) -- (supply);

% هدف
\node[rectangle, draw=red!60, fill=red!10, font=\tiny, align=center,
      rounded corners=3pt] at (0,-3.5) {
    هدف: از ۱۱۵ میلیارد m³ به ۷۵ میلیارد m³ در سال ۱۰
};

\end{tikzpicture}
\caption{چهار محور برنامه ملی احیای آب}
\label{fig:water-program}
\end{figure}

\subsection{اهداف کمّی بخش آب}

\begin{table}[htbp]
\centering
\caption{اهداف کمّی احیای آب در فاز سوم}
\label{tab:water-targets}
\begin{tabular}{>{\columncolor{cyan!8}}r p{4cm} c c c}
\toprule
\rowcolor{cyan!25}
\textbf{شاخص} & \textbf{توضیح} & \textbf{وضعیت فعلی} & \textbf{هدف سال ۱۰} & \textbf{کاهش} \\
\midrule
مصرف کشاورزی & میلیارد مترمکعب/سال & ۹۲ & ۶۰ & ۳۵٪ \\
\rowcolor{gray!10}
راندمان آبیاری & درصد & ۳۵٪ & ۶۰٪ & — \\
هدررفت شبکه شهری & درصد & ۳۰٪ & ۱۵٪ & ۵۰٪ \\
\rowcolor{gray!10}
بازیافت فاضلاب & درصد & ۱۰٪ & ۵۰٪ & — \\
شیرین‌سازی & میلیون مترمکعب/روز & ۰.۵ & ۵ & — \\
\rowcolor{gray!10}
سطح سفره‌های زیرزمینی & درصد ظرفیت & ۴۵٪ & ۵۵٪ & +۱۰٪ \\
\bottomrule
\end{tabular}
\end{table}

\subsection{احیای دریاچه‌ها و رودخانه‌ها}

\begin{table}[htbp]
\centering
\caption{برنامه احیای اکوسیستم‌های آبی}
\label{tab:ecosystem-restoration}
\begin{tabular}{>{\columncolor{blue!8}}r p{2.5cm} p{3cm} p{3cm} p{3.5cm}}
\toprule
\rowcolor{blue!25}
\textbf{اکوسیستم} & \textbf{وضعیت فعلی} & \textbf{هدف سال ۱۰} & \textbf{اقدام اصلی} & \textbf{بودجه (میلیارد \$)} \\
\midrule
دریاچه ارومیه & ۱۰٪ حجم اولیه & ۵۰٪ حجم اولیه & کاهش مصرف کشاورزی & ۵ \\
\rowcolor{gray!10}
زاینده‌رود & خشک | فصلی & جریان پایدار & تخصیص مجدد آب & ۲ \\
هورالعظیم & ۳۰٪ مساحت & ۷۰٪ مساحت & آب‌رسانی از کارون & ۱.۵ \\
\rowcolor{gray!10}
خلیج گرگان & در آستانه مرگ & احیای نسبی & اتصال به دریا & ۱ \\
کارون & آلوده | کم‌آب & استاندارد WHO & تصفیه + تخصیص & ۳ \\
\bottomrule
\end{tabular}
\end{table}

\begin{olgoobox}
\textbf{الگوی دریای آرال: درس‌های تلخ}

دریای آرال (ازبکستان/قزاقستان) از ۶۸,۰۰۰ km² به کمتر از ۱۰,۰۰۰ km² رسید — بزرگ‌ترین فاجعه زیست‌محیطی قرن بیستم. علت: انحراف آب برای پنبه‌کاری. پیامدها:
\begin{itemize}[nosep]
    \item نابودی صنعت ماهیگیری (۴۰,۰۰۰ شغل)
    \item طوفان‌های نمکی و بیماری‌های تنفسی
    \item مهاجرت میلیونی
    \item فروپاشی اقتصاد منطقه
\end{itemize}
دریاچه ارومیه در همین مسیر است مگر اقدام فوری صورت گیرد.
\end{olgoobox}

\subsection{انتقال به انرژی پاک}

\begin{figure}[htbp]
\centering
\begin{tikzpicture}
\begin{axis}[
    width=13cm,
    height=7cm,
    xlabel={سال},
    ylabel={سهم از تولید برق (درصد)},
    xmin=2024, xmax=2040,
    ymin=0, ymax=100,
    grid=major,
    grid style={dashed, gray!30},
    legend style={at={(0.02,0.98)}, anchor=north west},
    legend cell align=left,
    area style,
]

% سوخت فسیلی
\addplot[thick, fill=gray!40, draw=gray!60] coordinates {
    (2024,93) (2027,85) (2030,70) (2035,50) (2040,30)
} \closedcycle;

% گاز طبیعی (پاک‌تر)
\addplot[thick, fill=blue!30, draw=blue!60] coordinates {
    (2024,93) (2027,88) (2030,80) (2035,65) (2040,50)
};

% خورشیدی
\addplot[thick, fill=yellow!50, draw=yellow!70] coordinates {
    (2024,1) (2027,5) (2030,12) (2035,25) (2040,35)
};

% بادی
\addplot[thick, fill=green!40, draw=green!60] coordinates {
    (2024,0.5) (2027,3) (2030,8) (2035,15) (2040,20)
};

% هسته‌ای
\addplot[thick, fill=purple!30, draw=purple!60] coordinates {
    (2024,2) (2027,4) (2030,7) (2035,10) (2040,12)
};

% آبی
\addplot[thick, fill=cyan!40, draw=cyan!60] coordinates {
    (2024,3.5) (2027,4) (2030,5) (2035,5) (2040,5)
};

\legend{نفت/ذغال, گاز طبیعی, خورشیدی, بادی, هسته‌ای, برق‌آبی}

\end{axis}
\end{tikzpicture}
\caption{مسیر انتقال به انرژی پاک: سهم منابع تولید برق}
\label{fig:energy-transition}
\end{figure}

%───────────────────────────────────────────────────────────────────────────────
\section{همگرایی منطقه‌ای}
\label{sec:regional-integration}
%───────────────────────────────────────────────────────────────────────────────

\begin{naghlbox}
«ایران دموکراتیک می‌تواند قطب ثبات و همکاری در خاورمیانه باشد — نه صادرکننده انقلاب، بلکه صادرکننده توسعه.»
\sourceline{تحلیل‌گر}
\end{naghlbox}

\subsection{اهداف سیاست خارجی فاز سوم}

\begin{table}[htbp]
\centering
\caption{اهداف سیاست خارجی سال‌های ۶ تا ۱۰}
\label{tab:foreign-policy-phase3}
\begin{tabular}{>{\columncolor{purple!8}}r p{3.5cm} p{4cm} p{4cm}}
\toprule
\rowcolor{purple!25}
\textbf{محور} & \textbf{هدف} & \textbf{اقدام کلیدی} & \textbf{شاخص موفقیت} \\
\midrule
همسایگان & روابط عادی با همه & پیمان عدم تجاوز منطقه‌ای & صفر درگیری مرزی \\
\rowcolor{gray!10}
اعراب خلیج فارس & همکاری اقتصادی | امنیتی & شورای امنیت خلیج فارس & تجارت ۵۰ میلیارد \$ \\
اسرائیل & به رسمیت شناختن مشروط & در چارچوب راه‌حل دو دولت & روابط دیپلماتیک \\
\rowcolor{gray!10}
اروپا & شراکت استراتژیک & موافقت‌نامه جامع EU-Iran & تجارت ۱۰۰ میلیارد € \\
آمریکا & عادی‌سازی کامل & سفارت | لغو همه تحریم‌ها & روابط کامل دیپلماتیک \\
\rowcolor{gray!10}
سازمان‌های بین‌المللی & عضویت فعال & WTO | FATF | شورای حقوق بشر & عضویت کامل \\
\bottomrule
\end{tabular}
\end{table}

\subsection{نقشه همگرایی منطقه‌ای}

\begin{figure}[htbp]
\centering
\begin{tikzpicture}[
    scale=0.7,
    transform shape,
    country/.style={
        ellipse,
        draw=#1!70!black,
        fill=#1!20,
        minimum width=2cm,
        minimum height=1.2cm,
        font=\scriptsize\bfseries,
        align=center
    },
    relation/.style={
        -,
        thick,
        draw=#1!60
    }
]

% ایران در مرکز
\node[country=blue, minimum width=3cm, minimum height=2cm] (iran) at (0,0) {ایران};

% همسایگان و شرکا
\node[country=green] (turkey) at (-5,3) {ترکیه};
\node[country=green] (pakistan) at (5,3) {پاکستان};
\node[country=green] (iraq) at (-5,-3) {عراق};
\node[country=orange] (saudi) at (5,-3) {عربستان};
\node[country=orange] (uae) at (3,-4) {امارات};
\node[country=green] (afghan) at (6,0) {افغانستان};
\node[country=yellow] (azer) at (-3,4) {آذربایجان};
\node[country=yellow] (armenia) at (-4,2) {ارمنستان};
\node[country=green] (india) at (7,2) {هند};
\node[country=teal] (china) at (7,-2) {چین};
\node[country=purple] (russia) at (-6,0) {روسیه};
\node[country=cyan] (eu) at (0,5) {اتحادیه اروپا};

% روابط
\draw[relation=green] (iran) -- (turkey) node[midway, above, font=\tiny] {تجارت آزاد};
\draw[relation=green] (iran) -- (pakistan) node[midway, above, font=\tiny] {انرژی};
\draw[relation=green] (iran) -- (iraq) node[midway, left, font=\tiny] {بازسازی};
\draw[relation=orange] (iran) -- (saudi) node[midway, right, font=\tiny] {تنش‌زدایی};
\draw[relation=orange] (iran) -- (uae) node[midway, right, font=\tiny] {تجارت};
\draw[relation=green] (iran) -- (afghan) node[midway, above, font=\tiny] {آب | امنیت};
\draw[relation=cyan] (iran) -- (eu) node[midway, left, font=\tiny] {شراکت جامع};
\draw[relation=teal] (iran) -- (china) node[midway, right, font=\tiny] {BRI};
\draw[relation=green] (iran) -- (india) node[midway, above, font=\tiny] {چابهار};

% راهنما
\node[rectangle, draw=gray!40, fill=gray!5, font=\tiny, align=right] at (-7,-4) {
    \textcolor{green!60!black}{سبز:} همکاری فعال\\
    \textcolor{orange!60!black}{نارنجی:} در حال بهبود\\
    \textcolor{yellow!60!black}{زرد:} نیازمند توجه
};

\end{tikzpicture}
\caption{نقشه روابط منطقه‌ای ایران در افق سال ۱۰}
\label{fig:regional-map}
\end{figure}

\subsection{پیمان همکاری خلیج فارس}

\begin{olgoobox}
\textbf{پیشنهاد: شورای همکاری خلیج فارس جدید}

پیمانی شامل ایران + شش کشور GCC با محورهای:
\begin{itemize}[nosep]
    \item \textbf{امنیتی:} پیمان عدم تجاوز، کاهش تسلیحات، مبارزه با تروریسم
    \item \textbf{اقتصادی:} منطقه تجارت آزاد، هماهنگی انرژی، سرمایه‌گذاری مشترک
    \item \textbf{زیست‌محیطی:} حفاظت از خلیج فارس، مدیریت آب‌های مشترک
    \item \textbf{فرهنگی:} تبادل دانشجو، گردشگری، رسانه مشترک
\end{itemize}
این پیمان می‌تواند صلح پایدار را در پرتنش‌ترین منطقه جهان برقرار کند.
\end{olgoobox}

%───────────────────────────────────────────────────────────────────────────────
\section{فرهنگ دموکراتیک}
\label{sec:democratic-culture}
%───────────────────────────────────────────────────────────────────────────────

دموکراسی پایدار نیازمند فرهنگ دموکراتیک است — ارزش‌ها و رفتارهایی که از قانون فراتر می‌روند.

\subsection{ارکان فرهنگ دموکراتیک}

\begin{figure}[htbp]
\centering
\begin{tikzpicture}[
    scale=0.85,
    transform shape,
    value/.style={
        rectangle,
        rounded corners=5pt,
        minimum width=3cm,
        minimum height=1.5cm,
        draw=#1!70!black,
        fill=#1!15,
        text=#1!30!black,
        font=\small\bfseries,
        align=center
    }
]

% مرکز
\node[ellipse, draw=purple!70, fill=purple!10, minimum width=4cm, 
      minimum height=2.5cm, font=\bfseries, text=purple!70] (center) at (0,0) {
    \begin{tabular}{c}
    فرهنگ\\
    دموکراتیک
    \end{tabular}
};

% ارزش‌ها
\node[value=blue] (tolerance) at (-4,3) {تساهل و مدارا};
\node[value=green] (participation) at (0,4) {مشارکت فعال};
\node[value=orange] (trust) at (4,3) {اعتماد اجتماعی};
\node[value=red] (rule) at (-4,-3) {احترام به قانون};
\node[value=teal] (critical) at (0,-4) {تفکر انتقادی};
\node[value=yellow] (plural) at (4,-3) {پذیرش تنوع};

% اتصالات
\draw[->, gray!50] (tolerance) -- (center);
\draw[->, gray!50] (participation) -- (center);
\draw[->, gray!50] (trust) -- (center);
\draw[->, gray!50] (rule) -- (center);
\draw[->, gray!50] (critical) -- (center);
\draw[->, gray!50] (plural) -- (center);

\end{tikzpicture}
\caption{شش رکن فرهنگ دموکراتیک}
\label{fig:democratic-culture}
\end{figure}

\subsection{برنامه‌های ترویج فرهنگ دموکراتیک}

\begin{table}[htbp]
\centering
\caption{برنامه‌های ترویج فرهنگ دموکراتیک}
\label{tab:culture-programs}
\begin{tabular}{>{\columncolor{orange!8}}r p{3cm} p{4.5cm} p{4cm}}
\toprule
\rowcolor{orange!25}
\textbf{حوزه} & \textbf{برنامه} & \textbf{محتوا} & \textbf{مخاطب} \\
\midrule
آموزش رسمی & تربیت شهروندی & حقوق و مسئولیت‌ها، مشارکت مدنی & دانش‌آموزان \\
\rowcolor{gray!10}
رسانه & کمپین‌های آگاهی‌بخشی & مدارا، گفتگو، احترام به تفاوت & عموم مردم \\
جامعه مدنی & کارگاه‌های دموکراسی & مهارت‌های مشارکت، نظارت & فعالان مدنی \\
\rowcolor{gray!10}
احزاب & آموزش حزبی & رقابت سالم، پذیرش شکست & اعضای احزاب \\
محلی & شوراهای محله & تصمیم‌گیری مشارکتی & شهروندان \\
\rowcolor{gray!10}
هنر | ادبیات & حمایت از آثار دموکراتیک & فیلم، کتاب، تئاتر | نمایشگاه & هنرمندان و مخاطبان \\
\bottomrule
\end{tabular}
\end{table}

%───────────────────────────────────────────────────────────────────────────────
\section{شاخص‌های موفقیت فاز سوم}
\label{sec:phase3-kpis}
%───────────────────────────────────────────────────────────────────────────────

\begin{table}[htbp]
\centering
\caption{شاخص‌های کلیدی موفقیت (KPI) فاز سوم}
\label{tab:phase3-kpis}
\begin{tabular}{>{\columncolor{blue!8}}r p{5cm} c c}
\toprule
\rowcolor{blue!25}
\textbf{کد} & \textbf{شاخص} & \textbf{هدف سال ۱۰} & \textbf{وزن} \\
\midrule
C01 & انتقال مسالمت‌آمیز قدرت (بله/خیر) & بله & ۱۵٪ \\
\rowcolor{gray!10}
C02 & شاخص دموکراسی EIU & ۶.۵+ (دموکراسی ناقص) & ۱۰٪ \\
C03 & رشد GDP سالانه & ۶٪+ & ۱۰٪ \\
\rowcolor{gray!10}
C04 & نرخ تورم & زیر ۵٪ & ۸٪ \\
C05 & نرخ بیکاری & زیر ۶٪ & ۸٪ \\
\rowcolor{gray!10}
C06 & سرمایه‌گذاری خارجی سالانه & ۲۵ میلیارد \$+ & ۷٪ \\
C07 & کاهش مصرف آب & ۳۵٪ نسبت به سال پایه & ۱۰٪ \\
\rowcolor{gray!10}
C08 & سهم انرژی تجدیدپذیر & ۲۵٪+ & ۷٪ \\
C09 & رضایت عمومی از دموکراسی & ۷۰٪+ & ۱۰٪ \\
\rowcolor{gray!10}
C10 & روابط دیپلماتیک کامل & با همه همسایگان + آمریکا + EU & ۱۰٪ \\
C11 & شاخص ادراک فساد & ۴۵+ (از ۱۰۰) & ۵٪ \\
\bottomrule
\end{tabular}
\end{table}

%───────────────────────────────────────────────────────────────────────────────
\section{جمع‌بندی فصل}
\label{sec:phase3-conclusion}
%───────────────────────────────────────────────────────────────────────────────

\begin{kholasebox}
\textbf{خلاصه فصل ۱۰:}
\begin{enumerate}
    \item \textbf{آزمون انتقال قدرت:} انتخابات سال ۹ محک اصلی تحکیم دموکراسی است
    \item \textbf{تثبیت اقتصادی:} رشد ۶٪+، تورم ۵٪-، تنوع‌بخشی از نفت
    \item \textbf{بحران آب:} کاهش ۳۵٪ مصرف، احیای دریاچه ارومیه، اصلاح کشاورزی
    \item \textbf{انتقال انرژی:} ۲۵٪ سهم تجدیدپذیرها تا سال ۱۰
    \item \textbf{همگرایی منطقه‌ای:} عادی‌سازی با همه همسایگان، پیمان خلیج فارس
    \item \textbf{فرهنگ دموکراتیک:} نهادینه‌سازی ارزش‌های تساهل و مشارکت
    \item پایان فاز سوم: دموکراسی‌ای که دیگر وابسته به افراد نیست
\end{enumerate}
\end{kholasebox}

% نمودار جمع‌بندی
\begin{figure}[htbp]
\centering
\begin{tikzpicture}[
    scale=0.75,
    transform shape,
    achieve/.style={
        rectangle,
        rounded corners=5pt,
        minimum width=2.5cm,
        minimum height=1.5cm,
        draw=#1!70!black,
        fill=#1!15,
        text=#1!30!black,
        font=\scriptsize\bfseries,
        align=center
    }
]

% ورودی
\node[achieve=gray, minimum width=4cm] (input) at (0,0) {
    \begin{tabular}{c}
    ورودی از فاز ۲:\\
    نهادهای نوپا
    \end{tabular}
};

% دستاوردهای فاز ۳
\node[achieve=green] at (5,2) {انتقال مسالمت‌آمیز قدرت};
\node[achieve=blue] at (5,0) {رشد اقتصادی پایدار};
\node[achieve=cyan] at (5,-2) {احیای محیط زیست};
\node[achieve=orange] at (10,2) {همگرایی منطقه‌ای};
\node[achieve=purple] at (10,0) {فرهنگ دموکراتیک};
\node[achieve=yellow] at (10,-2) {نهادهای مستحکم};

% خروجی
\node[achieve=red, minimum width=4.5cm, minimum height=2cm] (output) at (15,0) {
    \begin{tabular}{c}
    خروجی به فاز ۴:\\
    \textbf{دموکراسی تحکیم‌شده}\\
    آماده برای بلوغ
    \end{tabular}
};

% فلش‌ها
\draw[->, thick, gray!60] (input) -- (3,0);
\draw[->, thick, gray!60] (7.5,0) -- (12.5,0);
\draw[->, thick, gray!60] (12.5,0) -- (output);

\end{tikzpicture}
\caption{مسیر فاز سوم: از نهادسازی به تحکیم}
\label{fig:phase3-summary}
\end{figure}

%───────────────────────────────────────────────────────────────────────────────
% منابع فصل
%───────────────────────────────────────────────────────────────────────────────

\vspace{1cm}
\begin{refsection}

\textbf{\large منابع فصل دهم}

\vspace{0.5cm}

\begin{enumerate}[label={[\arabic*]}, nosep, leftmargin=*]
    \item Linz, J. \& Stepan, A. (1996). \textit{Problems of Democratic Transition and Consolidation}. Johns Hopkins University Press.
    
    \item Huntington, S. (1991). \textit{The Third Wave: Democratization in the Late Twentieth Century}. University of Oklahoma Press.
    
    \item Diamond, L. (1999). \textit{Developing Democracy: Toward Consolidation}. Johns Hopkins University Press.
    
    \item Lipset, S.M. (1959). "Some Social Requisites of Democracy." \textit{American Political Science Review}, 53(1), 69-105.
    
    \item World Bank. (2023). \textit{Iran Economic Monitor}.
    
    \item IMF. (2023). \textit{World Economic Outlook: Iran}.
    
    \item UN-Water. (2023). \textit{Water Scarcity Report: Middle East}.
    
    \item IRENA. (2023). \textit{Renewable Energy Statistics}.
    
    \item Madani, K. (2014). "Water Management in Iran: What is Causing the Looming Crisis?" \textit{Journal of Environmental Studies and Sciences}, 4(4), 315-328.
    
    \item وزارت نیرو. (۱۴۰۲). \textit{ترازنامه آب ایران}.
    
    \item مرکز آمار ایران. (۱۴۰۲). \textit{سالنامه آماری کشور}.
    
    \item Economist Intelligence Unit. (2024). \textit{Democracy Index 2023}.
    
    \item Freedom House. (2024). \textit{Freedom in the World 2024}.
    
    \item World Values Survey. (2022). \textit{Wave 7 Results}.
    
    \item Inglehart, R. \& Welzel, C. (2005). \textit{Modernization, Cultural Change, and Democracy}. Cambridge University Press.
\end{enumerate}

\end{refsection}
	% ch11-phase45.tex
% فصل یازدهم: فاز ۴ و ۵ — بلوغ و تعالی دموکراتیک
% نویسنده: مهدی سالم | ریچموندهیل | ۱۴۰۴

\chapter{فاز ۴ و ۵: بلوغ و تعالی دموکراتیک (سال ۱۱-۲۵)}
\chapterheader{۱۱}{فاز ۴ و ۵: بلوغ و تعالی}{افق روشن ایران در تراز جهانی}{AccentGold}
\label{ch:phase45}

\begin{kholasebox}
این فصل به دو فاز پایانی گذار دموکراتیک می‌پردازد: \textbf{فاز چهارم (بلوغ)} در سال‌های ۱۱ تا ۱۵ که طی آن نهادها به بلوغ کامل می‌رسند و رویه‌های دموکراتیک درونی می‌شوند، و \textbf{فاز پنجم (تعالی)} در سال‌های ۱۶ تا ۲۵ که ایران به الگوی توسعه منطقه‌ای تبدیل شده و به استانداردهای کشورهای توسعه‌یافته دست می‌یابد. هدف نهایی: ایرانی آباد، آزاد و سربلند در جامعه جهانی با کیفیت زندگی در سطح OECD.
\end{kholasebox}

%═══════════════════════════════════════════════════════════════════════════════
\section{مقدمه: از تثبیت به تعالی}
%═══════════════════════════════════════════════════════════════════════════════

پس از یک دهه تلاش فشرده برای گذار، نهادسازی و تثبیت، ایران در آستانه ورود به مرحله‌ای کیفی متفاوت قرار می‌گیرد. اگر سه فاز نخست را می‌توان «ساختن خانه» نامید، فازهای چهارم و پنجم «زندگی در خانه» و «زیباسازی آن» هستند.

\begin{naghlbox}
«دموکراسی تنها زمانی تکمیل می‌شود که نه‌تنها نخبگان، بلکه توده مردم نیز آن را تنها بازی ممکن بدانند — وقتی هیچ‌کس جدی به بازگشت اقتدارگرایی نیندیشد.»
\sourceline{خوان لینز و آلفرد استپان، «مشکلات گذار و تثبیت دموکراتیک»، ۱۹۹۶}
\end{naghlbox}

\subsection{تفاوت ماهوی فازهای پایانی}

\begin{table}[htbp]
\centering
\caption{مقایسه ماهیت فازهای مختلف گذار دموکراتیک}
\label{tab:phase-comparison}
\begin{tabular}{>{\columncolor{blue!8}}r p{3cm} p{3cm} p{4cm}}
\toprule
\rowcolor{blue!25}
\textbf{فاز} & \textbf{ماهیت اصلی} & \textbf{چالش کلیدی} & \textbf{معیار موفقیت} \\
\midrule
۱ (گذار) & مدیریت بحران & بقا و ثبات & عدم بازگشت به خشونت \\
\rowcolor{gray!10}
۲ (نهادسازی) & ساخت زیرساخت & طراحی و اجرا & نهادهای کارآمد \\
۳ (تحکیم) & آزمون نهادها & انتقال قدرت & چرخش مسالمت‌آمیز \\
\rowcolor{gray!10}
۴ (بلوغ) & درونی‌سازی & فرهنگ‌سازی عمیق & دموکراسی به‌مثابه عادت \\
۵ (تعالی) & الگوسازی & نوآوری و پیشتازی & استانداردهای جهانی \\
\bottomrule
\end{tabular}
\end{table}

\subsection{پیش‌شرط‌های ورود به فاز چهارم}

برای ورود موفق به فاز بلوغ، دستاوردهای زیر باید در پایان سال دهم محقق شده باشند:

\begin{enumerate}[nosep]
\item \textbf{انتقال مسالمت‌آمیز قدرت}: حداقل یک چرخش کامل قدرت از طریق انتخابات
\item \textbf{ثبات اقتصادی}: تورم تک‌رقمی، رشد پایدار بالای ۵٪
\item \textbf{اجماع ملی}: پذیرش قانون اساسی توسط بیش از ۷۵٪ مردم
\item \textbf{نهادهای کارآمد}: عملکرد مطلوب سه قوه و نهادهای نظارتی
\item \textbf{امنیت پایدار}: نبود تهدید جدی داخلی یا خارجی
\item \textbf{جامعه مدنی پویا}: بیش از ۸۰,۰۰۰ سازمان مردم‌نهاد فعال
\end{enumerate}

%═══════════════════════════════════════════════════════════════════════════════
\section{فاز چهارم: بلوغ دموکراتیک (سال ۱۱-۱۵)}
\label{sec:phase4}
%═══════════════════════════════════════════════════════════════════════════════

\subsection{چشم‌انداز فاز چهارم}

\begin{figure}[htbp]
\centering
\begin{tikzpicture}[
    scale=0.85,
    transform shape,
    box/.style={
        rectangle,
        rounded corners=5pt,
        minimum width=3.5cm, text width=3.5cm,
        minimum height=1.5cm,
        align=center,
        font=\small\bfseries,
        draw=bleurepublique,
        fill=bleulight,
        line width=1.2pt
    },
    arrow/.style={->, >=Stealth, thick, color=goldphoenix}
]

% عنوان
\node[rectangle, rounded corners=8pt, draw=goldphoenix, fill=goldphoenix, text=white,
      minimum width=12cm, text width=12cm, minimum height=1cm, font=\large\bfseries] (title) at (0,6) {\rl{چشم‌انداز فاز چهارم: ایران بالغ دموکراتیک}};

% ردیف اول
\node[box] (n1) at (-4,3.5) {\rl{نهادها}\\ \tiny \rl{پختگی و کارآمدی}};
\node[box] (n2) at (0,3.5) {\rl{فرهنگ}\\ \tiny \rl{درونی‌سازی ارزش‌ها}};
\node[box] (n3) at (4,3.5) {\rl{اقتصاد}\\ \tiny \rl{دانش‌بنیان و متنوع}};

% ردیف دوم
\node[box] (n4) at (-2,1) {\rl{منطقه}\\ \tiny \rl{قدرت متوسط مؤثر}};
\node[box] (n5) at (2,1) {\rl{جهان}\\ \tiny \rl{عضو محترم جامعه بین‌الملل}};

% هدف
\node[rectangle, rounded corners=5pt, draw=goldphoenix, fill=goldlight, text=goldphoenix,
      minimum width=8cm, text width=8cm, minimum height=1cm, font=\bfseries] (goal) at (0,-1.5) {\rl{هدف: دموکراسی تثبیت‌شده و برگشت‌ناپذیر}};

% اتصالات
\draw[arrow] (title) -- (n1);
\draw[arrow] (title) -- (n2);
\draw[arrow] (title) -- (n3);
\draw[arrow] (n1) -- (n4);
\draw[arrow] (n2) -- (n4);
\draw[arrow] (n2) -- (n5);
\draw[arrow] (n3) -- (n5);
\draw[arrow] (n4) -- (goal);
\draw[arrow] (n5) -- (goal);

\end{tikzpicture}
\caption{چشم‌انداز بلوغ دموکراتیک فاز چهارم}
\end{figure}

\subsection{ستون اول: نهادینه‌سازی کامل رویه‌ها}

در این مرحله، نهادها از «اجرای قوانین» به «عادت‌واره شدن» گذر می‌کنند.

\subsubsection{بلوغ قوه مقننه}

\begin{table}[htbp]
\centering
\caption{شاخص‌های بلوغ قوه مقننه در پایان فاز چهارم}
\label{tab:legislature-maturity}
\begin{tabular}{>{\columncolor{green!8}}r p{5cm} c c}
\toprule
\rowcolor{green!25}
\textbf{ردیف} & \textbf{شاخص} & \textbf{سال ۱۰} & \textbf{هدف سال ۱۵} \\
\midrule
۱ & نرخ مشارکت انتخاباتی & ۶۵٪ & ۷۵٪ \\
\rowcolor{gray!10}
۲ & نسبت زنان در مجلس & ۲۵٪ & ۳۵٪ \\
۳ & میانگین سن نمایندگان & ۵۲ سال & ۴۵ سال \\
\rowcolor{gray!10}
۴ & تحصیلات دانشگاهی نمایندگان & ۸۵٪ & ۹۵٪ \\
۵ & رضایت عمومی از مجلس & ۴۵٪ & ۶۵٪ \\
\rowcolor{gray!10}
۶ & کیفیت قانون‌گذاری (ارزیابی مستقل) & ۶.۵/۱۰ & ۸/۱۰ \\
۷ & شفافیت فرآیند تصمیم‌گیری & ۷۰٪ & ۹۵٪ \\
\bottomrule
\end{tabular}
\end{table}

\textbf{اقدامات کلیدی}:
\begin{itemize}[nosep]
\item تقویت دفاتر تحقیقاتی مجلس با ۵۰۰ کارشناس متخصص
\item راه‌اندازی سامانه مشارکت الکترونیک شهروندان در قانون‌گذاری
\item برگزاری منظم جلسات پاسخگویی نمایندگان در حوزه‌های انتخابیه
\item ایجاد مکانیزم نظارت مردمی بر هزینه‌های نمایندگان
\end{itemize}

\subsubsection{تکامل قوه قضائیه}

\begin{olgoobox}
\textbf{الگوی موفق: دادگستری آلمان پس از جنگ}

دادگستری آلمان غربی پس از ۱۹۴۵ نمونه‌ای موفق از بازسازی است:
\begin{itemize}[nosep]
\item پاکسازی قضات نازی با حفظ ظرفیت سیستم
\item آموزش نسل جدید با تأکید بر حقوق بشر
\item استقلال کامل مالی و اداری
\item دادگاه قانون اساسی فدرال به‌عنوان نگهبان دموکراسی
\item پس از ۲۰ سال: یکی از معتبرترین نظام‌های قضایی جهان
\end{itemize}
\end{olgoobox}

\begin{table}[htbp]
\centering
\caption{اهداف کیفی قوه قضائیه در فاز چهارم}
\label{tab:judiciary-phase4}
\begin{tabular}{>{\columncolor{purple!8}}r p{4cm} p{6cm}}
\toprule
\rowcolor{purple!25}
\textbf{حوزه} & \textbf{هدف کمّی} & \textbf{اقدامات اجرایی} \\
\midrule
استقلال & امتیاز ۸/۱۰ در شاخص‌های بین‌المللی & قانون تصدی مادام‌العمر قضات عالی \\
\rowcolor{gray!10}
کارآمدی & میانگین رسیدگی ۹۰ روز & دادرسی الکترونیک کامل، افزایش ۵۰٪ قضات \\
دسترسی & معاضدت قضایی برای ۱۰۰٪ نیازمندان & ۵۰۰۰ وکیل معاضدتی، کلینیک‌های حقوقی \\
\rowcolor{gray!10}
اعتماد & رضایت ۶۵٪ مردم & شفافیت آراء، نظارت مردمی، آموزش عمومی \\
تخصص & ۲۰ شعبه تخصصی جدید & محاکم محیط‌زیست، فناوری، تجارت بین‌الملل \\
\bottomrule
\end{tabular}
\end{table}

\subsubsection{پختگی قوه مجریه}

دولت در این مرحله باید از «مدیریت بحران» به «برنامه‌ریزی بلندمدت» گذر کند.

\begin{enumerate}[nosep]
\item \textbf{سیاستگذاری مبتنی بر شواهد}:
    \begin{itemize}[nosep]
    \item مرکز ملی آمار و داده با استقلال کامل
    \item الزام ارزیابی تأثیر برای همه سیاست‌ها
    \item بانک داده باز (Open Data) برای پژوهشگران
    \end{itemize}

\item \textbf{دولت هوشمند}:
    \begin{itemize}[nosep]
    \item ۹۵٪ خدمات دولتی آنلاین
    \item هویت دیجیتال یکپارچه برای همه شهروندان
    \item هوش مصنوعی در خدمات عمومی (با نظارت اخلاقی)
    \end{itemize}

\item \textbf{بروکراسی کارآمد}:
    \begin{itemize}[nosep]
    \item کاهش ۴۰٪ حجم دستگاه دولتی نسبت به ۱۴۰۳
    \item حقوق رقابتی برای جذب نخبگان
    \item ارزیابی عملکرد شفاف و مبتنی بر نتیجه
    \end{itemize}
\end{enumerate}

\subsection{ستون دوم: اقتصاد دانش‌بنیان}

\subsubsection{تحول ساختاری اقتصاد}

\begin{figure}[htbp]
\centering
\begin{tikzpicture}
\begin{axis}[
    width=13cm,
    height=8cm,
    ybar stacked,
    bar width=1.1cm,
    ylabel={\rl{درصد از GDP}},
    xlabel={\rl{سال}},
    ymin=0, ymax=100,
    xtick=data,
    xticklabels={\rl{۱۴۰۳}, \rl{سال ۵}, \rl{سال ۱۰}, \rl{سال ۱۵}, \rl{سال ۲۵}},
    legend style={at={(0.5,-0.2)}, anchor=north, legend columns=2, font=\tiny},
    axis line style={bleurepublique},
    tick style={bleurepublique},
    label style={font=\tiny\bfseries, color=bleurepublique}
]

% نفت
\addplot[fill=bleurepublique!80] coordinates {(1,35) (2,28) (3,20) (4,15) (5,10)};
% خدمات و صنعت
\addplot[fill=goldphoenix!80] coordinates {(1,55) (2,56) (3,57) (4,57) (5,56)};
% دانش‌بنیان
\addplot[fill=bleulight] coordinates {(1,5) (2,10) (3,15) (4,20) (5,28)};
% گردشگری
\addplot[fill=goldlight] coordinates {(1,5) (2,6) (3,8) (4,8) (5,6)};

\legend{\rl{نفت و گاز}, \rl{صنعت و خدمات}, \rl{دانش‌بنیان و فناوری}, \rl{گردشگری}}

\end{axis}
\end{tikzpicture}
\caption{تحول ساختار اقتصادی ایران در افق ۲۵ ساله}
\end{figure}

\subsubsection{زیست‌بوم نوآوری}

\begin{table}[htbp]
\centering
\caption{اهداف زیست‌بوم نوآوری در پایان فاز چهارم (سال ۱۵)}
\label{tab:innovation-ecosystem}
\begin{tabular}{>{\columncolor{orange!8}}r l c c c}
\toprule
\rowcolor{orange!25}
\textbf{ردیف} & \textbf{شاخص} & \textbf{سال ۱۰} & \textbf{سال ۱۵} & \textbf{رتبه جهانی} \\
\midrule
۱ & هزینه R\&D (٪ GDP) & ۱.۵٪ & ۲.۵٪ & در ۳۰ کشور برتر \\
\rowcolor{gray!10}
۲ & ثبت اختراع سالانه (بین‌المللی) & ۲,۰۰۰ & ۵,۰۰۰ & در ۲۵ کشور برتر \\
۳ & تعداد استارتاپ‌های فعال & ۱۵,۰۰۰ & ۴۰,۰۰۰ & — \\
\rowcolor{gray!10}
۴ & جذب سرمایه‌گذاری خطرپذیر (میلیارد دلار) & ۲ & ۸ & در ۲۰ کشور برتر \\
۵ & تعداد یونیکورن‌ها & ۵ & ۱۵ & — \\
\rowcolor{gray!10}
۶ & صادرات فناوری (میلیارد دلار) & ۵ & ۲۰ & — \\
۷ & شاخص نوآوری جهانی (GII) & رتبه ۵۰ & رتبه ۳۵ & — \\
\bottomrule
\end{tabular}
\end{table}

\textbf{حوزه‌های اولویت‌دار فناوری}:

\begin{figure}[htbp]
\centering
\begin{tikzpicture}[
    scale=0.85,
    transform shape,
    priority/.style={
        rectangle,
        rounded corners=5pt,
        minimum width=3.5cm, text width=3.5cm,
        minimum height=1.3cm,
        align=center,
        font=\small\bfseries,
        draw=bleurepublique,
        fill=bleulight,
        line width=1.2pt
    },
    label/.style={font=\tiny\bfseries, color=goldphoenix}
]

% ردیف اول
\node[priority, draw=goldphoenix, fill=goldphoenix, text=white] (ai) at (0,0) {\rl{هوش مصنوعی}\\ \tiny \rl{پردازش زبان فارسی}};
\node[priority, draw=goldphoenix, fill=goldphoenix, text=white] (bio) at (-4,0) {\rl{زیست‌فناوری}\\ \tiny \rl{دارو و تجهیزات}};
\node[priority, draw=goldphoenix, fill=goldphoenix, text=white] (energy) at (4,0) {\rl{انرژی پاک}\\ \tiny \rl{خورشیدی و هیدروژن}};

% ردیف دوم
\node[priority] (water) at (0,-2) {\rl{فناوری آب}\\ \tiny \rl{بازیافت و تصفیه}};
\node[priority] (space) at (-4,-2) {\rl{فناوری فضایی}\\ \tiny \rl{ماهواره و سنجش}};
\node[priority] (nano) at (4,-2) {\rl{نانوفناوری}\\ \tiny \rl{مواد پیشرفته}};

% برچسب‌ها
\node[label] at (-6,0) {\rl{اولویت اول}};
\node[label] at (-6,-2) {\rl{اولویت دوم}};

\end{tikzpicture}
\caption{حوزه‌های اولویت‌دار فناوری در فاز چهارم}
\end{figure}

\subsubsection{سرمایه انسانی}

\begin{enghelabbox}
\textbf{هشدار: فرار مغزها همچنان تهدید است}

حتی در فاز چهارم، بدون توجه جدی به سرمایه انسانی، ریسک فرار مغزها باقی می‌ماند:
\begin{itemize}[nosep]
\item ایران سالانه ۱۵۰,۰۰۰ فارغ‌التحصیل دانشگاهی تولید می‌کند
\item در دهه گذشته، تخمین‌ها نشان می‌دهد سالانه ۵۰,۰۰۰ نفر تحصیل‌کرده مهاجرت کرده‌اند
\item \textbf{راه‌حل}: ایجاد فرصت‌های شغلی جذاب، آزادی آکادمیک، و کیفیت زندگی بالا
\item هدف: تبدیل «فرار مغزها» به «چرخش مغزها» — ایرانیان خارج به‌عنوان پل ارتباطی
\end{itemize}
\end{enghelabbox}

\begin{table}[htbp]
\centering
\caption{برنامه توسعه سرمایه انسانی در فاز چهارم}
\label{tab:human-capital}
\begin{tabular}{>{\columncolor{cyan!8}}r p{4cm} p{5.5cm}}
\toprule
\rowcolor{cyan!25}
\textbf{حوزه} & \textbf{هدف کمّی} & \textbf{برنامه‌های کلیدی} \\
\midrule
آموزش عالی & ورود به ۲۰۰ دانشگاه برتر جهان & ۵ قطب دانشگاهی بین‌المللی \\
\rowcolor{gray!10}
آموزش فنی & ۴۰٪ فارغ‌التحصیلان در رشته‌های فنی & اصلاح کنکور، ارتقای منزلت فنی‌حرفه‌ای \\
جذب نخبگان & بازگشت ۵۰,۰۰۰ متخصص & ویزای طلایی، معافیت مالیاتی \\
\rowcolor{gray!10}
مهارت‌افزایی & آموزش ۵ میلیون شاغل & پلتفرم ملی آموزش مهارت \\
زبان انگلیسی & ۵۰٪ جمعیت مسلط & آموزش از ابتدایی، محتوای رایگان \\
\bottomrule
\end{tabular}
\end{table}

\subsection{ستون سوم: جایگاه منطقه‌ای ایران}

\subsubsection{قدرت متوسط مسئول}

ایران در فاز چهارم باید به «قدرت متوسط مسئول» در منطقه تبدیل شود — کشوری که هم توانایی دارد و هم مسئولیت‌پذیر است.

\begin{naghlbox}
«قدرت واقعی در قرن بیست‌ویکم، توانایی شکل‌دادن به محیط از طریق جذابیت است، نه اجبار. کشورهایی موفق‌ترند که دیگران بخواهند با آن‌ها همکاری کنند.»
\sourceline{جوزف نای، «قدرت نرم»، ۲۰۰۴}
\end{naghlbox}

\begin{figure}[htbp]
\centering
\begin{tikzpicture}[
    scale=0.85,
    transform shape,
    circle_node/.style={
        circle,
        draw=bleurepublique,
        fill=bleulight,
        minimum size=2.8cm, text width=2.8cm,
        align=center,
        font=\small\bfseries,
        line width=1.5pt
    },
    flow/.style={-, thick, bleurepublique!50, dashed}
]

% مرکز
\node[circle_node, draw=goldphoenix, fill=goldphoenix, text=white, minimum size=3.2cm, text width=3.2cm] (iran) at (0,0) {\rl{ایران}\\ \tiny \rl{قدرت متوسط مسئول}};

% اطراف
\node[circle_node] (eco) at (0,3.5) {\rl{اقتصادی}\\ \tiny \rl{هاب تجاری}};
\node[circle_node] (sec) at (0,-3.5) {\rl{امنیتی}\\ \tiny \rl{ثبات‌بخش}};
\node[circle_node] (cul) at (-4,0) {\rl{فرهنگی}\\ \tiny \rl{تمدن‌ساز}};
\node[circle_node] (dip) at (4,0) {\rl{دیپلماتیک}\\ \tiny \rl{میانجی}};

% اتصالات
\draw[flow] (iran) -- (eco);
\draw[flow] (iran) -- (sec);
\draw[flow] (iran) -- (cul);
\draw[flow] (iran) -- (dip);

\end{tikzpicture}
\caption{نقش‌های راهبردی ایران در جایگاه قدرت متوسط مسئول}
\end{figure}

\subsubsection{نقش‌های منطقه‌ای ایران}

\begin{table}[htbp]
\centering
\caption{نقش‌های هدف ایران در منطقه تا پایان فاز چهارم}
\label{tab:regional-roles}
\begin{tabular}{>{\columncolor{teal!8}}r p{2.5cm} p{7cm}}
\toprule
\rowcolor{teal!25}
\textbf{نقش} & \textbf{جغرافیا} & \textbf{محتوای عملی} \\
\midrule
هاب انرژی & خاورمیانه-آسیای مرکزی & صادرات برق پاک، ترانزیت گاز، ذخیره‌سازی \\
\rowcolor{gray!10}
دالان تجاری & شمال-جنوب، شرق-غرب & INSTC، راه ابریشم جدید، بنادر چابهار و بندرعباس \\
مرکز مالی & خلیج فارس & بورس منطقه‌ای، بانکداری اسلامی، فین‌تک \\
\rowcolor{gray!10}
قطب علمی & جهان اسلام & ۱۰ دانشگاه در رتبه‌بندی جهانی \\
میانجی صلح & خاورمیانه & میزبان مذاکرات، نیروی حافظ صلح \\
\rowcolor{gray!10}
الگوی توسعه & کشورهای درحال‌گذار & انتقال تجربه، مشاوره فنی \\
\bottomrule
\end{tabular}
\end{table}

\subsubsection{سازمان‌های منطقه‌ای}

\begin{itemize}[nosep]
\item \textbf{پیمان همکاری خلیج فارس}: 
    شامل ایران، عراق، کویت، بحرین، قطر، امارات، عمان، عربستان
    \begin{itemize}[nosep]
    \item منطقه آزاد تجاری (تا سال ۱۲)
    \item پول مشترک یا سامانه پرداخت منطقه‌ای (تا سال ۱۸)
    \item مکانیزم حل اختلاف (فوری)
    \end{itemize}

\item \textbf{سازمان همکاری آسیای مرکزی و قفقاز}:
    \begin{itemize}[nosep]
    \item مدیریت مشترک منابع آب (آمودریا، ارس)
    \item شبکه انرژی متصل
    \item همکاری ضد تروریسم
    \end{itemize}

\item \textbf{اتحادیه کشورهای فارسی‌زبان}:
    ایران، افغانستان، تاجیکستان + اقلیت‌های فارسی‌زبان
    \begin{itemize}[nosep]
    \item همکاری فرهنگی و رسانه‌ای
    \item استانداردسازی زبان فارسی دیجیتال
    \item تبادل دانشگاهی
    \end{itemize}
\end{itemize}

\subsection{ستون چهارم: فرهنگ دموکراتیک عمیق}

\subsubsection{از قواعد به عادات}

\begin{olgoobox}
\textbf{الگوی موفق: دموکراسی اسکاندیناوی}

کشورهای اسکاندیناوی نمونه بارز «دموکراسی به‌مثابه عادت» هستند:
\begin{itemize}[nosep]
\item اعتماد اجتماعی بالا (بیش از ۷۰٪ مردم به یکدیگر اعتماد دارند)
\item مشارکت داوطلبانه گسترده (۵۰٪+ در سازمان‌های مدنی)
\item شفافیت به‌عنوان هنجار فرهنگی
\item رواداری و پذیرش تنوع
\item \textbf{درس کلیدی}: دموکراسی پایدار نیازمند زیرساخت فرهنگی است
\end{itemize}
\end{olgoobox}

\begin{table}[htbp]
\centering
\caption{شاخص‌های فرهنگ دموکراتیک در پایان فاز چهارم}
\label{tab:democratic-culture}
\begin{tabular}{>{\columncolor{violet!8}}r p{4.5cm} c c}
\toprule
\rowcolor{violet!25}
\textbf{شاخص} & \textbf{توضیح} & \textbf{سال ۱۰} & \textbf{هدف سال ۱۵} \\
\midrule
اعتماد بین‌فردی & «به اکثر مردم می‌توان اعتماد کرد» & ۲۵٪ & ۴۰٪ \\
\rowcolor{gray!10}
اعتماد به نهادها & میانگین اعتماد به نهادهای دولتی & ۳۵٪ & ۵۵٪ \\
رواداری سیاسی & پذیرش مخالفان به‌عنوان همسایه & ۴۵٪ & ۶۵٪ \\
\rowcolor{gray!10}
مشارکت مدنی & عضویت در حداقل یک سازمان & ۱۵٪ & ۳۵٪ \\
سواد رسانه‌ای & توانایی تشخیص اخبار جعلی & ۳۰٪ & ۶۰٪ \\
\rowcolor{gray!10}
برابری جنسیتی (نگرش) & موافقت با برابری حقوق & ۶۵٪ & ۸۵٪ \\
\bottomrule
\end{tabular}
\end{table}

\subsubsection{برنامه‌های فرهنگ‌سازی}

\begin{enumerate}[nosep]
\item \textbf{آموزش شهروندی}:
    \begin{itemize}[nosep]
    \item درس «شهروندی و دموکراسی» از ابتدایی تا دانشگاه
    \item شبیه‌سازی انتخابات و مجلس در مدارس
    \item اردوهای آشنایی با تنوع قومی-فرهنگی
    \end{itemize}

\item \textbf{رسانه‌های مسئول}:
    \begin{itemize}[nosep]
    \item شورای اخلاق رسانه‌ای خودانتظام
    \item حمایت از روزنامه‌نگاری تحقیقی
    \item پلتفرم ملی fact-checking
    \end{itemize}

\item \textbf{گفتگوی ملی مستمر}:
    \begin{itemize}[nosep]
    \item جشنواره سالانه دموکراسی
    \item دیالوگ‌های بین‌قومی و بین‌مذهبی
    \item یادبود قربانیان استبداد
    \end{itemize}
\end{enumerate}

\subsection{تقویم فاز چهارم}

\begin{table}[htbp]
\centering
\caption{نقاط عطف کلیدی فاز چهارم (سال ۱۱-۱۵)}
\label{tab:phase4-timeline}
\begin{tabular}{>{\columncolor{blue!8}}c p{3.5cm} p{6cm}}
\toprule
\rowcolor{blue!25}
\textbf{سال} & \textbf{رویداد کلیدی} & \textbf{شاخص‌های موفقیت} \\
\midrule
۱۱ & آغاز برنامه پنج‌ساله سوم & تصویب در مجلس با اجماع گسترده \\
\rowcolor{gray!10}
۱۲ & انتخابات ریاست‌جمهوری & چرخش مسالمت‌آمیز قدرت (دوم یا سوم) \\
۱۲ & راه‌اندازی منطقه آزاد خلیج فارس & امضای پیمان توسط ۶+ کشور \\
\rowcolor{gray!10}
۱۳ & میزبانی نشست بین‌المللی & اولین اجلاس سران در تهران پس از گذار \\
۱۴ & ارزیابی میان‌دوره‌ای & دستیابی به ۸۰٪ اهداف فاز \\
\rowcolor{gray!10}
۱۵ & پایان فاز چهارم & اعلام «دموکراسی تثبیت‌شده» \\
\bottomrule
\end{tabular}
\end{table}

%═══════════════════════════════════════════════════════════════════════════════
\section{فاز پنجم: تعالی (سال ۱۶-۲۵)}
\label{sec:phase5}
%═══════════════════════════════════════════════════════════════════════════════

\subsection{چشم‌انداز ایران ۱۴۲۹}

\begin{center}
\begin{tikzpicture}
% کادر اصلی
\node[rectangle, rounded corners=15pt, draw=green!70!black, fill=green!5,
      thick, minimum width=14cm, text width=14cm, minimum height=10cm] (main) {};

% عنوان
\node[above=0.2cm of main.north, font=\Large\bfseries] 
    {چشم‌انداز ایران ۱۴۲۹: کشوری آباد، آزاد، سربلند};

% محورها
\node[rectangle, rounded corners=8pt, draw=blue!70, fill=blue!15,
      minimum width=4cm, text width=4cm, minimum height=1.5cm, align=center]
      at (-4, 3) (p1) {\textbf{سیاسی}\\ دموکراسی پایدار};
      
\node[rectangle, rounded corners=8pt, draw=green!70, fill=green!15,
      minimum width=4cm, text width=4cm, minimum height=1.5cm, align=center]
      at (0, 3) (p2) {\textbf{اقتصادی}\\ ۲۵ اقتصاد برتر};
      
\node[rectangle, rounded corners=8pt, draw=orange!70, fill=orange!15,
      minimum width=4cm, text width=4cm, minimum height=1.5cm, align=center]
      at (4, 3) (p3) {\textbf{اجتماعی}\\ HDI بالای ۰.۸۵};

\node[rectangle, rounded corners=8pt, draw=purple!70, fill=purple!15,
      minimum width=4cm, text width=4cm, minimum height=1.5cm, align=center]
      at (-4, 0.5) (p4) {\textbf{فرهنگی}\\ قطب تمدنی جهان};
      
\node[rectangle, rounded corners=8pt, draw=teal!70, fill=teal!15,
      minimum width=4cm, text width=4cm, minimum height=1.5cm, align=center]
      at (0, 0.5) (p5) {\textbf{محیط‌زیستی}\\ پایداری و سبز};
      
\node[rectangle, rounded corners=8pt, draw=red!70, fill=red!15,
      minimum width=4cm, text width=4cm, minimum height=1.5cm, align=center]
      at (4, 0.5) (p6) {\textbf{بین‌المللی}\\ الگوی منطقه‌ای};

% شعار
\node[rectangle, rounded corners=10pt, draw=yellow!70!black, fill=yellow!20,
      thick, minimum width=10cm, text width=10cm, minimum height=1.2cm, align=center]
      at (0, -2) {\large\textbf{«ایرانی آباد برای همه ایرانیان — در صلح با جهان»}};
\end{tikzpicture}
\end{center}

\subsection{شاخص‌های کلان هدف سال ۲۵}

\begin{table}[htbp]
\centering
\caption{اهداف کلان ایران در پایان سال ۲۵ (۱۴۲۹)}
\label{tab:vision-2050}
\begin{tabular}{>{\columncolor{green!8}}r p{4cm} c c c}
\toprule
\rowcolor{green!25}
\textbf{حوزه} & \textbf{شاخص} & \textbf{۱۴۰۳} & \textbf{هدف ۱۴۲۹} & \textbf{معادل جهانی} \\
\midrule
اقتصاد & GDP سرانه (PPP) & \$۱۵,۰۰۰ & \$۴۵,۰۰۰ & پرتغال امروز \\
\rowcolor{gray!10}
اقتصاد & رتبه اقتصادی جهان & ۲۱ & ۱۵ & — \\
توسعه انسانی & شاخص HDI & ۰.۷۷۴ & ۰.۸۷۰ & کره جنوبی امروز \\
\rowcolor{gray!10}
سلامت & امید به زندگی & ۷۷ سال & ۸۳ سال & ژاپن امروز \\
آموزش & میانگین سال‌های تحصیل & ۱۰ سال & ۱۴ سال & آلمان امروز \\
\rowcolor{gray!10}
دموکراسی & شاخص دموکراسی (EIU) & — & ۸+ از ۱۰ & «دموکراسی کامل» \\
فساد & شاخص CPI & ۲۵ & ۷۰+ & اسپانیا امروز \\
\rowcolor{gray!10}
محیط‌زیست & انتشار CO2 سرانه & ۸.۵ تن & ۴ تن & میانگین جهانی \\
نوآوری & شاخص GII & ۶۰+ & ۲۵ & در ۳۰ کشور برتر \\
\bottomrule
\end{tabular}
\end{table}

\subsection{محور اول: الگوی توسعه برای منطقه}

\subsubsection{ایران به‌عنوان مدل}

\begin{naghlbox}
«موفقیت واقعی یک گذار دموکراتیک زمانی است که کشورهای دیگر بخواهند از آن الگوبرداری کنند. ایران با تاریخ و جغرافیای منحصربه‌فردش می‌تواند نشان دهد که دموکراسی و توسعه با هویت بومی سازگارند.»
\sourceline{نویسنده}
\end{naghlbox}

\textbf{ابعاد الگوسازی}:

\begin{enumerate}[nosep]
\item \textbf{گذار موفق از اقتدارگرایی}: 
    مدل ایرانی گذار می‌تواند برای کشورهای مشابه (مصر، الجزایر، پاکستان) الهام‌بخش باشد
    
\item \textbf{مدیریت تنوع قومی}:
    نشان‌دادن امکان وحدت در کثرت بدون تجزیه
    
\item \textbf{اسلام و دموکراسی}:
    اثبات سازگاری دموکراسی با جوامع مسلمان
    
\item \textbf{توسعه پایدار}:
    مدیریت بحران آب و انتقال انرژی در شرایط دشوار
    
\item \textbf{رهایی از تحریم}:
    الگویی برای بازگشت به جامعه بین‌الملل
\end{enumerate}

\subsubsection{مؤسسات انتقال تجربه}

\begin{table}[htbp]
\centering
\caption{نهادهای انتقال تجربه ایرانی به جهان}
\label{tab:knowledge-transfer}
\begin{tabular}{>{\columncolor{blue!8}}r p{3.5cm} p{6cm}}
\toprule
\rowcolor{blue!25}
\textbf{نهاد} & \textbf{مأموریت} & \textbf{فعالیت‌های کلیدی} \\
\midrule
آکادمی گذار دموکراتیک تهران & آموزش فعالان مدنی منطقه & دوره‌های ۳-۶ ماهه، بورسیه سالانه ۵۰۰ نفر \\
\rowcolor{gray!10}
مرکز تنوع و همزیستی & مطالعات قومی-مذهبی & تحقیقات، میانجی‌گری، مشاوره \\
بنیاد توسعه پایدار ایران & همکاری فنی & انتقال فناوری آب، انرژی، کشاورزی \\
\rowcolor{gray!10}
شبکه زنان خاورمیانه & توانمندسازی زنان & حمایت از فعالان زن، شبکه‌سازی \\
رسانه بین‌المللی فارسی & دیپلماسی عمومی & پخش به ۲۰+ کشور، محتوای چندزبانه \\
\bottomrule
\end{tabular}
\end{table}

\subsection{محور دوم: نوآوری و فناوری پیشرفته}

\subsubsection{اهداف بلندپروازانه فناوری}

\begin{center}
\begin{tikzpicture}
\begin{axis}[
    width=13cm,
    height=7cm,
    xlabel={سال},
    ylabel={رتبه جهانی (کمتر بهتر)},
    xmin=0, xmax=26,
    ymin=0, ymax=80,
    xtick={0,5,10,15,20,25},
    xticklabels={۱۴۰۳, سال ۵, سال ۱۰, سال ۱۵, سال ۲۰, سال ۲۵},
    ytick={0,10,20,30,40,50,60,70,80},
    legend pos=north east,
    grid=major,
    grid style={dashed, gray!30}
]
% شاخص نوآوری
\addplot[color=blue, mark=*, thick, mark size=3pt] coordinates {
    (0, 65) (5, 55) (10, 45) (15, 35) (20, 28) (25, 22)
};
% فناوری اطلاعات
\addplot[color=green!70!black, mark=square*, thick, mark size=3pt] coordinates {
    (0, 70) (5, 55) (10, 40) (15, 30) (20, 22) (25, 18)
};
% آموزش عالی
\addplot[color=red, mark=triangle*, thick, mark size=3pt] coordinates {
    (0, 50) (5, 45) (10, 38) (15, 32) (20, 27) (25, 23)
};
% ثبت اختراع
\addplot[color=orange, mark=diamond*, thick, mark size=3pt] coordinates {
    (0, 40) (5, 35) (10, 28) (15, 22) (20, 18) (25, 15)
};

\legend{شاخص نوآوری جهانی, توسعه ICT, کیفیت آموزش عالی, ثبت اختراع بین‌المللی}
\end{axis}
\end{tikzpicture}
\captionof{figure}{مسیر صعود ایران در شاخص‌های فناوری و نوآوری}
\label{fig:tech-trajectory}
\end{center}

\subsubsection{پروژه‌های کلان فناوری}

\begin{table}[htbp]
\centering
\caption{پروژه‌های کلان فناوری در فاز پنجم}
\label{tab:mega-projects}
\begin{tabular}{>{\columncolor{orange!8}}r p{3cm} p{2.5cm} p{4cm}}
\toprule
\rowcolor{orange!25}
\textbf{پروژه} & \textbf{هدف} & \textbf{سرمایه‌گذاری} & \textbf{دستاورد مورد انتظار} \\
\midrule
شهر هوشمند تهران ۲.۰ & بازسازی پایتخت & ۵۰ میلیارد دلار & کاهش ۵۰٪ آلودگی، ترافیک \\
\rowcolor{gray!10}
شبکه ملی برق هوشمند & انرژی پایدار & ۳۰ میلیارد دلار & ۶۰٪ تجدیدپذیر \\
کریدور علم و فناوری & اقتصاد دانش‌بنیان & ۲۰ میلیارد دلار & ۵ شهرک فناوری \\
\rowcolor{gray!10}
زیرساخت فضایی & استقلال فضایی & ۱۵ میلیارد دلار & ماهواره‌های بومی، خدمات سنجش \\
ابررایانه ایرانی & محاسبات پیشرفته & ۵ میلیارد دلار & در ۵۰ ابررایانه برتر جهان \\
\rowcolor{gray!10}
هاب زیست‌فناوری & صنعت دارو & ۱۰ میلیارد دلار & صادرات ۱۰ میلیارد دلاری \\
\bottomrule
\end{tabular}
\end{table}

\subsection{محور سوم: کیفیت زندگی در سطح OECD}

\subsubsection{شاخص‌های کیفیت زندگی}

\begin{table}[htbp]
\centering
\caption{اهداف کیفیت زندگی در پایان سال ۲۵}
\label{tab:quality-of-life}
\begin{tabular}{>{\columncolor{teal!8}}r p{3.5cm} c c c}
\toprule
\rowcolor{teal!25}
\textbf{حوزه} & \textbf{شاخص} & \textbf{۱۴۰۳} & \textbf{۱۴۲۹} & \textbf{میانگین OECD} \\
\midrule
درآمد & درآمد خانوار (سرانه ماهانه) & ۵۰۰\$ & ۲,۵۰۰\$ & ۳,۰۰۰\$ \\
\rowcolor{gray!10}
مسکن & متراژ سرانه مسکن & ۲۵ م۲ & ۴۰ م۲ & ۴۰ م۲ \\
سلامت & تخت بیمارستان/۱۰۰۰ نفر & ۱.۵ & ۴ & ۵ \\
\rowcolor{gray!10}
آموزش & نرخ دانشگاه‌رفتگان & ۳۵٪ & ۶۰٪ & ۵۰٪ \\
اشتغال & نرخ بیکاری & ۱۲٪ & ۵٪ & ۵٪ \\
\rowcolor{gray!10}
تعادل کار-زندگی & ساعات کار هفتگی & ۴۸ & ۴۰ & ۳۸ \\
امنیت & نرخ جرم (قتل/۱۰۰هزار) & ۳ & ۱.۵ & ۲ \\
\rowcolor{gray!10}
محیط‌زیست & روزهای هوای پاک تهران & ۸۰ & ۲۵۰ & — \\
رضایت & رضایت از زندگی (۱-۱۰) & ۴.۵ & ۷ & ۷ \\
\bottomrule
\end{tabular}
\end{table}

\subsubsection{تأمین اجتماعی فراگیر}

\begin{olgoobox}
\textbf{الگوی موفق: دولت رفاه اسکاندیناوی}

مدل اسکاندیناوی نشان می‌دهد رشد اقتصادی و عدالت اجتماعی با هم سازگارند:
\begin{itemize}[nosep]
\item پوشش بیمه‌ای ۱۰۰٪ جمعیت
\item آموزش رایگان از مهدکودک تا دکتری
\item مرخصی والدین ۱۲+ ماه با حقوق
\item بیمه بیکاری سخاوتمندانه با آموزش اجباری
\item مستمری بازنشستگی تضمین‌شده
\item \textbf{نتیجه}: کمترین نابرابری، بالاترین شادی در جهان
\end{itemize}
\end{olgoobox}

\begin{table}[htbp]
\centering
\caption{برنامه تأمین اجتماعی فراگیر تا سال ۲۵}
\label{tab:social-security}
\begin{tabular}{>{\columncolor{purple!8}}r p{5cm} p{4.5cm}}
\toprule
\rowcolor{purple!25}
\textbf{برنامه} & \textbf{پوشش هدف} & \textbf{منابع مالی} \\
\midrule
بیمه سلامت همگانی & ۱۰۰٪ جمعیت & مالیات بر درآمد، مالیات بر مصرف، مالیات تصاعدی، کاهش هزینه‌ی نظامی\\
\rowcolor{gray!10}
بازنشستگی پایدار & حداقل مستمری ۶۰٪ حداقل دستمزد & اصلاح صندوق‌ها، سرمایه‌گذاری \\
بیمه بیکاری & ۸۰٪ حقوق به مدت ۱۲ ماه & مشارکت کارفرما و دولت \\
\rowcolor{gray!10}
حمایت از خانواده & کمک‌هزینه فرزند تا ۱۸ سالگی & بودجه عمومی \\
مسکن اجتماعی & ۲ میلیون واحد مسکن حمایتی & زمین دولتی، وام کم‌بهره \\
\rowcolor{gray!10}
آموزش رایگان & مهدکودک تا کارشناسی & بودجه آموزش ۶٪ GDP \\
\bottomrule
\end{tabular}
\end{table}

\subsubsection{شهرها و زیرساخت‌ها}

\begin{table}[htbp]
\centering
\caption{برنامه توسعه شهری و زیرساختی تا سال ۲۵}
\label{tab:urban-infrastructure}
\begin{tabular}{>{\columncolor{cyan!8}}r p{3cm} c c p{3.5cm}}
\toprule
\rowcolor{cyan!25}
\textbf{حوزه} & \textbf{شاخص} & \textbf{۱۴۰۳} & \textbf{۱۴۲۹} & \textbf{پروژه‌های کلیدی} \\
\midrule
حمل‌ونقل ریلی & طول مترو (کیلومتر) & ۳۵۰ & ۱,۵۰۰ & مترو ۱۰ کلان‌شهر \\
\rowcolor{gray!10}
راه‌آهن & شبکه راه‌آهن سریع & ۱۵۰ کیلومتر & ۳,۰۰۰ کیلومتر & تهران-مشهد-اصفهان \\
فرودگاه & ظرفیت سالانه (میلیون) & ۵۰ & ۱۵۰ & فرودگاه جدید تهران \\
\rowcolor{gray!10}
بندر & ظرفیت کانتینری (TEU) & ۳ میلیون & ۱۵ میلیون & توسعه چابهار \\
دیجیتال & پوشش اینترنت پرسرعت & ۶۰٪ & ۹۸٪ & فیبر نوری سراسری \\
\rowcolor{gray!10}
انرژی & ظرفیت تجدیدپذیر & ۵٪ & ۶۰٪ & ۵۰ گیگاوات خورشیدی \\
\bottomrule
\end{tabular}
\end{table}

\subsection{محور چهارم: محیط‌زیست و پایداری}

\subsubsection{بحران آب: راه‌حل نهایی}

\begin{figure}[htbp]
\centering
\begin{tikzpicture}
\begin{axis}[
    width=13cm,
    height=8cm,
    xlabel={\rl{سال}},
    ylabel={\rl{میلیارد مترمکعب}},
    xmin=0, xmax=26,
    ymin=0, ymax=150,
    xtick={0,5,10,15,20,25},
    xticklabels={\rl{۱۴۰۳}, \rl{سال ۵}, \rl{سال ۱۰}, \rl{سال ۱۵}, \rl{سال ۲۰}, \rl{سال ۲۵}},
    legend style={at={(0.5,-0.2)}, anchor=north, legend columns=2, font=\tiny},
    grid=major,
    axis line style={bleurepublique},
    tick style={bleurepublique},
    label style={font=\tiny\bfseries, color=bleurepublique}
]

% عرضه آب
\addplot[color=bleurepublique, mark=*, ultra thick] coordinates {
    (0, 90) (5, 88) (10, 92) (15, 100) (20, 105) (25, 110)
};
% تقاضا (با مدیریت)
\addplot[color=goldphoenix, mark=square*, ultra thick] coordinates {
    (0, 105) (5, 98) (10, 90) (15, 85) (20, 82) (25, 80)
};
% تقاضا (بدون مدیریت)
\addplot[color=bleurepublique!30, mark=triangle*, dashed, thick] coordinates {
    (0, 105) (5, 112) (10, 120) (15, 128) (20, 135) (25, 142)
};

\legend{\rl{عرضه پایدار}, \rl{تقاضا (با مدیریت)}, \rl{تقاضا (بدون مدیریت)}}

% ناحیه تعادل
\fill[goldlight, opacity=0.3] (axis cs:12,80) rectangle (axis cs:26,120);
\node[font=\tiny\bfseries, color=goldphoenix] at (axis cs:20,100) {\rl{ناحیه پایداری آبی}};

\end{axis}
\end{tikzpicture}
\caption{مسیر دستیابی به تعادل آبی تا سال ۲۵}
\end{figure}

\begin{table}[htbp]
\centering
\caption{استراتژی جامع مدیریت آب در فاز پنجم}
\label{tab:water-strategy-phase5}
\begin{tabular}{>{\columncolor{blue!8}}r p{2.5cm} p{2cm} p{5cm}}
\toprule
\rowcolor{blue!25}
\textbf{راهکار} & \textbf{پتانسیل (MCM/سال)} & \textbf{هزینه} & \textbf{اقدامات کلیدی} \\
\midrule
کاهش مصرف کشاورزی & ۲۵,۰۰۰ & متوسط & آبیاری نوین ۱۰۰٪، تغییر الگوی کشت \\
\rowcolor{gray!10}
بازیافت فاضلاب & ۸,۰۰۰ & متوسط & تصفیه‌خانه ۱۰۰ شهر، استفاده صنعتی-کشاورزی \\
شیرین‌سازی & ۵,۰۰۰ & بالا & ۲۰ واحد خلیج فارس و عمان \\
\rowcolor{gray!10}
جمع‌آوری باران & ۳,۰۰۰ & پایین & سازه‌های تغذیه مصنوعی \\
کاهش تبخیر & ۲,۰۰۰ & متوسط & پوشش کانال‌ها، لوله‌کشی \\
\rowcolor{gray!10}
انتقال بین‌حوضه‌ای & ۲,۰۰۰ & بالا & پروژه‌های محدود و پایدار \\
\midrule
\textbf{مجموع} & \textbf{۴۵,۰۰۰} & — & رفع کامل کسری آب \\
\bottomrule
\end{tabular}
\end{table}

\subsubsection{انتقال انرژی}

\begin{enghelabbox}
\textbf{هشدار: پنجره زمانی محدود}

ایران برای انتقال انرژی پنجره زمانی محدودی دارد:
\begin{itemize}[nosep]
\item ذخایر نفت با نرخ فعلی: ۸۰-۱۰۰ سال
\item اما تقاضای جهانی برای نفت تا ۲۰۵۰ به شدت کاهش می‌یابد
\item اگر منتظر بمانیم، منابع نفتی بی‌ارزش می‌شوند (stranded assets)
\item \textbf{فرصت}: استفاده از درآمدهای فعلی نفت برای سرمایه‌گذاری در انرژی پاک
\item هدف: ایران از صادرکننده نفت به صادرکننده برق پاک و هیدروژن سبز تبدیل شود
\end{itemize}
\end{enghelabbox}

\begin{table}[htbp]
\centering
\caption{اهداف انتقال انرژی ایران تا سال ۲۵}
\label{tab:energy-transition}
\begin{tabular}{>{\columncolor{green!8}}r p{4cm} c c}
\toprule
\rowcolor{green!25}
\textbf{شاخص} & \textbf{توضیح} & \textbf{۱۴۰۳} & \textbf{۱۴۲۹} \\
\midrule
سهم تجدیدپذیر در برق & خورشیدی، بادی، آبی & ۸٪ & ۶۰٪ \\
\rowcolor{gray!10}
ظرفیت خورشیدی & گیگاوات نصب‌شده & ۱ & ۵۰ \\
ظرفیت بادی & گیگاوات نصب‌شده & ۰.۵ & ۲۰ \\
\rowcolor{gray!10}
خودروهای برقی & درصد فروش سالانه & ۱٪ & ۸۰٪ \\
انتشار CO2 & تن سرانه & ۸.۵ & ۴ \\
\rowcolor{gray!10}
صادرات برق پاک & میلیارد دلار سالانه & ۰.۵ & ۱۰ \\
تولید هیدروژن سبز & میلیون تن سالانه & ۰ & ۵ \\
\rowcolor{gray!10}
کارایی انرژی صنعتی & بهبود نسبت به ۱۴۰۳ & — & ۴۰٪+ \\
\bottomrule
\end{tabular}
\end{table}

\subsubsection{بازسازی محیط‌زیست}

\begin{table}[htbp]
\centering
\caption{اهداف زیست‌محیطی در افق ۲۵ ساله}
\label{tab:environment-goals}
\begin{tabular}{>{\columncolor{teal!8}}r p{4cm} c c}
\toprule
\rowcolor{teal!25}
\textbf{حوزه} & \textbf{شاخص} & \textbf{۱۴۰۳} & \textbf{۱۴۲۹} \\
\midrule
جنگل‌ها & پوشش جنگلی کشور & ۷٪ & ۱۲٪ \\
\rowcolor{gray!10}
دریاچه ارومیه & حجم آب (میلیارد م۳) & ۲ & ۱۵ (احیای کامل) \\
تالاب‌ها & تالاب‌های احیاشده & ۳۰٪ & ۸۰٪ \\
\rowcolor{gray!10}
مناطق حفاظت‌شده & درصد مساحت کشور & ۱۰٪ & ۱۸٪ \\
کیفیت هوای شهرها & روزهای سالم (تهران) & ۸۰ & ۳۰۰ \\
\rowcolor{gray!10}
بازیافت زباله & درصد بازیافت شهری & ۱۰٪ & ۶۰٪ \\
فرسایش خاک & کاهش نسبت به ۱۴۰۳ & — & ۵۰٪ \\
\bottomrule
\end{tabular}
\end{table}

\subsection{محور پنجم: جایگاه بین‌المللی}

\subsubsection{ایران در نظم جهانی}

\begin{figure}[htbp]
\centering
\begin{tikzpicture}[
    scale=0.85,
    transform shape,
    org/.style={
        rectangle,
        rounded corners=5pt,
        minimum width=3.2cm, text width=3.2cm,
        minimum height=1.5cm,
        align=center,
        font=\tiny\bfseries,
        draw=bleurepublique,
        fill=bleulight,
        line width=1.2pt
    },
    flow/.style={-, thick, bleurepublique!50, dashed}
]

% مرکز
\node[rectangle, rounded corners=8pt, draw=goldphoenix, fill=goldphoenix, text=white,
      minimum width=4cm, text width=4cm, minimum height=1.8cm, align=center, font=\small\bfseries] (iran) at (0,0) {\rl{ایران ۱۴۲۹}\\ \tiny \rl{قدرت متوسط جهانی}};

% سازمان‌ها
\node[org] (un) at (-4,2.5) {\rl{سازمان ملل}\\ \tiny \rl{عضو فعال حاکمیتی}};
\node[org] (wto) at (4,2.5) {\rl{WTO}\\ \tiny \rl{عضویت کامل و فعال}};
\node[org] (oecd) at (-4,-2.5) {\rl{OECD}\\ \tiny \rl{همسویی با استانداردها}};
\node[org] (gulf) at (4,-2.5) {\rl{پیمان پارس}\\ \tiny \rl{بنیان‌گذار نظم منطقه}};

% اتصالات
\draw[flow] (iran) -- (un);
\draw[flow] (iran) -- (wto);
\draw[flow] (iran) -- (oecd);
\draw[flow] (iran) -- (gulf);

\end{tikzpicture}
\caption{جایگاه ایران در سازمان‌های بین‌المللی تا سال ۲۵}
\end{figure}

\subsubsection{روابط با قدرت‌های بزرگ}

\begin{table}[htbp]
\centering
\caption{چشم‌انداز روابط ایران با قدرت‌های جهانی در سال ۲۵}
\label{tab:great-power-relations}
\begin{tabular}{>{\columncolor{violet!8}}r p{2.5cm} p{3cm} p{4cm}}
\toprule
\rowcolor{violet!25}
\textbf{کشور/بلوک} & \textbf{ماهیت رابطه} & \textbf{حوزه همکاری} & \textbf{چالش‌های احتمالی} \\
\midrule
اتحادیه اروپا & مشارکت استراتژیک & تجارت، انرژی، فناوری & مهاجرت، حقوق بشر \\
\rowcolor{gray!10}
آمریکا & همکاری رقابتی & تجارت، امنیت منطقه‌ای & اسرائیل، رقابت منطقه‌ای \\
چین & مشارکت اقتصادی & راه ابریشم، انرژی & وابستگی نامتقارن \\
\rowcolor{gray!10}
روسیه & همکاری انتخابی & انرژی، امنیت & دریای خزر، آسیای مرکزی \\
هند & شراکت رو به رشد & ترانزیت، انرژی، فناوری & پاکستان \\
\rowcolor{gray!10}
ژاپن-کره & همکاری اقتصادی & سرمایه‌گذاری، فناوری & محدود \\
\bottomrule
\end{tabular}
\end{table}

\subsubsection{دیپلماسی فرهنگی}

\begin{table}[htbp]
\centering
\caption{برنامه قدرت نرم و دیپلماسی فرهنگی}
\label{tab:soft-power}
\begin{tabular}{>{\columncolor{red!8}}r p{5cm} p{4.5cm}}
\toprule
\rowcolor{red!25}
\textbf{ابزار} & \textbf{هدف} & \textbf{شاخص موفقیت} \\
\midrule
گردشگری & جذب ۲۰ میلیون گردشگر سالانه & درآمد ۵۰ میلیارد دلار \\
\rowcolor{gray!10}
سینما و هنر & حضور در جشنواره‌های بین‌المللی & ۳ جایزه بزرگ سینمایی \\
ورزش & میزبانی رویدادهای بزرگ & جام جهانی، المپیک \\
\rowcolor{gray!10}
زبان فارسی & گسترش آموزش فارسی & ۱ میلیون زبان‌آموز غیرایرانی \\
مراکز فرهنگی & خانه‌های ایران در جهان & ۱۰۰ مرکز در ۵۰ کشور \\
\rowcolor{gray!10}
دانشگاه‌ها & جذب دانشجوی بین‌المللی & ۱۰۰,۰۰۰ دانشجوی خارجی \\
\bottomrule
\end{tabular}
\end{table}

\subsection{تقویم فاز پنجم}

\begin{table}[htbp]
\centering
\caption{نقاط عطف کلیدی فاز پنجم (سال ۱۶-۲۵)}
\label{tab:phase5-timeline}
\begin{tabular}{>{\columncolor{green!8}}c p{3.5cm} p{6cm}}
\toprule
\rowcolor{green!25}
\textbf{سال} & \textbf{رویداد کلیدی} & \textbf{شاخص موفقیت} \\
\midrule
۱۶ & آغاز برنامه پنج‌ساله چهارم & تمرکز بر تعالی و کیفیت \\
\rowcolor{gray!10}
۱۷ & عضویت کامل در WTO & رفع کامل تحریم‌ها \\
۱۸ & انتخابات ریاست‌جمهوری & چهارمین چرخش مسالمت‌آمیز \\
\rowcolor{gray!10}
۱۸ & راه‌اندازی پول مشترک منطقه‌ای & پیمان پولی خلیج فارس \\
۲۰ & میزبانی بازی‌های آسیایی & نمایش ایران نوین به جهان \\
\rowcolor{gray!10}
۲۱ & آغاز برنامه پنج‌ساله پنجم & تمرکز بر پایداری \\
۲۲ & GDP سرانه ۴۰,۰۰۰ دلار & ورود به گروه کشورهای پردرآمد \\
\rowcolor{gray!10}
۲۳ & تعادل آبی کامل & پایان بحران آب \\
۲۴ & انتخابات ریاست‌جمهوری & پنجمین چرخش مسالمت‌آمیز \\
\rowcolor{gray!10}
۲۵ & جشن ۲۵ سالگی دموکراسی & اعلام موفقیت گذار و تعالی \\
\bottomrule
\end{tabular}
\end{table}

%═══════════════════════════════════════════════════════════════════════════════
\section{شاخص‌های جامع پایش بلندمدت}
\label{sec:long-term-monitoring}
%═══════════════════════════════════════════════════════════════════════════════

\subsection{داشبورد ملی پیشرفت}

\begin{figure}[htbp]
\centering
\begin{tikzpicture}[
    scale=0.85,
    transform shape,
    box/.style={
        rectangle,
        rounded corners=5pt,
        minimum width=4cm, text width=4cm,
        minimum height=2.2cm,
        align=center,
        font=\small\bfseries,
        draw=bleurepublique,
        fill=bleulight,
        line width=1.2pt
    },
    label/.style={font=\tiny\bfseries, color=goldphoenix}
]

% کادر داشبورد
\node[rectangle, rounded corners=10pt, draw=bleurepublique!50, fill=gray!5,
      minimum width=14cm, text width=14cm, minimum height=9cm] (dash) at (0,0) {};

% ردیف اول
\node[box] (b1) at (-4.5,2.5) {\rl{دموکراسی (EIU)}\\ \tiny \rl{هدف: ۸.۵/۱۰}\\\rl{\tiny وضعیت: کامل}};
\node[box] (b2) at (0,2.5) {\rl{توسعه انسانی (HDI)}\\ \tiny \rl{هدف: ۰.۸۷}\\\rl{\tiny رتبه: استاندارد OECD}};
\node[box] (b3) at (4.5,2.5) {\rl{ضریب جینی}\\ \tiny \rl{هدف: ۰.۳۵}\\\rl{\tiny بازتوزیع عادلانه ثروت}};

% ردیف دوم
\node[box] (b4) at (-4.5,-0.5) {\rl{فساد (CPI)}\\ \tiny \rl{هدف: ۷۰/۱۰۰}\\\rl{\tiny شفافیت سیستمی}};
\node[box] (b5) at (0,-0.5) {\rl{محیط‌زیست (EPI)}\\ \tiny \rl{هدف: ۶۰/۱۰۰}\\\rl{\tiny پایداری اکولوژیک}};
\node[box] (b6) at (4.5,-0.5) {\rl{نوآوری (GII)}\\ \tiny \rl{رتبه هدف: ۲۵}\\\rl{\tiny اقتصاد دانش‌بنیان}};

\node[below=0.2cm of dash, font=\bfseries, color=bleurepublique] {\rl{داشبورد ملی پایش پیشرفت ایران ۱۴۲۹}};

\end{tikzpicture}
\caption{داشبورد شاخص‌های کلیدی پیشرفت ملی در فاز تعالی}
\end{figure}

\subsection{مسیر حرکت شاخص‌ها}

\begin{table}[htbp]
\centering
\caption{مسیر حرکت شاخص‌های کلیدی در افق ۲۵ ساله}
\label{tab:indicator-trajectory}
\small
\begin{tabular}{>{\columncolor{gray!8}}r l c c c c c c}
\toprule
\rowcolor{gray!25}
& \textbf{شاخص} & \textbf{۱۴۰۳} & \textbf{س۵} & \textbf{س۱۰} & \textbf{س۱۵} & \textbf{س۲۰} & \textbf{س۲۵} \\
\midrule
۱ & دموکراسی (EIU) & ۲.۲ & ۵.۰ & ۶.۵ & ۷.۵ & ۸.۰ & ۸.۵ \\
\rowcolor{gray!10}
۲ & HDI & ۰.۷۷ & ۰.۷۹ & ۰.۸۲ & ۰.۸۴ & ۰.۸۶ & ۰.۸۷ \\
۳ & GDP سرانه (\$K) & ۱۵ & ۲۰ & ۲۸ & ۳۵ & ۴۰ & ۴۵ \\
\rowcolor{gray!10}
۴ & CPI فساد & ۲۵ & ۳۵ & ۴۵ & ۵۵ & ۶۵ & ۷۰ \\
۵ & نرخ بیکاری (\%) & ۱۲ & ۱۰ & ۸ & ۶ & ۵ & ۵ \\
\rowcolor{gray!10}
۶ & تورم (\%) & ۴۵ & ۱۵ & ۷ & ۴ & ۳ & ۲ \\
۷ & ضریب جینی & ۰.۴۲ & ۰.۴۰ & ۰.۳۸ & ۰.۳۷ & ۰.۳۶ & ۰.۳۵ \\
\rowcolor{gray!10}
۸ & زنان در مجلس (\%) & ۶ & ۲۰ & ۲۵ & ۳۵ & ۴۰ & ۴۵ \\
۹ & امید به زندگی (سال) & ۷۷ & ۷۸ & ۸۰ & ۸۱ & ۸۲ & ۸۳ \\
\rowcolor{gray!10}
۱۰ & سهم تجدیدپذیر (\%) & ۸ & ۱۵ & ۲۵ & ۴۰ & ۵۰ & ۶۰ \\
\bottomrule
\end{tabular}
\end{table}

%═══════════════════════════════════════════════════════════════════════════════
\section{ریسک‌ها و سناریوهای جایگزین}
\label{sec:risks-scenarios}
%═══════════════════════════════════════════════════════════════════════════════

\subsection{ریسک‌های بلندمدت}

\begin{enghelabbox}
\textbf{هشدار: موفقیت تضمین‌شده نیست}

حتی با بهترین برنامه‌ریزی، ریسک‌های جدی می‌توانند مسیر را منحرف کنند:
\begin{itemize}[nosep]
\item \textbf{خستگی دموکراتیک}: کاهش مشارکت و علاقه مردم پس از هیجان اولیه
\item \textbf{پوپولیسم جدید}: ظهور رهبران کاریزماتیک با وعده‌های غیرواقعی
\item \textbf{بحران اقتصادی جهانی}: رکود بزرگ می‌تواند دستاوردها را تهدید کند
\item \textbf{تنش‌های قومی}: شکست در مدیریت تنوع می‌تواند به تجزیه‌طلبی بینجامد
\item \textbf{تغییرات اقلیمی}: خشکسالی‌های شدیدتر از پیش‌بینی
\item \textbf{فناوری مخرب}: هوش مصنوعی و اتوماسیون می‌تواند بیکاری انبوه ایجاد کند
\end{itemize}
\end{enghelabbox}

\subsection{سه سناریوی بلندمدت}

\begin{table}[htbp]
\centering
\caption{سه سناریوی محتمل برای ایران در سال ۲۵}
\label{tab:scenarios}
\begin{tabular}{>{\columncolor{green!8}}r p{3.5cm} p{3.5cm} p{3.5cm}}
\toprule
\rowcolor{gray!25}
\textbf{بُعد} & \textbf{سناریوی خوش‌بینانه} & \textbf{سناریوی محتمل} & \textbf{سناریوی بدبینانه} \\
\midrule
\rowcolor{green!15}
نام & ایران طلایی & ایران در حال پیشرفت & ایران سرگردان \\
\midrule
دموکراسی & دموکراسی کامل (۸.۵+) & دموکراسی ناقص (۷-۸) & رژیم هیبریدی (۵-۶) \\
\rowcolor{gray!10}
اقتصاد & \$۵۰K سرانه & \$۴۰K سرانه & \$۲۵K سرانه \\
وحدت ملی & همبستگی قوی & تنش‌های مدیریت‌شده & بحران هویت \\
\rowcolor{gray!10}
منطقه‌ای & رهبر منطقه & بازیگر مؤثر & بازیگر حاشیه‌ای \\
محیط‌زیست & تعادل پایدار & بهبود نسبی & بحران ادامه‌دار \\
\rowcolor{gray!10}
احتمال & ۲۵٪ & ۵۰٪ & ۲۵٪ \\
\bottomrule
\end{tabular}
\end{table}

\subsection{مکانیزم‌های تاب‌آوری}

برای افزایش احتمال سناریوهای مثبت و کاهش آسیب‌پذیری:

\begin{enumerate}[nosep]
\item \textbf{نهادهای ضربه‌گیر}:
    \begin{itemize}[nosep]
    \item صندوق ذخیره ارزی (هدف: ۵۰۰ میلیارد دلار)
    \item ذخایر استراتژیک غذا و انرژی
    \item شبکه امنیت اجتماعی خودکار
    \end{itemize}

\item \textbf{انعطاف‌پذیری سیاسی}:
    \begin{itemize}[nosep]
    \item مکانیزم اصلاح قانون اساسی
    \item فرآیند همه‌پرسی برای تصمیمات بزرگ
    \item دادگاه قانون اساسی به‌عنوان داور نهایی
    \end{itemize}

\item \textbf{تنوع اقتصادی}:
    \begin{itemize}[nosep]
    \item کاهش وابستگی به نفت به زیر ۱۰٪
    \item شرکای تجاری متنوع (هیچ شریک بالای ۲۰٪)
    \item اقتصاد دانش‌بنیان مقاوم به تحریم
    \end{itemize}

\item \textbf{سرمایه اجتماعی}:
    \begin{itemize}[nosep]
    \item اعتماد بین‌فردی بالا
    \item جامعه مدنی قوی
    \item رسانه‌های متکثر و مسئول
    \end{itemize}
\end{enumerate}

%═══════════════════════════════════════════════════════════════════════════════
\section{پیام پایانی: میراثی برای نسل‌های آینده}
\label{sec:legacy}
%═══════════════════════════════════════════════════════════════════════════════

\begin{naghlbox}
«ما این سرزمین را از نیاکان خود به ارث نبرده‌ایم؛ آن را از فرزندان خود به امانت گرفته‌ایم. نسل ما مسئولیت دارد ایرانی آباد، آزاد و سربلند به نسل‌های آینده تحویل دهد — نه صرفاً یک کشور، بلکه یک تمدن زنده و بالنده.»
\sourceline{نویسنده}
\end{naghlbox}

\subsection{عهد با آیندگان}

این کتاب بر یک اصل بنیادین استوار است: \textbf{عدالت بین‌نسلی}. تصمیمات امروز ما، آینده فرزندان ما را شکل می‌دهد.

\begin{olgoobox}
\textbf{چه ایرانی می‌خواهیم به نسل آینده تحویل دهیم؟}

\begin{itemize}[nosep]
\item کشوری که در آن هر کودک، فارغ از قومیت، جنسیت، یا طبقه، فرصت برابر داشته باشد
\item سرزمینی که رودخانه‌هایش جاری، جنگل‌هایش سرسبز، و هوایش پاک باشد
\item جامعه‌ای که حرمت و آزادی هر انسان در آن محترم باشد
\item ملتی که در صلح با همسایگان و جهان زندگی کند
\item تمدنی که به میراث گذشته خود افتخار کند و در عین حال به آینده بنگرد
\item دموکراسی‌ای که نه به زور، بلکه از قلب مردم برخاسته باشد
\end{itemize}
\end{olgoobox}

\subsection{آخرین سخن}

مسیر از بحران تا بالندگی طولانی و پرپیچ‌وخم است. ۲۵ سال زمان زیادی به نظر می‌رسد، اما در تاریخ یک تمدن ۲۵۰۰ ساله، لحظه‌ای بیش نیست.

نسل کنونی ایرانیان در نقطه عطفی تاریخی ایستاده است. می‌توانیم مسیر تاریخ را تغییر دهیم — اگر بخواهیم، اگر با هم باشیم، و اگر به آینده ایمان داشته باشیم.

\begin{figure}[htbp]
\centering
\begin{tikzpicture}
\node[rectangle, rounded corners=15pt, draw=goldphoenix, fill=bleulight,
      thick, minimum width=12cm, text width=12cm, minimum height=3.5cm, align=center] {
    {\Large\textbf{ایران ۱۴۲۹: یک رؤیای دست‌یافتنی}}\\[0.4cm]
    {\large آباد | آزاد | سربلند}\\[0.2cm]
    {\normalsize\rl{وطن ما، مسئولیت ما، آینده ما}}
};
\end{tikzpicture}
\end{figure}

%═══════════════════════════════════════════════════════════════════════════════
% منابع فصل
%═══════════════════════════════════════════════════════════════════════════════

\section*{منابع فصل یازدهم}
\addcontentsline{toc}{section}{منابع فصل یازدهم}

\begin{itemize}[nosep, font=\small]
\item Linz, J. J., \& Stepan, A. (1996). \textit{Problems of Democratic Transition and Consolidation}. Johns Hopkins University Press.
\item Nye, J. S. (2004). \textit{Soft Power: The Means to Success in World Politics}. Public Affairs.
\item Acemoglu, D., \& Robinson, J. A. (2012). \textit{Why Nations Fail}. Crown Business.
\item Fukuyama, F. (2014). \textit{Political Order and Political Decay}. Farrar, Straus and Giroux.
\item Diamond, L. (2019). \textit{Ill Winds: Saving Democracy from Russian Rage, Chinese Ambition, and American Complacency}. Penguin Press.
\item World Bank. (2023). \textit{World Development Indicators}.
\item UNDP. (2022). \textit{Human Development Report}.
\item Economist Intelligence Unit. (2023). \textit{Democracy Index}.
\item Transparency International. (2023). \textit{Corruption Perceptions Index}.
\item Global Innovation Index. (2023). WIPO.
\item International Energy Agency. (2023). \textit{World Energy Outlook}.
\item مرکز آمار ایران. (۱۴۰۲). \textit{سالنامه آماری کشور}.
\item موسسه مطالعات انرژی. (۱۴۰۲). \textit{ترازنامه انرژی ایران}.
\end{itemize}

\end{latex}
	
	%──────────────────────────────────────────────────────────────────────────────
	% بخش چهارم: حوزه‌های تخصصی
	%──────────────────────────────────────────────────────────────────────────────
	\part{حوزه‌های تخصصی}
	
	% ch12-diversity.tex
% فصل دوازدهم: مدیریت تنوع قومی-فرهنگی
% نویسنده: مهدی سالم | ریچموندهیل | ۱۴۰۴

\chapter{مدیریت تنوع قومی-فرهنگی: وحدت در کثرت}
\label{ch:diversity}

\begin{kholasebox}
ایران سرزمینی با تنوع قومی-زبانی غنی است: فارس‌ها (۵۵٪)، آذری‌ها (۲۰٪)، کردها (۱۰٪)، لرها (۶٪)، عرب‌ها (۳٪)، بلوچ‌ها (۲٪)، ترکمن‌ها (۱٪) و دیگر اقوام. این تنوع هم فرصت است و هم چالش. این فصل مدل \textbf{«فدرالیسم همبسته»} را ارائه می‌دهد: ساختاری که حقوق فرهنگی-زبانی اقوام را تضمین می‌کند، نابرابری‌های تاریخی را جبران می‌نماید، و در عین حال وحدت ملی و تمامیت ارضی را حفظ می‌کند. اصل راهنما: \textbf{«ایرانی بودن هویت مشترک ماست، تنوع قومی ثروت مشترک ما»}.
\end{kholasebox}

%═══════════════════════════════════════════════════════════════════════════════
\section{مقدمه: تنوع به‌مثابه ثروت یا تهدید؟}
%═══════════════════════════════════════════════════════════════════════════════

\begin{naghlbox}
«تنوع قومی ذاتاً نه خوب است نه بد. این نهادها و سیاست‌ها هستند که تعیین می‌کنند تنوع به همزیستی مسالمت‌آمیز می‌انجامد یا به خشونت و تجزیه. سوئیس و یوگسلاوی هر دو چندقومی بودند — سرنوشتشان بسیار متفاوت شد.»
\sourceline{آرنت لیپهارت، «دموکراسی در جوامع چندپاره»، ۱۹۷۷}
\end{naghlbox}

ایران در طول تاریخ همواره سرزمینی چندقومی بوده است. امپراتوری‌های ایرانی از هخامنشیان تا صفویان، تنوع را نه‌تنها تحمل، بلکه اغلب جشن می‌گرفتند. اما دوران مدرن، به‌ویژه از مشروطه به بعد، شاهد تنش فزاینده میان «ملت‌سازی متمرکز» و «حقوق اقوام» بوده است.

\subsection{دوگانه کاذب: وحدت یا تنوع}

\begin{center}
\begin{tikzpicture}[
    node distance=2cm,
    box/.style={
        rectangle,
        rounded corners=8pt,
        draw=#1!70,
        fill=#1!15,
        thick,
        minimum width=4.5cm,
        minimum height=2cm,
        align=center,
        font=\small
    }
]
% دو رویکرد افراطی
\node[box=red] (assim) {\textbf{همگون‌سازی اجباری}\\ \scriptsize سرکوب تنوع\\ \scriptsize «یک ملت، یک زبان»};
\node[box=red, right=3cm of assim] (separ) {\textbf{تجزیه‌طلبی}\\ \scriptsize انکار وحدت\\ \scriptsize «هر قوم، یک کشور»};

% راه سوم
\node[box=green, below=2.5cm of assim, xshift=4cm] (third) {\textbf{فدرالیسم همبسته}\\ \scriptsize وحدت در کثرت\\ \scriptsize «یک ایران، چند صدا»};

% فلش‌ها
\draw[->, thick, red!60] (assim) -- node[left, font=\scriptsize] {شکست} ++(0,-1.5) -| (third);
\draw[->, thick, red!60] (separ) -- node[right, font=\scriptsize] {فاجعه} ++(0,-1.5) -| (third);

% برچسب
\node[above=0.5cm of assim, xshift=4cm, font=\large\bfseries] {دو افراط و یک راه میانه};
\end{tikzpicture}
\end{center}

\subsection{چرا این موضوع حیاتی است؟}

\begin{table}[htbp]
\centering
\caption{اهمیت مدیریت صحیح تنوع قومی}
\label{tab:diversity-importance}
\begin{tabular}{>{\columncolor{blue!8}}r p{5cm} p{5cm}}
\toprule
\rowcolor{blue!25}
\textbf{بُعد} & \textbf{در صورت موفقیت} & \textbf{در صورت شکست} \\
\midrule
ثبات سیاسی & مشروعیت فراگیر نظام جدید & بی‌ثباتی مزمن، شورش‌های قومی \\
\rowcolor{gray!10}
تمامیت ارضی & حفظ مرزها با رضایت همه & ریسک تجزیه و جنگ داخلی \\
توسعه اقتصادی & بهره‌گیری از همه استعدادها & هدررفت سرمایه انسانی مناطق \\
\rowcolor{gray!10}
امنیت ملی & مرزهای امن با حمایت محلی & آسیب‌پذیری در مناطق قومی \\
وجهه بین‌المللی & الگوی همزیستی & بهانه مداخله خارجی \\
\bottomrule
\end{tabular}
\end{table}

%═══════════════════════════════════════════════════════════════════════════════
\section{نقشه قومی-زبانی ایران}
\label{sec:ethnic-map}
%═══════════════════════════════════════════════════════════════════════════════

\subsection{ترکیب جمعیتی}

\begin{table}[htbp]
\centering
\caption{ترکیب قومی-زبانی ایران (تخمین ۱۴۰۳)}
\label{tab:ethnic-composition-detail}
\begin{tabular}{>{\columncolor{teal!8}}r l r r p{4cm}}
\toprule
\rowcolor{teal!25}
\textbf{ردیف} & \textbf{قوم/زبان} & \textbf{جمعیت (میلیون)} & \textbf{درصد} & \textbf{مناطق اصلی سکونت} \\
\midrule
۱ & فارس & ۴۸ & ۵۵٪ & مرکز، شرق، جنوب \\
\rowcolor{gray!10}
۲ & آذری (ترک) & ۱۷.۵ & ۲۰٪ & آذربایجان شرقی و غربی، اردبیل، زنجان \\
۳ & کرد & ۸.۷ & ۱۰٪ & کردستان، کرمانشاه، ایلام، آذربایجان غربی \\
\rowcolor{gray!10}
۴ & لر (لر و بختیاری) & ۵.۲ & ۶٪ & لرستان، چهارمحال، خوزستان شمالی \\
۵ & عرب & ۲.۶ & ۳٪ & خوزستان، بوشهر، هرمزگان \\
\rowcolor{gray!10}
۶ & بلوچ & ۱.۷ & ۲٪ & سیستان و بلوچستان \\
۷ & ترکمن & ۰.۹ & ۱٪ & گلستان، خراسان شمالی \\
\rowcolor{gray!10}
۸ & سایر & ۲.۴ & ۳٪ & گیلک، مازنی، تالش، قشقایی و... \\
\midrule
& \textbf{مجموع} & \textbf{۸۷} & \textbf{۱۰۰٪} & \\
\bottomrule
\end{tabular}
\end{table}

\subsection{توزیع جغرافیایی}

\begin{center}
\begin{tikzpicture}[scale=0.9]
% نقشه ساده‌شده ایران
\draw[thick, fill=gray!10] plot[smooth cycle] coordinates {
    (0,2) (1,4) (3,5) (5,5.5) (7,5) (9,4.5) (10,3) (9.5,1) (8,0) (6,-0.5) (4,0) (2,0.5) (0.5,1)
};

% مناطق قومی (ساده‌شده)
% آذربایجان
\fill[blue!30, opacity=0.7] plot[smooth cycle] coordinates {(0.5,3) (1,4.5) (2.5,4.5) (3,3.5) (2,2.5)};
\node[font=\tiny\bfseries] at (1.7,3.5) {آذری};

% کردستان
\fill[green!40, opacity=0.7] plot[smooth cycle] coordinates {(0.5,1.5) (0.3,2.5) (1.5,3) (2.5,2.5) (2,1.5)};
\node[font=\tiny\bfseries] at (1.3,2.2) {کرد};

% لرستان
\fill[orange!40, opacity=0.7] plot[smooth cycle] coordinates {(2.5,1.5) (2.5,2.5) (3.5,2.5) (4,1.5) (3,1)};
\node[font=\tiny\bfseries] at (3,1.8) {لر};

% عرب - خوزستان
\fill[yellow!50, opacity=0.7] plot[smooth cycle] coordinates {(2,0.5) (2.5,1.5) (3.5,1.5) (3.5,0.5) (2.5,0)};
\node[font=\tiny\bfseries] at (2.8,0.8) {عرب};

% بلوچستان
\fill[purple!30, opacity=0.7] plot[smooth cycle] coordinates {(8,0) (9.5,1) (10,2) (9,2.5) (7.5,1.5) (7,0.5)};
\node[font=\tiny\bfseries] at (8.5,1.2) {بلوچ};

% ترکمن
\fill[red!30, opacity=0.7] plot[smooth cycle] coordinates {(6,4.5) (7,5) (8,4.5) (7.5,4) (6.5,4)};
\node[font=\tiny\bfseries] at (7,4.5) {ترکمن};

% فارس (مرکز)
\node[font=\small\bfseries] at (5.5,2.5) {فارس};

% راهنما
\node[anchor=north west, font=\scriptsize] at (10.5,5) {
    \begin{tabular}{ll}
    \textcolor{blue!60}{$\blacksquare$} & آذری \\
    \textcolor{green!60}{$\blacksquare$} & کرد \\
    \textcolor{orange!60}{$\blacksquare$} & لر \\
    \textcolor{yellow!70}{$\blacksquare$} & عرب \\
    \textcolor{purple!50}{$\blacksquare$} & بلوچ \\
    \textcolor{red!50}{$\blacksquare$} & ترکمن \\
    \end{tabular}
};
\end{tikzpicture}
\captionof{figure}{نقشه ساده‌شده توزیع قومی در ایران}
\label{fig:ethnic-map}
\end{center}

\subsection{تنوع مذهبی}

\begin{table}[htbp]
\centering
\caption{ترکیب مذهبی ایران}
\label{tab:religious-composition}
\begin{tabular}{>{\columncolor{purple!8}}r l r p{5cm}}
\toprule
\rowcolor{purple!25}
\textbf{ردیف} & \textbf{مذهب/دین} & \textbf{درصد تخمینی} & \textbf{توضیحات} \\
\midrule
۱ & شیعه دوازده‌امامی & ۸۵-۹۰٪ & اکثریت، رسمی در نظام فعلی \\
\rowcolor{gray!10}
۲ & سنی & ۸-۱۰٪ & کرد، بلوچ، ترکمن، بخشی از عرب‌ها \\
۳ & مسیحی & ۰.۵٪ & ارمنی، آشوری، کلدانی \\
\rowcolor{gray!10}
۴ & یهودی & کمتر از ۰.۱٪ & بزرگ‌ترین جامعه یهودی خاورمیانه خارج از اسرائیل \\
۵ & زرتشتی & کمتر از ۰.۱٪ & یزد، کرمان، تهران \\
\rowcolor{gray!10}
۶ & بهایی & تخمین متفاوت & غیررسمی، تحت فشار \\
۷ & بی‌دین/لاادری & نامشخص & در حال افزایش \\
\bottomrule
\end{tabular}
\end{table}

%═══════════════════════════════════════════════════════════════════════════════
\section{میراث تاریخی: بی‌اعتمادی حاشیه به مرکز}
\label{sec:historical-legacy}
%═══════════════════════════════════════════════════════════════════════════════

\subsection{سیاست‌های همگون‌سازی}

\begin{enghelabbox}
\textbf{درس تاریخ: همگون‌سازی اجباری همیشه شکست خورده است}

سیاست‌های «یک‌سان‌سازی» در ایران معاصر نتایج معکوس داشته:
\begin{itemize}[nosep]
\item \textbf{دوره رضاشاه}: ممنوعیت لباس محلی، تحمیل زبان فارسی، سرکوب عشایر
    \begin{itemize}[nosep]
    \item نتیجه: شورش‌های قومی، تقویت هویت‌طلبی پنهان
    \end{itemize}
\item \textbf{دوره پهلوی دوم}: تمرکز توسعه در مرکز، بی‌توجهی به مناطق قومی
    \begin{itemize}[nosep]
    \item نتیجه: شکاف توسعه‌ای، نارضایتی مناطق
    \end{itemize}
\item \textbf{جمهوری اسلامی}: تبعیض مذهبی علیه سنی‌ها، محدودیت‌های زبانی
    \begin{itemize}[nosep]
    \item نتیجه: احساس شهروند درجه دو در اقلیت‌ها
    \end{itemize}
\end{itemize}
\textbf{درس}: سرکوب هویت قومی آن را تقویت می‌کند، نه تضعیف.
\end{enghelabbox}

\subsection{شکاف توسعه‌ای}

\begin{table}[htbp]
\centering
\caption{شاخص‌های توسعه به تفکیک استان‌های قومی (۱۴۰۲)}
\label{tab:development-gap}
\begin{tabular}{>{\columncolor{red!8}}l c c c c c}
\toprule
\rowcolor{red!25}
\textbf{شاخص} & \textbf{تهران} & \textbf{آذربایجان شرقی} & \textbf{کردستان} & \textbf{خوزستان} & \textbf{سیستان و بلوچستان} \\
\midrule
HDI استانی & ۰.۸۲ & ۰.۷۵ & ۰.۷۱ & ۰.۷۲ & ۰.۶۵ \\
\rowcolor{gray!10}
نرخ بیکاری & ۸٪ & ۱۱٪ & ۱۵٪ & ۱۴٪ & ۱۸٪ \\
درآمد سرانه (به تهران=۱۰۰) & ۱۰۰ & ۶۵ & ۵۵ & ۶۰ & ۴۰ \\
\rowcolor{gray!10}
دسترسی به اینترنت پرسرعت & ۸۵٪ & ۶۵٪ & ۵۵٪ & ۶۰٪ & ۳۵٪ \\
پزشک به ازای هر ۱۰۰۰ نفر & ۳.۵ & ۱.۸ & ۱.۲ & ۱.۵ & ۰.۸ \\
\rowcolor{gray!10}
نرخ باسوادی & ۹۷٪ & ۹۳٪ & ۹۰٪ & ۹۱٪ & ۸۲٪ \\
\bottomrule
\end{tabular}
\end{table}

\begin{center}
\begin{tikzpicture}
\begin{axis}[
    width=13cm,
    height=6cm,
    ybar,
    bar width=12pt,
    ylabel={شاخص توسعه انسانی (HDI)},
    xlabel={استان},
    ymin=0.5,
    ymax=0.9,
    xtick=data,
    xticklabels={تهران, اصفهان, آذربایجان شرقی, کردستان, خوزستان, سیستان و بلوچستان},
    x tick label style={rotate=45, anchor=east, font=\small},
    nodes near coords,
    every node near coord/.append style={font=\tiny},
    legend pos=north east,
    grid=major,
    grid style={dashed, gray!30}
]
\addplot[fill=blue!60, draw=blue!80] coordinates {
    (1, 0.82) (2, 0.79) (3, 0.75) (4, 0.71) (5, 0.72) (6, 0.65)
};
% خط میانگین کشوری
\addplot[red, thick, dashed, domain=0.5:6.5] {0.774};
\node[red, font=\tiny] at (axis cs:5.5,0.79) {میانگین کشور: ۰.۷۷۴};
\end{axis}
\end{tikzpicture}
\captionof{figure}{شکاف توسعه‌ای بین استان‌های مرکزی و حاشیه‌ای}
\label{fig:development-gap}
\end{center}

\subsection{نارضایتی‌های انباشته}

\begin{table}[htbp]
\centering
\caption{مطالبات اصلی اقوام مختلف ایران}
\label{tab:ethnic-demands}
\begin{tabular}{>{\columncolor{orange!8}}l p{4cm} p{4cm} p{3cm}}
\toprule
\rowcolor{orange!25}
\textbf{قوم} & \textbf{مطالبات فرهنگی-زبانی} & \textbf{مطالبات سیاسی-اقتصادی} & \textbf{سطح تنش} \\
\midrule
آذری & آموزش به زبان ترکی، رسانه ترکی & سهم عادلانه از قدرت، توسعه & متوسط \\
\rowcolor{gray!10}
کرد & زبان کردی رسمی، خودمختاری & پایان تبعیض، توسعه منطقه & بالا \\
لر & حفظ فرهنگ و زبان & توسعه اقتصادی & پایین \\
\rowcolor{gray!10}
عرب & زبان عربی در مدارس & سهم از درآمد نفت، محیط‌زیست & متوسط-بالا \\
بلوچ & حقوق مذهبی (سنی)، زبان & امنیت، توسعه، پایان تبعیض & بالا \\
\rowcolor{gray!10}
ترکمن & حفظ فرهنگ، زبان & توسعه اقتصادی & پایین \\
\bottomrule
\end{tabular}
\end{table}

%═══════════════════════════════════════════════════════════════════════════════
\section{چارچوب نظری: مدل‌های مدیریت تنوع}
\label{sec:theoretical-framework}
%═══════════════════════════════════════════════════════════════════════════════

\subsection{طیف گزینه‌ها}

\begin{center}
\begin{tikzpicture}
% محور
\draw[thick, ->] (0,0) -- (14,0);
\node[below] at (7,-0.5) {طیف گزینه‌های مدیریت تنوع};

% نقاط
\foreach \x/\label/\desc in {
    1/دولت واحد متمرکز/{\scriptsize فرانسه},
    4/منطقه‌ای‌سازی/{\scriptsize ایتالیا},
    7/فدرالیسم/{\scriptsize آلمان، هند},
    10/کنفدراسیون/{\scriptsize سوئیس},
    13/تجزیه/{\scriptsize چکسلواکی}
} {
    \fill[blue!70] (\x,0) circle (4pt);
    \node[above, font=\small\bfseries, align=center] at (\x,0.3) {\label};
    \node[below, font=\scriptsize, align=center] at (\x,-0.2) {\desc};
}

% ناحیه پیشنهادی
\fill[green!30, opacity=0.5] (5.5,-0.8) rectangle (8.5,1.2);
\node[font=\small\bfseries, green!50!black] at (7,0.9) {ناحیه پیشنهادی};

% فلش‌های خطر
\draw[red, thick, ->] (1,0.5) -- (1,1.2) node[above, font=\tiny, align=center] {سرکوب\\ناپایدار};
\draw[red, thick, ->] (13,0.5) -- (13,1.2) node[above, font=\tiny, align=center] {تجزیه\\فاجعه‌بار};
\end{tikzpicture}
\end{center}

\subsection{مدل‌های موفق جهانی}

\begin{olgoobox}
\textbf{الگوی موفق: سوئیس — آزمایشگاه همزیستی}

سوئیس با ۴ زبان رسمی (آلمانی ۶۳٪، فرانسوی ۲۳٪، ایتالیایی ۸٪، رومانش ۱٪) نمونه بارز مدیریت موفق تنوع است:
\begin{itemize}[nosep]
\item \textbf{ساختار}: ۲۶ کانتون با خودمختاری گسترده
\item \textbf{زبان}: هر کانتون زبان رسمی خود را تعیین می‌کند؛ سطح فدرال چهارزبانه
\item \textbf{سیاست}: دموکراسی مستقیم، شورای فدرال هفت‌نفره با چرخش ریاست
\item \textbf{نتیجه}: یکی از باثبات‌ترین و مرفه‌ترین کشورهای جهان
\item \textbf{درس برای ایران}: تمرکززدایی واقعی + هویت مشترک فراقومی = موفقیت
\end{itemize}
\end{olgoobox}

\begin{table}[htbp]
\centering
\caption{مقایسه مدل‌های مدیریت تنوع در کشورهای موفق}
\label{tab:diversity-models}
\begin{tabular}{>{\columncolor{green!8}}l p{2.5cm} p{2.5cm} p{2.5cm} p{3cm}}
\toprule
\rowcolor{green!25}
\textbf{کشور} & \textbf{مدل} & \textbf{ویژگی کلیدی} & \textbf{موفقیت} & \textbf{درس برای ایران} \\
\midrule
سوئیس & فدرال-کنفدرال & خودمختاری بالا & خیلی بالا & تمرکززدایی واقعی \\
\rowcolor{gray!10}
کانادا & فدرال نامتقارن & حقوق خاص کبک & بالا & انعطاف در برابر تفاوت‌ها \\
هند & فدرال زبانی & ایالت‌ها بر اساس زبان & متوسط-بالا & مدیریت پیچیدگی \\
\rowcolor{gray!10}
اسپانیا & منطقه‌ای خودمختار & خودمختاری‌های نامتقارن & متوسط & توازن مرکز-پیرامون \\
بلژیک & فدرال جماعتی & جماعت‌های زبانی & متوسط & تقسیم قدرت \\
\bottomrule
\end{tabular}
\end{table}

\subsection{مدل‌های ناموفق: درس‌های منفی}

\begin{enghelabbox}
\textbf{هشدار: از یوگسلاوی و عراق بیاموزیم}

\textbf{یوگسلاوی}:
\begin{itemize}[nosep]
\item تنوع مدیریت‌نشده + ناسیونالیسم افراطی = جنگ داخلی خونین
\item ۱۴۰,۰۰۰ کشته، ۴ میلیون آواره
\item تجزیه به ۷ کشور با کینه‌های پایدار
\end{itemize}

\textbf{عراق}:
\begin{itemize}[nosep]
\item سرکوب طولانی کردها و شیعیان توسط رژیم بعث
\item پس از ۲۰۰۳: بی‌ثباتی مزمن، تنش‌های قومی-مذهبی
\item کردستان عراق: خودمختاری دوفاکتو، تمایل به استقلال
\end{itemize}

\textbf{درس}: سرکوب مسئله را حل نمی‌کند، فقط به تعویق می‌اندازد و تشدید می‌کند.
\end{enghelabbox}

%═══════════════════════════════════════════════════════════════════════════════
\section{مدل پیشنهادی: فدرالیسم همبسته}
\label{sec:cohesive-federalism}
%═══════════════════════════════════════════════════════════════════════════════

\subsection{اصول بنیادین}

\begin{center}
\begin{tikzpicture}[
    node distance=1.5cm,
    principle/.style={
        rectangle,
        rounded corners=8pt,
        draw=blue!70,
        fill=blue!15,
        thick,
        minimum width=5cm,
        minimum height=1.3cm,
        align=center,
        font=\small
    }
]
% عنوان
\node[rectangle, rounded corners=10pt, draw=green!70!black, fill=green!15,
      thick, minimum width=12cm, minimum height=1.2cm, font=\large\bfseries]
      (title) {هفت اصل فدرالیسم همبسته};

% اصول در دو ستون
\node[principle, below left=1.5cm and 0.5cm of title] (p1) 
    {\textbf{۱. یکپارچگی ارضی}\\ مرزهای ایران خط قرمز};
\node[principle, below=0.3cm of p1] (p2) 
    {\textbf{۲. برابری شهروندی}\\ همه ایرانیان برابر};
\node[principle, below=0.3cm of p2] (p3) 
    {\textbf{۳. حقوق فرهنگی-زبانی}\\ تضمین شده در قانون اساسی};
\node[principle, below=0.3cm of p3] (p4) 
    {\textbf{۴. خودمختاری متناسب}\\ بر اساس نیاز، نه یکسان};

\node[principle, below right=1.5cm and 0.5cm of title] (p5) 
    {\textbf{۵. همبستگی ملی}\\ مکانیزم‌های پیونددهنده};
\node[principle, below=0.3cm of p5] (p6) 
    {\textbf{۶. عدالت توزیعی}\\ جبران نابرابری‌های تاریخی};
\node[principle, below=0.3cm of p6] (p7) 
    {\textbf{۷. مشارکت در قدرت}\\ همه در تصمیم‌گیری ملی};

% هویت مشترک در مرکز
\node[ellipse, draw=red!70, fill=red!15, thick, 
      minimum width=3.5cm, minimum height=2cm, align=center,
      below=4.5cm of title] (identity)
    {\textbf{هویت مشترک}\\ ایرانی بودن};

% اتصالات
\draw[thick, gray, dashed] (p4.south) -- ++(0,-0.5) -| (identity);
\draw[thick, gray, dashed] (p7.south) -- ++(0,-0.5) -| (identity);
\end{tikzpicture}
\end{center}

\subsection{ساختار پیشنهادی: پنج منطقه خودمختار}

\begin{table}[htbp]
\centering
\caption{ساختار مناطق خودمختار پیشنهادی}
\label{tab:autonomous-regions-detail}
\begin{tabular}{>{\columncolor{cyan!8}}r p{2.5cm} p{2.5cm} r p{3.5cm}}
\toprule
\rowcolor{cyan!25}
\textbf{منطقه} & \textbf{مرکز} & \textbf{استان‌ها} & \textbf{جمعیت (م)} & \textbf{زبان‌های رسمی منطقه‌ای} \\
\midrule
آذربایجان & تبریز & شرقی، غربی، اردبیل، زنجان & ۱۲ & فارسی + ترکی آذری \\
\rowcolor{gray!10}
کردستان بزرگ & سنندج & کردستان، کرمانشاه، ایلام & ۶ & فارسی + کردی \\
عربستان ایران & اهواز & خوزستان & ۵ & فارسی + عربی \\
\rowcolor{gray!10}
بلوچستان & زاهدان & سیستان و بلوچستان & ۳ & فارسی + بلوچی \\
ترکمن‌صحرا & گنبدکاووس & بخشی از گلستان & ۱ & فارسی + ترکمنی \\
\midrule
\multicolumn{4}{l}{\textbf{سایر استان‌ها}: در چارچوب استانی معمول} & \\
\bottomrule
\end{tabular}
\end{table}

\begin{naghlbox}
«خودمختاری به معنای جدایی نیست. برعکس، خودمختاری درست‌طراحی‌شده، انگیزه جدایی را کاهش می‌دهد. وقتی مردم احساس کنند صدایشان شنیده می‌شود و حقوقشان محترم است، دلیلی برای جدایی‌طلبی ندارند.»
\sourceline{ویل کیملیکا، «شهروندی چندفرهنگی»، ۱۹۹۵}
\end{naghlbox}

\subsection{صلاحیت‌های هر سطح}

\begin{table}[htbp]
\centering
\caption{تقسیم صلاحیت‌ها بین سطوح حکومتی}
\label{tab:competency-division}
\begin{tabular}{>{\columncolor{blue!8}}p{3cm} p{4cm} p{4cm}}
\toprule
\rowcolor{blue!25}
\textbf{حوزه} & \textbf{صلاحیت فدرال (ملی)} & \textbf{صلاحیت منطقه‌ای/استانی} \\
\midrule
دفاع و امنیت & ارتش، مرزبانی، ضدتروریسم & پلیس محلی (با هماهنگی) \\
\rowcolor{gray!10}
سیاست خارجی & انحصاری & — \\
اقتصاد کلان & پول، بانک مرکزی، تجارت خارجی & توسعه منطقه‌ای، مالیات محلی \\
\rowcolor{gray!10}
آموزش & استانداردها، دانشگاه‌های ملی & مدارس، زبان آموزش، برنامه محلی \\
فرهنگ و رسانه & صداوسیمای ملی، میراث ملی & رسانه محلی، میراث منطقه‌ای \\
\rowcolor{gray!10}
بهداشت & بیمه ملی، استانداردها & بیمارستان‌های استانی \\
زیرساخت & بزرگراه‌های ملی، راه‌آهن & جاده‌های محلی، حمل‌ونقل شهری \\
\rowcolor{gray!10}
محیط‌زیست & قوانین کلان، منابع آب بین‌استانی & اجرا، منابع محلی \\
\bottomrule
\end{tabular}
\end{table}

\subsection{مجلس اقوام و مناطق}

\begin{table}[htbp]
\centering
\caption{ترکیب مجلس اقوام و مناطق (۱۰۰ کرسی)}
\label{tab:senate-composition-detail}
\begin{tabular}{>{\columncolor{purple!8}}r l c p{5cm}}
\toprule
\rowcolor{purple!25}
\textbf{ردیف} & \textbf{دسته‌بندی} & \textbf{تعداد کرسی} & \textbf{نحوه انتخاب} \\
\midrule
۱ & نمایندگان استان‌ها (۳۱×۲) & ۶۲ & انتخاب مستقیم، ۲ نفر از هر استان \\
\rowcolor{gray!10}
۲ & نمایندگان اقوام & ۲۰ & انتخاب توسط شوراهای قومی \\
۳ & نمایندگان اقلیت‌های دینی & ۸ & ارمنی(۲)، آشوری(۱)، یهودی(۱)، زرتشتی(۱)، سنی(۳) \\
\rowcolor{gray!10}
۴ & نخبگان و شخصیت‌های ملی & ۱۰ & انتصاب با تأیید مجلس ملی \\
\midrule
& \textbf{مجموع} & \textbf{۱۰۰} & \\
\bottomrule
\end{tabular}
\end{table}

\textbf{صلاحیت‌های ویژه مجلس اقوام}:
\begin{itemize}[nosep]
\item حق وتو در قوانین مربوط به حقوق اقوام و مناطق
\item تصویب تغییرات مرزی استان‌ها
\item نظارت بر توزیع عادلانه بودجه ملی
\item تأیید انتصاب مقامات مناطق خودمختار
\end{itemize}

%═══════════════════════════════════════════════════════════════════════════════
\section{حقوق زبانی و فرهنگی}
\label{sec:language-rights}
%═══════════════════════════════════════════════════════════════════════════════

\subsection{سیاست زبانی}

\begin{center}
\begin{tikzpicture}[
    level/.style={
        rectangle,
        rounded corners=8pt,
        draw=#1!70,
        fill=#1!15,
        thick,
        minimum width=10cm,
        minimum height=1.5cm,
        align=center
    }
]
% سطوح زبانی
\node[level=blue] (l1) at (0,4) {\textbf{سطح ۱: زبان ملی مشترک}\\ فارسی — زبان ارتباط ملی و بین‌المللی};

\node[level=green] (l2) at (0,2) {\textbf{سطح ۲: زبان‌های رسمی منطقه‌ای}\\ ترکی آذری، کردی، عربی، بلوچی، ترکمنی — رسمی در مناطق};

\node[level=orange] (l3) at (0,0) {\textbf{سطح ۳: زبان‌های محلی حمایت‌شده}\\ گیلکی، مازنی، لری، تالشی و... — حفاظت و ترویج};

% فلش‌ها
\draw[thick, ->] (l1.south) -- (l2.north);
\draw[thick, ->] (l2.south) -- (l3.north);

% توضیح
\node[right=0.5cm of l1, font=\scriptsize, align=left] {همه باید بیاموزند\\مدارک رسمی ملی};
\node[right=0.5cm of l2, font=\scriptsize, align=left] {آموزش، دادگاه، اداره\\در منطقه مربوطه};
\node[right=0.5cm of l3, font=\scriptsize, align=left] {حمایت فرهنگی\\بدون الزام رسمی};
\end{tikzpicture}
\captionof{figure}{هرم سیاست زبانی پیشنهادی}
\label{fig:language-pyramid}
\end{center}

\subsection{برنامه آموزش چندزبانه}

\begin{table}[htbp]
\centering
\caption{برنامه آموزش زبان در مدارس مناطق قومی}
\label{tab:multilingual-education-detail}
\begin{tabular}{>{\columncolor{green!8}}l c c c c}
\toprule
\rowcolor{green!25}
\textbf{مقطع} & \textbf{زبان مادری} & \textbf{فارسی} & \textbf{انگلیسی} & \textbf{زبان انتخابی} \\
\midrule
پیش‌دبستان & ۱۰۰٪ & — & — & — \\
\rowcolor{gray!10}
ابتدایی (۱-۳) & ۶۰٪ & ۴۰٪ & — & — \\
ابتدایی (۴-۶) & ۴۰٪ & ۵۰٪ & ۱۰٪ & — \\
\rowcolor{gray!10}
متوسطه اول & ۳۰٪ & ۵۰٪ & ۲۰٪ & — \\
متوسطه دوم & ۲۰٪ & ۵۰٪ & ۲۵٪ & ۵٪ \\
\rowcolor{gray!10}
دانشگاه & — & اصلی & انتخابی & زبان مادری اختیاری \\
\bottomrule
\end{tabular}
\end{table}

\textbf{نکات کلیدی}:
\begin{itemize}[nosep]
\item کودکان در پایه‌های اولیه به زبان مادری آموزش می‌بینند
\item فارسی به‌تدریج معرفی شده و غالب می‌شود
\item هدف: شهروندانی دوزبانه یا سه‌زبانه
\item انتخاب زبان آموزش با خانواده و شورای منطقه‌ای
\end{itemize}

\subsection{حقوق فرهنگی}

\begin{table}[htbp]
\centering
\caption{منشور حقوق فرهنگی اقوام}
\label{tab:cultural-rights}
\begin{tabular}{>{\columncolor{teal!8}}r p{5cm} p{5cm}}
\toprule
\rowcolor{teal!25}
\textbf{ردیف} & \textbf{حق} & \textbf{تضمین اجرایی} \\
\midrule
۱ & حق آموزش به زبان مادری & مدارس دوزبانه، معلمان بومی \\
\rowcolor{gray!10}
۲ & حق رسانه به زبان محلی & تلویزیون و رادیوی منطقه‌ای، مطبوعات \\
۳ & حق استفاده از زبان در دادگاه & مترجم رسمی، قاضی بومی‌زبان \\
\rowcolor{gray!10}
۴ & حق نام‌گذاری به زبان محلی & نام فرزند، کسب‌وکار، خیابان \\
۵ & حق حفظ آداب و سنن & تعطیلات محلی، جشن‌های قومی \\
\rowcolor{gray!10}
۶ & حق پوشش محلی & آزادی لباس سنتی \\
۷ & حق مذهب & مساجد سنی، کلیسا، کنیسه، آتشکده \\
\rowcolor{gray!10}
۸ & حق مشارکت در تصمیم‌گیری & شوراهای قومی، سهمیه در نهادها \\
\bottomrule
\end{tabular}
\end{table}

%═══════════════════════════════════════════════════════════════════════════════
\section{عدالت اقتصادی و جبران نابرابری‌ها}
\label{sec:economic-justice}
%═══════════════════════════════════════════════════════════════════════════════

\subsection{برنامه توسعه متوازن}

\begin{naghlbox}
«عدالت اقتصادی پایه صلح قومی است. وقتی مردم یک منطقه احساس کنند منابع سرزمینشان استخراج می‌شود اما سهمی از آن نمی‌برند، بذر نارضایتی کاشته شده است.»
\sourceline{نویسنده}
\end{naghlbox}

\begin{table}[htbp]
\centering
\caption{برنامه جبران شکاف توسعه‌ای (افق ۱۵ ساله)}
\label{tab:development-compensation}
\begin{tabular}{>{\columncolor{orange!8}}l c c c p{3.5cm}}
\toprule
\rowcolor{orange!25}
\textbf{منطقه} & \textbf{HDI فعلی} & \textbf{هدف سال ۱۰} & \textbf{هدف سال ۱۵} & \textbf{سرمایه‌گذاری ویژه} \\
\midrule
سیستان و بلوچستان & ۰.۶۵ & ۰.۷۵ & ۰.۸۰ & ۵۰ میلیارد دلار \\
\rowcolor{gray!10}
کردستان & ۰.۷۱ & ۰.۷۸ & ۰.۸۲ & ۳۰ میلیارد دلار \\
خوزستان & ۰.۷۲ & ۰.۷۹ & ۰.۸۳ & ۴۰ میلیارد دلار \\
\rowcolor{gray!10}
آذربایجان غربی & ۰.۷۳ & ۰.۷۹ & ۰.۸۳ & ۲۵ میلیارد دلار \\
لرستان & ۰.۷۲ & ۰.۷۸ & ۰.۸۲ & ۲۰ میلیارد دلار \\
\midrule
\multicolumn{4}{r}{\textbf{مجموع سرمایه‌گذاری ویژه توسعه مناطق محروم}} & \textbf{۱۶۵ میلیارد دلار} \\
\bottomrule
\end{tabular}
\end{table}

\subsection{تقسیم عادلانه درآمدهای ملی}

\begin{center}
\begin{tikzpicture}
\begin{axis}[
    width=12cm,
    height=7cm,
    ybar stacked,
    bar width=25pt,
    ylabel={درصد درآمد نفت و گاز},
    xlabel={مدل توزیع},
    ymin=0,
    ymax=100,
    xtick={1,2,3},
    xticklabels={وضع موجود, مدل پیشنهادی, مدل بلندمدت},
    legend style={
        at={(0.5,-0.2)},
        anchor=north,
        legend columns=3,
        font=\small
    },
    nodes near coords,
    every node near coord/.append style={font=\tiny},
    grid=major,
    grid style={dashed, gray!30}
]
% دولت مرکزی
\addplot[fill=blue!60, draw=blue!80] coordinates {(1,85) (2,50) (3,40)};
% استان تولیدکننده
\addplot[fill=green!60, draw=green!80] coordinates {(1,5) (2,25) (3,30)};
% صندوق توسعه مناطق محروم
\addplot[fill=orange!60, draw=orange!80] coordinates {(1,5) (2,15) (3,15)};
% صندوق نسل‌های آینده
\addplot[fill=purple!60, draw=purple!80] coordinates {(1,5) (2,10) (3,15)};

\legend{دولت مرکزی, استان تولیدکننده, صندوق مناطق محروم, صندوق نسل‌های آینده}
\end{axis}
\end{tikzpicture}
\captionof{figure}{مدل پیشنهادی توزیع درآمدهای نفت و گاز}
\label{fig:revenue-distribution}
\end{center}

\subsection{اصل سهم منطقه از منابع}

\begin{table}[htbp]
\centering
\caption{سهم استان‌ها از درآمد منابع طبیعی}
\label{tab:resource-sharing}
\begin{tabular}{>{\columncolor{green!8}}l p{6cm} c}
\toprule
\rowcolor{green!25}
\textbf{نوع منبع} & \textbf{قاعده توزیع} & \textbf{سهم استان مبدأ} \\
\midrule
نفت و گاز & ۲۵٪ به استان، ۱۵٪ صندوق توسعه، ۱۰٪ نسل آینده & ۲۵٪ \\
\rowcolor{gray!10}
معادن & ۳۰٪ به استان، ۱۰٪ صندوق محیط‌زیست & ۳۰٪ \\
آب & مدیریت حوضه‌ای، حق‌آبه تضمین‌شده برای مبدأ & متغیر \\
\rowcolor{gray!10}
گردشگری & ۵۰٪ درآمد ورودی به استان & ۵۰٪ \\
\bottomrule
\end{tabular}
\end{table}

\begin{olgoobox}
\textbf{الگوی موفق: نروژ و صندوق نفتی}

نروژ با مدیریت هوشمندانه درآمد نفت، الگویی برای همه کشورهای نفتی شده است:
\begin{itemize}[nosep]
\item \textbf{صندوق بازنشستگی دولتی}: بیش از ۱.۴ تریلیون دلار دارایی
\item \textbf{قاعده مالی}: فقط ۳٪ صندوق سالانه قابل برداشت
\item \textbf{شفافیت}: همه سرمایه‌گذاری‌ها عمومی
\item \textbf{نتیجه}: ثروت پایدار برای نسل‌های آینده
\item \textbf{درس برای ایران}: درآمد نفت متعلق به همه نسل‌ها و همه مناطق است
\end{itemize}
\end{olgoobox}

\subsection{پروژه‌های کلان توسعه مناطق قومی}

\begin{table}[htbp]
\centering
\caption{پروژه‌های کلان توسعه در مناطق قومی}
\label{tab:regional-mega-projects}
\begin{tabular}{>{\columncolor{cyan!8}}l p{3.5cm} c p{4cm}}
\toprule
\rowcolor{cyan!25}
\textbf{منطقه} & \textbf{پروژه کلیدی} & \textbf{سرمایه (میلیارد \$)} & \textbf{دستاورد مورد انتظار} \\
\midrule
آذربایجان & هاب لجستیک قفقاز & ۱۵ & ۵۰۰,۰۰۰ شغل، ترانزیت \\
\rowcolor{gray!10}
کردستان & شهرک صنعتی-گردشگری & ۱۰ & ۲۰۰,۰۰۰ شغل، گردشگری \\
خوزستان & احیای کارون و تالاب‌ها & ۲۰ & محیط‌زیست، کشاورزی \\
\rowcolor{gray!10}
بلوچستان & بندر چابهار و منطقه آزاد & ۲۵ & هاب تجارت هند-آسیای مرکزی \\
ترکمن‌صحرا & کریدور انرژی خزر & ۸ & صادرات برق، گاز \\
\rowcolor{gray!10}
لرستان & قطب گردشگری طبیعت & ۵ & ۱۰۰,۰۰۰ شغل \\
عرب‌خوزستان & پتروشیمی سبز & ۱۵ & اشتغال محلی، ارزش‌افزوده \\
\bottomrule
\end{tabular}
\end{table}

%═══════════════════════════════════════════════════════════════════════════════
\section{مشارکت در قدرت ملی}
\label{sec:power-sharing}
%═══════════════════════════════════════════════════════════════════════════════

\subsection{اصل حضور متناسب}

\begin{naghlbox}
«دموکراسی در جوامع چندپاره نمی‌تواند صرفاً حکومت اکثریت باشد. باید مکانیزم‌هایی وجود داشته باشد که همه گروه‌های مهم در تصمیم‌گیری‌های کلیدی مشارکت داشته باشند.»
\sourceline{آرنت لیپهارت، «الگوهای دموکراسی»، ۱۹۹۹}
\end{naghlbox}

\begin{table}[htbp]
\centering
\caption{اصل تناسب قومی در نهادهای ملی}
\label{tab:proportional-representation}
\begin{tabular}{>{\columncolor{purple!8}}l c p{5cm}}
\toprule
\rowcolor{purple!25}
\textbf{نهاد} & \textbf{هدف تناسب} & \textbf{مکانیزم تضمین} \\
\midrule
کابینه دولت & حداقل ۳۰٪ غیرفارس & الزام قانونی، نظارت مجلس اقوام \\
\rowcolor{gray!10}
قوه قضائیه & ۲۵٪ از اقوام غیرفارس & سهمیه در آزمون قضاوت، آموزش ویژه \\
ارتش (فرماندهان) & ۲۰٪ در سطوح بالا & برنامه ارتقای متنوع \\
\rowcolor{gray!10}
سفرا & ۲۵٪ از اقوام & انتصاب متوازن \\
مدیران ارشد دولتی & ۲۵٪ از مناطق قومی & امتیاز در استخدام \\
\rowcolor{gray!10}
دانشگاه‌های ملی & سهمیه منطقه‌ای & کنکور استانی، بورسیه \\
\bottomrule
\end{tabular}
\end{table}

\subsection{شورای عالی اقوام}

\begin{table}[htbp]
\centering
\caption{ساختار و وظایف شورای عالی اقوام ایران}
\label{tab:ethnic-council}
\begin{tabular}{>{\columncolor{blue!8}}r p{10cm}}
\toprule
\rowcolor{blue!25}
\textbf{مشخصه} & \textbf{توضیح} \\
\midrule
ترکیب & ۲۱ عضو: ۳ نفر از هر یک از ۷ گروه قومی اصلی \\
\rowcolor{gray!10}
انتخاب & انتخاب توسط شوراهای قومی منطقه‌ای \\
دوره & ۶ سال با تمدید یک‌بار \\
\rowcolor{gray!10}
ریاست & چرخشی سالانه بین اقوام \\
\midrule
\rowcolor{blue!15}
\multicolumn{2}{c}{\textbf{وظایف و اختیارات}} \\
\midrule
مشورتی & نظر مشورتی در همه قوانین مرتبط با اقوام \\
\rowcolor{gray!10}
نظارتی & گزارش سالانه وضعیت حقوق اقوام \\
پیشنهادی & ارائه لوایح به مجلس از طریق نمایندگان \\
\rowcolor{gray!10}
حل اختلاف & میانجی‌گری در تنش‌های بین‌قومی \\
وتوی تعلیقی & توقف ۶ ماهه قوانین مغایر با حقوق اقوام برای بازنگری \\
\bottomrule
\end{tabular}
\end{table}

\subsection{نمایندگی در نهادهای کلیدی}

\begin{center}
\begin{tikzpicture}
\begin{axis}[
    width=13cm,
    height=7cm,
    ybar,
    bar width=10pt,
    ylabel={درصد},
    xlabel={قوم},
    ymin=0,
    ymax=60,
    xtick=data,
    xticklabels={فارس, آذری, کرد, لر, عرب, بلوچ, ترکمن},
    legend style={
        at={(0.5,-0.2)},
        anchor=north,
        legend columns=2,
        font=\small
    },
    legend cell align={right},
    nodes near coords,
    every node near coord/.append style={font=\tiny},
    grid=major,
    grid style={dashed, gray!30}
]
% جمعیت
\addplot[fill=blue!60, draw=blue!80] coordinates {
    (1,55) (2,20) (3,10) (4,6) (5,3) (6,2) (7,1)
};
% نمایندگی هدف در مجلس
\addplot[fill=green!60, draw=green!80] coordinates {
    (1,50) (2,22) (3,12) (4,7) (5,4) (6,3) (7,2)
};
\legend{سهم جمعیتی, هدف نمایندگی در مجلس}
\end{axis}
\end{tikzpicture}
\captionof{figure}{مقایسه سهم جمعیتی و هدف نمایندگی اقوام}
\label{fig:representation-target}
\end{center}

%═══════════════════════════════════════════════════════════════════════════════
\section{هویت ملی مشترک: چتر فراگیر}
\label{sec:national-identity}
%═══════════════════════════════════════════════════════════════════════════════

\subsection{بازتعریف «ایرانی بودن»}

\begin{center}
\begin{tikzpicture}[
    node distance=1.5cm,
    identity/.style={
        ellipse,
        draw=#1!70,
        fill=#1!20,
        thick,
        minimum width=3cm,
        minimum height=1.5cm,
        align=center,
        font=\small
    }
]
% چتر ملی
\node[rectangle, rounded corners=15pt, draw=blue!70, fill=blue!10,
      thick, minimum width=13cm, minimum height=8cm] (umbrella) {};
\node[above=0.1cm of umbrella.north, font=\Large\bfseries, blue!70] 
    {هویت ایرانی: چتر فراگیر};

% هویت مشترک در مرکز
\node[identity=red, minimum width=4cm, minimum height=2cm] at (0,1) (core) 
    {\textbf{هسته مشترک}\\ تاریخ، سرزمین، آینده};

% هویت‌های قومی اطراف
\node[identity=blue, above left=1cm of core] (azeri) {آذری-ایرانی};
\node[identity=green, above right=1cm of core] (kurd) {کرد-ایرانی};
\node[identity=orange, left=1.5cm of core] (lor) {لر-ایرانی};
\node[identity=purple, right=1.5cm of core] (arab) {عرب-ایرانی};
\node[identity=teal, below left=1cm of core] (baluch) {بلوچ-ایرانی};
\node[identity=darkyellow, below right=1cm of core] (turkmen) {ترکمن-ایرانی};
\node[identity=gray, below=1.5cm of core] (fars) {فارس-ایرانی};

% خطوط اتصال
\foreach \x in {azeri, kurd, lor, arab, baluch, turkmen, fars} {
    \draw[thick, dashed, gray!50] (core) -- (\x);
}

% توضیح
\node[font=\scriptsize, align=center] at (0,-3.5) 
    {هر ایرانی هم به هویت قومی خود تعلق دارد و هم به هویت ملی مشترک\\ 
     این دو مکمل‌اند، نه رقیب};
\end{tikzpicture}
\end{center}

\subsection{عناصر هویت مشترک}

\begin{table}[htbp]
\centering
\caption{عناصر هویت ملی مشترک ایرانی}
\label{tab:shared-identity}
\begin{tabular}{>{\columncolor{red!8}}r p{3cm} p{7cm}}
\toprule
\rowcolor{red!25}
\textbf{عنصر} & \textbf{توصیف} & \textbf{نمونه‌های عینی} \\
\midrule
تاریخ مشترک & میراث تمدنی مشترک & هخامنشیان، ساسانیان، صفویان — همه اقوام سهیم بودند \\
\rowcolor{gray!10}
سرزمین مشترک & فلات ایران & جغرافیای طبیعی که اقوام را به هم پیوند می‌دهد \\
زبان پیونددهنده & فارسی به‌عنوان زبان مشترک & نه جایگزین زبان‌های محلی، بلکه پل ارتباط \\
\rowcolor{gray!10}
فرهنگ مشترک & نوروز، شب یلدا، ادبیات & جشن‌ها و سنت‌های مشترک فراقومی \\
سرنوشت مشترک & آینده‌ای که با هم می‌سازیم & پروژه‌های ملی، چالش‌های مشترک (آب، توسعه) \\
\rowcolor{gray!10}
ارزش‌های مشترک & آزادی، عدالت، دموکراسی & قانون اساسی به‌عنوان میثاق مشترک \\
\bottomrule
\end{tabular}
\end{table}

\subsection{برنامه‌های تقویت همبستگی}

\begin{table}[htbp]
\centering
\caption{برنامه‌های عملی تقویت همبستگی ملی}
\label{tab:solidarity-programs}
\begin{tabular}{>{\columncolor{green!8}}r p{3.5cm} p{6cm}}
\toprule
\rowcolor{green!25}
\textbf{برنامه} & \textbf{گروه هدف} & \textbf{محتوا و هدف} \\
\midrule
سربازی مشترک & جوانان ۱۸-۲۰ & خدمت در استان‌های دیگر، آشنایی با تنوع \\
\rowcolor{gray!10}
اردوهای ملی & دانش‌آموزان & بازدید متقابل مدارس مناطق مختلف \\
جشنواره اقوام & عموم مردم & جشنواره سالانه فرهنگ اقوام در تهران و شهرها \\
\rowcolor{gray!10}
موزه تنوع ایران & گردشگران و شهروندان & موزه ملی تنوع قومی-فرهنگی \\
پروژه‌های ملی مشترک & همه & راه‌آهن سراسری، شبکه آب، انرژی \\
\rowcolor{gray!10}
تیم‌های ورزشی ملی & ورزشکاران & تیم‌های متنوع قومی با پرچم واحد \\
رسانه ملی چندزبانه & مخاطبان رسانه & شبکه‌های تلویزیونی به زبان‌های مختلف \\
\bottomrule
\end{tabular}
\end{table}

\begin{naghlbox}
«ملت‌ها ساخته می‌شوند، نه کشف. هویت ملی محصول انتخاب آگاهانه برای زندگی مشترک است. ایران می‌تواند نشان دهد که تنوع و وحدت با هم ممکن‌اند — اگر همه احساس کنند در این خانه مشترک جایی دارند.»
\sourceline{ارنست رنان، «ملت چیست؟»، ۱۸۸۲ — بازخوانی برای ایران}
\end{naghlbox}

%═══════════════════════════════════════════════════════════════════════════════
\section{مدیریت تنش‌ها و پیشگیری از بحران}
\label{sec:conflict-prevention}
%═══════════════════════════════════════════════════════════════════════════════

\subsection{سیستم هشدار زودهنگام}

\begin{center}
\begin{tikzpicture}[
    node distance=2cm,
    level/.style={
        rectangle,
        rounded corners=5pt,
        draw=#1!70,
        fill=#1!20,
        thick,
        minimum width=11cm,
        minimum height=1.2cm,
        align=center,
        font=\small
    }
]
% سطوح هشدار
\node[level=green] (l1) at (0,4) {\textbf{سطح سبز}: وضعیت عادی — پایش روتین، گفتگوی مستمر};
\node[level=yellow] (l2) at (0,2.5) {\textbf{سطح زرد}: نارضایتی محسوس — تقویت گفتگو، رسیدگی فوری به شکایات};
\node[level=orange] (l3) at (0,1) {\textbf{سطح نارنجی}: تنش فعال — میانجی‌گری، اقدام اصلاحی، حضور مسئولان ارشد};
\node[level=red] (l4) at (0,-0.5) {\textbf{سطح قرمز}: بحران — مدیریت بحران، مذاکره مستقیم، تعلیق اقدامات تحریک‌کننده};

% شاخص‌ها
\node[right=0.5cm of l1, font=\scriptsize, align=left] {شاخص‌ها: رسانه‌ها، نظرسنجی، گزارش محلی};
\node[right=0.5cm of l2, font=\scriptsize, align=left] {شاخص‌ها: اعتراضات کوچک، شکایات مکرر};
\node[right=0.5cm of l3, font=\scriptsize, align=left] {شاخص‌ها: تظاهرات، درگیری محدود};
\node[right=0.5cm of l4, font=\scriptsize, align=left] {شاخص‌ها: خشونت، اعتصاب گسترده};
\end{tikzpicture}
\captionof{figure}{سیستم هشدار زودهنگام تنش‌های قومی}
\label{fig:early-warning}
\end{center}

\subsection{مکانیزم‌های حل اختلاف}

\begin{table}[htbp]
\centering
\caption{ساختار چندلایه حل اختلافات قومی}
\label{tab:dispute-resolution}
\begin{tabular}{>{\columncolor{blue!8}}r p{3cm} p{3.5cm} p{3.5cm}}
\toprule
\rowcolor{blue!25}
\textbf{سطح} & \textbf{نهاد مسئول} & \textbf{ابزار} & \textbf{زمان‌بندی} \\
\midrule
محلی & شورای بزرگان محلی & گفتگو، میانجی‌گری سنتی & ۱-۷ روز \\
\rowcolor{gray!10}
استانی & دفتر حقوق اقوام استان & تحقیق، توصیه، مذاکره & ۱-۴ هفته \\
منطقه‌ای & شورای منطقه خودمختار & تصمیم‌گیری الزام‌آور & ۱-۳ ماه \\
\rowcolor{gray!10}
ملی & شورای عالی اقوام & میانجی‌گری، توصیه به دولت & ۱-۶ ماه \\
قضایی & دادگاه قانون اساسی & رأی نهایی و الزام‌آور & ۳-۱۲ ماه \\
\bottomrule
\end{tabular}
\end{table}

\subsection{خطوط قرمز و قواعد بازی}

\begin{enghelabbox}
\textbf{خطوط قرمز غیرقابل مذاکره}

برای حفظ توازن بین حقوق اقوام و وحدت ملی، خطوط قرمز روشن ضروری است:

\textbf{خطوط قرمز برای دولت مرکزی}:
\begin{itemize}[nosep]
\item ممنوعیت سرکوب نظامی اعتراضات مسالمت‌آمیز
\item ممنوعیت محدودیت زبان و فرهنگ
\item ممنوعیت تبعیض در استخدام و خدمات
\end{itemize}

\textbf{خطوط قرمز برای جنبش‌های قومی}:
\begin{itemize}[nosep]
\item ممنوعیت خشونت و تروریسم
\item ممنوعیت همکاری با دشمنان خارجی علیه ایران
\item ممنوعیت فعالیت تجزیه‌طلبانه مسلحانه
\end{itemize}

\textbf{قاعده طلایی}: اختلافات از طریق گفتگو و نهادهای قانونی حل می‌شوند، نه زور.
\end{enghelabbox}

%═══════════════════════════════════════════════════════════════════════════════
\section{تقویم اجرایی}
\label{sec:diversity-timeline}
%═══════════════════════════════════════════════════════════════════════════════

\begin{table}[htbp]
\centering
\caption{تقویم اجرای سیاست‌های مدیریت تنوع}
\label{tab:diversity-timeline}
\begin{tabular}{>{\columncolor{teal!8}}c p{4cm} p{6cm}}
\toprule
\rowcolor{teal!25}
\textbf{زمان} & \textbf{اقدام کلیدی} & \textbf{شاخص موفقیت} \\
\midrule
ماه ۱-۶ & اعلام منشور حقوق اقوام & پذیرش توسط رهبران قومی \\
\rowcolor{gray!10}
سال ۱ & تأسیس شورای عالی اقوام & برگزاری اولین نشست \\
سال ۱-۲ & شروع آموزش دوزبانه آزمایشی & ۵۰۰ مدرسه پایلوت \\
\rowcolor{gray!10}
سال ۲ & تصویب قانون مناطق خودمختار & تصویب در مجلس مؤسسان \\
سال ۳ & انتخابات شوراهای منطقه‌ای & مشارکت ۶۰٪+ در مناطق قومی \\
\rowcolor{gray!10}
سال ۳-۵ & اجرای برنامه توسعه متوازن & کاهش ۲۰٪ شکاف HDI \\
سال ۵ & ارزیابی میان‌دوره‌ای & نظرسنجی رضایت اقوام \\
\rowcolor{gray!10}
سال ۵-۱۰ & گسترش کامل آموزش چندزبانه & همه مناطق قومی \\
سال ۱۰ & ارزیابی جامع & کاهش ۵۰٪ شکاف توسعه \\
\rowcolor{gray!10}
سال ۱۵ & تحقق اهداف برابری & HDI همه مناطق بالای ۰.۸۰ \\
\bottomrule
\end{tabular}
\end{table}

%═══════════════════════════════════════════════════════════════════════════════
\section{شاخص‌های پایش}
\label{sec:diversity-indicators}
%═══════════════════════════════════════════════════════════════════════════════

\begin{table}[htbp]
\centering
\caption{شاخص‌های کلیدی پایش وضعیت اقوام}
\label{tab:diversity-kpis}
\begin{tabular}{>{\columncolor{orange!8}}r p{4cm} c c c}
\toprule
\rowcolor{orange!25}
\textbf{شاخص} & \textbf{توضیح} & \textbf{مبدأ} & \textbf{سال ۱۰} & \textbf{سال ۲۵} \\
\midrule
شکاف HDI & اختلاف بالاترین و پایین‌ترین استان & ۰.۱۷ & ۰.۱۰ & ۰.۰۵ \\
\rowcolor{gray!10}
رضایت اقوام & درصد «راضی» از حقوق فرهنگی & ۳۰٪ & ۶۵٪ & ۸۵٪ \\
نمایندگی در دولت & درصد مقامات از اقلیت‌های قومی & ۱۰٪ & ۲۵٪ & ۳۵٪ \\
\rowcolor{gray!10}
آموزش دوزبانه & پوشش مدارس در مناطق قومی & ۰٪ & ۷۰٪ & ۱۰۰٪ \\
رسانه محلی & ساعات پخش به زبان‌های محلی & ۵٪ & ۳۰٪ & ۵۰٪ \\
\rowcolor{gray!10}
تنش‌های قومی & تعداد حوادث خشونت‌آمیز/سال & — & ۵۰٪- & ۹۰٪- \\
هویت دوگانه & «هم قوم‌ام، هم ایرانی» & ۴۰٪ & ۶۵٪ & ۸۵٪ \\
\bottomrule
\end{tabular}
\end{table}

%═══════════════════════════════════════════════════════════════════════════════
\section{جمع‌بندی: وحدت در کثرت}
\label{sec:diversity-conclusion}
%═══════════════════════════════════════════════════════════════════════════════

\begin{olgoobox}
\textbf{پیام کلیدی فصل}

ایران می‌تواند نشان دهد که:
\begin{itemize}[nosep]
\item \textbf{تنوع قومی ثروت است}، نه تهدید — اگر درست مدیریت شود
\item \textbf{وحدت و تنوع با هم ممکن‌اند} — سوئیس، کانادا و هند این را ثابت کرده‌اند
\item \textbf{سرکوب راه‌حل نیست} — فقط مسئله را پنهان و تشدید می‌کند
\item \textbf{فدرالیسم همبسته} مدلی است که هم حقوق اقوام را تضمین می‌کند، هم وحدت ملی را
\item \textbf{عدالت اقتصادی پایه صلح قومی است} — توسعه متوازن ضروری است
\item \textbf{هویت ایرانی چتری فراگیر} است که همه را در بر می‌گیرد
\end{itemize}

\textbf{شعار}: «ایرانی بودن هویت مشترک ماست، تنوع قومی ثروت مشترک ما»
\end{olgoobox}

%═══════════════════════════════════════════════════════════════════════════════
% منابع فصل
%═══════════════════════════════════════════════════════════════════════════════

\section*{منابع فصل دوازدهم}
\addcontentsline{toc}{section}{منابع فصل دوازدهم}

\begin{itemize}[nosep, font=\small]
\item Lijphart, A. (1977). \textit{Democracy in Plural Societies}. Yale University Press.
\item Lijphart, A. (1999). \textit{Patterns of Democracy}. Yale University Press.
\item Kymlicka, W. (1995). \textit{Multicultural Citizenship}. Oxford University Press.
\item Horowitz, D. L. (1985). \textit{Ethnic Groups in Conflict}. University of California Press.
\item Gurr, T. R. (2000). \textit{Peoples Versus States: Minorities at Risk in the New Century}. USIP Press.
\item McGarry, J., \& O'Leary, B. (1993). \textit{The Politics of Ethnic Conflict Regulation}. Routledge.
\item Renan, E. (1882). \textit{What is a Nation?}
\item Brubaker, R. (1996). \textit{Nationalism Reframed}. Cambridge University Press.
\item احمدی، حمید. (۱۳۸۷). \textit{قومیت و قوم‌گرایی در ایران}. نشر نی.
\item کاتوزیان، محمدعلی همایون. (۱۳۹۲). \textit{تضاد دولت و ملت در ایران}. نشر نی.
\item آبراهامیان، یرواند. (۱۳۸۹). \textit{ایران بین دو انقلاب}. نشر نی.
\item UNDP. (2020). \textit{Human Development Report: Iran}.
\item مرکز آمار ایران. (۱۴۰۲). \textit{سرشماری نفوس و مسکن}.
\end{itemize}
	

% ch13-economy.tex
% فصل سیزدهم: بازسازی اقتصادی و رهایی از تحریم
% نویسنده: مهدی سالم | ریچموندهیل | ۱۴۰۴

\chapter{بازسازی اقتصادی و رهایی از تحریم}
\label{ch:economy}

\begin{kholasebox}
اقتصاد ایران با چالش‌های ساختاری عمیق مواجه است: تورم مزمن (۴۵-۵۰٪)، بیکاری بالا (۱۲٪ رسمی، واقعی بیشتر)، وابستگی به نفت (۳۵٪ بودجه)، تحریم‌های گسترده (۳۸۰۰+ تحریم)، فساد سیستماتیک (رتبه ۱۴۹ جهان)، و فرار سرمایه و مغزها. این فصل نقشه راهی برای بازسازی اقتصادی ارائه می‌دهد: رفع تحریم‌ها در کوتاه‌مدت، تثبیت اقتصاد کلان در میان‌مدت، و تحول ساختاری به اقتصاد متنوع و دانش‌بنیان در بلندمدت. اصل راهنما: \textbf{«آبادانی ملموس و فوری»} — مردم باید بهبود را در زندگی روزمره خود احساس کنند.
\end{kholasebox}

%═══════════════════════════════════════════════════════════════════════════════
\section{مقدمه: اقتصاد بیمار، دموکراسی شکننده}
%═══════════════════════════════════════════════════════════════════════════════

\begin{naghlbox}
«دموکراسی‌ها به ندرت در فقر زنده می‌مانند. هیچ دموکراسی‌ای با درآمد سرانه زیر ۶۰۰۰ دلار تاکنون پایدار نمانده است. اقتصاد سالم شرط لازم — هرچند نه کافی — برای دموکراسی پایدار است.»
\sourceline{آدام پرزورسکی، «دموکراسی و توسعه»، ۲۰۰۰}
\end{naghlbox}

موفقیت گذار دموکراتیک به شدت به عملکرد اقتصادی گره خورده است. مردم از نظام جدید انتظار بهبود زندگی دارند. اگر این انتظار برآورده نشود، مشروعیت دموکراسی زیر سؤال می‌رود و راه برای بازگشت اقتدارگرایی یا هرج‌ومرج باز می‌شود.

\subsection{چرا اقتصاد اولویت است؟}

\begin{center}
\begin{tikzpicture}[
    scale=0.85, every node/.style={scale=0.85},
    node distance=1.8cm,
    box/.style={
        rectangle,
        rounded corners=8pt,
        draw=#1!70,
        fill=#1!5,
        thick,
        minimum width=3.8cm,
        minimum height=1.6cm,
        align=center,
        font=\small
    },
    arrow/.style={->, thick, >=stealth}
]
% چرخه مثبت
\node[box=vertnapoleon] (econ) {\textbf{اقتصاد سالم}\\ رشد، اشتغال، ثبات};
\node[box=bleurepublique, right=2cm of econ] (legit) {\textbf{مشروعیت}\\ اعتماد به نظام جدید};
\node[box=bleurepublique, below=1.8cm of legit] (stab) {\textbf{ثبات سیاسی}\\ امکان اصلاحات بیشتر};
\node[box=goldphoenix, below=1.8cm of econ] (invest) {\textbf{سرمایه‌گذاری}\\ داخلی و خارجی};

\draw[arrow, green!60!black] (econ) -- (legit);
\draw[arrow, blue!60!black] (legit) -- (stab);
\draw[arrow, purple!60!black] (stab) -- (invest);
\draw[arrow, orange!60!black] (invest) -- (econ);

% عنوان
\node[above=0.5cm of econ, xshift=2cm, font=\large\bfseries] {چرخه فضیلت اقتصاد و دموکراسی};

% چرخه منفی (کوچک‌تر، در کنار)
\node[box=red, scale=0.7, right=5cm of legit] (crisis) {\textbf{بحران اقتصادی}};
\node[box=red, scale=0.7, below=1cm of crisis] (distrust) {\textbf{بی‌اعتمادی}};
\node[box=red, scale=0.7, below=1cm of distrust] (instab) {\textbf{بی‌ثباتی}};

\draw[arrow, red!60] (crisis) -- (distrust);
\draw[arrow, red!60] (distrust) -- (instab);
\draw[arrow, red!60] (instab.west) -- ++(-0.5,0) |- (crisis.west);

\node[above=0.3cm of crisis, font=\small\bfseries, red!70] {چرخه باطل};
\end{tikzpicture}
\end{center}

%═══════════════════════════════════════════════════════════════════════════════
\section{تشخیص: وضعیت اقتصاد ایران}
\label{sec:economic-diagnosis}
%═══════════════════════════════════════════════════════════════════════════════

\subsection{شاخص‌های کلان اقتصادی}

\begin{table}[htbp]
\centering
\caption{شاخص‌های کلیدی اقتصاد ایران (۱۴۰۳/۲۰۲۴)}
\label{tab:economic-indicators}
\begin{tabularx}{\textwidth}{C{1cm} R{4cm} Z Z Y}
\toprule
\headmark رد & \headmark شاخص & \headmark مقدار & \headmark رتبه جهانی & \headmark وضعیت \\
\midrule
۱ & GDP (PPP) & ۱.۳ تریلیون \$ & ۲۱ & متوسط \\
\rowcolor{goldlight}
۲ & GDP سرانه (PPP) & ۱۵,۰۰۰ \$ & ۷۰ & پایین‌تر از پتانسیل \\
۳ & رشد اقتصادی & ۲-۳٪ & — & ناکافی \\
\rowcolor{goldlight}
۴ & نرخ تورم & ۴۵-۵۰٪ & ۵ بدترین & بحرانی \\
۵ & نرخ بیکاری & ۱۲٪ & — & بالا \\
\rowcolor{goldlight}
۶ & بیکاری جوانان & ۲۸٪ & — & بحرانی \\
\bottomrule
\end{tabularx}
\end{table}

\subsection{مشکلات ساختاری}

\begin{center}
\begin{tikzpicture}[
    scale=0.85, every node/.style={scale=0.85},
    problem/.style={
        rectangle,
        rounded corners=5pt,
        draw=bleurepublique!70,
        fill=bleulight,
        thick,
        minimum width=4.5cm,
        minimum height=1cm,
        align=center,
        font=\small
    }
]
% عنوان
\node[font=\large\bfseries, color=bleurepublique] at (0,4.5) {هشت مشکل ساختاری اقتصاد ایران};

% مشکلات
\node[problem] at (-3.8,3) (p1) {\textbf{۱. وابستگی به نفت}\\ ۳۵٪ بودجه، ۶۰٪ صادرات};
\node[problem] at (3.8,3) (p2) {\textbf{۲. تحریم‌های گسترده}\\ ۳۸۰۰+ تحریم فعال};
\node[problem] at (-3.8,1.5) (p3) {\textbf{۳. تورم مزمن}\\ ۴۵-۵۰٪ سالانه};
\node[problem] at (3.8,1.5) (p4) {\textbf{۴. فساد سیستماتیک}\\ رتبه ۱۴۹ از ۱۸۰};
\node[problem] at (-3.8,0) (p5) {\textbf{۵. ناکارآمدی دولتی}\\ ۴۰٪ اقتصاد دولتی};
\node[problem] at (3.8,0) (p6) {\textbf{۶. نظام بانکی ورشکسته}\\ NPL بالای ۲۰٪};
\node[problem] at (-3.8,-1.5) (p7) {\textbf{۷. فرار سرمایه}\\ ۵-۱۰ میلیارد \$/سال};
\node[problem] at (3.8,-1.5) (p8) {\textbf{۸. فرار مغزها}\\ ۱۵۰,۰۰۰ نخبه/سال};

% اتصال به مرکز
\node[ellipse, draw=goldphoenix, fill=goldlight, thick, minimum width=2.8cm] at (0,0.7) (center) 
    {\textbf{بحران اقتصادی}};
\foreach \p in {p1,p2,p3,p4,p5,p6,p7,p8} {
    \draw[thick, goldphoenix!40, dashed] (\p) -- (center);
}
\end{tikzpicture}
\end{center}

\subsection{تحریم‌ها: چالش اصلی}

\begin{table}[htbp]
\centering
\caption{تحریم‌های بین‌المللی علیه ایران}
\label{tab:sanctions}
\begin{tabularx}{\textwidth}{R{4cm} C{3cm} Y}
\toprule
\headmark نوع تحریم & \headmark تعداد تقریبی & \headmark تأثیر اصلی \\
\midrule
تحریم‌های آمریکا & ۲,۵۰۰+ & قطع از سیستم دلاری، تحریم ثانویه \\
\rowcolor{goldlight}
تحریم‌های اتحادیه اروپا & ۸۰۰+ & محدودیت تجارت، بانکی \\
تحریم‌های سازمان ملل & ۵۰+ & تسلیحاتی، هسته‌ای \\
\rowcolor{goldlight}
تحریم‌های سایر کشورها & ۴۰۰+ & کانادا، استرالیا، ژاپن و... \\
\midrule
\headmark مجموع & \textbf{۳,۸۰۰+} & قطع از سیستم مالی جهانی \\
\bottomrule
\end{tabularx}
\end{table}

\begin{enghelabbox}
\textbf{هشدار: هزینه واقعی تحریم‌ها}

تحریم‌ها هزینه‌های سنگینی بر اقتصاد ایران تحمیل کرده‌اند:
\begin{itemize}[nosep]
\item \textbf{صادرات نفت}: از ۲.۵ میلیون بشکه/روز به کمتر از ۱ میلیون
\item \textbf{درآمد ارزی}: کاهش ۶۰-۷۰٪
\item \textbf{ارزش ریال}: سقوط از ۳,۵۰۰ به ۵۰۰,۰۰۰+ در برابر دلار
\item \textbf{تجارت خارجی}: محدودیت شدید واردات و صادرات
\item \textbf{سرمایه‌گذاری خارجی}: نزدیک به صفر
\item \textbf{هزینه انسانی}: کمبود دارو، تجهیزات پزشکی، تورم و فقر
\end{itemize}
\textbf{نتیجه}: رفع تحریم‌ها اولویت اول است، اما کافی نیست — اصلاحات ساختاری ضروری است.
\end{enghelabbox}

%═══════════════════════════════════════════════════════════════════════════════
\section{استراتژی سه‌مرحله‌ای بازسازی اقتصادی}
\label{sec:economic-strategy}
%═══════════════════════════════════════════════════════════════════════════════

\begin{center}
\begin{tikzpicture}[
    scale=0.85, every node/.style={scale=0.85},
    phase/.style={
        rectangle,
        rounded corners=10pt,
        draw=#1!70,
        fill=#1!5,
        thick,
        minimum width=3.8cm,
        minimum height=2.8cm,
        align=center
    }
]
% سه فاز
\node[phase=red] (p1) at (0,0) {
    \textbf{فاز ۱}\\ 
    \textbf{تثبیت}\\ 
    (سال ۱-۲)\\[0.2cm]
    \scriptsize رفع تحریم\\
    \scriptsize کنترل تورم\\
    \scriptsize ثبات ارزی
};

\node[phase=goldphoenix] (p2) at (5,0) {
    \textbf{فاز ۲}\\
    \textbf{بازسازی}\\
    (سال ۳-۷)\\[0.2cm]
    \scriptsize اصلاحات ساختاری\\
    \scriptsize جذب سرمایه\\
    \scriptsize نوسازی صنایع
};

\node[phase=bleurepublique] (p3) at (10,0) {
    \textbf{فاز ۳}\\
    \textbf{تحول}\\
    (سال ۸-۱۵)\\[0.2cm]
    \scriptsize اقتصاد دانش‌بنیان\\
    \scriptsize تنوع صادرات\\
    \scriptsize رفاه فراگیر
};

% فلش‌ها
\draw[->, ultra thick, red!60] (p1) -- (p2);
\draw[->, ultra thick, orange!60] (p2) -- (p3);

% شاخص‌های هدف
\node[below=0.5cm of p1, font=\scriptsize, align=center] {تورم: ۵۰٪→۱۵٪\\ رشد: ۲٪→۵٪};
\node[below=0.5cm of p2, font=\scriptsize, align=center] {تورم: ۱۵٪→۵٪\\ رشد: ۵٪→۸٪};
\node[below=0.5cm of p3, font=\scriptsize, align=center] {تورم: ۵٪→۲٪\\ رشد: ۶٪ پایدار};
\end{tikzpicture}
\captionof{figure}{استراتژی سه‌مرحله‌ای بازسازی اقتصادی}
\label{fig:economic-strategy}
\end{center}

%═══════════════════════════════════════════════════════════════════════════════
\section{فاز اول: تثبیت (سال ۱-۲)}
\label{sec:stabilization}
%═══════════════════════════════════════════════════════════════════════════════

\subsection{اولویت اول: رفع تحریم‌ها}

\begin{table}[htbp]
\centering
\caption{نقشه راه رفع تحریم‌ها}
\label{tab:sanctions-roadmap}
\begin{tabularx}{\textwidth}{C{1.2cm} Y Y Z}
\toprule
\headmark مرحله & \headmark اقدام ایران & \headmark اقدام طرف مقابل & \headmark زمان‌بندی \\
\midrule
۱ & اعلام پایبندی به NPT، توقف غنی‌سازی & تعلیق تحریم‌های نفتی & ماه ۱-۳ \\
\rowcolor{goldlight}
۲ & پذیرش بازرسی‌های IAEA، شفافیت کامل & رفع تحریم‌های بانکی & ماه ۳-۶ \\
۳ & امضای پروتکل الحاقی & رفع تحریم‌های تجاری & ماه ۶-۱۲ \\
\rowcolor{goldlight}
۴ & تعهد به عدم توسعه سلاح & رفع تحریم‌های ثانوویه & ماه ۱۲-۱۸ \\
۵ & توافق جدید پایدار & آزادسازی دارایی‌ها & ماه ۱۸-۲۴ \\
\bottomrule
\end{tabularx}
\end{table}

\begin{naghlbox}
«رفع تحریم‌ها شرط لازم است، اما کافی نیست. کشورهای زیادی بدون تحریم هم فقیر مانده‌اند. آنچه تعیین‌کننده است، کیفیت نهادها و سیاست‌های اقتصادی است.»
\sourceline{دارون عجم‌اوغلو و جیمز رابینسون، «چرا ملت‌ها شکست می‌خورند»، ۲۰۱۲}
\end{naghlbox}

\subsection{کنترل تورم}

\begin{table}[htbp]
\centering
\caption{برنامه کنترل تورم در دو سال اول}
\label{tab:inflation-control}
\begin{tabularx}{\textwidth}{R{3cm} Y Y}
\toprule
\headmark ابزار & \headmark اقدام کلیدی & \headmark اثر مورد انتظار \\
\midrule
سیاست پولی & استقلال بانک مرکزی، هدف‌گذاری تورم & توقف چاپ پول، ثبات قیمت‌ها \\
\rowcolor{goldlight}
سیاست مالی & کاهش کسری بودجه به زیر ۳٪ GDP & کاهش تقاضای کاذب \\
سیاست ارزی & یکسان‌سازی نرخ ارز، شناورسازی & حذف رانت ارزی، بازگشت اعتماد \\
\rowcolor{goldlight}
یارانه‌ها & هدفمندسازی واقعی (پرداخت مستقیم) & حمایت از اقشار ضعیف \\
\bottomrule
\end{tabularx}
\end{table}

\begin{center}
\begin{tikzpicture}
\begin{axis}[
    width=0.95\textwidth,
    height=6cm,
    xlabel={\rl{ماه}},
    ylabel={\rl{تورم (درصد سالانه)}},
    xmin=0, xmax=25,
    ymin=0, ymax=60,
    xtick={0,6,12,18,24},
    xticklabels={\rl{شروع}, \rl{ماه ۶}, \rl{ماه ۱۲}, \rl{ماه ۱۸}, \rl{ماه ۲۴}},
    legend pos=north east,
    grid=major,
    grid style={dashed, gray!30},
    axis line style={bleurepublique!50, thick}
]
% مسیر تورم هدف
\addplot[color=goldphoenix, mark=*, thick, line width=1.5pt] coordinates {
    (0, 50) (3, 45) (6, 38) (9, 30) (12, 25) (15, 20) (18, 17) (21, 15) (24, 12)
};
% مسیر بدون اصلاحات
\addplot[color=black!40, mark=triangle*, dashed, thick] coordinates {
    (0, 50) (6, 52) (12, 55) (18, 53) (24, 50)
};

\legend{با اصلاحات, بدون اصلاحات}

% ناحیه هدف
\fill[green!20, opacity=0.3] (axis cs:18,0) rectangle (axis cs:24,15);
\node[font=\scriptsize] at (axis cs:21,8) {هدف};
\end{axis}
\end{tikzpicture}
\captionof{figure}{مسیر کاهش تورم در دو سال اول}
\label{fig:inflation-path}
\end{center}

\subsection{برنامه ۱۰۰ روز اول اقتصادی}

\begin{table}[htbp]
\centering
\caption{اقدامات اقتصادی فوری در ۱۰۰ روز اول}
\label{tab:100-days-economic}
\begin{tabular}{>{\columncolor{orange!8}}c p{4cm} p{5.5cm}}
\toprule
\rowcolor{orange!25}
\textbf{روز} & \textbf{اقدام} & \textbf{هدف فوری} \\
\midrule
۱-۷ & اعلام سیاست اقتصادی جدید & ایجاد اعتماد، جلوگیری از پانیک \\
\rowcolor{gray!10}
۱-۱۴ & تماس با IMF و بانک جهانی & درخواست مشاوره و کمک فنی \\
۱-۳۰ & یکسان‌سازی نرخ ارز & حذف رانت، شفافیت \\
\rowcolor{gray!10}
۱-۳۰ & آزادسازی واردات کالاهای اساسی & کاهش فوری قیمت‌ها \\
۳۰-۶۰ & اصلاح قانون بانک مرکزی & استقلال پولی \\
\rowcolor{gray!10}
۳۰-۶۰ & شروع مذاکرات رفع تحریم & سیگنال به بازارها \\
۶۰-۱۰۰ & ارائه لایحه بودجه اصلاح‌شده & کاهش کسری، شفافیت \\
\rowcolor{gray!10}
۶۰-۱۰۰ & پرداخت یارانه نقدی هدفمند & حمایت از اقشار آسیب‌پذیر \\
\bottomrule
\end{tabular}
\end{table}

\subsection{حمایت از اقشار آسیب‌پذیر}

\begin{olgoobox}
\textbf{الگوی موفق: برزیل — برنامه بولسا فامیلیا}

برزیل با برنامه یارانه نقدی مشروط (Bolsa Família) توانست فقر را به شدت کاهش دهد:
\begin{itemize}[nosep]
\item پرداخت نقدی به ۱۴ میلیون خانوار فقیر
\item شرط: فرستادن کودکان به مدرسه، واکسیناسیون، معاینات پزشکی
\item هزینه: کمتر از ۰.۵٪ GDP
\item نتیجه: کاهش فقر از ۲۶٪ به ۱۰٪ در ۱۵ سال
\item \textbf{درس برای ایران}: یارانه نقدی هدفمند + شرایط توانمندسازی
\end{itemize}
\end{olgoobox}

\begin{table}[htbp]
\centering
\caption{برنامه حمایت اجتماعی در دوره گذار}
\label{tab:social-protection}
\begin{tabular}{>{\columncolor{purple!8}}r p{4cm} c p{3.5cm}}
\toprule
\rowcolor{purple!25}
\textbf{برنامه} & \textbf{گروه هدف} & \textbf{بودجه سالانه} & \textbf{پوشش} \\
\midrule
یارانه نقدی تقویت‌شده & دهک‌های ۱-۴ & ۲۰ میلیارد \$ & ۳۰ میلیون نفر \\
\rowcolor{gray!10}
بیمه بیکاری موقت & کارگران تعدیل‌شده & ۵ میلیارد \$ & ۲ میلیون نفر \\
سبد کالای اساسی & فقر مطلق & ۳ میلیارد \$ & ۱۰ میلیون نفر \\
\rowcolor{gray!10}
کمک مسکن & بی‌سرپناهان، مستأجران & ۴ میلیارد \$ & ۵ میلیون نفر \\
بهداشت رایگان & همه & ۱۵ میلیارد \$ & ۸۷ میلیون نفر \\
\bottomrule
\end{tabular}
\end{table}

%═══════════════════════════════════════════════════════════════════════════════
\section{فاز دوم: بازسازی (سال ۳-۷)}
\label{sec:reconstruction}
%═══════════════════════════════════════════════════════════════════════════════

\subsection{اصلاحات ساختاری}

\begin{table}[htbp]
\centering
\caption{بسته اصلاحات ساختاری اقتصادی}
\label{tab:structural-reforms}
\begin{tabular}{>{\columncolor{blue!8}}r p{3cm} p{3.5cm} p{3.5cm}}
\toprule
\rowcolor{blue!25}
\textbf{حوزه} & \textbf{مشکل فعلی} & \textbf{اصلاح پیشنهادی} & \textbf{زمان‌بندی} \\
\midrule
بنگاه‌های دولتی & ناکارآمدی، زیان‌دهی & خصوصی‌سازی شفاف & سال ۳-۷ \\
\rowcolor{gray!10}
نظام بانکی & NPL بالا، ورشکستگی & تجدید ساختار، ادغام & سال ۲-۵ \\
نظام مالیاتی & فرار مالیاتی گسترده & اصلاح قانون، دیجیتال‌سازی & سال ۲-۴ \\
\rowcolor{gray!10}
بازار کار & صلبیت، غیررسمی بالا & انعطاف‌پذیری، حمایت از کارگر & سال ۳-۵ \\
رقابت & انحصارات گسترده & قانون رقابت، شکستن انحصار & سال ۲-۵ \\
\rowcolor{gray!10}
زمین و مسکن & سوداگری، قیمت بالا & مالیات بر زمین، عرضه مسکن & سال ۲-۷ \\
\bottomrule
\end{tabular}
\end{table}

\subsection{خصوصی‌سازی: این‌بار درست}

\begin{enghelabbox}
\textbf{هشدار: درس‌های خصوصی‌سازی ناموفق}

خصوصی‌سازی در ایران (اصل ۴۴) عمدتاً شکست خورده:
\begin{itemize}[nosep]
\item انتقال به «خودی‌ها» و نهادهای شبه‌دولتی
\item فساد گسترده در واگذاری‌ها
\item عدم بهبود کارایی
\item تمرکز ثروت در دست عده‌ای
\end{itemize}

\textbf{اصول خصوصی‌سازی صحیح}:
\begin{itemize}[nosep]
\item شفافیت کامل: مزایده علنی، قیمت‌گذاری مستقل
\item ممنوعیت خرید توسط نهادهای وابسته به دولت
\item حفظ حقوق کارگران: بازآموزی، بیمه بیکاری
\item رقابت: جلوگیری از انحصار خصوصی
\item نظارت مستقل بر فرآیند
\end{itemize}
\end{enghelabbox}

\begin{table}[htbp]
\centering
\caption{برنامه خصوصی‌سازی اولویت‌بندی‌شده}
\label{tab:privatization-plan}
\begin{tabular}{>{\columncolor{cyan!8}}l c c p{3.5cm}}
\toprule
\rowcolor{cyan!25}
\textbf{بخش} & \textbf{سهم دولت فعلی} & \textbf{هدف سال ۷} & \textbf{روش} \\
\midrule
مخابرات & ۶۰٪ & ۲۰٪ & فروش سهام در بورس \\
\rowcolor{gray!10}
بانک‌ها & ۷۰٪ & ۳۰٪ & ادغام و عرضه عمومی \\
خودروسازی & ۸۵٪ & ۲۰٪ & مشارکت خارجی + IPO \\
\rowcolor{gray!10}
فولاد و معدن & ۵۰٪ & ۲۵٪ & مزایده بین‌المللی \\
حمل‌ونقل & ۹۰٪ & ۴۰٪ & واگذاری تدریجی \\
\rowcolor{gray!10}
نفت و گاز & ۱۰۰٪ & ۸۰٪ & حفظ مالکیت ملی، قراردادهای جدید \\
\bottomrule
\end{tabular}
\end{table}

\subsection{جذب سرمایه‌گذاری خارجی}

\begin{table}[htbp]
\centering
\caption{اهداف جذب سرمایه‌گذاری خارجی (FDI)}
\label{tab:fdi-targets}
\begin{tabular}{>{\columncolor{green!8}}l c c c c c}
\toprule
\rowcolor{green!25}
\textbf{شاخص} & \textbf{۱۴۰۳} & \textbf{سال ۳} & \textbf{سال ۵} & \textbf{سال ۷} & \textbf{سال ۱۰} \\
\midrule
FDI سالانه (میلیارد \$) & ۱ & ۱۰ & ۲۵ & ۴۰ & ۶۰ \\
\rowcolor{gray!10}
موجودی FDI (میلیارد \$) & ۶۰ & ۸۰ & ۱۳۰ & ۲۰۰ & ۳۵۰ \\
رتبه سهولت کسب‌وکار & ۱۲۷ & ۸۰ & ۵۰ & ۳۵ & ۲۵ \\
\bottomrule
\end{tabular}
\end{table}

\begin{center}
\begin{tikzpicture}
\begin{axis}[
    width=12cm,
    height=6cm,
    ybar,
    bar width=15pt,
    ylabel={\rl{میلیارد دلار}},
    xlabel={\rl{سال}},
    ymin=0,
    ymax=70,
    xtick=data,
    xticklabels={\rl{۱۴۰۳}, \rl{سال ۳}, \rl{سال ۵}, \rl{سال ۷}, \rl{سال ۱۰}},
    nodes near coords,
    every node near coord/.append style={font=\small},
    legend pos=north west,
    grid=major,
    grid style={dashed, gray!30}
]
\addplot[fill=blue!60, draw=blue!80] coordinates {
    (1, 1) (2, 10) (3, 25) (4, 40) (5, 60)
};
% خط ترکیه برای مقایسه
\addplot[mark=*, red, thick] coordinates {
    (1, 13) (2, 13) (3, 13) (4, 13) (5, 13)
};

\legend{ایران (هدف), ترکیه (میانگین)}
\end{axis}
\end{tikzpicture}
\captionof{figure}{اهداف جذب سرمایه‌گذاری خارجی}
\label{fig:fdi-targets}
\end{center}

\textbf{اقدامات کلیدی برای جذب FDI}:
\begin{itemize}[nosep]
\item قانون جدید سرمایه‌گذاری خارجی با تضمین‌های قوی
\item داوری بین‌المللی برای اختلافات
\item مناطق آزاد واقعی (نه صوری)
\item حذف بوروکراسی: مجوز یکپارچه
\item ثبات سیاسی و حاکمیت قانون
\end{itemize}

\subsection{نوسازی صنایع}

\begin{table}[htbp]
\centering
\caption{برنامه نوسازی صنایع کلیدی}
\label{tab:industry-modernization}
\begin{tabularx}{\textwidth}{R{3cm} Y Z Y}
\toprule
\headmark صنعت & \headmark چالش فعلی & \headmark سرمایه (\$م) & \headmark هدف نهایی \\
\midrule
خودرو & کیفیت پایین، فناوری قدیمی & ۲۰ & مشارکت با برندهای جهانی \\
\rowcolor{goldlight}
پتروشیمی & خام‌فروشی، فناوری قدیمی & ۳۰ & زنجیره ارزش کامل \\
فولاد & انرژی‌بر، آلاینده & ۱۵ & فناوری سبز، صادرات \\
\rowcolor{goldlight}
نساجی & رقابت‌پذیری پایین & ۵ & نوسازی، برندسازی \\
غذایی & وابستگی به واردات & ۱۰ & خودکفایی، صادرات \\
\rowcolor{goldlight}
دارو و تجهیزات پزشکی & واردات بالا & ۸ & تولید داخلی ۸۰٪ \\
\bottomrule
\end{tabularx}
\end{table}

%═══════════════════════════════════════════════════════════════════════════════
\section{فاز سوم: تحول ساختاری (سال ۸-۱۵)}
\label{sec:transformation}
%═══════════════════════════════════════════════════════════════════════════════

\subsection{تنوع‌بخشی به اقتصاد}

\begin{center}
\begin{tikzpicture}
\begin{axis}[
    width=0.95\textwidth,
    height=7cm,
    ybar stacked,
    bar width=18pt,
    ylabel={\rl{درصد صادرات}},
    xlabel={},
    ymin=0, ymax=100,
    xtick={1,2,3,4},
    xticklabels={\rl{۱۴۰۳}, \rl{سال ۵}, \rl{سال ۱۰}, \rl{سال ۱۵}},
    legend style={
        at={(0.5,-0.15)},
        anchor=north,
        legend columns=3,
        font=\tiny
    },
    nodes near coords,
    every node near coord/.append style={font=\tiny},
    grid=major,
    grid style={dashed, gray!30},
    axis line style={bleurepublique!50, thick}
]
\addplot[fill=black!40, draw=black!60] coordinates {(1,60) (2,45) (3,30) (4,20)};
\addplot[fill=bleulight, draw=bleurepublique!70] coordinates {(1,20) (2,22) (3,22) (4,20)};
\addplot[fill=goldlight, draw=goldphoenix!70] coordinates {(1,10) (2,15) (3,22) (4,25)};
\addplot[fill=vertlight, draw=vertnapoleon!70] coordinates {(1,5) (2,10) (3,15) (4,20)};
\addplot[fill=bleurepublique!20, draw=bleurepublique] coordinates {(1,5) (2,8) (3,11) (4,15)};
\legend{\rl{نفت خام}, \rl{پتروشیمی}, \rl{صنایع}, \rl{گردشگری}, \rl{فناوری}}
\end{axis}
\end{tikzpicture}
\captionof{figure}{تنوع‌بخشی به سبد صادراتی ایران}
\end{center}

\subsection{موتورهای رشد جدید}

\begin{table}[htbp]
\centering
\caption{پنج موتور رشد اقتصادی آینده ایران}
\label{tab:growth-engines}
\begin{tabularx}{\textwidth}{R{2.8cm} Y Y Z Z}
\toprule
\headmark موتور & \headmark پتانسیل & \headmark هدف سال ۱۵ & \headmark سرمایه (\$م) & \headmark اشتغال \\
\midrule
گردشگری & ۵۰ سایت یونسکو & ۳۰ میلیون نفر & ۵۰ & ۳ م \\
\rowcolor{goldlight}
ترانزیت & هاب منطقه‌ای & ۵۰ میلیارد \$ درآمد & ۴۰ & ۱ م \\
فناوری اطلاعات & نیروی متخصص & ۲۰ میلیارد \$ صادرات & ۱۵ & ۱ م \\
\rowcolor{goldlight}
انرژی سبز & آفتاب و باد & ۱۰ میلیارد \$ صادرات & ۸۰ & ۰.۵ م \\
\bottomrule
\end{tabularx}
\end{table}

\subsubsection{گردشگری: گنج پنهان}

\begin{olgoobox}
\textbf{پتانسیل گردشگری ایران}

ایران یکی از ۱۰ کشور برتر جهان از نظر جاذبه‌های گردشگری است:
\begin{itemize}[nosep]
\item \textbf{میراث فرهنگی}: ۲۷ سایت ثبت یونسکو (رتبه ۱۰ جهان)
\item \textbf{تنوع طبیعی}: کویر، جنگل، کوه، ساحل
\item \textbf{تنوع فرهنگی}: موسیقی، غذا، صنایع‌دستی اقوام
\item \textbf{گردشگری سلامت}: پزشکی ارزان و باکیفیت
\item \textbf{گردشگری مذهبی}: مشهد، قم (۲۰+ میلیون زائر/سال)
\end{itemize}

\textbf{مقایسه}:
\begin{itemize}[nosep]
\item ترکیه: ۵۰ میلیون گردشگر، ۳۵ میلیارد دلار
\item امارات: ۲۰ میلیون گردشگر، ۴۰ میلیارد دلار
\item ایران (فعلی): ۵ میلیون گردشگر، ۳ میلیارد دلار
\item \textbf{ایران (هدف ۱۵ سال)}: ۳۰ میلیون گردشگر، ۵۰ میلیارد دلار
\end{itemize}
\end{olgoobox}

\subsubsection{ترانزیت: پل بین قاره‌ها}

\begin{center}
\begin{tikzpicture}[scale=0.75, every node/.style={scale=0.75}]
% نقشه ساده
\draw[thick, draw=bleurepublique!50, fill=bleulight] (0,0) ellipse (6.5cm and 3cm);
\node[font=\large\bfseries, color=bleurepublique] at (0,2.5) {ایران: هاب ترانزیت منطقه‌ای};

% مسیرها
\node[circle, draw=goldphoenix, fill=goldlight, minimum size=0.9cm] (n1) at (-5,0) {\scriptsize اروپا};
\node[circle, draw=goldphoenix, fill=goldlight, minimum size=0.9cm] (n2) at (5,0) {\scriptsize چین};
\node[circle, draw=goldphoenix, fill=goldlight, minimum size=0.9cm] (n3) at (0,2) {\scriptsize روسیه};
\node[circle, draw=goldphoenix, fill=goldlight, minimum size=0.9cm] (n4) at (0,-2) {\scriptsize هند};
\node[circle, draw=goldphoenix, fill=goldlight, minimum size=0.9cm] (n5) at (-3,-1.5) {\scriptsize خلیج فارس};
\node[circle, draw=goldphoenix, fill=goldlight, minimum size=0.9cm] (n6) at (3,1.5) {\scriptsize میانه};

% ایران در مرکز
\node[rectangle, rounded corners, draw=bleurepublique, fill=goldphoenix, text=white, minimum width=2cm, minimum height=1cm] (iran) at (0,0) {\textbf{ایران}};

% مسیرها
\foreach \n in {n1,n2,n3,n4,n5,n6} {
    \draw[ultra thick, bleurepublique!60, ->, >=stealth] (\n) -- (iran);
}

% کریدورها
\node[font=\scriptsize, below=3.5cm of iran, align=center, bleurepublique] {
    کریدور شمال-جنوب (INSTC) | کریدور شرق-غرب | راه ابریشم جدید
};
\end{tikzpicture}
\end{center}

\subsection{اشتغال‌زایی}

\begin{table}[htbp]
\centering
\caption{برنامه اشتغال‌زایی ۱۰ میلیون شغل در ۱۰ سال}
\label{tab:job-creation}
\begin{tabularx}{\textwidth}{R{2.8cm} C{1.2cm} C{1.2cm} Y}
\toprule
\headmark بخش & \headmark فعلی & \headmark هدف & \headmark راهبرد کلیدی \\
\midrule
گردشگری و هتل & ۰.۵ & ۳ & زیرساخت، ویزا، بازاریابی \\
\rowcolor{goldlight}
ساختمان و مسکن & ۳ & ۵ & پروژه نوسازی بافت فرسوده \\
صنایع تولیدی & ۴ & ۶ & نوسازی، توسعه صادرات \\
\rowcolor{goldlight}
فناوری اطلاعات & ۰.۳ & ۱.۳ & اقتصاد دیجیتال، استارتاپ‌ها \\
کشاورزی مدرن & ۴ & ۴.۵ & مکانیزاسیون، صرفه‌جویی آب \\
\rowcolor{goldlight}
خدمات و تجارت & ۸ & ۱۲ & لجستیک، مالی، آموزش \\
\midrule
\headmark مجموع & \textbf{۲۴} & \textbf{۳۴} & \textbf{۱۰ میلیون شغل جدید} \\
\bottomrule
\end{tabularx}
\end{table}

%═══════════════════════════════════════════════════════════════════════════════
\section{مبارزه با فساد اقتصادی}
\label{sec:anti-corruption}
%═══════════════════════════════════════════════════════════════════════════════

\begin{naghlbox}
«فساد بزرگ‌ترین دشمن توسعه است. هر دلاری که به جیب فاسدان می‌رود، از مدرسه، بیمارستان و زیرساخت کم می‌شود. بدون مبارزه جدی با فساد، هیچ برنامه اقتصادی موفق نخواهد شد.»
\sourceline{بانک جهانی، گزارش فساد و توسعه، ۲۰۱۷}
\end{naghlbox}

\begin{table}[htbp]
\centering
\caption{برنامه جامع مبارزه با فساد اقتصادی}
\label{tab:anti-corruption-plan}
\begin{tabularx}{\textwidth}{R{2.5cm} Y Y}
\toprule
\headmark محور & \headmark اقدام کلیدی & \headmark نتیجه مورد انتظار \\
\midrule
شفافیت & ثبت الکترونیک معاملات دولتی & حذف رانت اطلاعاتی \\
\rowcolor{goldlight}
نظارت & کمیسیون مستقل ضدفساد & بازداشت و محاکمه مفسدان \\
قوانین & قانون «از کجا آورده‌ای» & مصادره اموال نامشروع \\
\rowcolor{goldlight}
تعارض منافع & ممنوعیت فعالیت اقتصادی مقامات & جلوگیری از سوءاستفاده \\
مناقصات & سامانه علنی و الکترونیکی & سلامت قراردادهای بزرگ \\
\bottomrule
\end{tabularx}
\end{table}

\begin{center}
\begin{tikzpicture}
\begin{axis}[
    width=0.95\textwidth,
    height=6cm,
    xlabel={\rl{سال}},
    ylabel={\rl{رتبه شاخص CPI}},
    xmin=0, xmax=16,
    ymin=20, ymax=160,
    xtick={0,5,10,15},
    xticklabels={\rl{۱۴۰۳}, \rl{سال ۵}, \rl{سال ۱۰}, \rl{سال ۱۵}},
    legend pos=north east,
    grid=major,
    grid style={dashed, gray!30},
    y dir=reverse,
    axis line style={bleurepublique!50, thick}
]
\addplot[color=goldphoenix, mark=*, thick, line width=1.5pt] coordinates {
    (0, 149) (5, 100) (10, 70) (15, 45)
};
\addplot[color=black!40, dashed, thick] coordinates {
    (0, 101) (5, 101) (10, 101) (15, 101)
};
\legend{\rl{ایران (هدف)}, \rl{ترکیه (فعلی)}}
\end{axis}
\end{tikzpicture}
\captionof{figure}{مسیر بهبود رتبه ایران در شاخص فساد}
\end{center}

%═══════════════════════════════════════════════════════════════════════════════
\section{نظام مالیاتی جدید}
\label{sec:tax-reform}
%═══════════════════════════════════════════════════════════════════════════════

\subsection{اصلاح ساختار مالیاتی}

\begin{table}[htbp]
\centering
\caption{مقایسه نظام مالیاتی فعلی و پیشنهادی}
\label{tab:tax-comparison}
\begin{tabular}{>{\columncolor{blue!8}}l c c p{3.5cm}}
\toprule
\rowcolor{blue!25}
\textbf{شاخص} & \textbf{فعلی} & \textbf{هدف سال ۱۰} & \textbf{توضیح} \\
\midrule
نسبت مالیات به GDP & ۷٪ & ۱۸٪ & میانگین OECD: ۳۴٪ \\
\rowcolor{gray!10}
مالیات بر درآمد (حداکثر) & ۳۵٪ & ۴۵٪ & تصاعدی، عادلانه \\
مالیات بر شرکت‌ها & ۲۵٪ & ۲۰٪ & کاهش برای جذب سرمایه \\
\rowcolor{gray!10}
مالیات بر ارزش‌افزوده & ۹٪ & ۱۵٪ & استاندارد جهانی \\
مالیات بر ثروت & — & ۱-۲٪ & جدید، برای عدالت \\
\rowcolor{gray!10}
مالیات بر املاک & ناچیز & ۱٪ ارزش & ضد سوداگری \\
\bottomrule
\end{tabular}
\end{table}

\textbf{اصول نظام مالیاتی جدید}:
\begin{enumerate}[nosep]
\item \textbf{عدالت}: ثروتمندان بیشتر بپردازند (تصاعدی)
\item \textbf{کارایی}: حذف معافیت‌های ناکارآمد
\item \textbf{سادگی}: فرم‌های ساده، پرداخت الکترونیک
\item \textbf{شفافیت}: همه مالیات‌دهندگان ببینند پولشان کجا می‌رود
\item \textbf{اجرای قاطع}: مجازات سنگین برای فرار مالیاتی
\end{enumerate}

%═══════════════════════════════════════════════════════════════════════════════
\section{ادغام در اقتصاد جهانی}
\label{sec:global-integration}
%═══════════════════════════════════════════════════════════════════════════════

\begin{table}[htbp]
\centering
\caption{نقشه راه ادغام در اقتصاد جهانی}
\label{tab:global-integration}
\begin{tabular}{>{\columncolor{green!8}}c p{4cm} p{5.5cm}}
\toprule
\rowcolor{green!25}
\textbf{سال} & \textbf{اقدام} & \textbf{دستاورد} \\
\midrule
۱-۲ & رفع تحریم‌ها، اتصال به SWIFT & دسترسی به سیستم مالی جهانی \\
\rowcolor{gray!10}
۲-۳ & عضویت ناظر در WTO & شروع مذاکرات الحاق \\
۳-۵ & توافقات تجارت آزاد دوجانبه & ترکیه، هند، چین، اوراسیا \\
\rowcolor{gray!10}
۵-۷ & عضویت کامل در WTO & دسترسی به بازار ۱۶۴ کشور \\
۷-۱۰ & توافق تجاری با اتحادیه اروپا & بزرگ‌ترین بازار جهان \\
\rowcolor{gray!10}
۱۰-۱۵ & عضویت در OECD (هدف) & استانداردهای جهانی \\
\bottomrule
\end{tabular}
\end{table}

\subsection{توافقات تجاری هدف}

\begin{center}
\begin{tikzpicture}[
    partner/.style={
        ellipse,
        draw=#1!70,
        fill=#1!20,
        thick,
        minimum width=2.5cm,
        minimum height=1.2cm,
        align=center,
        font=\small
    }
]
% ایران در مرکز
\node[rectangle, rounded corners=10pt, draw=blue!70, fill=blue!20,
      thick, minimum width=2.5cm, minimum height=1.5cm] (iran) 
    {\textbf{ایران}\\هاب تجاری};

% شرکای تجاری
\node[partner=green, above=2cm of iran] (eu) {اتحادیه اروپا\\ \scriptsize ۴۵۰ میلیون نفر};
\node[partner=red, above right=1.5cm of iran] (china) {چین\\ \scriptsize ۱.۴ میلیارد};
\node[partner=orange, right=2.5cm of iran] (india) {هند\\ \scriptsize ۱.۴ میلیارد};
\node[partner=purple, below right=1.5cm of iran] (gulf) {خلیج فارس\\ \scriptsize ۵۰ میلیون};
\node[partner=teal, below=2cm of iran] (africa) {آفریقا\\ \scriptsize ۱.۴ میلیارد};
\node[partner=darkyellow, below left=1.5cm of iran] (eurasia) {اوراسیا\\ \scriptsize ۲۰۰ میلیون};
\node[partner=cyan, left=2.5cm of iran] (turkey) {ترکیه\\ \scriptsize ۸۵ میلیون};
\node[partner=pink, above left=1.5cm of iran] (cis) {آسیای مرکزی\\ \scriptsize ۷۵ میلیون};

% اتصالات با حجم تجارت هدف
\draw[thick, green!60, ->] (iran) -- node[right, font=\tiny] {۵۰B\$} (eu);
\draw[thick, red!60, ->] (iran) -- node[above, font=\tiny] {۴۰B\$} (china);
\draw[thick, orange!60, ->] (iran) -- node[above, font=\tiny] {۳۰B\$} (india);
\draw[thick, purple!60, ->] (iran) -- node[right, font=\tiny] {۲۵B\$} (gulf);
\draw[thick, teal!60, ->] (iran) -- node[left, font=\tiny] {۱۵B\$} (africa);
\draw[thick, yellow!60!black, ->] (iran) -- node[left, font=\tiny] {۲۰B\$} (eurasia);
\draw[thick, cyan!60, ->] (iran) -- node[above, font=\tiny] {۳۰B\$} (turkey);
\draw[thick, pink!60, ->] (iran) -- node[left, font=\tiny] {۱۵B\$} (cis);

% عنوان
\node[above=3.5cm of iran, font=\large\bfseries] {شبکه تجاری هدف ایران — افق ۱۵ ساله};
\end{tikzpicture}
\end{center}

\subsection{هدف صادرات}

\begin{table}[htbp]
\centering
\caption{اهداف صادراتی ایران در افق ۱۵ ساله}
\label{tab:export-targets}
\begin{tabular}{>{\columncolor{orange!8}}l c c c c}
\toprule
\rowcolor{orange!25}
\textbf{بخش} & \textbf{۱۴۰۳ (میلیارد \$)} & \textbf{سال ۵} & \textbf{سال ۱۰} & \textbf{سال ۱۵} \\
\midrule
نفت و گاز & ۵۰ & ۸۰ & ۱۰۰ & ۸۰ \\
\rowcolor{gray!10}
پتروشیمی & ۱۵ & ۲۵ & ۴۰ & ۵۰ \\
صنایع و معدن & ۱۰ & ۲۰ & ۳۵ & ۵۰ \\
\rowcolor{gray!10}
کشاورزی و غذا & ۵ & ۱۰ & ۱۸ & ۲۵ \\
خدمات (گردشگری، IT) & ۵ & ۱۵ & ۳۵ & ۶۰ \\
\rowcolor{gray!10}
فناوری و دانش‌بنیان & ۲ & ۸ & ۱۷ & ۳۵ \\
\midrule
\textbf{مجموع صادرات} & \textbf{۸۷} & \textbf{۱۵۸} & \textbf{۲۴۵} & \textbf{۳۰۰} \\
\bottomrule
\end{tabular}
\end{table}

%═══════════════════════════════════════════════════════════════════════════════
\section{نظام بانکی و مالی}
\label{sec:banking-reform}
%═══════════════════════════════════════════════════════════════════════════════

\subsection{بحران نظام بانکی}

\begin{enghelabbox}
\textbf{هشدار: بانک‌های ایران در وضعیت بحرانی هستند}

وضعیت نظام بانکی ایران نگران‌کننده است:
\begin{itemize}[nosep]
\item \textbf{مطالبات معوق (NPL)}: بالای ۲۰٪ (استاندارد جهانی: زیر ۵٪)
\item \textbf{کفایت سرمایه}: بسیاری زیر حداقل استاندارد
\item \textbf{بانک‌های ورشکسته}: چندین بانک عملاً ورشکسته
\item \textbf{بنگاه‌داری}: بانک‌ها مالک شرکت‌های زیان‌ده
\item \textbf{فساد}: وام‌های کلان به افراد مرتبط
\end{itemize}
\textbf{نتیجه}: بدون اصلاح نظام بانکی، رشد اقتصادی پایدار ممکن نیست.
\end{enghelabbox}

\begin{table}[htbp]
\centering
\caption{برنامه اصلاح نظام بانکی}
\label{tab:banking-reform}
\begin{tabular}{>{\columncolor{blue!8}}r p{4cm} p{5.5cm}}
\toprule
\rowcolor{blue!25}
\textbf{مرحله} & \textbf{اقدام} & \textbf{زمان‌بندی و هدف} \\
\midrule
۱ & ارزیابی مستقل (Asset Quality Review) & ماه ۱-۶: شناسایی وضعیت واقعی \\
\rowcolor{gray!10}
۲ & تفکیک بانک‌ها به سالم/قابل‌نجات/ورشکسته & ماه ۶-۱۲ \\
۳ & تزریق سرمایه به بانک‌های قابل‌نجات & سال ۱-۲: ۳۰ میلیارد دلار \\
\rowcolor{gray!10}
۴ & ادغام یا انحلال بانک‌های ورشکسته & سال ۱-۳ \\
۵ & فروش بنگاه‌های غیربانکی & سال ۲-۵ \\
\rowcolor{gray!10}
۶ & پیاده‌سازی استانداردهای بازل III & سال ۳-۷ \\
۷ & ورود بانک‌های خارجی & سال ۳+ \\
\bottomrule
\end{tabular}
\end{table}

\subsection{توسعه بازار سرمایه}

\begin{table}[htbp]
\centering
\caption{اهداف توسعه بازار سرمایه}
\label{tab:capital-market}
\begin{tabular}{>{\columncolor{green!8}}l c c c}
\toprule
\rowcolor{green!25}
\textbf{شاخص} & \textbf{۱۴۰۳} & \textbf{سال ۵} & \textbf{سال ۱۰} \\
\midrule
ارزش بازار سهام (٪ GDP) & ۳۵٪ & ۶۰٪ & ۱۰۰٪ \\
\rowcolor{gray!10}
ارزش اوراق بدهی (٪ GDP) & ۵٪ & ۲۰٪ & ۴۰٪ \\
تعداد شرکت‌های بورسی & ۷۰۰ & ۱,۲۰۰ & ۲,۰۰۰ \\
\rowcolor{gray!10}
سرمایه‌گذاران خارجی (٪) & ۱٪ & ۱۰٪ & ۲۵٪ \\
صندوق‌های سرمایه‌گذاری & ۲۰۰ & ۵۰۰ & ۱,۰۰۰ \\
\bottomrule
\end{tabular}
\end{table}

\subsection{فین‌تک و بانکداری دیجیتال}

\begin{olgoobox}
\textbf{فرصت: جهش به بانکداری دیجیتال}

ایران می‌تواند از «عقب‌ماندگی» به «پیشتازی» برسد:
\begin{itemize}[nosep]
\item \textbf{نسل جوان دیجیتال}: ۷۰٪ جمعیت زیر ۴۰ سال
\item \textbf{نفوذ اینترنت}: ۸۰٪+ جمعیت
\item \textbf{استارتاپ‌های فین‌تک}: زیرساخت موجود
\item \textbf{الگوهای موفق}: کنیا (M-Pesa)، هند (UPI)، چین (Alipay)
\end{itemize}

\textbf{اهداف}:
\begin{itemize}[nosep]
\item ۹۰٪ پرداخت‌ها دیجیتال تا سال ۵
\item ریال دیجیتال بانک مرکزی (CBDC) تا سال ۳
\item بانکداری باز (Open Banking) تا سال ۴
\item هویت دیجیتال یکپارچه تا سال ۲
\end{itemize}
\end{olgoobox}

%═══════════════════════════════════════════════════════════════════════════════
\section{انرژی و پایداری اقتصادی}
\label{sec:energy-economy}
%═══════════════════════════════════════════════════════════════════════════════

\subsection{اصلاح یارانه انرژی}

\begin{table}[htbp]
\centering
\caption{یارانه‌های انرژی و برنامه اصلاح}
\label{tab:energy-subsidies}
\begin{tabular}{>{\columncolor{red!8}}l c c c p{2.5cm}}
\toprule
\rowcolor{red!25}
\textbf{حامل} & \textbf{قیمت فعلی} & \textbf{قیمت جهانی} & \textbf{یارانه سالانه} & \textbf{هدف سال ۵} \\
\midrule
بنزین (لیتر) & ۳,۰۰۰ تومان & ۵۰,۰۰۰ تومان & ۳۰ میلیارد \$ & قیمت منطقه‌ای \\
\rowcolor{gray!10}
گازوئیل (لیتر) & ۱,۵۰۰ تومان & ۴۵,۰۰۰ تومان & ۲۰ میلیارد \$ & قیمت منطقه‌ای \\
گاز طبیعی (م۳) & ۱,۰۰۰ تومان & ۱۵,۰۰۰ تومان & ۲۵ میلیارد \$ & تدریجی \\
\rowcolor{gray!10}
برق (کیلووات) & ۵۰۰ تومان & ۵,۰۰۰ تومان & ۱۵ میلیارد \$ & تدریجی \\
\midrule
\textbf{مجموع یارانه انرژی} & & & \textbf{۹۰ میلیارد \$/سال} & \\
\bottomrule
\end{tabular}
\end{table}

\begin{naghlbox}
«یارانه انرژی در ایران معادل کل بودجه بهداشت و آموزش است. این پول عمدتاً به ثروتمندان می‌رسد که مصرف بیشتری دارند. اصلاح یارانه‌ها هم عدالت است، هم کارایی.»
\sourceline{صندوق بین‌المللی پول، گزارش یارانه انرژی، ۲۰۲۳}
\end{naghlbox}

\textbf{اصول اصلاح یارانه انرژی}:
\begin{enumerate}[nosep]
\item \textbf{تدریجی}: افزایش قیمت در ۵ سال، نه یک‌شبه
\item \textbf{جبرانی}: یارانه نقدی به دهک‌های ۱-۵
\item \textbf{صنعتی}: قیمت‌های متفاوت برای صنایع (دوره انتقال)
\item \textbf{حمل‌ونقل عمومی}: سوبسید برای اتوبوس و مترو
\item \textbf{شفاف}: نشان‌دادن هزینه واقعی به مردم
\end{enumerate}

\subsection{سرمایه‌گذاری در انرژی‌های تجدیدپذیر}

\begin{table}[htbp]
\centering
\caption{برنامه سرمایه‌گذاری در انرژی تجدیدپذیر}
\label{tab:renewable-investment}
\begin{tabular}{>{\columncolor{green!8}}l c c c c}
\toprule
\rowcolor{green!25}
\textbf{منبع} & \textbf{ظرفیت فعلی (GW)} & \textbf{سال ۵} & \textbf{سال ۱۰} & \textbf{سرمایه‌گذاری} \\
\midrule
خورشیدی & ۱ & ۱۵ & ۵۰ & ۵۰ میلیارد \$ \\
\rowcolor{gray!10}
بادی & ۰.۵ & ۵ & ۲۰ & ۲۵ میلیارد \$ \\
برق‌آبی & ۱۲ & ۱۴ & ۱۶ & ۱۰ میلیارد \$ \\
\rowcolor{gray!10}
سایر (زمین‌گرمایی، زیست‌توده) & ۰.۱ & ۱ & ۴ & ۵ میلیارد \$ \\
\midrule
\textbf{مجموع تجدیدپذیر} & \textbf{۱۳.۶} & \textbf{۳۵} & \textbf{۹۰} & \textbf{۹۰ میلیارد \$} \\
\textbf{سهم از تولید برق} & \textbf{۸٪} & \textbf{۲۵٪} & \textbf{۵۰٪} & \\
\bottomrule
\end{tabular}
\end{table}

%═══════════════════════════════════════════════════════════════════════════════
\section{مسکن و زمین}
\label{sec:housing}
%═══════════════════════════════════════════════════════════════════════════════

\subsection{بحران مسکن}

\begin{table}[htbp]
\centering
\caption{شاخص‌های بحران مسکن در ایران}
\label{tab:housing-crisis}
\begin{tabular}{>{\columncolor{orange!8}}l c c p{4cm}}
\toprule
\rowcolor{orange!25}
\textbf{شاخص} & \textbf{ایران} & \textbf{استاندارد جهانی} & \textbf{توضیح} \\
\midrule
نسبت قیمت مسکن به درآمد & ۲۵-۳۰ & ۳-۵ & بحرانی \\
\rowcolor{gray!10}
نرخ مالکیت مسکن & ۶۵٪ & ۶۵٪ & قابل قبول \\
متراژ سرانه مسکن & ۲۵ م۲ & ۴۰ م۲ & پایین \\
\rowcolor{gray!10}
واحدهای خالی & ۲.۵ میلیون & — & سوداگری \\
کمبود سالانه مسکن & ۷۰۰,۰۰۰ واحد & — & تولید ناکافی \\
\bottomrule
\end{tabular}
\end{table}

\subsection{برنامه جامع مسکن}

\begin{table}[htbp]
\centering
\caption{برنامه جامع مسکن ملی}
\label{tab:housing-program}
\begin{tabular}{>{\columncolor{blue!8}}r p{5cm} p{4.5cm}}
\toprule
\rowcolor{blue!25}
\textbf{محور} & \textbf{اقدام} & \textbf{هدف کمّی} \\
\midrule
عرضه زمین & آزادسازی زمین‌های دولتی برای مسکن & ۱۰۰,۰۰۰ هکتار در ۵ سال \\
\rowcolor{gray!10}
مالیات & مالیات ۲٪ بر خانه‌های خالی و دوم به بعد & کاهش سوداگری \\
تولید & پروژه ملی ساخت مسکن ارزان & ۱ میلیون واحد/سال \\
\rowcolor{gray!10}
تسهیلات & وام مسکن بلندمدت با نرخ واقعی & ۲۰ سال، نرخ تورم+۲٪ \\
اجاره & حمایت از مستأجران، یارانه اجاره & ۲ میلیون خانوار \\
\rowcolor{gray!10}
مسکن اجتماعی & ساخت مسکن استیجاری دولتی & ۵۰۰,۰۰۰ واحد در ۱۰ سال \\
\bottomrule
\end{tabular}
\end{table}

%═══════════════════════════════════════════════════════════════════════════════
\section{شاخص‌های کلیدی و اهداف}
\label{sec:economic-kpis}
%═══════════════════════════════════════════════════════════════════════════════

\begin{table}[htbp]
\centering
\caption{شاخص‌های کلیدی اقتصادی و اهداف ۱۵ ساله}
\label{tab:economic-kpis}
\small
\begin{tabular}{>{\columncolor{teal!8}}r l c c c c c}
\toprule
\rowcolor{teal!25}
& \textbf{شاخص} & \textbf{۱۴۰۳} & \textbf{س۲} & \textbf{س۵} & \textbf{س۱۰} & \textbf{س۱۵} \\
\midrule
۱ & رشد GDP (٪) & ۲ & ۵ & ۷ & ۶ & ۵ \\
\rowcolor{gray!10}
۲ & تورم (٪) & ۵۰ & ۱۵ & ۷ & ۴ & ۲ \\
۳ & بیکاری (٪) & ۱۲ & ۱۰ & ۸ & ۶ & ۵ \\
\rowcolor{gray!10}
۴ & GDP سرانه (\$K PPP) & ۱۵ & ۱۷ & ۲۲ & ۳۲ & ۴۵ \\
۵ & صادرات (میلیارد \$) & ۸۷ & ۱۱۰ & ۱۵۸ & ۲۴۵ & ۳۰۰ \\
\rowcolor{gray!10}
۶ & FDI سالانه (میلیارد \$) & ۱ & ۵ & ۲۵ & ۵۰ & ۶۰ \\
۷ & نسبت مالیات/GDP (٪) & ۷ & ۹ & ۱۲ & ۱۶ & ۱۸ \\
\rowcolor{gray!10}
۸ & سهم نفت از بودجه (٪) & ۳۵ & ۳۰ & ۲۲ & ۱۵ & ۱۰ \\
۹ & رتبه فساد (CPI) & ۱۴۹ & ۱۲۰ & ۸۵ & ۶۰ & ۴۵ \\
\rowcolor{gray!10}
۱۰ & رتبه کسب‌وکار & ۱۲۷ & ۹۰ & ۵۰ & ۳۵ & ۲۵ \\
\bottomrule
\end{tabular}
\end{table}

\begin{center}
\begin{tikzpicture}
\begin{axis}[
    width=13cm,
    height=7cm,
    xlabel={سال},
    ylabel={GDP سرانه (هزار دلار PPP)},
    xmin=0, xmax=16,
    ymin=10, ymax=50,
    xtick={0,2,5,10,15},
    xticklabels={۱۴۰۳, سال ۲, سال ۵, سال ۱۰, سال ۱۵},
    legend pos=north west,
    grid=major,
    grid style={dashed, gray!30}
]
% مسیر ایران
\addplot[color=blue, mark=*, thick, line width=2pt] coordinates {
    (0, 15) (2, 17) (5, 22) (10, 32) (15, 45)
};
% مقایسه با ترکیه
\addplot[color=orange, dashed, thick] coordinates {
    (0, 28) (5, 32) (10, 38) (15, 45)
};
% مقایسه با مالزی
\addplot[color=green!70!black, dashed, thick] coordinates {
    (0, 30) (5, 35) (10, 42) (15, 50)
};

\legend{ایران (هدف), ترکیه (پیش‌بینی), مالزی (پیش‌بینی)}

% نقطه همگرایی
\fill[red] (axis cs:15,45) circle (4pt);
\node[font=\scriptsize, above right] at (axis cs:15,45) {همگرایی};
\end{axis}
\end{tikzpicture}
\captionof{figure}{مسیر رشد GDP سرانه ایران در مقایسه با کشورهای مشابه}
\label{fig:gdp-trajectory}
\end{center}

%═══════════════════════════════════════════════════════════════════════════════
\section{ریسک‌ها و سناریوها}
\label{sec:economic-risks}
%═══════════════════════════════════════════════════════════════════════════════

\begin{table}[htbp]
\centering
\caption{ریسک‌های اصلی برنامه اقتصادی و راهکارها}
\label{tab:economic-risks}
\begin{tabular}{>{\columncolor{red!8}}l c p{5cm}}
\toprule
\rowcolor{red!25}
\textbf{ریسک} & \textbf{احتمال} & \textbf{راهکار کاهش} \\
\midrule
شکست مذاکرات رفع تحریم & متوسط & تنوع شرکا، اقتصاد مقاومتی هوشمند \\
\rowcolor{gray!10}
مقاومت گروه‌های ذی‌نفع & بالا & ائتلاف‌سازی، جبران بازندگان \\
بحران مالی جهانی & متوسط & صندوق ذخیره، تنوع صادرات \\
\rowcolor{gray!10}
ناآرامی اجتماعی از اصلاحات & متوسط & تدریج، حمایت از آسیب‌پذیران \\
فرار سرمایه و مغزها & بالا & ثبات، فرصت، کیفیت زندگی \\
\rowcolor{gray!10}
سقوط قیمت نفت & متوسط & تنوع‌بخشی سریع‌تر، صندوق ذخیره \\
\bottomrule
\end{tabular}
\end{table}

\subsection{سه سناریوی اقتصادی}

\begin{table}[htbp]
\centering
\caption{سه سناریوی اقتصادی برای ایران در سال ۱۵}
\label{tab:economic-scenarios}
\begin{tabular}{>{\columncolor{gray!8}}l p{3.5cm} p{3.5cm} p{3.5cm}}
\toprule
\rowcolor{gray!25}
\textbf{شاخص} & \textbf{خوش‌بینانه} & \textbf{محتمل} & \textbf{بدبینانه} \\
\midrule
\rowcolor{green!15}
GDP سرانه & \$۵۵,۰۰۰ & \$۴۵,۰۰۰ & \$۳۰,۰۰۰ \\
\rowcolor{gray!10}
تورم & ۱٪ & ۳٪ & ۱۰٪ \\
\rowcolor{green!15}
بیکاری & ۳٪ & ۵٪ & ۱۰٪ \\
\rowcolor{gray!10}
رتبه فساد & ۳۰ & ۴۵ & ۸۰ \\
\rowcolor{green!15}
FDI سالانه & ۸۰ میلیارد \$ & ۶۰ میلیارد \$ & ۲۰ میلیارد \$ \\
\rowcolor{gray!10}
سهم نفت از صادرات & ۱۰٪ & ۲۰٪ & ۴۰٪ \\
\midrule
\textbf{احتمال} & ۲۰٪ & ۵۵٪ & ۲۵٪ \\
\bottomrule
\end{tabular}
\end{table}

%═══════════════════════════════════════════════════════════════════════════════
\section{جمع‌بندی: اقتصادی که برای همه کار می‌کند}
\label{sec:economic-conclusion}
%═══════════════════════════════════════════════════════════════════════════════

\begin{olgoobox}
\textbf{پیام کلیدی فصل}

بازسازی اقتصادی ایران ممکن است — اگر این اصول رعایت شود:
\begin{itemize}[nosep]
\item \textbf{اولویت رفع تحریم}: بدون اتصال به اقتصاد جهانی، رشد پایدار ممکن نیست
\item \textbf{ثبات اقتصاد کلان}: کنترل تورم و ثبات ارزی، پیش‌شرط هر کاری
\item \textbf{اصلاحات ساختاری}: بدون تغییر نهادها، رشد موقتی خواهد بود
\item \textbf{تنوع‌بخشی}: رهایی از نفرین نفت با توسعه گردشگری، فناوری، صنایع
\item \textbf{مبارزه با فساد}: فساد بزرگ‌ترین مانع توسعه است
\item \textbf{عدالت}: رشد باید شامل همه شود، نه فقط ثروتمندان
\item \textbf{آبادانی ملموس}: مردم باید بهبود را در زندگی روزمره احساس کنند
\end{itemize}

\textbf{هدف}: اقتصادی که برای همه ایرانیان کار می‌کند — پویا، عادلانه، پایدار.
\end{olgoobox}

\begin{naghlbox}
«اقتصاد خوب همیشه به سیاست خوب نیاز دارد. بدون ثبات سیاسی، حاکمیت قانون، و نهادهای کارآمد، هیچ سیاست اقتصادی موفق نخواهد شد. برعکس، اقتصاد بیمار دموکراسی را تضعیف می‌کند. اقتصاد و سیاست دو روی یک سکه‌اند.»
\sourceline{نویسنده}
\end{naghlbox}

%═══════════════════════════════════════════════════════════════════════════════
% منابع فصل
%═══════════════════════════════════════════════════════════════════════════════

\section*{منابع فصل سیزدهم}
\addcontentsline{toc}{section}{منابع فصل سیزدهم}

\begin{itemize}[nosep, font=\small]
\item Acemoglu, D., \& Robinson, J. A. (2012). \textit{Why Nations Fail}. Crown Business.
\item Przeworski, A. et al. (2000). \textit{Democracy and Development}. Cambridge University Press.
\item Rodrik, D. (2007). \textit{One Economics, Many Recipes}. Princeton University Press.
\item Stiglitz, J. E. (2002). \textit{Globalization and Its Discontents}. W.W. Norton.
\item World Bank. (2023). \textit{Doing Business Report}.
\item IMF. (2023). \textit{World Economic Outlook}.
\item IMF. (2023). \textit{Energy Subsidies Report}.
\item Transparency International. (2023). \textit{Corruption Perceptions Index}.
\item UNCTAD. (2023). \textit{World Investment Report}.
\item WTO. (2023). \textit{Trade Statistics}.
\item بانک مرکزی ایران. (۱۴۰۲). \textit{گزارش اقتصادی و ترازنامه}.
\item مرکز پژوهش‌های مجلس. (۱۴۰۲). \textit{گزارش‌های اقتصادی}.
\item صندوق بین‌المللی پول. (۱۴۰۲). \textit{گزارش ماده چهار ایران}.
\item مرکز آمار ایران. (۱۴۰۲). \textit{حساب‌های ملی و شاخص‌های اقتصادی}.
\end{itemize}
	% ch14-environment.tex
% فصل چهاردهم: بحران آب و محیط زیست
% نویسنده: مهدی سالم | ریچموندهیل | ۱۴۰۴

\chapter{بحران آب و محیط زیست: چالش حیاتی}
\label{ch:environment}

\begin{kholasebox}
ایران با بحران زیست‌محیطی چندلایه مواجه است: \textbf{کسری آب سالانه ۱۵-۲۰ میلیارد مترمکعب}، خشک‌شدن دریاچه‌ها و تالاب‌ها (ارومیه، هامون، گاوخونی)، فرونشست زمین در ۳۰۰+ دشت، ریزگردها، آلودگی هوای شهرها، و تخریب جنگل‌ها. بدون حل بحران آب، نه توسعه پایدار ممکن است نه ثبات اجتماعی. این فصل راهبرد جامعی برای مدیریت بحران ارائه می‌دهد: \textbf{مدیریت تقاضا} (کاهش ۴۰٪ مصرف)، \textbf{افزایش عرضه پایدار} (شیرین‌سازی، بازیافت)، \textbf{حکمرانی نوین آب}، و \textbf{انتقال به اقتصاد سبز}. شعار: \textbf{«هر قطره آب، هر نفس پاک — حق نسل‌های آینده»}.
\end{kholasebox}

%═══════════════════════════════════════════════════════════════════════════════
\section{مقدمه: بحرانی که انکارش دیگر ممکن نیست}
%═══════════════════════════════════════════════════════════════════════════════

\begin{naghlbox}
«جنگ‌های آینده خاورمیانه بر سر آب خواهد بود، نه نفت. ایران در خط مقدم این بحران قرار دارد. اگر امروز اقدام نکنیم، فردا بسیار دیر خواهد بود.»
\sourceline{ایزمائیل سراج‌الدین، معاون سابق بانک جهانی، ۱۹۹۵}
\end{naghlbox}

بحران آب و محیط زیست تنها یک مسئله فنی نیست — یک \textbf{بحران امنیت ملی} است. خشکسالی‌های پیاپی، مهاجرت روستاییان، تنش‌های بین‌استانی بر سر آب، و ناآرامی‌های اجتماعی همه ریشه در همین بحران دارند. دموکراسی آینده ایران بدون حل این بحران پایدار نخواهد بود.

\subsection{چرا این موضوع اولویت است؟}

\begin{center}
\begin{tikzpicture}[
    node distance=1.8cm,
    box/.style={
        rectangle,
        rounded corners=8pt,
        draw=#1!70,
        fill=#1!15,
        thick,
        minimum width=3.8cm,
        minimum height=1.5cm,
        align=center,
        font=\small
    }
]
% بحران آب در مرکز
\node[ellipse, draw=red!70, fill=red!20, thick, minimum width=3cm, minimum height=2cm] (water) 
    {\textbf{بحران آب}};

% پیامدها
\node[box=orange, above=1.5cm of water] (food) {\textbf{امنیت غذایی}\\ کاهش تولید کشاورزی};
\node[box=purple, above right=1cm of water] (health) {\textbf{سلامت عمومی}\\ آب آلوده، بیماری};
\node[box=teal, right=2cm of water] (eco) {\textbf{اقتصاد}\\ زیان صنایع، بیکاری};
\node[box=blue, below right=1cm of water] (social) {\textbf{اجتماعی}\\ مهاجرت، تنش};
\node[box=darkgreen, below=1.5cm of water] (env) {\textbf{محیط زیست}\\ نابودی اکوسیستم};
\node[box=darkyellow, below left=1cm of water] (security) {\textbf{امنیت ملی}\\ تنش مرزی، ناآرامی};
\node[box=pink, left=2cm of water] (energy) {\textbf{انرژی}\\ برق‌آبی، خنک‌سازی};
\node[box=cyan, above left=1cm of water] (urban) {\textbf{شهرها}\\ جیره‌بندی، فرونشست};

% اتصالات
\foreach \n in {food, health, eco, social, env, security, energy, urban} {
    \draw[thick, red!50, ->] (water) -- (\n);
}
\end{tikzpicture}
\captionof{figure}{پیامدهای چندبُعدی بحران آب}
\label{fig:water-crisis-impacts}
\end{center}

%═══════════════════════════════════════════════════════════════════════════════
\section{تشخیص: ابعاد بحران زیست‌محیطی}
\label{sec:env-diagnosis}
%═══════════════════════════════════════════════════════════════════════════════

\subsection{بحران آب: اعداد و ارقام}

\begin{table}[htbp]
\centering
\caption{شاخص‌های کلیدی بحران آب در ایران}
\label{tab:water-indicators}
\begin{tabular}{>{\columncolor{blue!8}}r l c c p{3.5cm}}
\toprule
\rowcolor{blue!25}
\textbf{ردیف} & \textbf{شاخص} & \textbf{مقدار} & \textbf{وضعیت} & \textbf{توضیح} \\
\midrule
۱ & منابع آب تجدیدپذیر سالانه & ۱۳۰ میلیارد م۳ & — & کاهش‌یافته از ۱۵۰ \\
\rowcolor{gray!10}
۲ & مصرف سالانه آب & ۹۵-۱۰۰ میلیارد م۳ & بحرانی & برداشت بیش از تجدید \\
۳ & کسری سالانه & ۱۵-۲۰ میلیارد م۳ & بحرانی & از ذخایر زیرزمینی \\
\rowcolor{gray!10}
۴ & سرانه آب تجدیدپذیر & ۱,۵۰۰ م۳/نفر/سال & کم‌آب & خط فقر: ۱,۷۰۰ \\
۵ & افت سالانه سفره‌ها & ۱-۳ متر & بحرانی & غیرقابل بازگشت \\
\rowcolor{gray!10}
۶ & دشت‌های ممنوعه/بحرانی & ۳۵۰ از ۶۰۹ & ۵۷٪ & برداشت ممنوع \\
۷ & راندمان آبیاری & ۳۵-۴۰٪ & پایین & جهانی: ۷۰-۸۰٪ \\
\rowcolor{gray!10}
۸ & سهم کشاورزی از مصرف آب & ۹۲٪ & بیش از حد & جهانی: ۷۰٪ \\
\bottomrule
\end{tabular}
\end{table}

\subsection{توزیع مصرف آب}

\begin{center}
\begin{tikzpicture}
\pie[
    text=legend,
    radius=3,
    color={green!60, blue!60, orange!60, gray!60},
    explode={0.1, 0, 0, 0}
]{
    92/کشاورزی (۹۲٪),
    5/شرب و بهداشت (۵٪),
    2/صنعت (۲٪),
    1/سایر (۱٪)
}
\end{tikzpicture}
\captionof{figure}{توزیع مصرف آب در ایران}
\label{fig:water-consumption}
\end{center}

\subsection{خشک‌شدن منابع آبی}

\begin{table}[htbp]
\centering
\caption{وضعیت دریاچه‌ها و تالاب‌های اصلی ایران}
\label{tab:lakes-status}
\begin{tabular}{>{\columncolor{red!8}}l c c c p{3.5cm}}
\toprule
\rowcolor{red!25}
\textbf{منبع آبی} & \textbf{وسعت اولیه (کم۲)} & \textbf{وسعت فعلی} & \textbf{کاهش} & \textbf{وضعیت} \\
\midrule
دریاچه ارومیه & ۵,۰۰۰ & ۱,۰۰۰ & ۸۰٪ & بحرانی — در آستانه نابودی \\
\rowcolor{gray!10}
هامون & ۴,۰۰۰ & ۰-۵۰۰ & ۹۰٪+ & خشک شده \\
گاوخونی & ۴۷۰ & ۰ & ۱۰۰٪ & خشک شده \\
\rowcolor{gray!10}
بختگان & ۶۵۰ & ۵۰ & ۹۲٪ & تقریباً خشک \\
پریشان & ۴۲ & ۱۰ & ۷۶٪ & بحرانی \\
\rowcolor{gray!10}
تالاب انزلی & ۴۵۰ & ۱۸۰ & ۶۰٪ & در خطر \\
هورالعظیم & ۳,۰۰۰ & ۵۰۰ & ۸۳٪ & بحرانی \\
\bottomrule
\end{tabular}
\end{table}

\begin{enghelabbox}
\textbf{هشدار: فاجعه دریاچه ارومیه}

دریاچه ارومیه زمانی بزرگ‌ترین دریاچه داخلی خاورمیانه بود:
\begin{itemize}[nosep]
\item \textbf{کاهش حجم}: از ۳۰ میلیارد مترمکعب به ۳ میلیارد (۹۰٪ کاهش)
\item \textbf{کاهش سطح}: از ۵,۰۰۰ کیلومتر مربع به کمتر از ۱,۰۰۰
\item \textbf{علل}: سدسازی بی‌رویه، کشاورزی پرآب، تغییر اقلیم
\item \textbf{پیامدها}: طوفان نمک، تهدید سلامت ۱۴ میلیون نفر، نابودی اکوسیستم
\item \textbf{هشدار}: بدون اقدام فوری، طی ۱۰ سال کاملاً خشک می‌شود
\end{itemize}
\textbf{اهمیت}: احیای ارومیه نماد تعهد نظام جدید به محیط زیست است.
\end{enghelabbox}

\subsection{فرونشست زمین}

\begin{table}[htbp]
\centering
\caption{فرونشست زمین در دشت‌های بحرانی}
\label{tab:land-subsidence}
\begin{tabular}{>{\columncolor{orange!8}}l c c p{4.5cm}}
\toprule
\rowcolor{orange!25}
\textbf{دشت/منطقه} & \textbf{نرخ فرونشست (سانتی/سال)} & \textbf{مجموع فرونشست} & \textbf{پیامدها} \\
\midrule
دشت رفسنجان & ۲۵-۳۰ & ۵+ متر & شکاف زمین، تخریب زیرساخت \\
\rowcolor{gray!10}
دشت کاشمر & ۲۰-۲۵ & ۳+ متر & آسیب به ساختمان‌ها \\
جنوب تهران & ۲۵-۳۶ & ۴+ متر & تهدید متروی تهران \\
\rowcolor{gray!10}
دشت مشهد & ۱۵-۲۰ & ۲+ متر & خطر زیرساخت شهری \\
اصفهان & ۱۰-۱۵ & ۱.۵+ متر & آسیب به آثار تاریخی \\
\rowcolor{gray!10}
کرمان & ۲۰-۳۰ & ۳+ متر & تخریب کشاورزی \\
\bottomrule
\end{tabular}
\end{table}

\subsection{آلودگی هوا}

\begin{table}[htbp]
\centering
\caption{وضعیت آلودگی هوای کلان‌شهرهای ایران}
\label{tab:air-pollution}
\begin{tabular}{>{\columncolor{purple!8}}l c c c c}
\toprule
\rowcolor{purple!25}
\textbf{شهر} & \textbf{PM2.5 میانگین} & \textbf{استاندارد WHO} & \textbf{روزهای ناسالم/سال} & \textbf{مرگ زودرس/سال} \\
\midrule
تهران & ۳۵ & ۵ & ۲۵۰+ & ۲۵,۰۰۰+ \\
\rowcolor{gray!10}
اهواز & ۸۰+ & ۵ & ۳۰۰+ & تخمین بالا \\
تبریز & ۲۸ & ۵ & ۱۸۰ & ۵,۰۰۰+ \\
\rowcolor{gray!10}
اصفهان & ۳۲ & ۵ & ۲۲۰ & ۸,۰۰۰+ \\
مشهد & ۲۵ & ۵ & ۱۵۰ & ۶,۰۰۰+ \\
\rowcolor{gray!10}
شیراز & ۲۲ & ۵ & ۱۲۰ & ۳,۰۰۰+ \\
\midrule
\multicolumn{4}{r}{\textbf{مجموع مرگ‌های ناشی از آلودگی هوا در ایران}} & \textbf{۶۰,۰۰۰+/سال} \\
\bottomrule
\end{tabular}
\end{table}

\subsection{سایر بحران‌های زیست‌محیطی}

\begin{table}[htbp]
\centering
\caption{سایر چالش‌های زیست‌محیطی ایران}
\label{tab:other-env-challenges}
\begin{tabular}{>{\columncolor{teal!8}}l c p{6cm}}
\toprule
\rowcolor{teal!25}
\textbf{چالش} & \textbf{شدت} & \textbf{توضیح} \\
\midrule
ریزگردها & بحرانی & خوزستان، سیستان — منشأ داخلی و خارجی \\
\rowcolor{gray!10}
تخریب جنگل & بالا & کاهش ۳۰٪ جنگل‌های هیرکانی در ۵۰ سال \\
فرسایش خاک & بالا & ۲ میلیارد تن/سال فرسایش \\
\rowcolor{gray!10}
بیابان‌زایی & بالا & ۱۰٪ کشور در معرض بیابان‌زایی شدید \\
آلودگی آب & متوسط-بالا & نیترات، فلزات سنگین در سفره‌ها \\
\rowcolor{gray!10}
زباله و پسماند & بالا & بازیافت زیر ۱۰٪، دفن غیربهداشتی \\
انقراض گونه‌ها & بالا & یوزپلنگ، میش‌مرغ، و دهها گونه در خطر \\
\bottomrule
\end{tabular}
\end{table}

%═══════════════════════════════════════════════════════════════════════════════
\section{علل ریشه‌ای بحران}
\label{sec:root-causes}
%═══════════════════════════════════════════════════════════════════════════════

\begin{center}
\begin{tikzpicture}[
    cause/.style={
        rectangle,
        rounded corners=5pt,
        draw=red!70,
        fill=red!10,
        thick,
        minimum width=4.5cm,
        minimum height=1.3cm,
        align=center,
        font=\small
    }
]
% عنوان
\node[font=\large\bfseries] at (0,5.5) {ریشه‌های بحران آب و محیط زیست};

% علل ساختاری
\node[cause] at (-4.5,4) (c1) {\textbf{۱. حکمرانی ضعیف}\\ تمرکز، عدم هماهنگی};
\node[cause] at (0,4) (c2) {\textbf{۲. سیاست‌های نادرست}\\ خودکفایی غلط، یارانه آب};
\node[cause] at (4.5,4) (c3) {\textbf{۳. اقتصاد رانتی}\\ بی‌توجهی به پایداری};

% علل مستقیم
\node[cause] at (-4.5,2) (c4) {\textbf{۴. سدسازی بی‌رویه}\\ ۶۰۰+ سد، تخریب رودخانه‌ها};
\node[cause] at (0,2) (c5) {\textbf{۵. کشاورزی پرآب}\\ الگوی کشت نامناسب};
\node[cause] at (4.5,2) (c6) {\textbf{۶. برداشت بی‌رویه}\\ چاه‌های غیرمجاز};

% علل تشدیدکننده
\node[cause, fill=orange!10, draw=orange!70] at (-2.25,0) (c7) {\textbf{۷. تغییر اقلیم}\\ کاهش بارش، افزایش تبخیر};
\node[cause, fill=orange!10, draw=orange!70] at (2.25,0) (c8) {\textbf{۸. رشد جمعیت}\\ ۳ برابر در ۵۰ سال};

% بحران
\node[ellipse, draw=red!70, fill=red!30, thick, minimum width=4cm, minimum height=1.5cm] at (0,-2) (crisis)
    {\textbf{بحران آب و محیط زیست}};

% اتصالات
\foreach \c in {c1,c2,c3,c4,c5,c6,c7,c8} {
    \draw[thick, red!40, ->] (\c) -- (crisis);
}
\end{tikzpicture}
\end{center}

\begin{naghlbox}
«بحران آب ایران بیش از آنکه بحران طبیعی باشد، بحران مدیریت است. با همین میزان بارندگی، کشورهایی مثل اسرائیل و استرالیا وضعیت بهتری دارند. مشکل در سیاست‌ها و نهادهاست.»
\sourceline{کاوه مدنی، محقق محیط زیست، ۲۰۱۴}
\end{naghlbox}

%═══════════════════════════════════════════════════════════════════════════════
\section{استراتژی جامع مدیریت آب}
\label{sec:water-strategy}
%═══════════════════════════════════════════════════════════════════════════════

\begin{center}
\begin{tikzpicture}[
    pillar/.style={
        rectangle,
        rounded corners=8pt,
        draw=#1!70,
        fill=#1!15,
        thick,
        minimum width=3cm,
        minimum height=4cm,
        align=center
    }
]
% چهار ستون
\node[pillar=blue] (p1) at (0,0) {
    \textbf{مدیریت تقاضا}\\[0.3cm]
    \scriptsize کاهش ۴۰٪\\
    \scriptsize مصرف آب\\[0.2cm]
    \tiny کشاورزی کم‌آب\\
    \tiny قیمت‌گذاری\\
    \tiny فرهنگ‌سازی
};

\node[pillar=green] (p2) at (4,0) {
    \textbf{افزایش عرضه}\\[0.3cm]
    \scriptsize پایدار\\[0.2cm]
    \tiny شیرین‌سازی\\
    \tiny بازیافت\\
    \tiny جمع‌آوری باران
};

\node[pillar=orange] (p3) at (8,0) {
    \textbf{حکمرانی نوین}\\[0.3cm]
    \scriptsize نهادسازی\\[0.2cm]
    \tiny مدیریت حوضه‌ای\\
    \tiny شفافیت داده\\
    \tiny مشارکت مردم
};

\node[pillar=purple] (p4) at (12,0) {
    \textbf{احیای اکوسیستم}\\[0.3cm]
    \scriptsize بازسازی\\[0.2cm]
    \tiny دریاچه‌ها\\
    \tiny تالاب‌ها\\
    \tiny رودخانه‌ها
};

% سقف
\draw[thick, fill=teal!20] (-1.5,2.5) -- (13.5,2.5) -- (13.5,3.5) -- (-1.5,3.5) -- cycle;
\node[font=\large\bfseries] at (6,3) {استراتژی جامع آب: تعادل پایدار تا سال ۱۵};

% پایه
\draw[thick, fill=gray!20] (-1.5,-2.5) -- (13.5,-2.5) -- (13.5,-1.8) -- (-1.5,-1.8) -- cycle;
\node[font=\small\bfseries] at (6,-2.15) {پایه: داده محور، علم‌محور، مشارکتی، بین‌نسلی};
\end{tikzpicture}
\captionof{figure}{چهار ستون استراتژی جامع مدیریت آب}
\label{fig:water-strategy}
\end{center}

%═══════════════════════════════════════════════════════════════════════════════
\section{ستون اول: مدیریت تقاضا}
\label{sec:demand-management}
%═══════════════════════════════════════════════════════════════════════════════

\subsection{اصلاح بخش کشاورزی}

کشاورزی با ۹۲٪ مصرف آب، کلید حل بحران است.

\begin{table}[htbp]
\centering
\caption{برنامه کاهش مصرف آب کشاورزی}
\label{tab:agriculture-water}
\begin{tabular}{>{\columncolor{green!8}}r p{4cm} c p{4cm}}
\toprule
\rowcolor{green!25}
\textbf{راهکار} & \textbf{توضیح} & \textbf{پتانسیل صرفه‌جویی} & \textbf{اقدامات} \\
\midrule
آبیاری نوین & قطره‌ای، بارانی & ۲۰ میلیارد م۳ & پوشش ۱۰۰٪ در ۱۰ سال \\
\rowcolor{gray!10}
تغییر الگوی کشت & حذف محصولات پرآب & ۱۵ میلیارد م۳ & ممنوعیت برنج در کویر \\
کاهش سطح زیرکشت & تناسب با آب موجود & ۱۰ میلیارد م۳ & خروج زمین‌های نامناسب \\
\rowcolor{gray!10}
کشت گلخانه‌ای & مصرف ۱۰٪ آب مزرعه & ۵ میلیارد م۳ & ۱۰۰,۰۰۰ هکتار گلخانه \\
\midrule
\multicolumn{2}{r}{\textbf{مجموع صرفه‌جویی کشاورزی}} & \textbf{۵۰ میلیارد م۳} & \\
\bottomrule
\end{tabular}
\end{table}

\begin{olgoobox}
\textbf{الگوی موفق: اسرائیل — انقلاب آب}

اسرائیل با بارش کمتر از ایران، به خودکفایی آب رسیده است:
\begin{itemize}[nosep]
\item \textbf{آبیاری قطره‌ای}: ۹۰٪ کشاورزی — اختراع اسرائیلی
\item \textbf{بازیافت فاضلاب}: ۸۵٪ فاضلاب بازیافت و استفاده مجدد
\item \textbf{شیرین‌سازی}: ۷۰٪ آب شرب از شیرین‌سازی
\item \textbf{قیمت‌گذاری}: قیمت واقعی آب
\item \textbf{نتیجه}: صادرکننده محصولات کشاورزی با آب کمتر
\item \textbf{درس}: مدیریت، نه فقط منابع، تعیین‌کننده است
\end{itemize}
\end{olgoobox}

\subsection{قیمت‌گذاری واقعی آب}

\begin{table}[htbp]
\centering
\caption{اصلاح قیمت‌گذاری آب}
\label{tab:water-pricing}
\begin{tabular}{>{\columncolor{blue!8}}l c c c}
\toprule
\rowcolor{blue!25}
\textbf{بخش} & \textbf{قیمت فعلی} & \textbf{قیمت واقعی} & \textbf{هدف سال ۵} \\
\midrule
کشاورزی (م۳) & ۵۰۰ ریال & ۱۵,۰۰۰ ریال & ۸,۰۰۰ ریال (تدریجی) \\
\rowcolor{gray!10}
صنعت (م۳) & ۵,۰۰۰ ریال & ۲۰,۰۰۰ ریال & ۱۵,۰۰۰ ریال \\
خانگی-پایه (م۳) & ۳,۰۰۰ ریال & ۱۰,۰۰۰ ریال & ۵,۰۰۰ ریال \\
\rowcolor{gray!10}
خانگی-پرمصرف (م۳) & ۱۰,۰۰۰ ریال & ۳۰,۰۰۰ ریال & ۲۵,۰۰۰ ریال \\
\bottomrule
\end{tabular}
\end{table}

\textbf{اصول قیمت‌گذاری}:
\begin{itemize}[nosep]
\item پلکانی: مصرف بیشتر = قیمت بالاتر
\item حمایت از مصرف پایه: سهمیه ارزان برای نیاز اساسی
\item تدریجی: افزایش در ۵ سال
\item بازتوزیع: درآمد صرف کارآمدسازی و حمایت از فقرا
\end{itemize}

\subsection{کاهش مصرف شهری و صنعتی}

\begin{table}[htbp]
\centering
\caption{برنامه کاهش مصرف آب شهری و صنعتی}
\label{tab:urban-water-saving}
\begin{tabular}{>{\columncolor{cyan!8}}l p{5cm} c}
\toprule
\rowcolor{cyan!25}
\textbf{راهکار} & \textbf{اقدام} & \textbf{صرفه‌جویی} \\
\midrule
کاهش تلفات شبکه & نوسازی لوله‌ها، کنتور هوشمند & ۱.۵ میلیارد م۳ \\
\rowcolor{gray!10}
لوازم کم‌مصرف & شیرآلات، توالت، ماشین لباسشویی & ۰.۵ میلیارد م۳ \\
بازیافت در صنایع & چرخه بسته آب در کارخانجات & ۰.۵ میلیارد م۳ \\
\rowcolor{gray!10}
فضای سبز کم‌آب & گیاهان بومی، آبیاری هوشمند & ۰.۳ میلیارد م۳ \\
آموزش و فرهنگ‌سازی & کمپین ملی صرفه‌جویی & ۰.۲ میلیارد م۳ \\
\midrule
\multicolumn{2}{r}{\textbf{مجموع صرفه‌جویی شهری و صنعتی}} & \textbf{۳ میلیارد م۳} \\
\bottomrule
\end{tabular}
\end{table}

%═══════════════════════════════════════════════════════════════════════════════
\section{ستون دوم: افزایش عرضه پایدار}
\label{sec:supply-increase}
%═══════════════════════════════════════════════════════════════════════════════

\subsection{منابع جدید آب}

\begin{table}[htbp]
\centering
\caption{برنامه افزایش عرضه آب پایدار}
\label{tab:water-supply}
\begin{tabular}{>{\columncolor{green!8}}l c c c p{3cm}}
\toprule
\rowcolor{green!25}
\textbf{منبع} & \textbf{فعلی (میلیارد م۳)} & \textbf{هدف س۱۰} & \textbf{سرمایه‌گذاری} & \textbf{توضیح} \\
\midrule
شیرین‌سازی دریا & ۰.۵ & ۵ & ۲۵ میلیارد \$ & خلیج فارس و عمان \\
\rowcolor{gray!10}
بازیافت فاضلاب & ۱ & ۸ & ۱۵ میلیارد \$ & استفاده در کشاورزی \\
جمع‌آوری باران & ۰.۱ & ۲ & ۵ میلیارد \$ & سازه‌های ذخیره \\
\rowcolor{gray!10}
کاهش تبخیر & — & ۲ & ۳ میلیارد \$ & پوشش کانال‌ها \\
تغذیه مصنوعی & ۱ & ۳ & ۵ میلیارد \$ & احیای سفره‌ها \\
\midrule
\multicolumn{2}{r}{\textbf{مجموع افزایش عرضه}} & \textbf{۲۰ میلیارد م۳} & \textbf{۵۳ میلیارد \$} & \\
\bottomrule
\end{tabular}
\end{table}

\subsection{شیرین‌سازی آب دریا}

\begin{center}
\begin{tikzpicture}
% نقشه ساده جنوب ایران
\draw[thick, fill=blue!10] (0,0) -- (10,0) -- (10,3) -- (0,3) -- cycle;
\node[font=\small] at (5,2.5) {جنوب ایران};
\draw[thick, fill=blue!40] (0,0) -- (10,0) -- (10,-1) -- (0,-1) -- cycle;
\node[font=\small, white] at (5,-0.5) {خلیج فارس و دریای عمان};

% واحدهای شیرین‌سازی
\foreach \x/\name/\cap in {1/بوشهر/۵۰۰, 3/عسلویه/۸۰۰, 5/بندرعباس/۱۰۰۰, 7/جاسک/۴۰۰, 9/چابهار/۳۰۰} {
    \node[circle, fill=green!60, minimum size=0.5cm] at (\x, 0.5) {};
    \node[font=\tiny, above] at (\x, 0.8) {\name};
    \node[font=\tiny, below] at (\x, 0.2) {\cap MCM};
}

% خطوط انتقال
\draw[thick, red, ->] (1, 1) -- (1, 2.5);
\draw[thick, red, ->] (3, 1) -- (4, 2.5);
\draw[thick, red, ->] (5, 1) -- (5, 2.5);
\draw[thick, red, ->] (7, 1) -- (7, 2.5);
\draw[thick, red, ->] (9, 1) -- (9, 2.5);

% عنوان
\node[font=\bfseries] at (5, 3.5) {شبکه شیرین‌سازی ساحلی — هدف: ۵ میلیارد م۳/سال};

% راهنما
\node[font=\scriptsize] at (5, -1.8) {ظرفیت به میلیون مترمکعب در سال (MCM)};
\end{tikzpicture}
\captionof{figure}{طرح شبکه شیرین‌سازی آب دریا}
\label{fig:desalination-network}
\end{center}

\begin{table}[htbp]
\centering
\caption{پروژه‌های کلان شیرین‌سازی}
\label{tab:desalination-projects}
\begin{tabular}{>{\columncolor{blue!8}}l c c c c}
\toprule
\rowcolor{blue!25}
\textbf{پروژه} & \textbf{ظرفیت (MCM/سال)} & \textbf{سرمایه (میلیارد \$)} & \textbf{منطقه خدمات} & \textbf{زمان} \\
\midrule
بندرعباس-کرمان & ۱,۰۰۰ & ۸ & کرمان، یزد & سال ۲-۶ \\
\rowcolor{gray!10}
عسلویه-فارس & ۸۰۰ & ۶ & شیراز، فارس & سال ۲-۵ \\
بوشهر-خوزستان & ۵۰۰ & ۴ & بوشهر، خوزستان & سال ۳-۶ \\
\rowcolor{gray!10}
چابهار-بلوچستان & ۳۰۰ & ۲.۵ & سیستان و بلوچستان & سال ۲-۵ \\
جاسک-هرمزگان & ۴۰۰ & ۳ & هرمزگان & سال ۳-۶ \\
\midrule
\textbf{مجموع فاز اول} & \textbf{۳,۰۰۰} & \textbf{۲۳.۵} & & \\
\bottomrule
\end{tabular}
\end{table}

\subsection{بازیافت فاضلاب}

\begin{table}[htbp]
\centering
\caption{برنامه بازیافت فاضلاب}
\label{tab:wastewater-recycling}
\begin{tabular}{>{\columncolor{purple!8}}l c c c}
\toprule
\rowcolor{purple!25}
\textbf{شاخص} & \textbf{فعلی} & \textbf{هدف سال ۵} & \textbf{هدف سال ۱۰} \\
\midrule
پوشش شبکه فاضلاب شهری & ۵۰٪ & ۷۵٪ & ۹۵٪ \\
\rowcolor{gray!10}
تصفیه فاضلاب & ۴۰٪ & ۷۰٪ & ۹۰٪ \\
بازیافت و استفاده مجدد & ۱۵٪ & ۵۰٪ & ۸۰٪ \\
\rowcolor{gray!10}
حجم بازیافتی (میلیارد م۳) & ۱ & ۴ & ۸ \\
\bottomrule
\end{tabular}
\end{table}

%═══════════════════════════════════════════════════════════════════════════════
\section{ستون سوم: حکمرانی نوین آب}
\label{sec:water-governance}
%═══════════════════════════════════════════════════════════════════════════════

\subsection{اصلاح ساختار نهادی}

\begin{enghelabbox}
\textbf{مشکل فعلی: حکمرانی چندپاره و ناکارآمد}

مدیریت آب در ایران بین نهادهای متعدد تقسیم شده:
\begin{itemize}[nosep]
\item وزارت نیرو: آب شرب و صنعت
\item وزارت جهاد کشاورزی: آب کشاورزی
\item سازمان محیط زیست: حفاظت منابع (بدون قدرت)
\item استانداری‌ها: مدیریت محلی
\item شرکت‌های آب منطقه‌ای: اجرایی
\end{itemize}
\textbf{نتیجه}: تضاد منافع، عدم هماهنگی، غلبه مصرف بر حفاظت
\end{enghelabbox}

\begin{table}[htbp]
\centering
\caption{ساختار پیشنهادی حکمرانی آب}
\label{tab:water-governance}
\begin{tabular}{>{\columncolor{teal!8}}l p{4cm} p{5.5cm}}
\toprule
\rowcolor{teal!25}
\textbf{نهاد} & \textbf{وظیفه} & \textbf{ویژگی‌ها} \\
\midrule
شورای عالی آب & سیاستگذاری کلان & ریاست رئیس‌جمهور، همه ذی‌نفعان \\
\rowcolor{gray!10}
سازمان ملی آب & تنظیم‌گری و نظارت & مستقل، غیرسیاسی، فنی \\
شرکت‌های حوضه آبریز & مدیریت اجرایی حوضه & ۶ حوضه اصلی، خودگردان \\
\rowcolor{gray!10}
شوراهای محلی آب & مدیریت مشارکتی & کشاورزان، شهرها، صنایع \\
دادگاه آب & حل اختلافات & قضات متخصص \\
\bottomrule
\end{tabular}
\end{table}

\subsection{مدیریت حوضه‌ای}

\begin{center}
\begin{tikzpicture}[scale=0.7]
% نقشه ساده ایران با حوضه‌ها
\draw[thick] plot[smooth cycle] coordinates {
    (0,2) (1,4) (3,5) (5,5.5) (7,5) (9,4.5) (10,3) (9.5,1) (8,0) (6,-0.5) (4,0) (2,0.5) (0.5,1)
};

% حوضه‌ها (ساده‌شده)
\draw[thick, blue!50, fill=blue!10] plot[smooth cycle] coordinates {(1,2) (2,3.5) (4,3) (3,1.5)};
\node[font=\tiny] at (2.5,2.5) {دریای خزر};

\draw[thick, green!50, fill=green!10] plot[smooth cycle] coordinates {(4,2) (5,4) (7,3.5) (6,1.5)};
\node[font=\tiny] at (5.5,2.8) {فلات مرکزی};

\draw[thick, orange!50, fill=orange!10] plot[smooth cycle] coordinates {(6.5,2) (8,4) (9.5,2.5) (8,0.5)};
\node[font=\tiny] at (8,2) {شرق};

\draw[thick, purple!50, fill=purple!10] plot[smooth cycle] coordinates {(3,0.5) (4.5,2) (6,1) (5,-0.3)};
\node[font=\tiny] at (4.5,0.8) {خلیج فارس};

\draw[thick, red!50, fill=red!10] plot[smooth cycle] coordinates {(0.5,1) (2,2) (3,1) (2,0.3)};
\node[font=\tiny] at (1.8,1.2) {ارومیه};

\draw[thick, cyan!50, fill=cyan!10] plot[smooth cycle] coordinates {(7.5,3.5) (8.5,4.5) (9.5,3.5) (8.5,3)};
\node[font=\tiny] at (8.5,3.8) {سرخس};

% عنوان
\node[font=\bfseries] at (5,6.5) {شش حوضه آبریز اصلی ایران};
\end{tikzpicture}
\captionof{figure}{حوضه‌های آبریز اصلی ایران}
\label{fig:watersheds}
\end{center}

\begin{table}[htbp]
\centering
\caption{مشخصات شش حوضه آبریز اصلی}
\label{tab:watershed-details}
\begin{tabular}{>{\columncolor{blue!8}}l c c c c}
\toprule
\rowcolor{blue!25}
\textbf{حوضه} & \textbf{مساحت (کم۲)} & \textbf{جمعیت (م)} & \textbf{کسری آب} & \textbf{اولویت} \\
\midrule
فلات مرکزی & ۸۳۰,۰۰۰ & ۳۵ & بسیار بالا & ۱ \\
\rowcolor{gray!10}
خلیج فارس و عمان & ۲۳۰,۰۰۰ & ۲۰ & متوسط & ۳ \\
دریاچه ارومیه & ۵۲,۰۰۰ & ۶ & بحرانی & ۱ \\
\rowcolor{gray!10}
دریای خزر & ۱۷۵,۰۰۰ & ۱۵ & کم & ۴ \\
شرق (هامون و...) & ۲۷۰,۰۰۰ & ۶ & بالا & ۲ \\
\rowcolor{gray!10}
مرزی (ارس، هیرمند) & ۸۰,۰۰۰ & ۵ & متوسط & ۳ \\
\bottomrule
\end{tabular}
\end{table}

\subsection{شفافیت و داده‌محوری}

\begin{table}[htbp]
\centering
\caption{برنامه شفافیت داده‌های آب}
\label{tab:water-transparency}
\begin{tabular}{>{\columncolor{green!8}}r p{5cm} p{4.5cm}}
\toprule
\rowcolor{green!25}
\textbf{اقدام} & \textbf{توضیح} & \textbf{زمان‌بندی} \\
\midrule
پایگاه ملی آب & داده‌های لحظه‌ای سفره‌ها، سدها، مصرف & سال ۱-۲ \\
\rowcolor{gray!10}
کنتور هوشمند & همه چاه‌ها و مصرف‌کنندگان بزرگ & سال ۲-۵ \\
گزارش سالانه آب & وضعیت هر حوضه، عمومی & سال ۱+ \\
\rowcolor{gray!10}
نقشه آنلاین خشکسالی & پایش لحظه‌ای با ماهواره & سال ۱-۲ \\
حسابداری آب & ترازنامه آب هر استان & سال ۲-۳ \\
\bottomrule
\end{tabular}
\end{table}

%═══════════════════════════════════════════════════════════════════════════════
\section{ستون چهارم: احیای اکوسیستم‌ها}
\label{sec:ecosystem-restoration}
%═══════════════════════════════════════════════════════════════════════════════

\subsection{برنامه احیای دریاچه ارومیه}

\begin{table}[htbp]
\centering
\caption{برنامه ۱۰ ساله احیای دریاچه ارومیه}
\label{tab:urmia-restoration}
\begin{tabular}{>{\columncolor{cyan!8}}r p{4.5cm} c p{3.5cm}}
\toprule
\rowcolor{cyan!25}
\textbf{مرحله} & \textbf{اقدام} & \textbf{اثر (میلیارد م۳)} & \textbf{زمان} \\
\midrule
۱ & کاهش ۴۰٪ مصرف کشاورزی حوضه & ۲.۰ & سال ۱-۵ \\
\rowcolor{gray!10}
۲ & آزادسازی حقابه زیست‌محیطی از سدها & ۱.۵ & سال ۱-۳ \\
۳ & توقف پروژه‌های انتقال آب از حوضه & ۰.۵ & فوری \\
\rowcolor{gray!10}
۴ & بازیافت فاضلاب تبریز و ارومیه & ۰.۳ & سال ۲-۵ \\
۵ & جمع‌آوری رواناب‌ها & ۰.۲ & سال ۳-۷ \\
\midrule
\multicolumn{2}{r}{\textbf{مجموع آب اضافی برای دریاچه}} & \textbf{۴.۵ میلیارد م۳} & \\
\textbf{هدف} & \multicolumn{3}{l}{رساندن حجم دریاچه به ۱۵ میلیارد م۳ (نصف ظرفیت اصلی)} \\
\bottomrule
\end{tabular}
\end{table}

\begin{olgoobox}
\textbf{الگوی موفق: احیای دریای آرال}

قزاقستان موفق شد بخش شمالی دریای آرال را احیا کند:
\begin{itemize}[nosep]
\item ساخت سد کوکارال برای جداسازی و حفظ بخش شمالی
\item کاهش مصرف آب کشاورزی در حوضه
\item نتیجه: افزایش سطح آب از ۳۸ به ۴۲ متر
\item بازگشت ماهیگیری و معیشت محلی
\item \textbf{درس}: احیا ممکن است — اگر اراده باشد
\end{itemize}
\end{olgoobox}

\subsection{احیای سایر اکوسیستم‌ها}

\begin{table}[htbp]
\centering
\caption{برنامه احیای تالاب‌ها و رودخانه‌ها}
\label{tab:ecosystem-restoration-env}
\begin{tabular}{>{\columncolor{green!8}}l p{4cm} c c}
\toprule
\rowcolor{green!25}
\textbf{اکوسیستم} & \textbf{اقدامات کلیدی} & \textbf{سرمایه‌گذاری} & \textbf{زمان} \\
\midrule
هامون & مذاکره با افغانستان، کاهش مصرف داخلی & ۲ میلیارد \$ & ۱-۱۰ سال \\
\rowcolor{gray!10}
زاینده‌رود & تعادل‌بخشی، حقابه گاوخونی & ۱.۵ میلیارد \$ & ۱-۷ سال \\
هورالعظیم & رهاسازی آب، مذاکره با عراق & ۱ میلیارد \$ & ۱-۵ سال \\
\rowcolor{gray!10}
تالاب انزلی & تصفیه ورودی‌ها، لاروبی & ۰.۵ میلیارد \$ & ۲-۵ سال \\
جنگل‌های هیرکانی & توقف تخریب، احیای ۵۰۰,۰۰۰ هکتار & ۳ میلیارد \$ & ۱-۱۵ سال \\
\rowcolor{gray!10}
کارون & پاکسازی، تنظیم رهاسازی سدها & ۱ میلیارد \$ & ۲-۷ سال \\
\bottomrule
\end{tabular}
\end{table}

%═══════════════════════════════════════════════════════════════════════════════
\section{تغییرات اقلیمی و سازگاری}
\label{sec:climate-change}
%═══════════════════════════════════════════════════════════════════════════════

\begin{naghlbox}
«ایران یکی از آسیب‌پذیرترین کشورها در برابر تغییرات اقلیمی است. پیش‌بینی‌ها نشان می‌دهد تا ۲۰۵۰ دما ۲-۴ درجه افزایش و بارش ۲۰-۳۰٪ کاهش خواهد یافت. آنچه امروز بحران است، فردا فاجعه خواهد بود — اگر اقدام نکنیم.»
\sourceline{گزارش IPCC، ۲۰۲۲}
\end{naghlbox}

\subsection{پیش‌بینی‌های اقلیمی برای ایران}

\begin{table}[htbp]
\centering
\caption{پیش‌بینی تغییرات اقلیمی در ایران}
\label{tab:climate-projections}
\begin{tabular}{>{\columncolor{red!8}}l c c c}
\toprule
\rowcolor{red!25}
\textbf{شاخص} & \textbf{۲۰۳۰} & \textbf{۲۰۵۰} & \textbf{۲۰۷۰} \\
\midrule
افزایش دما (درجه سانتی‌گراد) & ۱-۱.۵ & ۲-۳ & ۳-۴.۵ \\
\rowcolor{gray!10}
کاهش بارش (درصد) & ۵-۱۰ & ۱۵-۲۵ & ۲۰-۳۵ \\
کاهش رواناب (درصد) & ۱۰-۲۰ & ۲۵-۴۰ & ۳۵-۵۵ \\
\rowcolor{gray!10}
افزایش تبخیر (درصد) & ۵-۱۰ & ۱۵-۲۵ & ۲۵-۴۰ \\
روزهای گرم فرین (بیش از ۴۰°) & ۳۰+ روز/سال & ۵۰+ روز & ۷۰+ روز \\
\bottomrule
\end{tabular}
\end{table}

\subsection{استراتژی سازگاری}

\begin{table}[htbp]
\centering
\caption{اقدامات سازگاری با تغییرات اقلیمی}
\label{tab:climate-adaptation}
\begin{tabular}{>{\columncolor{orange!8}}l p{5.5cm} p{3.5cm}}
\toprule
\rowcolor{orange!25}
\textbf{بخش} & \textbf{اقدامات سازگاری} & \textbf{سرمایه‌گذاری} \\
\midrule
کشاورزی & ارقام مقاوم به خشکی، کشت زودرس، بیمه اقلیمی & ۱۰ میلیارد \$ \\
\rowcolor{gray!10}
شهری & خنک‌سازی سبز، ساختمان کم‌مصرف، سایه‌بان & ۱۵ میلیارد \$ \\
آب & ذخیره‌سازی بیشتر، شیرین‌سازی، بازیافت & ۳۰ میلیارد \$ \\
\rowcolor{gray!10}
سلامت & سیستم هشدار گرما، مراقبت از سالمندان & ۵ میلیارد \$ \\
زیرساخت & مقاوم‌سازی در برابر سیل و خشکسالی & ۲۰ میلیارد \$ \\
\rowcolor{gray!10}
اکوسیستم & کریدورهای حیات‌وحش، مناطق حفاظت‌شده & ۵ میلیارد \$ \\
\bottomrule
\end{tabular}
\end{table}

%═══════════════════════════════════════════════════════════════════════════════
\section{کاهش انتشار و انتقال انرژی}
\label{sec:emission-reduction}
%═══════════════════════════════════════════════════════════════════════════════

\subsection{وضعیت فعلی انتشار}

\begin{table}[htbp]
\centering
\caption{پروفایل انتشار گازهای گلخانه‌ای ایران}
\label{tab:emission-profile}
\begin{tabular}{>{\columncolor{gray!8}}l c c p{4cm}}
\toprule
\rowcolor{gray!25}
\textbf{شاخص} & \textbf{مقدار} & \textbf{رتبه جهانی} & \textbf{توضیح} \\
\midrule
کل انتشار CO2 & ۷۵۰ میلیون تن/سال & ۸ & یکی از بزرگ‌ترین منتشرکنندگان \\
\rowcolor{gray!10}
انتشار سرانه & ۸.۵ تن/نفر/سال & بالا & دو برابر میانگین جهانی \\
سهم از انتشار جهانی & ۱.۸٪ & — & با ۱.۱٪ جمعیت جهان \\
\rowcolor{gray!10}
رشد انتشار & ۳٪/سال & — & نیاز به کاهش \\
سهم انرژی از انتشار & ۷۵٪ & — & نفت و گاز \\
\bottomrule
\end{tabular}
\end{table}

\subsection{منابع انتشار}

\begin{center}
\begin{tikzpicture}
\pie[
    text=legend,
    radius=3,
    color={red!60, orange!60, blue!60, green!60, purple!60, gray!60}
]{
    35/نیروگاه‌ها (۳۵٪),
    25/صنایع (۲۵٪),
    20/حمل‌ونقل (۲۰٪),
    10/ساختمان (۱۰٪),
    5/کشاورزی (۵٪),
    5/سایر (۵٪)
}
\end{tikzpicture}
\captionof{figure}{سهم بخش‌های مختلف از انتشار گازهای گلخانه‌ای}
\label{fig:emission-sources}
\end{center}

\subsection{اهداف کاهش انتشار}

\begin{table}[htbp]
\centering
\caption{اهداف کاهش انتشار گازهای گلخانه‌ای}
\label{tab:emission-targets}
\begin{tabular}{>{\columncolor{green!8}}l c c c c}
\toprule
\rowcolor{green!25}
\textbf{شاخص} & \textbf{۱۴۰۳} & \textbf{سال ۵} & \textbf{سال ۱۵} & \textbf{سال ۲۵} \\
\midrule
انتشار کل (میلیون تن) & ۷۵۰ & ۷۰۰ & ۵۵۰ & ۳۵۰ \\
\rowcolor{gray!10}
انتشار سرانه (تن) & ۸.۵ & ۷.۵ & ۵.۵ & ۳.۵ \\
سهم تجدیدپذیر در برق & ۸٪ & ۲۰٪ & ۵۰٪ & ۸۰٪ \\
\rowcolor{gray!10}
خودروهای برقی (سهم فروش) & ۱٪ & ۱۵٪ & ۶۰٪ & ۹۵٪ \\
کارایی انرژی (بهبود) & مبدأ & +۱۵٪ & +۴۰٪ & +۶۰٪ \\
\bottomrule
\end{tabular}
\end{table}

\begin{center}
\begin{tikzpicture}
\begin{axis}[
    width=13cm,
    height=7cm,
    xlabel={سال},
    ylabel={انتشار CO2 (میلیون تن/سال)},
    xmin=0, xmax=26,
    ymin=200, ymax=800,
    xtick={0,5,10,15,20,25},
    xticklabels={۱۴۰۳, سال ۵, سال ۱۰, سال ۱۵, سال ۲۰, سال ۲۵},
    legend pos=north east,
    grid=major,
    grid style={dashed, gray!30}
]
% مسیر بدون اقدام
\addplot[color=red, mark=triangle*, dashed, thick] coordinates {
    (0, 750) (5, 820) (10, 900) (15, 980) (20, 1050) (25, 1100)
};
% مسیر با اقدام
\addplot[color=green!70!black, mark=*, thick, line width=1.5pt] coordinates {
    (0, 750) (5, 700) (10, 620) (15, 550) (20, 450) (25, 350)
};
% خط هدف پاریس
\addplot[color=blue, dashed, thick] coordinates {
    (0, 750) (25, 375)
};

\legend{بدون اقدام, با برنامه پیشنهادی, هدف توافق پاریس}

% ناحیه سبز
\fill[green!20, opacity=0.3] (axis cs:20,200) rectangle (axis cs:25,400);
\node[font=\scriptsize, green!70!black] at (axis cs:22.5,300) {هدف};
\end{axis}
\end{tikzpicture}
\captionof{figure}{مسیر کاهش انتشار گازهای گلخانه‌ای}
\label{fig:emission-pathway}
\end{center}

\subsection{برنامه انتقال انرژی}

\begin{table}[htbp]
\centering
\caption{برنامه جامع انتقال انرژی}
\label{tab:energy-transition-plan}
\begin{tabular}{>{\columncolor{blue!8}}l p{4cm} c p{3.5cm}}
\toprule
\rowcolor{blue!25}
\textbf{بخش} & \textbf{اقدام} & \textbf{سرمایه‌گذاری} & \textbf{نتیجه} \\
\midrule
برق & ۶۰ گیگاوات خورشیدی و بادی & ۸۰ میلیارد \$ & ۵۰٪ برق تجدیدپذیر \\
\rowcolor{gray!10}
حمل‌ونقل & برقی‌سازی، حمل‌ونقل عمومی & ۴۰ میلیارد \$ & ۵۰٪ کاهش انتشار \\
صنعت & کارایی انرژی، هیدروژن سبز & ۳۰ میلیارد \$ & ۳۰٪ کاهش انتشار \\
\rowcolor{gray!10}
ساختمان & عایق‌بندی، گرمایش/سرمایش نوین & ۲۰ میلیارد \$ & ۵۰٪ کاهش مصرف \\
متان & کاهش سوختن و نشت گاز & ۱۰ میلیارد \$ & ۵۰٪ کاهش متان \\
\midrule
\textbf{مجموع} & & \textbf{۱۸۰ میلیارد \$} & ۵۰٪ کاهش کل انتشار \\
\bottomrule
\end{tabular}
\end{table}

%═══════════════════════════════════════════════════════════════════════════════
\section{هوای پاک: حق شهروندان}
\label{sec:clean-air}
%═══════════════════════════════════════════════════════════════════════════════

\subsection{برنامه هوای پاک شهرها}

\begin{table}[htbp]
\centering
\caption{برنامه جامع کاهش آلودگی هوای شهرها}
\label{tab:clean-air-program}
\begin{tabular}{>{\columncolor{cyan!8}}r p{4.5cm} p{4.5cm}}
\toprule
\rowcolor{cyan!25}
\textbf{منبع آلودگی} & \textbf{اقدامات} & \textbf{هدف کمّی} \\
\midrule
خودروها & از رده خارج کردن ۵ میلیون خودروی فرسوده، استانداردهای یورو ۶ & ۵۰٪ کاهش آلاینده‌ها \\
\rowcolor{gray!10}
موتورسیکلت & برقی‌سازی ۳ میلیون موتور & ۸۰٪ کاهش \\
صنایع & فیلتر اجباری، انتقال خارج شهر & ۷۰٪ کاهش \\
\rowcolor{gray!10}
نیروگاه‌ها & گازسوز کردن، فیلتر، انتقال & ۶۰٪ کاهش \\
ساختمان‌ها & بخاری‌های استاندارد، گاز به جای مازوت & ۵۰٪ کاهش \\
\rowcolor{gray!10}
گرد و غبار & فضای سبز، آب‌پاشی، کمربند سبز & ۴۰٪ کاهش \\
\bottomrule
\end{tabular}
\end{table}

\begin{table}[htbp]
\centering
\caption{اهداف کیفیت هوای کلان‌شهرها}
\label{tab:air-quality-targets}
\begin{tabular}{>{\columncolor{purple!8}}l c c c c}
\toprule
\rowcolor{purple!25}
\textbf{شهر} & \textbf{PM2.5 فعلی} & \textbf{هدف س۵} & \textbf{هدف س۱۰} & \textbf{WHO} \\
\midrule
تهران & ۳۵ & ۲۰ & ۱۰ & ۵ \\
\rowcolor{gray!10}
اهواز & ۸۰ & ۴۰ & ۲۰ & ۵ \\
تبریز & ۲۸ & ۱۵ & ۸ & ۵ \\
\rowcolor{gray!10}
اصفهان & ۳۲ & ۱۸ & ۱۰ & ۵ \\
مشهد & ۲۵ & ۱۵ & ۸ & ۵ \\
\bottomrule
\end{tabular}
\end{table}

\begin{olgoobox}
\textbf{الگوی موفق: پکن — از آلوده‌ترین به بهبود چشمگیر}

پکن در یک دهه آلودگی هوا را به شدت کاهش داد:
\begin{itemize}[nosep]
\item \textbf{۲۰۱۳}: PM2.5 میانگین ۸۹ — یکی از آلوده‌ترین شهرهای جهان
\item \textbf{۲۰۲۳}: PM2.5 میانگین ۳۲ — کاهش ۶۴٪
\item \textbf{اقدامات}: بستن نیروگاه‌های زغالی، محدودیت خودرو، صنایع پاک
\item \textbf{سرمایه‌گذاری}: ۱۲۰ میلیارد دلار در ۱۰ سال
\item \textbf{درس}: با اراده سیاسی و سرمایه‌گذاری، بهبود ممکن است
\end{itemize}
\end{olgoobox}

%═══════════════════════════════════════════════════════════════════════════════
\section{حفاظت از تنوع زیستی}
\label{sec:biodiversity}
%═══════════════════════════════════════════════════════════════════════════════

\subsection{ثروت زیستی ایران}

\begin{table}[htbp]
\centering
\caption{تنوع زیستی ایران}
\label{tab:biodiversity}
\begin{tabular}{>{\columncolor{green!8}}l c c p{4.5cm}}
\toprule
\rowcolor{green!25}
\textbf{گروه} & \textbf{تعداد گونه} & \textbf{بومی ایران} & \textbf{گونه‌های در خطر} \\
\midrule
پستانداران & ۱۹۵ & ۱۲ & یوزپلنگ آسیایی (کمتر از ۵۰) \\
\rowcolor{gray!10}
پرندگان & ۵۲۷ & ۵ & میش‌مرغ، هوبره \\
خزندگان & ۲۴۰ & ۵۵ & لاک‌پشت‌های دریایی \\
\rowcolor{gray!10}
ماهیان & ۲۲۰ & ۶۰ & ماهیان خاویاری خزر \\
گیاهان & ۸,۰۰۰+ & ۱,۷۰۰ & گونه‌های دارویی \\
\bottomrule
\end{tabular}
\end{table}

\subsection{برنامه حفاظت از حیات‌وحش}

\begin{table}[htbp]
\centering
\caption{برنامه حفاظت از تنوع زیستی}
\label{tab:wildlife-protection}
\begin{tabular}{>{\columncolor{teal!8}}r p{5cm} c}
\toprule
\rowcolor{teal!25}
\textbf{اقدام} & \textbf{توضیح} & \textbf{هدف} \\
\midrule
گسترش مناطق حفاظت‌شده & از ۱۰٪ به ۱۸٪ مساحت کشور & ۱۵ میلیون هکتار جدید \\
\rowcolor{gray!10}
کریدورهای حیات‌وحش & اتصال زیستگاه‌ها & ۲۰ کریدور اصلی \\
مبارزه با شکار غیرمجاز & محیط‌بانی قوی، جریمه سنگین & ۸۰٪ کاهش شکار \\
\rowcolor{gray!10}
برنامه نجات یوزپلنگ & تکثیر در اسارت، حفاظت زیستگاه & افزایش به ۲۰۰ قلاده \\
حفاظت دریایی & مناطق حفاظت‌شده دریایی & ۱۰٪ آب‌های ایران \\
\rowcolor{gray!10}
احیای گونه‌ها & برنامه بازگرداندن گونه‌های منقرض محلی & شیر ایرانی، گورخر \\
\bottomrule
\end{tabular}
\end{table}

%═══════════════════════════════════════════════════════════════════════════════
\section{اقتصاد سبز و مشاغل جدید}
\label{sec:green-economy}
%═══════════════════════════════════════════════════════════════════════════════

\begin{naghlbox}
«انتقال به اقتصاد سبز تهدید نیست، فرصت است. مطالعات نشان می‌دهد سرمایه‌گذاری در انرژی‌های تجدیدپذیر، ۳ برابر بیشتر از سوخت‌های فسیلی شغل ایجاد می‌کند.»
\sourceline{آژانس بین‌المللی انرژی‌های تجدیدپذیر (IRENA)، ۲۰۲۳}
\end{naghlbox}

\subsection{مشاغل سبز جدید}

\begin{table}[htbp]
\centering
\caption{پتانسیل اشتغال‌زایی اقتصاد سبز}
\label{tab:green-jobs}
\begin{tabular}{>{\columncolor{green!8}}l c c p{4cm}}
\toprule
\rowcolor{green!25}
\textbf{بخش} & \textbf{شغل فعلی (هزار)} & \textbf{هدف س۱۰} & \textbf{نوع مشاغل} \\
\midrule
انرژی خورشیدی & ۲۰ & ۳۰۰ & نصب، تعمیر، تولید پنل \\
\rowcolor{gray!10}
انرژی بادی & ۵ & ۸۰ & نصب، نگهداری توربین \\
خودروهای برقی & ۱۰ & ۲۰۰ & تولید، شارژ، باتری \\
\rowcolor{gray!10}
بازیافت و پسماند & ۵۰ & ۲۵۰ & جمع‌آوری، فرآوری \\
کشاورزی ارگانیک & ۳۰ & ۲۰۰ & تولید، گواهی، بازاریابی \\
\rowcolor{gray!10}
ساختمان سبز & ۲۰ & ۱۵۰ & معماری، عایق، سیستم‌ها \\
اکوتوریسم & ۵۰ & ۳۰۰ & راهنما، اقامتگاه، حفاظت \\
\rowcolor{gray!10}
آب (شیرین‌سازی، بازیافت) & ۳۰ & ۱۵۰ & مهندسی، اپراتوری \\
\midrule
\textbf{مجموع مشاغل سبز} & \textbf{۲۱۵} & \textbf{۱,۶۳۰} & ۱.۴ میلیون شغل جدید \\
\bottomrule
\end{tabular}
\end{table}

\subsection{فرصت‌های اقتصادی سبز}

\begin{table}[htbp]
\centering
\caption{بازارهای جدید اقتصاد سبز ایران}
\label{tab:green-markets}
\begin{tabular}{>{\columncolor{blue!8}}l c c p{4cm}}
\toprule
\rowcolor{blue!25}
\textbf{بازار} & \textbf{اندازه فعلی} & \textbf{هدف س۱۰} & \textbf{مزیت ایران} \\
\midrule
صادرات برق پاک & ۰.۵ میلیارد \$ & ۱۰ میلیارد \$ & آفتاب فراوان، موقعیت \\
\rowcolor{gray!10}
تجهیزات انرژی نو & ۰.۲ میلیارد \$ & ۵ میلیارد \$ & صنعت موجود \\
هیدروژن سبز & ۰ & ۵ میلیارد \$ & انرژی ارزان \\
\rowcolor{gray!10}
گردشگری اکو & ۱ میلیارد \$ & ۱۰ میلیارد \$ & تنوع طبیعی \\
فناوری آب & ۰.۵ میلیارد \$ & ۳ میلیارد \$ & تجربه بحران \\
\rowcolor{gray!10}
محصولات ارگانیک & ۰.۳ میلیارد \$ & ۳ میلیارد \$ & کشاورزی سنتی \\
\midrule
\textbf{مجموع بازار سبز} & \textbf{۲.۵ میلیارد \$} & \textbf{۳۶ میلیارد \$} & \\
\bottomrule
\end{tabular}
\end{table}

%═══════════════════════════════════════════════════════════════════════════════
\section{حکمرانی زیست‌محیطی}
\label{sec:env-governance}
%═══════════════════════════════════════════════════════════════════════════════

\subsection{ضعف‌های نهادی فعلی}

\begin{enghelabbox}
\textbf{مشکل: محیط زیست بدون قدرت}

سازمان حفاظت محیط زیست ایران:
\begin{itemize}[nosep]
\item بودجه ناچیز: کمتر از ۰.۲٪ بودجه کشور
\item بدون قدرت: نمی‌تواند جلوی پروژه‌های مخرب را بگیرد
\item سیاسی: رئیس سازمان منصوب سیاسی
\item ضعیف در اجرا: محیط‌بانان کم، تجهیزات ناکافی
\item \textbf{نتیجه}: رشد اقتصادی همیشه بر محیط زیست ترجیح داده شده
\end{itemize}
\end{enghelabbox}

\subsection{ساختار پیشنهادی}

\begin{table}[htbp]
\centering
\caption{ساختار نوین حکمرانی زیست‌محیطی}
\label{tab:env-governance-structure}
\begin{tabular}{>{\columncolor{green!8}}l p{4.5cm} p{4.5cm}}
\toprule
\rowcolor{green!25}
\textbf{نهاد} & \textbf{وظیفه} & \textbf{ویژگی کلیدی} \\
\midrule
وزارت محیط زیست & سیاستگذاری، اجرا، نظارت & وزیر در کابینه، بودجه ۱٪+ \\
\rowcolor{gray!10}
آژانس حفاظت محیط زیست & تنظیم‌گری، صدور مجوز، جریمه & مستقل، فنی، غیرسیاسی \\
دادگاه محیط زیست & رسیدگی به جرایم زیست‌محیطی & قضات متخصص، مجازات سنگین \\
\rowcolor{gray!10}
صندوق ملی محیط زیست & تأمین مالی پروژه‌ها & ۵ میلیارد دلار سرمایه اولیه \\
نیروی محیط‌بانی & حفاظت میدانی & ۲۰,۰۰۰ محیط‌بان (از ۵,۰۰۰ فعلی) \\
\bottomrule
\end{tabular}
\end{table}

\subsection{حقوق زیست‌محیطی در قانون اساسی}

\begin{table}[htbp]
\centering
\caption{حقوق زیست‌محیطی پیشنهادی برای قانون اساسی}
\label{tab:env-rights}
\begin{tabular}{>{\columncolor{teal!8}}r p{9cm}}
\toprule
\rowcolor{teal!25}
\textbf{حق} & \textbf{متن پیشنهادی} \\
\midrule
حق محیط زیست سالم & «هر شهروند حق زندگی در محیط زیست سالم را دارد.» \\
\rowcolor{gray!10}
حق آب پاک & «دسترسی به آب آشامیدنی سالم و کافی حق بنیادین است.» \\
حق هوای پاک & «دولت موظف به تأمین هوای پاک در شهرهاست.» \\
\rowcolor{gray!10}
حق اطلاع‌رسانی & «شهروندان حق دسترسی به داده‌های زیست‌محیطی را دارند.» \\
حق نسل‌های آینده & «دولت موظف به حفظ منابع طبیعی برای نسل‌های آینده است.» \\
\rowcolor{gray!10}
حق شکایت & «هر شهروند حق شکایت از تخریب محیط زیست را دارد.» \\
\bottomrule
\end{tabular}
\end{table}

%═══════════════════════════════════════════════════════════════════════════════
\section{همکاری بین‌المللی}
\label{sec:env-international}
%═══════════════════════════════════════════════════════════════════════════════

\subsection{توافقات و تعهدات}

\begin{table}[htbp]
\centering
\caption{تعهدات بین‌المللی زیست‌محیطی ایران}
\label{tab:env-international}
\begin{tabular}{>{\columncolor{blue!8}}l p{4cm} p{5cm}}
\toprule
\rowcolor{blue!25}
\textbf{توافق} & \textbf{وضعیت فعلی} & \textbf{اقدام پیشنهادی} \\
\midrule
توافق پاریس (اقلیم) & امضا، بدون اجرای جدی & تصویب مجلس، تعهد ۵۰٪ کاهش \\
\rowcolor{gray!10}
کنوانسیون تنوع زیستی & عضو & اجرای کامل، گزارش‌دهی \\
کنوانسیون رامسر (تالاب‌ها) & عضو مؤسس & احیای ۲۵ تالاب \\
\rowcolor{gray!10}
کنوانسیون بیابان‌زدایی & عضو & برنامه ملی مبارزه \\
پروتکل مونترال (اوزون) & عضو & تکمیل حذف CFCs \\
\bottomrule
\end{tabular}
\end{table}

\subsection{همکاری منطقه‌ای آب}

\begin{table}[htbp]
\centering
\caption{توافقات آبی منطقه‌ای}
\label{tab:water-agreements}
\begin{tabular}{>{\columncolor{cyan!8}}l l p{6cm}}
\toprule
\rowcolor{cyan!25}
\textbf{رودخانه/حوضه} & \textbf{کشورهای شریک} & \textbf{موضوع مذاکره} \\
\midrule
هیرمند & افغانستان & حقابه ایران، معاهده ۱۹۷۳ \\
\rowcolor{gray!10}
ارس & ترکیه، ارمنستان، آذربایجان & مدیریت مشترک، کیفیت آب \\
اروندرود & عراق & حقابه، کشتیرانی \\
\rowcolor{gray!10}
دجله و فرات (غیرمستقیم) & ترکیه، عراق، سوریه & تأثیر بر هورالعظیم \\
خزر & روسیه، قزاقستان، ترکمنستان، آذربایجان & رژیم حقوقی، حفاظت \\
\bottomrule
\end{tabular}
\end{table}

%═══════════════════════════════════════════════════════════════════════════════
\section{تقویم اجرایی}
\label{sec:env-timeline}
%═══════════════════════════════════════════════════════════════════════════════

\begin{table}[htbp]
\centering
\caption{تقویم اجرای برنامه‌های زیست‌محیطی}
\label{tab:env-timeline}
\begin{tabular}{>{\columncolor{orange!8}}c p{4.5cm} p{5cm}}
\toprule
\rowcolor{orange!25}
\textbf{زمان} & \textbf{اقدام کلیدی} & \textbf{شاخص موفقیت} \\
\midrule
ماه ۱-۶ & اعلام وضعیت اضطراری آب و محیط زیست & تصویب قانون اضطراری \\
\rowcolor{gray!10}
سال ۱ & تشکیل وزارت محیط زیست & ساختار جدید فعال \\
سال ۱-۲ & شروع پروژه‌های شیرین‌سازی & کلنگ ۵ پروژه بزرگ \\
\rowcolor{gray!10}
سال ۲ & قیمت‌گذاری واقعی آب کشاورزی (فاز ۱) & ۲۰٪ افزایش قیمت \\
سال ۳ & راه‌اندازی سامانه ملی پایش آب & داده لحظه‌ای همه حوضه‌ها \\
\rowcolor{gray!10}
سال ۵ & ارزیابی میان‌دوره‌ای & کاهش ۲۰٪ مصرف آب \\
سال ۵ & ۲۰٪ برق از تجدیدپذیر & ظرفیت ۲۰ گیگاوات \\
\rowcolor{gray!10}
سال ۱۰ & تعادل نسبی آب & کسری به ۵ میلیارد م۳ \\
سال ۱۰ & ۵۰٪ برق از تجدیدپذیر & ظرفیت ۶۰ گیگاوات \\
\rowcolor{gray!10}
سال ۱۵ & تعادل کامل آب & کسری صفر \\
سال ۱۵ & احیای ارومیه به ۵۰٪ ظرفیت & ۱۵ میلیارد م۳ حجم \\
\bottomrule
\end{tabular}
\end{table}

%═══════════════════════════════════════════════════════════════════════════════
\section{شاخص‌های پایش}
\label{sec:env-kpis}
%═══════════════════════════════════════════════════════════════════════════════

\begin{table}[htbp]
\centering
\caption{شاخص‌های کلیدی پایش زیست‌محیطی}
\label{tab:env-kpis}
\small
\begin{tabular}{>{\columncolor{green!8}}r l c c c c}
\toprule
\rowcolor{green!25}
& \textbf{شاخص} & \textbf{۱۴۰۳} & \textbf{س۵} & \textbf{س۱۰} & \textbf{س۱۵} \\
\midrule
۱ & کسری آب (میلیارد م۳) & ۱۸ & ۱۲ & ۵ & ۰ \\
\rowcolor{gray!10}
۲ & راندمان آبیاری (٪) & ۳۸ & ۵۵ & ۷۰ & ۸۵ \\
۳ & حجم دریاچه ارومیه (میلیارد م۳) & ۳ & ۶ & ۱۰ & ۱۵ \\
\rowcolor{gray!10}
۴ & PM2.5 تهران (میکروگرم) & ۳۵ & ۲۰ & ۱۲ & ۸ \\
۵ & سهم تجدیدپذیر در برق (٪) & ۸ & ۲۰ & ۵۰ & ۷۰ \\
\rowcolor{gray!10}
۶ & انتشار CO2 (میلیون تن) & ۷۵۰ & ۷۰۰ & ۵۵۰ & ۴۰۰ \\
۷ & پوشش جنگلی (٪) & ۷ & ۸ & ۱۰ & ۱۲ \\
\rowcolor{gray!10}
۸ & مناطق حفاظت‌شده (٪ مساحت) & ۱۰ & ۱۲ & ۱۵ & ۱۸ \\
۹ & بازیافت پسماند (٪) & ۸ & ۲۵ & ۴۵ & ۶۰ \\
\rowcolor{gray!10}
۱۰ & شاخص EPI (از ۱۰۰) & ۴۰ & ۵۰ & ۶۵ & ۷۵ \\
\bottomrule
\end{tabular}
\end{table}

%═══════════════════════════════════════════════════════════════════════════════
\section{جمع‌بندی: سبز یا نابود}
\label{sec:env-conclusion}
%═══════════════════════════════════════════════════════════════════════════════

\begin{olgoobox}
\textbf{پیام کلیدی فصل}

بحران آب و محیط زیست ایران:
\begin{itemize}[nosep]
\item \textbf{واقعی و فوری است}: نه تهدید آینده، بلکه بحران امروز
\item \textbf{قابل حل است}: اگر اراده سیاسی و سرمایه‌گذاری باشد
\item \textbf{نیازمند تحول است}: تغییر بنیادین در مصرف آب، انرژی، کشاورزی
\item \textbf{فرصت اقتصادی است}: ۱.۵ میلیون شغل سبز، بازارهای جدید
\item \textbf{مسئولیت بین‌نسلی است}: آنچه امروز نجات ندهیم، فردا نخواهد بود
\item \textbf{پایه ثبات سیاسی است}: بدون آب و محیط سالم، دموکراسی پایدار نخواهد بود
\end{itemize}

\textbf{شعار}: «هر قطره آب، هر نفس پاک — حق ما و حق نسل‌های آینده»
\end{olgoobox}

\begin{naghlbox}
«ما زمین را از نیاکان به ارث نبرده‌ایم؛ از فرزندانمان به امانت گرفته‌ایم. نسل ما آخرین نسلی است که می‌تواند فاجعه را متوقف کند — و اولین نسلی که پیامدهای آن را خواهد دید.»
\sourceline{ضرب‌المثل بومیان آمریکا، بازخوانی برای ایران}
\end{naghlbox}

%═══════════════════════════════════════════════════════════════════════════════
% منابع فصل
%═══════════════════════════════════════════════════════════════════════════════

\section*{منابع فصل چهاردهم}
\addcontentsline{toc}{section}{منابع فصل چهاردهم}

\begin{itemize}[nosep, font=\small]
\item IPCC. (2022). \textit{Climate Change 2022: Impacts, Adaptation and Vulnerability}. Cambridge University Press.
\item Madani, K. (2014). Water management in Iran: What is causing the looming crisis? \textit{Journal of Environmental Studies and Sciences}, 4(4), 315-328.
\item World Bank. (2016). \textit{High and Dry: Climate Change, Water, and the Economy}. Washington, DC.
\item FAO. (2021). \textit{The State of the World's Land and Water Resources for Food and Agriculture}. Rome.
\item IRENA. (2023). \textit{Renewable Energy and Jobs: Annual Review}. Abu Dhabi.
\item IEA. (2023). \textit{World Energy Outlook}. Paris.
\item WHO. (2021). \textit{WHO Global Air Quality Guidelines}. Geneva.
\item UNEP. (2022). \textit{Global Environment Outlook}. Nairobi.
\item Foltz, R. C. (2002). \textit{Environmentalism in the Muslim World}. Nova Science.
\item وزارت نیرو. (۱۴۰۲). \textit{ترازنامه آب ایران}.
\item سازمان حفاظت محیط زیست. (۱۴۰۲). \textit{گزارش وضعیت محیط زیست کشور}.
\item مرکز پژوهش‌های مجلس. (۱۴۰۲). \textit{گزارش‌های بحران آب}.
\item مؤسسه تحقیقات آب. (۱۴۰۲). \textit{وضعیت سفره‌های آب زیرزمینی}.
\item کتاب سبز ایران. (۱۴۰۱). \textit{گزارش تنوع زیستی}. سازمان حفاظت محیط زیست.
\end{itemize}
	
	%──────────────────────────────────────────────────────────────────────────────
	% بخش پنجم: اجرا و پایش
	%──────────────────────────────────────────────────────────────────────────────
	\part{اجرا و پایش}
	
	% ch15-monitoring.tex
% فصل پانزدهم: پایش، ارزیابی و مدیریت ریسک
% نویسنده: مهدی سالم | ریچموندهیل | ۱۴۰۴

\chapter{پایش، ارزیابی و مدیریت ریسک}
\label{ch:monitoring}

\begin{kholasebox}
هر برنامه‌ای بدون سیستم پایش و ارزیابی محکوم به شکست است. این فصل چارچوب جامعی برای \textbf{پایش پیشرفت گذار دموکراتیک}، \textbf{ارزیابی دوره‌ای}، و \textbf{مدیریت ریسک‌ها} ارائه می‌دهد. سیستم پیشنهادی شامل: \textbf{داشبورد ملی} با ۵۰ شاخص کلیدی، \textbf{گزارش‌دهی شفاف} به مردم، \textbf{نظام هشدار زودهنگام} برای شناسایی انحرافات، و \textbf{مکانیزم اصلاح مسیر} است. اصل راهنما: \textbf{«آنچه اندازه‌گیری نشود، مدیریت نمی‌شود — و آنچه شفاف نباشد، فاسد می‌شود»}.
\end{kholasebox}

%═══════════════════════════════════════════════════════════════════════════════
\section{مقدمه: چرا پایش حیاتی است؟}
%═══════════════════════════════════════════════════════════════════════════════

\begin{naghlbox}
«برنامه‌ریزی بدون پایش مثل دویدن با چشم‌های بسته است. شاید بدانید کجا می‌خواهید بروید، اما نمی‌دانید کجا هستید و آیا در مسیر درست حرکت می‌کنید یا نه.»
\sourceline{پیتر دراکر، پدر مدیریت مدرن}
\end{naghlbox}

گذار دموکراتیک فرآیندی پیچیده، غیرخطی و پر از غافلگیری است. بهترین برنامه‌ها هم در برخورد با واقعیت تغییر می‌کنند. سیستم پایش و ارزیابی ابزاری است برای:

\begin{itemize}[nosep]
\item \textbf{فهم واقعیت}: کجا هستیم؟ چقدر پیشرفت کرده‌ایم؟
\item \textbf{شناسایی انحرافات}: آیا از مسیر خارج شده‌ایم؟
\item \textbf{پاسخگویی}: آیا مسئولان به تعهداتشان عمل کرده‌اند؟
\item \textbf{یادگیری}: چه چیزی کار کرد؟ چه چیزی نه؟
\item \textbf{اصلاح مسیر}: چه تغییراتی لازم است؟
\end{itemize}

\subsection{تفاوت پایش و ارزیابی}

\begin{table}[htbp]
\centering
\caption{تفاوت پایش و ارزیابی}
\label{tab:monitoring-vs-evaluation}
\begin{tabular}{>{\columncolor{blue!8}}l p{5cm} p{5cm}}
\toprule
\rowcolor{blue!25}
\textbf{بُعد} & \textbf{پایش (Monitoring)} & \textbf{ارزیابی (Evaluation)} \\
\midrule
زمان & مستمر و لحظه‌ای & دوره‌ای (سالانه، میان‌دوره، پایانی) \\
\rowcolor{gray!10}
سؤال اصلی & آیا در مسیر هستیم؟ & آیا به هدف رسیدیم؟ چرا؟ \\
تمرکز & فعالیت‌ها و خروجی‌ها & نتایج و تأثیرات \\
\rowcolor{gray!10}
مسئول & واحدهای اجرایی & ارزیابان مستقل \\
روش & داده‌های اداری، داشبورد & تحقیق، نظرسنجی، مصاحبه \\
\rowcolor{gray!10}
کاربرد & اصلاح فوری & یادگیری، سیاستگذاری \\
\bottomrule
\end{tabular}
\end{table}

%═══════════════════════════════════════════════════════════════════════════════
\section{چارچوب پایش: مدل منطقی}
\label{sec:logic-model}
%═══════════════════════════════════════════════════════════════════════════════

\begin{center}
\begin{tikzpicture}[
    node distance=1.5cm,
    box/.style={
        rectangle,
        rounded corners=5pt,
        draw=#1!70,
        fill=#1!15,
        thick,
        minimum width=2.5cm,
        minimum height=1.5cm,
        align=center,
        font=\small
    },
    arrow/.style={->, thick, >=stealth}
]
% زنجیره نتایج
\node[box=blue] (input) {\textbf{ورودی‌ها}\\ \scriptsize بودجه، نیروی انسانی،\\ \scriptsize قوانین};
\node[box=green, right=1.5cm of input] (activity) {\textbf{فعالیت‌ها}\\ \scriptsize اصلاحات، پروژه‌ها،\\ \scriptsize برنامه‌ها};
\node[box=orange, right=1.5cm of activity] (output) {\textbf{خروجی‌ها}\\ \scriptsize قوانین تصویب‌شده،\\ \scriptsize نهادهای ایجادشده};
\node[box=purple, right=1.5cm of output] (outcome) {\textbf{پیامدها}\\ \scriptsize رفتار تغییریافته،\\ \scriptsize دسترسی بهبودیافته};
\node[box=red, right=1.5cm of outcome] (impact) {\textbf{تأثیرات}\\ \scriptsize دموکراسی پایدار،\\ \scriptsize رفاه فراگیر};

% فلش‌ها
\draw[arrow] (input) -- (activity);
\draw[arrow] (activity) -- (output);
\draw[arrow] (output) -- (outcome);
\draw[arrow] (outcome) -- (impact);

% برچسب‌های پایش
\node[below=0.8cm of input, font=\tiny, align=center] {پایش منابع};
\node[below=0.8cm of activity, font=\tiny, align=center] {پایش اجرا};
\node[below=0.8cm of output, font=\tiny, align=center] {پایش خروجی};
\node[below=0.8cm of outcome, font=\tiny, align=center] {ارزیابی پیامد};
\node[below=0.8cm of impact, font=\tiny, align=center] {ارزیابی تأثیر};

% عنوان
\node[above=1.5cm of output, font=\large\bfseries] {زنجیره نتایج و سطوح پایش};

% کادر پایش vs ارزیابی
\draw[dashed, blue!50, thick] (-1,-1.5) rectangle (6.5,-2.5);
\node[blue!70, font=\scriptsize] at (2.75,-2) {پایش مستمر};
\draw[dashed, red!50, thick] (6.7,-1.5) rectangle (12.5,-2.5);
\node[red!70, font=\scriptsize] at (9.6,-2) {ارزیابی دوره‌ای};
\end{tikzpicture}
\end{center}

%═══════════════════════════════════════════════════════════════════════════════
\section{شاخص‌های کلیدی عملکرد (KPIs)}
\label{sec:kpis}
%═══════════════════════════════════════════════════════════════════════════════

\subsection{اصول انتخاب شاخص}

\begin{table}[htbp]
\centering
\caption{معیارهای SMART برای شاخص‌های خوب}
\label{tab:smart-criteria}
\begin{tabular}{>{\columncolor{green!8}}c l p{7cm}}
\toprule
\rowcolor{green!25}
\textbf{حرف} & \textbf{معیار} & \textbf{توضیح} \\
\midrule
S & Specific (مشخص) & دقیقاً چه چیزی اندازه‌گیری می‌شود \\
\rowcolor{gray!10}
M & Measurable (قابل اندازه‌گیری) & عدد و داده قابل جمع‌آوری \\
A & Achievable (دست‌یافتنی) & هدف واقع‌بینانه \\
\rowcolor{gray!10}
R & Relevant (مرتبط) & به هدف اصلی کمک می‌کند \\
T & Time-bound (زمان‌دار) & مهلت مشخص دارد \\
\bottomrule
\end{tabular}
\end{table}

\subsection{شاخص‌های حوزه سیاسی}

\begin{table}[htbp]
\centering
\caption{شاخص‌های کلیدی حوزه سیاسی و دموکراتیک}
\label{tab:political-kpis}
\small
\begin{tabular}{>{\columncolor{blue!8}}r p{3.5cm} c c c c c}
\toprule
\rowcolor{blue!25}
& \textbf{شاخص} & \textbf{مبدأ} & \textbf{س۲} & \textbf{س۵} & \textbf{س۱۰} & \textbf{س۱۵} \\
\midrule
۱ & شاخص دموکراسی (EIU, از ۱۰) & ۲.۲ & ۴.۵ & ۶.۰ & ۷.۵ & ۸.۰ \\
\rowcolor{gray!10}
۲ & آزادی مطبوعات (RSF, از ۱۸۰) & ۱۷۶ & ۱۲۰ & ۸۰ & ۵۰ & ۳۵ \\
۳ & مشارکت انتخاباتی (٪) & ۴۰ & ۵۵ & ۶۵ & ۷۵ & ۷۵ \\
\rowcolor{gray!10}
۴ & زنان در مجلس (٪) & ۶ & ۱۵ & ۲۵ & ۳۵ & ۴۵ \\
۵ & اعتماد به دولت (٪) & ۱۵ & ۳۰ & ۴۵ & ۵۵ & ۶۵ \\
\rowcolor{gray!10}
۶ & حاکمیت قانون (WJP, از ۱) & ۰.۴ & ۰.۵ & ۰.۶ & ۰.۷ & ۰.۸ \\
۷ & استقلال قضایی (از ۱۰) & ۳ & ۵ & ۶ & ۷ & ۸ \\
\rowcolor{gray!10}
۸ & تعداد NGOهای فعال (هزار) & ۲۰ & ۴۰ & ۷۰ & ۱۰۰ & ۱۵۰ \\
۹ & شاخص فساد (CPI, از ۱۰۰) & ۲۵ & ۳۵ & ۴۵ & ۵۵ & ۷۰ \\
\rowcolor{gray!10}
۱۰ & آزادی اینترنت (از ۱۰۰) & ۱۶ & ۴۰ & ۶۰ & ۷۵ & ۸۵ \\
\bottomrule
\end{tabular}
\end{table}

\subsection{شاخص‌های حوزه اقتصادی}

\begin{table}[htbp]
\centering
\caption{شاخص‌های کلیدی حوزه اقتصادی}
\label{tab:economic-kpis-monitoring}
\small
\begin{tabular}{>{\columncolor{green!8}}r p{3.5cm} c c c c c}
\toprule
\rowcolor{green!25}
& \textbf{شاخص} & \textbf{مبدأ} & \textbf{س۲} & \textbf{س۵} & \textbf{س۱۰} & \textbf{س۱۵} \\
\midrule
۱ & رشد GDP (٪) & ۲ & ۵ & ۷ & ۶ & ۵ \\
\rowcolor{gray!10}
۲ & تورم (٪) & ۵۰ & ۱۵ & ۷ & ۴ & ۲ \\
۳ & بیکاری (٪) & ۱۲ & ۱۰ & ۸ & ۶ & ۵ \\
\rowcolor{gray!10}
۴ & GDP سرانه (هزار \$ PPP) & ۱۵ & ۱۷ & ۲۲ & ۳۲ & ۴۵ \\
۵ & ضریب جینی & ۰.۴۲ & ۰.۴۰ & ۰.۳۸ & ۰.۳۶ & ۰.۳۵ \\
\rowcolor{gray!10}
۶ & نرخ فقر (٪) & ۳۰ & ۲۵ & ۱۸ & ۱۰ & ۵ \\
۷ & FDI سالانه (میلیارد \$) & ۱ & ۵ & ۲۵ & ۵۰ & ۶۰ \\
\rowcolor{gray!10}
۸ & سهم نفت از بودجه (٪) & ۳۵ & ۳۰ & ۲۲ & ۱۵ & ۱۰ \\
۹ & رتبه کسب‌وکار (از ۱۹۰) & ۱۲۷ & ۹۰ & ۵۰ & ۳۵ & ۲۵ \\
\rowcolor{gray!10}
۱۰ & نسبت مالیات/GDP (٪) & ۷ & ۹ & ۱۲ & ۱۶ & ۱۸ \\
\bottomrule
\end{tabular}
\end{table}

\subsection{شاخص‌های حوزه اجتماعی}

\begin{table}[htbp]
\centering
\caption{شاخص‌های کلیدی حوزه اجتماعی و انسانی}
\label{tab:social-kpis}
\small
\begin{tabular}{>{\columncolor{purple!8}}r p{3.5cm} c c c c c}
\toprule
\rowcolor{purple!25}
& \textbf{شاخص} & \textbf{مبدأ} & \textbf{س۲} & \textbf{س۵} & \textbf{س۱۰} & \textbf{س۱۵} \\
\midrule
۱ & شاخص توسعه انسانی (HDI) & ۰.۷۷ & ۰.۷۸ & ۰.۸۱ & ۰.۸۴ & ۰.۸۷ \\
\rowcolor{gray!10}
۲ & امید به زندگی (سال) & ۷۷ & ۷۷.۵ & ۷۹ & ۸۱ & ۸۳ \\
۳ & میانگین سال تحصیل & ۱۰ & ۱۰.۵ & ۱۱.۵ & ۱۲.۵ & ۱۴ \\
\rowcolor{gray!10}
۴ & شکاف جنسیتی (GGI, از ۱) & ۰.۵۷ & ۰.۶۲ & ۰.۷۰ & ۰.۷۵ & ۰.۸۰ \\
۵ & رضایت از زندگی (از ۱۰) & ۴.۵ & ۵.۵ & ۶.۵ & ۷.۰ & ۷.۵ \\
\rowcolor{gray!10}
۶ & اعتماد بین‌فردی (٪) & ۲۰ & ۲۵ & ۳۵ & ۴۵ & ۵۵ \\
۷ & پوشش بیمه سلامت (٪) & ۸۵ & ۹۰ & ۹۵ & ۹۸ & ۱۰۰ \\
\rowcolor{gray!10}
۸ & دسترسی به اینترنت (٪) & ۸۰ & ۸۵ & ۹۲ & ۹۷ & ۹۹ \\
۹ & مسکن مناسب (٪ خانوار) & ۷۰ & ۷۵ & ۸۲ & ۸۸ & ۹۵ \\
\rowcolor{gray!10}
۱۰ & رضایت اقوام از حقوق (٪) & ۳۰ & ۴۵ & ۶۰ & ۷۵ & ۸۵ \\
\bottomrule
\end{tabular}
\end{table}

\subsection{شاخص‌های حوزه محیط زیست}

\begin{table}[htbp]
\centering
\caption{شاخص‌های کلیدی حوزه محیط زیست}
\label{tab:environmental-kpis}
\small
\begin{tabular}{>{\columncolor{teal!8}}r p{3.5cm} c c c c c}
\toprule
\rowcolor{teal!25}
& \textbf{شاخص} & \textbf{مبدأ} & \textbf{س۲} & \textbf{س۵} & \textbf{س۱۰} & \textbf{س۱۵} \\
\midrule
۱ & کسری آب (میلیارد م۳) & ۱۸ & ۱۵ & ۱۰ & ۵ & ۰ \\
\rowcolor{gray!10}
۲ & PM2.5 تهران (میکروگرم) & ۳۵ & ۲۸ & ۲۰ & ۱۲ & ۸ \\
۳ & سهم انرژی تجدیدپذیر (٪) & ۸ & ۱۲ & ۲۵ & ۵۰ & ۷۰ \\
\rowcolor{gray!10}
۴ & انتشار CO2 (میلیون تن) & ۷۵۰ & ۷۲۰ & ۶۵۰ & ۵۰۰ & ۴۰۰ \\
۵ & حجم ارومیه (میلیارد م۳) & ۳ & ۴ & ۷ & ۱۲ & ۱۵ \\
\rowcolor{gray!10}
۶ & پوشش جنگلی (٪) & ۷ & ۷.۵ & ۸.۵ & ۱۰ & ۱۲ \\
۷ & بازیافت پسماند (٪) & ۸ & ۱۵ & ۳۰ & ۴۵ & ۶۰ \\
\rowcolor{gray!10}
۸ & مناطق حفاظت‌شده (٪) & ۱۰ & ۱۱ & ۱۳ & ۱۶ & ۱۸ \\
۹ & راندمان آبیاری (٪) & ۳۸ & ۴۵ & ۵۵ & ۷۰ & ۸۵ \\
\rowcolor{gray!10}
۱۰ & شاخص EPI (از ۱۰۰) & ۴۰ & ۴۵ & ۵۵ & ۶۵ & ۷۵ \\
\bottomrule
\end{tabular}
\end{table}

\subsection{شاخص‌های حوزه امنیت و بین‌الملل}

\begin{table}[htbp]
\centering
\caption{شاخص‌های کلیدی امنیت و روابط بین‌الملل}
\label{tab:security-kpis}
\small
\begin{tabular}{>{\columncolor{red!8}}r p{3.5cm} c c c c c}
\toprule
\rowcolor{red!25}
& \textbf{شاخص} & \textbf{مبدأ} & \textbf{س۲} & \textbf{س۵} & \textbf{س۱۰} & \textbf{س۱۵} \\
\midrule
۱ & شاخص صلح جهانی (GPI) & ۱۴۰ & ۱۱۰ & ۸۰ & ۵۵ & ۴۰ \\
\rowcolor{gray!10}
۲ & تعداد تحریم‌های فعال & ۳۸۰۰ & ۲۰۰۰ & ۵۰۰ & ۱۰۰ & ۰ \\
۳ & روابط دیپلماتیک عادی & ۸۰ & ۱۲۰ & ۱۶۰ & ۱۸۰ & ۱۹۰ \\
\rowcolor{gray!10}
۴ & عضویت در سازمان‌های بین‌المللی & ۵۰ & ۶۰ & ۸۰ & ۱۰۰ & ۱۲۰ \\
۵ & گردشگر ورودی (میلیون) & ۵ & ۸ & ۱۵ & ۲۵ & ۳۵ \\
\rowcolor{gray!10}
۶ & رتبه پاسپورت ایرانی & ۱۸۰+ & ۱۵۰ & ۱۰۰ & ۶۰ & ۴۰ \\
۷ & صادرات غیرنفتی (میلیارد \$) & ۳۵ & ۵۰ & ۸۰ & ۱۴۰ & ۲۲۰ \\
\rowcolor{gray!10}
۸ & نرخ جرم (قتل/۱۰۰ هزار) & ۳ & ۲.۵ & ۲ & ۱.۵ & ۱.۲ \\
\bottomrule
\end{tabular}
\end{table}

%═══════════════════════════════════════════════════════════════════════════════
\section{داشبورد ملی پیشرفت}
\label{sec:national-dashboard}
%═══════════════════════════════════════════════════════════════════════════════

\begin{center}
\begin{tikzpicture}[scale=0.95]
% کادر اصلی داشبورد
\draw[thick, rounded corners=15pt, fill=gray!5] (-7,-6) rectangle (7,5.5);
\node[font=\large\bfseries] at (0,5) {داشبورد ملی پیشرفت گذار دموکراتیک};

% ۹ کادر شاخص
% ردیف اول
\node[rectangle, rounded corners=5pt, draw=blue!70, fill=blue!15,
      minimum width=3.8cm, minimum height=1.8cm, align=center] at (-4.5,3) 
    {\textbf{دموکراسی}\\ \Large ۶.۵/۱۰\\ \scriptsize هدف: ۸.۰};
\node[rectangle, rounded corners=5pt, draw=green!70, fill=green!15,
      minimum width=3.8cm, minimum height=1.8cm, align=center] at (0,3) 
    {\textbf{توسعه انسانی}\\ \Large ۰.۸۲\\ \scriptsize هدف: ۰.۸۷};
\node[rectangle, rounded corners=5pt, draw=orange!70, fill=orange!15,
      minimum width=3.8cm, minimum height=1.8cm, align=center] at (4.5,3) 
    {\textbf{رشد اقتصادی}\\ \Large ۶٪\\ \scriptsize هدف: ۵٪+};

% ردیف دوم
\node[rectangle, rounded corners=5pt, draw=red!70, fill=red!15,
      minimum width=3.8cm, minimum height=1.8cm, align=center] at (-4.5,0.5) 
    {\textbf{تورم}\\ \Large ۱۲٪\\ \scriptsize هدف: ۵٪};
\node[rectangle, rounded corners=5pt, draw=purple!70, fill=purple!15,
      minimum width=3.8cm, minimum height=1.8cm, align=center] at (0,0.5) 
    {\textbf{بیکاری}\\ \Large ۸٪\\ \scriptsize هدف: ۵٪};
\node[rectangle, rounded corners=5pt, draw=teal!70, fill=teal!15,
      minimum width=3.8cm, minimum height=1.8cm, align=center] at (4.5,0.5) 
    {\textbf{فساد (CPI)}\\ \Large ۵۰/۱۰۰\\ \scriptsize هدف: ۷۰};

% ردیف سوم
\node[rectangle, rounded corners=5pt, draw=cyan!70, fill=cyan!15,
      minimum width=3.8cm, minimum height=1.8cm, align=center] at (-4.5,-2) 
    {\textbf{محیط زیست}\\ \Large ۵۵/۱۰۰\\ \scriptsize هدف: ۷۵};
\node[rectangle, rounded corners=5pt, draw=yellow!70!black, fill=yellow!15,
      minimum width=3.8cm, minimum height=1.8cm, align=center] at (0,-2) 
    {\textbf{برابری جنسیتی}\\ \Large ۰.۷۰\\ \scriptsize هدف: ۰.۸۰};
\node[rectangle, rounded corners=5pt, draw=pink!70!black, fill=pink!15,
      minimum width=3.8cm, minimum height=1.8cm, align=center] at (4.5,-2) 
    {\textbf{رضایت مردم}\\ \Large ۶.۵/۱۰\\ \scriptsize هدف: ۷.۵};

% نوار وضعیت کلی
\draw[thick, fill=green!30] (-6,-4) rectangle (6,-3.5);
\node[font=\small\bfseries] at (0,-3.75) {وضعیت کلی: در مسیر | پیشرفت: ۶۵٪ از اهداف سال ۱۰};

% نوار زمان
\draw[thick] (-6,-5) -- (6,-5);
\foreach \x/\label in {-6/۱۴۰۳, -3/س۵, 0/س۱۰, 3/س۱۵, 6/س۲۵} {
    \fill (\x,-5) circle (3pt);
    \node[below, font=\tiny] at (\x,-5) {\label};
}
\fill[red] (-1.5,-5) circle (5pt);
\node[above, font=\tiny\bfseries, red] at (-1.5,-4.8) {اکنون};
\end{tikzpicture}
\captionof{figure}{نمونه داشبورد ملی پیشرفت (فرضی — سال ۷)}
\label{fig:dashboard-sample}
\end{center}

\subsection{ویژگی‌های داشبورد}

\begin{table}[htbp]
\centering
\caption{ویژگی‌های داشبورد ملی پیشرفت}
\label{tab:dashboard-features}
\begin{tabular}{>{\columncolor{blue!8}}r p{4cm} p{5.5cm}}
\toprule
\rowcolor{blue!25}
\textbf{ویژگی} & \textbf{توضیح} & \textbf{اهمیت} \\
\midrule
دسترسی عمومی & آنلاین، رایگان، به فارسی و انگلیسی & شفافیت و پاسخگویی \\
\rowcolor{gray!10}
به‌روزرسانی & ماهانه برای اکثر شاخص‌ها & اطلاعات تازه \\
سطوح مختلف & ملی، استانی، شهری & جزئیات برای هر منطقه \\
\rowcolor{gray!10}
مقایسه بین‌المللی & جایگاه ایران در جهان & بنچ‌مارکینگ \\
روند تاریخی & نمودار تغییرات در زمان & دیدن پیشرفت \\
\rowcolor{gray!10}
دانلود داده & داده‌های خام قابل دانلود & تحقیق و تحلیل مستقل \\
API & برای توسعه‌دهندگان & اکوسیستم داده باز \\
\bottomrule
\end{tabular}
\end{table}

%═══════════════════════════════════════════════════════════════════════════════
\section{نظام ارزیابی دوره‌ای}
\label{sec:periodic-evaluation}
%═══════════════════════════════════════════════════════════════════════════════

\subsection{تقویم ارزیابی}

\begin{table}[htbp]
\centering
\caption{تقویم ارزیابی‌های دوره‌ای}
\label{tab:evaluation-calendar}
\begin{tabular}{>{\columncolor{orange!8}}l c p{4cm} p{3.5cm}}
\toprule
\rowcolor{orange!25}
\textbf{نوع ارزیابی} & \textbf{زمان} & \textbf{محتوا} & \textbf{مسئول} \\
\midrule
گزارش ماهانه & هر ماه & شاخص‌های کلیدی، انحرافات & دفتر رئیس‌جمهور \\
\rowcolor{gray!10}
گزارش فصلی & هر ۳ ماه & تحلیل عمیق‌تر، اصلاحات & دولت به مجلس \\
گزارش سالانه & هر سال & ارزیابی جامع، نظرسنجی & نهاد مستقل ارزیابی \\
\rowcolor{gray!10}
ارزیابی میان‌دوره‌ای & هر ۵ سال & ارزیابی استراتژیک & پانل مستقل داخلی-خارجی \\
ارزیابی فاز & پایان هر فاز & آیا به اهداف رسیدیم؟ & کمیته ملی ارزیابی \\
\rowcolor{gray!10}
ممیزی بین‌المللی & هر ۲ سال & توسط نهادهای بین‌المللی & UN, EU, IMF, WB \\
\bottomrule
\end{tabular}
\end{table}

\subsection{معیارهای ارزیابی موفقیت}

\begin{olgoobox}
\textbf{سه سؤال کلیدی ارزیابی}

هر ارزیابی باید به سه سؤال پاسخ دهد:
\begin{enumerate}[nosep]
\item \textbf{اثربخشی}: آیا به اهداف رسیدیم؟ چقدر؟
\item \textbf{کارایی}: با چه هزینه‌ای؟ آیا راه بهتری بود؟
\item \textbf{پایداری}: آیا دستاوردها ماندگارند؟
\end{enumerate}

\textbf{سؤالات تکمیلی}:
\begin{itemize}[nosep]
\item \textbf{مرتبط بودن}: آیا اهداف درست انتخاب شدند؟
\item \textbf{انسجام}: آیا سیاست‌ها هماهنگ بودند؟
\item \textbf{تأثیر}: چه تغییر واقعی در زندگی مردم ایجاد شد؟
\end{itemize}
\end{olgoobox}

%═══════════════════════════════════════════════════════════════════════════════
\section{مدیریت ریسک}
\label{sec:risk-management}
%═══════════════════════════════════════════════════════════════════════════════

\subsection{ماتریس ریسک‌های گذار}

\begin{table}[htbp]
\centering
\caption{ماتریس ریسک‌های اصلی گذار دموکراتیک}
\label{tab:risk-matrix}
\small
\begin{tabular}{>{\columncolor{red!8}}l c c p{4.5cm}}
\toprule
\rowcolor{red!25}
\textbf{ریسک} & \textbf{احتمال} & \textbf{تأثیر} & \textbf{استراتژی کاهش} \\
\midrule
بازگشت اقتدارگرایی & متوسط & بحرانی & نهادسازی قوی، جامعه مدنی \\
\rowcolor{gray!10}
جنگ داخلی/خشونت & پایین & فاجعه‌بار & عدالت انتقالی، دیالوگ ملی \\
شکست اقتصادی & متوسط & بالا & سیاست‌های احتیاطی، FDI \\
\rowcolor{gray!10}
تجزیه‌طلبی & پایین-متوسط & بحرانی & فدرالیسم، حقوق اقوام \\
مداخله خارجی & متوسط & بالا & دیپلماسی فعال، موازنه \\
\rowcolor{gray!10}
فساد سیستماتیک & بالا & بالا & شفافیت، نهاد ضدفساد مستقل \\
بحران آب/محیط زیست & بالا & بالا & اقدام فوری، سرمایه‌گذاری \\
\rowcolor{gray!10}
پوپولیسم & متوسط-بالا & متوسط & آموزش، رسانه مسئول \\
خستگی اصلاحات & متوسط & متوسط & دستاوردهای ملموس، ارتباط \\
\rowcolor{gray!10}
تنش با همسایگان & متوسط & متوسط-بالا & دیپلماسی منطقه‌ای \\
\bottomrule
\end{tabular}
\end{table}

\subsection{نقشه حرارتی ریسک}

\begin{center}
\begin{tikzpicture}
% محورها
\draw[thick, ->] (0,0) -- (10,0) node[right] {احتمال};
\draw[thick, ->] (0,0) -- (0,8) node[above] {تأثیر};

% برچسب‌های محورها
\node[below] at (2,0) {پایین};
\node[below] at (5,0) {متوسط};
\node[below] at (8,0) {بالا};
\node[left] at (0,2) {پایین};
\node[left] at (0,4) {متوسط};
\node[left] at (0,6) {بالا};
\node[left] at (0,7.5) {بحرانی};

% مناطق رنگی
\fill[green!30] (0,0) rectangle (3.3,4);
\fill[yellow!30] (3.3,0) rectangle (6.6,4);
\fill[yellow!30] (0,4) rectangle (3.3,6);
\fill[orange!30] (6.6,0) rectangle (10,4);
\fill[orange!30] (3.3,4) rectangle (6.6,6);
\fill[orange!30] (0,6) rectangle (3.3,8);
\fill[red!30] (6.6,4) rectangle (10,8);
\fill[red!30] (3.3,6) rectangle (6.6,8);

% خطوط شبکه
\draw[gray, dashed] (3.3,0) -- (3.3,8);
\draw[gray, dashed] (6.6,0) -- (6.6,8);
\draw[gray, dashed] (0,4) -- (10,4);
\draw[gray, dashed] (0,6) -- (10,6);

% ریسک‌ها
\node[circle, fill=black, minimum size=8pt, label=right:{\scriptsize جنگ داخلی}] at (2,7.5) {};
\node[circle, fill=black, minimum size=8pt, label=right:{\scriptsize تجزیه‌طلبی}] at (3.5,7) {};
\node[circle, fill=black, minimum size=8pt, label=left:{\scriptsize بازگشت اقتدارگرایی}] at (5,6.5) {};
\node[circle, fill=black, minimum size=8pt, label=right:{\scriptsize شکست اقتصادی}] at (5,5.5) {};
\node[circle, fill=black, minimum size=8pt, label=above:{\scriptsize بحران آب}] at (7.5,5.5) {};
\node[circle, fill=black, minimum size=8pt, label=right:{\scriptsize فساد}] at (7,4.5) {};
\node[circle, fill=black, minimum size=8pt, label=below:{\scriptsize مداخله خارجی}] at (5,4.5) {};
\node[circle, fill=black, minimum size=8pt, label=right:{\scriptsize پوپولیسم}] at (6,3) {};

% راهنما
\node[fill=green!30, minimum width=0.5cm, minimum height=0.3cm] at (8.5,7.5) {};
\node[right, font=\tiny] at (8.8,7.5) {کم‌خطر};
\node[fill=yellow!30, minimum width=0.5cm, minimum height=0.3cm] at (8.5,7) {};
\node[right, font=\tiny] at (8.8,7) {هشدار};
\node[fill=orange!30, minimum width=0.5cm, minimum height=0.3cm] at (8.5,6.5) {};
\node[right, font=\tiny] at (8.8,6.5) {جدی};
\node[fill=red!30, minimum width=0.5cm, minimum height=0.3cm] at (8.5,6) {};
\node[right, font=\tiny] at (8.8,6) {بحرانی};
\end{tikzpicture}
\captionof{figure}{نقشه حرارتی ریسک‌های گذار دموکراتیک}
\label{fig:risk-heatmap}
\end{center}

\subsection{سیستم هشدار زودهنگام}

\begin{table}[htbp]
\centering
\caption{شاخص‌های هشدار زودهنگام}
\label{tab:early-warning}
\begin{tabular}{>{\columncolor{orange!8}}l p{4.5cm} p{4.5cm}}
\toprule
\rowcolor{orange!25}
\textbf{حوزه} & \textbf{شاخص‌های هشدار} & \textbf{آستانه بحران} \\
\midrule
سیاسی & کاهش مشارکت، افزایش اعتراضات & مشارکت زیر ۴۰٪، اعتراض ۱۰۰K+ \\
\rowcolor{gray!10}
اقتصادی & تورم، بیکاری، فرار سرمایه & تورم بالای ۳۰٪، بیکاری بالای ۱۵٪ \\
قومی & تنش‌های قومی، خشونت محلی & بیش از ۱۰ حادثه/ماه \\
\rowcolor{gray!10}
اجتماعی & رضایت، اعتماد به نهادها & رضایت زیر ۳۰٪ \\
امنیتی & جرم، تروریسم، تنش مرزی & افزایش ۵۰٪+ \\
\rowcolor{gray!10}
رسانه‌ای & سانسور، تهدید خبرنگاران & بازگشت محدودیت‌ها \\
\bottomrule
\end{tabular}
\end{table}

\begin{enghelabbox}
\textbf{پروتکل پاسخ به هشدار}

\textbf{سطح زرد} (هشدار اولیه):
\begin{itemize}[nosep]
\item تشکیل کارگروه بررسی
\item گزارش به مقامات ارشد
\item تقویت پایش
\end{itemize}

\textbf{سطح نارنجی} (تنش جدی):
\begin{itemize}[nosep]
\item جلسه اضطراری کابینه
\item اقدام اصلاحی فوری
\item ارتباط عمومی شفاف
\end{itemize}

\textbf{سطح قرمز} (بحران):
\begin{itemize}[nosep]
\item فعال‌سازی پروتکل مدیریت بحران
\item گفتگوی ملی/مذاکره
\item درخواست کمک بین‌المللی در صورت نیاز
\end{itemize}
\end{enghelabbox}

%═══════════════════════════════════════════════════════════════════════════════
\section{ساختار نهادی پایش}
\label{sec:institutional-framework}
%═══════════════════════════════════════════════════════════════════════════════

\begin{center}
\begin{tikzpicture}[
    node distance=1.5cm,
    institution/.style={
        rectangle,
        rounded corners=5pt,
        draw=#1!70,
        fill=#1!15,
        thick,
        minimum width=4cm,
        minimum height=1.3cm,
        align=center,
        font=\small
    }
]
% نهادها
\node[institution=blue] (stats) at (0,4) {\textbf{مرکز ملی آمار}\\ جمع‌آوری داده، پایگاه‌ها};
\node[institution=green] (eval) at (0,2) {\textbf{سازمان ملی ارزیابی}\\ ارزیابی مستقل، گزارش‌دهی};
\node[institution=orange] (audit) at (0,0) {\textbf{دیوان محاسبات}\\ ممیزی مالی و عملکردی};
\node[institution=purple] (ombuds) at (0,-2) {\textbf{آمبودزمان ملی}\\ شکایات مردمی، نظارت};

% نهادهای ناظر بیرونی
\node[institution=teal, right=2.5cm of stats] (civil) {\textbf{جامعه مدنی}\\ رصدخانه‌های مردمی};
\node[institution=red, right=2.5cm of eval] (media) {\textbf{رسانه‌های مستقل}\\ روزنامه‌نگاری تحقیقی};
\node[institution=cyan, right=2.5cm of audit] (intl) {\textbf{نهادهای بین‌المللی}\\ UN, WB, IMF, EU};
\node[institution=darkyellow, right=2.5cm of ombuds] (acad) {\textbf{دانشگاه‌ها}\\ تحقیق و تحلیل};

% کادر دولتی و مستقل
\draw[dashed, blue!50, thick, rounded corners] (-2.5,-3) rectangle (2.5,5);
\node[blue!70, font=\scriptsize] at (0,5.3) {نهادهای رسمی};

\draw[dashed, green!50, thick, rounded corners] (4,-3) rectangle (9,5);
\node[green!70, font=\scriptsize] at (6.5,5.3) {نظارت مستقل};

% اتصالات
\draw[thick, <->] (stats) -- (civil);
\draw[thick, <->] (eval) -- (media);
\draw[thick, <->] (audit) -- (intl);
\draw[thick, <->] (ombuds) -- (acad);
\end{tikzpicture}
\captionof{figure}{ساختار نهادی پایش و ارزیابی}
\label{fig:monitoring-institutions}
\end{center}

\begin{table}[htbp]
\centering
\caption{وظایف نهادهای پایش و ارزیابی}
\label{tab:monitoring-institutions}
\begin{tabular}{>{\columncolor{blue!8}}l p{4.5cm} p{4.5cm}}
\toprule
\rowcolor{blue!25}
\textbf{نهاد} & \textbf{وظایف اصلی} & \textbf{ویژگی کلیدی} \\
\midrule
مرکز ملی آمار & جمع‌آوری داده، سرشماری، نظرسنجی & استقلال، متدولوژی استاندارد \\
\rowcolor{gray!10}
سازمان ملی ارزیابی & ارزیابی سیاست‌ها و برنامه‌ها & مستقل از دولت، گزارش به مجلس \\
دیوان محاسبات & ممیزی مالی، ارزیابی عملکرد & گزارش علنی، قدرت پیگیری \\
\rowcolor{gray!10}
آمبودزمان ملی & شکایات مردمی، نظارت بر حقوق & دسترسی آسان، قدرت تحقیق \\
کمیته پایش مجلس & نظارت بر قوه مجریه & جلسات علنی، استیضاح \\
\bottomrule
\end{tabular}
\end{table}

%═══════════════════════════════════════════════════════════════════════════════
\section{پایش مشارکتی: صدای مردم}
\label{sec:participatory-monitoring}
%═══════════════════════════════════════════════════════════════════════════════

\begin{naghlbox}
«دموکراسی فقط رأی دادن هر چند سال نیست. شهروندان باید بتوانند روزانه بر عملکرد دولت نظارت کنند و صدایشان شنیده شود.»
\sourceline{نویسنده}
\end{naghlbox}

\subsection{ابزارهای پایش مردمی}

\begin{table}[htbp]
\centering
\caption{ابزارهای پایش مشارکتی}
\label{tab:participatory-tools}
\begin{tabular}{>{\columncolor{green!8}}r p{3.5cm} p{5.5cm}}
\toprule
\rowcolor{green!25}
\textbf{ابزار} & \textbf{توضیح} & \textbf{کاربرد} \\
\midrule
نظرسنجی ملی & پیمایش سالانه ۵۰,۰۰۰ نفری & سنجش رضایت، اولویت‌ها \\
\rowcolor{gray!10}
اپلیکیشن شهروندی & گزارش مشکلات، امتیازدهی & فیدبک لحظه‌ای \\
کارت امتیاز اجتماعی & ارزیابی خدمات محلی توسط مردم & پاسخگویی محلی \\
\rowcolor{gray!10}
بودجه مشارکتی & تصمیم‌گیری مردم درباره بخشی از بودجه & مشارکت مستقیم \\
جلسات عمومی & حضور مقامات برای پاسخگویی & تعامل مستقیم \\
\rowcolor{gray!10}
رصدخانه‌های مدنی & NGOهای ناظر بر حوزه‌های خاص & نظارت تخصصی \\
پلتفرم شفافیت & دسترسی به اسناد و قراردادهای دولتی & مبارزه با فساد \\
\bottomrule
\end{tabular}
\end{table}

\begin{olgoobox}
\textbf{الگوی موفق: کره جنوبی — e-People}

سامانه e-People کره جنوبی نمونه‌ای از پایش مشارکتی است:
\begin{itemize}[nosep]
\item سامانه آنلاین ثبت شکایات و پیشنهادات شهروندی
\item الزام دولت به پاسخ در ۱۴ روز
\item پیگیری آنلاین توسط شهروند
\item انتشار آمار شکایات و نحوه رسیدگی
\item سالانه ۲ میلیون+ شکایت و پیشنهاد
\item \textbf{نتیجه}: افزایش پاسخگویی، کاهش فساد
\end{itemize}
\end{olgoobox}

\subsection{نظرسنجی ملی سالانه}

\begin{table}[htbp]
\centering
\caption{محتوای نظرسنجی ملی سالانه}
\label{tab:annual-survey}
\begin{tabular}{>{\columncolor{purple!8}}r p{4cm} p{5cm}}
\toprule
\rowcolor{purple!25}
\textbf{بخش} & \textbf{موضوعات} & \textbf{سؤالات نمونه} \\
\midrule
رضایت کلی & زندگی، آینده، کشور & آیا کشور در مسیر درست است؟ \\
\rowcolor{gray!10}
اعتماد به نهادها & دولت، مجلس، قضا، ارتش & چقدر به مجلس اعتماد دارید؟ \\
کیفیت خدمات & بهداشت، آموزش، حمل‌ونقل & کیفیت مدارس دولتی چطور است؟ \\
\rowcolor{gray!10}
آزادی‌ها & بیان، تجمع، رسانه & آیا می‌توانید آزادانه نظر دهید؟ \\
امنیت & جانی، مالی، اجتماعی & چقدر احساس امنیت می‌کنید؟ \\
\rowcolor{gray!10}
اقتصاد & وضعیت مالی، اشتغال & وضعیت مالی‌تان بهتر شده؟ \\
اقوام و هویت & حقوق قومی، تعلق & آیا به حقوق قومی‌تان احترام گذاشته می‌شود؟ \\
\rowcolor{gray!10}
محیط زیست & آب، هوا، آینده & وضعیت محیط زیست بهتر شده؟ \\
\bottomrule
\end{tabular}
\end{table}

%═══════════════════════════════════════════════════════════════════════════════
\section{مقایسه بین‌المللی (بنچ‌مارکینگ)}
\label{sec:benchmarking}
%═══════════════════════════════════════════════════════════════════════════════

\subsection{شاخص‌های بین‌المللی مرجع}

\begin{table}[htbp]
\centering
\caption{شاخص‌های بین‌المللی مرجع برای پایش}
\label{tab:international-indices}
\small
\begin{tabular}{>{\columncolor{cyan!8}}l l c c c}
\toprule
\rowcolor{cyan!25}
\textbf{شاخص} & \textbf{سازمان} & \textbf{رتبه فعلی ایران} & \textbf{هدف س۱۰} & \textbf{هدف س۱۵} \\
\midrule
شاخص دموکراسی & EIU & ۱۵۴ از ۱۶۷ & ۸۰ & ۵۰ \\
\rowcolor{gray!10}
آزادی جهان & Freedom House & Not Free & Partly Free & Free \\
شاخص فساد (CPI) & Transparency Intl & ۱۴۹ از ۱۸۰ & ۷۰ & ۴۵ \\
\rowcolor{gray!10}
حاکمیت قانون & WJP & ۱۱۸ از ۱۴۲ & ۶۰ & ۴۰ \\
آزادی مطبوعات & RSF & ۱۷۶ از ۱۸۰ & ۸۰ & ۴۰ \\
\rowcolor{gray!10}
توسعه انسانی (HDI) & UNDP & ۷۶ از ۱۹۱ & ۵۵ & ۴۰ \\
شکاف جنسیتی & WEF & ۱۴۳ از ۱۴۶ & ۸۰ & ۵۰ \\
\rowcolor{gray!10}
کسب‌وکار & World Bank & ۱۲۷ از ۱۹۰& ۳۵ & ۲۵ \\
شاخص صلح جهانی (GPI) & IEP & ۱۴۰ از ۱۶۳ & ۶۰ & ۴۰ \\
\rowcolor{gray!10}
عملکرد محیط زیستی (EPI) & Yale & ۱۲۰ از ۱۸۰ & ۶۰ & ۴۰ \\
نوآوری جهانی (GII) & WIPO & ۶۵ از ۱۳۲ & ۴۰ & ۲۵ \\
\bottomrule
\end{tabular}
\end{table}

\subsection{کشورهای مرجع برای مقایسه}

\begin{table}[htbp]
\centering
\caption{کشورهای مرجع برای بنچ‌مارکینگ}
\label{tab:benchmark-countries}
\begin{tabular}{>{\columncolor{teal!8}}l p{3.5cm} p{5.5cm}}
\toprule
\rowcolor{teal!25}
\textbf{دسته} & \textbf{کشورها} & \textbf{دلیل انتخاب} \\
\midrule
گذار موفق & کره جنوبی، تایوان، اسپانیا، شیلی & الگوی گذار از اقتدارگرایی \\
\rowcolor{gray!10}
مشابه منطقه‌ای & ترکیه، مصر، عربستان & مقایسه منطقه‌ای \\
مشابه اقتصادی & مالزی، مکزیک، آرژانتین & سطح درآمد مشابه \\
\rowcolor{gray!10}
چندقومی موفق & سوئیس، کانادا، هند & مدیریت تنوع \\
پیشرو محیط زیست & کاستاریکا، دانمارک & الگوی سبز \\
\rowcolor{gray!10}
هدف بلندمدت & پرتغال، چک، استونی & جایی که می‌خواهیم برسیم \\
\bottomrule
\end{tabular}
\end{table}

%═══════════════════════════════════════════════════════════════════════════════
\section{گزارش‌دهی و شفافیت}
\label{sec:reporting}
%═══════════════════════════════════════════════════════════════════════════════

\subsection{انواع گزارش‌ها}

\begin{table}[htbp]
\centering
\caption{نظام گزارش‌دهی پیشرفت گذار}
\label{tab:reporting-system}
\begin{tabular}{>{\columncolor{orange!8}}l c p{4cm} p{3cm}}
\toprule
\rowcolor{orange!25}
\textbf{نوع گزارش} & \textbf{دوره} & \textbf{محتوا} & \textbf{مخاطب} \\
\midrule
داشبورد آنلاین & لحظه‌ای & شاخص‌های کلیدی & عموم مردم \\
\rowcolor{gray!10}
خبرنامه ماهانه & ماهانه & خلاصه پیشرفت‌ها & عموم، رسانه‌ها \\
گزارش فصلی & فصلی & تحلیل عملکرد & مجلس، نخبگان \\
\rowcolor{gray!10}
گزارش سالانه & سالانه & ارزیابی جامع & همه ذی‌نفعان \\
گزارش تفصیلی بخشی & سالانه & هر حوزه جداگانه & متخصصان \\
\rowcolor{gray!10}
گزارش به مردم & سالانه & زبان ساده، تصویری & عموم مردم \\
\bottomrule
\end{tabular}
\end{table}

\subsection{اصول گزارش‌دهی شفاف}

\begin{olgoobox}
\textbf{هفت اصل گزارش‌دهی خوب}

\begin{enumerate}[nosep]
\item \textbf{صداقت}: هم موفقیت‌ها، هم شکست‌ها
\item \textbf{به‌موقع}: داده‌های تازه، نه کهنه
\item \textbf{قابل فهم}: زبان ساده برای عموم
\item \textbf{قابل دسترس}: آنلاین، رایگان، چندزبانه
\item \textbf{قابل مقایسه}: با گذشته و با دیگران
\item \textbf{قابل راستی‌آزمایی}: داده‌های خام قابل دانلود
\item \textbf{پاسخگو}: توضیح انحرافات و اقدامات اصلاحی
\end{enumerate}
\end{olgoobox}

%═══════════════════════════════════════════════════════════════════════════════
\section{مکانیزم اصلاح مسیر}
\label{sec:course-correction}
%═══════════════════════════════════════════════════════════════════════════════

\begin{naghlbox}
«هیچ برنامه‌ای از برخورد با واقعیت سالم بیرون نمی‌آید. مهم این نیست که برنامه اولیه بی‌نقص باشد؛ مهم این است که توانایی یادگیری و اصلاح داشته باشیم.»
\sourceline{دوایت آیزنهاور}
\end{naghlbox}

\subsection{فرآیند اصلاح مسیر}

\begin{center}
\begin{tikzpicture}[
    node distance=1.8cm,
    step/.style={
        rectangle,
        rounded corners=8pt,
        draw=blue!70,
        fill=blue!15,
        thick,
        minimum width=3cm,
        minimum height=1.2cm,
        align=center,
        font=\small
    },
    arrow/.style={->, thick, >=stealth, blue!60}
]
% مراحل
\node[step] (detect) {\textbf{۱. شناسایی}\\ انحراف از هدف};
\node[step, right=1.5cm of detect] (analyze) {\textbf{۲. تحلیل}\\ علت‌یابی};
\node[step, right=1.5cm of analyze] (design) {\textbf{۳. طراحی}\\ راه‌حل};
\node[step, below=1.5cm of design] (decide) {\textbf{۴. تصمیم}\\ تصویب اصلاح};
\node[step, left=1.5cm of decide] (implement) {\textbf{۵. اجرا}\\ پیاده‌سازی};
\node[step, left=1.5cm of implement] (monitor) {\textbf{۶. پایش}\\ ارزیابی اثر};

% فلش‌ها
\draw[arrow] (detect) -- (analyze);
\draw[arrow] (analyze) -- (design);
\draw[arrow] (design) -- (decide);
\draw[arrow] (decide) -- (implement);
\draw[arrow] (implement) -- (monitor);
\draw[arrow, dashed] (monitor) -- (detect);

% چرخه
\node[font=\scriptsize, gray] at (1.5,-2.5) {چرخه مستمر بهبود};
\end{tikzpicture}
\captionof{figure}{فرآیند اصلاح مسیر}
\label{fig:course-correction}
\end{center}

\subsection{سطوح اصلاح}

\begin{table}[htbp]
\centering
\caption{سطوح مختلف اصلاح مسیر}
\label{tab:correction-levels}
\begin{tabular}{>{\columncolor{purple!8}}l p{3.5cm} p{3.5cm} p{2.5cm}}
\toprule
\rowcolor{purple!25}
\textbf{سطح} & \textbf{نوع تغییر} & \textbf{مرجع تصمیم} & \textbf{زمان} \\
\midrule
عملیاتی & تنظیم فعالیت‌ها & مدیران اجرایی & فوری \\
\rowcolor{gray!10}
تاکتیکی & اصلاح برنامه‌ها & وزرا و معاونان & ۱-۳ ماه \\
استراتژیک & تغییر اولویت‌ها & کابینه & ۳-۶ ماه \\
\rowcolor{gray!10}
ساختاری & تغییر نهادها/قوانین & مجلس & ۶-۱۲ ماه \\
بنیادین & بازنگری اهداف کلان & مجلس + همه‌پرسی & ۱-۲ سال \\
\bottomrule
\end{tabular}
\end{table}

%═══════════════════════════════════════════════════════════════════════════════
\section{ارزیابی تأثیر (Impact Evaluation)}
\label{sec:impact-evaluation}
%═══════════════════════════════════════════════════════════════════════════════

\subsection{روش‌های ارزیابی تأثیر}

\begin{table}[htbp]
\centering
\caption{روش‌های ارزیابی تأثیر سیاست‌ها}
\label{tab:impact-methods}
\begin{tabular}{>{\columncolor{green!8}}l p{5cm} p{4cm}}
\toprule
\rowcolor{green!25}
\textbf{روش} & \textbf{توضیح} & \textbf{کاربرد} \\
\midrule
آزمایش تصادفی (RCT) & مقایسه گروه آزمایش و کنترل & برنامه‌های اجتماعی \\
\rowcolor{gray!10}
تفاوت در تفاوت & مقایسه قبل/بعد در دو گروه & سیاست‌های جدید \\
رگرسیون ناپیوستگی & استفاده از آستانه‌های سیاستی & برنامه‌های هدفمند \\
\rowcolor{gray!10}
تطبیق (Matching) & مقایسه موارد مشابه & وقتی RCT ممکن نیست \\
مطالعه موردی کیفی & تحلیل عمیق چند مورد & فهم مکانیزم‌ها \\
\rowcolor{gray!10}
نظرسنجی قبل/بعد & سنجش تغییر در نگرش‌ها & ارزیابی فرهنگی \\
\bottomrule
\end{tabular}
\end{table}

\subsection{سؤالات کلیدی ارزیابی تأثیر}

\begin{itemize}[nosep]
\item آیا زندگی مردم واقعاً بهتر شده است؟
\item کدام گروه‌ها بیشتر/کمتر بهره‌مند شده‌اند؟
\item آیا تغییرات پایدار خواهند بود؟
\item چه عوارض ناخواسته‌ای ایجاد شده؟
\item آیا با همین هزینه، نتیجه بهتری ممکن بود؟
\end{itemize}

%═══════════════════════════════════════════════════════════════════════════════
\section{یادگیری سازمانی}
\label{sec:organizational-learning}
%═══════════════════════════════════════════════════════════════════════════════

\begin{table}[htbp]
\centering
\caption{مکانیزم‌های یادگیری سازمانی}
\label{tab:learning-mechanisms}
\begin{tabular}{>{\columncolor{cyan!8}}r p{4.5cm} p{4.5cm}}
\toprule
\rowcolor{cyan!25}
\textbf{مکانیزم} & \textbf{توضیح} & \textbf{خروجی} \\
\midrule
جلسات درس‌آموخته & بازنگری پس از هر پروژه بزرگ & گزارش درس‌آموخته‌ها \\
\rowcolor{gray!10}
پایگاه دانش & ذخیره تجربیات و بهترین شیوه‌ها & ویکی داخلی، راهنماها \\
آموزش مستمر & دوره‌های توانمندسازی کارکنان & کارکنان ماهرتر \\
\rowcolor{gray!10}
تبادل بین‌المللی & یادگیری از کشورهای دیگر & بهترین شیوه‌های جهانی \\
تحقیق و توسعه سیاستی & آزمایش پایلوت قبل از گسترش & سیاست‌های اثبات‌شده \\
\rowcolor{gray!10}
بازخورد مردمی & گوش‌دادن به شکایات و پیشنهادات & بهبود خدمات \\
\bottomrule
\end{tabular}
\end{table}

%═══════════════════════════════════════════════════════════════════════════════
\section{چارچوب پاسخگویی}
\label{sec:accountability}
%═══════════════════════════════════════════════════════════════════════════════

\begin{center}
\begin{tikzpicture}[
    node distance=2cm,
    actor/.style={
        ellipse,
        draw=#1!70,
        fill=#1!20,
        thick,
        minimum width=2.5cm,
        minimum height=1.5cm,
        align=center,
        font=\small
    }
]
% مرکز - دولت
\node[actor=blue, minimum width=3cm] (gov) {\textbf{دولت}\\قوه مجریه};

% حلقه‌های پاسخگویی
\node[actor=green, above=1.5cm of gov] (parl) {\textbf{مجلس}\\نظارت قانونی};
\node[actor=purple, above right=1cm of gov] (court) {\textbf{قضا}\\نظارت قضایی};
\node[actor=orange, right=2cm of gov] (audit) {\textbf{دیوان محاسبات}\\ممیزی};
\node[actor=red, below right=1cm of gov] (media) {\textbf{رسانه}\\افکار عمومی};
\node[actor=teal, below=1.5cm of gov] (civil) {\textbf{جامعه مدنی}\\نظارت مردمی};
\node[actor=darkyellow, below left=1cm of gov] (intl) {\textbf{بین‌المللی}\\استانداردها};
\node[actor=pink, left=2cm of gov] (party) {\textbf{احزاب}\\رقابت سیاسی};
\node[actor=cyan, above left=1cm of gov] (ombuds) {\textbf{آمبودزمان}\\شکایات};

% فلش‌ها
\foreach \n in {parl, court, audit, media, civil, intl, party, ombuds} {
    \draw[thick, <->, gray!60] (gov) -- (\n);
}

% عنوان
\node[above=3cm of gov, font=\large\bfseries] {اکوسیستم پاسخگویی};
\end{tikzpicture}
\captionof{figure}{چارچوب چندلایه پاسخگویی}
\label{fig:accountability-framework}
\end{center}

\begin{table}[htbp]
\centering
\caption{ابزارهای پاسخگویی}
\label{tab:accountability-tools}
\begin{tabular}{>{\columncolor{blue!8}}l p{4cm} p{5cm}}
\toprule
\rowcolor{blue!25}
\textbf{نوع پاسخگویی} & \textbf{ابزار} & \textbf{ضمانت اجرا} \\
\midrule
قانونی & استیضاح، رأی عدم اعتماد & برکناری \\
\rowcolor{gray!10}
قضایی & محاکمه، جریمه & مجازات قانونی \\
مالی & ممیزی، شفافیت بودجه & جریمه، استرداد \\
\rowcolor{gray!10}
انتخاباتی & رأی مردم & عدم انتخاب مجدد \\
اجتماعی & افکار عمومی، رسانه & فشار سیاسی \\
\rowcolor{gray!10}
حرفه‌ای & استانداردهای اخلاقی & تعلیق، اخراج \\
\bottomrule
\end{tabular}
\end{table}

%═══════════════════════════════════════════════════════════════════════════════
\section{تقویم پایش و ارزیابی}
\label{sec:me-timeline}
%═══════════════════════════════════════════════════════════════════════════════

\begin{table}[htbp]
\centering
\caption{تقویم فعالیت‌های پایش و ارزیابی}
\label{tab:me-timeline}
\begin{tabular}{>{\columncolor{orange!8}}l p{5cm} p{4cm}}
\toprule
\rowcolor{orange!25}
\textbf{زمان} & \textbf{فعالیت} & \textbf{مسئول} \\
\midrule
ماه ۱ & راه‌اندازی داشبورد ملی (نسخه اولیه) & مرکز آمار \\
\rowcolor{gray!10}
ماه ۶ & اولین گزارش شش‌ماهه & دولت \\
سال ۱ & نظرسنجی ملی پایه (Baseline) & سازمان ارزیابی \\
\rowcolor{gray!10}
سال ۱ & تأسیس سازمان ملی ارزیابی & مجلس \\
سال ۲ & ارزیابی فاز گذار (فاز ۱) & پانل مستقل \\
\rowcolor{gray!10}
سال ۳ & اولین گزارش جامع به مردم & دولت + NGOها \\
سال ۵ & ارزیابی میان‌دوره‌ای بزرگ & پانل داخلی-خارجی \\
\rowcolor{gray!10}
سال ۵ & ارزیابی فاز نهادسازی (فاز ۲) & پانل مستقل \\
سال ۱۰ & ارزیابی دهه اول گذار & کمیته ملی + بین‌المللی \\
\rowcolor{gray!10}
سال ۱۵ & ارزیابی تکمیلی گذار & کمیته ملی \\
سال ۲۵ & ارزیابی نهایی و جشن موفقیت & ملی و بین‌المللی \\
\bottomrule
\end{tabular}
\end{table}

%═══════════════════════════════════════════════════════════════════════════════
\section{جمع‌بندی: پایش برای موفقیت}
\label{sec:monitoring-conclusion}
%═══════════════════════════════════════════════════════════════════════════════

\begin{olgoobox}
\textbf{پیام کلیدی فصل}

سیستم پایش و ارزیابی:
\begin{itemize}[nosep]
\item \textbf{چشم برنامه است}: بدون آن، کور حرکت می‌کنیم
\item \textbf{ابزار پاسخگویی است}: مردم باید بدانند چه می‌شود
\item \textbf{موتور یادگیری است}: از اشتباهات درس می‌گیریم
\item \textbf{مبنای اصلاح است}: بدون داده، اصلاح ممکن نیست
\item \textbf{پیش‌شرط شفافیت است}: آنچه پنهان است، فاسد می‌شود
\item \textbf{مشارکتی است}: مردم هم ناظرند، هم ارزیاب
\end{itemize}

\textbf{شعار}: «آنچه اندازه‌گیری نشود، مدیریت نمی‌شود — آنچه شفاف نباشد، فاسد می‌شود»
\end{olgoobox}

\begin{naghlbox}
«در نهایت، موفقیت گذار دموکراتیک با یک معیار ساده سنجیده می‌شود: آیا زندگی مردم عادی بهتر شده است؟ آیا آزادترند؟ آیا امنیت بیشتری دارند؟ آیا به آینده امیدوارترند؟ اگر پاسخ مثبت است، در مسیر درست هستیم.»
\sourceline{نویسنده}
\end{naghlbox}

%═══════════════════════════════════════════════════════════════════════════════
% منابع فصل
%═══════════════════════════════════════════════════════════════════════════════

\section*{منابع فصل پانزدهم}
\addcontentsline{toc}{section}{منابع فصل پانزدهم}

\begin{itemize}[nosep, font=\small]
\item Kusek, J. Z., \& Rist, R. C. (2004). \textit{Ten Steps to a Results-Based Monitoring and Evaluation System}. World Bank.
\item Gertler, P. J. et al. (2016). \textit{Impact Evaluation in Practice}. World Bank.
\item OECD. (2019). \textit{Governance at a Glance}. Paris.
\item World Bank. (2017). \textit{World Development Report: Governance and the Law}. Washington, DC.
\item Transparency International. (2023). \textit{Corruption Perceptions Index Methodology}.
\item Freedom House. (2023). \textit{Freedom in the World Methodology}.
\item Economist Intelligence Unit. (2023). \textit{Democracy Index Methodology}.
\item World Justice Project. (2023). \textit{Rule of Law Index}.
\item UNDP. (2022). \textit{Human Development Report}.
\item Institute for Economics and Peace. (2023). \textit{Global Peace Index}.
\item Reporters Without Borders. (2023). \textit{World Press Freedom Index}.
\item World Economic Forum. (2023). \textit{Global Gender Gap Report}.
\item Drucker, P. F. (1993). \textit{Management: Tasks, Responsibilities, Practices}. Harper Business.
\item Senge, P. M. (1990). \textit{The Fifth Discipline: The Art and Practice of the Learning Organization}. Doubleday.
\end{itemize}
	
	%══════════════════════════════════════════════════════════════════════════════
	%                              پیوست‌ها
	%══════════════════════════════════════════════════════════════════════════════
	\appendix
	\part{پیوست‌ها}
	
	% ═══════════════════════════════════════════════════════════════════════════════
% پیوست ۱: متن قانون اساسی پیشنهادی جمهوری فدرال ایران
% فایل: app01-constitution.tex
% ═══════════════════════════════════════════════════════════════════════════════

\chapter{متن قانون اساسی پیشنهادی}
\label{app:constitution}

\begin{kholasebox}
این پیوست متن کامل پیش‌نویس قانون اساسی جمهوری فدرال ایران را ارائه می‌دهد. این سند حقوقی بنیادین، حاصل تلفیق تجارب موفق جهانی با واقعیت‌های تاریخی-فرهنگی ایران است. ساختار این قانون اساسی بر سه ستون استوار است: \textbf{دموکراسی مشارکتی}، \textbf{فدرالیسم همبسته}، و \textbf{حقوق بنیادین تضمین‌شده}. متن شامل ۱۸۷ اصل در هفت فصل به همراه مقررات انتقالی است.
\end{kholasebox}

% ═══════════════════════════════════════════════════════════════════════════════
\section{دیباچه قانون اساسی}
% ═══════════════════════════════════════════════════════════════════════════════

\begin{center}
\begin{tikzpicture}
    \node[
        draw=bleurepublique,
        line width=3pt,
        fill=bleulight,
        rounded corners=15pt,
        inner sep=20pt,
        text width=12cm,
        align=center
    ] (preamble) {
        {\Huge\rl{\textbf{قانون اساسی}}}\\[0.4cm]
        {\LARGE\rl{\textbf{جمهوری فدرال ایران}}}\\[0.6cm]
        {\large \rl{مصوب مجلس مؤسسان | سال ۱۴۰۵ هجری شمسی}}
    };
    \node[above=0.2cm of preamble, bleurepublique] {\Large \rl{★ ★ ★}};
\end{tikzpicture}
\end{center}

\vspace{10pt}

\begin{naghlbox}
\textbf{دیباچه}

ما، ملت ایران،

وارثان تمدنی کهن با پیشینه‌ای درخشان در تاریخ بشری؛

مردمانی متکثر از تبارها، زبان‌ها، فرهنگ‌ها و باورهای گوناگون که در طول هزاره‌ها در این سرزمین در کنار یکدیگر زیسته‌ایم؛

با یادآوری رنج‌های مشترک از استبداد، تبعیض و بی‌عدالتی؛

با درس‌آموزی از تجربه‌های تلخ و شیرین تاریخ معاصر؛

با ایمان به کرامت ذاتی انسان و برابری همه شهروندان؛

با تعهد به آزادی، عدالت، دموکراسی و حاکمیت قانون؛

با احترام به تنوع فرهنگی به‌مثابه ثروت ملی؛

با مسئولیت در برابر نسل‌های آینده و محیط زیست؛

با امید به ساختن آینده‌ای بهتر برای همه فرزندان این سرزمین؛

\textbf{این قانون اساسی را به‌عنوان میثاق ملی خود تصویب و اعلام می‌کنیم.}
\sourceline{مجلس مؤسسان جمهوری فدرال ایران}
\end{naghlbox}

% ═══════════════════════════════════════════════════════════════════════════════
\section{فصل اول: اصول کلی و هویت ملی}
\label{sec:const-ch1}
% ═══════════════════════════════════════════════════════════════════════════════

\subsection{بخش یکم: ماهیت و شکل حکومت}

\begin{center}
\begin{tikzpicture}[
    scale=0.9,
    transform shape,
    box/.style={
        rectangle,
        rounded corners=8pt,
        draw=bleurepublique,
        fill=bleulight,
        line width=1.5pt,
        minimum width=5.5cm,
        minimum height=1.2cm,
        text centered,
        font=\small\bfseries
    },
    content/.style={
        rectangle,
        rounded corners=8pt,
        draw=goldphoenix!50,
        fill=goldlight,
        line width=1pt,
        minimum width=5.5cm,
        minimum height=1.2cm,
        text centered,
        font=\small
    },
    arrow/.style={->, ultra thick, bleurepublique!40, >=stealth}
]
\node[box] (l1) at (-3.5, 4.5) {\rl{اصل ۱: ماهیت حکومت}};
\node[content] (r1) at (3.5, 4.5) {\rl{جمهوری فدرال دموکراتیک}};

\node[box] (l2) at (-3.5, 3.0) {\rl{اصل ۲: منشأ حاکمیت}};
\node[content] (r2) at (3.5, 3.0) {\rl{حاکمیت ملی متعلق به مردم}};

\node[box] (l3) at (-3.5, 1.5) {\rl{اصل ۳: وحدت ملی}};
\node[content] (r3) at (3.5, 1.5) {\rl{تمامیت ارضی و فدرالیسم}};

\node[box] (l4) at (-3.5, 0) {\rl{اصل ۴: تفکیک قوا}};
\node[content] (r4) at (3.5, 0) {\rl{استقلال قوای سه‌گانه}};

\node[box] (l5) at (-3.5, -1.5) {\rl{اصل ۵: حاکمیت قانون}};
\node[content] (r5) at (3.5, -1.5) {\rl{برتری قانون اساسی}};

\foreach \i in {1,...,5} {
    \draw[arrow] (l\i) -- (r\i);
}
\end{tikzpicture}
\end{center}

\begin{table}[htbp]
\centering
\caption{اصول منتخب فصل اول: اصول کلی و هویت ملی}
\label{tab:const-ch1}
\begin{tabularx}{\textwidth}{R{2.5cm} Y}
\toprule
\headmark شماره اصل & \headmark متن و محتوای اصل \\
\midrule
\textbf{اصل ۱} & \textbf{ماهیت حکومت:} ایران کشوری است با نظام جمهوری فدرال دموکراتیک که بر پایه حاکمیت ملی، حقوق بشر، و حکومت قانون بنیان نهاده شده است. \\
\rowcolor{goldlight}
\textbf{اصل ۲} & \textbf{منشأ حاکمیت:} حاکمیت ملی به‌طور انحصاری متعلق به مردم ایران است. هیچ قدرتی برتر از خواست دموکراتیک مردم نیست. \\
\textbf{اصل ۳} & \textbf{وحدت ملی:} تمامیت ارضی و همبستگی ملی از طریق ساختار فدرال و احترام به تنوع تضمین می‌شود. \\
\rowcolor{goldlight}
\textbf{اصل ۶} & \textbf{جدایی دین از دولت:} دولت نسبت به همه باورها بی‌طرف است. هیچ دینی، دین رسمی نیست. \\
\textbf{اصل ۷} & \textbf{زبان‌ها:} فارسی زبان رسمی و مشترک؛ سایر زبان‌ها ملی و در مناطق خود رسمی هستند. \\
\rowcolor{goldlight}
\textbf{اصل ۱۱} & \textbf{اصول غیرقابل تغییر:} دموکراسی، فدرالیسم و لایی‌سیته ستون‌های ابدی نظام هستند. \\
\bottomrule
\end{tabularx}
\end{table}

\subsection{بخش دوم: اصول غیرقابل تغییر}

\begin{enghelabbox}
\textbf{⚠️ اصول غیرقابل تغییر (اصل ۱۱)}

اصول زیر غیرقابل تغییر هستند و هیچ بازنگری در قانون اساسی نمی‌تواند آنها را نقض کند:

\begin{enumerate}[nosep, label=\alph*)]
    \item ماهیت جمهوری و دموکراتیک نظام
    \item حاکمیت ملی و انتخابی بودن حکومت
    \item حقوق بنیادین شهروندان مندرج در فصل دوم
    \item تفکیک و استقلال قوای سه‌گانه
    \item جدایی دین از دولت
    \item حقوق اقوام و ساختار فدرال
    \item استقلال و تمامیت ارضی کشور
\end{enumerate}
\end{enghelabbox}

% ═══════════════════════════════════════════════════════════════════════════════
\section{فصل دوم: حقوق بنیادین شهروندان}
\label{sec:const-ch2}
% ═══════════════════════════════════════════════════════════════════════════════

\begin{center}
\begin{tikzpicture}[
    scale=0.9,
    transform shape,
    centernode/.style={
        circle,
        draw=bleurepublique,
        fill=bleulight,
        line width=2pt,
        minimum size=3.5cm,
        text centered,
        font=\large\bfseries
    },
    leafnode/.style={
        rectangle,
        rounded corners=8pt,
        draw=goldphoenix,
        fill=goldlight,
        line width=1.2pt,
        minimum width=3cm,
        minimum height=1cm,
        text centered,
        font=\small\bfseries
    },
    connector/.style={->, ultra thick, bleurepublique!30, >=stealth}
]
\node[centernode] (center) at (0,0) {\rl{حقوق بنیادین}\\\rl{شهروندان}};

\foreach \ang/\label/\name in {90/حقوق مدنی/civil, 30/حقوق سیاسی/pol, -30/حقوق اقتصادی/econ, -90/حقوق اجتماعی/social, -150/حقوق فرهنگی/cult, 150/حقوق قضایی/jud} {
    \node[leafnode] (\name) at (\ang:4.8) {\rl{\label}};
    \draw[connector] (center) -- (\name);
}
\end{tikzpicture}
\end{center}

\subsection{بخش یکم: حقوق مدنی و آزادی‌های فردی (اصول ۱۲-۳۵)}

\begin{longtable}{|>{\columncolor{bleulight}}r|p{12cm}|}
\hline
\rowcolor{bleurepublique!20}
\headmark شماره اصل & \headmark متن کامل اصل (حقوق مدنی و آزادی‌های فردی) \\
\hline
\endfirsthead
\hline
\rowcolor{bleurepublique!20}
\headmark شماره اصل & \headmark متن کامل اصل \\
\hline
\endhead

\textbf{اصل ۱۲} &
\textbf{کرامت انسانی:} کرامت ذاتی انسان مصون از تعرض است. احترام به این کرامت و حمایت از آن وظیفه همه نهادهای حکومتی است. \\
\hline

\rowcolor{goldlight}
\textbf{اصل ۱۳} &
\textbf{برابری در برابر قانون:} همه شهروندان در برابر قانون برابرند. هرگونه تبعیض بر اساس جنسیت، قومیت، نژاد، زبان، دین، مذهب، باور سیاسی، منشأ اجتماعی، وضعیت اقتصادی، معلولیت، سن، یا هر وضعیت دیگر ممنوع است. \\
\hline

\textbf{اصل ۱۴} &
\textbf{حق حیات:} حق حیات هر انسان از تولد تا مرگ طبیعی محترم و تضمین‌شده است. مجازات اعدام ملغی است. \\
\hline

\rowcolor{goldlight}
\textbf{اصل ۱۵} &
\textbf{ممنوعیت شکنجه:} شکنجه، رفتار غیرانسانی، تحقیرآمیز یا بی‌رحمانه در هر شرایطی مطلقاً ممنوع است. هرگونه اعتراف یا مدرکی که از طریق شکنجه به دست آمده باشد فاقد اعتبار قانونی است. \\
\hline

\textbf{اصل ۱۶} &
\textbf{آزادی فردی و امنیت شخصی:}
الف) هیچ‌کس را نمی‌توان بازداشت کرد مگر به موجب حکم مقام قضایی صالح و طبق قانون.
ب) هر بازداشت‌شده باید ظرف ۲۴ ساعت به دادگاه معرفی شود.
ج) هر کس حق دارد فوراً از دلایل بازداشت خود مطلع شود و به وکیل دسترسی داشته باشد.
د) بازداشت موقت استثنایی است و مدت آن نمی‌تواند از حدود قانونی تجاوز کند. \\
\hline

\rowcolor{goldlight}
\textbf{اصل ۱۷} &
\textbf{حریم خصوصی:} حریم خصوصی افراد، مسکن، مکاتبات، ارتباطات و داده‌های شخصی مصون از تعرض است. تفتیش و نظارت تنها با حکم قضایی و در چارچوب قانون مجاز است. \\
\hline

\textbf{اصل ۱۸} &
\textbf{آزادی رفت‌وآمد:} هر شهروند حق دارد آزادانه در سراسر کشور رفت‌وآمد کند، محل سکونت خود را انتخاب نماید، از کشور خارج و به آن بازگردد. سلب یا محدودیت این حق تنها به موجب قانون و با حکم قضایی ممکن است. \\
\hline

\rowcolor{goldlight}
\textbf{اصل ۱۹} &
\textbf{آزادی عقیده و بیان:} 
الف) هر کس حق دارد آزادانه عقاید خود را داشته و ابراز کند.
ب) آزادی بیان، نوشتار، هنر و رسانه تضمین می‌شود.
ج) سانسور ممنوع است. محدودیت آزادی بیان تنها برای حمایت از حقوق دیگران، امنیت ملی یا نظم عمومی و به موجب قانون مجاز است.
د) تحریک به خشونت، نفرت‌پراکنی و انکار جنایات علیه بشریت ممنوع است. \\
\hline

\textbf{اصل ۲۰} &
\textbf{آزادی دین و وجدان:}
الف) هر کس حق دارد دین، مذهب یا باور خود را آزادانه انتخاب کند یا بدون دین باشد.
ب) هر کس حق دارد دین خود را به‌صورت فردی یا جمعی، در خلوت یا علن، عبادت و آموزش دهد.
ج) هیچ‌کس را نمی‌توان به پذیرش یا ترک دینی مجبور کرد.
د) تغییر دین حق هر فرد است و مجازات ندارد. \\
\hline

\rowcolor{goldlight}
\textbf{اصل ۲۱} &
\textbf{آزادی تجمع و تشکل:}
الف) شهروندان حق برگزاری تجمعات و راهپیمایی‌های مسالمت‌آمیز بدون سلاح را دارند.
ب) تجمعات عمومی نیازمند اطلاع‌رسانی قبلی هستند، نه مجوز.
ج) تشکیل احزاب، انجمن‌ها، سندیکاها و سازمان‌های مدنی آزاد است. \\
\hline

\textbf{اصل ۲۲} &
\textbf{حق مالکیت:}
الف) مالکیت خصوصی به رسمیت شناخته شده و حمایت می‌شود.
ب) سلب مالکیت تنها برای مصالح عمومی، با پرداخت غرامت عادلانه و به موجب قانون مجاز است.
ج) مالکیت متضمن مسئولیت اجتماعی است. \\
\hline
\end{longtable}

\subsection{بخش دوم: حقوق سیاسی (اصول ۳۶-۴۵)}

\begin{longtable}{|>{\columncolor{bleulight}}r|p{12cm}|}
\hline
\rowcolor{bleurepublique!20}
\headmark شماره اصل & \headmark متن کامل اصل (حقوق سیاسی) \\
\hline
\endfirsthead
\hline
\rowcolor{bleurepublique!20}
\headmark شماره اصل & \headmark متن کامل اصل \\
\hline
\endhead

\textbf{اصل ۳۶} &
\textbf{حق رأی:} 
الف) هر شهروند ایرانی که به سن ۱۸ سال رسیده باشد حق رأی دارد.
ب) رأی‌گیری آزاد، برابر، مستقیم و مخفی است.
ج) شرکت در انتخابات حق شهروندی است، نه تکلیف. \\
\hline

\rowcolor{goldlight}
\textbf{اصل ۳۷} &
\textbf{حق انتخاب شدن:}
الف) هر شهروندی که واجد شرایط قانونی باشد حق نامزد شدن برای مناصب انتخابی را دارد.
ب) هیچ نهادی حق رد صلاحیت بر اساس معیارهای غیرقانونی یا سیاسی را ندارد.
ج) احراز صلاحیت تنها بر اساس معیارهای عینی قانونی (سن، تابعیت، سابقه کیفری) و با حق اعتراض قضایی انجام می‌شود. \\
\hline

\textbf{اصل ۳۸} &
\textbf{آزادی احزاب:}
الف) تشکیل و فعالیت احزاب سیاسی آزاد است.
ب) احزاب باید در ساختار و عملکرد خود دموکراتیک باشند.
ج) انحلال حزب تنها به حکم دیوان عالی قانون اساسی و به دلیل نقض اصول دموکراتیک ممکن است. \\
\hline

\rowcolor{goldlight}
\textbf{اصل ۳۹} &
\textbf{حق همه‌پرسی:}
الف) شهروندان می‌توانند از طریق همه‌پرسی در تصمیمات ملی مشارکت کنند.
ب) ابتکار عمل مردمی: یک میلیون امضا می‌تواند برگزاری همه‌پرسی را الزامی کند.
ج) همه‌پرسی نمی‌تواند ناقض حقوق بنیادین یا اصول غیرقابل تغییر باشد. \\
\hline

\textbf{اصل ۴۰} &
\textbf{حق دادخواهی:} هر شهروند حق دارد از نهادهای دولتی دادخواهی کند و پاسخ مکتوب دریافت نماید. \\
\hline

\rowcolor{goldlight}
\textbf{اصل ۴۱} &
\textbf{دسترسی به اطلاعات:}
الف) هر شهروند حق دسترسی به اطلاعات و اسناد دولتی را دارد.
ب) طبقه‌بندی اسناد باید محدود، موجه و زمان‌دار باشد.
ج) افشاگری فساد (سوت‌زنی) تحت حمایت قانون است. \\
\hline

\textbf{اصل ۴۲} &
\textbf{حق اعتراض و نافرمانی مدنی:}
الف) اعتراض مسالمت‌آمیز حق شهروندی است.
ب) نافرمانی مدنی در برابر قوانین ناقض حقوق بنیادین مشروع است.
ج) حق مقاومت در برابر کودتا یا تلاش برای براندازی نظام دموکراتیک به رسمیت شناخته می‌شود. \\
\hline
\end{longtable}

\subsection{بخش سوم: حقوق اقتصادی و اجتماعی (اصول ۴۶-۶۵)}

\begin{longtable}{|>{\columncolor{bleulight}}r|p{12cm}|}
\hline
\rowcolor{bleurepublique!20}
\headmark شماره اصل & \headmark متن کامل اصل (حقوق اقتصادی، اجتماعی و محیط زیست) \\
\hline
\endfirsthead
\hline
\rowcolor{bleurepublique!20}
\headmark شماره اصل & \headmark متن کامل اصل \\
\hline
\endhead

\textbf{اصل ۴۶} &
\textbf{حق کار:} 
الف) هر شهروند حق کار، انتخاب آزادانه شغل، و شرایط کار عادلانه را دارد.
ب) کار اجباری ممنوع است.
ج) دولت موظف به تلاش برای ایجاد اشتغال کامل است. \\
\hline

\rowcolor{goldlight}
\textbf{اصل ۴۷} &
\textbf{حقوق کارگران:}
الف) حق تشکیل و عضویت در سندیکا و اتحادیه‌های کارگری تضمین می‌شود.
ب) حق اعتصاب به رسمیت شناخته می‌شود.
ج) دستمزد باید عادلانه و تضمین‌کننده زندگی با کرامت باشد.
د) حداقل دستمزد به موجب قانون تعیین و سالانه بازنگری می‌شود. \\
\hline

\textbf{اصل ۴۸} &
\textbf{حق تأمین اجتماعی:}
الف) هر شهروند حق بهره‌مندی از تأمین اجتماعی را دارد.
ب) بیمه بیکاری، بازنشستگی، ازکارافتادگی و بیمه درمانی همگانی تضمین می‌شود.
ج) حمایت ویژه از اقشار آسیب‌پذیر وظیفه دولت است. \\
\hline

\rowcolor{goldlight}
\textbf{اصل ۴۹} &
\textbf{حق بهداشت و درمان:}
الف) هر شهروند حق دسترسی به خدمات بهداشتی و درمانی را دارد.
ب) خدمات بهداشت اولیه رایگان است.
ج) بیمه درمانی پایه همگانی و اجباری است. \\
\hline

\textbf{اصل ۵۰} &
\textbf{حق مسکن:} دولت موظف است زمینه دسترسی همه شهروندان به مسکن مناسب را فراهم سازد. بی‌خانمانی باید ریشه‌کن شود. \\
\hline

\rowcolor{goldlight}
\textbf{اصل ۵۱} &
\textbf{حق آموزش:}
الف) آموزش رایگان و اجباری تا پایان دوره متوسطه است.
ب) آموزش عالی باید در دسترس همگان بر اساس شایستگی باشد.
ج) دولت موظف به حمایت از آموزش مادام‌العمر است. \\
\hline

\textbf{اصل ۵۲} &
\textbf{حق آب:}
الف) دسترسی به آب سالم و کافی حق بنیادین هر شهروند است.
ب) دولت موظف به تأمین آب شرب سالم برای همه است.
ج) منابع آب ثروت ملی و متعلق به نسل‌های حال و آینده است. \\
\hline

\rowcolor{goldlight}
\textbf{اصل ۵۳} &
\textbf{حقوق محیط زیست:}
الف) هر شهروند حق زندگی در محیط زیست سالم را دارد.
ب) حمایت از محیط زیست وظیفه دولت و حق و تکلیف همه شهروندان است.
ج) آلوده‌سازی محیط زیست جرم است. \\
\end{longtable}

\subsection{بخش چهارم: حقوق فرهنگی و اقوام (اصول ۶۶-۷۵)}

\begin{longtable}{|>{\columncolor{bleulight}}r|p{12cm}|}
\hline
\rowcolor{bleurepublique!20}
\headmark شماره اصل & \headmark متن کامل اصل (حقوق فرهنگی و اقوام) \\
\hline
\endfirsthead
\hline
\rowcolor{bleurepublique!20}
\headmark شماره اصل & \headmark متن کامل اصل \\
\hline
\endhead

\textbf{اصل ۶۶} &
\textbf{تنوع فرهنگی:} تنوع فرهنگی، قومی، زبانی و مذهبی ایران ثروت ملی است. دولت موظف به حمایت و حفاظت از این تنوع است. \\
\hline

\rowcolor{goldlight}
\textbf{اصل ۶۷} &
\textbf{حقوق زبانی:} 
الف) هر شهروند حق استفاده از زبان مادری خود در عرصه‌های عمومی و خصوصی را دارد.
ب) آموزش به زبان مادری در کنار زبان فارسی حق هر کودک است.
ج) خدمات عمومی در مناطق دوزبانه به هر دو زبان ارائه می‌شود. \\
\hline

\textbf{اصل ۶۸} &
\textbf{حقوق قومی:}
الف) هیچ قومی بر قوم دیگر برتری ندارد.
ب) هر قوم حق حفظ و توسعه هویت، فرهنگ، زبان و سنن خود را دارد.
ج) مشارکت عادلانه اقوام در قدرت سیاسی و اقتصادی تضمین می‌شود. \\
\hline

\rowcolor{goldlight}
\textbf{اصل ۶۹} &
\textbf{میراث فرهنگی:} حفاظت از میراث فرهنگی، تاریخی و باستانی ایران وظیفه دولت است. غارت، تخریب یا قاچاق آثار تاریخی جرم سنگین است. \\
\hline

\textbf{اصل ۷۰} &
\textbf{آزادی علم و هنر:} تحقیق علمی و خلاقیت هنری آزاد است. دولت موظف به حمایت از علم، فناوری و هنر است. \\
\hline
\end{longtable}

\subsection{بخش پنجم: حقوق قضایی (اصول ۷۶-۸۵)}

\begin{longtable}{|>{\columncolor{bleulight}}r|p{12cm}|}
\hline
\rowcolor{bleurepublique!20}
\headmark شماره اصل & \headmark متن کامل اصل (حقوق قضایی و دادرسی منصفانه) \\
\hline
\endfirsthead
\hline
\rowcolor{bleurepublique!20}
\headmark شماره اصل & \headmark متن کامل اصل \\
\hline
\endhead

\textbf{اصل ۷۶} &
\textbf{اصل برائت:} هر کس بی‌گناه است تا زمانی که جرم او در دادگاه صالح به اثبات برسد. بار اثبات بر عهده دادستان است. \\
\hline

\rowcolor{goldlight}
\textbf{اصل ۷۷} &
\textbf{حق محاکمه عادلانه:}
الف) هر کس حق دارد در مدت معقول توسط دادگاه مستقل و بی‌طرف محاکمه شود.
ب) محاکمات علنی است مگر در موارد استثنایی قانونی.
ج) هر کس حق دارد از اتهامات علیه خود مطلع شود و فرصت کافی برای دفاع داشته باشد. \\
\hline

\textbf{اصل ۷۸} &
\textbf{حق وکیل:}
الف) هر کس در تمام مراحل رسیدگی حق داشتن وکیل را دارد.
ب) اگر متهم توانایی مالی نداشته باشد، دولت وکیل تسخیری تعیین می‌کند.
ج) ارتباط بین وکیل و موکل محرمانه است. \\
\hline

\rowcolor{goldlight}
\textbf{اصل ۷۹} &
\textbf{اصل قانونی بودن جرم و مجازات:}
الف) هیچ عملی جرم نیست مگر به موجب قانون.
ب) هیچ مجازاتی اعمال نمی‌شود مگر به موجب قانون.
ج) قوانین کیفری عطف به ماسبق نمی‌شوند مگر به نفع متهم. \\
\hline

\textbf{اصل ۸۰} &
\textbf{ممنوعیت محاکمه مجدد:} هیچ‌کس را نمی‌توان برای جرمی که قبلاً محاکمه و تبرئه یا محکوم شده دوباره محاکمه کرد. \\
\hline

\rowcolor{goldlight}
\textbf{اصل ۸۱} &
\textbf{حق تجدیدنظر:} هر محکوم حق درخواست تجدیدنظر در حکم را دارد. \\
\hline

\textbf{اصل ۸۲} &
\textbf{حق جبران خسارت:} هر کس که به‌ناحق بازداشت یا محکوم شده باشد حق جبران خسارت مادی و معنوی را دارد. \\
\hline

\rowcolor{goldlight}
\textbf{اصل ۸۳} &
\textbf{اصلاح مجرمان:} هدف نظام کیفری اصلاح و بازپروری است، نه انتقام. شرایط زندان‌ها باید انسانی و اصلاحی باشد. \\
\hline
\end{longtable}

% ═══════════════════════════════════════════════════════════════════════════════
\section{فصل سوم: ساختار حکومت}
\label{sec:const-ch3}
% ═══════════════════════════════════════════════════════════════════════════════

\begin{figure}[htbp]
\centering
\begin{tikzpicture}[
    scale=0.85,
    transform shape,
    branch/.style={
        rectangle,
        rounded corners=10pt,
        draw=bleurepublique,
        fill=bleulight,
        line width=2pt,
        minimum width=4.5cm,
        minimum height=2.5cm,
        text centered,
        font=\bfseries
    },
    sub/.style={
        rectangle,
        rounded corners=5pt,
        draw=goldphoenix!50,
        fill=goldlight,
        line width=1pt,
        minimum width=3cm,
        minimum height=0.8cm,
        text centered,
        font=\tiny\bfseries
    },
    arrow/.style={->, ultra thick, bleurepublique!20, >=stealth}
]
\node[branch] (leg) at (-5.5, 3) {\rl{قوه مقننه}\\\rl{\tiny اصول ۸۶-۱۱۰}};
\node[branch] (exe) at (0, 3) {\rl{قوه مجریه}\\\rl{\tiny اصول ۱۱۱-۱۳۵}};
\node[branch] (jud) at (5.5, 3) {\rl{قوه قضائیه}\\\rl{\tiny اصول ۱۳۶-۱۵۵}};

\node[sub] (mn) at (-6.8, 0.5) {\rl{مجلس ملی}};
\node[sub] (ma) at (-3.8, 0.5) {\rl{مجلس اقوام}};
\node[sub] (pr) at (-0.8, 0.5) {\rl{رئیس‌جمهور}};
\node[sub] (cab) at (2.2, 0.5) {\rl{کابینه}};
\node[sub] (sc) at (4.2, 0.5) {\rl{دیوان عالی}};
\node[sub] (cc) at (7.2, 0.5) {\rl{دیوان قانون اساسی}};

\foreach \p/\c1/\c2 in {leg/mn/ma, exe/pr/cab, jud/sc/cc} {
    \draw[arrow] (\p) -- (\c1);
    \draw[arrow] (\p) -- (\c2);
}

\node[
    draw=goldphoenix,
    fill=goldlight,
    line width=2pt,
    rounded corners=12pt,
    minimum width=12cm,
    minimum height=1.5cm,
    font=\large\bfseries
] at (0, 6.5) {\rl{ساختار کلان حاکمیت در جمهوری فدرال ایران}};

\end{tikzpicture}
\caption{نمودار کلان تفکیک قوا و نهادهای بنیادین در میثاق ملی نوین}
\label{fig:gov-structure}
\end{figure}

\subsection{بخش یکم: قوه مقننه (اصول ۸۶-۱۱۰)}

\subsubsection{الف) ساختار دومجلسی}

\begin{longtable}{|>{\columncolor{bleulight}}r|p{12cm}|}
\hline
\rowcolor{bleurepublique!20}
\headmark شماره اصل & \headmark متن کامل اصل (ساختار قوه مقننه) \\
\hline
\endfirsthead
\hline
\rowcolor{bleurepublique!20}
\headmark شماره اصل & \headmark متن کامل اصل \\
\hline
\endhead

\textbf{اصل ۸۶} &
\textbf{پارلمان فدرال:} قوه مقننه متشکل از دو مجلس است:
الف) \textbf{مجلس ملی} (مجلس اول): نمایندگی مستقیم مردم؛
ب) \textbf{مجلس اقوام} (مجلس دوم): نمایندگی مناطق و اقوام.
هر دو مجلس در تصویب قوانین مشارکت دارند. \\
\hline

\rowcolor{goldlight}
\textbf{اصل ۸۷} &
\textbf{مجلس ملی:}
الف) مجلس ملی متشکل از ۳۵۰ نماینده است که با رأی مستقیم و مخفی مردم انتخاب می‌شوند.
ب) انتخابات به روش نسبی-فهرستی با آستانه ۳٪ برگزار می‌شود.
ج) دوره نمایندگی چهار سال است.
د) حداقل ۳۰٪ کاندیداها باید زن باشند. \\
\hline

\textbf{اصل ۸۸} &
\textbf{مجلس اقوام:}
الف) مجلس اقوام متشکل از ۱۲۰ نماینده است:
- ۶۰ نماینده: ۴ نماینده از هر استان (انتخابی)
- ۳۰ نماینده: نمایندگان مناطق خودمختار
- ۲۰ نماینده: نمایندگان اقلیت‌های زبانی-فرهنگی
- ۱۰ نماینده: نمایندگان ایرانیان خارج از کشور
ب) دوره نمایندگی شش سال است و هر دو سال یک‌سوم اعضا تجدید می‌شوند.
ج) رئیس مجلس اقوام معاون اول رئیس‌جمهور در تشریفات است. \\
\hline

\rowcolor{goldlight}
\textbf{اصل ۸۹} &
\textbf{شرایط نمایندگی:}
الف) تابعیت ایرانی؛
ب) سن حداقل ۲۵ سال برای مجلس ملی و ۳۵ سال برای مجلس اقوام؛
ج) برخورداری از حقوق مدنی؛
د) نداشتن محکومیت کیفری مؤثر؛
هـ) حداقل مدرک کارشناسی. \\
\hline

\textbf{اصل ۹۰} &
\textbf{مصونیت پارلمانی:}
الف) نمایندگان به خاطر اظهارات و آرایشان در مجلس قابل تعقیب نیستند.
ب) بازداشت نماینده در حال انجام وظیفه مستلزم اجازه مجلس است، مگر در جرم مشهود.
ج) مصونیت شامل جرایم عادی خارج از وظایف نمایندگی نمی‌شود. \\
\hline

\rowcolor{goldlight}
\textbf{اصل ۹۱} &
\textbf{تعارض منافع:}
الف) نمایندگان نمی‌توانند همزمان شغل دولتی داشته باشند.
ب) نمایندگان موظف به اعلام دارایی‌ها و منافع مالی خود هستند.
ج) رأی‌دادن در موضوعاتی که نماینده ذی‌نفع است ممنوع است. \\
\hline
\end{longtable}

\subsubsection{ب) فرآیند قانون‌گذاری}

\begin{figure}[htbp]
\centering
\begin{tikzpicture}[
    scale=0.9,
    transform shape,
    step/.style={
        rectangle,
        rounded corners=8pt,
        draw=bleurepublique,
        fill=bleulight,
        line width=1.5pt,
        minimum width=2.8cm,
        minimum height=1.2cm,
        text centered,
        font=\small\bfseries
    },
    arrow/.style={->, ultra thick, bleurepublique!30, >=stealth}
]
\node[step] (s1) at (0,0) {\rl{پیشنهاد لایحه/طرح}};
\node[step] (s2) at (3.5,0) {\rl{کمیسیون تخصصی}};
\node[step] (s3) at (7,0) {\rl{بررسی مجلس ملی}};
\node[step] (s4) at (10.5,0) {\rl{تصویب مجلس ملی}};

\node[step] (s5) at (10.5,-2.5) {\rl{بررسی مجلس اقوام}};
\node[step] (s8) at (7,-2.5) {\rl{تأیید رئیس‌جمهور}};
\node[step, draw=goldphoenix, fill=goldlight] (s9) at (3.5,-2.5) {\rl{انتشار و اجرا}};

\draw[arrow] (s1) -- (s2);
\draw[arrow] (s2) -- (s3);
\draw[arrow] (s3) -- (s4);
\draw[arrow] (s4) -- (s5);
\draw[arrow] (s5) -- (s8);
\draw[arrow] (s8) -- (s9);

\node[
    draw=goldphoenix,
    fill=goldlight,
    rounded corners=5pt,
    minimum width=4cm,
    font=\tiny\bfseries
] at (0,-2.5) {\rl{کمیته حل اختلاف در صورت تعارض}};
\end{tikzpicture}
\caption{فرآیند دموکراتیک قانون‌گذاری و تعادل قوا در ساختار دومجلسی}
\label{fig:legislative-flow}
\end{figure}

\begin{longtable}{|>{\columncolor{bleulight}}r|p{12cm}|}
\hline
\rowcolor{bleurepublique!20}
\headmark شماره اصل & \headmark متن کامل اصل (فرآیند قانون‌گذاری و نظارت) \\
\hline
\endfirsthead
\hline
\rowcolor{bleurepublique!20}
\headmark شماره اصل & \headmark متن کامل اصل \\
\hline
\endhead

\textbf{اصل ۹۲} & \textbf{ابتکار قانون‌گذاری:} حق پیشنهاد قانون متعلق به هیئت وزیران (لایحه)، ۱۵ نماینده مجلس ملی (طرح)، ۱۰ نماینده مجلس اقوام (طرح)، و ۵۰۰,۰۰۰ شهروند (ابتکار مردمی) است. \\
\hline

\rowcolor{goldlight}
\textbf{اصل ۹۳} & \textbf{تصویب قوانین عادی:} الف) قوانین عادی با اکثریت ساده هر دو مجلس تصویب می‌شوند. ب) در صورت اختلاف، کمیته مشترک تشکیل می‌شود. ج) اگر توافق حاصل نشد، مجلس ملی با اکثریت مطلق تصمیم نهایی را می‌گیرد. \\
\hline

\textbf{اصل ۹۴} & \textbf{قوانین اساسی و بنیادین:} قوانینی که به حقوق اقوام، ساختار فدرال، یا حقوق بنیادین مربوط می‌شوند، نیازمند تصویب دوسوم هر دو مجلس هستند. \\
\hline

\rowcolor{goldlight}
\textbf{اصل ۹۵} & \textbf{حق وتوی مجلس اقوام:} مجلس اقوام در موضوعات فدرال، تقسیمات کشوری، حقوق اقوام، توزیع منابع ملی و انتصابات عالی قضایی حق وتوی مطلق دارد. \\
\hline

\textbf{اصل ۹۶} & \textbf{وتوی رئیس‌جمهور:} الف) رئیس‌جمهور حق بازگرداندن قانون ظرف ۱۵ روز را دارد. ب) مجلس با دوسوم آرا می‌تواند وتو را رد کند. ج) امضا و ابلاغ نهایی ظرف ۱۰ روز الزامی است. \\
\hline

\rowcolor{goldlight}
\textbf{اصل ۹۷} & \textbf{بودجه سالانه:} دولت موظف به تقدیم لایحه بودجه تا پایان آبان است. مجلس ملی تصویب‌کننده و مجلس اقوام ناظر بر توزیع عادلانه بین مناطق است. \\
\hline

\textbf{اصل ۹۸} & \textbf{نظارت پارلمانی:} الف) وزرا در برابر مجلس ملی مسئول‌اند. ب) حق سؤال و استیضاح برای نمایندگان محفوظ است. ج) رأی عدم اعتماد با اکثریت مطلق نمایندگان ممکن است. \\
\hline
\end{longtable}

\subsection{بخش دوم: قوه مجریه (اصول ۱۱۱-۱۳۵)}

\subsubsection{الف) رئیس‌جمهور}

\begin{figure}[htbp]
\centering
\begin{tikzpicture}[
    scale=0.9,
    transform shape,
    box/.style={
        rectangle,
        rounded corners=12pt,
        draw=bleurepublique,
        fill=bleulight,
        line width=2pt,
        minimum width=12cm,
        minimum height=5cm
    },
    label/.style={
        rectangle,
        fill=bleurepublique,
        text=white,
        rounded corners=5pt,
        font=\large\bfseries,
        minimum width=6cm
    }
]
\node[box] (main) at (0,0) {};
\node[label] at (0,2.5) {\rl{رئیس‌جمهور جمهوری فدرال ایران}};

\node[align=right, text width=5.5cm, font=\small] at (-3,0.5) {
    \textbf{\rl{انتخاب:}} \rl{رأی مستقیم مردم}\\[3pt]
    \textbf{\rl{دوره:}} \rl{۴ سال (حداکثر ۲ دوره)}\\[3pt]
    \textbf{\rl{شرایط سن:}} \rl{حداقل ۴۰ سال}
};

\node[align=right, text width=5.5cm, font=\small] at (3,0.5) {
    \textbf{\rl{نقش:}} \rl{رئیس دولت و کشور}\\[3pt]
    \textbf{\rl{مسئولیت:}} \rl{در برابر ملت و پارلمان}\\[3pt]
    \textbf{\rl{اختیار:}} \rl{فرماندهی کل قوا}
};

\node[
    draw=goldphoenix,
    fill=goldlight,
    rounded corners=8pt,
    minimum width=10cm,
    minimum height=1.2cm,
    text centered,
    font=\small\bfseries,
    text=bleurepublique
] at (0,-1.5) {\rl{وظایف: اجرای قانون اساسی | سیاست خارجی | انتصاب وزرا | امضای قوانین}};

\end{tikzpicture}
\caption{جایگاه، صلاحیت‌ها و فرآیند پاسخگویی نهاد ریاست جمهوری}
\label{fig:president-role}
\end{figure}

\begin{longtable}{|>{\columncolor{bleulight}}r|p{12cm}|}
\hline
\rowcolor{bleurepublique!20}
\headmark شماره اصل & \headmark متن کامل اصل (نهاد ریاست جمهوری) \\
\hline
\endfirsthead
\hline
\rowcolor{bleurepublique!20}
\headmark شماره اصل & \headmark متن کامل اصل \\
\hline
\endhead

\textbf{اصل ۱۱۱} & \textbf{جایگاه رئیس‌جمهور:} رئیس‌جمهور بالاترین مقام رسمی کشور پس از ملت، رئیس قوه مجریه، رئیس دولت و نماینده عالی جمهوری در روابط بین‌المللی است. \\
\hline

\rowcolor{goldlight}
\textbf{اصل ۱۱۲} & \textbf{انتخاب:} الف) انتخاب با رأی مستقیم و مخفی. ب) کسب اکثریت مطلق (۵۰٪ + ۱) آرا الزامی است. ج) در غیر این صورت، دور دوم بین دو نفر اول برگزار می‌شود. \\
\hline

\textbf{اصل ۱۱۳} & \textbf{شرایط:} تابعیت ایرانی، حداقل ۴۰ سال سن، مدرک کارشناسی ارشد، و فقدان محکومیت کیفری. حمایت ۱۰۰ هزار شهروند یا ۵۰ نماینده برای کاندیداتوری لازم است. \\
\hline

\rowcolor{goldlight}
\textbf{اصل ۱۱۶} & \textbf{وظایف:} تعیین سیاست‌های دولت، تشکیل هیئت وزیران، فرماندهی کل قوا، امضای معاهدات و ابلاغ قوانین. \\
\hline

\textbf{اصل ۱۱۸} & \textbf{استیضاح:} یک‌سوم نمایندگان مجلس ملی می‌توانند طرح استیضاح را مطرح کنند. رأی عدم اعتماد نیازمند دوسوم آرای مجلس ملی و تأیید مجلس اقوام است. \\
\hline
\end{longtable}

\subsubsection{ب) هیئت وزیران}

\begin{longtable}{|>{\columncolor{bleulight}}r|p{12cm}|}
\hline
\rowcolor{bleurepublique!20}
\headmark شماره اصل & \headmark متن کامل اصل (هیئت وزیران) \\
\hline
\endfirsthead
\hline
\rowcolor{bleurepublique!20}
\headmark شماره اصل & \headmark متن کامل اصل \\
\hline
\endhead

\textbf{اصل ۱۲۰} & \textbf{ترکیب:} الف) رئیس‌جمهور، معاونان و وزرا. ب) حداکثر ۲۰ وزارتخانه. ج) رعایت تنوع جغرافیایی، قومی و جنسیتی (حداقل ۳۰٪ زن) الزامی است. \\
\hline

\rowcolor{goldlight}
\textbf{اصل ۱۲۱} & \textbf{رأی اعتماد:} وزرای پیشنهادی باید رأی اعتماد اکثریت نمایندگان مجلس ملی را کسب کنند. در صورت سه بار رد کلیات کابینه، انتخابات زودهنگام برگزار می‌شود. \\
\hline

\textbf{اصل ۱۲۲} & \textbf{وظایف:} تدوین سیاست‌های دولت، تهیه لوایح و بودجه سالانه، تصویب آیین‌نامه‌ها و تأمین امنیت ملی. \\
\hline

\rowcolor{goldlight}
\textbf{اصل ۱۲۴} & \textbf{تعارض منافع:} ممنوعیت شغل دوم برای وزرا، اعلام کامل دارایی‌ها در ابتدا و انتها، و محدودیت ۲ ساله برای ورود به بخش خصوصی مرتبط پس از خدمت. \\
\hline
\end{longtable}

\subsubsection{ج) نیروهای مسلح}

\begin{longtable}{|>{\columncolor{bleulight}}r|p{12cm}|}
\hline
\rowcolor{bleurepublique!20}
\headmark شماره اصل & \headmark متن کامل اصل (نیروهای مسلح) \\
\hline
\endfirsthead
\hline
\rowcolor{bleurepublique!20}
\headmark شماره اصل & \headmark متن کامل اصل \\
\hline
\endhead

\textbf{اصل ۱۲۵} & \textbf{وظیفه نیروهای مسلح:} دفاع از استقلال و تمامیت ارضی، حمایت از نظام قانون اساسی و نظم دموکراتیک، و کمک در بلایای طبیعی. نیروهای مسلح ابزار سیاست تهاجمی نیستند. \\
\hline

\rowcolor{goldlight}
\textbf{اصل ۱۲۶} & \textbf{فرماندهی:} رئیس‌جمهور فرمانده کل است. اعلام جنگ و صلح و اعزام نیرو به خارج مستلزم تصویب دوسوم مجلس ملی است. \\
\hline

\textbf{اصل ۱۲۷} & \textbf{غیرسیاسی بودن:} ممنوعیت فعالیت حزبی و دخالت در امور سیاسی. نظامیان در دوران خدمت حق کاندیداتوری ندارند. نظارت پارلمانی بر ارتش الزامی است. \\
\hline

\rowcolor{goldlight}
\textbf{اصل ۱۲۸} & \textbf{بودجه دفاعی:} بودجه نظامی باید به تصویب مجلس برسد و کمیسیون دفاع بر هزینه‌ها نظارت دقیق دارد. \\
\hline

\textbf{اصل ۱۲۹} & \textbf{ممنوعیت کودتا:} هرگونه اقدام نظامی برای تغییر نظام قانون اساسی کودتا و جرم سنگین است. حق مقاومت شهروندی در برابر کودتا به رسمیت شناخته می‌شود. \\
\hline
\end{longtable}

\subsection{بخش سوم: قوه قضائیه (اصول ۱۳۶-۱۵۵)}

\begin{figure}[htbp]
\centering
\begin{tikzpicture}[
    scale=0.9,
    transform shape,
    court/.style={
        rectangle,
        rounded corners=8pt,
        draw=bleurepublique,
        fill=bleulight,
        line width=1.5pt,
        minimum width=4cm,
        minimum height=1.5cm,
        text centered,
        font=\small\bfseries
    },
    arrow/.style={->, ultra thick, bleurepublique!30, >=stealth}
]
\node[court, draw=goldphoenix, fill=goldlight] (cc) at (0,2.5) {\rl{دیوان عالی قانون اساسی}\\\rl{\tiny ۱۵ قاضی}};
\node[court] (sc) at (-4,0.5) {\rl{دیوان عالی کشور}\\\rl{\tiny رئیس + ۲۴ قاضی}};
\node[court] (ac) at (4,0.5) {\rl{دیوان عدالت اداری}\\\rl{\tiny ۱۲ قاضی}};
\node[court, minimum height=1cm] (app) at (-4,-1.5) {\rl{دادگاه‌های استیناف}};
\node[court, minimum height=1cm] (first) at (4,-1.5) {\rl{دادگاه‌های بدوی}};
\node[court, draw=gray, fill=gray!10] (spec) at (0,-1.5) {\rl{دادگاه‌های تخصصی}};

\foreach \s/\t in {cc/sc, cc/ac, sc/app, app/first, spec/sc} {
    \draw[arrow] (\s) -- (\t);
}

\node[
    draw=goldphoenix,
    fill=goldlight,
    rounded corners=5pt,
    minimum width=6cm,
    minimum height=1cm,
    font=\bfseries
] at (0,-3.5) {\rl{شورای عالی قضایی (نهاد راهبری)}};
\end{tikzpicture}
\caption{ساختار سلسله‌مراتبی قوه قضائیه در نظام فدرال دموکراتیک}
\label{fig:jud-structure}
\end{figure}

\begin{longtable}{|>{\columncolor{bleulight}}r|p{12cm}|}
\hline
\rowcolor{bleurepublique!20}
\headmark شماره اصل & \headmark متن کامل اصل (قوه قضائیه و دیوان قانون اساسی) \\
\hline
\endfirsthead
\hline
\rowcolor{bleurepublique!20}
\headmark شماره اصل & \headmark متن کامل اصل \\
\hline
\endhead

\textbf{اصل ۱۳۶} & \textbf{استقلال قضایی:} قوه قضائیه مستقل است. قضات در صدور رأی تنها تابع قانون هستند. بودجه قضایی مستقل از قوه مجریه و توسط پارلمان تأمین می‌شود. \\
\hline

\rowcolor{goldlight}
\textbf{اصل ۱۳۷} & \textbf{شورای عالی قضایی:} بالاترین نهاد راهبری قضایی متشکل از رئیس دیوان عالی، دادستان کل، قضات منتخب و حقوقدانان. وظیفه انتصاب، ترفیع و نظارت بر قضات را دارد. \\
\hline

\textbf{اصل ۱۳۸} & \textbf{انتصاب قضات:} قضات توسط شورا منصوب می‌شوند. قضات دیوان عالی و دیوان قانون اساسی نیازمند تأیید مجلس اقوام و مجلس ملی هستند. \\
\hline

\rowcolor{goldlight}
\textbf{اصل ۱۳۹} & \textbf{تصدی قضاوت:} قضات دارای تصدی مادام‌العمر تا ۷۰ سالگی هستند. عزل یا انتقال آنها تنها در موارد استثنایی قانونی و حکم دادگاه انتظامی ممکن است. \\
\hline

\textbf{اصل ۱۴۰} & \textbf{دیوان عالی کشور:} عالی‌ترین مرجع قضایی مدنی و کیفری، ناظر بر حسن اجرای قوانین و ایجاد وحدت رویه قضایی. \\
\hline

\rowcolor{goldlight}
\textbf{اصل ۱۴۱} & \textbf{دیوان عالی قانون اساسی:} پاسدار میثاق ملی و حقوق بنیادین، متشکل از ۱۵ قاضی. صلاحیت تفسیر قانون اساسی و ابطال قوانین مغایر را دارد. \\
\hline

\textbf{اصل ۱۴۲} & \textbf{دسترسی به دیوان:} هر شهروند در صورت نقض حقوق بنیادین حق شکایت به دیوان قانون اساسی را دارد. مراجع سیاسی و دادگاه‌ها نیز حق ارجاع دارند. \\
\hline

\rowcolor{goldlight}
\textbf{اصل ۱۴۳} & \textbf{دیوان عدالت اداری:} مرجع رسیدگی به شکایات شهروندان از تصمیمات و آیین‌نامه‌های دولتی و ابطال اقدامات غیرقانونی. \\
\hline

\textbf{اصل ۱۴۴} & \textbf{دادستانی کل:} مسئول تعقیب جرایم و حفاظت از حقوق عامه. دادستان توسط شورا پیشنهاد و با تأیید مجلس منصوب می‌شود. \\
\hline

\rowcolor{goldlight}
\textbf{اصل ۱۴۵} & \textbf{دادرسی منصفانه:} علنی بودن محاکمات، حق داشتن وکیل، مستدل بودن احکام و حق تجدیدنظرخواهی همگانی. \\
\hline
\end{longtable}

ب) وظایف: نظارت بر حسن اجرای قوانین، ایجاد رویه قضایی واحد، رسیدگی به فرجام‌خواهی

ج) آرای دیوان در ایجاد رویه قضایی برای دادگاه‌های پایین‌تر الزام‌آور است. \\
\hline

\textbf{اصل ۱۴۱} &
\textbf{دیوان عالی قانون اساسی}

الف) دیوان عالی قانون اساسی پاسدار قانون اساسی و حقوق بنیادین است.

ب) ترکیب: ۱۵ قاضی با دوره ۱۲ ساله غیرقابل تمدید (هر ۴ سال، ۵ نفر تجدید)

ج) صلاحیت‌ها:
\begin{itemize}[nosep]
    \item بررسی انطباق قوانین با قانون اساسی
    \item رسیدگی به شکایات نقض حقوق بنیادین
    \item حل اختلاف بین قوا و بین مرکز و مناطق
    \item تفسیر قانون اساسی
    \item نظارت بر انتخابات ملی و همه‌پرسی
\end{itemize} \\
\hline

\textbf{اصل ۱۴۲} &
\textbf{دسترسی به دیوان قانون اساسی}

حق طرح شکایت در دیوان قانون اساسی متعلق است به:

الف) هر شهروند که حقوق بنیادینش نقض شده باشد

ب) رئیس‌جمهور، رئیس هر مجلس

ج) یک‌پنجم نمایندگان هر مجلس

د) دولت‌های محلی (در موضوعات فدرالی)

هـ) دادگاه‌ها هنگام شک در قانون اساسی بودن قانون \\
\hline

\textbf{اصل ۱۴۳} &
\textbf{دیوان عدالت اداری}

الف) دیوان عدالت اداری مرجع رسیدگی به شکایات از دستگاه‌های دولتی است.

ب) هر شهروند حق شکایت از تصمیمات اداری را دارد.

ج) دیوان می‌تواند تصمیمات غیرقانونی را ابطال و حکم به جبران خسارت دهد. \\
\hline

\textbf{اصل ۱۴۴} &
\textbf{دادستانی کل}

الف) دادستان کل مسئول تعقیب جرایم و نظارت بر اجرای قانون است.

ب) دادستان کل توسط شورای عالی قضایی پیشنهاد و با تأیید مجلس منصوب می‌شود.

ج) دادستانی مستقل است و تحت نظارت مجلس قرار دارد. \\
\hline

\textbf{اصل ۱۴۵} &
\textbf{دادرسی منصفانه}

الف) محاکمات باید علنی باشد مگر در موارد استثنایی قانونی.

ب) هر فرد حق داشتن وکیل و فرصت کافی برای دفاع را دارد.

ج) احکام باید مستدل و مکتوب باشند.

د) حق تجدیدنظر در همه احکام تضمین می‌شود. \\
\hline

\end{longtable}

% ═══════════════════════════════════════════════════════════════════════════════
\section{فصل چهارم: مجلس اقوام و ساختار فدرال}
\label{sec:const-ch4}
% ═══════════════════════════════════════════════════════════════════════════════

\begin{olgoobox}
\textbf{فدرالیسم همبسته: الگوی ایرانی}

ساختار فدرال جمهوری فدرال ایران بر اساس اصل «وحدت در کثرت» طراحی شده است. این ساختار:

\begin{itemize}[nosep]
    \item تنوع قومی-فرهنگی را به رسمیت می‌شناسد
    \item خودمختاری منطقه‌ای را تضمین می‌کند
    \item همبستگی و وحدت ملی را حفظ می‌کند
    \item از تجزیه‌طلبی جلوگیری می‌کند
\end{itemize}
\end{olgoobox}

\begin{figure}[htbp]
\centering
\begin{tikzpicture}[
    scale=0.9,
    transform shape,
    level/.style={
        rectangle,
        rounded corners=8pt,
        draw=bleurepublique,
        fill=bleulight,
        line width=1.5pt,
        minimum width=12cm,
        minimum height=1.2cm,
        text centered,
        font=\bfseries
    },
    arrow/.style={<->, ultra thick, bleurepublique!30, >=stealth}
]
\node[level, fill=bleurepublique, text=white] (fed) at (0,3) {\rl{سطح فدرال: حاکمیت ملی، دفاع، سیاست خارجی، پولی}};
\node[level, draw=goldphoenix, fill=goldlight] (reg) at (0,1) {\rl{سطح منطقه‌ای: ۵ منطقه خودمختار + ۱۵ استان}};
\node[level] (prov) at (0,-1) {\rl{سطح استانی: آموزش، بهداشت، قانون‌گذاری محلی}};
\node[level, fill=white] (loc) at (0,-3) {\rl{سطح محلی: شهرداری‌ها و دموکراسی مستقیم شورایی}};

\draw[arrow] (fed) -- (reg);
\draw[arrow] (reg) -- (prov);
\draw[arrow] (prov) -- (loc);
\end{tikzpicture}
\caption{هرم ساختار فدرال جمهوری فدرال ایران از مرکز تا جوامع محلی}
\label{fig:federal-pyramid}
\end{figure}

\begin{longtable}{|>{\columncolor{bleulight}}r|p{12cm}|}
\hline
\rowcolor{bleurepublique!20}
\headmark شماره اصل & \headmark متن کامل اصل (ساختار فدرال و تقسیمات کشوری) \\
\hline
\endfirsthead
\hline
\rowcolor{bleurepublique!20}
\headmark شماره اصل & \headmark متن کامل اصل \\
\hline
\endhead

\textbf{اصل ۱۴۶} & \textbf{تقسیمات کشوری:} ایران متشکل از ۵ منطقه خودمختار (آذربایجان، کردستان، بلوچستان، خوزستان و ترکمن‌صحرا) و ۱۵ استان فدرال است. تغییر در مرزها نیازمند همه‌پرسی محلی و تصویب دوسوم پارلمان است. \\
\hline

\rowcolor{goldlight}
\textbf{اصل ۱۴۷} & \textbf{خودمختاری منطقه‌ای:} الف) هر منطقه دارای پارلمان و دولت منطقه‌ای منتخب است. ب) زبان‌های بومی در کنار زبان فارسی در مناطق مربوطه رسمیت دارند. ج) مناطق در امور فرهنگی، آموزشی و عمران محلی خودمختارند. \\
\hline

\textbf{اصل ۱۴۸} & \textbf{صلاحیت‌های انحصاری فدرال:} حاکمیت ملی، دفاع، سیاست خارجی، سیاست پولی، گمرک، و منابع استراتژیک (نفت و گاز) در صلاحیت انحصاری دولت فدرال است. \\
\hline

\rowcolor{goldlight}
\textbf{اصل ۱۴۹} & \textbf{صلاحیت‌های مشترک:} محیط زیست، بهداشت، آموزش عالی و جنگل‌ها در صلاحیت مشترک مرکز و مناطق است. در صورت تعارض، قوانین فدرال اولویت دارند. \\
\hline

\textbf{اصل ۱۵۰} & \textbf{همبستگی فدرال:} مناطق موظف به رعایت وحدت ملی و تمامیت ارضی هستند. هیچ منطقه‌ای حق جدایی یک‌جانبه را ندارد. \\
\hline

\rowcolor{goldlight}
\textbf{اصل ۱۵۱} & \textbf{توازن مالی (عدالت توزیعی):} دولت فدرال موظف به توزیع عادلانه درآمدهای ملی بین مناطق بر اساس جمعیت و شاخص‌های محرومیت است تا شکاف‌های توسعه‌ای برطرف شود. \\
\hline

\textbf{اصل ۱۵۲} & \textbf{شورای فرمانداران:} برای هماهنگی بین مرکز و مناطق، شورای فرمانداران به ریاست رئیس‌جمهور تشکیل می‌شود. اختلافات بین‌منطقه‌ای توسط دیوان قانون اساسی حل می‌شود. \\
\hline
\end{longtable}

% ═══════════════════════════════════════════════════════════════════════════════
\section{فصل پنجم: نهادهای مستقل نظارتی}
\label{sec:const-ch5}
% ═══════════════════════════════════════════════════════════════════════════════

\begin{figure}[htbp]
\centering
\begin{tikzpicture}[
    scale=0.9,
    transform shape,
    center/.style={
        rectangle,
        rounded corners=12pt,
        draw=bleurepublique,
        fill=bleulight,
        line width=2pt,
        minimum width=5cm,
        minimum height=1.5cm,
        font=\large\bfseries
    },
    node/.style={
        rectangle,
        rounded corners=8pt,
        draw=goldphoenix!70,
        fill=goldlight,
        minimum width=3.2cm,
        minimum height=1.2cm,
        align=center,
        font=\tiny\bfseries
    },
    arrow/.style={->, thick, bleurepublique!40, >=stealth}
]
\node[center] (c) at (0,0) {\rl{نهادهای نظارتی مستقل}};

\foreach \ang/\label/\desc in {60/کمیسیون انتخابات/نظارت بر آرا, 120/بانک مرکزی/سیاست پولی, 180/سازمان حسابرسی/شفافیت مالی, 240/کمیسیون حقوق بشر/حمایت از کرامت, 300/سازمان ضد فساد/مبارزه با رانت, 0/شورای عالی رسانه/آزادی بیان} {
    \node[node] (n\ang) at (\ang:4.5) {\rl{\label}\\\rl{\tiny \desc}};
    \draw[arrow] (c) -- (n\ang);
}
\end{tikzpicture}
\caption{منظومه نهادهای مستقل نظارتی (رکن چهارم دموکراسی)}
\label{fig:oversight}
\end{figure}

\begin{longtable}{|>{\columncolor{bleulight}}r|p{12cm}|}
\hline
\rowcolor{bleurepublique!20}
\headmark شماره اصل & \headmark متن کامل اصل (نهادهای مستقل نظارتی) \\
\hline
\endfirsthead
\hline
\rowcolor{bleurepublique!20}
\headmark شماره اصل & \headmark متن کامل اصل \\
\hline
\endhead

\textbf{اصل ۱۵۳} & \textbf{جایگاه نهادهای مستقل:} این نهادها رکن چهارم نظام‌اند. از قوای سه‌گانه مستقل بوده، بودجه مجزا دارند و رؤسای آنها با رأی دوسوم مجلس منصوب می‌شوند. \\
\hline

\rowcolor{goldlight}
\textbf{اصل ۱۵۴} & \textbf{کمیسیون مستقل انتخابات:} مسئول برگزاری و نظارت بر کلیه انتخابات و همه‌پرسی‌ها. اعضا باید فاقد وابستگی حزبی بوده و توسط دیوان عالی تأیید شوند. \\
\hline

\textbf{اصل ۱۵۵} & \textbf{بانک مرکزی:} هدف اصلی ثبات قیمت‌ها و حفظ ارزش پول است. دولت حق استقراض مستقیم یا دخالت در سیاست‌های پولی را ندارد. \\
\hline

\rowcolor{goldlight}
\textbf{اصل ۱۵۶} & \textbf{سازمان حسابرسی:} نظارت بر سلامت مالی و بودجه‌ای کلیه دستگاه‌های حکومتی و شرکت‌های عمومی و ارائه گزارش شفاف به ملت. \\
\hline

\textbf{اصل ۱۵۷} & \textbf{کمیسیون حقوق بشر:} صیانت از حقوق بنیادین، دریافت شکایات شهروندان، و نظارت بر حسن اجرای کنوانسیون‌های بین‌المللی حقوق بشر در ایران. \\
\hline

\rowcolor{goldlight}
\textbf{اصل ۱۵۸} & \textbf{سازمان ضد فساد:} پیشگیری و مبارزه با فساد در سطوح عالی حاکمیتی، حمایت از سوت‌زنان و شفاف‌سازی دارایی مقامات. \\
\hline

\textbf{اصل ۱۵۹} & \textbf{شورای عالی رسانه:} تضمین آزادی بیان، تکثرگرایی رسانه‌ای و ممنوعیت سانسور. نظارت بر رسانه‌های عمومی برای اطمینان از بی‌طرفی. \\
\hline

\rowcolor{goldlight}
\textbf{اصل ۱۶۰} & \textbf{نهاد بازرسی کل:} رسیدگی به شکایات شهروندان از سوءمدیریت اداری و تضمین پاسخگویی مقامات در برابر قانون. \\
\hline
\end{longtable}

% ═══════════════════════════════════════════════════════════════════════════════
\section{فصل ششم: حکومت محلی و شوراها}
\label{sec:const-ch6}
% ═══════════════════════════════════════════════════════════════════════════════

\begin{longtable}{|>{\columncolor{bleulight}}r|p{12cm}|}
\hline
\rowcolor{bleurepublique!20}
\headmark شماره اصل & \headmark متن کامل اصل (حکومت محلی و شوراها) \\
\hline
\endfirsthead
\hline
\rowcolor{bleurepublique!20}
\headmark شماره اصل & \headmark متن کامل اصل \\
\hline
\endhead

\textbf{اصل ۱۶۱} & \textbf{تمرکززدایی:} اداره امور محلی بر اساس اصل تقرب (واگذاری اختیارات به پایین‌ترین سطح ممکن) و مشارکت مستقیم شهروندان انجام می‌شود. \\
\hline

\rowcolor{goldlight}
\textbf{اصل ۱۶۲} & \textbf{شوراهای محلی:} در هر شهر و روستا شورای منتخب با رأی مستقیم مردم برای ۴ سال تشکیل می‌شود. شورا ناظر بر مدیریت محلی و تصویب‌کننده بودجه شهرداری است. \\
\hline

\textbf{اصل ۱۶۳} & \textbf{شهرداری‌ها:} شهردار توسط شورا انتخاب می‌شود. شهرداری‌ها استقلال مالی داشته و در قبال شورا و شهروندان پاسخگو هستند. \\
\hline

\rowcolor{goldlight}
\textbf{اصل ۱۶۴} & \textbf{استانداران:} استاندار هماهنگ‌کننده دولت فدرال و محلی است که با پیشنهاد وزیر کشور و تأیید شورای منتخب منطقه منصوب می‌شود. \\
\hline

\textbf{اصل ۱۶۵} & \textbf{شوراهای منطقه‌ای:} مسئول برنامه‌ریزی توسعه کلان منطقه، هماهنگی بین استانی و مدیریت منابع مشترک منطقه‌ای هستند. \\
\hline

\rowcolor{goldlight}
\textbf{اصل ۱۶۶} & \textbf{مشارکت و همه‌پرسی محلی:} حق شهروندان برای مشارکت در بودجه‌نویسی و الزام برگزاری رفراندوم محلی برای طرح‌های حساس زیست‌محیطی یا عمرانی محلی. \\
\hline
\end{longtable}

% ═══════════════════════════════════════════════════════════════════════════════
\section{فصل هفتم: بازنگری در قانون اساسی}
\label{sec:const-ch7}
% ═══════════════════════════════════════════════════════════════════════════════

\begin{center}
\begin{tikzpicture}[
    step/.style={
        draw=DemocracyBlue,
        fill=DemocracyBlue!15,
        rounded corners=8pt,
        minimum width=3.5cm,
        minimum height=2cm,
        align=center,
        font=\small
    }
]
    % عنوان
    \node[
        fill=DemocracyBlue,
        text=white,
        rounded corners=5pt,
        font=\large\bfseries,
        minimum width=10cm
    ] at (0,4) {فرآیند بازنگری قانون اساسی};
    
    % مراحل
    \node[step] (s1) at (-5,1.5) {\textbf{مرحله ۱}\\پیشنهاد\\(دوسوم هر مجلس\\یا ۲ میلیون امضا)};
    
    \node[step] (s2) at (0,1.5) {\textbf{مرحله ۲}\\بررسی\\(کمیته ویژه\\بازنگری)};
    
    \node[step] (s3) at (5,1.5) {\textbf{مرحله ۳}\\تصویب\\(سه‌چهارم\\هر دو مجلس)};
    
    \node[step, fill=SuccessGreen!20, draw=SuccessGreen] (s4) at (0,-1.5) {\textbf{مرحله ۴}\\همه‌پرسی\\(اکثریت مطلق\\با مشارکت ۵۰٪)};
    
    % فلش‌ها
    \draw[->, thick, DemocracyBlue] (s1) -- (s2);
    \draw[->, thick, DemocracyBlue] (s2) -- (s3);
    \draw[->, thick, DemocracyBlue] (s3) -- (s4);
    
    % خط قرمز
    \node[
        draw=WarningRed,
        fill=WarningRed!10,
        rounded corners=5pt,
        minimum width=10cm,
        font=\small
    ] at (0,-3.5) {\textbf{⚠️ اصول غیرقابل تغییر (اصل ۱۱) قابل بازنگری نیستند}};
\end{tikzpicture}
\end{center}

\begin{longtable}{|>{\columncolor{bleulight}}r|p{12cm}|}
\hline
\rowcolor{bleurepublique!20}
\headmark شماره اصل & \headmark متن کامل اصل (بازنگری در قانون اساسی) \\
\hline
\endfirsthead
\hline
\rowcolor{bleurepublique!20}
\headmark شماره اصل & \headmark متن کامل اصل \\
\hline
\endhead

\textbf{اصل ۱۶۷} & \textbf{اصل بازنگری:} این میثاق ملی سندی زنده است که بر اساس خرد جمعی و نیازهای زمان قابل بازنگری است، مشروط به حفظ هویت دموکراتیک و حقوق بنیادین. \\
\hline

\rowcolor{goldlight}
\textbf{اصل ۱۶۸} & \textbf{پیشنهاد بازنگری:} حق پیشنهاد بر عهده رئیس‌جمهور، دوسوم هر یک از مجالس، یا با ابتکار مردمی ۲ میلیون شهروند است. \\
\hline

\textbf{اصل ۱۶۹} & \textbf{کمیته بازنگری:} پس از تأیید ضرورت، کمیته‌ای متشکل از نمایندگان مردم، قضات عالی و حقوقدانان ظرف ۶ ماه پیش‌نویس طرح بازنگری را تهیه می‌کنند. \\
\hline

\rowcolor{goldlight}
\textbf{اصل ۱۷۰} & \textbf{تصویب پارلمانی:} طرح بازنگری نیازمند تصویب سه‌چهارم آرای هر دو مجلس (مجلس ملی و مجلس اقوام) در فاصله‌ای حداقل ۶۰ روزه است. \\
\hline

\textbf{اصل ۱۷۱} & \textbf{تأیید ملی (همه‌پرسی):} هرگونه تغییر در قانون اساسی پس از تصویب پارلمان، باید به همه‌پرسی عمومی گذاشته شده و با اکثریت مطلق آرا تأیید شود. \\
\hline

\rowcolor{goldlight}
\textbf{اصل ۱۷۲} & \textbf{اصول غیرقابل تغییر:} اصول مربوط به کرامت انسانی، نظام دموکراتیک، تفکیک قوا، فدرالیسم، و لغو مجازات اعدام به هیچ‌وجه قابل بازنگری نیستند. \\
\hline

\textbf{اصل ۱۷۳} & \textbf{اعلام و نفاذ:} بازنگری مصوب ظرف ۱۵ روز توسط رئیس‌جمهور ابلاغ و در سراسر کشور لازم‌الاجرا می‌گردد. \\
\hline
\end{longtable}

% ═══════════════════════════════════════════════════════════════════════════════
\section{فصل هشتم: مقررات انتقالی}
\label{sec:const-ch8}
% ═══════════════════════════════════════════════════════════════════════════════

\begin{enghelabbox}
\textbf{⚠️ اهمیت مقررات انتقالی}

مقررات انتقالی پل ارتباطی بین نظام قدیم و نظام جدید هستند. این مقررات:
\begin{itemize}[nosep]
    \item چارچوب زمانی مشخص برای اجرای قانون اساسی تعیین می‌کنند
    \item از خلأ قانونی و هرج‌ومرج جلوگیری می‌کنند
    \item حقوق مکتسبه را در حد معقول حفظ می‌کنند
    \item پس از اتمام دوره انتقال، خودبه‌خود منقضی می‌شوند
\end{itemize}
\end{enghelabbox}

\begin{longtable}{|>{\columncolor{bleulight}}r|p{12cm}|}
\hline
\rowcolor{bleurepublique!20}
\headmark شماره اصل & \headmark متن کامل اصل (مقررات انتقالی و نهایی) \\
\hline
\endfirsthead
\hline
\rowcolor{bleurepublique!20}
\headmark شماره اصل & \headmark متن کامل اصل \\
\hline
\endhead

\textbf{اصل ۱۷۴} & \textbf{لازم‌الاجرا شدن:} این قانون اساسی از تاریخ تأیید در همه‌پرسی ملی لازم‌الاجراست. دولت موقت موظف به زمینه‌سازی فوری برای اجرای آن است. \\
\hline

\rowcolor{goldlight}
\textbf{اصل ۱۷۵} & \textbf{دوره انتقال:} بازه زمانی حداکثر ۱۸ ماهه از تصویب قانون اساسی تا استقرار کامل نهادهای برگزیده، که طی آن امنیت گذار و تداوم خدمات عمومی تضمین می‌شود. \\
\hline

\textbf{اصل ۱۷۶} & \textbf{اولین انتخابات:} برگزاری اولین انتخابات سراسری تحت نظارت کمیسیون موقت و نهادهای بین‌المللی ظرف ۱۲ ماه از تصویب. \\
\hline

\rowcolor{goldlight}
\textbf{اصل ۱۷۷} & \textbf{جدول استقرار:} تعیین ضرب‌الاجل‌های ۳ تا ۲۴ ماهه برای تشکیل کمیسیون انتخابات، پارلمان‌ها، دیوان عالی قانون اساسی و شوراهای محلی. \\
\hline

\textbf{اصل ۱۷۸} & \textbf{تطبیق قوانین:} قوانین پیشین تا زمان نسخ یا اصلاح معتبرند، مشروط به عدم مغایرت بنیادین. پارلمان اول ۵ سال فرصت برای تطبیق کامل قوانین دارد. \\
\hline

\rowcolor{goldlight}
\textbf{اصل ۱۷۹} & \textbf{اداره امور:} حفظ بدنه کارشناسی دولت و امنیت شغلی کارکنانی که در نقض حقوق بشر مشارکت نداشته‌اند. \\
\hline

\textbf{اصل ۱۸۰} & \textbf{عدالت انتقالی:} تشکیل کمیسیون حقیقت و آشتی برای رسیدگی به مظالم گذشته، جبران خسارت قربانیان و پاسخگو کردن ناقضان حقوق مردم. \\
\hline

\rowcolor{goldlight}
\textbf{اصل ۱۸۱} & \textbf{اصلاح نیروهای مسلح:} ادغام نیروها در ساختار واحد تحت فرماندهی دولت فدرال و خروج کامل نظامیان از سیاست و اقتصاد. \\
\hline

\textbf{اصل ۱۸۲} & \textbf{استرداد اموال:} شناسایی و بازگرداندن کلیه اموال عمومی و خصوصی که به ناحق در نظام پیشین تصاحب شده است. \\
\hline

\rowcolor{goldlight}
\textbf{اصل ۱۸۳} & \textbf{جانشینی بین‌المللی:} تعهد به قراردادهای بین‌المللی مطابق با منافع ملی و الحاق فوری به کنوانسیون‌های جهانی حقوق بشر. \\
\hline

\textbf{اصل ۱۸۴} & \textbf{ساختار فدرال:} استقرار تدریجی مدیریت‌های منطقه‌ای و انتقال اختیارات از مرکز به مناطق ظرف ۵ سال. \\
\hline

\rowcolor{goldlight}
\textbf{اصل ۱۸۵} & \textbf{چندزبانی رسمی:} آغاز برنامه ملی آموزش و استانداردسازی زبان‌های اقوام برای کاربرد رسمی در سطوح فدرال و منطقه‌ای. \\
\hline

\textbf{اصل ۱۸۶} & \textbf{انقضا:} این فصل پس از تحقق اهداف گذار یا پایان مهلت ۵ ساله خودبه‌خود از متن قانون اساسی خارج می‌شود. \\
\hline

\rowcolor{goldlight}
\textbf{اصل ۱۸۷} & \textbf{میثاق وفاداری:} این قانون اساسی عهد ملت ایران برای آزادی و عدالت است. هیچ مقامی فراتر از آن نیست و صیانت از آن تکلیف همگانی است. \\
\hline
\end{longtable}

% ═══════════════════════════════════════════════════════════════════════════════
\section{جداول تطبیقی با قوانین اساسی سایر کشورها}
\label{sec:const-comparative}
% ═══════════════════════════════════════════════════════════════════════════════

\subsection{مقایسه ساختار کلی}

\begin{center}
\begin{small}
\begin{longtable}{|>{\columncolor{DemocracyBlue!10}}p{2.5cm}|p{2.2cm}|p{2.2cm}|p{2.2cm}|p{2.2cm}|p{2.2cm}|}
\hline
\rowcolor{DemocracyBlue!30}
\textbf{ویژگی} & \textbf{ایران (پیشنهادی)} & \textbf{آلمان} & \textbf{هند} & \textbf{آفریقای جنوبی} & \textbf{اسپانیا} \\
\hline
\endfirsthead
\hline
\rowcolor{DemocracyBlue!30}
\textbf{ویژگی} & \textbf{ایران (پیشنهادی)} & \textbf{آلمان} & \textbf{هند} & \textbf{آفریقای جنوبی} & \textbf{اسپانیا} \\
\hline
\endhead

نوع نظام & جمهوری فدرال & جمهوری فدرال & جمهوری فدرال & جمهوری واحد & پادشاهی مشروطه \\
\hline

رئیس کشور & رئیس‌جمهور منتخب & رئیس‌جمهور (غیرمستقیم) & رئیس‌جمهور (غیرمستقیم) & رئیس‌جمهور (پارلمانی) & پادشاه \\
\hline

رئیس دولت & رئیس‌جمهور & صدراعظم & نخست‌وزیر & رئیس‌جمهور & نخست‌وزیر \\
\hline

پارلمان & دومجلسی & دومجلسی & دومجلسی & دومجلسی & دومجلسی \\
\hline

تعداد واحدهای فدرال & ۵ منطقه + ۱۵ استان & ۱۶ لَند & ۲۸ ایالت + ۸ منطقه & ۹ استان & ۱۷ منطقه خودمختار \\
\hline

دادگاه قانون اساسی & بله (۱۵ قاضی) & بله (۱۶ قاضی) & بله (دیوان عالی) & بله (۱۱ قاضی) & بله (۱۲ قاضی) \\
\hline

اصول غیرقابل تغییر & بله (۷ مورد) & بله (۲ مورد) & خیر (رویه قضایی) & بله (ارزش‌های بنیادین) & خیر \\
\hline

جدایی دین و دولت & کامل & بله & سکولار & بله & بله \\
\hline

\end{longtable}
\end{small}
\end{center}

\subsection{مقایسه حقوق بنیادین}

\begin{center}
\begin{small}
\begin{longtable}{|>{\columncolor{SuccessGreen!10}}p{3cm}|c|c|c|c|c|}
\hline
\rowcolor{SuccessGreen!30}
\textbf{حق بنیادین} & \textbf{ایران} & \textbf{آلمان} & \textbf{هند} & \textbf{آفریقا} & \textbf{کانادا} \\
\hline
\endfirsthead
\hline
\rowcolor{SuccessGreen!30}
\textbf{حق بنیادین} & \textbf{ایران} & \textbf{آلمان} & \textbf{هند} & \textbf{آفریقا} & \textbf{کانادا} \\
\hline
\endhead

لغو مجازات اعدام & ✓ & ✓ & ✗ & ✓ & ✓ \\
\hline

آزادی دین و تغییر دین & ✓ & ✓ & ✓ & ✓ & ✓ \\
\hline

برابری جنسیتی صریح & ✓ & ✓ & ✓ & ✓ & ✓ \\
\hline

حقوق زبانی اقلیت‌ها & ✓✓ & ✗ & ✓✓ & ✓✓ & ✓✓ \\
\hline

حق آب & ✓ & ✗ & ✗ & ✓ & ✗ \\
\hline

حق محیط زیست سالم & ✓ & ✓ & ✓ & ✓ & ✗ \\
\hline

حق مسکن & ✓ & ✗ & ✗ & ✓ & ✗ \\
\hline

ممنوعیت تبعیض گرایش جنسی & ✓ & ✓ & ✓ & ✓ & ✓ \\
\hline

حق دسترسی به اطلاعات & ✓ & ✓ & ✓ & ✓ & ✓ \\
\hline

حفاظت از افشاگران & ✓ & ✓ & ✗ & ✓ & ✓ \\
\hline

\multicolumn{6}{l}{\scriptsize ✓✓ = حمایت ویژه | ✓ = تضمین‌شده | ✗ = بدون تضمین صریح} \\

\end{longtable}
\end{small}
\end{center}

\subsection{مقایسه ساختار فدرال}

\begin{center}
\begin{tikzpicture}[scale=0.9]
    % محور عمودی - درجه تمرکززدایی
    \draw[->, thick] (0,0) -- (0,7) node[above, font=\small, align=center] {تمرکززدایی\\بیشتر};
    
    % محور افقی - تنوع قومی
    \draw[->, thick] (0,0) -- (12,0) node[right, font=\small, align=center] {تنوع قومی\\بیشتر};
    
    % کشورها
    \node[circle, draw=DemocracyBlue, fill=DemocracyBlue!30, minimum size=1.2cm, font=\tiny] at (8,5.5) {ایران};
    \node[circle, draw=SuccessGreen, fill=SuccessGreen!30, minimum size=1cm, font=\tiny] at (3,5) {آلمان};
    \node[circle, draw=WisdomGold, fill=WisdomGold!30, minimum size=1cm, font=\tiny] at (10,4.5) {هند};
    \node[circle, draw=purple, fill=purple!30, minimum size=1cm, font=\tiny] at (7,3.5) {اسپانیا};
    \node[circle, draw=orange, fill=orange!30, minimum size=1cm, font=\tiny] at (9,4) {آفریقا};
    \node[circle, draw=WarningRed, fill=WarningRed!30, minimum size=1cm, font=\tiny] at (5,6) {سوئیس};
    \node[circle, draw=gray, fill=gray!30, minimum size=1cm, font=\tiny] at (4,2) {فرانسه};
    
    % راهنما
    \node[font=\scriptsize, align=right] at (2,6.5) {فدرال};
    \node[font=\scriptsize, align=right] at (2,1) {متمرکز};
\end{tikzpicture}
\end{center}

\subsection{مقایسه فرآیند بازنگری قانون اساسی}

\begin{center}
\begin{small}
\begin{longtable}{|>{\columncolor{bleulight}}r|p{3.5cm}|p{3.5cm}|p{2cm}|p{3cm}|}
\hline
\rowcolor{bleurepublique!20}
\headmark کشور & \headmark پیشنهاددهنده & \headmark تصویب پارلمان & \headmark همه‌پرسی & \headmark اصول غیرقابل تغییر \\
\hline
\endfirsthead
\hline
\rowcolor{bleurepublique!20}
\headmark کشور & \headmark پیشنهاددهنده & \headmark تصویب پارلمان & \headmark همه‌پرسی & \headmark اصول غیرقابل تغییر \\
\hline
\endhead

\textbf{ایران (پیشنهادی)} & دوسوم هر مجلس یا ۲ میلیون امضا & سه‌چهارم هر دو مجلس & اجباری & کرامت انسانی، دموکراسی، فدرالیسم \\
\hline

\rowcolor{goldlight}
\textbf{آلمان} & دوسوم هر مجلس & دوسوم هر دو مجلس & خیر & کرامت انسانی، ساختار فدرال \\
\hline

\textbf{فرانسه} & رئیس‌جمهور یا پارلمان & سه‌پنجم کنگره & اختیاری & شکل جمهوری \\
\hline

\rowcolor{goldlight}
\textbf{سوئیس} & پارلمان یا ۱۰۰,۰۰۰ امضا & اکثریت ساده & اجباری & خیر \\
\hline
\end{longtable}
\end{small}
\end{center}

% ═══════════════════════════════════════════════════════════════════════════════
\section{نمودار کلی ساختار حکومت}
\label{sec:const-diagram}
% ═══════════════════════════════════════════════════════════════════════════════

\begin{figure}[htbp]
\centering
\begin{tikzpicture}[
    scale=0.8,
    transform shape,
    box/.style={
        rectangle,
        rounded corners=8pt,
        draw=bleurepublique,
        fill=bleulight,
        line width=1.5pt,
        minimum width=3cm,
        minimum height=1.2cm,
        text centered,
        font=\small\bfseries
    },
    nationnode/.style={
        rectangle,
        rounded corners=15pt,
        draw=goldphoenix,
        fill=goldlight,
        line width=3pt,
        minimum width=14cm,
        minimum height=1.6cm,
        font=\Large\bfseries
    },
    arrow/.style={->, ultra thick, bleurepublique!30, >=stealth}
]
\node[nationnode] (nation) at (0,8) {\rl{ملت ایران — منشأ حاکمیت}};

\node[box] (nat) at (-5,5.5) {\rl{مجلس ملی}};
\node[box] (sen) at (-1.5,5.5) {\rl{مجلس اقوام}};
\node[box, draw=goldphoenix, fill=goldlight] (pres) at (2,5.5) {\rl{رئیس‌جمهور}};
\node[box] (cc) at (5.5,5.5) {\rl{دیوان قانون اساسی}};

\node[box, minimum height=1cm] (reg) at (0,3) {\rl{دولت‌های منطقه‌ای و استانی}};
\node[box, minimum height=1cm] (ind) at (0,1) {\rl{نهادهای مستقل نظارتی (رکن چهارم)}};

\draw[arrow] (nation) -- (nat);
\draw[arrow] (nation) -- (sen);
\draw[arrow] (nation) -- (pres);
\draw[arrow] (nation) to[bend left] (cc);
\draw[arrow] (nation) to[bend right] (reg);
\draw[arrow] (cc) -- (nat);
\draw[arrow] (cc) -- (pres);

\end{tikzpicture}
\caption{ساختار کلان حکومت جمهوری فدرال ایران (تفکیک قوا و تعادل فدرال)}
\label{fig:gov-overall}
\end{figure}

% ═══════════════════════════════════════════════════════════════════════════════
\section{خلاصه و جمع‌بندی}
\label{sec:const-summary}
% ═══════════════════════════════════════════════════════════════════════════════

\begin{kholasebox}
\textbf{خلاصه قانون اساسی پیشنهادی جمهوری فدرال ایران}

\begin{center}
\begin{longtable}{|>{\columncolor{bleulight}}r|p{8cm}|}
\hline
\rowcolor{bleurepublique!20}
\headmark شاخص ساختاری & \headmark جزئیات و آمار کلی \\
\hline
\endfirsthead
\hline
\rowcolor{bleurepublique!20}
\headmark شاخص ساختاری & \headmark جزئیات و آمار کلی \\
\hline
\endhead

\textbf{تعداد کل اصول} & ۱۸۷ اصل در ۸ فصل \\
\hline

\rowcolor{goldlight}
\textbf{اصول غیرقابل تغییر} & ۷ اصل (پاسدار هویت نظام) \\
\hline

\textbf{حقوق بنیادین} & ۷۴ اصل (جامع‌ترین منشور حقوق منطقه) \\
\hline

\rowcolor{goldlight}
\textbf{واحدهای فدرال} & ۵ منطقه خودمختار و ۱۵ استان فدرال \\
\hline

\textbf{نهادهای مستقل} & ۸ سازمان نظارتی رکن چهارم \\
\hline

\rowcolor{goldlight}
\textbf{دوره انتقالی} & حداکثر ۱۸ ماه برای استقرار نهادها \\
\hline
\end{longtable}
\end{center}

\textbf{ویژگی‌های متمایز این قانون اساسی:}

\begin{enumerate}[nosep]
    \item \textbf{جدایی کامل دین از دولت} — برخلاف قانون اساسی فعلی
    \item \textbf{فدرالیسم همبسته} — پاسخ به مطالبات قومی بدون تجزیه
    \item \textbf{مجلس اقوام} — نهاد ویژه نمایندگی تنوع
    \item \textbf{لغو مجازات اعدام} — پیشرو در منطقه
    \item \textbf{حق آب} — پاسخ به بحران آب
    \item \textbf{نهادهای مستقل قوی} — پیشگیری از استبداد
    \item \textbf{اصول غیرقابل تغییر} — خطوط قرمز دموکراسی
    \item \textbf{عدالت انتقالی} — گذار مسالمت‌آمیز اما عادلانه
\end{enumerate}
\end{kholasebox}

\vspace{10pt}

\begin{olgoobox}
\textbf{منابع الهام‌بخش این قانون اساسی:}

\begin{itemize}[nosep]
    \item قانون اساسی آلمان (۱۹۴۹): ساختار فدرال، دادگاه قانون اساسی، اصول غیرقابل تغییر
    \item قانون اساسی آفریقای جنوبی (۱۹۹۶): منشور حقوق، عدالت انتقالی، حقوق اقتصادی-اجتماعی
    \item قانون اساسی هند (۱۹۵۰): مدیریت تنوع، فدرالیسم نامتقارن
    \item قانون اساسی اسپانیا (۱۹۷۸): مناطق خودمختار، گذار دموکراتیک
    \item قانون اساسی سوئیس (۱۹۹۹): دموکراسی مستقیم، فدرالیسم زبانی
    \item منشور حقوق بشر اروپا و میثاقین بین‌المللی حقوق بشر
\end{itemize}
\end{olgoobox}

% ═══════════════════════════════════════════════════════════════════════════════
% پایان پیوست ۱
% ═══════════════════════════════════════════════════════════════════════════════
	% ═══════════════════════════════════════════════════════════════════════════════
% پیوست ۲: منشور حقوق اقوام ایران
% فایل: app02-charter.tex
% ═══════════════════════════════════════════════════════════════════════════════

\chapter{منشور حقوق اقوام}
\label{app:charter}

\begin{kholasebox}
این منشور سند حقوقی تکمیلی قانون اساسی است که حقوق فردی و جمعی اقوام ایران را به تفصیل تضمین می‌کند. منشور بر این باور استوار است که \textbf{تنوع قومی-فرهنگی ایران نه تهدید بلکه ثروت ملی} است. این سند با الهام از اعلامیه حقوق اقلیت‌های ملل متحد (۱۹۹۲)، کنوانسیون چارچوب حمایت از اقلیت‌های ملی شورای اروپا (۱۹۹۵)، و تجربیات موفق کشورهای چندقومی تدوین شده است.
\end{kholasebox}

% ═══════════════════════════════════════════════════════════════════════════════
\section{دیباچه منشور}
% ═══════════════════════════════════════════════════════════════════════════════

\begin{center}
\begin{tikzpicture}
    % کادر اصلی
    \node[
        draw=bleurepublique,
        line width=3pt,
        fill=bleurepublique!5,
        rounded corners=15pt,
        inner sep=20pt,
        text width=14cm,
        align=justify
    ] (preamble) {
        \begin{center}
            {\Huge\textbf{\rl{منشور حقوق اقوام}}}\\[5pt]
            {\LARGE\textbf{\rl{جمهوری فدرال ایران}}}\\[10pt]
            {\large \rl{ضمیمه قانون اساسی | دارای قوت قانون اساسی}}
        \end{center}
    };
    
    % نماد تنوع
    \foreach \i/\color in {1/bleurepublique, 2/goldlight, 3/bleulight, 4/golddark, 5/bleurepublique, 6/goldlight} {
        \node[circle, fill=\color, minimum size=8pt] at (-3.5+\i*1.2, 4.2) {};
    }
\end{tikzpicture}
\end{center}

\vspace{10pt}

\begin{naghlbox}
\textbf{دیباچه}

ما، نمایندگان ملت ایران در مجلس مؤسسان،

با اذعان به اینکه ایران سرزمینی است با پیشینه تمدنی چندهزارساله که در آن اقوام گوناگون — فارس، آذری، کرد، لر، عرب، بلوچ، ترکمن، گیلک، مازندرانی، و دیگران — در طول تاریخ در کنار یکدیگر زیسته و این سرزمین را آباد کرده‌اند؛

با یادآوری رنج‌هایی که از تبعیض، سرکوب فرهنگی، و بی‌عدالتی اقتصادی بر اقوام رفته است؛

با باور به اینکه هر قومی حق دارد هویت، زبان، فرهنگ و سنت‌های خود را حفظ و توسعه دهد؛

با تعهد به ساختن آینده‌ای که در آن تنوع نه منشأ تفرقه بلکه سرچشمه ثروت و همبستگی باشد؛

با احترام به اصل تمامیت ارضی و وحدت ملی در چارچوب فدرالیسم دموکراتیک؛

\textbf{این منشور را به عنوان میثاق زندگی مشترک اقوام ایران تصویب و اعلام می‌کنیم.}

\sourceline{مجلس مؤسسان جمهوری فدرال ایران}
\end{naghlbox}

% ═══════════════════════════════════════════════════════════════════════════════
\section{فصل اول: اصول کلی و تعاریف}
\label{sec:charter-ch1}
% ═══════════════════════════════════════════════════════════════════════════════

\subsection{ماده ۱: هدف منشور}

این منشور به منظور تضمین حقوق اقوام ایران، حفاظت از تنوع فرهنگی، و ایجاد چارچوب حقوقی برای زندگی مشترک برابر و عادلانه تدوین شده است.

\subsection{ماده ۲: تعاریف}

\begin{longtable}{|>{\columncolor{bleurepublique!15}}p{3cm}|p{11cm}|}
\hline
\rowcolor{bleurepublique!30}
\textbf{\rl{اصطلاح}} & \textbf{\rl{تعریف}} \\
\hline
\endfirsthead
\hline
\rowcolor{bleurepublique!30}
\textbf{\rl{اصطلاح}} & \textbf{\rl{تعریف}} \\
\hline
\endhead

\textbf{\rl{قوم}} & \rl{گروهی از شهروندان که دارای ویژگی‌های مشترک زبانی، فرهنگی، تاریخی یا سرزمینی هستند و خود را متعلق به آن گروه می‌دانند.} \\
\hline

\textbf{\rl{زبان ملی}} & \rl{زبانی که توسط یکی از اقوام اصلی ایران به عنوان زبان مادری استفاده می‌شود و در این منشور به رسمیت شناخته شده است.} \\
\hline

\textbf{\rl{منطقه خودمختار}} & \rl{واحد سرزمینی که طبق قانون اساسی از درجه‌ای از خودمختاری در امور داخلی برخوردار است.} \\
\hline

\textbf{\rl{سرزمین تاریخی}} & \rl{منطقه‌ای که یک قوم به‌طور سنتی و تاریخی در آن ساکن بوده است.} \\
\hline

\textbf{\rl{حقوق جمعی}} & \rl{حقوقی که به یک گروه قومی به‌عنوان یک کلیت تعلق دارد.} \\
\hline

\textbf{\rl{حقوق فردی}} & \rl{حقوقی که به هر فرد عضو یک قوم تعلق دارد.} \\
\hline

\textbf{\rl{تبعیض مثبت}} & \rl{اقدامات موقت برای جبران نابرابری‌های تاریخی و ایجاد برابری واقعی.} \\
\hline

\textbf{\rl{همسان‌سازی اجباری}} & \rl{هرگونه سیاست یا اقدام برای از بین بردن هویت قومی یا فرهنگی.} \\
\hline

\end{longtable}

\subsection{ماده ۳: اقوام به رسمیت شناخته‌شده}

\begin{center}
\begin{tikzpicture}[
    ethnicity/.style={
        draw=#1,
        line width=1.5pt,
        fill=#1!20,
        rounded corners=8pt,
        minimum width=2.8cm,
        minimum height=1.8cm,
        align=center,
        font=\small\bfseries
    }
]
    % عنوان
    \node[
        fill=bleurepublique,
        text=white,
        rounded corners=5pt,
        font=\large\bfseries,
        minimum width=10cm
    ] at (0,5) {\rl{اقوام اصلی ایران}};
    
    % ردیف اول
    \node[ethnicity=bleurepublique] at (-5,3) {\shortstack{\rl{فارس}\\\rl{≈۵۵٪}}};
    \node[ethnicity=goldlight] at (-2,3) {\shortstack{\rl{آذری}\\\\rl{≈۲۰٪}}};
    \node[ethnicity=bleulight] at (1,3) {\shortstack{\rl{کرد}\\\rl{≈۱۰٪}}};
    \node[ethnicity=golddark] at (4,3) {\shortstack{\rl{لر}\\\rl{≈۶٪}}};
    
    % ردیف دوم
    \node[ethnicity=bleurepublique!80] at (-4,0.5) {\shortstack{\rl{عرب}\\\rl{≈۳٪}}};
    \node[ethnicity=goldlight!80] at (-1,0.5) {\shortstack{\rl{بلوچ}\\\rl{≈۲٪}}};
    \node[ethnicity=bleulight!80] at (2,0.5) {\shortstack{\rl{ترکمن}\\\rl{≈۱٪}}};
    \node[ethnicity=golddark!80] at (5,0.5) {\shortstack{\rl{گیلک و مازنی}\\\rl{≈۳٪}}};
    
    % یادداشت
    \node[font=\scriptsize, text=gray] at (0,-1.5) {\rl{* درصدها تخمینی است. هر فرد می‌تواند هویت چندگانه داشته باشد.}};
\end{tikzpicture}
\end{center}

\begin{longtable}{|>{\columncolor{bleurepublique!10}}p{2cm}|p{3cm}|p{3cm}|p{3cm}|p{3cm}|}
\hline
\rowcolor{bleurepublique!30}
\textbf{\rl{قوم}} & \textbf{\rl{زبان اصلی}} & \textbf{\rl{سرزمین تاریخی}} & \textbf{\rl{جمعیت تخمینی}} & \textbf{\rl{وضعیت}} \\
\hline
\endfirsthead
\hline
\rowcolor{bleurepublique!30}
\textbf{\rl{قوم}} & \textbf{\rl{زبان اصلی}} & \textbf{\rl{سرزمین تاریخی}} & \textbf{\rl{جمعیت تخمینی}} & \textbf{\rl{وضعیت}} \\
\hline
\endhead

\rl{فارس} & \rl{فارسی} & \rl{فلات مرکزی، خراسان، فارس} & \rl{≈۴۷ میلیون} & \rl{قوم اکثریت} \\
\hline

\rl{آذری} & \rl{ترکی آذربایجانی} & \rl{آذربایجان شرقی و غربی، اردبیل، زنجان} & \rl{≈۱۷ میلیون} & \rl{منطقه خودمختار} \\
\hline

\rl{کرد} & \rl{کردی (سورانی، کرمانجی)} & \rl{کردستان، کرمانشاه، ایلام، آذربایجان غربی} & \rl{≈۸.۵ میلیون} & \rl{منطقه خودمختار} \\
\hline

\rl{لر} & \rl{لری (لکی، بختیاری)} & \rl{لرستان، چهارمحال، کهگیلویه، خوزستان} & \rl{≈۵ میلیون} & \rl{حقوق ویژه} \\
\hline

\rl{عرب} & \rl{عربی} & \rl{خوزستان} & \rl{≈۲.۵ میلیون} & \rl{منطقه خودمختار} \\
\hline

\rl{بلوچ} & \rl{بلوچی} & \rl{سیستان و بلوچستان} & \rl{≈۱.۷ میلیون} & \rl{منطقه خودمختار} \\
\hline

\rl{ترکمن} & \rl{ترکمنی} & \rl{گلستان (ترکمن‌صحرا)} & \rl{≈۸۵۰ هزار} & \rl{منطقه خودمختار} \\
\hline

\rl{گیلک} & \rl{گیلکی} & \rl{گیلان} & \rl{≈۳ میلیون} & \rl{حقوق ویژه} \\
\hline

\rl{مازندرانی} & \rl{مازندرانی} & \rl{مازندران} & \rl{≈۳ میلیون} & \rl{حقوق ویژه} \\
\hline

\multicolumn{5}{l}{\scriptsize \rl{سایر گروه‌های قومی-زبانی (تات، تالش، قشقایی، و غیره) نیز از حمایت این منشور برخوردارند.}} \\

\end{longtable}

\subsection{ماده ۴: اصول بنیادین}

\begin{olgoobox}
\textbf{اصول حاکم بر این منشور:}

\begin{enumerate}[nosep]
    \item \textbf{اصل برابری}: همه اقوام برابرند؛ هیچ قومی بر قوم دیگر برتری ندارد
    \item \textbf{اصل عدم تبعیض}: تبعیض بر اساس قومیت در هر شکل ممنوع است
    \item \textbf{اصل حفاظت}: دولت موظف به حمایت فعال از تنوع قومی است
    \item \textbf{اصل مشارکت}: اقوام حق مشارکت در تصمیمات مؤثر بر زندگی خود را دارند
    \item \textbf{اصل همبستگی}: تنوع در خدمت وحدت ملی و انسجام اجتماعی است
    \item \textbf{اصل توسعه متوازن}: همه مناطق حق توسعه برابر دارند
    \item \textbf{اصل تمامیت ارضی}: حقوق اقوام در چارچوب وحدت و یکپارچگی سرزمینی تضمین می‌شود
\end{enumerate}
\end{olgoobox}

% ═══════════════════════════════════════════════════════════════════════════════
\section{فصل دوم: حقوق فردی اعضای اقوام}
\label{sec:charter-ch2}
% ═══════════════════════════════════════════════════════════════════════════════

\begin{center}
\begin{tikzpicture}
    % عنوان
    \node[
        fill=bleurepublique,
        text=white,
        rounded corners=5pt,
        font=\large\bfseries,
        minimum width=8cm
    ] at (0,4) {\rl{حقوق فردی اعضای اقوام}};
    
    % دسته‌بندی حقوق
    \foreach \i/\title/\items in {
        1/{\rl{حق هویت}}/{\rl{انتخاب آزادانه، ثبت هویت قومی، عدم اجبار}},
        2/{\rl{حق زبان}}/{\rl{استفاده آزاد، آموزش، خدمات دولتی}},
        3/{\rl{حق فرهنگ}}/{\rl{مشارکت، حفظ سنت‌ها، دسترسی}},
        4/{\rl{حق عدم تبعیض}}/{\rl{برابری در استخدام، آموزش، خدمات}}
    } {
        \node[
            draw=bleurepublique!70,
            fill=bleurepublique!10,
            rounded corners=5pt,
            minimum width=6cm,
            minimum height=1.5cm,
            align=center,
            font=\small
        ] at (0, 2.5-\i*1.8) {\textbf{\title}\\\scriptsize\items};
    }
\end{tikzpicture}
\end{center}

\subsection{ماده ۵: حق هویت قومی}

\begin{longtable}{|>{\columncolor{bleurepublique!15}}r|p{12cm}|}
\hline
\rowcolor{bleurepublique!30}
\textbf{\rl{بند}} & \textbf{\rl{متن}} \\
\hline
\endfirsthead

۵.۱ & \rl{هر فرد حق دارد آزادانه هویت قومی خود را تعیین کند، اعلام کند یا اعلام نکند.} \\
\hline

۵.۲ & \rl{هیچ‌کس را نمی‌توان به پذیرش یا انکار هویت قومی خاصی مجبور کرد.} \\
\hline

۵.۳ & \rl{هر فرد می‌تواند هویت قومی چندگانه داشته باشد (مثلاً فرزند خانواده کرد-فارس).} \\
\hline

۵.۴ & \rl{ثبت هویت قومی در اسناد رسمی اختیاری است و تنها برای مقاصد آماری و با رضایت فرد انجام می‌شود.} \\
\hline

۵.۵ & \rl{هیچ پیامد منفی حقوقی، اداری یا اجتماعی نباید بر اعلام یا عدم اعلام هویت قومی مترتب شود.} \\
\hline

\end{longtable}

\subsection{ماده ۶: حق استفاده از زبان مادری}

\begin{longtable}{|>{\columncolor{DemocracyBlue!15}}r|p{12cm}|}
\hline
\rowcolor{DemocracyBlue!30}
\textbf{بند} & \textbf{متن} \\
\hline
\endfirsthead

۶.۱ & هر فرد حق دارد در زندگی خصوصی و عمومی از زبان مادری خود استفاده کند. \\
\hline

۶.۲ & هر کودک حق دارد به زبان مادری خود آموزش ببیند. \\
\hline

۶.۳ & هر فرد حق دارد در دادگاه‌ها و نهادهای دولتی از مترجم رایگان بهره‌مند شود. \\
\hline

۶.۴ & هر فرد حق دارد نام خانوادگی و نام فرزندان خود را به زبان مادری انتخاب کند. \\
\hline

۶.۵ & استفاده از زبان مادری در محیط کار نمی‌تواند دلیل اخراج یا تبعیض باشد. \\
\hline

\end{longtable}

\subsection{ماده ۷: حق مشارکت در زندگی فرهنگی}

\begin{longtable}{|>{\columncolor{bleurepublique!15}}r|p{12cm}|}
\hline
\rowcolor{bleurepublique!30}
\textbf{\rl{بند}} & \textbf{\rl{متن}} \\
\hline
\endfirsthead

۷.۱ & \rl{هر فرد حق دارد در زندگی فرهنگی قوم خود مشارکت کند.} \\
\hline

۷.۲ & \rl{هر فرد حق دارد آداب، رسوم و مراسم قومی خود را برگزار کند.} \\
\hline

۷.۳ & \rl{هر فرد حق دارد لباس سنتی قومی خود را بپوشد.} \\
\hline

۷.۴ & \rl{هر فرد حق دارد به میراث فرهنگی قوم خود دسترسی داشته باشد.} \\
\hline

۷.۵ & \rl{هر فرد حق دارد تاریخ و ادبیات قوم خود را بیاموزد و بیاموزاند.} \\
\hline

\end{longtable}

\subsection{ماده ۸: حق عدم تبعیض}

\begin{enghelabbox}
\textbf{⚠️ تبعیض قومی جرم است}

\textbf{ماده ۸.۱:} هرگونه تبعیض بر اساس قومیت، زبان یا فرهنگ ممنوع و جرم است.

\textbf{ماده ۸.۲:} تبعیض شامل اما نه محدود به موارد زیر است:
\begin{itemize}[nosep]
    \item تبعیض در استخدام، ارتقا یا اخراج
    \item تبعیض در دسترسی به آموزش
    \item تبعیض در دسترسی به خدمات عمومی
    \item تبعیض در دسترسی به مسکن
    \item توهین و تحقیر قومی
    \item نفرت‌پراکنی قومی
\end{itemize}

\textbf{ماده ۸.۳:} قربانیان تبعیض حق شکایت و جبران خسارت مادی و معنوی دارند.

\textbf{ماده ۸.۴:} بار اثبات عدم تبعیض در دعاوی استخدامی بر عهده کارفرماست.
\end{enghelabbox}

% ═══════════════════════════════════════════════════════════════════════════════
\section{فصل سوم: حقوق جمعی اقوام}
\label{sec:charter-ch3}
% ═══════════════════════════════════════════════════════════════════════════════

\begin{center}
\begin{tikzpicture}
    % عنوان مرکزی
    \node[
        draw=golddark,
        line width=2pt,
        fill=golddark!20,
        rounded corners=10pt,
        minimum width=4cm,
        minimum height=1.5cm,
        font=\large\bfseries
    ] (center) at (0,0) {\rl{حقوق جمعی}};
    
    % شاخه‌ها
    \foreach \angle/\title in {
        90/{\rl{حق وجود و هویت}},
        45/{\rl{حق خودمختاری}},
        0/{\rl{حق توسعه}},
        -45/{\rl{حق نمایندگی}},
        -90/{\rl{حق فرهنگی}},
        -135/{\rl{حق زبانی}},
        180/{\rl{حق سرزمینی}},
        135/{\rl{حق مشورت}}
    } {
        \node[
            draw=golddark!70,
            fill=golddark!10,
            rounded corners=5pt,
            minimum width=2.5cm,
            align=center,
            font=\scriptsize\bfseries
        ] at (\angle:3.5cm) {\title};
        
        \draw[->, thick, golddark!60] (center) -- (\angle:2.8cm);
    }
\end{tikzpicture}
\end{center}

\subsection{ماده ۹: حق وجود و حفظ هویت جمعی}

\begin{longtable}{|>{\columncolor{SuccessGreen!15}}r|p{12cm}|}
\hline
\rowcolor{SuccessGreen!30}
\textbf{بند} & \textbf{متن} \\
\hline
\endfirsthead

۹.۱ & هر قوم حق وجود به‌عنوان یک گروه متمایز را دارد. \\
\hline

۹.۲ & هرگونه سیاست یا اقدام برای نابودی، همسان‌سازی اجباری یا محو هویت یک قوم ممنوع و جرم علیه بشریت است. \\
\hline

۹.۳ & هر قوم حق دارد هویت جمعی خود را حفظ، توسعه و به نسل‌های آینده منتقل کند. \\
\hline

۹.۴ & دولت موظف به حمایت فعال از حفظ تنوع قومی است. \\
\hline

\end{longtable}

\subsection{ماده ۱۰: حق خودمختاری داخلی}

\begin{longtable}{|>{\columncolor{SuccessGreen!15}}r|p{12cm}|}
\hline
\rowcolor{SuccessGreen!30}
\textbf{بند} & \textbf{متن} \\
\hline
\endfirsthead

۱۰.۱ & اقوامی که در سرزمین تاریخی خود اکثریت دارند، حق تشکیل منطقه خودمختار را در چارچوب قانون اساسی دارند. \\
\hline

۱۰.۲ & خودمختاری شامل حق قانون‌گذاری در حوزه‌های محلی، اداره امور داخلی، و مدیریت منابع محلی است. \\
\hline

۱۰.۳ & خودمختاری در چارچوب تمامیت ارضی و وحدت ملی تفسیر می‌شود. \\
\hline

۱۰.۴ & اقوامی که منطقه خودمختار ندارند، از طریق مجلس اقوام و نهادهای محلی نمایندگی می‌شوند. \\
\hline

\end{longtable}

\subsection{ماده ۱۱: حق مشارکت سیاسی}

\begin{longtable}{|>{\columncolor{SuccessGreen!15}}r|p{12cm}|}
\hline
\rowcolor{SuccessGreen!30}
\textbf{بند} & \textbf{متن} \\
\hline
\endfirsthead

۱۱.۱ & هر قوم حق نمایندگی عادلانه در نهادهای ملی را دارد. \\
\hline

۱۱.۲ & مجلس اقوام نهاد نمایندگی جمعی اقوام در سطح ملی است. \\
\hline

۱۱.۳ & در انتصابات ملی (کابینه، دیوان عالی، نهادهای مستقل) باید تنوع قومی رعایت شود. \\
\hline

۱۱.۴ & احزاب قومی-منطقه‌ای حق فعالیت آزاد دارند مشروط به پایبندی به قانون اساسی و اصول دموکراتیک. \\
\hline

۱۱.۵ & هیچ قانون یا سیاستی که مستقیماً بر یک قوم تأثیر می‌گذارد، نباید بدون مشورت با نمایندگان آن قوم تصویب شود. \\
\hline

\end{longtable}

\subsection{ماده ۱۲: حق توسعه و منابع}

\begin{longtable}{|>{\columncolor{goldlight!15}}r|p{12cm}|}
\hline
\rowcolor{goldlight!30}
\textbf{\rl{بند}} & \textbf{\rl{متن}} \\
\hline
\endfirsthead

۱۲.۱ & \rl{هر قوم حق بهره‌مندی عادلانه از منابع ملی را دارد.} \\
\hline

۱۲.۲ & \rl{مناطق قومی محروم حق دریافت منابع جبرانی برای توسعه را دارند.} \\
\hline

۱۲.۳ & \rl{منابع طبیعی واقع در سرزمین یک قوم، متعلق به کل ملت است اما آن منطقه سهم ویژه‌ای دریافت می‌کند.} \\
\hline

۱۲.۴ & \rl{پروژه‌های توسعه در مناطق قومی باید با مشارکت و رضایت جوامع محلی انجام شود.} \\
\hline

۱۲.۵ & \rl{جابجایی اجباری جمعیت از سرزمین‌های سنتی ممنوع است مگر در شرایط اضطراری و با جبران کامل.} \\
\hline

\end{longtable}

% ═══════════════════════════════════════════════════════════════════════════════
\section{فصل چهارم: حقوق زبانی}
\label{sec:charter-ch4}
% ═══════════════════════════════════════════════════════════════════════════════

\begin{center}
\begin{tikzpicture}
    % پیرامید زبانی
    \node[
        draw=golddark,
        line width=2pt,
        fill=golddark!30,
        trapezium,
        trapezium left angle=70,
        trapezium right angle=110,
        minimum width=10cm,
        minimum height=1.5cm,
        font=\bfseries
    ] (l1) at (0,3) {\rl{زبان فارسی: زبان رسمی مشترک سراسری}};
    
    \node[
        draw=golddark,
        line width=1.5pt,
        fill=golddark!20,
        trapezium,
        trapezium left angle=70,
        trapezium right angle=110,
        minimum width=12cm,
        minimum height=1.5cm,
        font=\bfseries
    ] (l2) at (0,1) {\rl{زبان‌های ملی: رسمی در مناطق مربوطه}};
    
    \node[
        draw=golddark,
        line width=1pt,
        fill=golddark!10,
        trapezium,
        trapezium left angle=70,
        trapezium right angle=110,
        minimum width=14cm,
        minimum height=1.5cm,
        font=\small
    ] (l3) at (0,-1) {\rl{زبان‌های محلی و گویش‌ها: حمایت و حفاظت}};
    
    % فهرست زبان‌ها
    \node[font=\scriptsize, align=center] at (0,0) {\rl{آذری | کردی | عربی | بلوچی | ترکمنی | لری | گیلکی | مازندرانی}};
\end{tikzpicture}
\end{center}

\subsection{ماده ۱۳: وضعیت زبان‌ها}

\begin{longtable}{|>{\columncolor{WisdomGold!15}}r|p{12cm}|}
\hline
\rowcolor{WisdomGold!30}
\textbf{بند} & \textbf{متن} \\
\hline
\endfirsthead

۱۳.۱ & زبان فارسی زبان رسمی مشترک جمهوری فدرال ایران است که همه شهروندان حق و وظیفه یادگیری آن را دارند. \\
\hline

۱۳.۲ & زبان‌های آذری، کردی، عربی، بلوچی، ترکمنی، لری، گیلکی و مازندرانی زبان‌های ملی ایران هستند. \\
\hline

۱۳.۳ & در مناطق خودمختار، زبان قومی آن منطقه زبان رسمی دوم است. \\
\hline

۱۳.۴ & همه زبان‌ها و گویش‌های موجود در ایران، صرف‌نظر از تعداد گویشوران، شایسته حفاظت هستند. \\
\hline

\end{longtable}

\subsection{ماده ۱۴: حقوق زبانی در آموزش}

\begin{longtable}{|>{\columncolor{WisdomGold!15}}r|p{12cm}|}
\hline
\rowcolor{WisdomGold!30}
\textbf{بند} & \textbf{متن} \\
\hline
\endfirsthead

۱۴.۱ & آموزش به زبان مادری در مقطع پیش‌دبستانی و دبستان حق هر کودک است. \\
\hline

۱۴.۲ & زبان فارسی از سال سوم دبستان به‌عنوان زبان دوم آموزش داده می‌شود و به‌تدریج زبان آموزش می‌شود. \\
\hline

۱۴.۳ & در مناطق دوزبانه، نظام آموزشی دوزبانه (زبان مادری + فارسی) برقرار می‌شود. \\
\hline

۱۴.۴ & یادگیری یک زبان ملی دیگر (غیر از زبان مادری و فارسی) در مدارس تشویق می‌شود. \\
\hline

۱۴.۵ & دولت موظف به تربیت معلم، تألیف کتاب و تأمین منابع آموزشی به زبان‌های ملی است. \\
\hline

۱۴.۶ & دانشگاه‌ها می‌توانند رشته‌هایی به زبان‌های ملی ارائه دهند. \\
\hline

\end{longtable}

\subsection{ماده ۱۵: حقوق زبانی در خدمات عمومی}

\begin{center}
\begin{small}
\begin{center}
\begin{small}
\begin{longtable}{|>{\columncolor{goldlight!10}}p{3cm}|p{4cm}|p{4cm}|p{3cm}|}
\hline
\rowcolor{goldlight!30}
\textbf{\rl{حوزه}} & \textbf{\rl{مناطق خودمختار}} & \textbf{\rl{مناطق مختلط}} & \textbf{\rl{سایر مناطق}} \\
\hline
\endfirsthead

\rl{اسناد هویتی} & \rl{دوزبانه (فارسی + محلی)} & \rl{دوزبانه به درخواست} & \rl{فارسی} \\
\hline

\rl{تابلوها و علائم} & \rl{دوزبانه اجباری} & \rl{دوزبانه در شهرهای بزرگ} & \rl{فارسی} \\
\hline

\rl{خدمات اداری} & \rl{دوزبانه} & \rl{مترجم در دسترس} & \rl{فارسی + مترجم} \\
\hline

\rl{دادگاه‌ها} & \rl{دوزبانه} & \rl{مترجم رایگان} & \rl{مترجم رایگان} \\
\hline

\rl{بهداشت و درمان} & \rl{دوزبانه} & \rl{مترجم بهداشتی} & \rl{مترجم در دسترس} \\
\hline

\rl{رسانه عمومی} & \rl{شبکه محلی به زبان قومی} & \rl{برنامه‌های چندزبانه} & \rl{برنامه‌های زبان‌های ملی} \\
\hline

\end{longtable}
\end{small}
\end{center}
\end{small}
\end{center}

\subsection{ماده ۱۶: آکادمی زبان‌های ایران}

\begin{olgoobox}
\textbf{آکادمی زبان‌های ایران}

\textbf{ماده ۱۶.۱:} آکادمی زبان‌های ایران به عنوان نهاد مستقل علمی تأسیس می‌شود.

\textbf{وظایف آکادمی:}
\begin{itemize}[nosep]
    \item پژوهش، مستندسازی و حفاظت از همه زبان‌ها و گویش‌های ایران
    \item استانداردسازی خط و املای زبان‌های ملی
    \item تدوین فرهنگ لغت و دستور زبان
    \item حفاظت از زبان‌های در خطر انقراض
    \item ترویج چندزبانگی و یادگیری زبان‌ها
\end{itemize}

\textbf{ماده ۱۶.۲:} هیئت‌مدیره آکادمی شامل نمایندگان همه زبان‌های ملی است.

\textbf{ماده ۱۶.۳:} بودجه آکادمی باید کافی و تضمین‌شده باشد.
\end{olgoobox}

% ═══════════════════════════════════════════════════════════════════════════════
\section{فصل پنجم: حقوق فرهنگی}
\label{sec:charter-ch5}
% ═══════════════════════════════════════════════════════════════════════════════

\subsection{ماده ۱۷: حفاظت از میراث فرهنگی}

\begin{longtable}{|>{\columncolor{purple!15}}r|p{12cm}|}
\hline
\rowcolor{purple!30}
\textbf{بند} & \textbf{متن} \\
\hline
\endfirsthead

۱۷.۱ & میراث فرهنگی هر قوم بخشی از میراث ملی ایران است و تحت حمایت دولت قرار دارد. \\
\hline

۱۷.۲ & آثار تاریخی، باستانی و فرهنگی هر منطقه باید در همان منطقه حفظ و نگهداری شود. \\
\hline

۱۷.۳ & نام‌های تاریخی شهرها، روستاها و مکان‌ها باید حفظ شود. تغییر اجباری نام‌های تاریخی ممنوع است. \\
\hline

۱۷.۴ & بازگرداندن نام‌های تاریخی که در گذشته به‌اجبار تغییر کرده‌اند، مجاز است. \\
\hline

۱۷.۵ & موزه‌های قومی-منطقه‌ای برای حفظ و نمایش میراث فرهنگی تأسیس می‌شوند. \\
\hline

\end{longtable}

\subsection{ماده ۱۸: هنر و ادبیات}

\begin{longtable}{|>{\columncolor{purple!15}}r|p{12cm}|}
\hline
\rowcolor{purple!30}
\textbf{بند} & \textbf{متن} \\
\hline
\endfirsthead

۱۸.۱ & هنرمندان و نویسندگان هر قوم حق آفرینش آزاد به زبان و سبک خود را دارند. \\
\hline

۱۸.۲ & دولت موظف به حمایت از تولید آثار ادبی و هنری به زبان‌های ملی است. \\
\hline

۱۸.۳ & ادبیات کودک به زبان‌های ملی باید تولید و در دسترس قرار گیرد. \\
\hline

۱۸.۴ & جشنواره‌های فرهنگی قومی با حمایت دولت برگزار می‌شود. \\
\hline

۱۸.۵ & ترجمه آثار ادبی بین زبان‌های ملی تشویق و حمایت می‌شود. \\
\hline

\end{longtable}

\subsection{ماده ۱۹: رسانه و ارتباطات}

\begin{longtable}{|>{\columncolor{purple!15}}r|p{12cm}|}
\hline
\rowcolor{purple!30}
\textbf{بند} & \textbf{متن} \\
\hline
\endfirsthead

۱۹.۱ & هر قوم حق داشتن رسانه‌های چاپی، صوتی، تصویری و دیجیتال به زبان خود را دارد. \\
\hline

۱۹.۲ & صدا و سیمای ملی موظف به پخش برنامه به همه زبان‌های ملی به نسبت جمعیت است. \\
\hline

۱۹.۳ & هر منطقه خودمختار حق داشتن شبکه رادیویی و تلویزیونی محلی به زبان قومی را دارد. \\
\hline

۱۹.۴ & اینترنت و فضای مجازی به زبان‌های ملی باید توسعه یابد. \\
\hline

۱۹.۵ & دولت از تولید محتوای دیجیتال به زبان‌های ملی حمایت می‌کند. \\
\hline

\end{longtable}

\subsection{ماده ۲۰: آداب، رسوم و جشن‌ها}

\begin{longtable}{|>{\columncolor{purple!15}}r|p{12cm}|}
\hline
\rowcolor{purple!30}
\textbf{بند} & \textbf{متن} \\
\hline
\endfirsthead

۲۰.۱ & هر قوم حق برگزاری جشن‌ها و مراسم سنتی خود را دارد. \\
\hline

۲۰.۲ & روزهای مهم فرهنگی هر قوم (مانند نوروز، یلدا، سده، مهرگان، عید فطر، عید قربان) می‌تواند در منطقه مربوطه تعطیل رسمی باشد. \\
\hline

۲۰.۳ & تقویم ملی باید شامل روزهای فرهنگی همه اقوام باشد. \\
\hline

۲۰.۴ & موسیقی، رقص و هنرهای نمایشی سنتی هر قوم باید حفظ و آموزش داده شود. \\
\hline

\end{longtable}

% ═══════════════════════════════════════════════════════════════════════════════
\section{فصل ششم: حقوق سرزمینی و توسعه}
\label{sec:charter-ch6}
% ═══════════════════════════════════════════════════════════════════════════════

\begin{center}
\begin{tikzpicture}[scale=0.9]
    % نقشه شماتیک ایران با مناطق
    \draw[thick, bleurepublique, fill=bleurepublique!10, rounded corners=10pt] 
        (-6,-3) rectangle (6,4);
    
    % عنوان
    \node[font=\large\bfseries] at (0,3.5) {\rl{مناطق خودمختار جمهوری فدرال ایران}};
    
    % مناطق (شماتیک)
    \node[
        draw=bleurepublique,
        fill=bleurepublique!30,
        rounded corners=5pt,
        minimum width=2.5cm,
        minimum height=1.5cm,
        align=center,
        font=\small
    ] at (-4,1.5) {\textbf{\rl{آذربایجان}}\\\rl{تبریز}};
    
    \node[
        draw=goldlight,
        fill=goldlight!30,
        rounded corners=5pt,
        minimum width=2.5cm,
        minimum height=1.5cm,
        align=center,
        font=\small
    ] at (-1.5,1.5) {\textbf{\rl{کردستان}}\\\rl{سنندج}};
    
    \node[
        draw=bleulight,
        fill=bleulight!30,
        rounded corners=5pt,
        minimum width=2.5cm,
        minimum height=1.5cm,
        align=center,
        font=\small
    ] at (-4,-1) {\textbf{\rl{خوزستان}}\\\rl{اهواز}};
    
    \node[
        draw=golddark,
        fill=golddark!30,
        rounded corners=5pt,
        minimum width=2.5cm,
        minimum height=1.5cm,
        align=center,
        font=\small
    ] at (4,-1) {\textbf{\rl{بلوچستان}}\\\rl{زاهدان}};
    
    \node[
        draw=bleurepublique!70,
        fill=bleurepublique!20,
        rounded corners=5pt,
        minimum width=2.5cm,
        minimum height=1.5cm,
        align=center,
        font=\small
    ] at (2,1.5) {\textbf{\rl{ترکمن‌صحرا}}\\\rl{گنبد}};
    
    % مرکز (پایتخت)
    \node[
        draw=gray,
        fill=gray!20,
        circle,
        minimum size=1.5cm,
        font=\small\bfseries
    ] at (0,0) {\rl{تهران}};
\end{tikzpicture}
\end{center}

\subsection{ماده ۲۱: حق سرزمین}

\begin{longtable}{|>{\columncolor{golddark!15}}r|p{12cm}|}
\hline
\rowcolor{golddark!30}
\textbf{\rl{بند}} & \textbf{\rl{متن}} \\
\hline
\endfirsthead

۲۱.۱ & \rl{هر قوم حق زندگی در سرزمین تاریخی خود را دارد.} \\
\hline

۲۱.۲ & \rl{تغییر ترکیب جمعیتی عمدی مناطق قومی (مهندسی جمعیتی) ممنوع است.} \\
\hline

۲۱.۳ & \rl{جابجایی اجباری جمعیت از سرزمین‌های سنتی ممنوع است مگر در موارد اضطراری طبیعی و با جبران کامل و حق بازگشت.} \\
\hline

۲۱.۴ & \rl{مرزهای مناطق خودمختار تنها با رضایت ساکنان (همه‌پرسی) قابل تغییر است.} \\
\hline

۲۱.۵ & \rl{شهروندان همه اقوام حق سکونت در هر نقطه از کشور را دارند.} \\
\hline

\end{longtable}

\subsection{ماده ۲۲: توسعه متوازن}

\begin{center}
\begin{tikzpicture}
    % فرمول توزیع منابع
    \node[
        draw=SuccessGreen,
        line width=2pt,
        fill=SuccessGreen!10,
        rounded corners=10pt,
        minimum width=14cm,
        minimum height=3.5cm
    ] (formula) at (0,0) {};
    
    \node[font=\large\bfseries, SuccessGreen] at (0,1.2) {فرمول توزیع منابع صندوق توازن منطقه‌ای};
    
    % اجزای فرمول
    \node[
        draw=DemocracyBlue,
        fill=DemocracyBlue!20,
        rounded corners=5pt,
        minimum width=2cm,
        minimum height=1cm,
        font=\small
    ] at (-5,-0.5) {\shortstack{جمعیت\\۵۰٪}};
    
    \node[font=\Large] at (-3.5,-0.5) {+};
    
    \node[
        draw=WisdomGold,
        fill=WisdomGold!20,
        rounded corners=5pt,
        minimum width=2cm,
        minimum height=1cm,
        font=\small
    ] at (-2,-0.5) {\shortstack{مساحت\\۲۰٪}};
    
    \node[font=\Large] at (-0.5,-0.5) {+};
    
    \node[
        draw=WarningRed,
        fill=WarningRed!20,
        rounded corners=5pt,
        minimum width=2cm,
        minimum height=1cm,
        font=\small
    ] at (1,-0.5) {\shortstack{محرومیت\\۲۰٪}};
    
    \node[font=\Large] at (2.5,-0.5) {+};
    
    \node[
        draw=purple,
        fill=purple!20,
        rounded corners=5pt,
        minimum width=2cm,
        minimum height=1cm,
        font=\small
    ] at (4,-0.5) {\shortstack{عملکرد\\۱۰٪}};
\end{tikzpicture}
\end{center}

\begin{longtable}{|>{\columncolor{orange!15}}r|p{12cm}|}
\hline
\rowcolor{orange!30}
\textbf{بند} & \textbf{متن} \\
\hline
\endfirsthead

۲۲.۱ & دولت موظف به تضمین توسعه متوازن همه مناطق کشور است. \\
\hline

۲۲.۲ & صندوق توازن منطقه‌ای برای کاهش شکاف توسعه بین مناطق تأسیس می‌شود. \\
\hline

۲۲.۳ & مناطق محروم حق دریافت منابع ویژه توسعه را دارند. \\
\hline

۲۲.۴ & شاخص‌های توسعه (درآمد، آموزش، بهداشت، زیرساخت) در همه مناطق باید به‌طور سالانه اندازه‌گیری و منتشر شود. \\
\hline

۲۲.۵ & هدف: رسیدن به حداکثر ۲۰٪ تفاوت در شاخص توسعه بین پیشرفته‌ترین و محروم‌ترین مناطق در افق ۲۰ ساله. \\
\hline

\end{longtable}

\subsection{ماده ۲۳: منابع طبیعی}

\begin{longtable}{|>{\columncolor{orange!15}}r|p{12cm}|}
\hline
\rowcolor{orange!30}
\textbf{بند} & \textbf{متن} \\
\hline
\endfirsthead

۲۳.۱ & منابع طبیعی (نفت، گاز، معادن، آب) متعلق به کل ملت ایران است. \\
\hline

۲۳.۲ & مناطق تولیدکننده منابع طبیعی سهم ویژه (۵-۱۵٪ از درآمد) دریافت می‌کنند. \\
\hline

۲۳.۳ & بهره‌برداری از منابع طبیعی باید با رعایت حقوق محیط زیست و جوامع محلی باشد. \\
\hline

۲۳.۴ & جوامع محلی حق اطلاع و مشورت در پروژه‌های استخراجی را دارند. \\
\hline

۲۳.۵ & خسارات زیست‌محیطی به مناطق باید جبران شود. \\
\hline

\end{longtable}

% ═══════════════════════════════════════════════════════════════════════════════
\section{فصل هفتم: سازوکارهای حمایتی و نظارتی}
\label{sec:charter-ch7}
% ═══════════════════════════════════════════════════════════════════════════════

\begin{center}
\begin{tikzpicture}[
    inst/.style={
        draw=#1,
        line width=1.5pt,
        fill=#1!15,
        rounded corners=8pt,
        minimum width=4cm,
        minimum height=2cm,
        align=center,
        font=\small\bfseries
    }
]
    % عنوان
    \node[
        fill=WarningRed,
        text=white,
        rounded corners=5pt,
        font=\large\bfseries,
        minimum width=10cm
    ] at (0,5) {نهادهای حمایت از حقوق اقوام};
    
    % نهادها
    \node[inst=DemocracyBlue] (ma) at (-4,2.5) {\shortstack{مجلس اقوام\\(نمایندگی)}};
    
    \node[inst=SuccessGreen] (cm) at (0,2.5) {\shortstack{کمیسیون ملی\\حقوق اقوام\\(نظارت)}};
    
    \node[inst=WisdomGold] (cc) at (4,2.5) {\shortstack{دیوان قانون اساسی\\(دادرسی)}};
    
    \node[inst=purple] (om) at (-2,0) {\shortstack{دفاتر نماینده\\حقوق اقوام\\(میانجیگری)}};
    
    \node[inst=orange] (la) at (2,0) {\shortstack{آکادمی زبان‌ها\\(علمی-فرهنگی)}};
    
    % اتصالات
    \draw[<->, thick, gray] (ma) -- (cm);
    \draw[<->, thick, gray] (cm) -- (cc);
    \draw[<->, thick, gray] (cm) -- (om);
    \draw[<->, thick, gray] (cm) -- (la);
\end{tikzpicture}
\end{center}

\subsection{ماده ۲۴: مجلس اقوام}

\begin{longtable}{|>{\columncolor{DemocracyBlue!15}}r|p{12cm}|}
\hline
\rowcolor{DemocracyBlue!30}
\textbf{بند} & \textbf{متن} \\
\hline
\endfirsthead

۲۴.۱ & مجلس اقوام (مجلس دوم پارلمان) نهاد عالی نمایندگی اقوام در سطح ملی است. \\
\hline

۲۴.۲ & ترکیب: ۱۲۰ نماینده شامل نمایندگان استان‌ها، مناطق خودمختار، اقلیت‌های زبانی و ایرانیان خارج. \\
\hline

۲۴.۳ & صلاحیت‌های ویژه مجلس اقوام:
\begin{itemize}[nosep]
    \item حق وتو بر قوانین مرتبط با حقوق اقوام
    \item حق وتو بر تغییرات در ساختار فدرال
    \item تأیید انتصاب قضات دیوان عالی و قانون اساسی
    \item نظارت بر توزیع عادلانه منابع ملی
\end{itemize} \\
\hline

۲۴.۴ & هر قوم به رسمیت شناخته‌شده حداقل ۲ نماینده در مجلس اقوام دارد. \\
\hline

\end{longtable}

\subsection{ماده ۲۵: کمیسیون ملی حقوق اقوام}

\begin{longtable}{|>{\columncolor{SuccessGreen!15}}r|p{12cm}|}
\hline
\rowcolor{SuccessGreen!30}
\textbf{بند} & \textbf{متن} \\
\hline
\endfirsthead

۲۵.۱ & کمیسیون ملی حقوق اقوام نهاد مستقل نظارت بر اجرای این منشور و حمایت از حقوق اقوام است. \\
\hline

۲۵.۲ & \textbf{ترکیب کمیسیون:}
\begin{itemize}[nosep]
    \item ۱۵ عضو از نمایندگان اقوام مختلف
    \item ۵ حقوقدان متخصص حقوق اقلیت‌ها
    \item ۳ نماینده سازمان‌های مدنی
    \item دوره عضویت: ۶ سال غیرقابل تمدید
\end{itemize} \\
\hline

۲۵.۳ & \textbf{وظایف کمیسیون:}
\begin{itemize}[nosep]
    \item دریافت و رسیدگی به شکایات نقض حقوق اقوام
    \item نظارت بر اجرای این منشور و قوانین مرتبط
    \item ارائه گزارش سالانه به مجلس و عموم
    \item پیشنهاد اصلاح قوانین به مجلس
    \item آموزش و آگاهی‌رسانی عمومی
    \item میانجیگری در اختلافات قومی
\end{itemize} \\
\hline

۲۵.۴ & کمیسیون دارای استقلال مالی و اداری است و بودجه آن باید کافی و تضمین‌شده باشد. \\
\hline

۲۵.۵ & کمیسیون حق بازرسی از همه نهادهای دولتی در موضوعات مرتبط با حقوق اقوام را دارد. \\
\hline

۲۵.۶ & تصمیمات کمیسیون لازم‌الاجرا است. عدم اجرا قابل پیگیری در دیوان عدالت اداری است. \\
\hline

\end{longtable}

\subsection{ماده ۲۶: دادرسی قضایی}

\begin{longtable}{|>{\columncolor{WisdomGold!15}}r|p{12cm}|}
\hline
\rowcolor{WisdomGold!30}
\textbf{بند} & \textbf{متن} \\
\hline
\endfirsthead

۲۶.۱ & هر فرد یا گروه قومی که حقوقش طبق این منشور نقض شده، حق طرح شکایت در دادگاه‌ها را دارد. \\
\hline

۲۶.۲ & دیوان عالی قانون اساسی مرجع نهایی تفسیر این منشور و رسیدگی به دعاوی اساسی حقوق اقوام است. \\
\hline

۲۶.۳ & شُعَب تخصصی حقوق اقوام در دادگاه‌های استان‌ها تشکیل می‌شود. \\
\hline

۲۶.۴ & در مناطق دوزبانه، محاکمات می‌تواند به زبان محلی برگزار شود. \\
\hline

۲۶.۵ & معاضدت قضایی رایگان برای دعاوی حقوق اقوام تضمین می‌شود. \\
\hline

۲۶.۶ & احکام قطعی در موضوعات حقوق اقوام به‌صورت عمومی منتشر می‌شود. \\
\hline

\end{longtable}

\subsection{ماده ۲۷: نماینده حقوق اقوام (آمبودزمان)}

\begin{olgoobox}
\textbf{دفتر نماینده حقوق اقوام}

\textbf{ماده ۲۷.۱:} در هر منطقه خودمختار و استان، دفتر نماینده حقوق اقوام تأسیس می‌شود.

\textbf{وظایف نماینده:}
\begin{itemize}[nosep]
    \item دریافت شکایات شهروندان در موضوعات قومی
    \item میانجیگری و حل‌وفصل اختلافات
    \item ارجاع موارد جدی به کمیسیون ملی یا دادگاه
    \item آموزش و آگاهی‌رسانی در سطح محلی
    \item ارائه گزارش دوره‌ای به کمیسیون ملی
\end{itemize}

\textbf{ماده ۲۷.۲:} نماینده باید از اهالی منطقه و آشنا به زبان و فرهنگ محلی باشد.

\textbf{ماده ۲۷.۳:} دسترسی به نماینده رایگان و بدون تشریفات است.
\end{olgoobox}

% ═══════════════════════════════════════════════════════════════════════════════
\section{فصل هشتم: ممنوعیت‌ها و ضمانت‌های اجرایی}
\label{sec:charter-ch8}
% ═══════════════════════════════════════════════════════════════════════════════

\subsection{ماده ۲۸: اقدامات ممنوع}

\begin{enghelabbox}
\textbf{⚠️ اقدامات ممنوع علیه اقوام}

اقدامات زیر ممنوع و جرم محسوب می‌شود:

\begin{enumerate}[nosep]
    \item \textbf{نسل‌کشی فرهنگی:} هرگونه اقدام سیستماتیک برای نابودی زبان، فرهنگ یا هویت یک قوم
    
    \item \textbf{همسان‌سازی اجباری:} اجبار به ترک زبان مادری، تغییر نام، یا انکار هویت قومی
    
    \item \textbf{تبعیض سیستماتیک:} محرومیت عمدی یک قوم از حقوق، منابع یا فرصت‌ها
    
    \item \textbf{مهندسی جمعیتی:} تغییر عمدی ترکیب جمعیتی مناطق برای کاهش نسبت یک قوم
    
    \item \textbf{جابجایی اجباری:} کوچاندن اجباری جمعیت از سرزمین‌های سنتی
    
    \item \textbf{نفرت‌پراکنی قومی:} تحریک به خشونت یا نفرت علیه یک قوم
    
    \item \textbf{توهین قومی:} تحقیر، استهزا یا اهانت به هویت، زبان یا فرهنگ یک قوم
    
    \item \textbf{سرکوب زبانی:} ممنوعیت یا محدودیت استفاده از زبان مادری
    
    \item \textbf{تخریب میراث فرهنگی:} نابودی عمدی آثار تاریخی، فرهنگی یا مذهبی یک قوم
    
    \item \textbf{انکار وجود:} انکار رسمی وجود یا هویت یک قوم
\end{enumerate}
\end{enghelabbox}

\subsection{ماده ۲۹: مجازات‌ها}

\begin{longtable}{|>{\columncolor{WarningRed!15}}p{4cm}|p{4cm}|p{5cm}|}
\hline
\rowcolor{WarningRed!30}
\textbf{جرم} & \textbf{مجازات افراد} & \textbf{مجازات مقامات/نهادها} \\
\hline
\endfirsthead

نسل‌کشی فرهنگی & ۱۰-۲۵ سال حبس & انحلال نهاد + پیگرد مسئولان \\
\hline

تبعیض سیستماتیک & ۵-۱۵ سال حبس + جریمه & انفصال دائم + جریمه سنگین \\
\hline

مهندسی جمعیتی & ۵-۱۰ سال حبس & ابطال تصمیمات + انفصال \\
\hline

نفرت‌پراکنی قومی & ۲-۷ سال حبس + جریمه & انفصال + محرومیت از فعالیت \\
\hline

توهین قومی & ۶ ماه تا ۳ سال حبس + جریمه & انفصال موقت + جریمه \\
\hline

سرکوب زبانی & ۱-۵ سال حبس & ابطال مقررات + جبران خسارت \\
\hline

تخریب میراث فرهنگی & ۵-۱۵ سال حبس + جبران & انحلال + جریمه سنگین \\
\hline

\end{longtable}

\subsection{ماده ۳۰: جبران خسارت}

\begin{longtable}{|>{\columncolor{WarningRed!15}}r|p{12cm}|}
\hline
\rowcolor{WarningRed!30}
\textbf{بند} & \textbf{متن} \\
\hline
\endfirsthead

۳۰.۱ & قربانیان نقض حقوق اقوام حق جبران کامل خسارت را دارند. \\
\hline

۳۰.۲ & جبران شامل: غرامت مالی، اعاده وضع، بازسازی، عذرخواهی رسمی و تضمین عدم تکرار. \\
\hline

۳۰.۳ & در موارد نقض سیستماتیک، جبران جمعی به نفع کل قوم آسیب‌دیده صورت می‌گیرد. \\
\hline

۳۰.۴ & صندوق جبران خسارت قربانیان تبعیض قومی تأسیس می‌شود. \\
\hline

۳۰.۵ & مرور زمان شامل جرایم علیه اقوام نمی‌شود. \\
\hline

\end{longtable}

% ═══════════════════════════════════════════════════════════════════════════════
\section{فصل نهم: عدالت انتقالی و جبران تاریخی}
\label{sec:charter-ch9}
% ═══════════════════════════════════════════════════════════════════════════════

\begin{center}
\begin{tikzpicture}
    % عنوان
    \node[
        fill=purple,
        text=white,
        rounded corners=5pt,
        font=\large\bfseries,
        minimum width=10cm
    ] at (0,4) {عدالت انتقالی برای اقوام};
    
    % چهار رکن
    \node[
        draw=DemocracyBlue,
        fill=DemocracyBlue!20,
        rounded corners=8pt,
        minimum width=3cm,
        minimum height=2cm,
        align=center,
        font=\small\bfseries
    ] (t1) at (-5,1.5) {\shortstack{حقیقت‌یابی\\\\کمیسیون حقیقت\\و آشتی}};
    
    \node[
        draw=SuccessGreen,
        fill=SuccessGreen!20,
        rounded corners=8pt,
        minimum width=3cm,
        minimum height=2cm,
        align=center,
        font=\small\bfseries
    ] (t2) at (-1.5,1.5) {\shortstack{جبران خسارت\\\\مادی و معنوی\\فردی و جمعی}};
    
    \node[
        draw=WisdomGold,
        fill=WisdomGold!20,
        rounded corners=8pt,
        minimum width=3cm,
        minimum height=2cm,
        align=center,
        font=\small\bfseries
    ] (t3) at (2,1.5) {\shortstack{پیگرد قضایی\\\\عاملان اصلی\\نقض حقوق}};
    
    \node[
        draw=WarningRed,
        fill=WarningRed!20,
        rounded corners=8pt,
        minimum width=3cm,
        minimum height=2cm,
        align=center,
        font=\small\bfseries
    ] (t4) at (5.5,1.5) {\shortstack{تضمین عدم تکرار\\\\اصلاحات نهادی\\آموزش}};
    
    % فلش مرکزی
    \node[
        draw=gray,
        fill=gray!10,
        rounded corners=5pt,
        minimum width=12cm,
        minimum height=1cm,
        font=\small
    ] at (0,-1) {هدف: آشتی ملی بر پایه حقیقت، عدالت و احترام متقابل};
\end{tikzpicture}
\end{center}

\subsection{ماده ۳۱: شناسایی بی‌عدالتی‌های تاریخی}

\begin{longtable}{|>{\columncolor{purple!15}}r|p{12cm}|}
\hline
\rowcolor{purple!30}
\textbf{بند} & \textbf{متن} \\
\hline
\endfirsthead

۳۱.۱ & جمهوری فدرال ایران به‌رسمیت می‌شناسد که در گذشته بی‌عدالتی‌هایی علیه اقوام مختلف روا داشته شده است. \\
\hline

۳۱.۲ & این بی‌عدالتی‌ها شامل اما نه محدود به موارد زیر است:
\begin{itemize}[nosep]
    \item ممنوعیت و سرکوب زبان‌های مادری
    \item تبعیض در توزیع منابع و توسعه
    \item سرکوب سیاسی جنبش‌های قومی
    \item تغییر اجباری نام‌های تاریخی
    \item نادیده گرفتن هویت و فرهنگ اقوام
    \item مهندسی جمعیتی در برخی مناطق
\end{itemize} \\
\hline

۳۱.۳ & شناسایی این بی‌عدالتی‌ها گام نخست به سوی آشتی ملی است. \\
\hline

\end{longtable}

\subsection{ماده ۳۲: کمیسیون حقیقت و آشتی قومی}

\begin{longtable}{|>{\columncolor{purple!15}}r|p{12cm}|}
\hline
\rowcolor{purple!30}
\textbf{بند} & \textbf{متن} \\
\hline
\endfirsthead

۳۲.۱ & کمیسیون حقیقت و آشتی قومی برای بررسی نقض حقوق اقوام در گذشته تشکیل می‌شود. \\
\hline

۳۲.۲ & \textbf{وظایف کمیسیون:}
\begin{itemize}[nosep]
    \item مستندسازی موارد نقض حقوق اقوام
    \item شنیدن شهادت قربانیان و شاهدان
    \item شناسایی عاملان و ساختارهای مسئول
    \item تدوین گزارش جامع و توصیه‌ها
    \item پیشنهاد مصادیق جبران خسارت
\end{itemize} \\
\hline

۳۲.۳ & \textbf{ترکیب:} ۲۱ عضو شامل نمایندگان همه اقوام، حقوقدانان، مورخان و فعالان حقوق بشر \\
\hline

۳۲.۴ & مدت فعالیت: ۳ سال با امکان تمدید یک‌ساله \\
\hline

۳۲.۵ & گزارش نهایی کمیسیون به‌صورت عمومی منتشر و در مدارس تدریس می‌شود. \\
\hline

\end{longtable}

\subsection{ماده ۳۳: اقدامات جبرانی جمعی}

\begin{longtable}{|>{\columncolor{purple!15}}r|p{12cm}|}
\hline
\rowcolor{purple!30}
\textbf{بند} & \textbf{متن} \\
\hline
\endfirsthead

۳۳.۱ & \textbf{اقدامات نمادین:}
\begin{itemize}[nosep]
    \item عذرخواهی رسمی دولت از اقوام آسیب‌دیده
    \item تعیین روز ملی تنوع و آشتی قومی
    \item احداث یادبودها و موزه‌های تاریخ اقوام
    \item گنجاندن تاریخ واقعی اقوام در کتب درسی
\end{itemize} \\
\hline

۳۳.۲ & \textbf{اقدامات مادی:}
\begin{itemize}[nosep]
    \item صندوق جبران برای توسعه مناطق محروم قومی
    \item بورسیه تحصیلی ویژه برای جوانان اقوام محروم
    \item سرمایه‌گذاری در زیرساخت‌های مناطق آسیب‌دیده
    \item حمایت از احیای زبان‌ها و فرهنگ‌های آسیب‌دیده
\end{itemize} \\
\hline

۳۳.۳ & \textbf{اقدامات ساختاری:}
\begin{itemize}[nosep]
    \item تبعیض مثبت موقت در استخدام دولتی
    \item سهمیه آموزش عالی برای مناطق محروم
    \item اولویت توسعه برای مناطق تاریخاً محروم
\end{itemize} \\
\hline

\end{longtable}

\subsection{ماده ۳۴: بازگرداندن نام‌های تاریخی}

\begin{longtable}{|>{\columncolor{purple!15}}r|p{12cm}|}
\hline
\rowcolor{purple!30}
\textbf{بند} & \textbf{متن} \\
\hline
\endfirsthead

۳۴.۱ & شهرها، روستاها، خیابان‌ها و مکان‌هایی که نام تاریخی آنها به‌اجبار تغییر کرده، می‌توانند نام اصلی خود را بازیابند. \\
\hline

۳۴.۲ & درخواست بازگرداندن نام توسط شورای محلی یا درخواست ۱۰٪ ساکنان مطرح می‌شود. \\
\hline

۳۴.۳ & تصمیم نهایی با همه‌پرسی محلی گرفته می‌شود. \\
\hline

۳۴.۴ & کمیته ملی نام‌های جغرافیایی برای بررسی و مشورت تشکیل می‌شود. \\
\hline

\end{longtable}

% ═══════════════════════════════════════════════════════════════════════════════
\section{فصل دهم: تبعیض مثبت و برابری واقعی}
\label{sec:charter-ch10}
% ═══════════════════════════════════════════════════════════════════════════════

\subsection{ماده ۳۵: اصل تبعیض مثبت}

\begin{olgoobox}
\textbf{تبعیض مثبت: ابزار برابری واقعی}

\textbf{ماده ۳۵.۱:} برای جبران نابرابری‌های تاریخی و ایجاد برابری واقعی، اقدامات تبعیض مثبت موقت به نفع اقوام محروم مجاز و توصیه‌شده است.

\textbf{ماده ۳۵.۲:} تبعیض مثبت نقض اصل برابری نیست، بلکه ابزار تحقق برابری واقعی است.

\textbf{ماده ۳۵.۳:} اقدامات تبعیض مثبت موقت هستند و پس از رفع نابرابری باید پایان یابند.

\textbf{ماده ۳۵.۴:} کمیسیون ملی حقوق اقوام بر اجرا و ضرورت ادامه این اقدامات نظارت دارد.
\end{olgoobox}

\subsection{ماده ۳۶: حوزه‌های تبعیض مثبت}

\begin{center}
\begin{small}
\begin{longtable}{|>{\columncolor{SuccessGreen!10}}p{2.5cm}|p{5cm}|p{3cm}|p{3cm}|}
\hline
\rowcolor{SuccessGreen!30}
\textbf{حوزه} & \textbf{اقدام} & \textbf{هدف} & \textbf{مدت پیش‌بینی} \\
\hline
\endfirsthead

استخدام دولتی & سهمیه متناسب با جمعیت قومی منطقه & نمایندگی عادلانه & ۱۵ سال \\
\hline

آموزش عالی & بورسیه و سهمیه برای مناطق محروم & دسترسی برابر & ۲۰ سال \\
\hline

انتصابات عالی & لحاظ تنوع قومی در کابینه و نهادها & مشارکت در قدرت & دائمی \\
\hline

بودجه توسعه & تخصیص ویژه به مناطق محروم & توسعه متوازن & تا رفع شکاف \\
\hline

رسانه & سهم متناسب در صدا و سیما & نمایندگی رسانه‌ای & ۱۰ سال \\
\hline

قضاوت & تربیت قاضی از اقوام مختلف & عدالت چندفرهنگی & ۱۵ سال \\
\hline

\end{longtable}
\end{small}
\end{center}

\subsection{ماده ۳۷: نظارت بر برابری}

\begin{longtable}{|>{\columncolor{SuccessGreen!15}}r|p{12cm}|}
\hline
\rowcolor{SuccessGreen!30}
\textbf{بند} & \textbf{متن} \\
\hline
\endfirsthead

۳۷.۱ & شاخص‌های برابری قومی به‌صورت سالانه اندازه‌گیری و منتشر می‌شود. \\
\hline

۳۷.۲ & \textbf{شاخص‌های کلیدی:}
\begin{itemize}[nosep]
    \item نرخ فقر به تفکیک قومی و منطقه‌ای
    \item سطح تحصیلات به تفکیک قومی
    \item نرخ بیکاری به تفکیک قومی
    \item نمایندگی در مشاغل دولتی عالی
    \item دسترسی به خدمات بهداشتی و آموزشی
    \item زیرساخت‌های توسعه در مناطق قومی
\end{itemize} \\
\hline

۳۷.۳ & هدف: رسیدن به حداکثر ۱۵٪ تفاوت در شاخص‌های توسعه بین اقوام در افق ۲۵ ساله. \\
\hline

۳۷.۴ & گزارش سالانه برابری قومی به مجلس و عموم ارائه می‌شود. \\
\hline

\end{longtable}

% ═══════════════════════════════════════════════════════════════════════════════
\section{فصل یازدهم: آموزش و آگاهی}
\label{sec:charter-ch11}
% ═══════════════════════════════════════════════════════════════════════════════

\subsection{ماده ۳۸: آموزش چندفرهنگی}

\begin{longtable}{|>{\columncolor{DemocracyBlue!15}}r|p{12cm}|}
\hline
\rowcolor{DemocracyBlue!30}
\textbf{بند} & \textbf{متن} \\
\hline
\endfirsthead

۳۸.۱ & نظام آموزشی باید ترویج‌کننده احترام به تنوع قومی و فرهنگی باشد. \\
\hline

۳۸.۲ & تاریخ، فرهنگ و ادبیات همه اقوام ایران در کتب درسی گنجانده می‌شود. \\
\hline

۳۸.۳ & کتب درسی باید عاری از هرگونه تعصب، کلیشه یا تحقیر قومی باشند. \\
\hline

۳۸.۴ & کمیته بازنگری کتب درسی با مشارکت نمایندگان همه اقوام تشکیل می‌شود. \\
\hline

۳۸.۵ & آموزش شهروندی شامل آموزش حقوق اقوام و زندگی مسالمت‌آمیز است. \\
\hline

۳۸.۶ & معلمان باید دوره‌های آموزش چندفرهنگی را بگذرانند. \\
\hline

\end{longtable}

\subsection{ماده ۳۹: آگاهی‌رسانی عمومی}

\begin{longtable}{|>{\columncolor{DemocracyBlue!15}}r|p{12cm}|}
\hline
\rowcolor{DemocracyBlue!30}
\textbf{بند} & \textbf{متن} \\
\hline
\endfirsthead

۳۹.۱ & دولت موظف به ترویج فرهنگ احترام به تنوع قومی در جامعه است. \\
\hline

۳۹.۲ & رسانه‌های عمومی موظف به بازنمایی مثبت و متوازن همه اقوام هستند. \\
\hline

۳۹.۳ & کمپین‌های ملی مبارزه با تعصب و نفرت قومی برگزار می‌شود. \\
\hline

۳۹.۴ & روز ملی تنوع فرهنگی برای جشن گرفتن کثرت قومی تعیین می‌شود. \\
\hline

۳۹.۵ & جشنواره‌های فرهنگی بین‌قومی برای تقویت آشنایی و همبستگی برگزار می‌شود. \\
\hline

\end{longtable}

% ═══════════════════════════════════════════════════════════════════════════════
\section{فصل دوازدهم: مقررات نهایی}
\label{sec:charter-ch12}
% ═══════════════════════════════════════════════════════════════════════════════

\subsection{ماده ۴۰: جایگاه حقوقی منشور}

\begin{longtable}{|>{\columncolor{gray!15}}r|p{12cm}|}
\hline
\rowcolor{gray!30}
\textbf{بند} & \textbf{متن} \\
\hline
\endfirsthead

۴۰.۱ & این منشور جزء لاینفک قانون اساسی و دارای همان قوت حقوقی است. \\
\hline

۴۰.۲ & هیچ قانون عادی نمی‌تواند مغایر با این منشور باشد. \\
\hline

۴۰.۳ & تفسیر این منشور بر عهده دیوان عالی قانون اساسی است. \\
\hline

۴۰.۴ & در تفسیر، اصل تفسیر به نفع حقوق اقوام اعمال می‌شود. \\
\hline

\end{longtable}

\subsection{ماده ۴۱: بازنگری در منشور}

\begin{longtable}{|>{\columncolor{gray!15}}r|p{12cm}|}
\hline
\rowcolor{gray!30}
\textbf{بند} & \textbf{متن} \\
\hline
\endfirsthead

۴۱.۱ & بازنگری در این منشور تابع قواعد بازنگری قانون اساسی است. \\
\hline

۴۱.۲ & هر بازنگری که حقوق اقوام را کاهش دهد، مستلزم تأیید دوسوم مجلس اقوام است. \\
\hline

۴۱.۳ & اصول بنیادین این منشور (برابری اقوام، حق هویت، حق زبان) غیرقابل تغییر است. \\
\hline

\end{longtable}

\subsection{ماده ۴۲: لازم‌الاجرا شدن}

\begin{longtable}{|>{\columncolor{gray!15}}r|p{12cm}|}
\hline
\rowcolor{gray!30}
\textbf{بند} & \textbf{متن} \\
\hline
\endfirsthead

۴۲.۱ & این منشور همزمان با قانون اساسی لازم‌الاجرا می‌شود. \\
\hline

۴۲.۲ & دولت موظف است ظرف ۲ سال قوانین اجرایی لازم را تدوین کند. \\
\hline

۴۲.۳ & تا تدوین قوانین اجرایی، اصول این منشور مستقیماً قابل استناد است. \\
\hline

\end{longtable}

% ═══════════════════════════════════════════════════════════════════════════════
\section{پیوست‌های منشور}
% ═══════════════════════════════════════════════════════════════════════════════

\subsection{پیوست الف: نقشه مناطق خودمختار}

\begin{center}
\begin{tikzpicture}[scale=0.85]
    % کادر کلی نقشه شماتیک
    \draw[thick, gray, rounded corners=5pt] (-7,-4) rectangle (7,5);
    
    % عنوان
    \node[font=\large\bfseries] at (0,4.5) {نقشه شماتیک مناطق خودمختار و استان‌ها};
    
    % مناطق خودمختار (رنگی)
    % آذربایجان
    \draw[fill=SuccessGreen!40, draw=SuccessGreen, thick, rounded corners=3pt]
        (-6,2) rectangle (-3,4);
    \node[font=\small\bfseries] at (-4.5,3) {آذربایجان};
    \node[font=\tiny] at (-4.5,2.5) {تبریز | اردبیل};
    
    % کردستان
    \draw[fill=WisdomGold!40, draw=WisdomGold, thick, rounded corners=3pt]
        (-6,-0.5) rectangle (-3,1.5);
    \node[font=\small\bfseries] at (-4.5,0.5) {کردستان};
    \node[font=\tiny] at (-4.5,0) {سنندج | کرمانشاه};
    
    % خوزستان
    \draw[fill=WarningRed!40, draw=WarningRed, thick, rounded corners=3pt]
        (-6,-3.5) rectangle (-3,-1);
    \node[font=\small\bfseries] at (-4.5,-2.25) {خوزستان};
    \node[font=\tiny] at (-4.5,-2.75) {اهواز};
    
    % بلوچستان
    \draw[fill=purple!40, draw=purple, thick, rounded corners=3pt]
        (3,-3.5) rectangle (6,-0.5);
    \node[font=\small\bfseries] at (4.5,-2) {بلوچستان};
    \node[font=\tiny] at (4.5,-2.5) {زاهدان};
    
    % ترکمن‌صحرا
    \draw[fill=teal!40, draw=teal, thick, rounded corners=3pt]
        (3,2.5) rectangle (6,4);
    \node[font=\small\bfseries] at (4.5,3.25) {ترکمن‌صحرا};
    \node[font=\tiny] at (4.5,2.8) {گنبد};
    
    % مرکز (استان‌ها)
    \draw[fill=gray!20, draw=gray, rounded corners=3pt]
        (-2,-2) rectangle (2,2);
    \node[font=\small\bfseries] at (0,0.5) {استان‌ها};
    \node[font=\tiny, align=center] at (0,-0.5) {تهران | اصفهان\\فارس | خراسان\\و سایر استان‌ها};
    
    % راهنما
    \node[font=\scriptsize, align=right] at (5,-3.9) {
        \textcolor{SuccessGreen}{■} آذربایجان |
        \textcolor{WisdomGold}{■} کردستان |
        \textcolor{WarningRed}{■} خوزستان |
        \textcolor{purple}{■} بلوچستان |
        \textcolor{teal}{■} ترکمن‌صحرا
    };
\end{tikzpicture}
\end{center}

\subsection{پیوست ب: جدول زبان‌های ملی}

\begin{center}
\begin{small}
\begin{longtable}{|>{\columncolor{WisdomGold!10}}p{2cm}|p{2.5cm}|p{2cm}|p{2.5cm}|p{4cm}|}
\hline
\rowcolor{WisdomGold!30}
\textbf{زبان} & \textbf{خانواده زبانی} & \textbf{خط} & \textbf{گویشوران (میلیون)} & \textbf{وضعیت رسمی} \\
\hline
\endfirsthead

فارسی & ایرانی (هندواروپایی) & فارسی-عربی & ≈۴۷ & زبان رسمی سراسری \\
\hline

آذری & ترکی (آلتایی) & فارسی-عربی / لاتین & ≈۱۷ & رسمی در منطقه آذربایجان \\
\hline

کردی & ایرانی (هندواروپایی) & فارسی-عربی / لاتین & ≈۸.۵ & رسمی در منطقه کردستان \\
\hline

عربی & سامی (آفروآسیایی) & عربی & ≈۲.۵ & رسمی در منطقه خوزستان \\
\hline

بلوچی & ایرانی (هندواروپایی) & فارسی-عربی & ≈۱.۷ & رسمی در منطقه بلوچستان \\
\hline

ترکمنی & ترکی (آلتایی) & فارسی-عربی / لاتین & ≈۰.۸۵ & رسمی در منطقه ترکمن‌صحرا \\
\hline

لری & ایرانی (هندواروپایی) & فارسی-عربی & ≈۵ & زبان ملی (حمایت ویژه) \\
\hline

گیلکی & ایرانی (هندواروپایی) & فارسی-عربی & ≈۳ & زبان ملی (حمایت ویژه) \\
\hline

مازندرانی & ایرانی (هندواروپایی) & فارسی-عربی & ≈۳ & زبان ملی (حمایت ویژه) \\
\hline

\end{longtable}
\end{small}
\end{center}

\subsection{پیوست ج: تقویم اجرایی منشور}

\begin{center}
\begin{tikzpicture}[scale=0.9]
    % محور زمانی
    \draw[->, thick, gray] (0,0) -- (14,0);
    \foreach \x/\year in {0/سال ۰, 2/سال ۱, 4/سال ۲, 6/سال ۳, 8/سال ۵, 10/سال ۱۰, 12/سال ۲۰} {
        \draw[thick] (\x,0.1) -- (\x,-0.1);
        \node[below, font=\tiny] at (\x,-0.2) {\year};
    }
    
    % اقدامات
    \node[
        draw=DemocracyBlue,
        fill=DemocracyBlue!20,
        rounded corners=3pt,
        font=\tiny,
        align=center,
        minimum width=1.8cm
    ] at (1,1) {\shortstack{تشکیل\\کمیسیون ملی}};
    
    \node[
        draw=SuccessGreen,
        fill=SuccessGreen!20,
        rounded corners=3pt,
        font=\tiny,
        align=center,
        minimum width=1.8cm
    ] at (3,1.5) {\shortstack{آغاز آموزش\\زبان مادری}};
    
    \node[
        draw=WisdomGold,
        fill=WisdomGold!20,
        rounded corners=3pt,
        font=\tiny,
        align=center,
        minimum width=1.8cm
    ] at (5,1) {\shortstack{تکمیل قوانین\\اجرایی}};
    
    \node[
        draw=purple,
        fill=purple!20,
        rounded corners=3pt,
        font=\tiny,
        align=center,
        minimum width=1.8cm
    ] at (7,1.5) {\shortstack{گزارش کمیسیون\\حقیقت و آشتی}};
    
    \node[
        draw=WarningRed,
        fill=WarningRed!20,
        rounded corners=3pt,
        font=\tiny,
        align=center,
        minimum width=1.8cm
    ] at (9,1) {\shortstack{ارزیابی میان‌دوره\\شاخص‌ها}};
    
    \node[
        draw=teal,
        fill=teal!20,
        rounded corners=3pt,
        font=\tiny,
        align=center,
        minimum width=1.8cm
    ] at (11,1.5) {\shortstack{دستیابی به\\برابری نسبی}};
    
    % فلش‌ها
    \foreach \x in {1,3,5,7,9,11} {
        \draw[->, gray] (\x,0.5) -- (\x,0.1);
    }
\end{tikzpicture}
\end{center}

% ═══════════════════════════════════════════════════════════════════════════════
\section{خلاصه و جمع‌بندی منشور}
% ═══════════════════════════════════════════════════════════════════════════════

\begin{kholasebox}
\textbf{خلاصه منشور حقوق اقوام}

\begin{center}
\begin{tabular}{r r}
\textbf{شاخص} & \textbf{جزئیات} \\
\hline
تعداد فصول & ۱۲ فصل \\
تعداد مواد & ۴۲ ماده \\
اقوام به رسمیت شناخته‌شده & ۹ قوم اصلی \\
زبان‌های ملی & ۹ زبان \\
مناطق خودمختار & ۵ منطقه \\
نهادهای حمایتی & ۵ نهاد اصلی \\
\end{tabular}
\end{center}

\textbf{دستاوردهای کلیدی این منشور:}

\begin{enumerate}[nosep]
    \item \textbf{به رسمیت شناختن تنوع}: همه اقوام ایران به رسمیت شناخته شدند
    \item \textbf{حقوق زبانی}: آموزش و خدمات به زبان مادری تضمین شد
    \item \textbf{خودمختاری}: ۵ منطقه خودمختار با اختیارات واقعی
    \item \textbf{توسعه متوازن}: فرمول عادلانه توزیع منابع
    \item \textbf{عدالت انتقالی}: جبران بی‌عدالتی‌های تاریخی
    \item \textbf{نهادهای نظارتی}: کمیسیون مستقل حقوق اقوام
    \item \textbf{تبعیض مثبت}: اقدامات جبرانی برای برابری واقعی
    \item \textbf{ضمانت اجرایی}: مجازات‌های سنگین برای نقض حقوق
\end{enumerate}
\end{kholasebox}

\begin{naghlbox}
«ایران خانه مشترک همه اقوام است. در این خانه، هیچ‌کس میهمان نیست؛ همه صاحب‌خانه‌اند. تنوع ما نه تهدید که ثروت ماست. با احترام متقابل و حقوق برابر، می‌توانیم آینده‌ای بسازیم که در آن هر ایرانی به هویت خود افتخار کند و به همبستگی ملی پایبند باشد.»

\sourceline{از دیباچه منشور حقوق اقوام}
\end{naghlbox}

% ═══════════════════════════════════════════════════════════════════════════════
% پایان پیوست ۲
% ═══════════════════════════════════════════════════════════════════════════════
	% ═══════════════════════════════════════════════════════════════════════════════
% پیوست ۳: اسناد پشتیبان
% فایل: app03-documents.tex
% ═══════════════════════════════════════════════════════════════════════════════

\chapter{اسناد پشتیبان}
\label{app:documents}

\begin{kholasebox}
این پیوست مجموعه‌ای از اسناد پشتیبان را ارائه می‌دهد که برای اجرای عملی طرح گذار دموکراتیک ضروری هستند. اسناد شامل پیش‌نویس قوانین کلیدی، پروتکل‌های عملیاتی، الگوهای توافقنامه، چک‌لیست‌های اجرایی، و فرم‌های نمونه است. این اسناد به‌عنوان نقطه شروع طراحی شده‌اند و باید متناسب با شرایط واقعی تطبیق داده شوند.
\end{kholasebox}

% ═══════════════════════════════════════════════════════════════════════════════
\section{بخش اول: پیش‌نویس قوانین کلیدی}
\label{sec:doc-laws}
% ═══════════════════════════════════════════════════════════════════════════════

\subsection{سند ۱: قانون شورای انتقالی}

\begin{center}
\begin{tikzpicture}
    \node[
        draw=DemocracyBlue,
        line width=2pt,
        fill=DemocracyBlue!5,
        rounded corners=10pt,
        inner sep=15pt,
        text width=14cm,
        align=center
    ] {
        {\Large\textbf{پیش‌نویس قانون شورای انتقالی}}\\[5pt]
        {\small مصوب روز صفر گذار | سند شماره ۱}
    };
\end{tikzpicture}
\end{center}

\begin{longtable}{|>{\columncolor{bleurepublique!10}}p{2.5cm}|p{11.5cm}|}
\hline
\rowcolor{bleurepublique!30}
\textbf{\rl{ماده}} & \textbf{\rl{متن}} \\
\hline
\endfirsthead
\hline
\rowcolor{bleurepublique!30}
\textbf{\rl{ماده}} & \textbf{\rl{متن}} \\
\hline
\endhead

\textbf{ماده ۱} \newline تأسیس & 
شورای انتقالی ملی به‌عنوان عالی‌ترین نهاد حاکمیتی در دوره گذار تأسیس می‌شود. این شورا تا تشکیل نهادهای منتخب دموکراتیک، مسئولیت هدایت کشور را بر عهده دارد. \\
\hline

\textbf{ماده ۲} \newline ترکیب &
شورای انتقالی متشکل از ۵۰ عضو به شرح زیر است:

\begin{tabular}{r r}
نمایندگان احزاب و جریان‌های سیاسی & ۱۵ نفر \\
نمایندگان اقوام & ۱۰ نفر \\
نمایندگان جامعه مدنی & ۸ نفر \\
نمایندگان زنان & ۵ نفر \\
نمایندگان جوانان و دانشجویان & ۴ نفر \\
متخصصان و تکنوکرات‌ها & ۵ نفر \\
نمایندگان کارگران و اصناف & ۳ نفر \\
\end{tabular} \\
\hline

\textbf{ماده ۳} \newline وظایف &
وظایف شورای انتقالی:

الف) تصویب اعلامیه‌ها و فرمان‌های دوره گذار

ب) نظارت بر دولت موقت

ج) تصویب بودجه اضطراری

د) تصمیم‌گیری درباره برگزاری همه‌پرسی قانون اساسی

هـ) نظارت بر روند انتخابات

و) حل‌وفصل اختلافات بین نهادهای انتقالی \\
\hline

\textbf{ماده ۴} \newline ریاست &
شورا یک رئیس و دو نایب‌رئیس انتخاب می‌کند. ریاست باید چرخشی و نماینده تنوع باشد. رئیس شورا، رئیس تشریفاتی کشور در دوره انتقال است. \\
\hline

\textbf{ماده ۵} \newline تصمیم‌گیری &
تصمیمات عادی با اکثریت ساده و تصمیمات اساسی (اعلامیه‌های بنیادین، بودجه، عزل دولت) با اکثریت دوسوم اتخاذ می‌شود. \\
\hline

\textbf{ماده ۶} \newline محدودیت‌ها &
شورای انتقالی نمی‌تواند:

الف) قانون اساسی دائمی تصویب کند (این وظیفه مجلس مؤسسان است)

ب) معاهدات بلندمدت منعقد کند

ج) اصلاحات ساختاری غیرقابل بازگشت انجام دهد

د) مدت خود را تمدید کند \\
\hline

\textbf{ماده ۷} \newline مدت &
مدت فعالیت شورای انتقالی حداکثر ۱۸ ماه است. با تشکیل مجلس مؤسسان یا پارلمان منتخب، وظایف شورا پایان می‌یابد. \\
\hline

\textbf{ماده ۸} \newline شفافیت &
جلسات شورا علنی است. مصوبات ظرف ۲۴ ساعت منتشر می‌شود. صورتجلسات در دسترس عموم قرار می‌گیرد. \\
\hline

\end{longtable}

\subsection{سند ۲: قانون دولت موقت}

\begin{longtable}{|>{\columncolor{goldlight!10}}p{2.5cm}|p{11.5cm}|}
\hline
\rowcolor{goldlight!30}
\textbf{\rl{ماده}} & \textbf{\rl{متن}} \\
\hline
\endfirsthead

\textbf{ماده ۱} \newline تشکیل &
دولت موقت جمهوری فدرال ایران برای اداره امور اجرایی کشور در دوره گذار تشکیل می‌شود. \\
\hline

\textbf{ماده ۲} \newline ترکیب &
دولت موقت متشکل است از:

الف) نخست‌وزیر موقت (منصوب شورای انتقالی)

ب) حداکثر ۱۸ وزیر

ج) ترکیب کابینه باید بازتاب‌دهنده تنوع قومی، جنسیتی و سیاسی باشد \\
\hline

\textbf{ماده ۳} \newline صلاحیت &
دولت موقت مسئول:

الف) حفظ نظم و امنیت عمومی

ب) ادامه خدمات عمومی ضروری

ج) مدیریت اقتصاد و جلوگیری از فروپاشی

د) آماده‌سازی انتخابات آزاد

هـ) اجرای مصوبات شورای انتقالی \\
\hline

\textbf{ماده ۴} \newline محدودیت‌ها &
دولت موقت بدون تأیید شورای انتقالی نمی‌تواند:

الف) استقراض بین‌المللی بلندمدت انجام دهد

ب) قرارداد نفتی جدید منعقد کند

ج) تغییرات ساختاری در نهادها ایجاد کند

د) انتصابات دائمی در مناصب عالی انجام دهد \\
\hline

\textbf{ماده ۵} \newline پاسخگویی &
دولت موقت در برابر شورای انتقالی پاسخگوست. شورا می‌تواند با رأی دوسوم، دولت را برکنار کند. \\
\hline

\textbf{ماده ۶} \newline انتقال قدرت &
دولت موقت موظف است پس از انتخابات، ظرف ۳۰ روز قدرت را به دولت منتخب واگذار کند. \\
\hline

\end{longtable}

\subsection{سند ۳: قانون انتخابات دوره گذار}

\begin{longtable}{|>{\columncolor{bleulight!10}}p{2.5cm}|p{11.5cm}|}
\hline
\rowcolor{bleulight!30}
\textbf{\rl{ماده}} & \textbf{\rl{متن}} \\
\hline
\endfirsthead

\textbf{ماده ۱} \newline اصول &
انتخابات بر اساس اصول زیر برگزار می‌شود:

الف) آزاد، منصفانه و شفاف

ب) رأی همگانی، برابر، مستقیم و مخفی

ج) رقابتی و چندحزبی

د) تحت نظارت داخلی و بین‌المللی \\
\hline

\textbf{ماده ۲} \newline کمیسیون &
کمیسیون مستقل انتخابات با ترکیب زیر تشکیل می‌شود:

\begin{tabular}{r r}
قضات بازنشسته & ۳ نفر \\
حقوقدانان مستقل & ۲ نفر \\
نمایندگان احزاب (چرخشی) & ۲ نفر \\
متخصصان انتخابات & ۲ نفر \\
نماینده سازمان‌های بین‌المللی (ناظر) & ۲ نفر \\
\end{tabular} \\
\hline

\textbf{ماده ۳} \newline تقویم &
\begin{tabular}{r r}
ثبت‌نام احزاب و کاندیداها & ماه ۶-۸ \\
تبلیغات انتخاباتی & ۴۵ روز قبل از انتخابات \\
همه‌پرسی قانون اساسی & ماه ۱۰-۱۲ \\
انتخابات پارلمانی & ماه ۱۲-۱۵ \\
انتخابات ریاست جمهوری & ماه ۱۴-۱۶ \\
\end{tabular} \\
\hline

\textbf{ماده ۴} \newline نامزدی &
شرایط نامزدی:

الف) تابعیت ایرانی

ب) سن قانونی (۲۵ سال برای مجلس، ۴۰ سال برای ریاست جمهوری)

ج) عدم همکاری فعال با سرکوب در نظام قبلی

د) جمع‌آوری امضای حمایتی یا معرفی حزب \\
\hline

\textbf{ماده ۵} \newline بررسی صلاحیت &
بررسی صلاحیت تنها بر اساس معیارهای عینی قانونی انجام می‌شود:

الف) کمیته بررسی صلاحیت: ۵ قاضی + ۲ حقوقدان

ب) رد صلاحیت باید مستدل و مکتوب باشد

ج) حق اعتراض به کمیسیون عالی انتخابات تضمین می‌شود

د) رد صلاحیت ایدئولوژیک یا سیاسی ممنوع است \\
\hline

\textbf{ماده ۶} \newline تبلیغات &
الف) دسترسی برابر به رسانه‌های عمومی

ب) سقف هزینه تبلیغاتی

ج) ممنوعیت استفاده از منابع دولتی

د) ممنوعیت تبلیغات مبتنی بر نفرت \\
\hline

\textbf{ماده ۷} \newline نظارت &
الف) ناظران داخلی از همه احزاب

ب) ناظران بین‌المللی (سازمان ملل، اتحادیه اروپا)

ج) رسانه‌های مستقل

د) سازمان‌های مدنی \\
\hline

\textbf{ماده ۸} \newline شکایات &
کمیسیون رسیدگی به شکایات انتخاباتی تشکیل می‌شود. آرای قطعی ظرف ۷ روز صادر می‌شود. \\
\hline

\end{longtable}

\subsection{سند ۴: قانون عدالت انتقالی}

\begin{enghelabbox}
\textbf{⚠️ قانون عدالت انتقالی — سند حیاتی دوره گذار}

این قانون چارچوب برخورد با گذشته و حرکت به سوی آینده را تعیین می‌کند. تعادل بین عدالت و آشتی، کلید موفقیت گذار است.
\end{enghelabbox}

\begin{longtable}{|>{\columncolor{golddark!10}}p{2.5cm}|p{11.5cm}|}
\hline
\rowcolor{golddark!30}
\textbf{\rl{ماده}} & \textbf{\rl{متن}} \\
\hline
\endfirsthead

\textbf{ماده ۱} \newline اهداف &
اهداف عدالت انتقالی:

الف) کشف حقیقت درباره نقض حقوق بشر

ب) اجرای عدالت برای قربانیان

ج) جبران خسارت مادی و معنوی

د) تضمین عدم تکرار

هـ) آشتی ملی بر پایه حقیقت و عدالت \\
\hline

\textbf{ماده ۲} \newline کمیسیون حقیقت &
کمیسیون حقیقت و آشتی ملی تشکیل می‌شود:

الف) ترکیب: ۱۵ عضو (قضات، حقوقدانان، روحانیون، فعالان حقوق بشر، نمایندگان قربانیان)

ب) صلاحیت: بررسی نقض حقوق بشر از سال ۱۳۵۷

ج) مدت: ۳ سال با امکان تمدید

د) اختیارات: احضار شهود، دسترسی به اسناد، برگزاری جلسات علنی \\
\hline

\textbf{ماده ۳} \newline دسته‌بندی مسئولان &
\begin{tabular}{|r|r|r|}
\hline
\rowcolor{gray!20}
\textbf{دسته} & \textbf{مصداق} & \textbf{برخورد} \\
\hline
دسته الف & عاملان جنایات علیه بشریت، شکنجه‌گران، آمران قتل & محاکمه کیفری \\
\hline
دسته ب & همکاران فعال سرکوب، مدیران میانی & بررسی انفرادی، عزل، محرومیت \\
\hline
دسته ج & کارمندان عادی بدون نقش در سرکوب & حفظ شغل با تعهدنامه \\
\hline
\end{tabular} \\
\hline

\textbf{ماده ۴} \newline جرایم غیرقابل عفو &
جرایم زیر مشمول عفو نیستند:

الف) جنایات علیه بشریت

ب) شکنجه

ج) اعدام‌های سیاسی

د) ناپدیدسازی اجباری

هـ) تجاوز جنسی

و) غارت گسترده اموال عمومی \\
\hline

\textbf{ماده ۵} \newline جبران خسارت &
حقوق قربانیان:

الف) شناسایی رسمی به‌عنوان قربانی

ب) غرامت مالی متناسب

ج) بازسازی (عذرخواهی، احیای حیثیت)

د) خدمات توان‌بخشی (پزشکی، روان‌درمانی)

هـ) تضمین عدم تکرار \\
\hline

\textbf{ماده ۶} \newline صندوق جبران &
صندوق ملی جبران خسارت قربانیان تأسیس می‌شود:

منابع:
\begin{itemize}[nosep]
    \item اموال مصادره‌شده عاملان
    \item بودجه دولتی
    \item کمک‌های بین‌المللی
\end{itemize} \\
\hline

\textbf{ماده ۷} \newline بررسی صلاحیت (وِتینگ) &
الف) کارکنان نهادهای امنیتی، قضایی و اجرایی بررسی می‌شوند

ب) معیار: مشارکت در نقض حقوق بشر

ج) حق دفاع و اعتراض تضمین می‌شود

د) هدف: پاک‌سازی نهادها، نه انتقام \\
\hline

\textbf{ماده ۸} \newline اصلاحات نهادی &
برای تضمین عدم تکرار:

الف) بازسازی نهادهای امنیتی تحت نظارت غیرنظامی

ب) استقلال قضایی

ج) آموزش حقوق بشر در مدارس و نهادها

د) آزادی رسانه

هـ) تقویت جامعه مدنی \\
\hline

\end{longtable}

% ═══════════════════════════════════════════════════════════════════════════════
\section{بخش دوم: پروتکل‌های عملیاتی}
\label{sec:doc-protocols}
% ═══════════════════════════════════════════════════════════════════════════════

\subsection{سند ۵: پروتکل مدیریت لحظه صفر}

\begin{center}
\begin{tikzpicture}
    % عنوان
    \node[
        fill=golddark,
        text=white,
        rounded corners=5pt,
        font=\large\bfseries,
        minimum width=12cm
    ] at (0,5) {\rl{پروتکل عملیاتی لحظه صفر (۷۲ ساعت اول)}};
    
    % تایم‌لاین
    \draw[->, thick, gray] (-6,3.5) -- (6,3.5);
    
    % نقاط زمانی
    \foreach \x/\t in {-5/۰, -2.5/۶, 0/۱۲, 2.5/۲۴, 5/۷۲} {
        \draw[thick] (\x,3.6) -- (\x,3.4);
        \node[below, font=\tiny] at (\x,3.3) {\rl{ساعت \t}};
    }
    
    % اقدامات
    \node[
        draw=golddark,
        fill=golddark!20,
        rounded corners=5pt,
        font=\tiny,
        align=center,
        minimum width=2cm
    ] at (-5,2) {\shortstack{\rl{اعلامیه}\\\rl{سقوط رژیم}}};
    
    \node[
        draw=bleurepublique,
        fill=bleurepublique!20,
        rounded corners=5pt,
        font=\tiny,
        align=center,
        minimum width=2cm
    ] at (-2.5,2) {\shortstack{\rl{تشکیل شورای}\\\rl{انتقالی}}};
    
    \node[
        draw=goldlight,
        fill=goldlight!20,
        rounded corners=5pt,
        font=\tiny,
        align=center,
        minimum width=2cm
    ] at (0,2) {\shortstack{\rl{کنترل نقاط}\\\rl{استراتژیک}}};
    
    \node[
        draw=bleulight,
        fill=bleulight!20,
        rounded corners=5pt,
        font=\tiny,
        align=center,
        minimum width=2cm
    ] at (2.5,2) {\shortstack{\rl{اعلامیه}\\\rl{آرامش}}};
    
    \node[
        draw=bleurepublique!80,
        fill=bleurepublique!20,
        rounded corners=5pt,
        font=\tiny,
        align=center,
        minimum width=2cm
    ] at (5,2) {\shortstack{\rl{معرفی}\\\rl{دولت موقت}}};
\end{tikzpicture}
\end{center}

\begin{longtable}{|>{\columncolor{WarningRed!10}}p{2cm}|p{4cm}|p{4cm}|p{3.5cm}|}
\hline
\rowcolor{WarningRed!30}
\textbf{زمان} & \textbf{اقدام} & \textbf{مسئول} & \textbf{خروجی} \\
\hline
\endfirsthead

ساعت ۰ & اعلام سقوط نظام قبلی & رهبران جنبش & بیانیه رسمی \\
\hline

ساعت ۰-۲ & تماس با فرماندهان نظامی & کمیته هماهنگی & تعهد بی‌طرفی ارتش \\
\hline

ساعت ۲-۴ & کنترل صدا و سیما & تیم رسانه & پخش زنده \\
\hline

ساعت ۴-۶ & جلسه اضطراری شورای انتقالی & اعضای منتخب & اولین فرمان‌ها \\
\hline

ساعت ۶-۱۲ & ایمن‌سازی نقاط حساس & نیروهای امنیتی همراه & کنترل فرودگاه، بانک مرکزی \\
\hline

ساعت ۱۲-۲۴ & اعلامیه آرامش و امنیت & نخست‌وزیر موقت & پخش سراسری \\
\hline

ساعت ۲۴-۴۸ & تماس با سفارتخانه‌ها & وزیر خارجه موقت & شناسایی دیپلماتیک \\
\hline

ساعت ۴۸-۷۲ & جلسه توجیهی اقتصادی & وزیر اقتصاد موقت & اطمینان‌بخشی بازار \\
\hline

\end{longtable}

\subsubsection{چک‌لیست لحظه صفر}

\begin{center}
\begin{tikzpicture}
    \node[
        draw=gray,
        fill=gray!5,
        rounded corners=10pt,
        inner sep=15pt,
        text width=14cm,
        align=right
    ] {
        \textbf{چک‌لیست اقدامات فوری — لحظه صفر}
        
        \vspace{10pt}
        
        \begin{tabular}{r r r}
            $\square$ & تأیید سقوط نظام قبلی & اولویت ۱ \\
            $\square$ & تماس با فرماندهان ارتش & اولویت ۱ \\
            $\square$ & فعال‌سازی شورای انتقالی & اولویت ۱ \\
            $\square$ & کنترل رسانه ملی & اولویت ۱ \\
            $\square$ & ایمن‌سازی بانک مرکزی & اولویت ۱ \\
            $\square$ & ایمن‌سازی فرودگاه‌ها & اولویت ۲ \\
            $\square$ & ایمن‌سازی مرزها & اولویت ۲ \\
            $\square$ & ایمن‌سازی تأسیسات نفتی & اولویت ۲ \\
            $\square$ & تماس با سازمان‌های بین‌المللی & اولویت ۲ \\
            $\square$ & اعلامیه حفظ آرامش & اولویت ۱ \\
            $\square$ & تضمین ادامه خدمات عمومی & اولویت ۱ \\
            $\square$ & جلوگیری از غارت و انتقام‌جویی & اولویت ۱ \\
            $\square$ & حفاظت از اسناد و آرشیوها & اولویت ۲ \\
            $\square$ & بازداشت متهمان اصلی & اولویت ۳ \\
            $\square$ & اعلام منع رفت‌وآمد موقت (در صورت نیاز) & اولویت ۳ \\
        \end{tabular}
    };
\end{tikzpicture}
\end{center}

\subsection{سند ۶: پروتکل امنیت دوره گذار}

\begin{longtable}{|>{\columncolor{bleurepublique!10}}p{3cm}|p{5cm}|p{5.5cm}|}
\hline
\rowcolor{bleurepublique!30}
\textbf{\rl{تهدید}} & \textbf{\rl{اقدام پیشگیرانه}} & \textbf{\rl{پاسخ اضطراری}} \\
\hline
\endfirsthead

کودتای نظامی & 
• چرخش فرماندهان \newline
• نظارت غیرنظامی \newline
• پراکندگی قدرت &
• فعال‌سازی مقاومت مدنی \newline
• درخواست حمایت بین‌المللی \newline
• بسیج رسانه‌ای \\
\hline

شورش عناصر رژیم قبلی &
• بررسی صلاحیت نیروها \newline
• کنترل سلاح‌ها \newline
• نظارت بر رهبران سابق &
• بازداشت سریع \newline
• قطع منابع مالی \newline
• اطلاع‌رسانی عمومی \\
\hline

درگیری قومی &
• گفتگوی پیشگیرانه \newline
• نمایندگی عادلانه \newline
• عدالت توزیعی &
• میانجیگری فوری \newline
• استقرار نیروی حفظ صلح \newline
• رسانه‌های آشتی \\
\hline

فروپاشی اقتصادی &
• تضمین ذخایر ارزی \newline
• مذاکره با IMF/WB \newline
• حفظ زنجیره تأمین &
• یارانه اضطراری \newline
• کنترل قیمت موقت \newline
• کمک بین‌المللی \\
\hline

تروریسم &
• اطلاعات پیشگیرانه \newline
• حفاظت از اهداف حساس \newline
• همکاری بین‌المللی &
• پاسخ سریع \newline
• اطلاع‌رسانی شفاف \newline
• حفظ آرامش \\
\hline

مداخله خارجی &
• دیپلماسی فعال \newline
• تضمین‌های بین‌المللی \newline
• بی‌طرفی فعال &
• بین‌المللی‌سازی بحران \newline
• مقاومت مسالمت‌آمیز \newline
• درخواست از شورای امنیت \\
\hline

\end{longtable}

\subsection{سند ۷: پروتکل ارتباطات بحران}

\begin{olgoobox}
\textbf{اصول ارتباطات در دوره گذار}

\begin{enumerate}[nosep]
    \item \textbf{شفافیت:} اطلاع‌رسانی صادقانه، حتی در مورد اخبار بد
    \item \textbf{سرعت:} پاسخ سریع به شایعات و اخبار کاذب
    \item \textbf{یکپارچگی:} پیام واحد از همه منابع رسمی
    \item \textbf{همدلی:} درک نگرانی‌های مردم
    \item \textbf{امید:} تأکید بر چشم‌انداز مثبت آینده
\end{enumerate}
\end{olgoobox}

\begin{longtable}{|>{\columncolor{DemocracyBlue!10}}p{3cm}|p{4cm}|p{3cm}|p{3.5cm}|}
\hline
\rowcolor{DemocracyBlue!30}
\textbf{کانال} & \textbf{مخاطب} & \textbf{فرکانس} & \textbf{محتوا} \\
\hline
\endfirsthead

تلویزیون ملی & عموم مردم & روزانه ۳ بار & اخبار رسمی، پیام‌های دولت \\
\hline

رادیو & مناطق روستایی & ساعتی & اطلاعات ضروری \\
\hline

شبکه‌های اجتماعی & جوانان، شهری & لحظه‌ای & پاسخ به شایعات \\
\hline

وب‌سایت رسمی & همه & به‌روز & اسناد، قوانین، آمار \\
\hline

پیامک ملی & همه & اضطراری & هشدارها، اطلاعات حیاتی \\
\hline

کنفرانس خبری & رسانه‌ها & روزانه & شفاف‌سازی، پرسش و پاسخ \\
\hline

جلسات محلی & جوامع محلی & هفتگی & گفتگوی مستقیم \\
\hline

\end{longtable}

% ═══════════════════════════════════════════════════════════════════════════════
\section{بخش سوم: الگوهای توافقنامه}
\label{sec:doc-agreements}
% ═══════════════════════════════════════════════════════════════════════════════

\subsection{سند ۸: میثاق ملی گذار دموکراتیک}

\begin{center}
\begin{tikzpicture}
    \node[
        draw=DemocracyBlue,
        line width=3pt,
        fill=DemocracyBlue!5,
        rounded corners=15pt,
        inner sep=20pt,
        text width=14cm,
        align=center
    ] {
        {\Huge\textbf{میثاق ملی}}\\[10pt]
        {\Large\textbf{گذار دموکراتیک ایران}}\\[15pt]
        {\normalsize امضاشده توسط نمایندگان احزاب، جنبش‌ها، اقوام و جامعه مدنی}
    };
\end{tikzpicture}
\end{center}

\vspace{10pt}

\begin{naghlbox}
\textbf{دیباچه میثاق}

ما، امضاکنندگان این میثاق، نمایندگان طیف‌های مختلف سیاسی، قومی و اجتماعی ایران،

با اعتقاد به ضرورت گذار مسالمت‌آمیز به دموکراسی؛

با تعهد به احترام متقابل و پذیرش تکثر؛

با هدف ساختن آینده‌ای آزاد، دموکراتیک و عادلانه؛

این میثاق را به‌عنوان چارچوب همکاری در دوره گذار امضا می‌کنیم.

\sourceline{میثاق ملی گذار دموکراتیک}
\end{naghlbox}

\begin{longtable}{|>{\columncolor{bleurepublique!10}}r|p{12cm}|}
\hline
\rowcolor{bleurepublique!30}
\textbf{\rl{اصل}} & \textbf{\rl{متن}} \\
\hline
\endfirsthead

\textbf{اصل ۱} &
\textbf{تعهد به گذار مسالمت‌آمیز:} همه طرف‌ها متعهد به گذار غیرخشونت‌آمیز و دموکراتیک هستند. \\
\hline

\textbf{اصل ۲} &
\textbf{پذیرش تکثرگرایی:} همه طرف‌ها حق وجود و فعالیت یکدیگر را به رسمیت می‌شناسند. \\
\hline

\textbf{اصل ۳} &
\textbf{تعهد به دموکراسی:} رقابت سیاسی تنها از طریق صندوق رأی و روش‌های دموکراتیک. \\
\hline

\textbf{اصل ۴} &
\textbf{احترام به حقوق بشر:} همه طرف‌ها متعهد به رعایت حقوق بنیادین شهروندان. \\
\hline

\textbf{اصل ۵} &
\textbf{حفظ تمامیت ارضی:} همه طرف‌ها متعهد به یکپارچگی سرزمینی ایران در چارچوب فدرالیسم. \\
\hline

\textbf{اصل ۶} &
\textbf{پذیرش نتایج انتخابات:} همه طرف‌ها نتایج انتخابات آزاد و منصفانه را می‌پذیرند. \\
\hline

\textbf{اصل ۷} &
\textbf{انتقال قدرت مسالمت‌آمیز:} برنده انتخابات حکومت می‌کند، بازنده در اپوزیسیون می‌ماند. \\
\hline

\textbf{اصل ۸} &
\textbf{عدالت انتقالی:} تعهد به حقیقت‌یابی و عدالت بدون انتقام‌جویی. \\
\hline

\textbf{اصل ۹} &
\textbf{شمول همگانی:} هیچ گروهی نباید از فرآیند گذار حذف شود. \\
\hline

\textbf{اصل ۱۰} &
\textbf{حل اختلاف مسالمت‌آمیز:} اختلافات از طریق گفتگو و میانجیگری حل می‌شود. \\
\hline

\end{longtable}

\subsection{سند ۹: توافقنامه همکاری بین‌قومی}

\begin{longtable}{|>{\columncolor{goldlight!10}}r|p{12cm}|}
\hline
\rowcolor{goldlight!30}
\textbf{\rl{ماده}} & \textbf{\rl{متن}} \\
\hline
\endfirsthead

\textbf{ماده ۱} &
\textbf{طرفین:} این توافقنامه بین نمایندگان رسمی اقوام ایران (فارس، آذری، کرد، عرب، بلوچ، ترکمن، لر، گیلک، مازندرانی و سایر اقوام) منعقد می‌شود. \\
\hline

\textbf{ماده ۲} &
\textbf{اصل برابری:} همه اقوام برابرند. هیچ قومی بر قوم دیگر برتری ندارد. \\
\hline

\textbf{ماده ۳} &
\textbf{تعهد به وحدت:} همه طرف‌ها متعهد به حفظ وحدت ملی و تمامیت ارضی ایران هستند. \\
\hline

\textbf{ماده ۴} &
\textbf{فدرالیسم:} طرف‌ها بر ساختار فدرال به‌عنوان چارچوب مدیریت تنوع توافق دارند. \\
\hline

\textbf{ماده ۵} &
\textbf{حقوق زبانی:} طرف‌ها حق آموزش و استفاده از زبان مادری را برای همه اقوام به رسمیت می‌شناسند. \\
\hline

\textbf{ماده ۶} &
\textbf{توزیع منابع:} طرف‌ها بر فرمول عادلانه توزیع منابع ملی توافق دارند. \\
\hline

\textbf{ماده ۷} &
\textbf{مشارکت سیاسی:} همه اقوام حق نمایندگی عادلانه در نهادهای ملی را دارند. \\
\hline

\textbf{ماده ۸} &
\textbf{حل اختلاف:} اختلافات بین‌قومی از طریق شورای آشتی و در نهایت دیوان قانون اساسی حل می‌شود. \\
\hline

\textbf{ماده ۹} &
\textbf{ممنوعیت خشونت:} طرف‌ها هرگونه خشونت قومی را محکوم و ممنوع می‌دانند. \\
\hline

\textbf{ماده ۱۰} &
\textbf{تعهد به آشتی:} طرف‌ها متعهد به آشتی تاریخی و فراموش کردن کینه‌های گذشته هستند. \\
\hline

\end{longtable}

\subsection{سند ۱۰: تفاهم‌نامه نیروهای مسلح}

\begin{longtable}{|>{\columncolor{golddark!10}}r|p{12cm}|}
\hline
\rowcolor{golddark!30}
\textbf{\rl{ماده}} & \textbf{\rl{متن}} \\
\hline
\endfirsthead

\textbf{ماده ۱} &
\textbf{طرفین:} این تفاهم‌نامه بین شورای انتقالی و فرماندهی نیروهای مسلح منعقد می‌شود. \\
\hline

\textbf{ماده ۲} &
\textbf{بی‌طرفی:} نیروهای مسلح در دوره گذار بی‌طرف می‌مانند و از هیچ جریان سیاسی حمایت نمی‌کنند. \\
\hline

\textbf{ماده ۳} &
\textbf{اطاعت از دولت غیرنظامی:} نیروهای مسلح تحت فرماندهی دولت موقت غیرنظامی قرار می‌گیرند. \\
\hline

\textbf{ماده ۴} &
\textbf{حفاظت از امنیت:} نیروهای مسلح موظف به حفظ امنیت ملی، مرزها و نظم عمومی هستند. \\
\hline

\textbf{ماده ۵} &
\textbf{عدم دخالت در سیاست:} نظامیان از فعالیت سیاسی و حزبی منع می‌شوند. \\
\hline

\textbf{ماده ۶} &
\textbf{اصلاحات:} نیروهای مسلح با برنامه اصلاحات و حرفه‌ای‌سازی همکاری می‌کنند. \\
\hline

\textbf{ماده ۷} &
\textbf{نیروهای موازی:} نیروهای شبه‌نظامی و موازی منحل می‌شوند. \\
\hline

\textbf{ماده ۸} &
\textbf{عدالت انتقالی:} فرماندهان متهم به نقض حقوق بشر بررسی می‌شوند. نظامیان عادی بدون سابقه سرکوب حفظ می‌شوند. \\
\hline

\textbf{ماده ۹} &
\textbf{بودجه:} بودجه دفاعی معقول و تحت نظارت پارلمانی تضمین می‌شود. \\
\hline

\textbf{ماده ۱۰} &
\textbf{ممنوعیت کودتا:} هرگونه تلاش برای کودتا جرم سنگین محسوب می‌شود. \\
\hline

\end{longtable}

% ═══════════════════════════════════════════════════════════════════════════════
\section{بخش چهارم: فرم‌ها و الگوهای اداری}
\label{sec:doc-forms}
% ═══════════════════════════════════════════════════════════════════════════════

\subsection{سند ۱۱: فرم اعلام دارایی مقامات}

\begin{center}
\begin{tikzpicture}
    \node[
        draw=gray,
        line width=1pt,
        fill=white,
        rounded corners=5pt,
        inner sep=15pt,
        text width=14cm,
        align=right
    ] {
        \begin{center}
            {\large\textbf{فرم اعلام دارایی مقامات دولتی}}\\
            {\small جمهوری فدرال ایران — سازمان شفافیت و مبارزه با فساد}
        \end{center}
        
        \vspace{10pt}
        \hrule
        \vspace{10pt}
        
        \textbf{بخش الف: مشخصات فردی}
        
        نام و نام خانوادگی: \dotfill
        
        شماره ملی: \dotfill سمت: \dotfill
        
        تاریخ انتصاب: \dotfill
        
        \vspace{10pt}
        
        \textbf{بخش ب: دارایی‌های غیرمنقول}
        
        \begin{tabular}{|r|r|r|r|r|}
        \hline
        ردیف & نوع ملک & مساحت & آدرس & ارزش تقریبی \\
        \hline
        ۱ & & & & \\
        \hline
        ۲ & & & & \\
        \hline
        \end{tabular}
        
        \vspace{10pt}
        
        \textbf{بخش ج: دارایی‌های منقول}
        
        \begin{tabular}{|r|r|r|}
        \hline
        نوع & تعداد/مقدار & ارزش تقریبی \\
        \hline
        خودرو & & \\
        \hline
        موجودی بانکی & & \\
        \hline
        سهام و اوراق & & \\
        \hline
        \end{tabular}
        
        \vspace{10pt}
        
        \textbf{بخش د: بدهی‌ها}
        
        مجموع بدهی‌ها: \dotfill ریال
        
        \vspace{10pt}
        
        \textbf{تعهد:} اینجانب صحت اطلاعات فوق را تأیید می‌کنم و متعهد می‌شوم هرگونه تغییر را ظرف ۳۰ روز اعلام کنم.
        
        \vspace{10pt}
        
        امضا: \dotfill تاریخ: \dotfill
    };
\end{tikzpicture}
\end{center}

\subsection{سند ۱۲: فرم گزارش فساد (سوت‌زنی)}

\begin{center}
\begin{tikzpicture}
    \node[
        draw=WarningRed,
        line width=1pt,
        fill=WarningRed!5,
        rounded corners=5pt,
        inner sep=15pt,
        text width=14cm,
        align=right
    ] {
        \begin{center}
            {\large\textbf{فرم محرمانه گزارش فساد}}\\
            {\small خط ویژه مبارزه با فساد — ۱۱۱}
        \end{center}
        
        \vspace{10pt}
        \hrule
        \vspace{5pt}
        {\scriptsize ⚠️ این فرم محرمانه است. هویت گزارش‌دهنده محفوظ می‌ماند.}
        \vspace{5pt}
        \hrule
        \vspace{10pt}
        
        \textbf{بخش الف: مشخصات گزارش‌دهنده (اختیاری)}
        
        نام (اختیاری): \dotfill کد پیگیری: \dotfill
        
        تلفن تماس (اختیاری): \dotfill
        
        \vspace{10pt}
        
        \textbf{بخش ب: مشخصات فساد}
        
        نوع فساد: $\square$ رشوه $\square$ اختلاس $\square$ سوءاستفاده $\square$ تبانی $\square$ سایر
        
        نهاد/سازمان مربوطه: \dotfill
        
        فرد/افراد متهم: \dotfill
        
        \vspace{10pt}
        
        \textbf{بخش ج: شرح ماجرا}
        
        \vspace{50pt}
        
        \textbf{بخش د: مستندات}
        
        $\square$ سند مکتوب $\square$ فایل صوتی $\square$ فایل تصویری $\square$ شاهد $\square$ سایر
        
        \vspace{10pt}
        
        \textbf{تعهد سازمان:} هویت شما محفوظ است. از شما در برابر هرگونه انتقام‌جویی حمایت می‌شود.
    };
\end{tikzpicture}
\end{center}

\subsection{سند ۱۳: الگوی تعهدنامه کارکنان دولت}

\begin{center}
\begin{tikzpicture}
    \node[
        draw=DemocracyBlue,
        line width=1pt,
        fill=DemocracyBlue!5,
        rounded corners=5pt,
        inner sep=15pt,
        text width=14cm,
        align=right
    ] {
        \begin{center}
            {\large\textbf{تعهدنامه خدمت در نظام جدید}}\\
            {\small جمهوری فدرال ایران}
        \end{center}
        
        \vspace{10pt}
        \hrule
        \vspace{10pt}
        
        اینجانب \dotfill فرزند \dotfill
        
        دارای شماره ملی \dotfill
        
        شاغل در \dotfill با سمت \dotfill
        
        \vspace{10pt}
        
        \textbf{با امضای این تعهدنامه:}
        
        \vspace{5pt}
        
        ۱. قانون اساسی جمهوری فدرال ایران را به رسمیت می‌شناسم و به آن پایبندم.
        
        ۲. متعهد به رعایت حقوق بشر و کرامت انسانی همه شهروندان هستم.
        
        ۳. از هرگونه تبعیض بر اساس قومیت، جنسیت، دین یا عقیده سیاسی خودداری می‌کنم.
        
        ۴. در انجام وظایف، صداقت، شفافیت و بی‌طرفی را رعایت می‌کنم.
        
        ۵. اطلاعات محرمانه را حفظ می‌کنم.
        
        ۶. از موقعیت شغلی برای منافع شخصی سوءاستفاده نمی‌کنم.
        
        ۷. هرگونه فساد را گزارش می‌دهم.
        
        ۸. آگاهم که نقض این تعهدات منجر به پیگرد قانونی و اخراج می‌شود.
        
        \vspace{15pt}
        
        امضا: \dotfill تاریخ: \dotfill
        
        گواه (مدیر): \dotfill
    };
\end{tikzpicture}
\end{center}

% ═══════════════════════════════════════════════════════════════════════════════
\section{بخش پنجم: چک‌لیست‌های اجرایی}
\label{sec:doc-checklists}
% ═══════════════════════════════════════════════════════════════════════════════

\subsection{سند ۱۴: چک‌لیست ۱۰۰ روز اول}

\begin{center}
\begin{tikzpicture}
    \node[
        fill=SuccessGreen,
        text=white,
        rounded corners=5pt,
        font=\large\bfseries,
        minimum width=12cm
    ] at (0,0) {چک‌لیست اقدامات ۱۰۰ روز اول دولت موقت};
\end{tikzpicture}
\end{center}

\begin{longtable}{|>{\columncolor{SuccessGreen!10}}r|p{6cm}|p{2.5cm}|p{2cm}|p{2cm}|}
\hline
\rowcolor{SuccessGreen!30}
\textbf{ردیف} & \textbf{اقدام} & \textbf{مسئول} & \textbf{مهلت} & \textbf{وضعیت} \\
\hline
\endfirsthead

\multicolumn{5}{c}{\textbf{— امنیت و ثبات —}} \\
\hline

۱ & ایمن‌سازی کامل مرزها & وزارت دفاع & روز ۷ & $\square$ \\
\hline

۲ & استقرار نظم در شهرهای بزرگ & وزارت کشور & روز ۱۴ & $\square$ \\
\hline

۳ & جمع‌آوری سلاح‌های غیرمجاز & وزارت کشور & روز ۳۰ & $\square$ \\
\hline

۴ & انحلال نیروهای موازی & وزارت دفاع & روز ۳۰ & $\square$ \\
\hline

\multicolumn{5}{c}{\textbf{— اقتصاد و معیشت —}} \\
\hline

۵ & تضمین ذخایر ارزی & بانک مرکزی & روز ۷ & $\square$ \\
\hline

۶ & پرداخت حقوق کارکنان دولت & وزارت اقتصاد & روز ۱۵ & $\square$ \\
\hline

۷ & تثبیت قیمت کالاهای اساسی & وزارت صمت & روز ۳۰ & $\square$ \\
\hline

۸ & پرداخت یارانه اضطراری & وزارت رفاه & روز ۴۵ & $\square$ \\
\hline

۹ & آغاز مذاکرات رفع تحریم & وزارت خارجه & روز ۳۰ & $\square$ \\
\hline

\multicolumn{5}{c}{\textbf{— سیاسی و حقوقی —}} \\
\hline

۱۰ & آزادی زندانیان سیاسی & وزارت دادگستری & روز ۱۴ & $\square$ \\
\hline

۱۱ & لغو قوانین سرکوبگرانه & شورای انتقالی & روز ۳۰ & $\square$ \\
\hline

۱۲ & تشکیل کمیسیون انتخابات & شورای انتقالی & روز ۴۵ & $\square$ \\
\hline

۱۳ & اعلام تقویم انتخابات & کمیسیون انتخابات & روز ۶۰ & $\square$ \\
\hline

۱۴ & ثبت‌نام احزاب سیاسی & وزارت کشور & روز ۹۰ & $\square$ \\
\hline

\multicolumn{5}{c}{\textbf{— خدمات عمومی —}} \\
\hline

۱۵ & تضمین برق و آب پایدار & وزارت نیرو & روز ۱۴ & $\square$ \\
\hline

۱۶ & بازگشایی مدارس و دانشگاه‌ها & وزارت آموزش & روز ۳۰ & $\square$ \\
\hline

۱۷ & تداوم خدمات بهداشتی & وزارت بهداشت & روز ۷ & $\square$ \\
\hline

۱۸ & بازگشایی بانک‌ها & بانک مرکزی & روز ۱۴ & $\square$ \\
\hline

\multicolumn{5}{c}{\textbf{— عدالت انتقالی —}} \\
\hline

۱۹ & ایمن‌سازی آرشیوها و اسناد & وزارت اطلاعات & روز ۷ & $\square$ \\
\hline

۲۰ & تشکیل کمیسیون حقیقت‌یابی & شورای انتقالی & روز ۶۰ & $\square$ \\
\hline

۲۱ & بازداشت متهمان اصلی & دادستانی & روز ۳۰ & $\square$ \\
\hline

۲۲ & آغاز بررسی صلاحیت مقامات & کمیته وِتینگ & روز ۴۵ & $\square$ \\
\hline

\multicolumn{5}{c}{\textbf{— روابط بین‌الملل —}} \\
\hline

۲۳ & ارسال پیام به سازمان ملل & وزارت خارجه & روز ۳ & $\square$ \\
\hline

۲۴ & درخواست شناسایی بین‌المللی & وزارت خارجه & روز ۱۴ & $\square$ \\
\hline

۲۵ & پذیرش ناظران بین‌المللی & وزارت خارجه & روز ۳۰ & $\square$ \\
\hline

\end{longtable}

\subsection{سند ۱۵: چک‌لیست برگزاری انتخابات}

\begin{longtable}{|>{\columncolor{WisdomGold!10}}r|p{6cm}|p{3cm}|p{3.5cm}|}
\hline
\rowcolor{WisdomGold!30}
\textbf{مرحله} & \textbf{اقدام} & \textbf{زمان‌بندی} & \textbf{وضعیت} \\
\hline
\endfirsthead

\multicolumn{4}{c}{\textbf{— مرحله ۱: آماده‌سازی (۶ ماه قبل) —}} \\
\hline

۱.۱ & تشکیل کمیسیون مستقل انتخابات & ماه ۱ & $\square$ \\
\hline

۱.۲ & تدوین قانون انتخابات موقت & ماه ۱-۲ & $\square$ \\
\hline

۱.۳ & تعیین حوزه‌های انتخاباتی & ماه ۲ & $\square$ \\
\hline

۱.۴ & ثبت‌نام و آموزش کارکنان انتخاباتی & ماه ۲-۴ & $\square$ \\
\hline

۱.۵ & به‌روزرسانی لیست رأی‌دهندگان & ماه ۲-۵ & $\square$ \\
\hline

۱.۶ & تأمین تجهیزات و مواد انتخاباتی & ماه ۳-۵ & $\square$ \\
\hline

\multicolumn{4}{c}{\textbf{— مرحله ۲: ثبت‌نام (۳ ماه قبل) —}} \\
\hline

۲.۱ & ثبت‌نام احزاب سیاسی & ماه ۳ & $\square$ \\
\hline

۲.۲ & ثبت‌نام کاندیداها & ماه ۳-۴ & $\square$ \\
\hline

۲.۳ & بررسی صلاحیت‌ها (معیارهای عینی) & ماه ۴ & $\square$ \\
\hline

۲.۴ & رسیدگی به اعتراضات & ماه ۴-۵ & $\square$ \\
\hline

۲.۵ & اعلام لیست نهایی کاندیداها & ماه ۵ & $\square$ \\
\hline

\multicolumn{4}{c}{\textbf{— مرحله ۳: تبلیغات (۴۵ روز قبل) —}} \\
\hline

۳.۱ & آغاز رسمی تبلیغات & روز ۴۵- & $\square$ \\
\hline

۳.۲ & مناظره‌های تلویزیونی & روز ۴۰-۱۵- & $\square$ \\
\hline

۳.۳ & نظارت بر تبلیغات & مستمر & $\square$ \\
\hline

۳.۴ & پایان تبلیغات & روز ۲- & $\square$ \\
\hline

۳.۵ & دوره سکوت انتخاباتی & روز ۲- تا روز انتخابات & $\square$ \\
\hline

\multicolumn{4}{c}{\textbf{— مرحله ۴: رأی‌گیری (روز انتخابات) —}} \\
\hline

۴.۱ & بازگشایی شعب & ساعت ۸ صبح & $\square$ \\
\hline

۴.۲ & نظارت بر رأی‌گیری & مستمر & $\square$ \\
\hline

۴.۳ & بستن شعب & ساعت ۲۲ & $\square$ \\
\hline

۴.۴ & شمارش آرا در شعبه & شب انتخابات & $\square$ \\
\hline

۴.۵ & ارسال نتایج به مرکز & شب انتخابات & $\square$ \\
\hline

\multicolumn{4}{c}{\textbf{— مرحله ۵: اعلام نتایج (پس از انتخابات) —}} \\
\hline

۵.۱ & جمع‌بندی نتایج استانی & روز ۱+ & $\square$ \\
\hline

۵.۲ & اعلام نتایج اولیه & روز ۲+ & $\square$ \\
\hline

۵.۳ & مهلت اعتراض & روز ۲-۷+ & $\square$ \\
\hline

۵.۴ & رسیدگی به اعتراضات & روز ۷-۱۴+ & $\square$ \\
\hline

۵.۵ & اعلام نتایج قطعی & روز ۱۴+ & $\square$ \\
\hline

۵.۶ & صدور اعتبارنامه & روز ۲۱+ & $\square$ \\
\hline

\end{longtable}

\subsection{سند ۱۶: چک‌لیست استقرار فدرالیسم}

\begin{longtable}{|>{\columncolor{purple!10}}r|p{5.5cm}|p{2.5cm}|p{2cm}|p{2cm}|}
\hline
\rowcolor{purple!30}
\textbf{ردیف} & \textbf{اقدام} & \textbf{مسئول} & \textbf{سال} & \textbf{وضعیت} \\
\hline
\endfirsthead

\multicolumn{5}{c}{\textbf{— سال اول: طراحی —}} \\
\hline

۱ & تصویب ساختار فدرال در قانون اساسی & مجلس مؤسسان & سال ۱ & $\square$ \\
\hline

۲ & تعیین مرزهای مناطق خودمختار & کمیته تقسیمات & سال ۱ & $\square$ \\
\hline

۳ & تدوین قانون فدرالیسم & مجلس ملی & سال ۱ & $\square$ \\
\hline

۴ & طراحی فرمول توزیع منابع & وزارت اقتصاد & سال ۱ & $\square$ \\
\hline

\multicolumn{5}{c}{\textbf{— سال ۲-۳: استقرار —}} \\
\hline

۵ & انتخابات پارلمان‌های منطقه‌ای & کمیسیون انتخابات & سال ۲ & $\square$ \\
\hline

۶ & تشکیل دولت‌های منطقه‌ای & مناطق & سال ۲ & $\square$ \\
\hline

۷ & انتقال صلاحیت‌های اولیه & دولت مرکزی & سال ۲-۳ & $\square$ \\
\hline

۸ & تأسیس صندوق توازن منطقه‌ای & وزارت اقتصاد & سال ۲ & $\square$ \\
\hline

۹ & آغاز آموزش دوزبانه & وزارت آموزش & سال ۲ & $\square$ \\
\hline

\multicolumn{5}{c}{\textbf{— سال ۴-۵: تثبیت —}} \\
\hline

۱۰ & انتقال کامل صلاحیت‌ها & دولت مرکزی & سال ۴-۵ & $\square$ \\
\hline

۱۱ & استقرار نظام مالیاتی فدرال & وزارت اقتصاد & سال ۴ & $\square$ \\
\hline

۱۲ & ارزیابی عملکرد سیستم & شورای فدرال & سال ۵ & $\square$ \\
\hline

۱۳ & اصلاحات لازم & مجلس ملی & سال ۵ & $\square$ \\
\hline

\end{longtable}

% ═══════════════════════════════════════════════════════════════════════════════
\section{بخش ششم: اعلامیه‌ها و بیانیه‌های نمونه}
\label{sec:doc-declarations}
% ═══════════════════════════════════════════════════════════════════════════════

\subsection{سند ۱۷: اعلامیه روز صفر}

\begin{center}
\begin{tikzpicture}
    \node[
        draw=DemocracyBlue,
        line width=3pt,
        fill=DemocracyBlue!5,
        rounded corners=15pt,
        inner sep=20pt,
        text width=14cm,
        align=justify
    ] {
        \begin{center}
            {\LARGE\textbf{اعلامیه شماره یک}}\\[5pt]
            {\large\textbf{شورای انتقالی ملی ایران}}\\[10pt]
            {\normalsize به نام ملت ایران}
        \end{center}
        
        \vspace{10pt}
        
        هم‌میهنان عزیز،
        
        پس از سال‌ها مبارزه و فداکاری، امروز صفحه‌ای نو در تاریخ ایران گشوده شد. نظام استبدادی پایان یافت و ملت ایران آزاد شد.
        
        \textbf{شورای انتقالی ملی} به‌عنوان نهاد موقت نمایندگی ملت، مسئولیت هدایت کشور به سوی دموکراسی را بر عهده گرفت.
        
        \textbf{اعلام می‌کنیم:}
        
        ۱. قانون اساسی نظام قبلی ملغی است.
        
        ۲. کلیه نهادهای سرکوب منحل می‌شوند.
        
        ۳. کلیه زندانیان سیاسی آزاد می‌شوند.
        
        ۴. آزادی بیان، تجمع و مطبوعات برقرار است.
        
        ۵. دولت موقت برای اداره امور کشور تشکیل می‌شود.
        
        ۶. انتخابات آزاد در اولین فرصت برگزار خواهد شد.
        
        \textbf{از همه شهروندان می‌خواهیم:}
        
        • آرامش خود را حفظ کنند
        
        • از انتقام‌جویی خودداری کنند
        
        • اموال عمومی را حفاظت کنند
        
        • به کار و زندگی عادی بازگردند
        
        \vspace{10pt}
        
        \textbf{ایران آزاد، دموکراتیک و سربلند در راه است.}
        
        \begin{flushright}
            شورای انتقالی ملی ایران\\
            تاریخ: [روز صفر]
        \end{flushright}
    };
\end{tikzpicture}
\end{center}

\subsection{سند ۱۸: پیام به جامعه جهانی}

\begin{center}
\begin{tikzpicture}
    \node[
        draw=SuccessGreen,
        line width=2pt,
        fill=SuccessGreen!5,
        rounded corners=10pt,
        inner sep=15pt,
        text width=14cm,
        align=justify
    ] {
        \begin{center}
            {\large\textbf{پیام جمهوری فدرال ایران}}\\[5pt]
            {\normalsize به جامعه جهانی و سازمان ملل متحد}
        \end{center}
        
        \vspace{10pt}
        
        عالی‌جنابان،
        
        با افتخار اعلام می‌کنیم که ملت ایران پس از دهه‌ها مبارزه، به آزادی دست یافته است. جمهوری فدرال ایران بر پایه دموکراسی، حقوق بشر و حاکمیت قانون بنا می‌شود.
        
        \textbf{تعهدات ما:}
        
        • احترام به حقوق بین‌الملل و منشور ملل متحد
        
        • پایبندی به تمام کنوانسیون‌های حقوق بشر
        
        • همزیستی مسالمت‌آمیز با همه کشورها
        
        • شفافیت در برنامه‌های هسته‌ای (صرفاً صلح‌آمیز)
        
        • مبارزه با تروریسم
        
        • همکاری منطقه‌ای و بین‌المللی
        
        \textbf{درخواست‌های ما:}
        
        • شناسایی دیپلماتیک دولت جدید
        
        • لغو تحریم‌های ظالمانه علیه ملت ایران
        
        • کمک به روند گذار دموکراتیک
        
        • اعزام ناظران بین‌المللی انتخابات
        
        ایران جدید دست دوستی به سوی همه ملت‌ها دراز می‌کند.
        
        \begin{flushright}
            وزیر امور خارجه موقت\\
            جمهوری فدرال ایران
        \end{flushright}
    };
\end{tikzpicture}
\end{center}

\subsection{سند ۱۹: بیانیه آشتی ملی}

\begin{naghlbox}
\textbf{بیانیه آشتی ملی}

ما، نمایندگان ملت ایران، در این لحظه تاریخی:

\textbf{می‌پذیریم} که در گذشته بی‌عدالتی‌هایی رخ داده است؛

\textbf{می‌دانیم} که زخم‌های عمیقی بر جان و روان هم‌میهنان ما نشسته است؛

\textbf{باور داریم} که آینده بهتر تنها با آشتی ممکن است؛

\textbf{متعهد می‌شویم} به:

• کشف حقیقت درباره آنچه گذشت

• اجرای عدالت برای قربانیان

• جبران خسارت به آسیب‌دیدگان

• تضمین عدم تکرار

\textbf{اما همچنین متعهد می‌شویم} به:

• پرهیز از انتقام‌جویی

• بخشش آنان که توبه می‌کنند

• ساختن آینده مشترک

• دیدن یکدیگر به‌عنوان هم‌میهن، نه دشمن

\textbf{ایران خانه همه ماست.} بیایید با هم بسازیمش.

\sourceline{شورای آشتی ملی}
\end{naghlbox}

% ═══════════════════════════════════════════════════════════════════════════════
\section{بخش هفتم: نمونه قراردادها و تفاهم‌نامه‌های بین‌المللی}
\label{sec:doc-international}
% ═══════════════════════════════════════════════════════════════════════════════

\subsection{سند ۲۰: چارچوب مذاکرات رفع تحریم}

\begin{center}
\begin{tikzpicture}
    % مراحل مذاکره
    \node[
        fill=DemocracyBlue,
        text=white,
        rounded corners=5pt,
        font=\bfseries,
        minimum width=12cm
    ] at (0,4) {نقشه راه رفع تحریم‌ها — چارچوب مذاکراتی};
    
    % مراحل
    \foreach \i/\title/\actions in {
        1/{مرحله ۱: اعتمادسازی (ماه ۱-۳)}/{اعلام تعهد به صلح، آزادسازی اتباع، شفافیت هسته‌ای},
        2/{مرحله ۲: توافق موقت (ماه ۴-۶)}/{تعلیق برخی تحریم‌ها، دسترسی به منابع بلوکه‌شده},
        3/{مرحله ۳: توافق جامع (ماه ۷-۱۲)}/{لغو تحریم‌های بانکی و نفتی، عادی‌سازی روابط},
        4/{مرحله ۴: اجرا و نظارت (سال ۲+)}/{نظارت بین‌المللی، لغو کامل تحریم‌ها}
    } {
        \node[
            draw=DemocracyBlue!70,
            fill=DemocracyBlue!10,
            rounded corners=5pt,
            minimum width=12cm,
            minimum height=1.5cm,
            align=center,
            font=\small
        ] at (0, 2.5-\i*1.8) {\textbf{\title}\\\scriptsize \actions};
    }
\end{tikzpicture}
\end{center}

\begin{longtable}{|>{\columncolor{DemocracyBlue!10}}p{3cm}|p{5cm}|p{5.5cm}|}
\hline
\rowcolor{DemocracyBlue!30}
\textbf{حوزه تحریم} & \textbf{اقدام ایران} & \textbf{اقدام طرف مقابل} \\
\hline
\endfirsthead

تحریم‌های هسته‌ای &
• پذیرش پروتکل الحاقی \newline
• بازرسی‌های گسترده IAEA \newline
• محدودیت غنی‌سازی &
• لغو تحریم‌های مرتبط با هسته‌ای \newline
• آزادسازی دارایی‌ها \newline
• همکاری فناوری صلح‌آمیز \\
\hline

تحریم‌های بانکی &
• پایبندی به FATF \newline
• شفافیت مالی \newline
• مبارزه با پولشویی &
• اتصال به سوئیفت \newline
• بازگشایی کانال‌های بانکی \newline
• دسترسی به منابع ارزی \\
\hline

تحریم‌های نفتی &
• شفافیت قراردادها \newline
• استانداردهای زیست‌محیطی &
• اجازه صادرات نفت \newline
• سرمایه‌گذاری در صنعت نفت \\
\hline

تحریم‌های تسلیحاتی &
• شفافیت بودجه دفاعی \newline
• عدم حمایت از گروه‌های مسلح &
• لغو تدریجی تحریم‌های تسلیحاتی \newline
• همکاری امنیتی منطقه‌ای \\
\hline

تحریم‌های حقوق بشری &
• آزادی زندانیان سیاسی \newline
• اصلاحات قضایی \newline
• لغو اعدام &
• لغو تحریم‌های فردی \newline
• همکاری حقوق بشری \\
\hline

\end{longtable}

\subsection{سند ۲۱: الگوی توافقنامه همکاری منطقه‌ای}

\begin{longtable}{|>{\columncolor{SuccessGreen!10}}r|p{12cm}|}
\hline
\rowcolor{SuccessGreen!30}
\textbf{ماده} & \textbf{متن} \\
\hline
\endfirsthead

\textbf{عنوان} &
\textbf{توافقنامه چارچوب همکاری منطقه‌ای}

بین جمهوری فدرال ایران و [کشور/کشورها] \\
\hline

\textbf{ماده ۱} &
\textbf{اهداف:}

الف) ارتقای صلح و ثبات منطقه‌ای

ب) توسعه همکاری‌های اقتصادی

ج) مدیریت مشترک منابع آب

د) مبارزه با تروریسم و قاچاق

هـ) همکاری فرهنگی و علمی \\
\hline

\textbf{ماده ۲} &
\textbf{اصول:}

الف) احترام به حاکمیت و استقلال

ب) عدم مداخله در امور داخلی

ج) حل مسالمت‌آمیز اختلافات

د) منافع متقابل \\
\hline

\textbf{ماده ۳} &
\textbf{حوزه‌های همکاری:}

الف) تجارت آزاد و تسهیل گمرکی

ب) ترانزیت و حمل‌ونقل

ج) انرژی (نفت، گاز، برق)

د) مدیریت آب‌های مشترک

هـ) گردشگری

و) آموزش و تحقیقات \\
\hline

\textbf{ماده ۴} &
\textbf{ساختار نهادی:}

الف) شورای عالی سران (سالانه)

ب) کمیسیون مشترک وزرا (شش‌ماهه)

ج) کمیته‌های تخصصی

د) دبیرخانه دائمی \\
\hline

\textbf{ماده ۵} &
\textbf{حل اختلاف:}

اختلافات از طریق مذاکره، میانجیگری، و در صورت نیاز داوری بین‌المللی حل می‌شود. \\
\hline

\textbf{ماده ۶} &
\textbf{مدت و بازنگری:}

این توافقنامه برای مدت ۱۰ سال معتبر است و خودبه‌خود تمدید می‌شود. بازنگری هر ۵ سال انجام می‌شود. \\
\hline

\end{longtable}

% ═══════════════════════════════════════════════════════════════════════════════
\section{بخش هشتم: راهنماهای آموزشی}
\label{sec:doc-guides}
% ═══════════════════════════════════════════════════════════════════════════════

\subsection{سند ۲۲: راهنمای ناظران انتخاباتی}

\begin{olgoobox}
\textbf{راهنمای ناظران انتخاباتی}

\textbf{وظایف ناظر:}
\begin{itemize}[nosep]
    \item حضور در شعبه از ابتدا تا انتهای رأی‌گیری
    \item نظارت بر رعایت مقررات
    \item ثبت تخلفات احتمالی
    \item حضور در شمارش آرا
    \item تهیه گزارش
\end{itemize}

\textbf{حقوق ناظر:}
\begin{itemize}[nosep]
    \item دسترسی کامل به شعبه
    \item مشاهده همه مراحل
    \item طرح سؤال از مسئولان
    \item اعتراض به تخلفات
    \item دریافت کپی صورتجلسه
\end{itemize}

\textbf{محدودیت‌های ناظر:}
\begin{itemize}[nosep]
    \item دخالت در فرآیند رأی‌گیری ممنوع
    \item تبلیغات ممنوع
    \item افشای نتایج قبل از اعلام رسمی ممنوع
\end{itemize}
\end{olgoobox}

\subsection{سند ۲۳: راهنمای شهروندی دوره گذار}

\begin{center}
\begin{tikzpicture}
    \node[
        draw=gray,
        fill=gray!5,
        rounded corners=10pt,
        inner sep=15pt,
        text width=14cm,
        align=right
    ] {
        \begin{center}
            {\large\textbf{حقوق و وظایف شهروندی در دوره گذار}}\\
            {\small راهنمای شهروندان}
        \end{center}
        
        \vspace{10pt}
        
        \textbf{حقوق شما:}
        
        ✓ آزادی بیان، تجمع و تشکل
        
        ✓ حق رأی در انتخابات آزاد
        
        ✓ حق دسترسی به اطلاعات
        
        ✓ حق شکایت از نهادهای دولتی
        
        ✓ برابری در برابر قانون
        
        ✓ مصونیت از بازداشت خودسرانه
        
        \vspace{10pt}
        
        \textbf{وظایف شما:}
        
        ◆ رعایت قانون و نظم عمومی
        
        ◆ احترام به حقوق دیگران
        
        ◆ مشارکت سازنده در فرآیند گذار
        
        ◆ پرهیز از خشونت و انتقام‌جویی
        
        ◆ حفاظت از اموال عمومی
        
        ◆ همکاری با نهادهای انتقالی
        
        \vspace{10pt}
        
        \textbf{در صورت مشکل:}
        
        ☎ خط اضطراری: ۱۱۰
        
        ☎ خط شکایات: ۱۱۱
        
        ☎ خط حقوق بشر: ۱۱۲
    };
\end{tikzpicture}
\end{center}

% ═══════════════════════════════════════════════════════════════════════════════
\section{خلاصه و فهرست اسناد}
\label{sec:doc-summary}
% ═══════════════════════════════════════════════════════════════════════════════

\begin{kholasebox}
\textbf{فهرست کامل اسناد پشتیبان}

\begin{center}
\begin{small}
\begin{longtable}{|r|p{6cm}|p{3cm}|p{3cm}|}
\hline
\rowcolor{gray!30}
\textbf{شماره} & \textbf{عنوان سند} & \textbf{نوع} & \textbf{کاربرد} \\
\hline
\endfirsthead

\multicolumn{4}{c}{\textbf{بخش ۱: پیش‌نویس قوانین}} \\
\hline
۱ & قانون شورای انتقالی & قانون & روز صفر \\
\hline
۲ & قانون دولت موقت & قانون & روز صفر \\
\hline
۳ & قانون انتخابات دوره گذار & قانون & ماه ۳-۶ \\
\hline
۴ & قانون عدالت انتقالی & قانون & ماه ۱-۳ \\
\hline

\multicolumn{4}{c}{\textbf{بخش ۲: پروتکل‌های عملیاتی}} \\
\hline
۵ & پروتکل مدیریت لحظه صفر & پروتکل & روز صفر \\
\hline
۶ & پروتکل امنیت دوره گذار & پروتکل & مستمر \\
\hline
۷ & پروتکل ارتباطات بحران & پروتکل & مستمر \\
\hline

\multicolumn{4}{c}{\textbf{بخش ۳: الگوهای توافقنامه}} \\
\hline
۸ & میثاق ملی گذار دموکراتیک & توافقنامه & پیش از گذار \\
\hline
۹ & توافقنامه همکاری بین‌قومی & توافقنامه & ماه ۱-۳ \\
\hline
۱۰ & تفاهم‌نامه نیروهای مسلح & توافقنامه & روز صفر \\
\hline

\multicolumn{4}{c}{\textbf{بخش ۴: فرم‌های اداری}} \\
\hline
۱۱ & فرم اعلام دارایی مقامات & فرم & مستمر \\
\hline
۱۲ & فرم گزارش فساد & فرم & مستمر \\
\hline
۱۳ & الگوی تعهدنامه کارکنان & فرم & ماه ۱-۶ \\
\hline

\multicolumn{4}{c}{\textbf{بخش ۵: چک‌لیست‌ها}} \\
\hline
۱۴ & چک‌لیست ۱۰۰ روز اول & چک‌لیست & روز ۱-۱۰۰ \\
\hline
۱۵ & چک‌لیست برگزاری انتخابات & چک‌لیست & ماه ۶-۱۸ \\
\hline
۱۶ & چک‌لیست استقرار فدرالیسم & چک‌لیست & سال ۱-۵ \\
\hline

\multicolumn{4}{c}{\textbf{بخش ۶: اعلامیه‌ها}} \\
\hline
۱۷ & اعلامیه روز صفر & اعلامیه & روز صفر \\
\hline
۱۸ & پیام به جامعه جهانی & اعلامیه & روز ۱-۷ \\
\hline
۱۹ & بیانیه آشتی ملی & بیانیه & ماه ۱-۳ \\
\hline

\multicolumn{4}{c}{\textbf{بخش ۷: اسناد بین‌المللی}} \\
\hline
۲۰ & چارچوب مذاکرات رفع تحریم & چارچوب & ماه ۱-۱۲ \\
\hline
۲۱ & الگوی توافقنامه منطقه‌ای & الگو & سال ۲+ \\
\hline

\multicolumn{4}{c}{\textbf{بخش ۸: راهنماها}} \\
\hline
۲۲ & راهنمای ناظران انتخاباتی & راهنما & ماه ۶+ \\
\hline
۲۳ & راهنمای شهروندی دوره گذار & راهنما & روز صفر+ \\
\hline

\end{longtable}
\end{small}
\end{center}

\textbf{نکته:} این اسناد به‌عنوان الگو و نقطه شروع طراحی شده‌اند. هر سند باید متناسب با شرایط واقعی، مشورت با متخصصان، و اجماع نیروهای سیاسی نهایی شود.
\end{kholasebox}

% ═══════════════════════════════════════════════════════════════════════════════
% پایان پیوست ۳
% ═══════════════════════════════════════════════════════════════════════════════
	% ═══════════════════════════════════════════════════════════════════════════════
% پیوست ۴: داده‌ها و جداول آماری
% فایل: app04-data.tex
% ═══════════════════════════════════════════════════════════════════════════════

\chapter{داده‌ها و جداول آماری}
\label{app:data}

\begin{kholasebox}
این پیوست مجموعه‌ای جامع از داده‌های آماری، جداول مقایسه‌ای، و نمودارهای تحلیلی را ارائه می‌دهد که پایه علمی و تجربی طرح گذار دموکراتیک را تشکیل می‌دهند. داده‌ها از منابع معتبر بین‌المللی (بانک جهانی، صندوق بین‌المللی پول، سازمان ملل، و مراکز پژوهشی) گردآوری شده‌اند. این اطلاعات برای برنامه‌ریزی، تصمیم‌گیری و پایش پیشرفت ضروری است.
\end{kholasebox}

% ═══════════════════════════════════════════════════════════════════════════════
\section{بخش اول: داده‌های جمعیتی و قومی}
\label{sec:data-demographic}
% ═══════════════════════════════════════════════════════════════════════════════

\subsection{جدول ۱: توزیع جمعیت بر اساس قومیت}

\begin{center}
\begin{tikzpicture}
    % نمودار دایره‌ای
    \pie[
        text=legend,
        radius=3,
        color={bleurepublique!70, goldlight!70, bleulight!70, golddark!70, bleurepublique!50, goldlight!50, bleulight!50, gray!50}
    ]{
        55/\rl{فارس},
        20/\rl{آذری},
        10/\rl{کرد},
        6/\rl{لر},
        3/\rl{عرب},
        2/\rl{بلوچ},
        1/\rl{ترکمن},
        3/\rl{سایر}
    }
    
    \node[font=\large\bfseries] at (0,-4.5) {توزیع قومی جمعیت ایران (تخمین ۱۴۰۳)};
\end{tikzpicture}
\end{center}

\begin{center}
\begin{small}
\begin{longtable}{|>{\columncolor{bleurepublique!10}}r|r|r|r|r|r|}
\hline
\rowcolor{bleurepublique!30}
\textbf{\rl{قوم}} & \textbf{\rl{جمعیت (میلیون)}} & \textbf{\rl{درصد}} & \textbf{\rl{زبان اصلی}} & \textbf{\rl{دین غالب}} & \textbf{\rl{استان‌های اصلی}} \\
\hline
\endfirsthead
\hline
\rowcolor{bleurepublique!30}
\textbf{\rl{قوم}} & \textbf{\rl{جمعیت (میلیون)}} & \textbf{\rl{درصد}} & \textbf{\rl{زبان اصلی}} & \textbf{\rl{دین غالب}} & \textbf{\rl{استان‌های اصلی}} \\
\hline
\endhead

فارس & ۴۷.۳ & ۵۵٪ & فارسی & شیعه & تهران، اصفهان، فارس، خراسان \\
\hline

آذری & ۱۷.۲ & ۲۰٪ & ترکی آذربایجانی & شیعه & آذربایجان شرقی و غربی، اردبیل، زنجان \\
\hline

کرد & ۸.۶ & ۱۰٪ & کردی & سنی/شیعه & کردستان، کرمانشاه، ایلام، آ.غربی \\
\hline

لر & ۵.۲ & ۶٪ & لری/بختیاری & شیعه & لرستان، چهارمحال، کهگیلویه \\
\hline

عرب & ۲.۶ & ۳٪ & عربی & شیعه & خوزستان \\
\hline

بلوچ & ۱.۷ & ۲٪ & بلوچی & سنی & سیستان و بلوچستان \\
\hline

ترکمن & ۰.۸۶ & ۱٪ & ترکمنی & سنی & گلستان \\
\hline

گیلک & ۲.۶ & ۳٪ & گیلکی & شیعه & گیلان \\
\hline

مازندرانی & ۲.۶ & ۳٪ & مازندرانی & شیعه & مازندران \\
\hline

سایر & ۱.۷ & ۲٪ & متنوع & متنوع & پراکنده \\
\hline

\rowcolor{gray!20}
\textbf{مجموع} & \textbf{۸۶.۰} & \textbf{۱۰۰٪} & — & — & — \\
\hline

\multicolumn{6}{l}{\scriptsize منبع: تخمین بر اساس داده‌های مرکز آمار ایران و مطالعات جمعیت‌شناختی | ارقام تقریبی است} \\

\end{longtable}
\end{small}
\end{center}

\subsection{جدول ۲: توزیع جمعیت بر اساس استان}

\begin{center}
\begin{small}
\begin{longtable}{|r|r|r|r|r|r|}
\hline
\rowcolor{goldlight!30}
\textbf{\rl{ردیف}} & \textbf{\rl{استان}} & \textbf{\rl{جمعیت (میلیون)}} & \textbf{\rl{مساحت (کیلومتر²)}} & \textbf{\rl{تراکم}} & \textbf{\rl{قوم غالب}} \\
\hline
\endfirsthead

۱ & تهران & ۱۴.۲ & ۱۳,۶۹۲ & ۱,۰۳۷ & فارس \\
\hline
۲ & خراسان رضوی & ۶.۸ & ۱۱۸,۸۵۴ & ۵۷ & فارس \\
\hline
۳ & اصفهان & ۵.۴ & ۱۰۷,۰۲۹ & ۵۰ & فارس \\
\hline
۴ & فارس & ۵.۰ & ۱۲۲,۶۰۸ & ۴۱ & فارس \\
\hline
۵ & آذربایجان شرقی & ۴.۰ & ۴۵,۶۵۱ & ۸۸ & آذری \\
\hline
۶ & خوزستان & ۴.۸ & ۶۴,۰۵۵ & ۷۵ & عرب/فارس/لر \\
\hline
۷ & آذربایجان غربی & ۳.۳ & ۳۷,۴۱۱ & ۸۸ & آذری/کرد \\
\hline
۸ & کرمان & ۳.۳ & ۱۸۰,۷۲۶ & ۱۸ & فارس \\
\hline
۹ & مازندران & ۳.۴ & ۲۳,۸۳۳ & ۱۴۳ & مازندرانی \\
\hline
۱۰ & گیلان & ۲.۶ & ۱۴,۰۴۲ & ۱۸۵ & گیلک \\
\hline
۱۱ & کردستان & ۱.۷ & ۲۹,۱۳۷ & ۵۸ & کرد \\
\hline
۱۲ & کرمانشاه & ۲.۰ & ۲۵,۰۰۹ & ۸۰ & کرد \\
\hline
۱۳ & سیستان و بلوچستان & ۲.۹ & ۱۸۰,۷۲۶ & ۱۶ & بلوچ \\
\hline
۱۴ & لرستان & ۱.۹ & ۲۸,۵۵۹ & ۶۷ & لر \\
\hline
۱۵ & گلستان & ۱.۹ & ۲۰,۱۹۵ & ۹۴ & فارس/ترکمن \\
\hline

\rowcolor{gray!20}
\multicolumn{2}{|r|}{\textbf{مجموع ایران}} & \textbf{۸۶.۰} & \textbf{۱,۶۴۸,۱۹۵} & \textbf{۵۲} & — \\
\hline

\end{longtable}
\end{small}
\end{center}

\subsection{جدول ۳: شاخص‌های توسعه انسانی به تفکیک منطقه}

\begin{center}
\begin{tikzpicture}
    % نمودار میله‌ای افقی
    \begin{axis}[
        xbar,
        width=14cm,
        height=10cm,
        xlabel={شاخص توسعه انسانی (HDI)},
        symbolic y coords={بلوچستان, کردستان, خوزستان, آذربایجان, لرستان, گیلان, فارس, اصفهان, تهران},
        ytick=data,
        xmin=0.5, xmax=0.9,
        bar width=12pt,
        nodes near coords,
        nodes near coords align={horizontal},
        every node near coord/.append style={font=\tiny},
    ]
    \addplot[fill=DemocracyBlue!70] coordinates {
        (0.82,تهران)
        (0.78,اصفهان)
        (0.76,فارس)
        (0.74,گیلان)
        (0.70,لرستان)
        (0.68,آذربایجان)
        (0.65,خوزستان)
        (0.62,کردستان)
        (0.55,بلوچستان)
    };
    \end{axis}
    
    \node[font=\small\bfseries] at (7,-1) {شاخص توسعه انسانی مناطق ایران (تخمین ۱۴۰۲)};
\end{tikzpicture}
\end{center}

\begin{center}
\begin{small}
\begin{longtable}{|>{\columncolor{bleurepublique!10}}p{3cm}|r|r|r|r|r|}
\hline
\rowcolor{bleurepublique!30}
\textbf{\rl{منطقه/استان}} & \textbf{\rl{HDI}} & \textbf{\rl{امید زندگی}} & \textbf{\rl{سواد (٪)}} & \textbf{\rl{درآمد سرانه (\$)}} & \textbf{\rl{شکاف با میانگین}} \\
\hline
\endfirsthead

تهران & ۰.۸۲ & ۷۷.۵ & ۹۸٪ & ۱۸,۵۰۰ & +۱۵٪ \\
\hline
اصفهان & ۰.۷۸ & ۷۶.۲ & ۹۶٪ & ۱۴,۲۰۰ & +۹٪ \\
\hline
فارس & ۰.۷۶ & ۷۵.۸ & ۹۵٪ & ۱۲,۸۰۰ & +۶٪ \\
\hline
\rowcolor{yellow!20}
\textbf{میانگین ملی} & \textbf{۰.۷۱} & \textbf{۷۴.۵} & \textbf{۹۲٪} & \textbf{۱۱,۰۰۰} & \textbf{—} \\
\hline
آذربایجان شرقی & ۰.۶۸ & ۷۴.۰ & ۹۰٪ & ۹,۸۰۰ & -۵٪ \\
\hline
کردستان & ۰.۶۲ & ۷۲.۵ & ۸۵٪ & ۷,۵۰۰ & -۱۳٪ \\
\hline
خوزستان & ۰.۶۵ & ۷۳.۰ & ۸۷٪ & ۸,۲۰۰ & -۹٪ \\
\hline
سیستان و بلوچستان & ۰.۵۵ & ۶۹.۵ & ۷۵٪ & ۴,۵۰۰ & -۲۳٪ \\
\hline

\end{longtable}
\end{small}
\end{center}

% ═══════════════════════════════════════════════════════════════════════════════
\section{بخش دوم: داده‌های اقتصادی}
\label{sec:data-economic}
% ═══════════════════════════════════════════════════════════════════════════════

\subsection{جدول ۴: شاخص‌های کلان اقتصادی ایران}

\begin{center}
\begin{small}
\begin{longtable}{|>{\columncolor{goldlight!10}}p{4cm}|r|r|r|r|r|}
\hline
\rowcolor{goldlight!30}
\textbf{\rl{شاخص}} & \textbf{\rl{۱۳۹۸}} & \textbf{\rl{۱۳۹۹}} & \textbf{\rl{۱۴۰۰}} & \textbf{\rl{۱۴۰۱}} & \textbf{\rl{۱۴۰۲ (تخمین)}} \\
\hline
\endfirsthead

تولید ناخالص داخلی (میلیارد \$) & ۴۴۵ & ۲۳۱ & ۳۵۹ & ۳۸۸ & ۴۰۱ \\
\hline

رشد اقتصادی (٪) & -۶.۸ & ۳.۳ & ۴.۷ & ۲.۹ & ۲.۵ \\
\hline

تورم سالانه (٪) & ۴۱.۲ & ۳۶.۴ & ۴۰.۲ & ۴۵.۸ & ۴۲.۵ \\
\hline

نرخ بیکاری (٪) & ۱۰.۷ & ۹.۴ & ۹.۲ & ۸.۹ & ۹.۱ \\
\hline

صادرات نفت (میلیون بشکه/روز) & ۰.۴ & ۰.۵ & ۰.۷ & ۱.۰ & ۱.۴ \\
\hline

درآمد نفتی (میلیارد \$) & ۸ & ۱۲ & ۲۵ & ۴۲ & ۵۵ \\
\hline

ذخایر ارزی (میلیارد \$) & ۸۶ & ۱۱۵ & ۱۲۲ & ۱۴۲ & ۱۵۵ \\
\hline

بدهی خارجی (میلیارد \$) & ۹.۳ & ۸.۷ & ۸.۲ & ۷.۸ & ۷.۵ \\
\hline

نرخ ارز (تومان/دلار) & ۱۱,۵۰۰ & ۲۵,۰۰۰ & ۲۷,۵۰۰ & ۴۲,۰۰۰ & ۵۵,۰۰۰ \\
\hline

\end{longtable}
\end{small}
\end{center}

\subsection{جدول ۵: ترکیب اقتصاد ایران}

\begin{center}
\begin{tikzpicture}
    % نمودار ستونی
    \begin{axis}[
        ybar,
        width=14cm,
        height=8cm,
        ylabel={درصد از GDP},
        symbolic x coords={نفت و گاز, صنعت, خدمات, کشاورزی, معدن, ساختمان},
        xtick=data,
        ymin=0, ymax=50,
        bar width=20pt,
        nodes near coords,
        every node near coord/.append style={font=\small},
        legend style={at={(0.5,-0.2)}, anchor=north},
    ]
    \addplot[fill=DemocracyBlue!70] coordinates {
        (نفت و گاز,18)
        (صنعت,22)
        (خدمات,45)
        (کشاورزی,9)
        (معدن,2)
        (ساختمان,4)
    };
    \end{axis}
    
    \node[font=\small\bfseries] at (7,-1.5) {ترکیب تولید ناخالص داخلی ایران (۱۴۰۲)};
\end{tikzpicture}
\end{center}

\subsection{جدول ۶: تحریم‌های بین‌المللی علیه ایران}

\begin{center}
\begin{small}
\begin{longtable}{|>{\columncolor{golddark!10}}p{2.5cm}|r|p{4cm}|p{5cm}|}
\hline
\rowcolor{golddark!30}
\textbf{\rl{منبع تحریم}} & \textbf{\rl{تعداد}} & \textbf{\rl{حوزه‌های اصلی}} & \textbf{\rl{تأثیر اقتصادی}} \\
\hline
\endfirsthead

آمریکا & ۳,۲۰۰+ & نفت، بانکی، تسلیحاتی، هسته‌ای، حقوق بشر & قطع از سوئیفت، محدودیت صادرات نفت، بلوکه دارایی‌ها \\
\hline

اتحادیه اروپا & ۴۵۰+ & نفت، بانکی، تسلیحاتی، حقوق بشر & محدودیت تجارت، ممنوعیت سرمایه‌گذاری \\
\hline

سازمان ملل & ۱۲۰+ & هسته‌ای، تسلیحاتی & ممنوعیت انتقال فناوری، تسلیحات \\
\hline

بریتانیا & ۲۸۰+ & مشابه EU + اقدامات مستقل & محدودیت‌های مالی \\
\hline

کانادا & ۱۵۰+ & حقوق بشر، هسته‌ای & محدودیت‌های تجاری و مالی \\
\hline

استرالیا & ۱۰۰+ & هسته‌ای، حقوق بشر & محدودیت‌های تجاری \\
\hline

\rowcolor{gray!20}
\textbf{مجموع (تخمین)} & \textbf{۳,۸۰۰+} & — & \textbf{خسارت سالانه: ۱۵۰-۲۰۰ میلیارد \$} \\
\hline

\end{longtable}
\end{small}
\end{center}

\subsection{جدول ۷: سناریوهای اقتصادی پس از رفع تحریم}

\begin{center}
\begin{small}
\begin{longtable}{|>{\columncolor{goldlight!10}}p{3cm}|p{3.5cm}|p{3.5cm}|p{3.5cm}|}
\hline
\rowcolor{goldlight!30}
\textbf{\rl{شاخص}} & \textbf{\rl{سناریوی حداقلی}} & \textbf{\rl{سناریوی میانه}} & \textbf{\rl{سناریوی حداکثری}} \\
\hline
\endfirsthead

\textbf{فرض پایه} & رفع جزئی تحریم‌ها & رفع اکثر تحریم‌ها & رفع کامل + اصلاحات \\
\hline

صادرات نفت (م.ب/روز) & ۲.۰ & ۳.۰ & ۴.۰ \\
\hline

درآمد نفتی (میلیارد \$/سال) & ۸۰ & ۱۲۰ & ۱۶۰ \\
\hline

رشد GDP (سال ۱-۵) & ۳-۴٪ & ۵-۷٪ & ۸-۱۰٪ \\
\hline

سرمایه‌گذاری خارجی (میلیارد \$/سال) & ۵-۱۰ & ۱۵-۲۵ & ۳۰-۵۰ \\
\hline

کاهش تورم (سال ۵) & ۲۵-۳۰٪ & ۱۵-۲۰٪ & ۸-۱۲٪ \\
\hline

کاهش بیکاری (سال ۵) & ۷-۸٪ & ۵-۶٪ & ۳-۴٪ \\
\hline

GDP سرانه (سال ۱۰) & ۱۵,۰۰۰ \$ & ۲۰,۰۰۰ \$ & ۲۵,۰۰۰ \$ \\
\hline

\end{longtable}
\end{small}
\end{center}

% ═══════════════════════════════════════════════════════════════════════════════
\section{بخش سوم: داده‌های بحران آب}
\label{sec:data-water}
% ═══════════════════════════════════════════════════════════════════════════════

\subsection{جدول ۸: بیلان آبی ایران}

\begin{center}
\begin{tikzpicture}
    % نمودار تعادل آبی
    \node[
        draw=DemocracyBlue,
        line width=2pt,
        fill=DemocracyBlue!20,
        rounded corners=10pt,
        minimum width=5cm,
        minimum height=2cm,
        font=\bfseries
    ] (supply) at (-4,0) {\shortstack{عرضه آب\\۹۷ میلیارد م³}};
    
    \node[
        draw=WarningRed,
        line width=2pt,
        fill=WarningRed!20,
        rounded corners=10pt,
        minimum width=5cm,
        minimum height=2cm,
        font=\bfseries
    ] (demand) at (4,0) {\shortstack{تقاضای آب\\۱۱۵ میلیارد م³}};
    
    \node[
        draw=WarningRed,
        line width=3pt,
        fill=WarningRed!40,
        rounded corners=10pt,
        minimum width=5cm,
        minimum height=1.5cm,
        font=\bfseries
    ] (deficit) at (0,-3) {\shortstack{کسری سالانه\\۱۸ میلیارد م³}};
    
    \draw[->, thick, DemocracyBlue] (supply) -- (deficit);
    \draw[->, thick, WarningRed] (demand) -- (deficit);
    
    % جزئیات
    \node[font=\scriptsize, align=center] at (-4,-1.5) {بارش: ۴۱۵ میلیمتر\\تبخیر: ۷۰٪\\رواناب: ۹۷ م³};
    
    \node[font=\scriptsize, align=center] at (4,-1.5) {کشاورزی: ۹۲٪\\صنعت: ۳٪\\شرب: ۵٪};
\end{tikzpicture}
\end{center}

\begin{center}
\begin{small}
\begin{longtable}{|>{\columncolor{bleurepublique!10}}p{4cm}|r|r|r|}
\hline
\rowcolor{bleurepublique!30}
\textbf{\rl{شاخص}} & \textbf{\rl{مقدار}} & \textbf{\rl{واحد}} & \textbf{\rl{وضعیت}} \\
\hline
\endfirsthead

بارش سالانه متوسط & ۲۵۰ & میلیمتر & ۱/۳ میانگین جهانی \\
\hline

منابع آب تجدیدپذیر & ۱۳۰ & میلیارد م³/سال & محدود \\
\hline

منابع آب قابل استحصال & ۹۷ & میلیارد م³/سال & — \\
\hline

مصرف کل آب & ۱۱۵ & میلیارد م³/سال & بیش از ظرفیت \\
\hline

\rowcolor{WarningRed!20}
کسری سالانه & ۱۸ & میلیارد م³/سال & بحرانی \\
\hline

برداشت از آب‌های زیرزمینی & ۶۵ & میلیارد م³/سال & ۱.۵ برابر ظرفیت \\
\hline

افت سالانه سطح آب زیرزمینی & ۵۰-۱۰۰ & سانتیمتر & بحرانی \\
\hline

تعداد دشت‌های ممنوعه & ۴۰۵ & از ۶۰۹ دشت & ۶۶٪ \\
\hline

سرانه آب تجدیدپذیر & ۱,۵۰۰ & م³/نفر/سال & کم‌آبی \\
\hline

\end{longtable}
\end{small}
\end{center}

\subsection{جدول ۹: وضعیت حوضه‌های آبریز}

\begin{center}
\begin{small}
\begin{longtable}{|>{\columncolor{bleurepublique!10}}r|p{3cm}|r|r|r|p{2.5cm}|}
\hline
\rowcolor{bleurepublique!30}
\textbf{\rl{ردیف}} & \textbf{\rl{حوضه آبریز}} & \textbf{\rl{مساحت (کیلومتر²)}} & \textbf{\rl{جمعیت (میلیون)}} & \textbf{\rl{کسری (م³/سال)}} & \textbf{\rl{وضعیت}} \\
\hline
\endfirsthead

۱ & دریاچه ارومیه & ۵۱,۸۷۶ & ۶.۴ & ۳.۵ میلیارد & \cellcolor{WarningRed!30}بحرانی \\
\hline

۲ & زاینده‌رود & ۴۱,۵۰۰ & ۵.۲ & ۲.۱ میلیارد & \cellcolor{WarningRed!30}بحرانی \\
\hline

۳ & کارون & ۶۵,۲۳۰ & ۴.۸ & ۱.۸ میلیارد & \cellcolor{orange!30}نگران‌کننده \\
\hline

۴ & هامون & ۱۰۶,۰۰۰ & ۲.۹ & ۲.۵ میلیارد & \cellcolor{WarningRed!30}بحرانی \\
\hline

۵ & قره‌سو & ۵۰,۲۰۰ & ۳.۱ & ۱.۲ میلیارد & \cellcolor{orange!30}نگران‌کننده \\
\hline

۶ & مرکزی & ۴۲۵,۰۰۰ & ۲۵.۰ & ۴.۵ میلیارد & \cellcolor{WarningRed!30}بحرانی \\
\hline

۷ & خلیج فارس & ۲۳۵,۰۰۰ & ۱۲.۵ & ۲.۰ میلیارد & \cellcolor{WisdomGold!30}متوسط \\
\hline

۸ & دریای خزر & ۱۸۰,۰۰۰ & ۱۵.۲ & ۰.۵ میلیارد & \cellcolor{SuccessGreen!30}نسبتاً خوب \\
\hline

\end{longtable}
\end{small}
\end{center}

\subsection{جدول ۱۰: برنامه شیرین‌سازی آب دریا}

\begin{center}
\begin{small}
\begin{longtable}{|>{\columncolor{goldlight!10}}p{3cm}|r|r|r|r|}
\hline
\rowcolor{goldlight!30}
\textbf{\rl{پروژه/منطقه}} & \textbf{\rl{ظرفیت (م³/روز)}} & \textbf{\rl{هزینه (میلیارد \$)}} & \textbf{\rl{زمان اجرا}} & \textbf{\rl{جمعیت بهره‌مند}} \\
\hline
\endfirsthead

بندرعباس-کرمان & ۴۰۰,۰۰۰ & ۲.۵ & ۱۴۰۵-۱۴۰۸ & ۳ میلیون \\
\hline

بوشهر-فارس & ۵۰۰,۰۰۰ & ۳.۲ & ۱۴۰۵-۱۴۰۹ & ۴ میلیون \\
\hline

چابهار-سیستان & ۳۰۰,۰۰۰ & ۲.۰ & ۱۴۰۶-۱۴۱۰ & ۲.۵ میلیون \\
\hline

خوزستان ساحلی & ۶۰۰,۰۰۰ & ۳.۸ & ۱۴۰۵-۱۴۱۰ & ۵ میلیون \\
\hline

خلیج فارس-یزد & ۳۵۰,۰۰۰ & ۲.۸ & ۱۴۰۷-۱۴۱۱ & ۳ میلیون \\
\hline

\rowcolor{yellow!20}
\textbf{مجموع فاز اول} & \textbf{۲,۱۵۰,۰۰۰} & \textbf{۱۴.۳} & \textbf{۱۴۰۵-۱۴۱۲} & \textbf{۱۷.۵ میلیون} \\
\hline

\end{longtable}
\end{small}
\end{center}

% ═══════════════════════════════════════════════════════════════════════════════
\section{بخش چهارم: شاخص‌های حکمرانی و دموکراسی}
\label{sec:data-governance}
% ═══════════════════════════════════════════════════════════════════════════════

\subsection{جدول ۱۱: رتبه ایران در شاخص‌های بین‌المللی}

\begin{center}
\begin{small}
\begin{longtable}{|>{\columncolor{bleulight!10}}p{4cm}|r|r|p{3cm}|p{3cm}|}
\hline
\rowcolor{bleulight!30}
\textbf{\rl{شاخص}} & \textbf{\rl{رتبه ایران}} & \textbf{\rl{از کل کشورها}} & \textbf{\rl{وضعیت}} & \textbf{\rl{هدف ۱۰ ساله}} \\
\hline
\endfirsthead

شاخص دموکراسی (EIU) & ۱۵۴ & ۱۶۷ & \cellcolor{WarningRed!30}استبدادی & رتبه ۸۰-۱۰۰ \\
\hline

آزادی مطبوعات (RSF) & ۱۷۶ & ۱۸۰ & \cellcolor{WarningRed!30}بسیار بد & رتبه ۱۰۰-۱۲۰ \\
\hline

شاخص فساد (TI) & ۱۴۹ & ۱۸۰ & \cellcolor{WarningRed!30}فاسد & رتبه ۸۰-۱۰۰ \\
\hline

حاکمیت قانون (WJP) & ۱۱۸ & ۱۴۲ & \cellcolor{WarningRed!30}ضعیف & رتبه ۶۰-۸۰ \\
\hline

آزادی اقتصادی (HF) & ۱۶۹ & ۱۷۶ & \cellcolor{WarningRed!30}سرکوب‌شده & رتبه ۱۰۰-۱۲۰ \\
\hline

توسعه انسانی (UNDP) & ۷۶ & ۱۹۱ & \cellcolor{WisdomGold!30}بالای متوسط & رتبه ۵۰-۶۰ \\
\hline

شاخص صلح جهانی & ۱۴۳ & ۱۶۳ & \cellcolor{WarningRed!30}پرتنش & رتبه ۸۰-۱۰۰ \\
\hline

سهولت کسب‌وکار (WB) & ۱۲۷ & ۱۹۰ & \cellcolor{orange!30}متوسط & رتبه ۵۰-۷۰ \\
\hline

شاخص نوآوری (WIPO) & ۶۲ & ۱۳۲ & \cellcolor{SuccessGreen!30}نسبتاً خوب & رتبه ۴۰-۵۰ \\
\hline

\end{longtable}
\end{small}
\end{center}

\subsection{جدول ۱۲: مقایسه با کشورهای منطقه}

\begin{center}
\begin{small}
\begin{longtable}{|>{\columncolor{bleurepublique!10}}p{2cm}|r|r|r|r|r|r|}
\hline
\rowcolor{bleurepublique!30}
\textbf{\rl{کشور}} & \textbf{\rl{جمعیت (م)}} & \textbf{\rl{GDP (\$ م)}} & \textbf{\rl{سرانه (\$)}} & \textbf{\rl{دموکراسی}} & \textbf{\rl{فساد}} & \textbf{\rl{HDI}} \\
\hline
\endfirsthead

\rowcolor{yellow!15}
ایران & ۸۶ & ۴۰۱ & ۴,۶۵۰ & ۱.۹۵ & ۲۵ & ۰.۷۷۴ \\
\hline

ترکیه & ۸۵ & ۹۰۵ & ۱۰,۶۵۰ & ۴.۳۵ & ۳۶ & ۰.۸۳۸ \\
\hline

عربستان & ۳۶ & ۱,۱۰۸ & ۳۰,۷۵۰ & ۲.۰۸ & ۵۳ & ۰.۸۷۵ \\
\hline

امارات & ۱۰ & ۵۰۷ & ۵۰,۶۵۰ & ۲.۷۶ & ۶۷ & ۰.۹۱۱ \\
\hline

پاکستان & ۲۳۱ & ۳۴۹ & ۱,۵۱۰ & ۴.۱۳ & ۲۷ & ۰.۵۴۴ \\
\hline

عراق & ۴۴ & ۲۶۷ & ۶,۰۷۰ & ۳.۶۲ & ۲۳ & ۰.۶۸۶ \\
\hline

مصر & ۱۱۰ & ۴۷۵ & ۴,۳۲۰ & ۲.۹۳ & ۳۰ & ۰.۷۳۱ \\
\hline

\multicolumn{7}{l}{\scriptsize منابع: بانک جهانی، EIU، Transparency International، UNDP | داده‌های ۲۰۲۳} \\

\end{longtable}
\end{small}
\end{center}

\subsection{جدول ۱۳: اهداف KPI برای دوره گذار}

\begin{center}
\begin{tikzpicture}
    % عنوان
    \node[
        fill=bleurepublique,
        text=white,
        rounded corners=5pt,
        font=\bfseries,
        minimum width=14cm
    ] at (0,6) {\rl{شاخص‌های کلیدی عملکرد (KPI) — اهداف ۲۵ ساله}};
    
    % جدول بصری
    \foreach \y/\name/\current/\y5/\y10/\y25 in {
        4.5/{شاخص دموکراسی}/۱.۹۵/۴.۵/۶.۵/۸.۰,
        3.5/{شاخص فساد}/۲۵/۳۵/۵۰/۷۰,
        2.5/{آزادی مطبوعات}/۱۷۶/۱۲۰/۸۰/۵۰,
        1.5/{HDI}/۰.۷۷/۰.۸۰/۰.۸۵/۰.۹۰,
        0.5/{GDP سرانه (\$)}/۴,۶۵۰/۸,۰۰۰/۱۵,۰۰۰/۳۰,۰۰۰
    } {
        % نام شاخص
        \node[font=\small, align=left] at (-5.5,\y) {\name};
        
        % مقادیر
        \node[font=\scriptsize, fill=WarningRed!30, rounded corners=2pt, minimum width=1.5cm] at (-2.5,\y) {\current};
        \node[font=\scriptsize, fill=orange!30, rounded corners=2pt, minimum width=1.5cm] at (-0.5,\y) {\y5};
        \node[font=\scriptsize, fill=WisdomGold!30, rounded corners=2pt, minimum width=1.5cm] at (1.5,\y) {\y10};
        \node[font=\scriptsize, fill=SuccessGreen!30, rounded corners=2pt, minimum width=1.5cm] at (3.5,\y) {\y25};
    }
    
    % عناوین ستون‌ها
    \node[font=\small\bfseries] at (-2.5,5.5) {فعلی};
    \node[font=\small\bfseries] at (-0.5,5.5) {سال ۵};
    \node[font=\small\bfseries] at (1.5,5.5) {سال ۱۰};
    \node[font=\small\bfseries] at (3.5,5.5) {سال ۲۵};
\end{tikzpicture}
\end{center}

% ═══════════════════════════════════════════════════════════════════════════════
\section{بخش پنجم: تجارب جهانی گذار دموکراتیک}
\label{sec:data-transitions}
% ═══════════════════════════════════════════════════════════════════════════════

\subsection{جدول ۱۴: مقایسه گذارهای موفق}

\begin{center}
\begin{small}
\begin{longtable}{|>{\columncolor{goldlight!10}}p{2cm}|r|p{2.5cm}|p{2.5cm}|p{2.5cm}|p{2.5cm}|}
\hline
\rowcolor{goldlight!30}
\textbf{\rl{کشور}} & \textbf{\rl{سال گذار}} & \textbf{\rl{نوع رژیم قبلی}} & \textbf{\rl{محرک گذار}} & \textbf{\rl{مدت تثبیت}} & \textbf{\rl{موفقیت کلی}} \\
\hline
\endfirsthead

اسپانیا & ۱۹۷۵-۷۸ & دیکتاتوری نظامی & فوت فرانکو & ۱۰ سال & \cellcolor{SuccessGreen!30}موفق \\
\hline

پرتغال & ۱۹۷۴ & دیکتاتوری & کودتای نظامی & ۱۲ سال & \cellcolor{SuccessGreen!30}موفق \\
\hline

شیلی & ۱۹۸۸-۹۰ & دیکتاتوری نظامی & همه‌پرسی & ۱۵ سال & \cellcolor{SuccessGreen!30}موفق \\
\hline

لهستان & ۱۹۸۹ & کمونیستی & میز گرد & ۸ سال & \cellcolor{SuccessGreen!30}موفق \\
\hline

آفریقای جنوبی & ۱۹۹۰-۹۴ & آپارتاید & مذاکره & ۱۰ سال & \cellcolor{SuccessGreen!30}موفق \\
\hline

اندونزی & ۱۹۹۸ & اقتدارگرا & بحران اقتصادی & ۱۵ سال & \cellcolor{WisdomGold!30}نسبتاً موفق \\
\hline

تونس & ۲۰۱۱ & دیکتاتوری & انقلاب & در حال تثبیت & \cellcolor{WisdomGold!30}چالش‌دار \\
\hline

\end{longtable}
\end{small}
\end{center}

\subsection{جدول ۱۵: مقایسه گذارهای ناموفق}

\begin{center}
\begin{small}
\begin{longtable}{|>{\columncolor{golddark!10}}p{2cm}|r|p{2.5cm}|p{2.5cm}|p{4cm}|}
\hline
\rowcolor{golddark!30}
\textbf{\rl{کشور}} & \textbf{\rl{سال}} & \textbf{\rl{نوع گذار}} & \textbf{\rl{نتیجه}} & \textbf{\rl{علل شکست}} \\
\hline
\endfirsthead

مصر & ۲۰۱۱-۱۳ & انقلاب & کودتا & عدم اجماع، اسلام‌گرایان، ارتش \\
\hline

لیبی & ۲۰۱۱ & سقوط رژیم & جنگ داخلی & فقدان نهاد، مداخله خارجی، قبیله‌گرایی \\
\hline

یمن & ۲۰۱۱ & قیام & جنگ داخلی & تنش‌های مذهبی، مداخله خارجی \\
\hline

سوریه & ۲۰۱۱ & قیام & جنگ داخلی & سرکوب، مداخله خارجی، تکثر مخالفان \\
\hline

عراق & ۲۰۰۳ & اشغال & بی‌ثباتی & مداخله نظامی، انحلال ارتش، فرقه‌گرایی \\
\hline

افغانستان & ۲۰۰۱ & اشغال & شکست & مداخله نظامی، فساد، طالبان \\
\hline

ونزوئلا & ۲۰۱۹ & بن‌بست & شکست & حمایت نظامی از رژیم، تحریم‌ها \\
\hline

\end{longtable}
\end{small}
\end{center}

\subsection{جدول ۱۶: درس‌های کلیدی از تجارب جهانی}

\begin{center}
\begin{tikzpicture}
    % دو ستون: موفقیت و شکست
    \node[
        draw=SuccessGreen,
        line width=2pt,
        fill=SuccessGreen!10,
        rounded corners=10pt,
        minimum width=6.5cm,
        minimum height=8cm,
        align=right
    ] (success) at (-3.5,0) {};
    
    \node[
        draw=WarningRed,
        line width=2pt,
        fill=WarningRed!10,
        rounded corners=10pt,
        minimum width=6.5cm,
        minimum height=8cm,
        align=right
    ] (failure) at (3.5,0) {};
    
    % عناوین
    \node[fill=goldlight, text=white, rounded corners=3pt, font=\bfseries] at (-3.5,3.5) {\rl{عوامل موفقیت}};
    \node[fill=golddark, text=white, rounded corners=3pt, font=\bfseries] at (3.5,3.5) {\rl{عوامل شکست}};
    
    % محتوای موفقیت
    \node[font=\small, align=right, text width=5.5cm] at (-3.5,1) {
        ✓ اجماع ملی گسترده\\[3pt]
        ✓ رهبری متحد و معتبر\\[3pt]
        ✓ حمایت یا بی‌طرفی ارتش\\[3pt]
        ✓ میثاق بین نیروها\\[3pt]
        ✓ عدالت انتقالی متوازن\\[3pt]
        ✓ حمایت بین‌المللی\\[3pt]
        ✓ اصلاحات اقتصادی سریع\\[3pt]
        ✓ نهادسازی مؤثر
    };
    
    % محتوای شکست
    \node[font=\small, align=right, text width=5.5cm] at (3.5,1) {
        ✗ تفرقه و رقابت نخبگان\\[3pt]
        ✗ مداخله نظامی خارجی\\[3pt]
        ✗ انتقام‌جویی و خشونت\\[3pt]
        ✗ فروپاشی اقتصادی\\[3pt]
        ✗ شکاف‌های قومی/مذهبی\\[3pt]
        ✗ فقدان نهادهای مدنی\\[3pt]
        ✗ بازگشت نظامیان\\[3pt]
        ✗ افراط‌گرایی
    };
\end{tikzpicture}
\end{center}

% ═══════════════════════════════════════════════════════════════════════════════
\section{بخش ششم: جداول زمان‌بندی پروژه}
\label{sec:data-timeline}
% ═══════════════════════════════════════════════════════════════════════════════

\subsection{جدول ۱۷: تقویم کلان ۲۵ ساله}

\begin{center}
\begin{tikzpicture}[scale=0.9]
    % محور زمانی
    \draw[->, thick, gray] (0,0) -- (15,0);
    
    % علامت‌گذاری سال‌ها
    \foreach \x/\year in {0/۰, 2.5/۵, 5/۱۰, 7.5/۱۵, 10/۲۰, 12.5/۲۵} {
        \draw[thick] (\x,0.15) -- (\x,-0.15);
        \node[below, font=\small] at (\x,-0.3) {سال \year};
    }
    
    % فازها
    \draw[fill=WarningRed!40, draw=WarningRed] (0,0.5) rectangle (1.5,1.5);
    \node[font=\tiny, align=center] at (0.75,1) {فاز ۱\\گذار};
    
    \draw[fill=orange!40, draw=orange] (1.5,0.5) rectangle (4,1.5);
    \node[font=\tiny, align=center] at (2.75,1) {فاز ۲\\نهادسازی};
    
    \draw[fill=WisdomGold!40, draw=WisdomGold] (4,0.5) rectangle (7.5,1.5);
    \node[font=\tiny, align=center] at (5.75,1) {فاز ۳\\تحکیم};
    
    \draw[fill=SuccessGreen!40, draw=SuccessGreen] (7.5,0.5) rectangle (10,1.5);
    \node[font=\tiny, align=center] at (8.75,1) {فاز ۴\\بلوغ};
    
    \draw[fill=DemocracyBlue!40, draw=DemocracyBlue] (10,0.5) rectangle (12.5,1.5);
    \node[font=\tiny, align=center] at (11.25,1) {فاز ۵\\تعالی};
    
    % رویدادهای کلیدی
    \node[circle, fill=WarningRed, minimum size=8pt] at (0.5,-0.8) {};
    \node[font=\tiny, below] at (0.5,-1) {انتخابات اول};
    
    \node[circle, fill=orange, minimum size=8pt] at (2.5,-0.8) {};
    \node[font=\tiny, below] at (2.5,-1) {قانون اساسی};
    
    \node[circle, fill=WisdomGold, minimum size=8pt] at (5,-0.8) {};
    \node[font=\tiny, below] at (5,-1) {رفع تحریم};
    
    \node[circle, fill=SuccessGreen, minimum size=8pt] at (7.5,-0.8) {};
    \node[font=\tiny, below] at (7.5,-1) {دموکراسی پایدار};
    
    \node[circle, fill=DemocracyBlue, minimum size=8pt] at (10,-0.8) {};
    \node[font=\tiny, below] at (10,-1) {توسعه‌یافتگی};
\end{tikzpicture}
\end{center}

\begin{center}
\begin{small}
\begin{longtable}{|>{\columncolor{gray!10}}p{1.5cm}|p{2cm}|p{3cm}|p{4cm}|p{3cm}|}
\hline
\rowcolor{gray!30}
\textbf{فاز} & \textbf{دوره} & \textbf{عنوان} & \textbf{اهداف اصلی} & \textbf{شاخص موفقیت} \\
\hline
\endfirsthead

\cellcolor{WarningRed!20} فاز ۱ & سال ۱-۲ & گذار اولیه & 
• انتخابات آزاد
• قانون اساسی
• ثبات اولیه &
انتخابات موفق، عدم خشونت \\
\hline

\cellcolor{orange!20} فاز ۲ & سال ۳-۵ & نهادسازی & 
• استقرار فدرالیسم
• اصلاح قضایی
• آغاز رفع تحریم &
نهادهای مستقل فعال \\
\hline

\cellcolor{WisdomGold!20} فاز ۳ & سال ۶-۱۰ & تحکیم & 
• رشد اقتصادی پایدار
• دموکراسی نهادینه
• انتقال قدرت مسالمت‌آمیز &
دو انتقال قدرت موفق \\
\hline

\cellcolor{SuccessGreen!20} فاز ۴ & سال ۱۱-۱۵ & بلوغ & 
• رسیدن به درآمد متوسط بالا
• عضویت در نهادهای بین‌المللی
• الگوی منطقه‌ای &
GDP سرانه ۲۰,۰۰۰\$ \\
\hline

\cellcolor{DemocracyBlue!20} فاز ۵ & سال ۱۶-۲۵ & تعالی & 
• دموکراسی کامل
• اقتصاد دانش‌بنیان
• قدرت منطقه‌ای سازنده &
جزو ۵۰ کشور برتر \\
\hline

\end{longtable}
\end{small}
\end{center}

\subsection{جدول ۱۸: بودجه تخمینی طرح گذار}

\begin{center}
\begin{small}
\begin{longtable}{|>{\columncolor{WisdomGold!10}}p{4cm}|r|r|r|r|r|}
\hline
\rowcolor{WisdomGold!30}
\textbf{بخش} & \textbf{سال ۱-۲} & \textbf{سال ۳-۵} & \textbf{سال ۶-۱۰} & \textbf{مجموع ۱۰ سال} & \textbf{منبع تأمین} \\
\hline
% ═══════════════════════════════════════════════════════════════════════════════
% ادامه پیوست ۴: داده‌ها و جداول آماری
% ادامه جداول بودجه و خلاصه
% ═══════════════════════════════════════════════════════════════════════════════

\endfirsthead

انتخابات و دموکراتیزاسیون & ۲.۵ & ۱.۵ & ۲.۰ & ۶.۰ & بودجه + کمک بین‌المللی \\
\hline

عدالت انتقالی & ۱.۰ & ۱.۵ & ۰.۵ & ۳.۰ & بودجه + مصادره \\
\hline

نهادسازی و اصلاحات & ۳.۰ & ۵.۰ & ۴.۰ & ۱۲.۰ & بودجه \\
\hline

زیرساخت‌های آب & ۵.۰ & ۱۵.۰ & ۳۰.۰ & ۵۰.۰ & بودجه + وام + سرمایه‌گذاری \\
\hline

زیرساخت‌های انرژی & ۳.۰ & ۱۰.۰ & ۲۵.۰ & ۳۸.۰ & بودجه + سرمایه‌گذاری \\
\hline

توسعه مناطق محروم & ۵.۰ & ۱۵.۰ & ۳۰.۰ & ۵۰.۰ & صندوق توازن \\
\hline

آموزش و بهداشت & ۸.۰ & ۲۰.۰ & ۴۰.۰ & ۶۸.۰ & بودجه جاری \\
\hline

حمایت اجتماعی & ۱۰.۰ & ۱۵.۰ & ۲۰.۰ & ۴۵.۰ & بودجه جاری \\
\hline

امنیت و دفاع & ۱۲.۰ & ۲۵.۰ & ۴۰.۰ & ۷۷.۰ & بودجه \\
\hline

\rowcolor{yellow!20}
\textbf{مجموع (میلیارد \$)} & \textbf{۴۹.۵} & \textbf{۱۰۸.۰} & \textbf{۱۹۱.۵} & \textbf{۳۴۹.۰} & — \\
\hline

\multicolumn{6}{l}{\scriptsize * ارقام تخمینی بر اساس تجارب مشابه و شرایط ایران | نیازمند بازنگری دقیق‌تر} \\

\end{longtable}
\end{small}
\end{center}

\subsection{جدول ۱۹: منابع تأمین مالی}

\begin{center}
\begin{tikzpicture}
    % نمودار دایره‌ای منابع
    \pie[
        text=legend,
        radius=3,
        color={DemocracyBlue!70, SuccessGreen!70, WisdomGold!70, orange!70, purple!70}
    ]{
        45/درآمد نفتی,
        25/مالیات و درآمد داخلی,
        15/سرمایه‌گذاری خارجی,
        10/وام بین‌المللی,
        5/کمک‌های بین‌المللی
    }
    
    \node[font=\bfseries] at (0,-4.5) {ترکیب تأمین مالی طرح گذار (۱۰ ساله)};
\end{tikzpicture}
\end{center}

\begin{center}
\begin{small}
\begin{longtable}{|>{\columncolor{SuccessGreen!10}}p{3.5cm}|r|p{4cm}|p{4cm}|}
\hline
\rowcolor{SuccessGreen!30}
\textbf{منبع} & \textbf{سهم (٪)} & \textbf{پیش‌شرط} & \textbf{ریسک} \\
\hline
\endfirsthead

درآمد نفت و گاز & ۴۵٪ & رفع تحریم، افزایش تولید & نوسانات قیمت، تحریم مجدد \\
\hline

مالیات و درآمد داخلی & ۲۵٪ & اصلاح نظام مالیاتی & فرار مالیاتی، ضعف اقتصاد \\
\hline

سرمایه‌گذاری خارجی (FDI) & ۱۵٪ & ثبات سیاسی، رفع تحریم & بی‌اعتمادی، رقابت منطقه‌ای \\
\hline

وام بین‌المللی (IMF/WB) & ۱۰٪ & اصلاحات ساختاری & شرایط سخت، بدهی \\
\hline

کمک‌های بین‌المللی & ۵٪ & روابط دیپلماتیک & محدود، مشروط \\
\hline

\end{longtable}
\end{small}
\end{center}

% ═══════════════════════════════════════════════════════════════════════════════
\section{بخش هفتم: داده‌های مقایسه‌ای فدرالیسم}
\label{sec:data-federalism}
% ═══════════════════════════════════════════════════════════════════════════════

\subsection{جدول ۲۰: مقایسه سیستم‌های فدرال جهان}

\begin{center}
\begin{small}
\begin{longtable}{|>{\columncolor{purple!10}}p{2cm}|r|r|p{2.5cm}|p{2.5cm}|p{3cm}|}
\hline
\rowcolor{purple!30}
\textbf{کشور} & \textbf{واحدها} & \textbf{جمعیت (م)} & \textbf{نوع فدرالیسم} & \textbf{مجلس دوم} & \textbf{ویژگی خاص} \\
\hline
\endfirsthead

آلمان & ۱۶ لَند & ۸۳ & همکارانه & بوندسرات (نمایندگی ایالات) & فدرالیسم مالی قوی \\
\hline

آمریکا & ۵۰ ایالت & ۳۳۱ & دوگانه & سنا (۲ نماینده هر ایالت) & حقوق ایالتی گسترده \\
\hline

هند & ۲۸ ایالت + ۸ منطقه & ۱,۴۰۰ & نامتقارن & راجیا سابها & تنوع زبانی بالا \\
\hline

سوئیس & ۲۶ کانتون & ۸.۷ & مشارکتی & شورای کانتون‌ها & دموکراسی مستقیم \\
\hline

کانادا & ۱۰ استان + ۳ منطقه & ۳۸ & نامتقارن & سنا (انتصابی) & دوزبانگی رسمی \\
\hline

استرالیا & ۶ ایالت + ۲ منطقه & ۲۶ & همکارانه & سنا (۱۲ از هر ایالت) & فدرالیسم مالی \\
\hline

بلژیک & ۳ منطقه + ۳ جامعه & ۱۱.۵ & زبانی & سنا & مبتنی بر زبان \\
\hline

اسپانیا & ۱۷ منطقه خودمختار & ۴۷ & نامتقارن & سنا & خودمختاری متفاوت \\
\hline

\rowcolor{yellow!15}
ایران (پیشنهادی) & ۵ منطقه + ۱۵ استان & ۸۶ & نامتقارن-همبسته & مجلس اقوام & فدرالیسم قومی \\
\hline

\end{longtable}
\end{small}
\end{center}

\subsection{جدول ۲۱: توزیع صلاحیت‌ها در نظام فدرال پیشنهادی}

\begin{center}
\begin{tikzpicture}
    % سه ستون
    \node[
        draw=DemocracyBlue,
        line width=2pt,
        fill=DemocracyBlue!15,
        rounded corners=10pt,
        minimum width=4.5cm,
        minimum height=9cm,
        align=center
    ] (fed) at (-5,0) {};
    
    \node[
        draw=SuccessGreen,
        line width=2pt,
        fill=SuccessGreen!15,
        rounded corners=10pt,
        minimum width=4.5cm,
        minimum height=9cm,
        align=center
    ] (shared) at (0,0) {};
    
    \node[
        draw=WisdomGold,
        line width=2pt,
        fill=WisdomGold!15,
        rounded corners=10pt,
        minimum width=4.5cm,
        minimum height=9cm,
        align=center
    ] (local) at (5,0) {};
    
    % عناوین
    \node[fill=DemocracyBlue, text=white, rounded corners=3pt, font=\small\bfseries] at (-5,4) {صلاحیت فدرال};
    \node[fill=SuccessGreen, text=white, rounded corners=3pt, font=\small\bfseries] at (0,4) {صلاحیت مشترک};
    \node[fill=WisdomGold, text=white, rounded corners=3pt, font=\small\bfseries] at (5,4) {صلاحیت منطقه‌ای};
    
    % محتوا
    \node[font=\scriptsize, align=right, text width=4cm] at (-5,1) {
        • دفاع ملی\\
        • سیاست خارجی\\
        • پول و بانک مرکزی\\
        • گمرک و تجارت خارجی\\
        • مهاجرت و تابعیت\\
        • ارتباطات ملی\\
        • انرژی هسته‌ای\\
        • حقوق اساسی
    };
    
    \node[font=\scriptsize, align=right, text width=4cm] at (0,1) {
        • آموزش عالی\\
        • بهداشت عمومی\\
        • محیط زیست\\
        • مدیریت آب\\
        • حمل‌ونقل بین‌استانی\\
        • انرژی\\
        • مالیات\\
        • امنیت داخلی
    };
    
    \node[font=\scriptsize, align=right, text width=4cm] at (5,1) {
        • آموزش پایه\\
        • فرهنگ و زبان\\
        • پلیس محلی\\
        • شهرسازی\\
        • خدمات اجتماعی\\
        • گردشگری\\
        • کشاورزی محلی\\
        • امور شهرداری‌ها
    };
\end{tikzpicture}
\end{center}

\subsection{جدول ۲۲: فرمول توزیع منابع صندوق توازن}

\begin{center}
\begin{small}
\begin{longtable}{|>{\columncolor{WisdomGold!10}}p{3cm}|r|p{5cm}|p{4cm}|}
\hline
\rowcolor{WisdomGold!30}
\textbf{معیار} & \textbf{وزن (٪)} & \textbf{شاخص سنجش} & \textbf{منطق} \\
\hline
\endfirsthead

جمعیت & ۵۰٪ & جمعیت رسمی سرشماری & عدالت توزیعی پایه \\
\hline

مساحت & ۲۰٪ & کیلومتر مربع & هزینه‌های زیرساختی \\
\hline

شاخص محرومیت & ۱۵٪ & ترکیب HDI، فقر، بیکاری & جبران عقب‌ماندگی \\
\hline

عملکرد & ۱۰٪ & شاخص‌های حکمرانی و کارایی & تشویق به بهبود \\
\hline

سهم تولیدکننده & ۵٪ & تولید منابع طبیعی & حقوق مناطق تولیدکننده \\
\hline

\end{longtable}
\end{small}
\end{center}

\subsubsection{نمونه محاسبه برای مناطق خودمختار}

\begin{center}
\begin{small}
\begin{longtable}{|>{\columncolor{gray!10}}p{2.5cm}|r|r|r|r|r|r|}
\hline
\rowcolor{gray!30}
\textbf{منطقه} & \textbf{جمعیت (٪)} & \textbf{مساحت (٪)} & \textbf{محرومیت} & \textbf{عملکرد} & \textbf{تولیدکننده} & \textbf{سهم نهایی (٪)} \\
\hline
\endfirsthead

آذربایجان & ۸.۵ & ۶.۵ & ۰.۷۵ & ۰.۸۰ & ۰.۵۰ & ۷.۸۵ \\
\hline

کردستان & ۴.۵ & ۵.۲ & ۱.۲۰ & ۰.۷۰ & ۰.۳۰ & ۵.۴۵ \\
\hline

خوزستان & ۵.۶ & ۳.۹ & ۱.۰۰ & ۰.۶۵ & ۳.۰۰ & ۶.۲۵ \\
\hline

بلوچستان & ۳.۴ & ۱۱.۰ & ۱.۸۰ & ۰.۵۰ & ۰.۲۰ & ۵.۹۰ \\
\hline

ترکمن‌صحرا & ۱.۰ & ۱.۲ & ۰.۹۰ & ۰.۶۰ & ۱.۵۰ & ۱.۴۵ \\
\hline

\rowcolor{yellow!20}
\textbf{جمع مناطق خودمختار} & \textbf{۲۳.۰} & \textbf{۲۷.۸} & — & — & — & \textbf{۲۶.۹۰} \\
\hline

\multicolumn{7}{l}{\scriptsize فرمول: سهم = (۰.۵ × جمعیت) + (۰.۲ × مساحت) + (۰.۱۵ × محرومیت) + (۰.۱ × عملکرد) + (۰.۰۵ × تولیدکننده)} \\

\end{longtable}
\end{small}
\end{center}

% ═══════════════════════════════════════════════════════════════════════════════
\section{بخش هشتم: شاخص‌های پایش و ارزیابی}
\label{sec:data-monitoring}
% ═══════════════════════════════════════════════════════════════════════════════

\subsection{جدول ۲۳: داشبورد ملی پایش گذار}

\begin{center}
\begin{tikzpicture}
    % کادر اصلی داشبورد
    \node[
        draw=DemocracyBlue,
        line width=2pt,
        fill=white,
        rounded corners=15pt,
        minimum width=15cm,
        minimum height=10cm
    ] at (0,0) {};
    
    % عنوان
    \node[fill=DemocracyBlue, text=white, rounded corners=5pt, font=\large\bfseries, minimum width=10cm] at (0,4.5) {داشبورد ملی پایش گذار دموکراتیک};
    
    % چهار بخش
    % بخش ۱: سیاسی
    \node[
        draw=WarningRed,
        fill=WarningRed!10,
        rounded corners=5pt,
        minimum width=6.5cm,
        minimum height=3.5cm
    ] at (-3.5,1.5) {};
    \node[fill=WarningRed, text=white, rounded corners=3pt, font=\small\bfseries] at (-3.5,3) {شاخص‌های سیاسی};
    \node[font=\tiny, align=right] at (-3.5,1.2) {
        دموکراسی: ۴.۵/۱۰ \quad {\color{orange}▲}\\
        آزادی مطبوعات: ۱۲۰/۱۸۰ \quad {\color{SuccessGreen}▲}\\
        مشارکت انتخاباتی: ۶۵٪ \quad {\color{SuccessGreen}▲}\\
        اعتماد به دولت: ۴۵٪ \quad {\color{orange}▲}
    };
    
    % بخش ۲: اقتصادی
    \node[
        draw=SuccessGreen,
        fill=SuccessGreen!10,
        rounded corners=5pt,
        minimum width=6.5cm,
        minimum height=3.5cm
    ] at (3.5,1.5) {};
    \node[fill=SuccessGreen, text=white, rounded corners=3pt, font=\small\bfseries] at (3.5,3) {شاخص‌های اقتصادی};
    \node[font=\tiny, align=right] at (3.5,1.2) {
        رشد GDP: ۵.۲٪ \quad {\color{SuccessGreen}▲}\\
        تورم: ۲۵٪ \quad {\color{orange}▼}\\
        بیکاری: ۸.۵٪ \quad {\color{SuccessGreen}▼}\\
        سرمایه‌گذاری خارجی: ۱۲B\$ \quad {\color{SuccessGreen}▲}
    };
    
    % بخش ۳: اجتماعی
    \node[
        draw=WisdomGold,
        fill=WisdomGold!10,
        rounded corners=5pt,
        minimum width=6.5cm,
        minimum height=3.5cm
    ] at (-3.5,-2.5) {};
    \node[fill=WisdomGold, text=white, rounded corners=3pt, font=\small\bfseries] at (-3.5,-1) {شاخص‌های اجتماعی};
    \node[font=\tiny, align=right] at (-3.5,-2.8) {
        HDI: ۰.۷۹ \quad {\color{SuccessGreen}▲}\\
        نرخ فقر: ۱۸٪ \quad {\color{orange}▼}\\
        نابرابری (ژینی): ۰.۳۸ \quad {\color{SuccessGreen}▼}\\
        دسترسی به بهداشت: ۸۵٪ \quad {\color{SuccessGreen}▲}
    };
    
    % بخش ۴: محیطی
    \node[
        draw=purple,
        fill=purple!10,
        rounded corners=5pt,
        minimum width=6.5cm,
        minimum height=3.5cm
    ] at (3.5,-2.5) {};
    \node[fill=purple, text=white, rounded corners=3pt, font=\small\bfseries] at (3.5,-1) {شاخص‌های محیطی};
    \node[font=\tiny, align=right] at (3.5,-2.8) {
        بیلان آبی: -۱۵B م³ \quad {\color{orange}▲}\\
        آلودگی هوا: ۱۲۰ AQI \quad {\color{WarningRed}—}\\
        انرژی تجدیدپذیر: ۸٪ \quad {\color{SuccessGreen}▲}\\
        جنگل: ۷.۵٪ \quad {\color{orange}▲}
    };
\end{tikzpicture}
\end{center}

\subsection{جدول ۲۴: ماتریس شاخص‌های KPI}

\begin{center}
\begin{small}
\begin{longtable}{|>{\columncolor{DemocracyBlue!10}}p{3cm}|p{2.5cm}|r|r|r|r|p{1.5cm}|}
\hline
\rowcolor{DemocracyBlue!30}
\textbf{شاخص} & \textbf{منبع داده} & \textbf{خط پایه} & \textbf{هدف سال ۵} & \textbf{هدف سال ۱۰} & \textbf{فعلی} & \textbf{وضعیت} \\
\hline
\endfirsthead

\multicolumn{7}{c}{\textbf{— حوزه سیاسی —}} \\
\hline

شاخص دموکراسی EIU & EIU & ۱.۹۵ & ۴.۵ & ۶.۵ & — & پایه \\
\hline

آزادی مطبوعات RSF & RSF & ۱۷۶ & ۱۲۰ & ۸۰ & — & پایه \\
\hline

مشارکت انتخاباتی & کمیسیون & — & ۶۵٪ & ۷۵٪ & — & — \\
\hline

زنان در پارلمان & IPU & ۵٪ & ۲۵٪ & ۳۵٪ & — & پایه \\
\hline

\multicolumn{7}{c}{\textbf{— حوزه اقتصادی —}} \\
\hline

GDP سرانه (\$) & WB & ۴,۶۵۰ & ۸,۰۰۰ & ۱۵,۰۰۰ & — & پایه \\
\hline

نرخ تورم (٪) & بانک مرکزی & ۴۵ & ۲۰ & ۱۰ & — & پایه \\
\hline

نرخ بیکاری (٪) & مرکز آمار & ۹.۱ & ۷ & ۵ & — & پایه \\
\hline

شاخص فساد TI & TI & ۲۵ & ۴۰ & ۵۵ & — & پایه \\
\hline

\multicolumn{7}{c}{\textbf{— حوزه اجتماعی —}} \\
\hline

شاخص توسعه انسانی & UNDP & ۰.۷۷ & ۰.۸۲ & ۰.۸۷ & — & پایه \\
\hline

نرخ فقر (٪) & WB & ۲۵ & ۱۵ & ۸ & — & پایه \\
\hline

ضریب جینی & WB & ۰.۴۲ & ۰.۳۸ & ۰.۳۲ & — & پایه \\
\hline

امید به زندگی (سال) & WHO & ۷۴.۵ & ۷۶ & ۷۸ & — & پایه \\
\hline

\multicolumn{7}{c}{\textbf{— حوزه محیط زیست —}} \\
\hline

کسری آب (میلیارد م³) & وزارت نیرو & ۱۸ & ۱۲ & ۵ & — & پایه \\
\hline

انرژی تجدیدپذیر (٪) & IEA & ۲ & ۱۰ & ۲۵ & — & پایه \\
\hline

انتشار CO2 سرانه (تن) & WB & ۸.۵ & ۷.۵ & ۶ & — & پایه \\
\hline

\end{longtable}
\end{small}
\end{center}

\subsection{جدول ۲۵: سیستم هشدار زودهنگام}

\begin{center}
\begin{small}
\begin{longtable}{|>{\columncolor{WarningRed!10}}p{3cm}|p{3.5cm}|p{2cm}|p{2cm}|p{3cm}|}
\hline
\rowcolor{WarningRed!30}
\textbf{ریسک} & \textbf{شاخص هشدار} & \textbf{آستانه زرد} & \textbf{آستانه قرمز} & \textbf{اقدام واکنشی} \\
\hline
\endfirsthead

بی‌ثباتی سیاسی & اعتراضات خشونت‌آمیز & ۵ مورد/ماه & ۱۵ مورد/ماه & گفتگوی ملی، امتیازات \\
\hline

تنش قومی & درگیری‌های قومی & ۳ مورد/ماه & ۱۰ مورد/ماه & شورای آشتی، میانجیگری \\
\hline

فروپاشی اقتصادی & کاهش GDP & -۳٪ & -۷٪ & بسته نجات، کمک بین‌المللی \\
\hline

تورم افسارگسیخته & نرخ تورم ماهانه & ۵٪ & ۱۰٪ & سیاست پولی انقباضی \\
\hline

بحران ارزی & کاهش ارزش ریال & ۳۰٪/سال & ۶۰٪/سال & مداخله ارزی، کنترل \\
\hline

کودتای نظامی & فعالیت‌های مشکوک & شایعات پایدار & حرکت نیروها & فعال‌سازی پروتکل ضدکودتا \\
\hline

مداخله خارجی & تحریکات مرزی & ۳ مورد/ماه & ۱۰ مورد/ماه & دیپلماسی فعال، هشدار UN \\
\hline

بحران انسانی & جابجایی داخلی & ۵۰,۰۰۰ نفر & ۲۰۰,۰۰۰ نفر & کمک اضطراری، UNHCR \\
\hline

\end{longtable}
\end{small}
\end{center}

% ═══════════════════════════════════════════════════════════════════════════════
\section{خلاصه و منابع داده‌ها}
\label{sec:data-summary}
% ═══════════════════════════════════════════════════════════════════════════════

\begin{kholasebox}
\textbf{خلاصه پیوست داده‌ها و جداول آماری}

\begin{center}
\begin{tabular}{r r}
\textbf{بخش} & \textbf{تعداد جداول} \\
\hline
داده‌های جمعیتی و قومی & ۳ جدول \\
داده‌های اقتصادی & ۴ جدول \\
داده‌های بحران آب & ۳ جدول \\
شاخص‌های حکمرانی & ۳ جدول \\
تجارب جهانی & ۳ جدول \\
جداول زمان‌بندی و بودجه & ۳ جدول \\
داده‌های فدرالیسم & ۳ جدول \\
شاخص‌های پایش & ۳ جدول \\
\hline
\textbf{مجموع} & \textbf{۲۵ جدول} \\
\end{tabular}
\end{center}
\end{kholasebox}

\subsection{منابع اصلی داده‌ها}

\begin{longtable}{|>{\columncolor{gray!10}}p{4cm}|p{5cm}|p{5cm}|}
\hline
\rowcolor{gray!30}
\textbf{سازمان} & \textbf{نوع داده} & \textbf{آدرس} \\
\hline
\endfirsthead

بانک جهانی (World Bank) & اقتصادی، توسعه، فقر & data.worldbank.org \\
\hline

صندوق بین‌المللی پول (IMF) & اقتصاد کلان، مالی & imf.org/data \\
\hline

سازمان ملل (UN) & جمعیت، HDI، محیط زیست & data.un.org \\
\hline

UNDP & توسعه انسانی & hdr.undp.org \\
\hline

Transparency International & فساد & transparency.org \\
\hline

Freedom House & آزادی سیاسی و مدنی & freedomhouse.org \\
\hline

Economist Intelligence Unit & شاخص دموکراسی & eiu.com \\
\hline

گزارشگران بدون مرز (RSF) & آزادی مطبوعات & rsf.org \\
\hline

مرکز آمار ایران & داده‌های داخلی & amar.org.ir \\
\hline

FAO & کشاورزی، آب، غذا & fao.org/faostat \\
\hline

IEA & انرژی & iea.org \\
\hline

WHO & بهداشت & who.int/data \\
\hline

\end{longtable}

\begin{enghelabbox}
\textbf{⚠️ تذکر مهم درباره داده‌ها}

\begin{itemize}[nosep]
    \item برخی داده‌های مربوط به ایران به دلیل محدودیت دسترسی، \textbf{تخمینی} هستند.
    \item داده‌های قومی بر اساس برآوردهای غیررسمی است زیرا سرشماری رسمی قومی وجود ندارد.
    \item شاخص‌های آینده بر اساس \textbf{سناریوهای خوش‌بینانه} با فرض اجرای کامل طرح است.
    \item داده‌ها باید پس از گذار با \textbf{سرشماری و آمارگیری جدید} به‌روزرسانی شوند.
    \item برای برنامه‌ریزی دقیق، همکاری با نهادهای بین‌المللی آماری ضروری است.
\end{itemize}
\end{enghelabbox}

% ═══════════════════════════════════════════════════════════════════════════════
% پایان پیوست ۴
% ═══════════════════════════════════════════════════════════════════════════════
	% ═══════════════════════════════════════════════════════════════════════════════
% پیوست ۵: واژه‌نامه تخصصی
% فایل: app05-glossary.tex
% ═══════════════════════════════════════════════════════════════════════════════

\chapter{واژه‌نامه تخصصی}
\label{app:glossary}

\begin{kholasebox}
این واژه‌نامه شامل تعریف اصطلاحات تخصصی به‌کاررفته در کتاب است. واژه‌ها به ترتیب موضوعی و سپس الفبایی (فارسی و انگلیسی) مرتب شده‌اند. هدف از این واژه‌نامه، ایجاد زبان مشترک برای بحث درباره گذار دموکراتیک و ارائه تعاریف دقیق علمی است. معادل‌های انگلیسی برای تسهیل ارجاع به منابع بین‌المللی ذکر شده‌اند.
\end{kholasebox}

% ═══════════════════════════════════════════════════════════════════════════════
\section{بخش اول: مفاهیم سیاسی و حکمرانی}
\label{sec:glossary-political}
% ═══════════════════════════════════════════════════════════════════════════════

\begin{longtable}{|>{\columncolor{bleurepublique!10}}p{3.5cm}|p{3cm}|p{7.5cm}|}
\hline
\rowcolor{bleurepublique!30}
\textbf{\rl{اصطلاح فارسی}} & \textbf{\rl{معادل انگلیسی}} & \textbf{\rl{تعریف}} \\
\hline
\endfirsthead
\hline
\rowcolor{bleurepublique!30}
\textbf{\rl{اصطلاح فارسی}} & \textbf{\rl{معادل انگلیسی}} & \textbf{\rl{تعریف}} \\
\hline
\endhead

استبداد & Autocracy / Despotism & نظام سیاسی که در آن قدرت در دست یک فرد یا گروه کوچک متمرکز است و مردم نقشی در تصمیم‌گیری ندارند. \\
\hline

اقتدارگرایی & Authoritarianism & نظامی که در آن قدرت سیاسی متمرکز است، آزادی‌های مدنی محدود است، اما ایدئولوژی فراگیر توتالیتر ندارد. \\
\hline

انتقال قدرت & Power Transfer & فرآیند واگذاری مسالمت‌آمیز قدرت از یک دولت به دولت دیگر، معمولاً پس از انتخابات. \\
\hline

پارلمانتاریسم & Parliamentarism & نظام حکومتی که در آن قوه مجریه از دل قوه مقننه برمی‌خیزد و در برابر آن پاسخگوست. \\
\hline

پاسخگویی & Accountability & الزام مقامات به توضیح تصمیمات و اقدامات خود و پذیرش پیامدهای آنها. \\
\hline

پلورالیسم & Pluralism & پذیرش و ارزش‌گذاری تنوع نظرات، باورها و گروه‌های اجتماعی در جامعه و سیاست. \\
\hline

تثبیت دموکراتیک & Democratic Consolidation & فرآیندی که طی آن دموکراسی جدید نهادینه می‌شود و «تنها بازی در شهر» می‌گردد. \\
\hline

تفکیک قوا & Separation of Powers & تقسیم قدرت حکومتی به سه قوه مجزا (مقننه، مجریه، قضائیه) برای جلوگیری از تمرکز قدرت. \\
\hline

توتالیتاریسم & Totalitarianism & نظام سیاسی که در آن دولت کنترل کامل بر همه جنبه‌های زندگی عمومی و خصوصی دارد. \\
\hline

جامعه مدنی & Civil Society & مجموعه نهادها، سازمان‌ها و انجمن‌های داوطلبانه مستقل از دولت که منافع شهروندان را نمایندگی می‌کنند. \\
\hline

جمهوری & Republic & نظام حکومتی که در آن قدرت متعلق به مردم است و رئیس کشور منتخب است (نه موروثی). \\
\hline

حاکمیت قانون & Rule of Law & اصلی که بر اساس آن همه، از جمله حاکمان، تابع قانون هستند و قانون عادلانه و برابر اجرا می‌شود. \\
\hline

حاکمیت ملی & National Sovereignty & اصل حقوقی که قدرت سیاسی نهایی را متعلق به ملت می‌داند. \\
\hline

حق تعیین سرنوشت & Self-Determination & حق مردم یک سرزمین برای تعیین وضعیت سیاسی و پیگیری توسعه اقتصادی، اجتماعی و فرهنگی خود. \\
\hline

حکمرانی خوب & Good Governance & اداره امور عمومی به شیوه‌ای شفاف، پاسخگو، کارآمد، مشارکتی و مبتنی بر قانون. \\
\hline

دموکراسی & Democracy & نظام حکومتی که در آن قدرت سیاسی از مردم ناشی می‌شود و از طریق انتخابات آزاد و منظم اعمال می‌گردد. \\
\hline

دموکراسی اجماعی & Consociational Democracy & مدل دموکراتیک برای جوامع چندپاره که بر ائتلاف بزرگ، وتوی متقابل، تناسب و خودمختاری بخشی استوار است. \\
\hline

دموکراسی مستقیم & Direct Democracy & نظامی که در آن شهروندان مستقیماً (نه از طریق نمایندگان) در تصمیم‌گیری سیاسی شرکت می‌کنند. \\
\hline

دموکراسی نمایندگی & Representative Democracy & نظامی که در آن شهروندان نمایندگانی انتخاب می‌کنند تا به نیابت از آنها تصمیم‌گیری کنند. \\
\hline

رأی همگانی & Universal Suffrage & حق رأی برای همه شهروندان بالغ بدون تبعیض. \\
\hline

ریاست‌جمهوری & Presidentialism & نظام حکومتی که در آن رئیس‌جمهور هم رئیس کشور و هم رئیس دولت است و مستقیماً انتخاب می‌شود. \\
\hline

سکولاریسم & Secularism & اصل جدایی دین از دولت و بی‌طرفی حکومت نسبت به ادیان. \\
\hline

شفافیت & Transparency & دسترسی آزاد شهروندان به اطلاعات دولتی و علنی بودن فرآیندهای تصمیم‌گیری. \\
\hline

کثرت‌گرایی & Pluralism & نظریه‌ای که تنوع گروه‌ها و منافع در جامعه را به رسمیت می‌شناسد و ارزش می‌نهد. \\
\hline

گذار دموکراتیک & Democratic Transition & فرآیند حرکت از یک نظام اقتدارگرا به یک نظام دموکراتیک. \\
\hline

لیبرالیسم & Liberalism & فلسفه سیاسی که بر آزادی فردی، حقوق طبیعی، حکومت محدود و برابری در برابر قانون تأکید دارد. \\
\hline

مشارکت سیاسی & Political Participation & فعالیت شهروندان در فرآیندهای سیاسی از جمله رأی‌دادن، عضویت در احزاب و فعالیت مدنی. \\
\hline

مشروطیت & Constitutionalism & اصل حکومت بر اساس قانون اساسی که قدرت حکومت را محدود می‌کند. \\
\hline

مصونیت پارلمانی & Parliamentary Immunity & حمایت قانونی از نمایندگان در برابر تعقیب به خاطر اظهارات و آرایشان در مجلس. \\
\hline

نظارت و موازنه & Checks and Balances & سازوکارهایی که هر قوه را قادر می‌سازد قدرت قوای دیگر را محدود کند. \\
\hline

نظام انتخاباتی & Electoral System & مجموعه قواعدی که چگونگی تبدیل آرا به کرسی‌ها را تعیین می‌کند. \\
\hline

همه‌پرسی & Referendum & رأی‌گیری مستقیم از مردم درباره یک موضوع یا قانون خاص. \\
\hline

\end{longtable}

% ═══════════════════════════════════════════════════════════════════════════════
\section{بخش دوم: مفاهیم فدرالیسم و تنوع}
\label{sec:glossary-federalism}
% ═══════════════════════════════════════════════════════════════════════════════

\begin{longtable}{|>{\columncolor{goldlight!10}}p{3.5cm}|p{3cm}|p{7.5cm}|}
\hline
\rowcolor{goldlight!30}
\textbf{\rl{اصطلاح فارسی}} & \textbf{\rl{معادل انگلیسی}} & \textbf{\rl{تعریف}} \\
\hline
\endfirsthead

اقلیت & Minority & گروهی از جمعیت که از نظر عددی کمتر از اکثریت است و معمولاً از نظر قدرت سیاسی در موقعیت ضعیف‌تری قرار دارد. \\
\hline

تبعیض مثبت & Affirmative Action & سیاست‌هایی که به نفع گروه‌های تاریخاً محروم برای جبران نابرابری‌های گذشته اعمال می‌شود. \\
\hline

تمرکززدایی & Decentralization & انتقال قدرت و منابع از حکومت مرکزی به سطوح پایین‌تر (استانی، محلی). \\
\hline

تنوع فرهنگی & Cultural Diversity & وجود فرهنگ‌ها، زبان‌ها و سنت‌های متفاوت در یک جامعه. \\
\hline

حقوق اقلیت‌ها & Minority Rights & حقوق ویژه‌ای که برای حمایت از گروه‌های اقلیت در برابر تصمیمات اکثریت تضمین می‌شود. \\
\hline

حقوق جمعی & Collective Rights & حقوقی که به یک گروه (نه فرد) تعلق دارد، مانند حق یک قوم برای حفظ زبان خود. \\
\hline

خودمختاری & Autonomy & درجه‌ای از استقلال در اداره امور داخلی یک منطقه در چارچوب یک کشور بزرگ‌تر. \\
\hline

دولت-ملت & Nation-State & واحد سیاسی که مرزهای دولت با مرزهای یک ملت (گروه قومی-فرهنگی) منطبق است. \\
\hline

سهمیه‌بندی & Quota System & تخصیص حداقلی از مناصب یا منابع به گروه‌های خاص برای تضمین نمایندگی. \\
\hline

صلاحیت انحصاری & Exclusive Competence & اختیاراتی که منحصراً به یک سطح حکومت (فدرال یا محلی) تعلق دارد. \\
\hline

صلاحیت مشترک & Concurrent/Shared Competence & اختیاراتی که بین سطوح مختلف حکومت تقسیم می‌شود. \\
\hline

فدرالیسم & Federalism & نظام حکومتی که قدرت بین حکومت مرکزی و واحدهای منطقه‌ای تقسیم می‌شود. \\
\hline

فدرالیسم مالی & Fiscal Federalism & توزیع منابع مالی و اختیارات مالیاتی بین سطوح مختلف حکومت. \\
\hline

فدرالیسم نامتقارن & Asymmetric Federalism & نظام فدرالی که واحدهای مختلف از درجات متفاوت خودمختاری برخوردارند. \\
\hline

فدرالیسم همبسته & Solidary Federalism & مدل فدرالی که بر همبستگی ملی و توازن بین وحدت و تنوع تأکید دارد. \\
\hline

قوم & Ethnic Group / Ethnicity & گروهی از مردم که ویژگی‌های مشترک فرهنگی، زبانی، تاریخی یا نیاکانی دارند. \\
\hline

قوم‌گرایی & Ethno-nationalism & جنبش سیاسی که خواهان خودمختاری یا استقلال برای یک گروه قومی است. \\
\hline

کنفدراسیون & Confederation & اتحادیه‌ای سست از واحدهای مستقل که برخی اختیارات را به نهاد مشترک واگذار می‌کنند. \\
\hline

مجلس اقوام & Chamber of Peoples/Regions & مجلس دوم پارلمان که نماینده مناطق یا گروه‌های قومی است. \\
\hline

مجلس مؤسسان & Constituent Assembly & نهاد منتخب موقت برای تدوین قانون اساسی جدید. \\
\hline

ملت & Nation & جمعیتی که هویت مشترک (فرهنگی، تاریخی یا سیاسی) دارند و معمولاً خواهان خودمختاری سیاسی هستند. \\
\hline

ملی‌گرایی & Nationalism & ایدئولوژی که بر هویت، منافع و وحدت یک ملت تأکید دارد. \\
\hline

ملی‌گرایی مدنی & Civic Nationalism & ملی‌گرایی مبتنی بر شهروندی و ارزش‌های مشترک (نه قومیت). \\
\hline

ملی‌گرایی قومی & Ethnic Nationalism & ملی‌گرایی مبتنی بر هویت قومی، نژادی یا خونی. \\
\hline

میثاق ملی & National Pact & توافق بنیادین بین گروه‌های مختلف یک جامعه برای زندگی مشترک. \\
\hline

هویت & Identity & درک فرد یا گروه از خود و تعلق به جمع بزرگ‌تر. \\
\hline

وحدت در کثرت & Unity in Diversity & اصلی که وحدت ملی را با احترام به تنوع فرهنگی ترکیب می‌کند. \\
\hline

\end{longtable}

% ═══════════════════════════════════════════════════════════════════════════════
\section{بخش سوم: مفاهیم عدالت انتقالی}
\label{sec:glossary-tj}
% ═══════════════════════════════════════════════════════════════════════════════

\begin{longtable}{|>{\columncolor{bleulight!10}}p{3.5cm}|p{3cm}|p{7.5cm}|}
\hline
\rowcolor{bleulight!30}
\textbf{\rl{اصطلاح فارسی}} & \textbf{\rl{معادل انگلیسی}} & \textbf{\rl{تعریف}} \\
\hline
\endfirsthead

آشتی ملی & National Reconciliation & فرآیند بازسازی روابط اجتماعی پس از دوره خشونت یا سرکوب. \\
\hline

احیای حیثیت & Rehabilitation & بازگرداندن حیثیت و شأن اجتماعی قربانیان نقض حقوق بشر. \\
\hline

اعاده وضع & Restitution & بازگرداندن قربانی به وضعیت قبل از نقض حقوق (مانند بازگشت اموال مصادره‌شده). \\
\hline

بررسی صلاحیت & Vetting / Lustration & فرآیند بررسی سوابق کارکنان دولتی برای شناسایی عاملان نقض حقوق بشر. \\
\hline

جبران خسارت & Reparation & اقدامات برای جبران آسیب‌های وارده به قربانیان، شامل غرامت، اعاده، احیا و تضمین عدم تکرار. \\
\hline

جرایم علیه بشریت & Crimes Against Humanity & جرایم سنگین و گسترده علیه غیرنظامیان شامل قتل، شکنجه، تجاوز سیستماتیک و غیره. \\
\hline

حافظه تاریخی & Historical Memory & یادآوری و بازنمایی جمعی گذشته، به‌ویژه دوره‌های خشونت و بی‌عدالتی. \\
\hline

حقیقت‌یابی & Truth-Seeking & فرآیند کشف و مستندسازی حقایق درباره نقض حقوق بشر در گذشته. \\
\hline

دادگاه بین‌المللی کیفری & International Criminal Court (ICC) & نهاد قضایی بین‌المللی برای محاکمه عاملان جنایات جنگی، نسل‌کشی و جرایم علیه بشریت. \\
\hline

شکنجه & Torture & هرگونه عمل عمدی که درد یا رنج شدید جسمی یا روحی به فرد وارد کند. \\
\hline

عدالت انتقالی & Transitional Justice & مجموعه اقدامات قضایی و غیرقضایی برای مواجهه با میراث نقض گسترده حقوق بشر. \\
\hline

عفو & Amnesty & بخشش جرایم گذشته، معمولاً به شرط افشای حقیقت یا همکاری. \\
\hline

عفو مشروط & Conditional Amnesty & عفوی که به شرط افشای کامل حقیقت یا سایر شرایط اعطا می‌شود. \\
\hline

غرامت & Compensation & پرداخت مالی به قربانیان برای جبران خسارت. \\
\hline

قربانی & Victim & فرد یا گروهی که از نقض حقوق بشر آسیب دیده است. \\
\hline

کمیسیون حقیقت & Truth Commission & نهاد موقت رسمی برای تحقیق درباره نقض حقوق بشر در گذشته و گزارش به عموم. \\
\hline

محاکمه & Prosecution & تعقیب قضایی عاملان نقض حقوق بشر در دادگاه‌های ملی یا بین‌المللی. \\
\hline

مصادره به مطلوب & Expropriation & تصاحب غیرقانونی اموال توسط دولت یا عوامل آن. \\
\hline

معافیت از مجازات & Impunity & وضعیتی که عاملان جرایم سنگین از مجازات فرار می‌کنند. \\
\hline

ناپدیدسازی اجباری & Enforced Disappearance & بازداشت افراد توسط عوامل دولتی بدون اعتراف و بدون امکان پیگیری سرنوشت آنها. \\
\hline

نسل‌کشی & Genocide & اقدامات عمدی برای نابودی کامل یا بخشی یک گروه ملی، قومی، نژادی یا مذهبی. \\
\hline

نسل‌کشی فرهنگی & Cultural Genocide & اقدامات برای نابودی هویت فرهنگی یک گروه بدون کشتار فیزیکی. \\
\hline

یادبود & Memorialization & اقدامات نمادین برای بزرگداشت قربانیان و یادآوری گذشته (موزه، بنای یادبود، روز ملی). \\
\hline

\end{longtable}

% ═══════════════════════════════════════════════════════════════════════════════
\section{بخش چهارم: مفاهیم حقوقی و قضایی}
\label{sec:glossary-legal}
% ═══════════════════════════════════════════════════════════════════════════════

\begin{longtable}{|>{\columncolor{golddark!10}}p{3.5cm}|p{3cm}|p{7.5cm}|}
\hline
\rowcolor{golddark!30}
\textbf{\rl{اصطلاح فارسی}} & \textbf{\rl{معادل انگلیسی}} & \textbf{\rl{تعریف}} \\
\hline
\endfirsthead

استقلال قضایی & Judicial Independence & اصل جدایی قوه قضائیه از قوای دیگر و مصونیت قضات از فشار سیاسی. \\
\hline

اصل برائت & Presumption of Innocence & فرض بی‌گناهی متهم تا زمان اثبات جرم در دادگاه. \\
\hline

اصل قانونی بودن جرم & Principle of Legality & هیچ عملی جرم نیست مگر به موجب قانون موجود در زمان ارتکاب. \\
\hline

بازنگری قضایی & Judicial Review & اختیار دادگاه برای بررسی انطباق قوانین و اقدامات دولت با قانون اساسی. \\
\hline

حبیس کورپوس & Habeas Corpus & حق بازداشت‌شده برای حضور نزد قاضی و بررسی قانونی بودن بازداشت. \\
\hline

حق دادرسی منصفانه & Right to Fair Trial & حق هر متهم به محاکمه عادلانه، علنی و توسط دادگاه مستقل. \\
\hline

حقوق بشر & Human Rights & حقوق بنیادین و غیرقابل سلب که هر انسان به صرف انسان بودن دارد. \\
\hline

حقوق طبیعی & Natural Rights & حقوقی که ذاتی و پیشینی نسبت به قوانین موضوعه است. \\
\hline

حقوق مدنی & Civil Rights & حقوق مربوط به آزادی‌های فردی و حمایت در برابر تبعیض دولتی. \\
\hline

دادخواهی & Petition & حق شهروندان برای درخواست از مقامات دولتی و دریافت پاسخ. \\
\hline

دادگاه قانون اساسی & Constitutional Court & دادگاه تخصصی برای بررسی انطباق قوانین با قانون اساسی. \\
\hline

دادرسی اداری & Administrative Justice & نظام رسیدگی به شکایات شهروندان از تصمیمات نهادهای دولتی. \\
\hline

دیوان عدالت اداری & Administrative Court & دادگاه تخصصی برای رسیدگی به شکایات از دستگاه‌های دولتی. \\
\hline

رویه قضایی & Jurisprudence / Case Law & مجموعه آرای دادگاه‌ها که به عنوان منبع حقوق به کار می‌رود. \\
\hline

سلسله‌مراتب هنجاری & Hierarchy of Norms & ترتیب برتری قواعد حقوقی (قانون اساسی > قانون عادی > آیین‌نامه). \\
\hline

صلاحیت جهانی & Universal Jurisdiction & اختیار دادگاه ملی برای محاکمه جرایم بین‌المللی صرف‌نظر از محل وقوع. \\
\hline

عدم عطف به ماسبق & Non-Retroactivity & اصل اینکه قوانین کیفری جدید به جرایم گذشته تسری نمی‌یابد. \\
\hline

قانون اساسی & Constitution & سند حقوقی بنیادین که ساختار حکومت و حقوق شهروندان را تعیین می‌کند. \\
\hline

مرور زمان & Statute of Limitations & مهلت قانونی برای تعقیب جرم یا اقامه دعوا. \\
\hline

منشور حقوق & Bill of Rights & بخشی از قانون اساسی که حقوق بنیادین شهروندان را فهرست می‌کند. \\
\hline

\end{longtable}

% ═══════════════════════════════════════════════════════════════════════════════
\section{بخش پنجم: مفاهیم اقتصادی}
\label{sec:glossary-economic}
% ═══════════════════════════════════════════════════════════════════════════════

\begin{longtable}{|>{\columncolor{orange!10}}p{3.5cm}|p{3cm}|p{7.5cm}|}
\hline
\rowcolor{orange!30}
\textbf{اصطلاح فارسی} & \textbf{معادل انگلیسی} & \textbf{تعریف} \\
\hline
\endfirsthead

اصلاحات ساختاری & Structural Reforms & تغییرات بنیادین در نظام اقتصادی برای بهبود کارایی و رقابت‌پذیری. \\
\hline

اقتصاد دانش‌بنیان & Knowledge-Based Economy & اقتصادی که محور آن تولید، توزیع و کاربرد دانش و اطلاعات است. \\
\hline

تحریم اقتصادی & Economic Sanctions & محدودیت‌های تجاری و مالی که کشورها علیه یکدیگر اعمال می‌کنند. \\
\hline

تنوع اقتصادی & Economic Diversification & کاهش وابستگی به یک بخش (مانند نفت) و توسعه بخش‌های متنوع. \\
\hline

تورم & Inflation & افزایش مستمر سطح عمومی قیمت‌ها و کاهش قدرت خرید پول. \\
\hline

توسعه پایدار & Sustainable Development & توسعه‌ای که نیازهای نسل حاضر را بدون به خطر انداختن نسل‌های آینده برآورده می‌کند. \\
\hline

خصوصی‌سازی & Privatization & انتقال مالکیت یا مدیریت بنگاه‌های دولتی به بخش خصوصی. \\
\hline

درآمد سرانه & Per Capita Income & میانگین درآمد هر فرد، معمولاً GDP تقسیم بر جمعیت. \\
\hline

رانت‌جویی & Rent-Seeking & تلاش برای کسب ثروت از طریق دستکاری سیاست‌ها به جای فعالیت مولد. \\
\hline

سرمایه‌گذاری مستقیم خارجی & Foreign Direct Investment (FDI) & سرمایه‌گذاری شرکت‌های خارجی در کشور میزبان با هدف کنترل یا نفوذ قابل توجه. \\
\hline

شاخص توسعه انسانی & Human Development Index (HDI) & شاخص ترکیبی که امید به زندگی، آموزش و درآمد را اندازه‌گیری می‌کند. \\
\hline

شاخص ژینی & Gini Coefficient & معیار نابرابری درآمد، از ۰ (برابری کامل) تا ۱ (نابرابری کامل). \\
\hline

صندوق بین‌المللی پول & International Monetary Fund (IMF) & نهاد بین‌المللی برای ثبات مالی جهانی و کمک به کشورهای دارای بحران. \\
\hline

فرار مغزها & Brain Drain & مهاجرت نیروی انسانی ماهر و تحصیل‌کرده از کشور. \\
\hline

فساد & Corruption & سوءاستفاده از قدرت عمومی برای منافع شخصی. \\
\hline

نفرین منابع & Resource Curse & پدیده‌ای که کشورهای غنی از منابع طبیعی اغلب رشد اقتصادی و دموکراسی ضعیف‌تری دارند. \\
\hline

نولیبرالیسم & Neoliberalism & رویکرد اقتصادی که بر بازار آزاد، کاهش دخالت دولت و خصوصی‌سازی تأکید دارد. \\
\hline

یارانه & Subsidy & کمک مالی دولت به تولیدکنندگان یا مصرف‌کنندگان برای کاهش قیمت. \\
\hline

\end{longtable}

% ═══════════════════════════════════════════════════════════════════════════════
\section{بخش ششم: مفاهیم محیط زیست و آب}
\label{sec:glossary-environment}
% ═══════════════════════════════════════════════════════════════════════════════

\begin{longtable}{|>{\columncolor{teal!10}}p{3.5cm}|p{3cm}|p{7.5cm}|}
\hline
\rowcolor{teal!30}
\textbf{اصطلاح فارسی} & \textbf{معادل انگلیسی} & \textbf{تعریف} \\
\hline
\endfirsthead

آب مجازی & Virtual Water & مقدار آب مصرف‌شده در تولید یک کالا، از کشاورزی تا مصرف نهایی. \\
\hline

آب‌های زیرزمینی & Groundwater & آب‌های ذخیره‌شده در لایه‌های زمین که از چاه استخراج می‌شود. \\
\hline

آبخوان & Aquifer & لایه‌ای از سنگ یا رسوب که آب زیرزمینی را در خود نگه می‌دارد. \\
\hline

اقتصاد سبز & Green Economy & اقتصادی که بر کاهش ریسک‌های محیطی و کمبود منابع تمرکز دارد. \\
\hline

انتقال انرژی & Energy Transition & حرکت از سوخت‌های فسیلی به منابع تجدیدپذیر انرژی. \\
\hline

انرژی تجدیدپذیر & Renewable Energy & انرژی از منابعی که به‌طور طبیعی بازسازی می‌شوند (خورشید، باد، آب). \\
\hline

بحران آب & Water Crisis & وضعیتی که منابع آب موجود برای تأمین نیازهای جمعیت کافی نیست. \\
\hline

بیابان‌زایی & Desertification & تخریب زمین در مناطق خشک که به گسترش بیابان منجر می‌شود. \\
\hline

بیلان آبی & Water Balance & تفاوت بین عرضه و تقاضای آب در یک منطقه. \\
\hline

تغییر اقلیم & Climate Change & تغییرات بلندمدت در الگوهای آب‌وهوایی، عمدتاً ناشی از فعالیت‌های بشری. \\
\hline

توسعه پایدار & Sustainable Development & توسعه‌ای که نیازهای حال را بدون به خطر انداختن توانایی نسل‌های آینده برآورده می‌کند. \\
\hline

حوضه آبریز & Watershed / River Basin & منطقه‌ای که آب‌های سطحی آن به یک رودخانه یا دریاچه می‌ریزد. \\
\hline

خشکسالی & Drought & دوره طولانی کمبود بارش که به کمبود آب منجر می‌شود. \\
\hline

ردپای آب & Water Footprint & کل آب مصرف‌شده توسط یک فرد، جامعه یا فرآیند تولیدی. \\
\hline

شیرین‌سازی & Desalination & فرآیند حذف نمک از آب دریا برای تولید آب شیرین. \\
\hline

فرونشست زمین & Land Subsidence & پایین رفتن سطح زمین به دلیل برداشت بیش از حد آب زیرزمینی. \\
\hline

کم‌آبی & Water Scarcity & وضعیتی که سرانه آب تجدیدپذیر کمتر از ۱۰۰۰ متر مکعب در سال باشد. \\
\hline

مدیریت یکپارچه منابع آب & Integrated Water Resources Management (IWRM) & رویکرد هماهنگ به مدیریت آب، زمین و منابع مرتبط. \\
\hline

\end{longtable}

% ═══════════════════════════════════════════════════════════════════════════════
\section{بخش هفتم: مفاهیم بین‌المللی}
\label{sec:glossary-international}
% ═══════════════════════════════════════════════════════════════════════════════

\begin{longtable}{|>{\columncolor{WarningRed!10}}p{3.5cm}|p{3cm}|p{7.5cm}|}
\hline
\rowcolor{WarningRed!30}
\textbf{اصطلاح فارسی} & \textbf{معادل انگلیسی} & \textbf{تعریف} \\
\hline
\endfirsthead

چندجانبه‌گرایی & Multilateralism & رویکرد همکاری بین چند کشور از طریق نهادهای بین‌المللی. \\
\hline

حقوق بین‌الملل & International Law & مجموعه قواعد حاکم بر روابط بین دولت‌ها و بازیگران بین‌المللی. \\
\hline

دیپلماسی & Diplomacy & هنر و علم مدیریت روابط بین کشورها از طریق مذاکره و گفتگو. \\
\hline

سازمان ملل متحد & United Nations (UN) & سازمان بین‌المللی متشکل از ۱۹۳ کشور برای حفظ صلح و همکاری بین‌المللی. \\
\hline

شورای امنیت & Security Council & ارگان اصلی سازمان ملل مسئول حفظ صلح و امنیت بین‌المللی. \\
\hline

عدم مداخله & Non-Intervention & اصل حقوق بین‌الملل مبنی بر منع دخالت در امور داخلی کشورها. \\
\hline

کنوانسیون & Convention & معاهده چندجانبه بین‌المللی درباره موضوع خاص. \\
\hline

مداخله بشردوستانه & Humanitarian Intervention & دخالت نظامی در کشور دیگر برای جلوگیری از نقض گسترده حقوق بشر. \\
\hline

منشور ملل متحد & UN Charter & سند تأسیس سازمان ملل که اصول حقوق بین‌الملل را تعیین می‌کند. \\
\hline

میثاق بین‌المللی & International Covenant & معاهده الزام‌آور حقوق بشری (مانند میثاق حقوق مدنی و سیاسی). \\
\hline

همگرایی منطقه‌ای & Regional Integration & فرآیند نزدیکی اقتصادی و سیاسی کشورهای یک منطقه. \\
\hline

\end{longtable}

% ═══════════════════════════════════════════════════════════════════════════════
\section{بخش هشتم: اختصارات}
\label{sec:glossary-abbreviations}
% ═══════════════════════════════════════════════════════════════════════════════

\begin{center}
\begin{small}
\begin{longtable}{|>{\columncolor{gray!10}}p{2cm}|p{6cm}|p{6cm}|}
\hline
\rowcolor{gray!30}
\textbf{اختصار} & \textbf{عبارت انگلیسی} & \textbf{معادل فارسی} \\
\hline
\endfirsthead

AU & African Union & اتحادیه آفریقا \\
\hline
CEDAW & Convention on Elimination of Discrimination Against Women & کنوانسیون رفع تبعیض علیه زنان \\
\hline
CSO & Civil Society Organization & سازمان جامعه مدنی \\
\hline
EIU & Economist Intelligence Unit & واحد اطلاعات اکونومیست \\
\hline
EU & European Union & اتحادیه اروپا \\
\hline
FATF & Financial Action Task Force & گروه ویژه اقدام مالی \\
\hline
FDI & Foreign Direct Investment & سرمایه‌گذاری مستقیم خارجی \\
\hline
GDP & Gross Domestic Product & تولید ناخالص داخلی \\
\hline
HDI & Human Development Index & شاخص توسعه انسانی \\
\hline
IAEA & International Atomic Energy Agency & آژانس بین‌المللی انرژی اتمی \\
\hline
ICC & International Criminal Court & دیوان بین‌المللی کیفری \\
\hline
ICCPR & International Covenant on Civil and Political Rights & میثاق بین‌المللی حقوق مدنی و سیاسی \\
\hline
ICESCR & International Covenant on Economic, Social and Cultural Rights & میثاق بین‌المللی حقوق اقتصادی، اجتماعی و فرهنگی \\
\hline
ICJ & International Court of Justice & دیوان بین‌المللی دادگستری \\
\hline
IEA & International Energy Agency & آژانس بین‌المللی انرژی \\
\hline
IMF & International Monetary Fund & صندوق بین‌المللی پول \\
\hline
IPU & Inter-Parliamentary Union & اتحادیه بین‌المجالس \\
\hline
IWRM & Integrated Water Resources Management & مدیریت یکپارچه منابع آب \\
\hline
KPI & Key Performance Indicator & شاخص کلیدی عملکرد \\
\hline
NGO & Non-Governmental Organization & سازمان غیردولتی \\
\hline
OECD & Organisation for Economic Co-operation and Development & سازمان همکاری و توسعه اقتصادی \\
\hline
RSF & Reporters Sans Frontières (Reporters Without Borders) & گزارشگران بدون مرز \\
\hline
SDGs & Sustainable Development Goals & اهداف توسعه پایدار \\
\hline
TI & Transparency International & شفافیت بین‌الملل \\
\hline
TJ & Transitional Justice & عدالت انتقالی \\
\hline
TRC & Truth and Reconciliation Commission & کمیسیون حقیقت و آشتی \\
\hline
UDHR & Universal Declaration of Human Rights & اعلامیه جهانی حقوق بشر \\
\hline
UN & United Nations & سازمان ملل متحد \\
\hline
UNDP & United Nations Development Programme & برنامه توسعه سازمان ملل \\
\hline
UNHCR & United Nations High Commissioner for Refugees & کمیساریای عالی پناهندگان سازمان ملل \\
\hline
WB & World Bank & بانک جهانی \\
\hline
WHO & World Health Organization & سازمان بهداشت جهانی \\
\hline
WJP & World Justice Project & پروژه عدالت جهانی \\
\hline

\end{longtable}
\end{small}
\end{center}

% ═══════════════════════════════════════════════════════════════════════════════
\section{نمایه واژگان فارسی-انگلیسی}
\label{sec:glossary-index}
% ═══════════════════════════════════════════════════════════════════════════════

\begin{center}
\begin{multicols}{2}
\begin{small}
\textbf{آ}\\
آب مجازی — Virtual Water\\
آشتی ملی — National Reconciliation\\

\textbf{ا}\\
استبداد — Autocracy\\
استقلال قضایی — Judicial Independence\\
اصل برائت — Presumption of Innocence\\
اقتدارگرایی — Authoritarianism\\
اقتصاد دانش‌بنیان — Knowledge-Based Economy\\
اقلیت — Minority\\

\textbf{ب}\\
بازنگری قضایی — Judicial Review\\
بحران آب — Water Crisis\\

\textbf{پ}\\
پارلمانتاریسم — Parliamentarism\\
پاسخگویی — Accountability\\
پلورالیسم — Pluralism\\

\textbf{ت}\\
تبعیض مثبت — Affirmative Action\\
تثبیت دموکراتیک — Democratic Consolidation\\
تحریم اقتصادی — Economic Sanctions\\
تفکیک قوا — Separation of Powers\\
تمرکززدایی — Decentralization\\
تورم — Inflation\\
توسعه پایدار — Sustainable Development\\

\textbf{ج}\\
جامعه مدنی — Civil Society\\
جبران خسارت — Reparation\\
جرایم علیه بشریت — Crimes Against Humanity\\
جمهوری — Republic\\

\textbf{ح}\\
حاکمیت قانون — Rule of Law\\
حاکمیت ملی — National Sovereignty\\
حق تعیین سرنوشت — Self-Determination\\
حقوق بشر — Human Rights\\
حقیقت‌یابی — Truth-Seeking\\

\textbf{خ}\\
خودمختاری — Autonomy\\
خصوصی‌سازی — Privatization\\

\textbf{د}\\
دموکراسی — Democracy\\
دموکراسی اجماعی — Consociational Democracy\\
دیپلماسی — Diplomacy\\

\textbf{ر}\\
رأی همگانی — Universal Suffrage\\
رانت‌جویی — Rent-Seeking\\

\textbf{س}\\
سکولاریسم — Secularism\\
سرمایه‌گذاری خارجی — FDI\\

\textbf{ش}\\
شفافیت — Transparency\\
شکنجه — Torture\\
شیرین‌سازی — Desalination\\

\textbf{ع}\\
عدالت انتقالی — Transitional Justice\\
عفو — Amnesty\\

\textbf{ف}\\
فدرالیسم — Federalism\\
فساد — Corruption\\

\textbf{ق}\\
قانون اساسی — Constitution\\
قوم — Ethnic Group\\

\textbf{ک}\\
کثرت‌گرایی — Pluralism\\
کمیسیون حقیقت — Truth Commission\\

\textbf{گ}\\
گذار دموکراتیک — Democratic Transition\\

\textbf{م}\\
مجلس اقوام — Chamber of Peoples\\
مجلس مؤسسان — Constituent Assembly\\
مشارکت سیاسی — Political Participation\\
ملت — Nation\\
ملی‌گرایی — Nationalism\\
میثاق ملی — National Pact\\

\textbf{ن}\\
نسل‌کشی — Genocide\\
نظارت و موازنه — Checks and Balances\\

\textbf{و}\\
وحدت در کثرت — Unity in Diversity\\

\textbf{ه}\\
همه‌پرسی — Referendum\\
هویت — Identity\\

\end{small}
\end{multicols}
\end{center}

% ═══════════════════════════════════════════════════════════════════════════════
% پایان پیوست ۵
% ═══════════════════════════════════════════════════════════════════════════════
	% ═══════════════════════════════════════════════════════════════════════════════
% پیوست ۶: کتاب‌شناسی
% فایل: app06-bibliography.tex
% ═══════════════════════════════════════════════════════════════════════════════

\chapter{کتاب‌شناسی}
\label{app:bibliography}

\begin{kholasebox}
این کتاب‌شناسی مجموعه‌ای جامع از منابع علمی و پژوهشی است که در تدوین این کتاب مورد استفاده قرار گرفته یا برای مطالعه بیشتر توصیه می‌شود. منابع به ترتیب موضوعی و سپس الفبایی (بر اساس نام خانوادگی نویسنده) مرتب شده‌اند. منابع فارسی در بخش جداگانه آمده است. برای سهولت دسترسی، شناسه دیجیتال (DOI) یا آدرس اینترنتی منابع در صورت وجود ذکر شده است.
\end{kholasebox}

% ═══════════════════════════════════════════════════════════════════════════════
\section{بخش اول: گذار دموکراتیک و تثبیت}
\label{sec:bib-transition}
% ═══════════════════════════════════════════════════════════════════════════════

\subsection{کتاب‌های بنیادین}

\begin{enumerate}[label={[\arabic*]}]

\item \textbf{Acemoglu, Daron \& Robinson, James A.} (2006). \textit{Economic Origins of Dictatorship and Democracy}. Cambridge University Press.
\begin{quote}
\small تحلیل اقتصاد سیاسی گذار دموکراتیک؛ نظریه‌پردازی درباره شرایط ظهور و تثبیت دموکراسی.
\end{quote}

\item \textbf{Carothers, Thomas} (2002). "The End of the Transition Paradigm." \textit{Journal of Democracy}, 13(1), 5-21.
\begin{quote}
\small نقد پارادایم سنتی گذار و ارائه دیدگاه‌های جایگزین.
\end{quote}

\item \textbf{Diamond, Larry} (1999). \textit{Developing Democracy: Toward Consolidation}. Johns Hopkins University Press.
\begin{quote}
\small بررسی جامع فرآیند تثبیت دموکراتیک و عوامل مؤثر بر آن.
\end{quote}

\item \textbf{Diamond, Larry \& Plattner, Marc F.} (eds.) (2009). \textit{Democracy: A Reader}. Johns Hopkins University Press.
\begin{quote}
\small مجموعه مقالات کلاسیک درباره نظریه و عمل دموکراسی.
\end{quote}

\item \textbf{Fukuyama, Francis} (2014). \textit{Political Order and Political Decay}. Farrar, Straus and Giroux.
\begin{quote}
\small تحلیل تاریخی-تطبیقی نظم سیاسی و زوال نهادها.
\end{quote}

\item \textbf{Huntington, Samuel P.} (1991). \textit{The Third Wave: Democratization in the Late Twentieth Century}. University of Oklahoma Press.
\begin{quote}
\small تحلیل موج سوم دموکراتیزاسیون و الگوهای گذار در دهه‌های ۱۹۷۰-۱۹۹۰.
\end{quote}

\item \textbf{Levitsky, Steven \& Way, Lucan A.} (2010). \textit{Competitive Authoritarianism: Hybrid Regimes After the Cold War}. Cambridge University Press.
\begin{quote}
\small تحلیل رژیم‌های هیبریدی و اقتدارگرایی رقابتی.
\end{quote}

\item \textbf{Linz, Juan J. \& Stepan, Alfred} (1996). \textit{Problems of Democratic Transition and Consolidation: Southern Europe, South America, and Post-Communist Europe}. Johns Hopkins University Press.
\begin{quote}
\small کتاب مرجع در مطالعات گذار؛ تحلیل تطبیقی گذارهای موفق و ناموفق.
\end{quote}

\item \textbf{O'Donnell, Guillermo \& Schmitter, Philippe C.} (1986). \textit{Transitions from Authoritarian Rule: Tentative Conclusions about Uncertain Democracies}. Johns Hopkins University Press.
\begin{quote}
\small اثر کلاسیک در نظریه گذار؛ مفاهیم کلیدی مانند «نرم‌شدگان» و «سخت‌گیران».
\end{quote}

\item \textbf{Przeworski, Adam} (1991). \textit{Democracy and the Market: Political and Economic Reforms in Eastern Europe and Latin America}. Cambridge University Press.
\begin{quote}
\small تحلیل همزمانی اصلاحات سیاسی و اقتصادی در دوره گذار.
\end{quote}

\item \textbf{Schedler, Andreas} (ed.) (2006). \textit{Electoral Authoritarianism: The Dynamics of Unfree Competition}. Lynne Rienner.
\begin{quote}
\small بررسی انتخابات در نظام‌های اقتدارگرا و کارکرد آنها.
\end{quote}

\end{enumerate}

\subsection{مطالعات موردی گذار}

\begin{enumerate}[label={[\arabic*]}, resume]

\item \textbf{Encarnación, Omar G.} (2008). \textit{Spanish Politics: Democracy After Dictatorship}. Polity Press.
\begin{quote}
\small تحلیل گذار اسپانیا از دیکتاتوری فرانکو به دموکراسی.
\end{quote}

\item \textbf{Ekiert, Grzegorz \& Hanson, Stephen E.} (eds.) (2003). \textit{Capitalism and Democracy in Central and Eastern Europe}. Cambridge University Press.
\begin{quote}
\small گذار دوگانه (سیاسی و اقتصادی) در اروپای شرقی.
\end{quote}

\item \textbf{Gunther, Richard; Diamandouros, P. Nikiforos \& Puhle, Hans-Jürgen} (eds.) (1995). \textit{The Politics of Democratic Consolidation: Southern Europe in Comparative Perspective}. Johns Hopkins University Press.
\begin{quote}
\small تثبیت دموکراسی در یونان، پرتغال و اسپانیا.
\end{quote}

\item \textbf{Haggard, Stephan \& Kaufman, Robert R.} (2016). \textit{Dictators and Democrats: Masses, Elites, and Regime Change}. Princeton University Press.
\begin{quote}
\small نقش توده‌ها و نخبگان در گذار دموکراتیک.
\end{quote}

\item \textbf{Mainwaring, Scott \& Pérez-Liñán, Aníbal} (2013). \textit{Democracies and Dictatorships in Latin America}. Cambridge University Press.
\begin{quote}
\small تحلیل تطبیقی دموکراسی و دیکتاتوری در آمریکای لاتین.
\end{quote}

\end{enumerate}

% ═══════════════════════════════════════════════════════════════════════════════
\section{بخش دوم: فدرالیسم و مدیریت تنوع}
\label{sec:bib-federalism}
% ═══════════════════════════════════════════════════════════════════════════════

\begin{enumerate}[label={[\arabic*]}, resume]

\item \textbf{Amoretti, Ugo M. \& Bermeo, Nancy} (eds.) (2004). \textit{Federalism and Territorial Cleavages}. Johns Hopkins University Press.
\begin{quote}
\small فدرالیسم به عنوان ابزار مدیریت شکاف‌های سرزمینی و قومی.
\end{quote}

\item \textbf{Anderson, George} (2008). \textit{Federalism: An Introduction}. Oxford University Press.
\begin{quote}
\small مقدمه‌ای جامع بر نظریه و عمل فدرالیسم.
\end{quote}

\item \textbf{Burgess, Michael} (2006). \textit{Comparative Federalism: Theory and Practice}. Routledge.
\begin{quote}
\small تحلیل تطبیقی نظام‌های فدرال در جهان.
\end{quote}

\item \textbf{Elazar, Daniel J.} (1987). \textit{Exploring Federalism}. University of Alabama Press.
\begin{quote}
\small نظریه‌پردازی درباره فدرالیسم و انواع آن.
\end{quote}

\item \textbf{Gagnon, Alain-G. \& Tully, James} (eds.) (2001). \textit{Multinational Democracies}. Cambridge University Press.
\begin{quote}
\small دموکراسی در جوامع چندملیتی و مدیریت تنوع.
\end{quote}

\item \textbf{Horowitz, Donald L.} (1985). \textit{Ethnic Groups in Conflict}. University of California Press.
\begin{quote}
\small کتاب مرجع در مطالعات تعارض قومی؛ ریشه‌ها و راه‌حل‌ها.
\end{quote}

\item \textbf{Horowitz, Donald L.} (2014). "Ethnic Power Sharing: Three Big Problems." \textit{Journal of Democracy}, 25(2), 5-20.
\begin{quote}
\small نقد مدل‌های تقسیم قدرت قومی.
\end{quote}

\item \textbf{Kymlicka, Will} (1995). \textit{Multicultural Citizenship: A Liberal Theory of Minority Rights}. Oxford University Press.
\begin{quote}
\small نظریه لیبرال حقوق اقلیت‌ها و شهروندی چندفرهنگی.
\end{quote}

\item \textbf{Kymlicka, Will \& Norman, Wayne} (eds.) (2000). \textit{Citizenship in Diverse Societies}. Oxford University Press.
\begin{quote}
\small شهروندی در جوامع متنوع و چالش‌های آن.
\end{quote}

\item \textbf{Lijphart, Arend} (1977). \textit{Democracy in Plural Societies: A Comparative Exploration}. Yale University Press.
\begin{quote}
\small نظریه دموکراسی اجماعی برای جوامع چندپاره.
\end{quote}

\item \textbf{Lijphart, Arend} (2012). \textit{Patterns of Democracy: Government Forms and Performance in Thirty-Six Countries} (2nd ed.). Yale University Press.
\begin{quote}
\small مقایسه الگوهای دموکراسی اکثریتی و اجماعی.
\end{quote}

\item \textbf{McGarry, John \& O'Leary, Brendan} (eds.) (1993). \textit{The Politics of Ethnic Conflict Regulation}. Routledge.
\begin{quote}
\small روش‌های مدیریت و تنظیم تعارضات قومی.
\end{quote}

\item \textbf{Requejo, Ferran \& Nagel, Klaus-Jürgen} (eds.) (2011). \textit{Federalism beyond Federations: Asymmetry and Processes of Resymmetrisation in Europe}. Ashgate.
\begin{quote}
\small فدرالیسم نامتقارن در تجربه اروپایی.
\end{quote}

\item \textbf{Stepan, Alfred} (1999). "Federalism and Democracy: Beyond the U.S. Model." \textit{Journal of Democracy}, 10(4), 19-34.
\begin{quote}
\small نقد الگوی آمریکایی و معرفی مدل‌های جایگزین فدرالیسم.
\end{quote}

\item \textbf{Watts, Ronald L.} (2008). \textit{Comparing Federal Systems} (3rd ed.). McGill-Queen's University Press.
\begin{quote}
\small مقایسه نظام‌های فدرال جهان؛ مرجع استاندارد.
\end{quote}

\end{enumerate}

% ═══════════════════════════════════════════════════════════════════════════════
\section{بخش سوم: عدالت انتقالی}
\label{sec:bib-tj}
% ═══════════════════════════════════════════════════════════════════════════════

\begin{enumerate}[label={[\arabic*]}, resume]

\item \textbf{Arthur, Paige} (2009). "How 'Transitions' Reshaped Human Rights: A Conceptual History of Transitional Justice." \textit{Human Rights Quarterly}, 31(2), 321-367.
\begin{quote}
\small تاریخ مفهومی عدالت انتقالی.
\end{quote}

\item \textbf{De Greiff, Pablo} (ed.) (2006). \textit{The Handbook of Reparations}. Oxford University Press.
\begin{quote}
\small راهنمای جامع جبران خسارت در عدالت انتقالی.
\end{quote}

\item \textbf{Elster, Jon} (2004). \textit{Closing the Books: Transitional Justice in Historical Perspective}. Cambridge University Press.
\begin{quote}
\small تحلیل تاریخی عدالت انتقالی از یونان باستان تا امروز.
\end{quote}

\item \textbf{Hayner, Priscilla B.} (2011). \textit{Unspeakable Truths: Transitional Justice and the Challenge of Truth Commissions} (2nd ed.). Routledge.
\begin{quote}
\small بررسی جامع کمیسیون‌های حقیقت در جهان.
\end{quote}

\item \textbf{Kritz, Neil J.} (ed.) (1995). \textit{Transitional Justice: How Emerging Democracies Reckon with Former Regimes} (3 vols.). United States Institute of Peace Press.
\begin{quote}
\small مجموعه سه‌جلدی مرجع در عدالت انتقالی.
\end{quote}

\item \textbf{Minow, Martha} (1998). \textit{Between Vengeance and Forgiveness: Facing History After Genocide and Mass Violence}. Beacon Press.
\begin{quote}
\small تعادل بین عدالت و آشتی پس از خشونت‌های گسترده.
\end{quote}

\item \textbf{Roht-Arriaza, Naomi \& Mariezcurrena, Javier} (eds.) (2006). \textit{Transitional Justice in the Twenty-First Century: Beyond Truth versus Justice}. Cambridge University Press.
\begin{quote}
\small تحولات عدالت انتقالی در قرن بیست‌ویکم.
\end{quote}

\item \textbf{Teitel, Ruti G.} (2000). \textit{Transitional Justice}. Oxford University Press.
\begin{quote}
\small نظریه‌پردازی حقوقی درباره عدالت انتقالی.
\end{quote}

\item \textbf{Truth and Reconciliation Commission of South Africa} (1998). \textit{Truth and Reconciliation Commission of South Africa Report} (5 vols.). Cape Town.
\begin{quote}
\small گزارش کامل کمیسیون حقیقت و آشتی آفریقای جنوبی.
\end{quote}

\item \textbf{Tutu, Desmond} (1999). \textit{No Future Without Forgiveness}. Doubleday.
\begin{quote}
\small تجربه شخصی رهبری کمیسیون حقیقت آفریقای جنوبی.
\end{quote}

\end{enumerate}

% ═══════════════════════════════════════════════════════════════════════════════
\section{بخش چهارم: حقوق بشر و حقوق اقلیت‌ها}
\label{sec:bib-rights}
% ═══════════════════════════════════════════════════════════════════════════════

\begin{enumerate}[label={[\arabic*]}, resume]

\item \textbf{Donnelly, Jack} (2013). \textit{Universal Human Rights in Theory and Practice} (3rd ed.). Cornell University Press.
\begin{quote}
\small نظریه و عمل حقوق بشر جهانی.
\end{quote}

\item \textbf{Freeman, Michael} (2011). \textit{Human Rights: An Interdisciplinary Approach} (2nd ed.). Polity Press.
\begin{quote}
\small رویکرد میان‌رشته‌ای به حقوق بشر.
\end{quote}

\item \textbf{Henrard, Kristin} (2000). \textit{Devising an Adequate System of Minority Protection}. Martinus Nijhoff.
\begin{quote}
\small طراحی نظام مناسب حمایت از اقلیت‌ها.
\end{quote}

\item \textbf{Pentassuglia, Gaetano} (2002). \textit{Minorities in International Law}. Council of Europe Publishing.
\begin{quote}
\small حقوق اقلیت‌ها در حقوق بین‌الملل.
\end{quote}

\item \textbf{Steiner, Henry J.; Alston, Philip \& Goodman, Ryan} (2008). \textit{International Human Rights in Context} (3rd ed.). Oxford University Press.
\begin{quote}
\small کتاب درسی مرجع در حقوق بین‌الملل بشر.
\end{quote}

\item \textbf{Thornberry, Patrick} (1991). \textit{International Law and the Rights of Minorities}. Oxford University Press.
\begin{quote}
\small حقوق بین‌الملل اقلیت‌ها.
\end{quote}

\end{enumerate}

% ═══════════════════════════════════════════════════════════════════════════════
\section{بخش پنجم: اقتصاد سیاسی و توسعه}
\label{sec:bib-economy}
% ═══════════════════════════════════════════════════════════════════════════════

\begin{enumerate}[label={[\arabic*]}, resume]

\item \textbf{Acemoglu, Daron \& Robinson, James A.} (2012). \textit{Why Nations Fail: The Origins of Power, Prosperity, and Poverty}. Crown Business.
\begin{quote}
\small نقش نهادها در توسعه و شکست ملت‌ها.
\end{quote}

\item \textbf{Easterly, William} (2006). \textit{The White Man's Burden: Why the West's Efforts to Aid the Rest Have Done So Much Ill and So Little Good}. Penguin.
\begin{quote}
\small نقد کمک‌های توسعه‌ای غربی.
\end{quote}

\item \textbf{Haber, Stephen \& Menaldo, Victor} (2011). "Do Natural Resources Fuel Authoritarianism? A Reappraisal of the Resource Curse." \textit{American Political Science Review}, 105(1), 1-26.
\begin{quote}
\small بازارزیابی نظریه نفرین منابع.
\end{quote}

\item \textbf{North, Douglass C.} (1990). \textit{Institutions, Institutional Change and Economic Performance}. Cambridge University Press.
\begin{quote}
\small نظریه نهادی توسعه اقتصادی.
\end{quote}

\item \textbf{North, Douglass C.; Wallis, John Joseph \& Weingast, Barry R.} (2009). \textit{Violence and Social Orders: A Conceptual Framework for Interpreting Recorded Human History}. Cambridge University Press.
\begin{quote}
\small چارچوب نظری برای فهم نظم اجتماعی و توسعه.
\end{quote}

\item \textbf{Rodrik, Dani} (2007). \textit{One Economics, Many Recipes: Globalization, Institutions, and Economic Growth}. Princeton University Press.
\begin{quote}
\small تنوع مسیرهای توسعه اقتصادی.
\end{quote}

\item \textbf{Ross, Michael L.} (2012). \textit{The Oil Curse: How Petroleum Wealth Shapes the Development of Nations}. Princeton University Press.
\begin{quote}
\small تأثیر ثروت نفتی بر توسعه سیاسی و اقتصادی.
\end{quote}

\item \textbf{Sachs, Jeffrey D. \& Warner, Andrew M.} (1995). "Natural Resource Abundance and Economic Growth." NBER Working Paper No. 5398.
\begin{quote}
\small مقاله کلاسیک درباره نفرین منابع.
\end{quote}

\item \textbf{Sen, Amartya} (1999). \textit{Development as Freedom}. Knopf.
\begin{quote}
\small توسعه به مثابه آزادی؛ رویکرد قابلیت‌محور.
\end{quote}

\end{enumerate}

% ═══════════════════════════════════════════════════════════════════════════════
\section{بخش ششم: بحران آب و محیط زیست}
\label{sec:bib-water}
% ═══════════════════════════════════════════════════════════════════════════════

\begin{enumerate}[label={[\arabic*]}, resume]

\item \textbf{Allan, Tony} (2011). \textit{Virtual Water: Tackling the Threat to Our Planet's Most Precious Resource}. I.B. Tauris.
\begin{quote}
\small مفهوم آب مجازی و کاربردهای آن.
\end{quote}

\item \textbf{Gleick, Peter H.} (ed.) (2014). \textit{The World's Water Volume 8: The Biennial Report on Freshwater Resources}. Island Press.
\begin{quote}
\small گزارش دوسالانه وضعیت منابع آب جهان.
\end{quote}

\item \textbf{Madani, Kaveh} (2014). "Water Management in Iran: What is Causing the Looming Crisis?" \textit{Journal of Environmental Studies and Sciences}, 4(4), 315-328.
\begin{quote}
\small تحلیل بحران آب ایران از دیدگاه مدیریتی.
\end{quote}

\item \textbf{Postel, Sandra} (1999). \textit{Pillar of Sand: Can the Irrigation Miracle Last?}. W.W. Norton.
\begin{quote}
\small پایداری کشاورزی آبیاری و چالش‌های آینده.
\end{quote}

\item \textbf{Rijsberman, Frank R.} (2006). "Water Scarcity: Fact or Fiction?" \textit{Agricultural Water Management}, 80(1-3), 5-22.
\begin{quote}
\small تحلیل کمبود آب جهانی.
\end{quote}

\item \textbf{Wolf, Aaron T.} (2007). "Shared Waters: Conflict and Cooperation." \textit{Annual Review of Environment and Resources}, 32, 241-269.
\begin{quote}
\small همکاری و تعارض بر سر منابع آب مشترک.
\end{quote}

\item \textbf{World Bank} (2017). \textit{Beyond Scarcity: Water Security in the Middle East and North Africa}. Washington, DC.
\begin{quote}
\small گزارش بانک جهانی درباره امنیت آبی خاورمیانه.
\end{quote}

\end{enumerate}

% ═══════════════════════════════════════════════════════════════════════════════
\section{بخش هفتم: ایران‌شناسی و مطالعات منطقه‌ای}
\label{sec:bib-iran}
% ═══════════════════════════════════════════════════════════════════════════════

\begin{enumerate}[label={[\arabic*]}, resume]

\item \textbf{Abrahamian, Ervand} (2008). \textit{A History of Modern Iran}. Cambridge University Press.
\begin{quote}
\small تاریخ جامع ایران مدرن از قاجار تا جمهوری اسلامی.
\end{quote}

\item \textbf{Ansari, Ali M.} (2003). \textit{Modern Iran Since 1921: The Pahlavis and After}. Longman.
\begin{quote}
\small تاریخ سیاسی ایران از رضاشاه تا انقلاب.
\end{quote}

\item \textbf{Axworthy, Michael} (2013). \textit{Revolutionary Iran: A History of the Islamic Republic}. Oxford University Press.
\begin{quote}
\small تاریخ جمهوری اسلامی ایران.
\end{quote}

\item \textbf{Elling, Rasmus Christian} (2013). \textit{Minorities in Iran: Nationalism and Ethnicity after Khomeini}. Palgrave Macmillan.
\begin{quote}
\small بررسی جامع اقلیت‌های قومی ایران پس از انقلاب.
\end{quote}

\item \textbf{Katouzian, Homa} (2009). \textit{The Persians: Ancient, Mediaeval and Modern Iran}. Yale University Press.
\begin{quote}
\small تاریخ فرهنگی-سیاسی ایران از باستان تا امروز.
\end{quote}

\item \textbf{Keddie, Nikki R.} (2006). \textit{Modern Iran: Roots and Results of Revolution} (Updated ed.). Yale University Press.
\begin{quote}
\small ریشه‌ها و پیامدهای انقلاب ایران.
\end{quote}

\item \textbf{Keshavarzian, Arang} (2007). \textit{Bazaar and State in Iran: The Politics of the Tehran Marketplace}. Cambridge University Press.
\begin{quote}
\small نقش بازار در سیاست ایران.
\end{quote}

\item \textbf{Milani, Abbas} (2011). \textit{The Shah}. Palgrave Macmillan.
\begin{quote}
\small زندگینامه سیاسی محمدرضا پهلوی.
\end{quote}

\item \textbf{Sadjadpour, Karim} (2009). \textit{Reading Khamenei: The World View of Iran's Most Powerful Leader}. Carnegie Endowment.
\begin{quote}
\small تحلیل جهان‌بینی علی خامنه‌ای.
\end{quote}

\item \textbf{Yapp, Malcolm} (1996). \textit{The Near East Since the First World War} (2nd ed.). Longman.
\begin{quote}
\small تاریخ خاورمیانه پس از جنگ جهانی اول.
\end{quote}

\end{enumerate}

% ═══════════════════════════════════════════════════════════════════════════════
\section{بخش هشتم: قانون اساسی و نظام‌های حقوقی}
\label{sec:bib-constitution}
% ═══════════════════════════════════════════════════════════════════════════════

\begin{enumerate}[label={[\arabic*]}, resume]

\item \textbf{Elkins, Zachary; Ginsburg, Tom \& Melton, James} (2009). \textit{The Endurance of National Constitutions}. Cambridge University Press.
\begin{quote}
\small چرا برخی قوانین اساسی ماندگارترند.
\end{quote}

\item \textbf{Ginsburg, Tom \& Simpser, Alberto} (eds.) (2014). \textit{Constitutions in Authoritarian Regimes}. Cambridge University Press.
\begin{quote}
\small نقش قانون اساسی در نظام‌های اقتدارگرا.
\end{quote}

\item \textbf{Ginsburg, Tom \& Huq, Aziz Z.} (2016). \textit{How to Save a Constitutional Democracy}. University of Chicago Press.
\begin{quote}
\small حفاظت از دموکراسی قانون اساسی در برابر تهدیدات.
\end{quote}

\item \textbf{Rosenfeld, Michel \& Sajó, András} (eds.) (2012). \textit{The Oxford Handbook of Comparative Constitutional Law}. Oxford University Press.
\begin{quote}
\small دستنامه جامع حقوق اساسی تطبیقی.
\end{quote}

\item \textbf{Tushnet, Mark} (2008). \textit{Weak Courts, Strong Rights: Judicial Review and Social Welfare Rights in Comparative Constitutional Law}. Princeton University Press.
\begin{quote}
\small نظارت قضایی و حقوق اجتماعی-اقتصادی.
\end{quote}

\end{enumerate}

% ═══════════════════════════════════════════════════════════════════════════════
\section{بخش نهم: منابع فارسی}
\label{sec:bib-persian}
% ═══════════════════════════════════════════════════════════════════════════════

\begin{enumerate}[label={[\arabic*]}, resume]

\item \textbf{آبراهامیان، یرواند} (۱۳۸۹). \textit{ایران بین دو انقلاب}. ترجمه احمد گل‌محمدی و محمدابراهیم فتاحی. تهران: نشر نی.
\begin{quote}
\small تاریخ اجتماعی-سیاسی ایران از مشروطه تا انقلاب ۵۷.
\end{quote}

\item \textbf{آجودانی، ماشاءالله} (۱۳۸۲). \textit{مشروطه ایرانی}. تهران: اختران.
\begin{quote}
\small تحلیل انقلاب مشروطه و زمینه‌های فکری آن.
\end{quote}

\item \textbf{بشیریه، حسین} (۱۳۸۰). \textit{درس‌های دموکراسی برای همه}. تهران: نشر نگاه معاصر.
\begin{quote}
\small مقدمه‌ای بر نظریه دموکراسی.
\end{quote}

\item \textbf{بشیریه، حسین} (۱۳۸۱). \textit{دیباچه‌ای بر جامعه‌شناسی سیاسی ایران}. تهران: نشر نگاه معاصر.
\begin{quote}
\small تحلیل جامعه‌شناختی سیاست در ایران.
\end{quote}

\item \textbf{زیباکلام، صادق} (۱۳۷۷). \textit{سنت و مدرنیته}. تهران: روزنه.
\begin{quote}
\small ریشه‌یابی ناکامی مدرنیزاسیون در ایران.
\end{quote}

\item \textbf{طباطبایی، سید جواد} (۱۳۸۵). \textit{نظریه حکومت قانون در ایران}. تهران: ستوده.
\begin{quote}
\small تاریخ اندیشه سیاسی و حکومت قانون در ایران.
\end{quote}

\item \textbf{کاتوزیان، محمدعلی همایون} (۱۳۷۹). \textit{دولت و جامعه در ایران}. ترجمه حسن افشار. تهران: نشر مرکز.
\begin{quote}
\small تحلیل رابطه دولت و جامعه در تاریخ ایران.
\end{quote}

\item \textbf{کدی، نیکی آر.} (۱۳۸۳). \textit{ریشه‌های انقلاب ایران}. ترجمه عبدالرحیم گواهی. تهران: قلم.
\begin{quote}
\small ریشه‌های اجتماعی و سیاسی انقلاب ۵۷.
\end{quote}

\item \textbf{مدنی، جلال‌الدین} (۱۳۶۹). \textit{حقوق اساسی و نهادهای سیاسی}. تهران: همراه.
\begin{quote}
\small کتاب درسی حقوق اساسی.
\end{quote}

\item \textbf{میرسپاسی، علی} (۱۳۸۴). \textit{روشنفکران ایران}. ترجمه عباس مخبر. تهران: نشر توسعه.
\begin{quote}
\small تحلیل روشنفکری ایرانی در قرن بیستم.
\end{quote}

\end{enumerate}

% ═══════════════════════════════════════════════════════════════════════════════
\section{بخش دهم: اسناد و گزارش‌های بین‌المللی}
\label{sec:bib-reports}
% ═══════════════════════════════════════════════════════════════════════════════

\subsection{اسناد سازمان ملل}

\begin{enumerate}[label={[\arabic*]}, resume]

\item \textbf{United Nations} (1948). \textit{Universal Declaration of Human Rights}. UN General Assembly Resolution 217A.
\begin{quote}
\small اعلامیه جهانی حقوق بشر.
\end{quote}

\item \textbf{United Nations} (1966). \textit{International Covenant on Civil and Political Rights}. UN Treaty Series, Vol. 999.
\begin{quote}
\small میثاق بین‌المللی حقوق مدنی و سیاسی.
\end{quote}

\item \textbf{United Nations} (1966). \textit{International Covenant on Economic, Social and Cultural Rights}. UN Treaty Series, Vol. 993.
\begin{quote}
\small میثاق بین‌المللی حقوق اقتصادی، اجتماعی و فرهنگی.
\end{quote}

\item \textbf{United Nations} (1992). \textit{Declaration on the Rights of Persons Belonging to National or Ethnic, Religious and Linguistic Minorities}. UN General Assembly Resolution 47/135.
\begin{quote}
\small اعلامیه حقوق اقلیت‌ها.
\end{quote}

\item \textbf{United Nations} (2007). \textit{United Nations Declaration on the Rights of Indigenous Peoples}. UN General Assembly Resolution 61/295.
\begin{quote}
\small اعلامیه حقوق مردمان بومی.
\end{quote}

\end{enumerate}

\subsection{گزارش‌های شاخص}

\begin{enumerate}[label={[\arabic*]}, resume]

\item \textbf{Economist Intelligence Unit} (Annual). \textit{Democracy Index}. London: EIU.
\begin{quote}
\small شاخص سالانه دموکراسی جهان.
\end{quote}

\item \textbf{Freedom House} (Annual). \textit{Freedom in the World}. Washington, DC: Freedom House.
\begin{quote}
\small گزارش سالانه آزادی در جهان.
\end{quote}

\item \textbf{Reporters Without Borders} (Annual). \textit{World Press Freedom Index}. Paris: RSF.
\begin{quote}
\small شاخص سالانه آزادی مطبوعات.
\end{quote}

\item \textbf{Transparency International} (Annual). \textit{Corruption Perceptions Index}. Berlin: TI.
\begin{quote}
\small شاخص سالانه درک فساد.
\end{quote}

\item \textbf{United Nations Development Programme} (Annual). \textit{Human Development Report}. New York: UNDP.
\begin{quote}
\small گزارش سالانه توسعه انسانی.
\end{quote}

\item \textbf{World Bank} (Annual). \textit{World Development Indicators}. Washington, DC: World Bank.
\begin{quote}
\small شاخص‌های توسعه جهانی.
\end{quote}

\item \textbf{World Justice Project} (Annual). \textit{Rule of Law Index}. Washington, DC: WJP.
\begin{quote}
\small شاخص سالانه حاکمیت قانون.
\end{quote}

\end{enumerate}

% ═══════════════════════════════════════════════════════════════════════════════
\section{بخش یازدهم: منابع آنلاین و پایگاه‌های داده}
\label{sec:bib-online}
% ═══════════════════════════════════════════════════════════════════════════════

\begin{center}
\begin{small}
\begin{longtable}{|>{\columncolor{bleurepublique!10}}p{4cm}|p{5cm}|p{5cm}|}
\hline
\rowcolor{bleurepublique!30}
\textbf{\rl{نام منبع}} & \textbf{\rl{آدرس}} & \textbf{\rl{محتوا}} \\
\hline
\endfirsthead

Constitute Project & constituteproject.org & متن قوانین اساسی جهان \\
\hline

International IDEA & idea.int & دموکراسی و انتخابات \\
\hline

V-Dem Institute & v-dem.net & شاخص‌های دموکراسی \\
\hline

Polity Project & systemicpeace.org/polity & داده‌های رژیم سیاسی \\
\hline

ICTJ & ictj.org & مرکز بین‌المللی عدالت انتقالی \\
\hline

Forum of Federations & forumfed.org & فدرالیسم تطبیقی \\
\hline

European Commission for Democracy through Law (Venice Commission) & venice.coe.int & استانداردهای قانون اساسی \\
\hline

OHCHR & ohchr.org & دفتر حقوق بشر سازمان ملل \\
\hline

World Bank Data & data.worldbank.org & داده‌های توسعه \\
\hline

IMF Data & imf.org/data & داده‌های اقتصادی \\
\hline

FAO AQUASTAT & fao.org/aquastat & داده‌های آب جهان \\
\hline

Iran Data Portal & irandataportal.syr.edu & داده‌های ایران \\
\hline

\end{longtable}
\end{small}
\end{center}

% ═══════════════════════════════════════════════════════════════════════════════
\section{راهنمای مطالعه بیشتر}
\label{sec:bib-guide}
% ═══════════════════════════════════════════════════════════════════════════════

\begin{olgoobox}
\textbf{توصیه برای مطالعه بیشتر بر اساس موضوع}

\textbf{برای درک نظریه گذار دموکراتیک:}
\begin{itemize}[nosep]
    \item با O'Donnell \& Schmitter (1986) شروع کنید
    \item سپس Linz \& Stepan (1996) را بخوانید
    \item برای دیدگاه انتقادی: Carothers (2002)
\end{itemize}

\textbf{برای فدرالیسم و مدیریت تنوع:}
\begin{itemize}[nosep]
    \item Watts (2008) به عنوان مرجع اصلی
    \item Lijphart (1977) برای دموکراسی اجماعی
    \item Horowitz (1985) برای تعارض قومی
\end{itemize}

\textbf{برای عدالت انتقالی:}
\begin{itemize}[nosep]
    \item Hayner (2011) برای کمیسیون‌های حقیقت
    \item Teitel (2000) برای چارچوب نظری
    \item گزارش TRC آفریقای جنوبی برای نمونه عملی
\end{itemize}

\textbf{برای تاریخ ایران:}
\begin{itemize}[nosep]
    \item Abrahamian (2008) برای دید کلی
    \item Keddie (2006) برای انقلاب
    \item Elling (2013) برای اقلیت‌های قومی
\end{itemize}
\end{olgoobox}

% ═══════════════════════════════════════════════════════════════════════════════
\section{درباره استناددهی}
% ═══════════════════════════════════════════════════════════════════════════════

\begin{kholasebox}
\textbf{شیوه استناددهی در این کتاب}

این کتاب از شیوه استناددهی \textbf{APA (ویرایش هفتم)} با تطبیق برای زبان فارسی استفاده کرده است.

\textbf{برای کتاب:}\\
نام خانوادگی، نام. (سال). \textit{عنوان کتاب}. ناشر.

\textbf{برای مقاله:}\\
نام خانوادگی، نام. (سال). عنوان مقاله. \textit{نام نشریه}، دوره(شماره)، صفحات.

\textbf{برای منبع آنلاین:}\\
نام خانوادگی، نام. (سال). عنوان. آدرس وب

\textbf{تذکر:} برای منابع فارسی، نام نویسنده به فارسی و برای منابع انگلیسی به انگلیسی درج شده است.
\end{kholasebox}

% ═══════════════════════════════════════════════════════════════════════════════
% پایان پیوست ۶ و پایان کتاب
% ═══════════════════════════════════════════════════════════════════════════════

\vspace{20pt}

\begin{center}
\begin{tikzpicture}
    \node[
        draw=bleurepublique,
        line width=3pt,
        fill=bleurepublique!10,
        rounded corners=15pt,
        inner sep=20pt,
        text width=12cm,
        align=center
    ] {
        {\Huge ★}\\[10pt]
        {\Large\textbf{\rl{پایان کتاب}}}\\[10pt]
        {\large \rl{از بحران تا بالندگی}}\\[5pt]
        {\normalsize \rl{طرح جامع تأسیس و تثبیت دموکراسی پایدار}}\\[5pt]
        {\normalsize \rl{در جامعه‌ای با میراث تمدنی و تنوع قومی-فرهنگی}}\\[15pt]
        {\small \rl{مهدی سالم | ریچموندهیل | ۱۴۰۴}}
    };
\end{tikzpicture}
\end{center}
	\chapter{بررسی تطبیقی قوانین اساسی و مدل‌های گذار}
\label{app:comparative}

این پیوست به بررسی و مقایسه قوانین اساسی کشورهایی می‌پردازد که در دهه‌های اخیر گذار به دموکراسی را تجربه کرده‌اند. هدف از این مقایسه، درک بهتر ظرافت‌های طراحی نهادی در دوران گذار و ارائه مبنای تحلیلی برای انتخاب‌های صورت گرفته در این پیش‌نویس است.

\section{ماتریس تطبیقی ابعاد حکمرانی دموکراتیک}

در این بخش، مدل‌های مختلف گذار بر اساس شاخص‌های کلیدی حقوقی، اقتصادی و نهادی مقایسه می‌شوند.

\begin{landscape}
\begin{small}
\begin{longtable}{|>{\columncolor{bleulight}}p{2.5cm}|p{3cm}|p{3cm}|p{3cm}|p{3cm}|p{3cm}|}
\hline
\rowcolor{bleurepublique!20}
\headmark بعد مقایسه & \headmark مدل آلمان (ثبات فدرال) & \headmark مدل تونس (توافق سکولار) & \headmark مدل آفریقای جنوبی (آشتی ملی) & \headmark مدل هند (تنوع قومی) & \headmark ایران (مدل پیشنهادی) \\
\hline
\endfirsthead
\hline
\rowcolor{bleurepublique!20}
\headmark بعد مقایسه & \headmark مدل آلمان & \headmark مدل تونس & \headmark مدل آفریقای جنوبی & \headmark مدل هند & \headmark ایران (پیشنهادی) \\
\hline
\endhead

\textbf{حقوق بنیادین} & منشور بر حقوق فردی و منع استبداد متمرکز است. & بر آزادی وجدان و برابری جنسیتی تأکید دارد. & شامل حقوق وسیع اجتماعی-اقتصادی و مسکن است. & شامل حقوق حمایتی برای اقلیت‌های زبانی و دینی. & جامع (منشور حقوق + لغو اعدام + حق آب). \\
\hline

\rowcolor{goldlight}
\textbf{منابع طبیعی} & مالکیت عمومی و استانی بر اساس قوانین فدرال. & نظارت ملی بر قراردادهای انرژی. & مالکیت ملی با هدف توزیع عادلانه و رفع آپارتاید. & مالکیت دولتی (فدرال) بر منابع استراتژیک. & مالکیت ملی فدرال با توزیع عادلانه منطقه‌ای. \\
\hline

\textbf{مدل اقتصادی} & اقتصاد بازار اجتماعی (Social Market). & اقتصاد مختلط با گرایش به حمایت‌های اجتماعی. & تمرکز بر بازتوزیع ثروت و کاهش شکاف طبقاتی. & اقتصاد دولتی-خصوصی با برنامه‌ریزی مرکزی. & اقتصاد آزاد دموکراتیک با چتر حمایتی سبز. \\
\hline

\rowcolor{goldlight}
\textbf{توزیع قدرت} & فدرالیسم کامل (ایالت‌های قوی). & نظام نیمه‌ریاستی با تمرکززدایی اداری. & نظام واحد با استان‌های دارای اختیار اداری. & فدرالیسم نامتقارن (برخی ایالت‌ها اختیارات بیشتر). & فدرالیسم همبسته (مناطق خودمختار قومی). \\
\hline

\textbf{عدالت انتقالی} & پاکسازی (Lustration) و جبران خسارت. & تمرکز بر حقیقت‌یابی و آشتی ملی (IVD). & کمیسیون حقیقت و آشتی (TRC). & رویه‌های قضایی برای حل اختلافات تاریخی. & کمیسیون حقیقت و بازمعماری نهادی. \\
\hline

\rowcolor{goldlight}
\textbf{تضمین شفافیت} & دیوان محاسبات و نظارت پارلمانی قوی. & نهادهای نظارتی مستقل از دولت. & نهادهای صیانت‌گر دموکراسی (Chapter 9). & کمیسیون انتخابات و دیوان فرادست مستقل. & رکن چهارم (نهادهای مستقل نظارتی ۸گانه). \\
\hline
\end{longtable}
\end{small}
\end{landscape}

\section{تحلیل درس‌آموخته‌ها برای گذار ایران}

با بررسی تجارب جهانی، سه درس کلیدی برای ایران شناسایی شده است:

\begin{enumerate}
    \item \textbf{برگشت‌ناپذیری دموکراسی (مدل آلمان)}: ایجاد نهادهایی که تغییر نظام را حتی با اکثریت پارلمانی سخت می‌کنند (اصول غیرقابل تغییر).
    \item \textbf{مدیریت تکثر (مدل هند و اسپانیا)}: پذیرش هویت‌های قومی در ساختار حقوقی به جای انکار آن‌ها، قدرتمندترین ابراز علیه تجزیه‌طلبی است.
    \item \textbf{پاسخگویی به نیازهای معیشتی (مدل آفریقای جنوبی)}: دموکراسی سیاسی بدون دموکراسی اقتصادی (دسترسی به آب، مسکن و کار) در جوامع در حال گذار بسیار شکننده خواهد بود.
\end{enumerate}

	
	%══════════════════════════════════════════════════════════════════════════════
	%                              صفحات پایانی
	%══════════════════════════════════════════════════════════════════════════════
	\backmatter
	
	%──────────────────────────────────────────────────────────────────────────────
	% درباره نویسنده
	%──────────────────────────────────────────────────────────────────────────────
	\chapter*{درباره نویسنده}
	\addcontentsline{toc}{chapter}{درباره نویسنده}
	
	\textbf{مهدی سالم} پژوهشگر حوزه توسعه سیاسی و حکمرانی دموکراتیک است. حوزه‌های پژوهشی او شامل گذارهای دموکراتیک، مدیریت تنوع قومی-فرهنگی، و توسعه پایدار در کشورهای در حال گذار می‌شود.
	
	\vspace{1cm}
	
	تماس: \\
	ریچموندهیل، کانادا
	mahhdy@live.com 
	
\end{document}