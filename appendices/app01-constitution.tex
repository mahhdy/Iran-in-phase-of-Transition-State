% ═══════════════════════════════════════════════════════════════════════════════
% پیوست ۱: متن قانون اساسی پیشنهادی جمهوری فدرال ایران
% فایل: app01-constitution.tex
% ═══════════════════════════════════════════════════════════════════════════════

\chapter{متن قانون اساسی پیشنهادی}
\label{app:constitution}

\begin{kholasebox}
این پیوست متن کامل پیش‌نویس قانون اساسی جمهوری فدرال ایران را ارائه می‌دهد. این سند حقوقی بنیادین، حاصل تلفیق تجارب موفق جهانی با واقعیت‌های تاریخی-فرهنگی ایران است. ساختار این قانون اساسی بر سه ستون استوار است: \textbf{دموکراسی مشارکتی}، \textbf{فدرالیسم همبسته}، و \textbf{حقوق بنیادین تضمین‌شده}. متن شامل ۱۸۷ اصل در هفت فصل به همراه مقررات انتقالی است.
\end{kholasebox}

% ═══════════════════════════════════════════════════════════════════════════════
\section{دیباچه قانون اساسی}
% ═══════════════════════════════════════════════════════════════════════════════

\begin{center}
\begin{tikzpicture}
    \node[
        draw=bleurepublique,
        line width=3pt,
        fill=bleulight,
        rounded corners=15pt,
        inner sep=20pt,
        text width=12cm,
        align=center
    ] (preamble) {
        {\Huge\rl{\textbf{قانون اساسی}}}\\[0.4cm]
        {\LARGE\rl{\textbf{جمهوری فدرال ایران}}}\\[0.6cm]
        {\large \rl{مصوب مجلس مؤسسان | سال ۱۴۰۵ هجری شمسی}}
    };
    \node[above=0.2cm of preamble, bleurepublique] {\Large \rl{★ ★ ★}};
\end{tikzpicture}
\end{center}

\vspace{10pt}

\begin{naghlbox}
\textbf{دیباچه}

ما، ملت ایران،

وارثان تمدنی کهن با پیشینه‌ای درخشان در تاریخ بشری؛

مردمانی متکثر از تبارها، زبان‌ها، فرهنگ‌ها و باورهای گوناگون که در طول هزاره‌ها در این سرزمین در کنار یکدیگر زیسته‌ایم؛

با یادآوری رنج‌های مشترک از استبداد، تبعیض و بی‌عدالتی؛

با درس‌آموزی از تجربه‌های تلخ و شیرین تاریخ معاصر؛

با ایمان به کرامت ذاتی انسان و برابری همه شهروندان؛

با تعهد به آزادی، عدالت، دموکراسی و حاکمیت قانون؛

با احترام به تنوع فرهنگی به‌مثابه ثروت ملی؛

با مسئولیت در برابر نسل‌های آینده و محیط زیست؛

با امید به ساختن آینده‌ای بهتر برای همه فرزندان این سرزمین؛

\textbf{این قانون اساسی را به‌عنوان میثاق ملی خود تصویب و اعلام می‌کنیم.}
\sourceline{مجلس مؤسسان جمهوری فدرال ایران}
\end{naghlbox}

% ═══════════════════════════════════════════════════════════════════════════════
\section{فصل اول: اصول کلی و هویت ملی}
\label{sec:const-ch1}
% ═══════════════════════════════════════════════════════════════════════════════

\subsection{بخش یکم: ماهیت و شکل حکومت}

\begin{center}
\begin{tikzpicture}[
    scale=0.9,
    transform shape,
    box/.style={
        rectangle,
        rounded corners=8pt,
        draw=bleurepublique,
        fill=bleulight,
        line width=1.5pt,
        minimum width=5.5cm,
        minimum height=1.2cm,
        text centered,
        font=\small\bfseries
    },
    content/.style={
        rectangle,
        rounded corners=8pt,
        draw=goldphoenix!50,
        fill=goldlight,
        line width=1pt,
        minimum width=5.5cm,
        minimum height=1.2cm,
        text centered,
        font=\small
    },
    arrow/.style={->, ultra thick, bleurepublique!40, >=stealth}
]
\node[box] (l1) at (-3.5, 4.5) {\rl{اصل ۱: ماهیت حکومت}};
\node[content] (r1) at (3.5, 4.5) {\rl{جمهوری فدرال دموکراتیک}};

\node[box] (l2) at (-3.5, 3.0) {\rl{اصل ۲: منشأ حاکمیت}};
\node[content] (r2) at (3.5, 3.0) {\rl{حاکمیت ملی متعلق به مردم}};

\node[box] (l3) at (-3.5, 1.5) {\rl{اصل ۳: وحدت ملی}};
\node[content] (r3) at (3.5, 1.5) {\rl{تمامیت ارضی و فدرالیسم}};

\node[box] (l4) at (-3.5, 0) {\rl{اصل ۴: تفکیک قوا}};
\node[content] (r4) at (3.5, 0) {\rl{استقلال قوای سه‌گانه}};

\node[box] (l5) at (-3.5, -1.5) {\rl{اصل ۵: حاکمیت قانون}};
\node[content] (r5) at (3.5, -1.5) {\rl{برتری قانون اساسی}};

\foreach \i in {1,...,5} {
    \draw[arrow] (l\i) -- (r\i);
}
\end{tikzpicture}
\end{center}

\begin{table}[htbp]
\centering
\caption{اصول منتخب فصل اول: اصول کلی و هویت ملی}
\label{tab:const-ch1}
\begin{tabularx}{\textwidth}{R{2.5cm} Y}
\toprule
\headmark شماره اصل & \headmark متن و محتوای اصل \\
\midrule
\textbf{اصل ۱} & \textbf{ماهیت حکومت:} ایران کشوری است با نظام جمهوری فدرال دموکراتیک که بر پایه حاکمیت ملی، حقوق بشر، و حکومت قانون بنیان نهاده شده است. \\
\rowcolor{goldlight}
\textbf{اصل ۲} & \textbf{منشأ حاکمیت:} حاکمیت ملی به‌طور انحصاری متعلق به مردم ایران است. هیچ قدرتی برتر از خواست دموکراتیک مردم نیست. \\
\textbf{اصل ۳} & \textbf{وحدت ملی:} تمامیت ارضی و همبستگی ملی از طریق ساختار فدرال و احترام به تنوع تضمین می‌شود. \\
\rowcolor{goldlight}
\textbf{اصل ۶} & \textbf{جدایی دین از دولت:} دولت نسبت به همه باورها بی‌طرف است. هیچ دینی، دین رسمی نیست. \\
\textbf{اصل ۷} & \textbf{زبان‌ها:} فارسی زبان رسمی و مشترک؛ سایر زبان‌ها ملی و در مناطق خود رسمی هستند. \\
\rowcolor{goldlight}
\textbf{اصل ۱۱} & \textbf{اصول غیرقابل تغییر:} دموکراسی، فدرالیسم و لایی‌سیته ستون‌های ابدی نظام هستند. \\
\bottomrule
\end{tabularx}
\end{table}

\subsection{بخش دوم: اصول غیرقابل تغییر}

\begin{enghelabbox}
\textbf{⚠️ اصول غیرقابل تغییر (اصل ۱۱)}

اصول زیر غیرقابل تغییر هستند و هیچ بازنگری در قانون اساسی نمی‌تواند آنها را نقض کند:

\begin{enumerate}[nosep, label=\alph*)]
    \item ماهیت جمهوری و دموکراتیک نظام
    \item حاکمیت ملی و انتخابی بودن حکومت
    \item حقوق بنیادین شهروندان مندرج در فصل دوم
    \item تفکیک و استقلال قوای سه‌گانه
    \item جدایی دین از دولت
    \item حقوق اقوام و ساختار فدرال
    \item استقلال و تمامیت ارضی کشور
\end{enumerate}
\end{enghelabbox}

% ═══════════════════════════════════════════════════════════════════════════════
\section{فصل دوم: حقوق بنیادین شهروندان}
\label{sec:const-ch2}
% ═══════════════════════════════════════════════════════════════════════════════

\begin{center}
\begin{tikzpicture}[
    scale=0.9,
    transform shape,
    centernode/.style={
        circle,
        draw=bleurepublique,
        fill=bleulight,
        line width=2pt,
        minimum size=3.5cm,
        text centered,
        font=\large\bfseries
    },
    leafnode/.style={
        rectangle,
        rounded corners=8pt,
        draw=goldphoenix,
        fill=goldlight,
        line width=1.2pt,
        minimum width=3cm,
        minimum height=1cm,
        text centered,
        font=\small\bfseries
    },
    connector/.style={->, ultra thick, bleurepublique!30, >=stealth}
]
\node[centernode] (center) at (0,0) {\rl{حقوق بنیادین}\\\rl{شهروندان}};

\foreach \ang/\label/\name in {90/حقوق مدنی/civil, 30/حقوق سیاسی/pol, -30/حقوق اقتصادی/econ, -90/حقوق اجتماعی/social, -150/حقوق فرهنگی/cult, 150/حقوق قضایی/jud} {
    \node[leafnode] (\name) at (\ang:4.8) {\rl{\label}};
    \draw[connector] (center) -- (\name);
}
\end{tikzpicture}
\end{center}

\subsection{بخش یکم: حقوق مدنی و آزادی‌های فردی (اصول ۱۲-۳۵)}

\begin{longtable}{|>{\columncolor{bleulight}}r|p{12cm}|}
\hline
\rowcolor{bleurepublique!20}
\headmark شماره اصل & \headmark متن کامل اصل (حقوق مدنی و آزادی‌های فردی) \\
\hline
\endfirsthead
\hline
\rowcolor{bleurepublique!20}
\headmark شماره اصل & \headmark متن کامل اصل \\
\hline
\endhead

\textbf{اصل ۱۲} &
\textbf{کرامت انسانی:} کرامت ذاتی انسان مصون از تعرض است. احترام به این کرامت و حمایت از آن وظیفه همه نهادهای حکومتی است. \\
\hline

\rowcolor{goldlight}
\textbf{اصل ۱۳} &
\textbf{برابری در برابر قانون:} همه شهروندان در برابر قانون برابرند. هرگونه تبعیض بر اساس جنسیت، قومیت، نژاد، زبان، دین، مذهب، باور سیاسی، منشأ اجتماعی، وضعیت اقتصادی، معلولیت، سن، یا هر وضعیت دیگر ممنوع است. \\
\hline

\textbf{اصل ۱۴} &
\textbf{حق حیات:} حق حیات هر انسان از تولد تا مرگ طبیعی محترم و تضمین‌شده است. مجازات اعدام ملغی است. \\
\hline

\rowcolor{goldlight}
\textbf{اصل ۱۵} &
\textbf{ممنوعیت شکنجه:} شکنجه، رفتار غیرانسانی، تحقیرآمیز یا بی‌رحمانه در هر شرایطی مطلقاً ممنوع است. هرگونه اعتراف یا مدرکی که از طریق شکنجه به دست آمده باشد فاقد اعتبار قانونی است. \\
\hline

\textbf{اصل ۱۶} &
\textbf{آزادی فردی و امنیت شخصی:}
الف) هیچ‌کس را نمی‌توان بازداشت کرد مگر به موجب حکم مقام قضایی صالح و طبق قانون.
ب) هر بازداشت‌شده باید ظرف ۲۴ ساعت به دادگاه معرفی شود.
ج) هر کس حق دارد فوراً از دلایل بازداشت خود مطلع شود و به وکیل دسترسی داشته باشد.
د) بازداشت موقت استثنایی است و مدت آن نمی‌تواند از حدود قانونی تجاوز کند. \\
\hline

\rowcolor{goldlight}
\textbf{اصل ۱۷} &
\textbf{حریم خصوصی:} حریم خصوصی افراد، مسکن، مکاتبات، ارتباطات و داده‌های شخصی مصون از تعرض است. تفتیش و نظارت تنها با حکم قضایی و در چارچوب قانون مجاز است. \\
\hline

\textbf{اصل ۱۸} &
\textbf{آزادی رفت‌وآمد:} هر شهروند حق دارد آزادانه در سراسر کشور رفت‌وآمد کند، محل سکونت خود را انتخاب نماید، از کشور خارج و به آن بازگردد. سلب یا محدودیت این حق تنها به موجب قانون و با حکم قضایی ممکن است. \\
\hline

\rowcolor{goldlight}
\textbf{اصل ۱۹} &
\textbf{آزادی عقیده و بیان:} 
الف) هر کس حق دارد آزادانه عقاید خود را داشته و ابراز کند.
ب) آزادی بیان، نوشتار، هنر و رسانه تضمین می‌شود.
ج) سانسور ممنوع است. محدودیت آزادی بیان تنها برای حمایت از حقوق دیگران، امنیت ملی یا نظم عمومی و به موجب قانون مجاز است.
د) تحریک به خشونت، نفرت‌پراکنی و انکار جنایات علیه بشریت ممنوع است. \\
\hline

\textbf{اصل ۲۰} &
\textbf{آزادی دین و وجدان:}
الف) هر کس حق دارد دین، مذهب یا باور خود را آزادانه انتخاب کند یا بدون دین باشد.
ب) هر کس حق دارد دین خود را به‌صورت فردی یا جمعی، در خلوت یا علن، عبادت و آموزش دهد.
ج) هیچ‌کس را نمی‌توان به پذیرش یا ترک دینی مجبور کرد.
د) تغییر دین حق هر فرد است و مجازات ندارد. \\
\hline

\rowcolor{goldlight}
\textbf{اصل ۲۱} &
\textbf{آزادی تجمع و تشکل:}
الف) شهروندان حق برگزاری تجمعات و راهپیمایی‌های مسالمت‌آمیز بدون سلاح را دارند.
ب) تجمعات عمومی نیازمند اطلاع‌رسانی قبلی هستند، نه مجوز.
ج) تشکیل احزاب، انجمن‌ها، سندیکاها و سازمان‌های مدنی آزاد است. \\
\hline

\textbf{اصل ۲۲} &
\textbf{حق مالکیت:}
الف) مالکیت خصوصی به رسمیت شناخته شده و حمایت می‌شود.
ب) سلب مالکیت تنها برای مصالح عمومی، با پرداخت غرامت عادلانه و به موجب قانون مجاز است.
ج) مالکیت متضمن مسئولیت اجتماعی است. \\
\hline
\end{longtable}

\subsection{بخش دوم: حقوق سیاسی (اصول ۳۶-۴۵)}

\begin{longtable}{|>{\columncolor{bleulight}}r|p{12cm}|}
\hline
\rowcolor{bleurepublique!20}
\headmark شماره اصل & \headmark متن کامل اصل (حقوق سیاسی) \\
\hline
\endfirsthead
\hline
\rowcolor{bleurepublique!20}
\headmark شماره اصل & \headmark متن کامل اصل \\
\hline
\endhead

\textbf{اصل ۳۶} &
\textbf{حق رأی:} 
الف) هر شهروند ایرانی که به سن ۱۸ سال رسیده باشد حق رأی دارد.
ب) رأی‌گیری آزاد، برابر، مستقیم و مخفی است.
ج) شرکت در انتخابات حق شهروندی است، نه تکلیف. \\
\hline

\rowcolor{goldlight}
\textbf{اصل ۳۷} &
\textbf{حق انتخاب شدن:}
الف) هر شهروندی که واجد شرایط قانونی باشد حق نامزد شدن برای مناصب انتخابی را دارد.
ب) هیچ نهادی حق رد صلاحیت بر اساس معیارهای غیرقانونی یا سیاسی را ندارد.
ج) احراز صلاحیت تنها بر اساس معیارهای عینی قانونی (سن، تابعیت، سابقه کیفری) و با حق اعتراض قضایی انجام می‌شود. \\
\hline

\textbf{اصل ۳۸} &
\textbf{آزادی احزاب:}
الف) تشکیل و فعالیت احزاب سیاسی آزاد است.
ب) احزاب باید در ساختار و عملکرد خود دموکراتیک باشند.
ج) انحلال حزب تنها به حکم دیوان عالی قانون اساسی و به دلیل نقض اصول دموکراتیک ممکن است. \\
\hline

\rowcolor{goldlight}
\textbf{اصل ۳۹} &
\textbf{حق همه‌پرسی:}
الف) شهروندان می‌توانند از طریق همه‌پرسی در تصمیمات ملی مشارکت کنند.
ب) ابتکار عمل مردمی: یک میلیون امضا می‌تواند برگزاری همه‌پرسی را الزامی کند.
ج) همه‌پرسی نمی‌تواند ناقض حقوق بنیادین یا اصول غیرقابل تغییر باشد. \\
\hline

\textbf{اصل ۴۰} &
\textbf{حق دادخواهی:} هر شهروند حق دارد از نهادهای دولتی دادخواهی کند و پاسخ مکتوب دریافت نماید. \\
\hline

\rowcolor{goldlight}
\textbf{اصل ۴۱} &
\textbf{دسترسی به اطلاعات:}
الف) هر شهروند حق دسترسی به اطلاعات و اسناد دولتی را دارد.
ب) طبقه‌بندی اسناد باید محدود، موجه و زمان‌دار باشد.
ج) افشاگری فساد (سوت‌زنی) تحت حمایت قانون است. \\
\hline

\textbf{اصل ۴۲} &
\textbf{حق اعتراض و نافرمانی مدنی:}
الف) اعتراض مسالمت‌آمیز حق شهروندی است.
ب) نافرمانی مدنی در برابر قوانین ناقض حقوق بنیادین مشروع است.
ج) حق مقاومت در برابر کودتا یا تلاش برای براندازی نظام دموکراتیک به رسمیت شناخته می‌شود. \\
\hline
\end{longtable}

\subsection{بخش سوم: حقوق اقتصادی و اجتماعی (اصول ۴۶-۶۵)}

\begin{longtable}{|>{\columncolor{bleulight}}r|p{12cm}|}
\hline
\rowcolor{bleurepublique!20}
\headmark شماره اصل & \headmark متن کامل اصل (حقوق اقتصادی، اجتماعی و محیط زیست) \\
\hline
\endfirsthead
\hline
\rowcolor{bleurepublique!20}
\headmark شماره اصل & \headmark متن کامل اصل \\
\hline
\endhead

\textbf{اصل ۴۶} &
\textbf{حق کار:} 
الف) هر شهروند حق کار، انتخاب آزادانه شغل، و شرایط کار عادلانه را دارد.
ب) کار اجباری ممنوع است.
ج) دولت موظف به تلاش برای ایجاد اشتغال کامل است. \\
\hline

\rowcolor{goldlight}
\textbf{اصل ۴۷} &
\textbf{حقوق کارگران:}
الف) حق تشکیل و عضویت در سندیکا و اتحادیه‌های کارگری تضمین می‌شود.
ب) حق اعتصاب به رسمیت شناخته می‌شود.
ج) دستمزد باید عادلانه و تضمین‌کننده زندگی با کرامت باشد.
د) حداقل دستمزد به موجب قانون تعیین و سالانه بازنگری می‌شود. \\
\hline

\textbf{اصل ۴۸} &
\textbf{حق تأمین اجتماعی:}
الف) هر شهروند حق بهره‌مندی از تأمین اجتماعی را دارد.
ب) بیمه بیکاری، بازنشستگی، ازکارافتادگی و بیمه درمانی همگانی تضمین می‌شود.
ج) حمایت ویژه از اقشار آسیب‌پذیر وظیفه دولت است. \\
\hline

\rowcolor{goldlight}
\textbf{اصل ۴۹} &
\textbf{حق بهداشت و درمان:}
الف) هر شهروند حق دسترسی به خدمات بهداشتی و درمانی را دارد.
ب) خدمات بهداشت اولیه رایگان است.
ج) بیمه درمانی پایه همگانی و اجباری است. \\
\hline

\textbf{اصل ۵۰} &
\textbf{حق مسکن:} دولت موظف است زمینه دسترسی همه شهروندان به مسکن مناسب را فراهم سازد. بی‌خانمانی باید ریشه‌کن شود. \\
\hline

\rowcolor{goldlight}
\textbf{اصل ۵۱} &
\textbf{حق آموزش:}
الف) آموزش رایگان و اجباری تا پایان دوره متوسطه است.
ب) آموزش عالی باید در دسترس همگان بر اساس شایستگی باشد.
ج) دولت موظف به حمایت از آموزش مادام‌العمر است. \\
\hline

\textbf{اصل ۵۲} &
\textbf{حق آب:}
الف) دسترسی به آب سالم و کافی حق بنیادین هر شهروند است.
ب) دولت موظف به تأمین آب شرب سالم برای همه است.
ج) منابع آب ثروت ملی و متعلق به نسل‌های حال و آینده است. \\
\hline

\rowcolor{goldlight}
\textbf{اصل ۵۳} &
\textbf{حقوق محیط زیست:}
الف) هر شهروند حق زندگی در محیط زیست سالم را دارد.
ب) حمایت از محیط زیست وظیفه دولت و حق و تکلیف همه شهروندان است.
ج) آلوده‌سازی محیط زیست جرم است. \\
\end{longtable}

\subsection{بخش چهارم: حقوق فرهنگی و اقوام (اصول ۶۶-۷۵)}

\begin{longtable}{|>{\columncolor{bleulight}}r|p{12cm}|}
\hline
\rowcolor{bleurepublique!20}
\headmark شماره اصل & \headmark متن کامل اصل (حقوق فرهنگی و اقوام) \\
\hline
\endfirsthead
\hline
\rowcolor{bleurepublique!20}
\headmark شماره اصل & \headmark متن کامل اصل \\
\hline
\endhead

\textbf{اصل ۶۶} &
\textbf{تنوع فرهنگی:} تنوع فرهنگی، قومی، زبانی و مذهبی ایران ثروت ملی است. دولت موظف به حمایت و حفاظت از این تنوع است. \\
\hline

\rowcolor{goldlight}
\textbf{اصل ۶۷} &
\textbf{حقوق زبانی:} 
الف) هر شهروند حق استفاده از زبان مادری خود در عرصه‌های عمومی و خصوصی را دارد.
ب) آموزش به زبان مادری در کنار زبان فارسی حق هر کودک است.
ج) خدمات عمومی در مناطق دوزبانه به هر دو زبان ارائه می‌شود. \\
\hline

\textbf{اصل ۶۸} &
\textbf{حقوق قومی:}
الف) هیچ قومی بر قوم دیگر برتری ندارد.
ب) هر قوم حق حفظ و توسعه هویت، فرهنگ، زبان و سنن خود را دارد.
ج) مشارکت عادلانه اقوام در قدرت سیاسی و اقتصادی تضمین می‌شود. \\
\hline

\rowcolor{goldlight}
\textbf{اصل ۶۹} &
\textbf{میراث فرهنگی:} حفاظت از میراث فرهنگی، تاریخی و باستانی ایران وظیفه دولت است. غارت، تخریب یا قاچاق آثار تاریخی جرم سنگین است. \\
\hline

\textbf{اصل ۷۰} &
\textbf{آزادی علم و هنر:} تحقیق علمی و خلاقیت هنری آزاد است. دولت موظف به حمایت از علم، فناوری و هنر است. \\
\hline
\end{longtable}

\subsection{بخش پنجم: حقوق قضایی (اصول ۷۶-۸۵)}

\begin{longtable}{|>{\columncolor{bleulight}}r|p{12cm}|}
\hline
\rowcolor{bleurepublique!20}
\headmark شماره اصل & \headmark متن کامل اصل (حقوق قضایی و دادرسی منصفانه) \\
\hline
\endfirsthead
\hline
\rowcolor{bleurepublique!20}
\headmark شماره اصل & \headmark متن کامل اصل \\
\hline
\endhead

\textbf{اصل ۷۶} &
\textbf{اصل برائت:} هر کس بی‌گناه است تا زمانی که جرم او در دادگاه صالح به اثبات برسد. بار اثبات بر عهده دادستان است. \\
\hline

\rowcolor{goldlight}
\textbf{اصل ۷۷} &
\textbf{حق محاکمه عادلانه:}
الف) هر کس حق دارد در مدت معقول توسط دادگاه مستقل و بی‌طرف محاکمه شود.
ب) محاکمات علنی است مگر در موارد استثنایی قانونی.
ج) هر کس حق دارد از اتهامات علیه خود مطلع شود و فرصت کافی برای دفاع داشته باشد. \\
\hline

\textbf{اصل ۷۸} &
\textbf{حق وکیل:}
الف) هر کس در تمام مراحل رسیدگی حق داشتن وکیل را دارد.
ب) اگر متهم توانایی مالی نداشته باشد، دولت وکیل تسخیری تعیین می‌کند.
ج) ارتباط بین وکیل و موکل محرمانه است. \\
\hline

\rowcolor{goldlight}
\textbf{اصل ۷۹} &
\textbf{اصل قانونی بودن جرم و مجازات:}
الف) هیچ عملی جرم نیست مگر به موجب قانون.
ب) هیچ مجازاتی اعمال نمی‌شود مگر به موجب قانون.
ج) قوانین کیفری عطف به ماسبق نمی‌شوند مگر به نفع متهم. \\
\hline

\textbf{اصل ۸۰} &
\textbf{ممنوعیت محاکمه مجدد:} هیچ‌کس را نمی‌توان برای جرمی که قبلاً محاکمه و تبرئه یا محکوم شده دوباره محاکمه کرد. \\
\hline

\rowcolor{goldlight}
\textbf{اصل ۸۱} &
\textbf{حق تجدیدنظر:} هر محکوم حق درخواست تجدیدنظر در حکم را دارد. \\
\hline

\textbf{اصل ۸۲} &
\textbf{حق جبران خسارت:} هر کس که به‌ناحق بازداشت یا محکوم شده باشد حق جبران خسارت مادی و معنوی را دارد. \\
\hline

\rowcolor{goldlight}
\textbf{اصل ۸۳} &
\textbf{اصلاح مجرمان:} هدف نظام کیفری اصلاح و بازپروری است، نه انتقام. شرایط زندان‌ها باید انسانی و اصلاحی باشد. \\
\hline
\end{longtable}

% ═══════════════════════════════════════════════════════════════════════════════
\section{فصل سوم: ساختار حکومت}
\label{sec:const-ch3}
% ═══════════════════════════════════════════════════════════════════════════════

\begin{figure}[htbp]
\centering
\begin{tikzpicture}[
    scale=0.85,
    transform shape,
    branch/.style={
        rectangle,
        rounded corners=10pt,
        draw=bleurepublique,
        fill=bleulight,
        line width=2pt,
        minimum width=4.5cm,
        minimum height=2.5cm,
        text centered,
        font=\bfseries
    },
    sub/.style={
        rectangle,
        rounded corners=5pt,
        draw=goldphoenix!50,
        fill=goldlight,
        line width=1pt,
        minimum width=3cm,
        minimum height=0.8cm,
        text centered,
        font=\tiny\bfseries
    },
    arrow/.style={->, ultra thick, bleurepublique!20, >=stealth}
]
\node[branch] (leg) at (-5.5, 3) {\rl{قوه مقننه}\\\rl{\tiny اصول ۸۶-۱۱۰}};
\node[branch] (exe) at (0, 3) {\rl{قوه مجریه}\\\rl{\tiny اصول ۱۱۱-۱۳۵}};
\node[branch] (jud) at (5.5, 3) {\rl{قوه قضائیه}\\\rl{\tiny اصول ۱۳۶-۱۵۵}};

\node[sub] (mn) at (-6.8, 0.5) {\rl{مجلس ملی}};
\node[sub] (ma) at (-3.8, 0.5) {\rl{مجلس اقوام}};
\node[sub] (pr) at (-0.8, 0.5) {\rl{رئیس‌جمهور}};
\node[sub] (cab) at (2.2, 0.5) {\rl{کابینه}};
\node[sub] (sc) at (4.2, 0.5) {\rl{دیوان عالی}};
\node[sub] (cc) at (7.2, 0.5) {\rl{دیوان قانون اساسی}};

\foreach \p/\c1/\c2 in {leg/mn/ma, exe/pr/cab, jud/sc/cc} {
    \draw[arrow] (\p) -- (\c1);
    \draw[arrow] (\p) -- (\c2);
}

\node[
    draw=goldphoenix,
    fill=goldlight,
    line width=2pt,
    rounded corners=12pt,
    minimum width=12cm,
    minimum height=1.5cm,
    font=\large\bfseries
] at (0, 6.5) {\rl{ساختار کلان حاکمیت در جمهوری فدرال ایران}};

\end{tikzpicture}
\caption{نمودار کلان تفکیک قوا و نهادهای بنیادین در میثاق ملی نوین}
\label{fig:gov-structure}
\end{figure}

\subsection{بخش یکم: قوه مقننه (اصول ۸۶-۱۱۰)}

\subsubsection{الف) ساختار دومجلسی}

\begin{longtable}{|>{\columncolor{bleulight}}r|p{12cm}|}
\hline
\rowcolor{bleurepublique!20}
\headmark شماره اصل & \headmark متن کامل اصل (ساختار قوه مقننه) \\
\hline
\endfirsthead
\hline
\rowcolor{bleurepublique!20}
\headmark شماره اصل & \headmark متن کامل اصل \\
\hline
\endhead

\textbf{اصل ۸۶} &
\textbf{پارلمان فدرال:} قوه مقننه متشکل از دو مجلس است:
الف) \textbf{مجلس ملی} (مجلس اول): نمایندگی مستقیم مردم؛
ب) \textbf{مجلس اقوام} (مجلس دوم): نمایندگی مناطق و اقوام.
هر دو مجلس در تصویب قوانین مشارکت دارند. \\
\hline

\rowcolor{goldlight}
\textbf{اصل ۸۷} &
\textbf{مجلس ملی:}
الف) مجلس ملی متشکل از ۳۵۰ نماینده است که با رأی مستقیم و مخفی مردم انتخاب می‌شوند.
ب) انتخابات به روش نسبی-فهرستی با آستانه ۳٪ برگزار می‌شود.
ج) دوره نمایندگی چهار سال است.
د) حداقل ۳۰٪ کاندیداها باید زن باشند. \\
\hline

\textbf{اصل ۸۸} &
\textbf{مجلس اقوام:}
الف) مجلس اقوام متشکل از ۱۲۰ نماینده است:
- ۶۰ نماینده: ۴ نماینده از هر استان (انتخابی)
- ۳۰ نماینده: نمایندگان مناطق خودمختار
- ۲۰ نماینده: نمایندگان اقلیت‌های زبانی-فرهنگی
- ۱۰ نماینده: نمایندگان ایرانیان خارج از کشور
ب) دوره نمایندگی شش سال است و هر دو سال یک‌سوم اعضا تجدید می‌شوند.
ج) رئیس مجلس اقوام معاون اول رئیس‌جمهور در تشریفات است. \\
\hline

\rowcolor{goldlight}
\textbf{اصل ۸۹} &
\textbf{شرایط نمایندگی:}
الف) تابعیت ایرانی؛
ب) سن حداقل ۲۵ سال برای مجلس ملی و ۳۵ سال برای مجلس اقوام؛
ج) برخورداری از حقوق مدنی؛
د) نداشتن محکومیت کیفری مؤثر؛
هـ) حداقل مدرک کارشناسی. \\
\hline

\textbf{اصل ۹۰} &
\textbf{مصونیت پارلمانی:}
الف) نمایندگان به خاطر اظهارات و آرایشان در مجلس قابل تعقیب نیستند.
ب) بازداشت نماینده در حال انجام وظیفه مستلزم اجازه مجلس است، مگر در جرم مشهود.
ج) مصونیت شامل جرایم عادی خارج از وظایف نمایندگی نمی‌شود. \\
\hline

\rowcolor{goldlight}
\textbf{اصل ۹۱} &
\textbf{تعارض منافع:}
الف) نمایندگان نمی‌توانند همزمان شغل دولتی داشته باشند.
ب) نمایندگان موظف به اعلام دارایی‌ها و منافع مالی خود هستند.
ج) رأی‌دادن در موضوعاتی که نماینده ذی‌نفع است ممنوع است. \\
\hline
\end{longtable}

\subsubsection{ب) فرآیند قانون‌گذاری}

\begin{figure}[htbp]
\centering
\begin{tikzpicture}[
    scale=0.9,
    transform shape,
    step/.style={
        rectangle,
        rounded corners=8pt,
        draw=bleurepublique,
        fill=bleulight,
        line width=1.5pt,
        minimum width=2.8cm,
        minimum height=1.2cm,
        text centered,
        font=\small\bfseries
    },
    arrow/.style={->, ultra thick, bleurepublique!30, >=stealth}
]
\node[step] (s1) at (0,0) {\rl{پیشنهاد لایحه/طرح}};
\node[step] (s2) at (3.5,0) {\rl{کمیسیون تخصصی}};
\node[step] (s3) at (7,0) {\rl{بررسی مجلس ملی}};
\node[step] (s4) at (10.5,0) {\rl{تصویب مجلس ملی}};

\node[step] (s5) at (10.5,-2.5) {\rl{بررسی مجلس اقوام}};
\node[step] (s8) at (7,-2.5) {\rl{تأیید رئیس‌جمهور}};
\node[step, draw=goldphoenix, fill=goldlight] (s9) at (3.5,-2.5) {\rl{انتشار و اجرا}};

\draw[arrow] (s1) -- (s2);
\draw[arrow] (s2) -- (s3);
\draw[arrow] (s3) -- (s4);
\draw[arrow] (s4) -- (s5);
\draw[arrow] (s5) -- (s8);
\draw[arrow] (s8) -- (s9);

\node[
    draw=goldphoenix,
    fill=goldlight,
    rounded corners=5pt,
    minimum width=4cm,
    font=\tiny\bfseries
] at (0,-2.5) {\rl{کمیته حل اختلاف در صورت تعارض}};
\end{tikzpicture}
\caption{فرآیند دموکراتیک قانون‌گذاری و تعادل قوا در ساختار دومجلسی}
\label{fig:legislative-flow}
\end{figure}

\begin{longtable}{|>{\columncolor{bleulight}}r|p{12cm}|}
\hline
\rowcolor{bleurepublique!20}
\headmark شماره اصل & \headmark متن کامل اصل (فرآیند قانون‌گذاری و نظارت) \\
\hline
\endfirsthead
\hline
\rowcolor{bleurepublique!20}
\headmark شماره اصل & \headmark متن کامل اصل \\
\hline
\endhead

\textbf{اصل ۹۲} & \textbf{ابتکار قانون‌گذاری:} حق پیشنهاد قانون متعلق به هیئت وزیران (لایحه)، ۱۵ نماینده مجلس ملی (طرح)، ۱۰ نماینده مجلس اقوام (طرح)، و ۵۰۰,۰۰۰ شهروند (ابتکار مردمی) است. \\
\hline

\rowcolor{goldlight}
\textbf{اصل ۹۳} & \textbf{تصویب قوانین عادی:} الف) قوانین عادی با اکثریت ساده هر دو مجلس تصویب می‌شوند. ب) در صورت اختلاف، کمیته مشترک تشکیل می‌شود. ج) اگر توافق حاصل نشد، مجلس ملی با اکثریت مطلق تصمیم نهایی را می‌گیرد. \\
\hline

\textbf{اصل ۹۴} & \textbf{قوانین اساسی و بنیادین:} قوانینی که به حقوق اقوام، ساختار فدرال، یا حقوق بنیادین مربوط می‌شوند، نیازمند تصویب دوسوم هر دو مجلس هستند. \\
\hline

\rowcolor{goldlight}
\textbf{اصل ۹۵} & \textbf{حق وتوی مجلس اقوام:} مجلس اقوام در موضوعات فدرال، تقسیمات کشوری، حقوق اقوام، توزیع منابع ملی و انتصابات عالی قضایی حق وتوی مطلق دارد. \\
\hline

\textbf{اصل ۹۶} & \textbf{وتوی رئیس‌جمهور:} الف) رئیس‌جمهور حق بازگرداندن قانون ظرف ۱۵ روز را دارد. ب) مجلس با دوسوم آرا می‌تواند وتو را رد کند. ج) امضا و ابلاغ نهایی ظرف ۱۰ روز الزامی است. \\
\hline

\rowcolor{goldlight}
\textbf{اصل ۹۷} & \textbf{بودجه سالانه:} دولت موظف به تقدیم لایحه بودجه تا پایان آبان است. مجلس ملی تصویب‌کننده و مجلس اقوام ناظر بر توزیع عادلانه بین مناطق است. \\
\hline

\textbf{اصل ۹۸} & \textbf{نظارت پارلمانی:} الف) وزرا در برابر مجلس ملی مسئول‌اند. ب) حق سؤال و استیضاح برای نمایندگان محفوظ است. ج) رأی عدم اعتماد با اکثریت مطلق نمایندگان ممکن است. \\
\hline
\end{longtable}

\subsection{بخش دوم: قوه مجریه (اصول ۱۱۱-۱۳۵)}

\subsubsection{الف) رئیس‌جمهور}

\begin{figure}[htbp]
\centering
\begin{tikzpicture}[
    scale=0.9,
    transform shape,
    box/.style={
        rectangle,
        rounded corners=12pt,
        draw=bleurepublique,
        fill=bleulight,
        line width=2pt,
        minimum width=12cm,
        minimum height=5cm
    },
    label/.style={
        rectangle,
        fill=bleurepublique,
        text=white,
        rounded corners=5pt,
        font=\large\bfseries,
        minimum width=6cm
    }
]
\node[box] (main) at (0,0) {};
\node[label] at (0,2.5) {\rl{رئیس‌جمهور جمهوری فدرال ایران}};

\node[align=right, text width=5.5cm, font=\small] at (-3,0.5) {
    \textbf{\rl{انتخاب:}} \rl{رأی مستقیم مردم}\\[3pt]
    \textbf{\rl{دوره:}} \rl{۴ سال (حداکثر ۲ دوره)}\\[3pt]
    \textbf{\rl{شرایط سن:}} \rl{حداقل ۴۰ سال}
};

\node[align=right, text width=5.5cm, font=\small] at (3,0.5) {
    \textbf{\rl{نقش:}} \rl{رئیس دولت و کشور}\\[3pt]
    \textbf{\rl{مسئولیت:}} \rl{در برابر ملت و پارلمان}\\[3pt]
    \textbf{\rl{اختیار:}} \rl{فرماندهی کل قوا}
};

\node[
    draw=goldphoenix,
    fill=goldlight,
    rounded corners=8pt,
    minimum width=10cm,
    minimum height=1.2cm,
    text centered,
    font=\small\bfseries,
    text=bleurepublique
] at (0,-1.5) {\rl{وظایف: اجرای قانون اساسی | سیاست خارجی | انتصاب وزرا | امضای قوانین}};

\end{tikzpicture}
\caption{جایگاه، صلاحیت‌ها و فرآیند پاسخگویی نهاد ریاست جمهوری}
\label{fig:president-role}
\end{figure}

\begin{longtable}{|>{\columncolor{bleulight}}r|p{12cm}|}
\hline
\rowcolor{bleurepublique!20}
\headmark شماره اصل & \headmark متن کامل اصل (نهاد ریاست جمهوری) \\
\hline
\endfirsthead
\hline
\rowcolor{bleurepublique!20}
\headmark شماره اصل & \headmark متن کامل اصل \\
\hline
\endhead

\textbf{اصل ۱۱۱} & \textbf{جایگاه رئیس‌جمهور:} رئیس‌جمهور بالاترین مقام رسمی کشور پس از ملت، رئیس قوه مجریه، رئیس دولت و نماینده عالی جمهوری در روابط بین‌المللی است. \\
\hline

\rowcolor{goldlight}
\textbf{اصل ۱۱۲} & \textbf{انتخاب:} الف) انتخاب با رأی مستقیم و مخفی. ب) کسب اکثریت مطلق (۵۰٪ + ۱) آرا الزامی است. ج) در غیر این صورت، دور دوم بین دو نفر اول برگزار می‌شود. \\
\hline

\textbf{اصل ۱۱۳} & \textbf{شرایط:} تابعیت ایرانی، حداقل ۴۰ سال سن، مدرک کارشناسی ارشد، و فقدان محکومیت کیفری. حمایت ۱۰۰ هزار شهروند یا ۵۰ نماینده برای کاندیداتوری لازم است. \\
\hline

\rowcolor{goldlight}
\textbf{اصل ۱۱۶} & \textbf{وظایف:} تعیین سیاست‌های دولت، تشکیل هیئت وزیران، فرماندهی کل قوا، امضای معاهدات و ابلاغ قوانین. \\
\hline

\textbf{اصل ۱۱۸} & \textbf{استیضاح:} یک‌سوم نمایندگان مجلس ملی می‌توانند طرح استیضاح را مطرح کنند. رأی عدم اعتماد نیازمند دوسوم آرای مجلس ملی و تأیید مجلس اقوام است. \\
\hline
\end{longtable}

\subsubsection{ب) هیئت وزیران}

\begin{longtable}{|>{\columncolor{bleulight}}r|p{12cm}|}
\hline
\rowcolor{bleurepublique!20}
\headmark شماره اصل & \headmark متن کامل اصل (هیئت وزیران) \\
\hline
\endfirsthead
\hline
\rowcolor{bleurepublique!20}
\headmark شماره اصل & \headmark متن کامل اصل \\
\hline
\endhead

\textbf{اصل ۱۲۰} & \textbf{ترکیب:} الف) رئیس‌جمهور، معاونان و وزرا. ب) حداکثر ۲۰ وزارتخانه. ج) رعایت تنوع جغرافیایی، قومی و جنسیتی (حداقل ۳۰٪ زن) الزامی است. \\
\hline

\rowcolor{goldlight}
\textbf{اصل ۱۲۱} & \textbf{رأی اعتماد:} وزرای پیشنهادی باید رأی اعتماد اکثریت نمایندگان مجلس ملی را کسب کنند. در صورت سه بار رد کلیات کابینه، انتخابات زودهنگام برگزار می‌شود. \\
\hline

\textbf{اصل ۱۲۲} & \textbf{وظایف:} تدوین سیاست‌های دولت، تهیه لوایح و بودجه سالانه، تصویب آیین‌نامه‌ها و تأمین امنیت ملی. \\
\hline

\rowcolor{goldlight}
\textbf{اصل ۱۲۴} & \textbf{تعارض منافع:} ممنوعیت شغل دوم برای وزرا، اعلام کامل دارایی‌ها در ابتدا و انتها، و محدودیت ۲ ساله برای ورود به بخش خصوصی مرتبط پس از خدمت. \\
\hline
\end{longtable}

\subsubsection{ج) نیروهای مسلح}

\begin{longtable}{|>{\columncolor{bleulight}}r|p{12cm}|}
\hline
\rowcolor{bleurepublique!20}
\headmark شماره اصل & \headmark متن کامل اصل (نیروهای مسلح) \\
\hline
\endfirsthead
\hline
\rowcolor{bleurepublique!20}
\headmark شماره اصل & \headmark متن کامل اصل \\
\hline
\endhead

\textbf{اصل ۱۲۵} & \textbf{وظیفه نیروهای مسلح:} دفاع از استقلال و تمامیت ارضی، حمایت از نظام قانون اساسی و نظم دموکراتیک، و کمک در بلایای طبیعی. نیروهای مسلح ابزار سیاست تهاجمی نیستند. \\
\hline

\rowcolor{goldlight}
\textbf{اصل ۱۲۶} & \textbf{فرماندهی:} رئیس‌جمهور فرمانده کل است. اعلام جنگ و صلح و اعزام نیرو به خارج مستلزم تصویب دوسوم مجلس ملی است. \\
\hline

\textbf{اصل ۱۲۷} & \textbf{غیرسیاسی بودن:} ممنوعیت فعالیت حزبی و دخالت در امور سیاسی. نظامیان در دوران خدمت حق کاندیداتوری ندارند. نظارت پارلمانی بر ارتش الزامی است. \\
\hline

\rowcolor{goldlight}
\textbf{اصل ۱۲۸} & \textbf{بودجه دفاعی:} بودجه نظامی باید به تصویب مجلس برسد و کمیسیون دفاع بر هزینه‌ها نظارت دقیق دارد. \\
\hline

\textbf{اصل ۱۲۹} & \textbf{ممنوعیت کودتا:} هرگونه اقدام نظامی برای تغییر نظام قانون اساسی کودتا و جرم سنگین است. حق مقاومت شهروندی در برابر کودتا به رسمیت شناخته می‌شود. \\
\hline
\end{longtable}

\subsection{بخش سوم: قوه قضائیه (اصول ۱۳۶-۱۵۵)}

\begin{figure}[htbp]
\centering
\begin{tikzpicture}[
    scale=0.9,
    transform shape,
    court/.style={
        rectangle,
        rounded corners=8pt,
        draw=bleurepublique,
        fill=bleulight,
        line width=1.5pt,
        minimum width=4cm,
        minimum height=1.5cm,
        text centered,
        font=\small\bfseries
    },
    arrow/.style={->, ultra thick, bleurepublique!30, >=stealth}
]
\node[court, draw=goldphoenix, fill=goldlight] (cc) at (0,2.5) {\rl{دیوان عالی قانون اساسی}\\\rl{\tiny ۱۵ قاضی}};
\node[court] (sc) at (-4,0.5) {\rl{دیوان عالی کشور}\\\rl{\tiny رئیس + ۲۴ قاضی}};
\node[court] (ac) at (4,0.5) {\rl{دیوان عدالت اداری}\\\rl{\tiny ۱۲ قاضی}};
\node[court, minimum height=1cm] (app) at (-4,-1.5) {\rl{دادگاه‌های استیناف}};
\node[court, minimum height=1cm] (first) at (4,-1.5) {\rl{دادگاه‌های بدوی}};
\node[court, draw=gray, fill=gray!10] (spec) at (0,-1.5) {\rl{دادگاه‌های تخصصی}};

\foreach \s/\t in {cc/sc, cc/ac, sc/app, app/first, spec/sc} {
    \draw[arrow] (\s) -- (\t);
}

\node[
    draw=goldphoenix,
    fill=goldlight,
    rounded corners=5pt,
    minimum width=6cm,
    minimum height=1cm,
    font=\bfseries
] at (0,-3.5) {\rl{شورای عالی قضایی (نهاد راهبری)}};
\end{tikzpicture}
\caption{ساختار سلسله‌مراتبی قوه قضائیه در نظام فدرال دموکراتیک}
\label{fig:jud-structure}
\end{figure}

\begin{longtable}{|>{\columncolor{bleulight}}r|p{12cm}|}
\hline
\rowcolor{bleurepublique!20}
\headmark شماره اصل & \headmark متن کامل اصل (قوه قضائیه و دیوان قانون اساسی) \\
\hline
\endfirsthead
\hline
\rowcolor{bleurepublique!20}
\headmark شماره اصل & \headmark متن کامل اصل \\
\hline
\endhead

\textbf{اصل ۱۳۶} & \textbf{استقلال قضایی:} قوه قضائیه مستقل است. قضات در صدور رأی تنها تابع قانون هستند. بودجه قضایی مستقل از قوه مجریه و توسط پارلمان تأمین می‌شود. \\
\hline

\rowcolor{goldlight}
\textbf{اصل ۱۳۷} & \textbf{شورای عالی قضایی:} بالاترین نهاد راهبری قضایی متشکل از رئیس دیوان عالی، دادستان کل، قضات منتخب و حقوقدانان. وظیفه انتصاب، ترفیع و نظارت بر قضات را دارد. \\
\hline

\textbf{اصل ۱۳۸} & \textbf{انتصاب قضات:} قضات توسط شورا منصوب می‌شوند. قضات دیوان عالی و دیوان قانون اساسی نیازمند تأیید مجلس اقوام و مجلس ملی هستند. \\
\hline

\rowcolor{goldlight}
\textbf{اصل ۱۳۹} & \textbf{تصدی قضاوت:} قضات دارای تصدی مادام‌العمر تا ۷۰ سالگی هستند. عزل یا انتقال آنها تنها در موارد استثنایی قانونی و حکم دادگاه انتظامی ممکن است. \\
\hline

\textbf{اصل ۱۴۰} & \textbf{دیوان عالی کشور:} عالی‌ترین مرجع قضایی مدنی و کیفری، ناظر بر حسن اجرای قوانین و ایجاد وحدت رویه قضایی. \\
\hline

\rowcolor{goldlight}
\textbf{اصل ۱۴۱} & \textbf{دیوان عالی قانون اساسی:} پاسدار میثاق ملی و حقوق بنیادین، متشکل از ۱۵ قاضی. صلاحیت تفسیر قانون اساسی و ابطال قوانین مغایر را دارد. \\
\hline

\textbf{اصل ۱۴۲} & \textbf{دسترسی به دیوان:} هر شهروند در صورت نقض حقوق بنیادین حق شکایت به دیوان قانون اساسی را دارد. مراجع سیاسی و دادگاه‌ها نیز حق ارجاع دارند. \\
\hline

\rowcolor{goldlight}
\textbf{اصل ۱۴۳} & \textbf{دیوان عدالت اداری:} مرجع رسیدگی به شکایات شهروندان از تصمیمات و آیین‌نامه‌های دولتی و ابطال اقدامات غیرقانونی. \\
\hline

\textbf{اصل ۱۴۴} & \textbf{دادستانی کل:} مسئول تعقیب جرایم و حفاظت از حقوق عامه. دادستان توسط شورا پیشنهاد و با تأیید مجلس منصوب می‌شود. \\
\hline

\rowcolor{goldlight}
\textbf{اصل ۱۴۵} & \textbf{دادرسی منصفانه:} علنی بودن محاکمات، حق داشتن وکیل، مستدل بودن احکام و حق تجدیدنظرخواهی همگانی. \\
\hline
\end{longtable}

ب) وظایف: نظارت بر حسن اجرای قوانین، ایجاد رویه قضایی واحد، رسیدگی به فرجام‌خواهی

ج) آرای دیوان در ایجاد رویه قضایی برای دادگاه‌های پایین‌تر الزام‌آور است. \\
\hline

\textbf{اصل ۱۴۱} &
\textbf{دیوان عالی قانون اساسی}

الف) دیوان عالی قانون اساسی پاسدار قانون اساسی و حقوق بنیادین است.

ب) ترکیب: ۱۵ قاضی با دوره ۱۲ ساله غیرقابل تمدید (هر ۴ سال، ۵ نفر تجدید)

ج) صلاحیت‌ها:
\begin{itemize}[nosep]
    \item بررسی انطباق قوانین با قانون اساسی
    \item رسیدگی به شکایات نقض حقوق بنیادین
    \item حل اختلاف بین قوا و بین مرکز و مناطق
    \item تفسیر قانون اساسی
    \item نظارت بر انتخابات ملی و همه‌پرسی
\end{itemize} \\
\hline

\textbf{اصل ۱۴۲} &
\textbf{دسترسی به دیوان قانون اساسی}

حق طرح شکایت در دیوان قانون اساسی متعلق است به:

الف) هر شهروند که حقوق بنیادینش نقض شده باشد

ب) رئیس‌جمهور، رئیس هر مجلس

ج) یک‌پنجم نمایندگان هر مجلس

د) دولت‌های محلی (در موضوعات فدرالی)

هـ) دادگاه‌ها هنگام شک در قانون اساسی بودن قانون \\
\hline

\textbf{اصل ۱۴۳} &
\textbf{دیوان عدالت اداری}

الف) دیوان عدالت اداری مرجع رسیدگی به شکایات از دستگاه‌های دولتی است.

ب) هر شهروند حق شکایت از تصمیمات اداری را دارد.

ج) دیوان می‌تواند تصمیمات غیرقانونی را ابطال و حکم به جبران خسارت دهد. \\
\hline

\textbf{اصل ۱۴۴} &
\textbf{دادستانی کل}

الف) دادستان کل مسئول تعقیب جرایم و نظارت بر اجرای قانون است.

ب) دادستان کل توسط شورای عالی قضایی پیشنهاد و با تأیید مجلس منصوب می‌شود.

ج) دادستانی مستقل است و تحت نظارت مجلس قرار دارد. \\
\hline

\textbf{اصل ۱۴۵} &
\textbf{دادرسی منصفانه}

الف) محاکمات باید علنی باشد مگر در موارد استثنایی قانونی.

ب) هر فرد حق داشتن وکیل و فرصت کافی برای دفاع را دارد.

ج) احکام باید مستدل و مکتوب باشند.

د) حق تجدیدنظر در همه احکام تضمین می‌شود. \\
\hline

\end{longtable}

% ═══════════════════════════════════════════════════════════════════════════════
\section{فصل چهارم: مجلس اقوام و ساختار فدرال}
\label{sec:const-ch4}
% ═══════════════════════════════════════════════════════════════════════════════

\begin{olgoobox}
\textbf{فدرالیسم همبسته: الگوی ایرانی}

ساختار فدرال جمهوری فدرال ایران بر اساس اصل «وحدت در کثرت» طراحی شده است. این ساختار:

\begin{itemize}[nosep]
    \item تنوع قومی-فرهنگی را به رسمیت می‌شناسد
    \item خودمختاری منطقه‌ای را تضمین می‌کند
    \item همبستگی و وحدت ملی را حفظ می‌کند
    \item از تجزیه‌طلبی جلوگیری می‌کند
\end{itemize}
\end{olgoobox}

\begin{figure}[htbp]
\centering
\begin{tikzpicture}[
    scale=0.9,
    transform shape,
    level/.style={
        rectangle,
        rounded corners=8pt,
        draw=bleurepublique,
        fill=bleulight,
        line width=1.5pt,
        minimum width=12cm,
        minimum height=1.2cm,
        text centered,
        font=\bfseries
    },
    arrow/.style={<->, ultra thick, bleurepublique!30, >=stealth}
]
\node[level, fill=bleurepublique, text=white] (fed) at (0,3) {\rl{سطح فدرال: حاکمیت ملی، دفاع، سیاست خارجی، پولی}};
\node[level, draw=goldphoenix, fill=goldlight] (reg) at (0,1) {\rl{سطح منطقه‌ای: ۵ منطقه خودمختار + ۱۵ استان}};
\node[level] (prov) at (0,-1) {\rl{سطح استانی: آموزش، بهداشت، قانون‌گذاری محلی}};
\node[level, fill=white] (loc) at (0,-3) {\rl{سطح محلی: شهرداری‌ها و دموکراسی مستقیم شورایی}};

\draw[arrow] (fed) -- (reg);
\draw[arrow] (reg) -- (prov);
\draw[arrow] (prov) -- (loc);
\end{tikzpicture}
\caption{هرم ساختار فدرال جمهوری فدرال ایران از مرکز تا جوامع محلی}
\label{fig:federal-pyramid}
\end{figure}

\begin{longtable}{|>{\columncolor{bleulight}}r|p{12cm}|}
\hline
\rowcolor{bleurepublique!20}
\headmark شماره اصل & \headmark متن کامل اصل (ساختار فدرال و تقسیمات کشوری) \\
\hline
\endfirsthead
\hline
\rowcolor{bleurepublique!20}
\headmark شماره اصل & \headmark متن کامل اصل \\
\hline
\endhead

\textbf{اصل ۱۴۶} & \textbf{تقسیمات کشوری:} ایران متشکل از ۵ منطقه خودمختار (آذربایجان، کردستان، بلوچستان، خوزستان و ترکمن‌صحرا) و ۱۵ استان فدرال است. تغییر در مرزها نیازمند همه‌پرسی محلی و تصویب دوسوم پارلمان است. \\
\hline

\rowcolor{goldlight}
\textbf{اصل ۱۴۷} & \textbf{خودمختاری منطقه‌ای:} الف) هر منطقه دارای پارلمان و دولت منطقه‌ای منتخب است. ب) زبان‌های بومی در کنار زبان فارسی در مناطق مربوطه رسمیت دارند. ج) مناطق در امور فرهنگی، آموزشی و عمران محلی خودمختارند. \\
\hline

\textbf{اصل ۱۴۸} & \textbf{صلاحیت‌های انحصاری فدرال:} حاکمیت ملی، دفاع، سیاست خارجی، سیاست پولی، گمرک، و منابع استراتژیک (نفت و گاز) در صلاحیت انحصاری دولت فدرال است. \\
\hline

\rowcolor{goldlight}
\textbf{اصل ۱۴۹} & \textbf{صلاحیت‌های مشترک:} محیط زیست، بهداشت، آموزش عالی و جنگل‌ها در صلاحیت مشترک مرکز و مناطق است. در صورت تعارض، قوانین فدرال اولویت دارند. \\
\hline

\textbf{اصل ۱۵۰} & \textbf{همبستگی فدرال:} مناطق موظف به رعایت وحدت ملی و تمامیت ارضی هستند. هیچ منطقه‌ای حق جدایی یک‌جانبه را ندارد. \\
\hline

\rowcolor{goldlight}
\textbf{اصل ۱۵۱} & \textbf{توازن مالی (عدالت توزیعی):} دولت فدرال موظف به توزیع عادلانه درآمدهای ملی بین مناطق بر اساس جمعیت و شاخص‌های محرومیت است تا شکاف‌های توسعه‌ای برطرف شود. \\
\hline

\textbf{اصل ۱۵۲} & \textbf{شورای فرمانداران:} برای هماهنگی بین مرکز و مناطق، شورای فرمانداران به ریاست رئیس‌جمهور تشکیل می‌شود. اختلافات بین‌منطقه‌ای توسط دیوان قانون اساسی حل می‌شود. \\
\hline
\end{longtable}

% ═══════════════════════════════════════════════════════════════════════════════
\section{فصل پنجم: نهادهای مستقل نظارتی}
\label{sec:const-ch5}
% ═══════════════════════════════════════════════════════════════════════════════

\begin{figure}[htbp]
\centering
\begin{tikzpicture}[
    scale=0.9,
    transform shape,
    center/.style={
        rectangle,
        rounded corners=12pt,
        draw=bleurepublique,
        fill=bleulight,
        line width=2pt,
        minimum width=5cm,
        minimum height=1.5cm,
        font=\large\bfseries
    },
    node/.style={
        rectangle,
        rounded corners=8pt,
        draw=goldphoenix!70,
        fill=goldlight,
        minimum width=3.2cm,
        minimum height=1.2cm,
        align=center,
        font=\tiny\bfseries
    },
    arrow/.style={->, thick, bleurepublique!40, >=stealth}
]
\node[center] (c) at (0,0) {\rl{نهادهای نظارتی مستقل}};

\foreach \ang/\label/\desc in {60/کمیسیون انتخابات/نظارت بر آرا, 120/بانک مرکزی/سیاست پولی, 180/سازمان حسابرسی/شفافیت مالی, 240/کمیسیون حقوق بشر/حمایت از کرامت, 300/سازمان ضد فساد/مبارزه با رانت, 0/شورای عالی رسانه/آزادی بیان} {
    \node[node] (n\ang) at (\ang:4.5) {\rl{\label}\\\rl{\tiny \desc}};
    \draw[arrow] (c) -- (n\ang);
}
\end{tikzpicture}
\caption{منظومه نهادهای مستقل نظارتی (رکن چهارم دموکراسی)}
\label{fig:oversight}
\end{figure}

\begin{longtable}{|>{\columncolor{bleulight}}r|p{12cm}|}
\hline
\rowcolor{bleurepublique!20}
\headmark شماره اصل & \headmark متن کامل اصل (نهادهای مستقل نظارتی) \\
\hline
\endfirsthead
\hline
\rowcolor{bleurepublique!20}
\headmark شماره اصل & \headmark متن کامل اصل \\
\hline
\endhead

\textbf{اصل ۱۵۳} & \textbf{جایگاه نهادهای مستقل:} این نهادها رکن چهارم نظام‌اند. از قوای سه‌گانه مستقل بوده، بودجه مجزا دارند و رؤسای آنها با رأی دوسوم مجلس منصوب می‌شوند. \\
\hline

\rowcolor{goldlight}
\textbf{اصل ۱۵۴} & \textbf{کمیسیون مستقل انتخابات:} مسئول برگزاری و نظارت بر کلیه انتخابات و همه‌پرسی‌ها. اعضا باید فاقد وابستگی حزبی بوده و توسط دیوان عالی تأیید شوند. \\
\hline

\textbf{اصل ۱۵۵} & \textbf{بانک مرکزی:} هدف اصلی ثبات قیمت‌ها و حفظ ارزش پول است. دولت حق استقراض مستقیم یا دخالت در سیاست‌های پولی را ندارد. \\
\hline

\rowcolor{goldlight}
\textbf{اصل ۱۵۶} & \textbf{سازمان حسابرسی:} نظارت بر سلامت مالی و بودجه‌ای کلیه دستگاه‌های حکومتی و شرکت‌های عمومی و ارائه گزارش شفاف به ملت. \\
\hline

\textbf{اصل ۱۵۷} & \textbf{کمیسیون حقوق بشر:} صیانت از حقوق بنیادین، دریافت شکایات شهروندان، و نظارت بر حسن اجرای کنوانسیون‌های بین‌المللی حقوق بشر در ایران. \\
\hline

\rowcolor{goldlight}
\textbf{اصل ۱۵۸} & \textbf{سازمان ضد فساد:} پیشگیری و مبارزه با فساد در سطوح عالی حاکمیتی، حمایت از سوت‌زنان و شفاف‌سازی دارایی مقامات. \\
\hline

\textbf{اصل ۱۵۹} & \textbf{شورای عالی رسانه:} تضمین آزادی بیان، تکثرگرایی رسانه‌ای و ممنوعیت سانسور. نظارت بر رسانه‌های عمومی برای اطمینان از بی‌طرفی. \\
\hline

\rowcolor{goldlight}
\textbf{اصل ۱۶۰} & \textbf{نهاد بازرسی کل:} رسیدگی به شکایات شهروندان از سوءمدیریت اداری و تضمین پاسخگویی مقامات در برابر قانون. \\
\hline
\end{longtable}

% ═══════════════════════════════════════════════════════════════════════════════
\section{فصل ششم: حکومت محلی و شوراها}
\label{sec:const-ch6}
% ═══════════════════════════════════════════════════════════════════════════════

\begin{longtable}{|>{\columncolor{bleulight}}r|p{12cm}|}
\hline
\rowcolor{bleurepublique!20}
\headmark شماره اصل & \headmark متن کامل اصل (حکومت محلی و شوراها) \\
\hline
\endfirsthead
\hline
\rowcolor{bleurepublique!20}
\headmark شماره اصل & \headmark متن کامل اصل \\
\hline
\endhead

\textbf{اصل ۱۶۱} & \textbf{تمرکززدایی:} اداره امور محلی بر اساس اصل تقرب (واگذاری اختیارات به پایین‌ترین سطح ممکن) و مشارکت مستقیم شهروندان انجام می‌شود. \\
\hline

\rowcolor{goldlight}
\textbf{اصل ۱۶۲} & \textbf{شوراهای محلی:} در هر شهر و روستا شورای منتخب با رأی مستقیم مردم برای ۴ سال تشکیل می‌شود. شورا ناظر بر مدیریت محلی و تصویب‌کننده بودجه شهرداری است. \\
\hline

\textbf{اصل ۱۶۳} & \textbf{شهرداری‌ها:} شهردار توسط شورا انتخاب می‌شود. شهرداری‌ها استقلال مالی داشته و در قبال شورا و شهروندان پاسخگو هستند. \\
\hline

\rowcolor{goldlight}
\textbf{اصل ۱۶۴} & \textbf{استانداران:} استاندار هماهنگ‌کننده دولت فدرال و محلی است که با پیشنهاد وزیر کشور و تأیید شورای منتخب منطقه منصوب می‌شود. \\
\hline

\textbf{اصل ۱۶۵} & \textbf{شوراهای منطقه‌ای:} مسئول برنامه‌ریزی توسعه کلان منطقه، هماهنگی بین استانی و مدیریت منابع مشترک منطقه‌ای هستند. \\
\hline

\rowcolor{goldlight}
\textbf{اصل ۱۶۶} & \textbf{مشارکت و همه‌پرسی محلی:} حق شهروندان برای مشارکت در بودجه‌نویسی و الزام برگزاری رفراندوم محلی برای طرح‌های حساس زیست‌محیطی یا عمرانی محلی. \\
\hline
\end{longtable}

% ═══════════════════════════════════════════════════════════════════════════════
\section{فصل هفتم: بازنگری در قانون اساسی}
\label{sec:const-ch7}
% ═══════════════════════════════════════════════════════════════════════════════

\begin{center}
\begin{tikzpicture}[
    step/.style={
        draw=DemocracyBlue,
        fill=DemocracyBlue!15,
        rounded corners=8pt,
        minimum width=3.5cm,
        minimum height=2cm,
        align=center,
        font=\small
    }
]
    % عنوان
    \node[
        fill=DemocracyBlue,
        text=white,
        rounded corners=5pt,
        font=\large\bfseries,
        minimum width=10cm
    ] at (0,4) {فرآیند بازنگری قانون اساسی};
    
    % مراحل
    \node[step] (s1) at (-5,1.5) {\textbf{مرحله ۱}\\پیشنهاد\\(دوسوم هر مجلس\\یا ۲ میلیون امضا)};
    
    \node[step] (s2) at (0,1.5) {\textbf{مرحله ۲}\\بررسی\\(کمیته ویژه\\بازنگری)};
    
    \node[step] (s3) at (5,1.5) {\textbf{مرحله ۳}\\تصویب\\(سه‌چهارم\\هر دو مجلس)};
    
    \node[step, fill=SuccessGreen!20, draw=SuccessGreen] (s4) at (0,-1.5) {\textbf{مرحله ۴}\\همه‌پرسی\\(اکثریت مطلق\\با مشارکت ۵۰٪)};
    
    % فلش‌ها
    \draw[->, thick, DemocracyBlue] (s1) -- (s2);
    \draw[->, thick, DemocracyBlue] (s2) -- (s3);
    \draw[->, thick, DemocracyBlue] (s3) -- (s4);
    
    % خط قرمز
    \node[
        draw=WarningRed,
        fill=WarningRed!10,
        rounded corners=5pt,
        minimum width=10cm,
        font=\small
    ] at (0,-3.5) {\textbf{⚠️ اصول غیرقابل تغییر (اصل ۱۱) قابل بازنگری نیستند}};
\end{tikzpicture}
\end{center}

\begin{longtable}{|>{\columncolor{bleulight}}r|p{12cm}|}
\hline
\rowcolor{bleurepublique!20}
\headmark شماره اصل & \headmark متن کامل اصل (بازنگری در قانون اساسی) \\
\hline
\endfirsthead
\hline
\rowcolor{bleurepublique!20}
\headmark شماره اصل & \headmark متن کامل اصل \\
\hline
\endhead

\textbf{اصل ۱۶۷} & \textbf{اصل بازنگری:} این میثاق ملی سندی زنده است که بر اساس خرد جمعی و نیازهای زمان قابل بازنگری است، مشروط به حفظ هویت دموکراتیک و حقوق بنیادین. \\
\hline

\rowcolor{goldlight}
\textbf{اصل ۱۶۸} & \textbf{پیشنهاد بازنگری:} حق پیشنهاد بر عهده رئیس‌جمهور، دوسوم هر یک از مجالس، یا با ابتکار مردمی ۲ میلیون شهروند است. \\
\hline

\textbf{اصل ۱۶۹} & \textbf{کمیته بازنگری:} پس از تأیید ضرورت، کمیته‌ای متشکل از نمایندگان مردم، قضات عالی و حقوقدانان ظرف ۶ ماه پیش‌نویس طرح بازنگری را تهیه می‌کنند. \\
\hline

\rowcolor{goldlight}
\textbf{اصل ۱۷۰} & \textbf{تصویب پارلمانی:} طرح بازنگری نیازمند تصویب سه‌چهارم آرای هر دو مجلس (مجلس ملی و مجلس اقوام) در فاصله‌ای حداقل ۶۰ روزه است. \\
\hline

\textbf{اصل ۱۷۱} & \textbf{تأیید ملی (همه‌پرسی):} هرگونه تغییر در قانون اساسی پس از تصویب پارلمان، باید به همه‌پرسی عمومی گذاشته شده و با اکثریت مطلق آرا تأیید شود. \\
\hline

\rowcolor{goldlight}
\textbf{اصل ۱۷۲} & \textbf{اصول غیرقابل تغییر:} اصول مربوط به کرامت انسانی، نظام دموکراتیک، تفکیک قوا، فدرالیسم، و لغو مجازات اعدام به هیچ‌وجه قابل بازنگری نیستند. \\
\hline

\textbf{اصل ۱۷۳} & \textbf{اعلام و نفاذ:} بازنگری مصوب ظرف ۱۵ روز توسط رئیس‌جمهور ابلاغ و در سراسر کشور لازم‌الاجرا می‌گردد. \\
\hline
\end{longtable}

% ═══════════════════════════════════════════════════════════════════════════════
\section{فصل هشتم: مقررات انتقالی}
\label{sec:const-ch8}
% ═══════════════════════════════════════════════════════════════════════════════

\begin{enghelabbox}
\textbf{⚠️ اهمیت مقررات انتقالی}

مقررات انتقالی پل ارتباطی بین نظام قدیم و نظام جدید هستند. این مقررات:
\begin{itemize}[nosep]
    \item چارچوب زمانی مشخص برای اجرای قانون اساسی تعیین می‌کنند
    \item از خلأ قانونی و هرج‌ومرج جلوگیری می‌کنند
    \item حقوق مکتسبه را در حد معقول حفظ می‌کنند
    \item پس از اتمام دوره انتقال، خودبه‌خود منقضی می‌شوند
\end{itemize}
\end{enghelabbox}

\begin{longtable}{|>{\columncolor{bleulight}}r|p{12cm}|}
\hline
\rowcolor{bleurepublique!20}
\headmark شماره اصل & \headmark متن کامل اصل (مقررات انتقالی و نهایی) \\
\hline
\endfirsthead
\hline
\rowcolor{bleurepublique!20}
\headmark شماره اصل & \headmark متن کامل اصل \\
\hline
\endhead

\textbf{اصل ۱۷۴} & \textbf{لازم‌الاجرا شدن:} این قانون اساسی از تاریخ تأیید در همه‌پرسی ملی لازم‌الاجراست. دولت موقت موظف به زمینه‌سازی فوری برای اجرای آن است. \\
\hline

\rowcolor{goldlight}
\textbf{اصل ۱۷۵} & \textbf{دوره انتقال:} بازه زمانی حداکثر ۱۸ ماهه از تصویب قانون اساسی تا استقرار کامل نهادهای برگزیده، که طی آن امنیت گذار و تداوم خدمات عمومی تضمین می‌شود. \\
\hline

\textbf{اصل ۱۷۶} & \textbf{اولین انتخابات:} برگزاری اولین انتخابات سراسری تحت نظارت کمیسیون موقت و نهادهای بین‌المللی ظرف ۱۲ ماه از تصویب. \\
\hline

\rowcolor{goldlight}
\textbf{اصل ۱۷۷} & \textbf{جدول استقرار:} تعیین ضرب‌الاجل‌های ۳ تا ۲۴ ماهه برای تشکیل کمیسیون انتخابات، پارلمان‌ها، دیوان عالی قانون اساسی و شوراهای محلی. \\
\hline

\textbf{اصل ۱۷۸} & \textbf{تطبیق قوانین:} قوانین پیشین تا زمان نسخ یا اصلاح معتبرند، مشروط به عدم مغایرت بنیادین. پارلمان اول ۵ سال فرصت برای تطبیق کامل قوانین دارد. \\
\hline

\rowcolor{goldlight}
\textbf{اصل ۱۷۹} & \textbf{اداره امور:} حفظ بدنه کارشناسی دولت و امنیت شغلی کارکنانی که در نقض حقوق بشر مشارکت نداشته‌اند. \\
\hline

\textbf{اصل ۱۸۰} & \textbf{عدالت انتقالی:} تشکیل کمیسیون حقیقت و آشتی برای رسیدگی به مظالم گذشته، جبران خسارت قربانیان و پاسخگو کردن ناقضان حقوق مردم. \\
\hline

\rowcolor{goldlight}
\textbf{اصل ۱۸۱} & \textbf{اصلاح نیروهای مسلح:} ادغام نیروها در ساختار واحد تحت فرماندهی دولت فدرال و خروج کامل نظامیان از سیاست و اقتصاد. \\
\hline

\textbf{اصل ۱۸۲} & \textbf{استرداد اموال:} شناسایی و بازگرداندن کلیه اموال عمومی و خصوصی که به ناحق در نظام پیشین تصاحب شده است. \\
\hline

\rowcolor{goldlight}
\textbf{اصل ۱۸۳} & \textbf{جانشینی بین‌المللی:} تعهد به قراردادهای بین‌المللی مطابق با منافع ملی و الحاق فوری به کنوانسیون‌های جهانی حقوق بشر. \\
\hline

\textbf{اصل ۱۸۴} & \textbf{ساختار فدرال:} استقرار تدریجی مدیریت‌های منطقه‌ای و انتقال اختیارات از مرکز به مناطق ظرف ۵ سال. \\
\hline

\rowcolor{goldlight}
\textbf{اصل ۱۸۵} & \textbf{چندزبانی رسمی:} آغاز برنامه ملی آموزش و استانداردسازی زبان‌های اقوام برای کاربرد رسمی در سطوح فدرال و منطقه‌ای. \\
\hline

\textbf{اصل ۱۸۶} & \textbf{انقضا:} این فصل پس از تحقق اهداف گذار یا پایان مهلت ۵ ساله خودبه‌خود از متن قانون اساسی خارج می‌شود. \\
\hline

\rowcolor{goldlight}
\textbf{اصل ۱۸۷} & \textbf{میثاق وفاداری:} این قانون اساسی عهد ملت ایران برای آزادی و عدالت است. هیچ مقامی فراتر از آن نیست و صیانت از آن تکلیف همگانی است. \\
\hline
\end{longtable}

% ═══════════════════════════════════════════════════════════════════════════════
\section{جداول تطبیقی با قوانین اساسی سایر کشورها}
\label{sec:const-comparative}
% ═══════════════════════════════════════════════════════════════════════════════

\subsection{مقایسه ساختار کلی}

\begin{center}
\begin{small}
\begin{longtable}{|>{\columncolor{DemocracyBlue!10}}p{2.5cm}|p{2.2cm}|p{2.2cm}|p{2.2cm}|p{2.2cm}|p{2.2cm}|}
\hline
\rowcolor{DemocracyBlue!30}
\textbf{ویژگی} & \textbf{ایران (پیشنهادی)} & \textbf{آلمان} & \textbf{هند} & \textbf{آفریقای جنوبی} & \textbf{اسپانیا} \\
\hline
\endfirsthead
\hline
\rowcolor{DemocracyBlue!30}
\textbf{ویژگی} & \textbf{ایران (پیشنهادی)} & \textbf{آلمان} & \textbf{هند} & \textbf{آفریقای جنوبی} & \textbf{اسپانیا} \\
\hline
\endhead

نوع نظام & جمهوری فدرال & جمهوری فدرال & جمهوری فدرال & جمهوری واحد & پادشاهی مشروطه \\
\hline

رئیس کشور & رئیس‌جمهور منتخب & رئیس‌جمهور (غیرمستقیم) & رئیس‌جمهور (غیرمستقیم) & رئیس‌جمهور (پارلمانی) & پادشاه \\
\hline

رئیس دولت & رئیس‌جمهور & صدراعظم & نخست‌وزیر & رئیس‌جمهور & نخست‌وزیر \\
\hline

پارلمان & دومجلسی & دومجلسی & دومجلسی & دومجلسی & دومجلسی \\
\hline

تعداد واحدهای فدرال & ۵ منطقه + ۱۵ استان & ۱۶ لَند & ۲۸ ایالت + ۸ منطقه & ۹ استان & ۱۷ منطقه خودمختار \\
\hline

دادگاه قانون اساسی & بله (۱۵ قاضی) & بله (۱۶ قاضی) & بله (دیوان عالی) & بله (۱۱ قاضی) & بله (۱۲ قاضی) \\
\hline

اصول غیرقابل تغییر & بله (۷ مورد) & بله (۲ مورد) & خیر (رویه قضایی) & بله (ارزش‌های بنیادین) & خیر \\
\hline

جدایی دین و دولت & کامل & بله & سکولار & بله & بله \\
\hline

\end{longtable}
\end{small}
\end{center}

\subsection{مقایسه حقوق بنیادین}

\begin{center}
\begin{small}
\begin{longtable}{|>{\columncolor{SuccessGreen!10}}p{3cm}|c|c|c|c|c|}
\hline
\rowcolor{SuccessGreen!30}
\textbf{حق بنیادین} & \textbf{ایران} & \textbf{آلمان} & \textbf{هند} & \textbf{آفریقا} & \textbf{کانادا} \\
\hline
\endfirsthead
\hline
\rowcolor{SuccessGreen!30}
\textbf{حق بنیادین} & \textbf{ایران} & \textbf{آلمان} & \textbf{هند} & \textbf{آفریقا} & \textbf{کانادا} \\
\hline
\endhead

لغو مجازات اعدام & ✓ & ✓ & ✗ & ✓ & ✓ \\
\hline

آزادی دین و تغییر دین & ✓ & ✓ & ✓ & ✓ & ✓ \\
\hline

برابری جنسیتی صریح & ✓ & ✓ & ✓ & ✓ & ✓ \\
\hline

حقوق زبانی اقلیت‌ها & ✓✓ & ✗ & ✓✓ & ✓✓ & ✓✓ \\
\hline

حق آب & ✓ & ✗ & ✗ & ✓ & ✗ \\
\hline

حق محیط زیست سالم & ✓ & ✓ & ✓ & ✓ & ✗ \\
\hline

حق مسکن & ✓ & ✗ & ✗ & ✓ & ✗ \\
\hline

ممنوعیت تبعیض گرایش جنسی & ✓ & ✓ & ✓ & ✓ & ✓ \\
\hline

حق دسترسی به اطلاعات & ✓ & ✓ & ✓ & ✓ & ✓ \\
\hline

حفاظت از افشاگران & ✓ & ✓ & ✗ & ✓ & ✓ \\
\hline

\multicolumn{6}{l}{\scriptsize ✓✓ = حمایت ویژه | ✓ = تضمین‌شده | ✗ = بدون تضمین صریح} \\

\end{longtable}
\end{small}
\end{center}

\subsection{مقایسه ساختار فدرال}

\begin{center}
\begin{tikzpicture}[scale=0.9]
    % محور عمودی - درجه تمرکززدایی
    \draw[->, thick] (0,0) -- (0,7) node[above, font=\small, align=center] {تمرکززدایی\\بیشتر};
    
    % محور افقی - تنوع قومی
    \draw[->, thick] (0,0) -- (12,0) node[right, font=\small, align=center] {تنوع قومی\\بیشتر};
    
    % کشورها
    \node[circle, draw=DemocracyBlue, fill=DemocracyBlue!30, minimum size=1.2cm, font=\tiny] at (8,5.5) {ایران};
    \node[circle, draw=SuccessGreen, fill=SuccessGreen!30, minimum size=1cm, font=\tiny] at (3,5) {آلمان};
    \node[circle, draw=WisdomGold, fill=WisdomGold!30, minimum size=1cm, font=\tiny] at (10,4.5) {هند};
    \node[circle, draw=purple, fill=purple!30, minimum size=1cm, font=\tiny] at (7,3.5) {اسپانیا};
    \node[circle, draw=orange, fill=orange!30, minimum size=1cm, font=\tiny] at (9,4) {آفریقا};
    \node[circle, draw=WarningRed, fill=WarningRed!30, minimum size=1cm, font=\tiny] at (5,6) {سوئیس};
    \node[circle, draw=gray, fill=gray!30, minimum size=1cm, font=\tiny] at (4,2) {فرانسه};
    
    % راهنما
    \node[font=\scriptsize, align=right] at (2,6.5) {فدرال};
    \node[font=\scriptsize, align=right] at (2,1) {متمرکز};
\end{tikzpicture}
\end{center}

\subsection{مقایسه فرآیند بازنگری قانون اساسی}

\begin{center}
\begin{small}
\begin{longtable}{|>{\columncolor{bleulight}}r|p{3.5cm}|p{3.5cm}|p{2cm}|p{3cm}|}
\hline
\rowcolor{bleurepublique!20}
\headmark کشور & \headmark پیشنهاددهنده & \headmark تصویب پارلمان & \headmark همه‌پرسی & \headmark اصول غیرقابل تغییر \\
\hline
\endfirsthead
\hline
\rowcolor{bleurepublique!20}
\headmark کشور & \headmark پیشنهاددهنده & \headmark تصویب پارلمان & \headmark همه‌پرسی & \headmark اصول غیرقابل تغییر \\
\hline
\endhead

\textbf{ایران (پیشنهادی)} & دوسوم هر مجلس یا ۲ میلیون امضا & سه‌چهارم هر دو مجلس & اجباری & کرامت انسانی، دموکراسی، فدرالیسم \\
\hline

\rowcolor{goldlight}
\textbf{آلمان} & دوسوم هر مجلس & دوسوم هر دو مجلس & خیر & کرامت انسانی، ساختار فدرال \\
\hline

\textbf{فرانسه} & رئیس‌جمهور یا پارلمان & سه‌پنجم کنگره & اختیاری & شکل جمهوری \\
\hline

\rowcolor{goldlight}
\textbf{سوئیس} & پارلمان یا ۱۰۰,۰۰۰ امضا & اکثریت ساده & اجباری & خیر \\
\hline
\end{longtable}
\end{small}
\end{center}

% ═══════════════════════════════════════════════════════════════════════════════
\section{نمودار کلی ساختار حکومت}
\label{sec:const-diagram}
% ═══════════════════════════════════════════════════════════════════════════════

\begin{figure}[htbp]
\centering
\begin{tikzpicture}[
    scale=0.8,
    transform shape,
    box/.style={
        rectangle,
        rounded corners=8pt,
        draw=bleurepublique,
        fill=bleulight,
        line width=1.5pt,
        minimum width=3cm,
        minimum height=1.2cm,
        text centered,
        font=\small\bfseries
    },
    nationnode/.style={
        rectangle,
        rounded corners=15pt,
        draw=goldphoenix,
        fill=goldlight,
        line width=3pt,
        minimum width=14cm,
        minimum height=1.6cm,
        font=\Large\bfseries
    },
    arrow/.style={->, ultra thick, bleurepublique!30, >=stealth}
]
\node[nationnode] (nation) at (0,8) {\rl{ملت ایران — منشأ حاکمیت}};

\node[box] (nat) at (-5,5.5) {\rl{مجلس ملی}};
\node[box] (sen) at (-1.5,5.5) {\rl{مجلس اقوام}};
\node[box, draw=goldphoenix, fill=goldlight] (pres) at (2,5.5) {\rl{رئیس‌جمهور}};
\node[box] (cc) at (5.5,5.5) {\rl{دیوان قانون اساسی}};

\node[box, minimum height=1cm] (reg) at (0,3) {\rl{دولت‌های منطقه‌ای و استانی}};
\node[box, minimum height=1cm] (ind) at (0,1) {\rl{نهادهای مستقل نظارتی (رکن چهارم)}};

\draw[arrow] (nation) -- (nat);
\draw[arrow] (nation) -- (sen);
\draw[arrow] (nation) -- (pres);
\draw[arrow] (nation) to[bend left] (cc);
\draw[arrow] (nation) to[bend right] (reg);
\draw[arrow] (cc) -- (nat);
\draw[arrow] (cc) -- (pres);

\end{tikzpicture}
\caption{ساختار کلان حکومت جمهوری فدرال ایران (تفکیک قوا و تعادل فدرال)}
\label{fig:gov-overall}
\end{figure}

% ═══════════════════════════════════════════════════════════════════════════════
\section{خلاصه و جمع‌بندی}
\label{sec:const-summary}
% ═══════════════════════════════════════════════════════════════════════════════

\begin{kholasebox}
\textbf{خلاصه قانون اساسی پیشنهادی جمهوری فدرال ایران}

\begin{center}
\begin{longtable}{|>{\columncolor{bleulight}}r|p{8cm}|}
\hline
\rowcolor{bleurepublique!20}
\headmark شاخص ساختاری & \headmark جزئیات و آمار کلی \\
\hline
\endfirsthead
\hline
\rowcolor{bleurepublique!20}
\headmark شاخص ساختاری & \headmark جزئیات و آمار کلی \\
\hline
\endhead

\textbf{تعداد کل اصول} & ۱۸۷ اصل در ۸ فصل \\
\hline

\rowcolor{goldlight}
\textbf{اصول غیرقابل تغییر} & ۷ اصل (پاسدار هویت نظام) \\
\hline

\textbf{حقوق بنیادین} & ۷۴ اصل (جامع‌ترین منشور حقوق منطقه) \\
\hline

\rowcolor{goldlight}
\textbf{واحدهای فدرال} & ۵ منطقه خودمختار و ۱۵ استان فدرال \\
\hline

\textbf{نهادهای مستقل} & ۸ سازمان نظارتی رکن چهارم \\
\hline

\rowcolor{goldlight}
\textbf{دوره انتقالی} & حداکثر ۱۸ ماه برای استقرار نهادها \\
\hline
\end{longtable}
\end{center}

\textbf{ویژگی‌های متمایز این قانون اساسی:}

\begin{enumerate}[nosep]
    \item \textbf{جدایی کامل دین از دولت} — برخلاف قانون اساسی فعلی
    \item \textbf{فدرالیسم همبسته} — پاسخ به مطالبات قومی بدون تجزیه
    \item \textbf{مجلس اقوام} — نهاد ویژه نمایندگی تنوع
    \item \textbf{لغو مجازات اعدام} — پیشرو در منطقه
    \item \textbf{حق آب} — پاسخ به بحران آب
    \item \textbf{نهادهای مستقل قوی} — پیشگیری از استبداد
    \item \textbf{اصول غیرقابل تغییر} — خطوط قرمز دموکراسی
    \item \textbf{عدالت انتقالی} — گذار مسالمت‌آمیز اما عادلانه
\end{enumerate}
\end{kholasebox}

\vspace{10pt}

\begin{olgoobox}
\textbf{منابع الهام‌بخش این قانون اساسی:}

\begin{itemize}[nosep]
    \item قانون اساسی آلمان (۱۹۴۹): ساختار فدرال، دادگاه قانون اساسی، اصول غیرقابل تغییر
    \item قانون اساسی آفریقای جنوبی (۱۹۹۶): منشور حقوق، عدالت انتقالی، حقوق اقتصادی-اجتماعی
    \item قانون اساسی هند (۱۹۵۰): مدیریت تنوع، فدرالیسم نامتقارن
    \item قانون اساسی اسپانیا (۱۹۷۸): مناطق خودمختار، گذار دموکراتیک
    \item قانون اساسی سوئیس (۱۹۹۹): دموکراسی مستقیم، فدرالیسم زبانی
    \item منشور حقوق بشر اروپا و میثاقین بین‌المللی حقوق بشر
\end{itemize}
\end{olgoobox}

% ═══════════════════════════════════════════════════════════════════════════════
% پایان پیوست ۱
% ═══════════════════════════════════════════════════════════════════════════════