% ═══════════════════════════════════════════════════════════════════════════════
% پیوست ۴: داده‌ها و جداول آماری
% فایل: app04-data.tex
% ═══════════════════════════════════════════════════════════════════════════════

\chapter{داده‌ها و جداول آماری}
\label{app:data}

\begin{kholasebox}
این پیوست مجموعه‌ای جامع از داده‌های آماری، جداول مقایسه‌ای، و نمودارهای تحلیلی را ارائه می‌دهد که پایه علمی و تجربی طرح گذار دموکراتیک را تشکیل می‌دهند. داده‌ها از منابع معتبر بین‌المللی (بانک جهانی، صندوق بین‌المللی پول، سازمان ملل، و مراکز پژوهشی) گردآوری شده‌اند. این اطلاعات برای برنامه‌ریزی، تصمیم‌گیری و پایش پیشرفت ضروری است.
\end{kholasebox}

% ═══════════════════════════════════════════════════════════════════════════════
\section{بخش اول: داده‌های جمعیتی و قومی}
\label{sec:data-demographic}
% ═══════════════════════════════════════════════════════════════════════════════

\subsection{جدول ۱: توزیع جمعیت بر اساس قومیت}

\begin{center}
\begin{tikzpicture}
    % نمودار دایره‌ای
    \pie[
        text=legend,
        radius=3,
        color={bleurepublique!70, goldlight!70, bleulight!70, golddark!70, bleurepublique!50, goldlight!50, bleulight!50, gray!50}
    ]{
        55/\rl{فارس},
        20/\rl{آذری},
        10/\rl{کرد},
        6/\rl{لر},
        3/\rl{عرب},
        2/\rl{بلوچ},
        1/\rl{ترکمن},
        3/\rl{سایر}
    }
    
    \node[font=\large\bfseries] at (0,-4.5) {توزیع قومی جمعیت ایران (تخمین ۱۴۰۳)};
\end{tikzpicture}
\end{center}

\begin{center}
\begin{small}
\begin{longtable}{|>{\columncolor{bleurepublique!10}}r|r|r|r|r|r|}
\hline
\rowcolor{bleurepublique!30}
\textbf{\rl{قوم}} & \textbf{\rl{جمعیت (میلیون)}} & \textbf{\rl{درصد}} & \textbf{\rl{زبان اصلی}} & \textbf{\rl{دین غالب}} & \textbf{\rl{استان‌های اصلی}} \\
\hline
\endfirsthead
\hline
\rowcolor{bleurepublique!30}
\textbf{\rl{قوم}} & \textbf{\rl{جمعیت (میلیون)}} & \textbf{\rl{درصد}} & \textbf{\rl{زبان اصلی}} & \textbf{\rl{دین غالب}} & \textbf{\rl{استان‌های اصلی}} \\
\hline
\endhead

فارس & ۴۷.۳ & ۵۵٪ & فارسی & شیعه & تهران، اصفهان، فارس، خراسان \\
\hline

آذری & ۱۷.۲ & ۲۰٪ & ترکی آذربایجانی & شیعه & آذربایجان شرقی و غربی، اردبیل، زنجان \\
\hline

کرد & ۸.۶ & ۱۰٪ & کردی & سنی/شیعه & کردستان، کرمانشاه، ایلام، آ.غربی \\
\hline

لر & ۵.۲ & ۶٪ & لری/بختیاری & شیعه & لرستان، چهارمحال، کهگیلویه \\
\hline

عرب & ۲.۶ & ۳٪ & عربی & شیعه & خوزستان \\
\hline

بلوچ & ۱.۷ & ۲٪ & بلوچی & سنی & سیستان و بلوچستان \\
\hline

ترکمن & ۰.۸۶ & ۱٪ & ترکمنی & سنی & گلستان \\
\hline

گیلک & ۲.۶ & ۳٪ & گیلکی & شیعه & گیلان \\
\hline

مازندرانی & ۲.۶ & ۳٪ & مازندرانی & شیعه & مازندران \\
\hline

سایر & ۱.۷ & ۲٪ & متنوع & متنوع & پراکنده \\
\hline

\rowcolor{gray!20}
\textbf{مجموع} & \textbf{۸۶.۰} & \textbf{۱۰۰٪} & — & — & — \\
\hline

\multicolumn{6}{l}{\scriptsize منبع: تخمین بر اساس داده‌های مرکز آمار ایران و مطالعات جمعیت‌شناختی | ارقام تقریبی است} \\

\end{longtable}
\end{small}
\end{center}

\subsection{جدول ۲: توزیع جمعیت بر اساس استان}

\begin{center}
\begin{small}
\begin{longtable}{|r|r|r|r|r|r|}
\hline
\rowcolor{goldlight!30}
\textbf{\rl{ردیف}} & \textbf{\rl{استان}} & \textbf{\rl{جمعیت (میلیون)}} & \textbf{\rl{مساحت (کیلومتر²)}} & \textbf{\rl{تراکم}} & \textbf{\rl{قوم غالب}} \\
\hline
\endfirsthead

۱ & تهران & ۱۴.۲ & ۱۳,۶۹۲ & ۱,۰۳۷ & فارس \\
\hline
۲ & خراسان رضوی & ۶.۸ & ۱۱۸,۸۵۴ & ۵۷ & فارس \\
\hline
۳ & اصفهان & ۵.۴ & ۱۰۷,۰۲۹ & ۵۰ & فارس \\
\hline
۴ & فارس & ۵.۰ & ۱۲۲,۶۰۸ & ۴۱ & فارس \\
\hline
۵ & آذربایجان شرقی & ۴.۰ & ۴۵,۶۵۱ & ۸۸ & آذری \\
\hline
۶ & خوزستان & ۴.۸ & ۶۴,۰۵۵ & ۷۵ & عرب/فارس/لر \\
\hline
۷ & آذربایجان غربی & ۳.۳ & ۳۷,۴۱۱ & ۸۸ & آذری/کرد \\
\hline
۸ & کرمان & ۳.۳ & ۱۸۰,۷۲۶ & ۱۸ & فارس \\
\hline
۹ & مازندران & ۳.۴ & ۲۳,۸۳۳ & ۱۴۳ & مازندرانی \\
\hline
۱۰ & گیلان & ۲.۶ & ۱۴,۰۴۲ & ۱۸۵ & گیلک \\
\hline
۱۱ & کردستان & ۱.۷ & ۲۹,۱۳۷ & ۵۸ & کرد \\
\hline
۱۲ & کرمانشاه & ۲.۰ & ۲۵,۰۰۹ & ۸۰ & کرد \\
\hline
۱۳ & سیستان و بلوچستان & ۲.۹ & ۱۸۰,۷۲۶ & ۱۶ & بلوچ \\
\hline
۱۴ & لرستان & ۱.۹ & ۲۸,۵۵۹ & ۶۷ & لر \\
\hline
۱۵ & گلستان & ۱.۹ & ۲۰,۱۹۵ & ۹۴ & فارس/ترکمن \\
\hline

\rowcolor{gray!20}
\multicolumn{2}{|r|}{\textbf{مجموع ایران}} & \textbf{۸۶.۰} & \textbf{۱,۶۴۸,۱۹۵} & \textbf{۵۲} & — \\
\hline

\end{longtable}
\end{small}
\end{center}

\subsection{جدول ۳: شاخص‌های توسعه انسانی به تفکیک منطقه}

\begin{center}
\begin{tikzpicture}
    % نمودار میله‌ای افقی
    \begin{axis}[
        xbar,
        width=14cm,
        height=10cm,
        xlabel={شاخص توسعه انسانی (HDI)},
        symbolic y coords={بلوچستان, کردستان, خوزستان, آذربایجان, لرستان, گیلان, فارس, اصفهان, تهران},
        ytick=data,
        xmin=0.5, xmax=0.9,
        bar width=12pt,
        nodes near coords,
        nodes near coords align={horizontal},
        every node near coord/.append style={font=\tiny},
    ]
    \addplot[fill=DemocracyBlue!70] coordinates {
        (0.82,تهران)
        (0.78,اصفهان)
        (0.76,فارس)
        (0.74,گیلان)
        (0.70,لرستان)
        (0.68,آذربایجان)
        (0.65,خوزستان)
        (0.62,کردستان)
        (0.55,بلوچستان)
    };
    \end{axis}
    
    \node[font=\small\bfseries] at (7,-1) {شاخص توسعه انسانی مناطق ایران (تخمین ۱۴۰۲)};
\end{tikzpicture}
\end{center}

\begin{center}
\begin{small}
\begin{longtable}{|>{\columncolor{bleurepublique!10}}p{3cm}|r|r|r|r|r|}
\hline
\rowcolor{bleurepublique!30}
\textbf{\rl{منطقه/استان}} & \textbf{\rl{HDI}} & \textbf{\rl{امید زندگی}} & \textbf{\rl{سواد (٪)}} & \textbf{\rl{درآمد سرانه (\$)}} & \textbf{\rl{شکاف با میانگین}} \\
\hline
\endfirsthead

تهران & ۰.۸۲ & ۷۷.۵ & ۹۸٪ & ۱۸,۵۰۰ & +۱۵٪ \\
\hline
اصفهان & ۰.۷۸ & ۷۶.۲ & ۹۶٪ & ۱۴,۲۰۰ & +۹٪ \\
\hline
فارس & ۰.۷۶ & ۷۵.۸ & ۹۵٪ & ۱۲,۸۰۰ & +۶٪ \\
\hline
\rowcolor{yellow!20}
\textbf{میانگین ملی} & \textbf{۰.۷۱} & \textbf{۷۴.۵} & \textbf{۹۲٪} & \textbf{۱۱,۰۰۰} & \textbf{—} \\
\hline
آذربایجان شرقی & ۰.۶۸ & ۷۴.۰ & ۹۰٪ & ۹,۸۰۰ & -۵٪ \\
\hline
کردستان & ۰.۶۲ & ۷۲.۵ & ۸۵٪ & ۷,۵۰۰ & -۱۳٪ \\
\hline
خوزستان & ۰.۶۵ & ۷۳.۰ & ۸۷٪ & ۸,۲۰۰ & -۹٪ \\
\hline
سیستان و بلوچستان & ۰.۵۵ & ۶۹.۵ & ۷۵٪ & ۴,۵۰۰ & -۲۳٪ \\
\hline

\end{longtable}
\end{small}
\end{center}

% ═══════════════════════════════════════════════════════════════════════════════
\section{بخش دوم: داده‌های اقتصادی}
\label{sec:data-economic}
% ═══════════════════════════════════════════════════════════════════════════════

\subsection{جدول ۴: شاخص‌های کلان اقتصادی ایران}

\begin{center}
\begin{small}
\begin{longtable}{|>{\columncolor{goldlight!10}}p{4cm}|r|r|r|r|r|}
\hline
\rowcolor{goldlight!30}
\textbf{\rl{شاخص}} & \textbf{\rl{۱۳۹۸}} & \textbf{\rl{۱۳۹۹}} & \textbf{\rl{۱۴۰۰}} & \textbf{\rl{۱۴۰۱}} & \textbf{\rl{۱۴۰۲ (تخمین)}} \\
\hline
\endfirsthead

تولید ناخالص داخلی (میلیارد \$) & ۴۴۵ & ۲۳۱ & ۳۵۹ & ۳۸۸ & ۴۰۱ \\
\hline

رشد اقتصادی (٪) & -۶.۸ & ۳.۳ & ۴.۷ & ۲.۹ & ۲.۵ \\
\hline

تورم سالانه (٪) & ۴۱.۲ & ۳۶.۴ & ۴۰.۲ & ۴۵.۸ & ۴۲.۵ \\
\hline

نرخ بیکاری (٪) & ۱۰.۷ & ۹.۴ & ۹.۲ & ۸.۹ & ۹.۱ \\
\hline

صادرات نفت (میلیون بشکه/روز) & ۰.۴ & ۰.۵ & ۰.۷ & ۱.۰ & ۱.۴ \\
\hline

درآمد نفتی (میلیارد \$) & ۸ & ۱۲ & ۲۵ & ۴۲ & ۵۵ \\
\hline

ذخایر ارزی (میلیارد \$) & ۸۶ & ۱۱۵ & ۱۲۲ & ۱۴۲ & ۱۵۵ \\
\hline

بدهی خارجی (میلیارد \$) & ۹.۳ & ۸.۷ & ۸.۲ & ۷.۸ & ۷.۵ \\
\hline

نرخ ارز (تومان/دلار) & ۱۱,۵۰۰ & ۲۵,۰۰۰ & ۲۷,۵۰۰ & ۴۲,۰۰۰ & ۵۵,۰۰۰ \\
\hline

\end{longtable}
\end{small}
\end{center}

\subsection{جدول ۵: ترکیب اقتصاد ایران}

\begin{center}
\begin{tikzpicture}
    % نمودار ستونی
    \begin{axis}[
        ybar,
        width=14cm,
        height=8cm,
        ylabel={درصد از GDP},
        symbolic x coords={نفت و گاز, صنعت, خدمات, کشاورزی, معدن, ساختمان},
        xtick=data,
        ymin=0, ymax=50,
        bar width=20pt,
        nodes near coords,
        every node near coord/.append style={font=\small},
        legend style={at={(0.5,-0.2)}, anchor=north},
    ]
    \addplot[fill=DemocracyBlue!70] coordinates {
        (نفت و گاز,18)
        (صنعت,22)
        (خدمات,45)
        (کشاورزی,9)
        (معدن,2)
        (ساختمان,4)
    };
    \end{axis}
    
    \node[font=\small\bfseries] at (7,-1.5) {ترکیب تولید ناخالص داخلی ایران (۱۴۰۲)};
\end{tikzpicture}
\end{center}

\subsection{جدول ۶: تحریم‌های بین‌المللی علیه ایران}

\begin{center}
\begin{small}
\begin{longtable}{|>{\columncolor{golddark!10}}p{2.5cm}|r|p{4cm}|p{5cm}|}
\hline
\rowcolor{golddark!30}
\textbf{\rl{منبع تحریم}} & \textbf{\rl{تعداد}} & \textbf{\rl{حوزه‌های اصلی}} & \textbf{\rl{تأثیر اقتصادی}} \\
\hline
\endfirsthead

آمریکا & ۳,۲۰۰+ & نفت، بانکی، تسلیحاتی، هسته‌ای، حقوق بشر & قطع از سوئیفت، محدودیت صادرات نفت، بلوکه دارایی‌ها \\
\hline

اتحادیه اروپا & ۴۵۰+ & نفت، بانکی، تسلیحاتی، حقوق بشر & محدودیت تجارت، ممنوعیت سرمایه‌گذاری \\
\hline

سازمان ملل & ۱۲۰+ & هسته‌ای، تسلیحاتی & ممنوعیت انتقال فناوری، تسلیحات \\
\hline

بریتانیا & ۲۸۰+ & مشابه EU + اقدامات مستقل & محدودیت‌های مالی \\
\hline

کانادا & ۱۵۰+ & حقوق بشر، هسته‌ای & محدودیت‌های تجاری و مالی \\
\hline

استرالیا & ۱۰۰+ & هسته‌ای، حقوق بشر & محدودیت‌های تجاری \\
\hline

\rowcolor{gray!20}
\textbf{مجموع (تخمین)} & \textbf{۳,۸۰۰+} & — & \textbf{خسارت سالانه: ۱۵۰-۲۰۰ میلیارد \$} \\
\hline

\end{longtable}
\end{small}
\end{center}

\subsection{جدول ۷: سناریوهای اقتصادی پس از رفع تحریم}

\begin{center}
\begin{small}
\begin{longtable}{|>{\columncolor{goldlight!10}}p{3cm}|p{3.5cm}|p{3.5cm}|p{3.5cm}|}
\hline
\rowcolor{goldlight!30}
\textbf{\rl{شاخص}} & \textbf{\rl{سناریوی حداقلی}} & \textbf{\rl{سناریوی میانه}} & \textbf{\rl{سناریوی حداکثری}} \\
\hline
\endfirsthead

\textbf{فرض پایه} & رفع جزئی تحریم‌ها & رفع اکثر تحریم‌ها & رفع کامل + اصلاحات \\
\hline

صادرات نفت (م.ب/روز) & ۲.۰ & ۳.۰ & ۴.۰ \\
\hline

درآمد نفتی (میلیارد \$/سال) & ۸۰ & ۱۲۰ & ۱۶۰ \\
\hline

رشد GDP (سال ۱-۵) & ۳-۴٪ & ۵-۷٪ & ۸-۱۰٪ \\
\hline

سرمایه‌گذاری خارجی (میلیارد \$/سال) & ۵-۱۰ & ۱۵-۲۵ & ۳۰-۵۰ \\
\hline

کاهش تورم (سال ۵) & ۲۵-۳۰٪ & ۱۵-۲۰٪ & ۸-۱۲٪ \\
\hline

کاهش بیکاری (سال ۵) & ۷-۸٪ & ۵-۶٪ & ۳-۴٪ \\
\hline

GDP سرانه (سال ۱۰) & ۱۵,۰۰۰ \$ & ۲۰,۰۰۰ \$ & ۲۵,۰۰۰ \$ \\
\hline

\end{longtable}
\end{small}
\end{center}

% ═══════════════════════════════════════════════════════════════════════════════
\section{بخش سوم: داده‌های بحران آب}
\label{sec:data-water}
% ═══════════════════════════════════════════════════════════════════════════════

\subsection{جدول ۸: بیلان آبی ایران}

\begin{center}
\begin{tikzpicture}
    % نمودار تعادل آبی
    \node[
        draw=DemocracyBlue,
        line width=2pt,
        fill=DemocracyBlue!20,
        rounded corners=10pt,
        minimum width=5cm,
        minimum height=2cm,
        font=\bfseries
    ] (supply) at (-4,0) {\shortstack{عرضه آب\\۹۷ میلیارد م³}};
    
    \node[
        draw=WarningRed,
        line width=2pt,
        fill=WarningRed!20,
        rounded corners=10pt,
        minimum width=5cm,
        minimum height=2cm,
        font=\bfseries
    ] (demand) at (4,0) {\shortstack{تقاضای آب\\۱۱۵ میلیارد م³}};
    
    \node[
        draw=WarningRed,
        line width=3pt,
        fill=WarningRed!40,
        rounded corners=10pt,
        minimum width=5cm,
        minimum height=1.5cm,
        font=\bfseries
    ] (deficit) at (0,-3) {\shortstack{کسری سالانه\\۱۸ میلیارد م³}};
    
    \draw[->, thick, DemocracyBlue] (supply) -- (deficit);
    \draw[->, thick, WarningRed] (demand) -- (deficit);
    
    % جزئیات
    \node[font=\scriptsize, align=center] at (-4,-1.5) {بارش: ۴۱۵ میلیمتر\\تبخیر: ۷۰٪\\رواناب: ۹۷ م³};
    
    \node[font=\scriptsize, align=center] at (4,-1.5) {کشاورزی: ۹۲٪\\صنعت: ۳٪\\شرب: ۵٪};
\end{tikzpicture}
\end{center}

\begin{center}
\begin{small}
\begin{longtable}{|>{\columncolor{bleurepublique!10}}p{4cm}|r|r|r|}
\hline
\rowcolor{bleurepublique!30}
\textbf{\rl{شاخص}} & \textbf{\rl{مقدار}} & \textbf{\rl{واحد}} & \textbf{\rl{وضعیت}} \\
\hline
\endfirsthead

بارش سالانه متوسط & ۲۵۰ & میلیمتر & ۱/۳ میانگین جهانی \\
\hline

منابع آب تجدیدپذیر & ۱۳۰ & میلیارد م³/سال & محدود \\
\hline

منابع آب قابل استحصال & ۹۷ & میلیارد م³/سال & — \\
\hline

مصرف کل آب & ۱۱۵ & میلیارد م³/سال & بیش از ظرفیت \\
\hline

\rowcolor{WarningRed!20}
کسری سالانه & ۱۸ & میلیارد م³/سال & بحرانی \\
\hline

برداشت از آب‌های زیرزمینی & ۶۵ & میلیارد م³/سال & ۱.۵ برابر ظرفیت \\
\hline

افت سالانه سطح آب زیرزمینی & ۵۰-۱۰۰ & سانتیمتر & بحرانی \\
\hline

تعداد دشت‌های ممنوعه & ۴۰۵ & از ۶۰۹ دشت & ۶۶٪ \\
\hline

سرانه آب تجدیدپذیر & ۱,۵۰۰ & م³/نفر/سال & کم‌آبی \\
\hline

\end{longtable}
\end{small}
\end{center}

\subsection{جدول ۹: وضعیت حوضه‌های آبریز}

\begin{center}
\begin{small}
\begin{longtable}{|>{\columncolor{bleurepublique!10}}r|p{3cm}|r|r|r|p{2.5cm}|}
\hline
\rowcolor{bleurepublique!30}
\textbf{\rl{ردیف}} & \textbf{\rl{حوضه آبریز}} & \textbf{\rl{مساحت (کیلومتر²)}} & \textbf{\rl{جمعیت (میلیون)}} & \textbf{\rl{کسری (م³/سال)}} & \textbf{\rl{وضعیت}} \\
\hline
\endfirsthead

۱ & دریاچه ارومیه & ۵۱,۸۷۶ & ۶.۴ & ۳.۵ میلیارد & \cellcolor{WarningRed!30}بحرانی \\
\hline

۲ & زاینده‌رود & ۴۱,۵۰۰ & ۵.۲ & ۲.۱ میلیارد & \cellcolor{WarningRed!30}بحرانی \\
\hline

۳ & کارون & ۶۵,۲۳۰ & ۴.۸ & ۱.۸ میلیارد & \cellcolor{orange!30}نگران‌کننده \\
\hline

۴ & هامون & ۱۰۶,۰۰۰ & ۲.۹ & ۲.۵ میلیارد & \cellcolor{WarningRed!30}بحرانی \\
\hline

۵ & قره‌سو & ۵۰,۲۰۰ & ۳.۱ & ۱.۲ میلیارد & \cellcolor{orange!30}نگران‌کننده \\
\hline

۶ & مرکزی & ۴۲۵,۰۰۰ & ۲۵.۰ & ۴.۵ میلیارد & \cellcolor{WarningRed!30}بحرانی \\
\hline

۷ & خلیج فارس & ۲۳۵,۰۰۰ & ۱۲.۵ & ۲.۰ میلیارد & \cellcolor{WisdomGold!30}متوسط \\
\hline

۸ & دریای خزر & ۱۸۰,۰۰۰ & ۱۵.۲ & ۰.۵ میلیارد & \cellcolor{SuccessGreen!30}نسبتاً خوب \\
\hline

\end{longtable}
\end{small}
\end{center}

\subsection{جدول ۱۰: برنامه شیرین‌سازی آب دریا}

\begin{center}
\begin{small}
\begin{longtable}{|>{\columncolor{goldlight!10}}p{3cm}|r|r|r|r|}
\hline
\rowcolor{goldlight!30}
\textbf{\rl{پروژه/منطقه}} & \textbf{\rl{ظرفیت (م³/روز)}} & \textbf{\rl{هزینه (میلیارد \$)}} & \textbf{\rl{زمان اجرا}} & \textbf{\rl{جمعیت بهره‌مند}} \\
\hline
\endfirsthead

بندرعباس-کرمان & ۴۰۰,۰۰۰ & ۲.۵ & ۱۴۰۵-۱۴۰۸ & ۳ میلیون \\
\hline

بوشهر-فارس & ۵۰۰,۰۰۰ & ۳.۲ & ۱۴۰۵-۱۴۰۹ & ۴ میلیون \\
\hline

چابهار-سیستان & ۳۰۰,۰۰۰ & ۲.۰ & ۱۴۰۶-۱۴۱۰ & ۲.۵ میلیون \\
\hline

خوزستان ساحلی & ۶۰۰,۰۰۰ & ۳.۸ & ۱۴۰۵-۱۴۱۰ & ۵ میلیون \\
\hline

خلیج فارس-یزد & ۳۵۰,۰۰۰ & ۲.۸ & ۱۴۰۷-۱۴۱۱ & ۳ میلیون \\
\hline

\rowcolor{yellow!20}
\textbf{مجموع فاز اول} & \textbf{۲,۱۵۰,۰۰۰} & \textbf{۱۴.۳} & \textbf{۱۴۰۵-۱۴۱۲} & \textbf{۱۷.۵ میلیون} \\
\hline

\end{longtable}
\end{small}
\end{center}

% ═══════════════════════════════════════════════════════════════════════════════
\section{بخش چهارم: شاخص‌های حکمرانی و دموکراسی}
\label{sec:data-governance}
% ═══════════════════════════════════════════════════════════════════════════════

\subsection{جدول ۱۱: رتبه ایران در شاخص‌های بین‌المللی}

\begin{center}
\begin{small}
\begin{longtable}{|>{\columncolor{bleulight!10}}p{4cm}|r|r|p{3cm}|p{3cm}|}
\hline
\rowcolor{bleulight!30}
\textbf{\rl{شاخص}} & \textbf{\rl{رتبه ایران}} & \textbf{\rl{از کل کشورها}} & \textbf{\rl{وضعیت}} & \textbf{\rl{هدف ۱۰ ساله}} \\
\hline
\endfirsthead

شاخص دموکراسی (EIU) & ۱۵۴ & ۱۶۷ & \cellcolor{WarningRed!30}استبدادی & رتبه ۸۰-۱۰۰ \\
\hline

آزادی مطبوعات (RSF) & ۱۷۶ & ۱۸۰ & \cellcolor{WarningRed!30}بسیار بد & رتبه ۱۰۰-۱۲۰ \\
\hline

شاخص فساد (TI) & ۱۴۹ & ۱۸۰ & \cellcolor{WarningRed!30}فاسد & رتبه ۸۰-۱۰۰ \\
\hline

حاکمیت قانون (WJP) & ۱۱۸ & ۱۴۲ & \cellcolor{WarningRed!30}ضعیف & رتبه ۶۰-۸۰ \\
\hline

آزادی اقتصادی (HF) & ۱۶۹ & ۱۷۶ & \cellcolor{WarningRed!30}سرکوب‌شده & رتبه ۱۰۰-۱۲۰ \\
\hline

توسعه انسانی (UNDP) & ۷۶ & ۱۹۱ & \cellcolor{WisdomGold!30}بالای متوسط & رتبه ۵۰-۶۰ \\
\hline

شاخص صلح جهانی & ۱۴۳ & ۱۶۳ & \cellcolor{WarningRed!30}پرتنش & رتبه ۸۰-۱۰۰ \\
\hline

سهولت کسب‌وکار (WB) & ۱۲۷ & ۱۹۰ & \cellcolor{orange!30}متوسط & رتبه ۵۰-۷۰ \\
\hline

شاخص نوآوری (WIPO) & ۶۲ & ۱۳۲ & \cellcolor{SuccessGreen!30}نسبتاً خوب & رتبه ۴۰-۵۰ \\
\hline

\end{longtable}
\end{small}
\end{center}

\subsection{جدول ۱۲: مقایسه با کشورهای منطقه}

\begin{center}
\begin{small}
\begin{longtable}{|>{\columncolor{bleurepublique!10}}p{2cm}|r|r|r|r|r|r|}
\hline
\rowcolor{bleurepublique!30}
\textbf{\rl{کشور}} & \textbf{\rl{جمعیت (م)}} & \textbf{\rl{GDP (\$ م)}} & \textbf{\rl{سرانه (\$)}} & \textbf{\rl{دموکراسی}} & \textbf{\rl{فساد}} & \textbf{\rl{HDI}} \\
\hline
\endfirsthead

\rowcolor{yellow!15}
ایران & ۸۶ & ۴۰۱ & ۴,۶۵۰ & ۱.۹۵ & ۲۵ & ۰.۷۷۴ \\
\hline

ترکیه & ۸۵ & ۹۰۵ & ۱۰,۶۵۰ & ۴.۳۵ & ۳۶ & ۰.۸۳۸ \\
\hline

عربستان & ۳۶ & ۱,۱۰۸ & ۳۰,۷۵۰ & ۲.۰۸ & ۵۳ & ۰.۸۷۵ \\
\hline

امارات & ۱۰ & ۵۰۷ & ۵۰,۶۵۰ & ۲.۷۶ & ۶۷ & ۰.۹۱۱ \\
\hline

پاکستان & ۲۳۱ & ۳۴۹ & ۱,۵۱۰ & ۴.۱۳ & ۲۷ & ۰.۵۴۴ \\
\hline

عراق & ۴۴ & ۲۶۷ & ۶,۰۷۰ & ۳.۶۲ & ۲۳ & ۰.۶۸۶ \\
\hline

مصر & ۱۱۰ & ۴۷۵ & ۴,۳۲۰ & ۲.۹۳ & ۳۰ & ۰.۷۳۱ \\
\hline

\multicolumn{7}{l}{\scriptsize منابع: بانک جهانی، EIU، Transparency International، UNDP | داده‌های ۲۰۲۳} \\

\end{longtable}
\end{small}
\end{center}

\subsection{جدول ۱۳: اهداف KPI برای دوره گذار}

\begin{center}
\begin{tikzpicture}
    % عنوان
    \node[
        fill=bleurepublique,
        text=white,
        rounded corners=5pt,
        font=\bfseries,
        minimum width=14cm
    ] at (0,6) {\rl{شاخص‌های کلیدی عملکرد (KPI) — اهداف ۲۵ ساله}};
    
    % جدول بصری
    \foreach \y/\name/\current/\y5/\y10/\y25 in {
        4.5/{شاخص دموکراسی}/۱.۹۵/۴.۵/۶.۵/۸.۰,
        3.5/{شاخص فساد}/۲۵/۳۵/۵۰/۷۰,
        2.5/{آزادی مطبوعات}/۱۷۶/۱۲۰/۸۰/۵۰,
        1.5/{HDI}/۰.۷۷/۰.۸۰/۰.۸۵/۰.۹۰,
        0.5/{GDP سرانه (\$)}/۴,۶۵۰/۸,۰۰۰/۱۵,۰۰۰/۳۰,۰۰۰
    } {
        % نام شاخص
        \node[font=\small, align=left] at (-5.5,\y) {\name};
        
        % مقادیر
        \node[font=\scriptsize, fill=WarningRed!30, rounded corners=2pt, minimum width=1.5cm] at (-2.5,\y) {\current};
        \node[font=\scriptsize, fill=orange!30, rounded corners=2pt, minimum width=1.5cm] at (-0.5,\y) {\y5};
        \node[font=\scriptsize, fill=WisdomGold!30, rounded corners=2pt, minimum width=1.5cm] at (1.5,\y) {\y10};
        \node[font=\scriptsize, fill=SuccessGreen!30, rounded corners=2pt, minimum width=1.5cm] at (3.5,\y) {\y25};
    }
    
    % عناوین ستون‌ها
    \node[font=\small\bfseries] at (-2.5,5.5) {فعلی};
    \node[font=\small\bfseries] at (-0.5,5.5) {سال ۵};
    \node[font=\small\bfseries] at (1.5,5.5) {سال ۱۰};
    \node[font=\small\bfseries] at (3.5,5.5) {سال ۲۵};
\end{tikzpicture}
\end{center}

% ═══════════════════════════════════════════════════════════════════════════════
\section{بخش پنجم: تجارب جهانی گذار دموکراتیک}
\label{sec:data-transitions}
% ═══════════════════════════════════════════════════════════════════════════════

\subsection{جدول ۱۴: مقایسه گذارهای موفق}

\begin{center}
\begin{small}
\begin{longtable}{|>{\columncolor{goldlight!10}}p{2cm}|r|p{2.5cm}|p{2.5cm}|p{2.5cm}|p{2.5cm}|}
\hline
\rowcolor{goldlight!30}
\textbf{\rl{کشور}} & \textbf{\rl{سال گذار}} & \textbf{\rl{نوع رژیم قبلی}} & \textbf{\rl{محرک گذار}} & \textbf{\rl{مدت تثبیت}} & \textbf{\rl{موفقیت کلی}} \\
\hline
\endfirsthead

اسپانیا & ۱۹۷۵-۷۸ & دیکتاتوری نظامی & فوت فرانکو & ۱۰ سال & \cellcolor{SuccessGreen!30}موفق \\
\hline

پرتغال & ۱۹۷۴ & دیکتاتوری & کودتای نظامی & ۱۲ سال & \cellcolor{SuccessGreen!30}موفق \\
\hline

شیلی & ۱۹۸۸-۹۰ & دیکتاتوری نظامی & همه‌پرسی & ۱۵ سال & \cellcolor{SuccessGreen!30}موفق \\
\hline

لهستان & ۱۹۸۹ & کمونیستی & میز گرد & ۸ سال & \cellcolor{SuccessGreen!30}موفق \\
\hline

آفریقای جنوبی & ۱۹۹۰-۹۴ & آپارتاید & مذاکره & ۱۰ سال & \cellcolor{SuccessGreen!30}موفق \\
\hline

اندونزی & ۱۹۹۸ & اقتدارگرا & بحران اقتصادی & ۱۵ سال & \cellcolor{WisdomGold!30}نسبتاً موفق \\
\hline

تونس & ۲۰۱۱ & دیکتاتوری & انقلاب & در حال تثبیت & \cellcolor{WisdomGold!30}چالش‌دار \\
\hline

\end{longtable}
\end{small}
\end{center}

\subsection{جدول ۱۵: مقایسه گذارهای ناموفق}

\begin{center}
\begin{small}
\begin{longtable}{|>{\columncolor{golddark!10}}p{2cm}|r|p{2.5cm}|p{2.5cm}|p{4cm}|}
\hline
\rowcolor{golddark!30}
\textbf{\rl{کشور}} & \textbf{\rl{سال}} & \textbf{\rl{نوع گذار}} & \textbf{\rl{نتیجه}} & \textbf{\rl{علل شکست}} \\
\hline
\endfirsthead

مصر & ۲۰۱۱-۱۳ & انقلاب & کودتا & عدم اجماع، اسلام‌گرایان، ارتش \\
\hline

لیبی & ۲۰۱۱ & سقوط رژیم & جنگ داخلی & فقدان نهاد، مداخله خارجی، قبیله‌گرایی \\
\hline

یمن & ۲۰۱۱ & قیام & جنگ داخلی & تنش‌های مذهبی، مداخله خارجی \\
\hline

سوریه & ۲۰۱۱ & قیام & جنگ داخلی & سرکوب، مداخله خارجی، تکثر مخالفان \\
\hline

عراق & ۲۰۰۳ & اشغال & بی‌ثباتی & مداخله نظامی، انحلال ارتش، فرقه‌گرایی \\
\hline

افغانستان & ۲۰۰۱ & اشغال & شکست & مداخله نظامی، فساد، طالبان \\
\hline

ونزوئلا & ۲۰۱۹ & بن‌بست & شکست & حمایت نظامی از رژیم، تحریم‌ها \\
\hline

\end{longtable}
\end{small}
\end{center}

\subsection{جدول ۱۶: درس‌های کلیدی از تجارب جهانی}

\begin{center}
\begin{tikzpicture}
    % دو ستون: موفقیت و شکست
    \node[
        draw=SuccessGreen,
        line width=2pt,
        fill=SuccessGreen!10,
        rounded corners=10pt,
        minimum width=6.5cm,
        minimum height=8cm,
        align=right
    ] (success) at (-3.5,0) {};
    
    \node[
        draw=WarningRed,
        line width=2pt,
        fill=WarningRed!10,
        rounded corners=10pt,
        minimum width=6.5cm,
        minimum height=8cm,
        align=right
    ] (failure) at (3.5,0) {};
    
    % عناوین
    \node[fill=goldlight, text=white, rounded corners=3pt, font=\bfseries] at (-3.5,3.5) {\rl{عوامل موفقیت}};
    \node[fill=golddark, text=white, rounded corners=3pt, font=\bfseries] at (3.5,3.5) {\rl{عوامل شکست}};
    
    % محتوای موفقیت
    \node[font=\small, align=right, text width=5.5cm] at (-3.5,1) {
        ✓ اجماع ملی گسترده\\[3pt]
        ✓ رهبری متحد و معتبر\\[3pt]
        ✓ حمایت یا بی‌طرفی ارتش\\[3pt]
        ✓ میثاق بین نیروها\\[3pt]
        ✓ عدالت انتقالی متوازن\\[3pt]
        ✓ حمایت بین‌المللی\\[3pt]
        ✓ اصلاحات اقتصادی سریع\\[3pt]
        ✓ نهادسازی مؤثر
    };
    
    % محتوای شکست
    \node[font=\small, align=right, text width=5.5cm] at (3.5,1) {
        ✗ تفرقه و رقابت نخبگان\\[3pt]
        ✗ مداخله نظامی خارجی\\[3pt]
        ✗ انتقام‌جویی و خشونت\\[3pt]
        ✗ فروپاشی اقتصادی\\[3pt]
        ✗ شکاف‌های قومی/مذهبی\\[3pt]
        ✗ فقدان نهادهای مدنی\\[3pt]
        ✗ بازگشت نظامیان\\[3pt]
        ✗ افراط‌گرایی
    };
\end{tikzpicture}
\end{center}

% ═══════════════════════════════════════════════════════════════════════════════
\section{بخش ششم: جداول زمان‌بندی پروژه}
\label{sec:data-timeline}
% ═══════════════════════════════════════════════════════════════════════════════

\subsection{جدول ۱۷: تقویم کلان ۲۵ ساله}

\begin{center}
\begin{tikzpicture}[scale=0.9]
    % محور زمانی
    \draw[->, thick, gray] (0,0) -- (15,0);
    
    % علامت‌گذاری سال‌ها
    \foreach \x/\year in {0/۰, 2.5/۵, 5/۱۰, 7.5/۱۵, 10/۲۰, 12.5/۲۵} {
        \draw[thick] (\x,0.15) -- (\x,-0.15);
        \node[below, font=\small] at (\x,-0.3) {سال \year};
    }
    
    % فازها
    \draw[fill=WarningRed!40, draw=WarningRed] (0,0.5) rectangle (1.5,1.5);
    \node[font=\tiny, align=center] at (0.75,1) {فاز ۱\\گذار};
    
    \draw[fill=orange!40, draw=orange] (1.5,0.5) rectangle (4,1.5);
    \node[font=\tiny, align=center] at (2.75,1) {فاز ۲\\نهادسازی};
    
    \draw[fill=WisdomGold!40, draw=WisdomGold] (4,0.5) rectangle (7.5,1.5);
    \node[font=\tiny, align=center] at (5.75,1) {فاز ۳\\تحکیم};
    
    \draw[fill=SuccessGreen!40, draw=SuccessGreen] (7.5,0.5) rectangle (10,1.5);
    \node[font=\tiny, align=center] at (8.75,1) {فاز ۴\\بلوغ};
    
    \draw[fill=DemocracyBlue!40, draw=DemocracyBlue] (10,0.5) rectangle (12.5,1.5);
    \node[font=\tiny, align=center] at (11.25,1) {فاز ۵\\تعالی};
    
    % رویدادهای کلیدی
    \node[circle, fill=WarningRed, minimum size=8pt] at (0.5,-0.8) {};
    \node[font=\tiny, below] at (0.5,-1) {انتخابات اول};
    
    \node[circle, fill=orange, minimum size=8pt] at (2.5,-0.8) {};
    \node[font=\tiny, below] at (2.5,-1) {قانون اساسی};
    
    \node[circle, fill=WisdomGold, minimum size=8pt] at (5,-0.8) {};
    \node[font=\tiny, below] at (5,-1) {رفع تحریم};
    
    \node[circle, fill=SuccessGreen, minimum size=8pt] at (7.5,-0.8) {};
    \node[font=\tiny, below] at (7.5,-1) {دموکراسی پایدار};
    
    \node[circle, fill=DemocracyBlue, minimum size=8pt] at (10,-0.8) {};
    \node[font=\tiny, below] at (10,-1) {توسعه‌یافتگی};
\end{tikzpicture}
\end{center}

\begin{center}
\begin{small}
\begin{longtable}{|>{\columncolor{gray!10}}p{1.5cm}|p{2cm}|p{3cm}|p{4cm}|p{3cm}|}
\hline
\rowcolor{gray!30}
\textbf{فاز} & \textbf{دوره} & \textbf{عنوان} & \textbf{اهداف اصلی} & \textbf{شاخص موفقیت} \\
\hline
\endfirsthead

\cellcolor{WarningRed!20} فاز ۱ & سال ۱-۲ & گذار اولیه & 
• انتخابات آزاد
• قانون اساسی
• ثبات اولیه &
انتخابات موفق، عدم خشونت \\
\hline

\cellcolor{orange!20} فاز ۲ & سال ۳-۵ & نهادسازی & 
• استقرار فدرالیسم
• اصلاح قضایی
• آغاز رفع تحریم &
نهادهای مستقل فعال \\
\hline

\cellcolor{WisdomGold!20} فاز ۳ & سال ۶-۱۰ & تحکیم & 
• رشد اقتصادی پایدار
• دموکراسی نهادینه
• انتقال قدرت مسالمت‌آمیز &
دو انتقال قدرت موفق \\
\hline

\cellcolor{SuccessGreen!20} فاز ۴ & سال ۱۱-۱۵ & بلوغ & 
• رسیدن به درآمد متوسط بالا
• عضویت در نهادهای بین‌المللی
• الگوی منطقه‌ای &
GDP سرانه ۲۰,۰۰۰\$ \\
\hline

\cellcolor{DemocracyBlue!20} فاز ۵ & سال ۱۶-۲۵ & تعالی & 
• دموکراسی کامل
• اقتصاد دانش‌بنیان
• قدرت منطقه‌ای سازنده &
جزو ۵۰ کشور برتر \\
\hline

\end{longtable}
\end{small}
\end{center}

\subsection{جدول ۱۸: بودجه تخمینی طرح گذار}

\begin{center}
\begin{small}
\begin{longtable}{|>{\columncolor{WisdomGold!10}}p{4cm}|r|r|r|r|r|}
\hline
\rowcolor{WisdomGold!30}
\textbf{بخش} & \textbf{سال ۱-۲} & \textbf{سال ۳-۵} & \textbf{سال ۶-۱۰} & \textbf{مجموع ۱۰ سال} & \textbf{منبع تأمین} \\
\hline
% ═══════════════════════════════════════════════════════════════════════════════
% ادامه پیوست ۴: داده‌ها و جداول آماری
% ادامه جداول بودجه و خلاصه
% ═══════════════════════════════════════════════════════════════════════════════

\endfirsthead

انتخابات و دموکراتیزاسیون & ۲.۵ & ۱.۵ & ۲.۰ & ۶.۰ & بودجه + کمک بین‌المللی \\
\hline

عدالت انتقالی & ۱.۰ & ۱.۵ & ۰.۵ & ۳.۰ & بودجه + مصادره \\
\hline

نهادسازی و اصلاحات & ۳.۰ & ۵.۰ & ۴.۰ & ۱۲.۰ & بودجه \\
\hline

زیرساخت‌های آب & ۵.۰ & ۱۵.۰ & ۳۰.۰ & ۵۰.۰ & بودجه + وام + سرمایه‌گذاری \\
\hline

زیرساخت‌های انرژی & ۳.۰ & ۱۰.۰ & ۲۵.۰ & ۳۸.۰ & بودجه + سرمایه‌گذاری \\
\hline

توسعه مناطق محروم & ۵.۰ & ۱۵.۰ & ۳۰.۰ & ۵۰.۰ & صندوق توازن \\
\hline

آموزش و بهداشت & ۸.۰ & ۲۰.۰ & ۴۰.۰ & ۶۸.۰ & بودجه جاری \\
\hline

حمایت اجتماعی & ۱۰.۰ & ۱۵.۰ & ۲۰.۰ & ۴۵.۰ & بودجه جاری \\
\hline

امنیت و دفاع & ۱۲.۰ & ۲۵.۰ & ۴۰.۰ & ۷۷.۰ & بودجه \\
\hline

\rowcolor{yellow!20}
\textbf{مجموع (میلیارد \$)} & \textbf{۴۹.۵} & \textbf{۱۰۸.۰} & \textbf{۱۹۱.۵} & \textbf{۳۴۹.۰} & — \\
\hline

\multicolumn{6}{l}{\scriptsize * ارقام تخمینی بر اساس تجارب مشابه و شرایط ایران | نیازمند بازنگری دقیق‌تر} \\

\end{longtable}
\end{small}
\end{center}

\subsection{جدول ۱۹: منابع تأمین مالی}

\begin{center}
\begin{tikzpicture}
    % نمودار دایره‌ای منابع
    \pie[
        text=legend,
        radius=3,
        color={DemocracyBlue!70, SuccessGreen!70, WisdomGold!70, orange!70, purple!70}
    ]{
        45/درآمد نفتی,
        25/مالیات و درآمد داخلی,
        15/سرمایه‌گذاری خارجی,
        10/وام بین‌المللی,
        5/کمک‌های بین‌المللی
    }
    
    \node[font=\bfseries] at (0,-4.5) {ترکیب تأمین مالی طرح گذار (۱۰ ساله)};
\end{tikzpicture}
\end{center}

\begin{center}
\begin{small}
\begin{longtable}{|>{\columncolor{SuccessGreen!10}}p{3.5cm}|r|p{4cm}|p{4cm}|}
\hline
\rowcolor{SuccessGreen!30}
\textbf{منبع} & \textbf{سهم (٪)} & \textbf{پیش‌شرط} & \textbf{ریسک} \\
\hline
\endfirsthead

درآمد نفت و گاز & ۴۵٪ & رفع تحریم، افزایش تولید & نوسانات قیمت، تحریم مجدد \\
\hline

مالیات و درآمد داخلی & ۲۵٪ & اصلاح نظام مالیاتی & فرار مالیاتی، ضعف اقتصاد \\
\hline

سرمایه‌گذاری خارجی (FDI) & ۱۵٪ & ثبات سیاسی، رفع تحریم & بی‌اعتمادی، رقابت منطقه‌ای \\
\hline

وام بین‌المللی (IMF/WB) & ۱۰٪ & اصلاحات ساختاری & شرایط سخت، بدهی \\
\hline

کمک‌های بین‌المللی & ۵٪ & روابط دیپلماتیک & محدود، مشروط \\
\hline

\end{longtable}
\end{small}
\end{center}

% ═══════════════════════════════════════════════════════════════════════════════
\section{بخش هفتم: داده‌های مقایسه‌ای فدرالیسم}
\label{sec:data-federalism}
% ═══════════════════════════════════════════════════════════════════════════════

\subsection{جدول ۲۰: مقایسه سیستم‌های فدرال جهان}

\begin{center}
\begin{small}
\begin{longtable}{|>{\columncolor{purple!10}}p{2cm}|r|r|p{2.5cm}|p{2.5cm}|p{3cm}|}
\hline
\rowcolor{purple!30}
\textbf{کشور} & \textbf{واحدها} & \textbf{جمعیت (م)} & \textbf{نوع فدرالیسم} & \textbf{مجلس دوم} & \textbf{ویژگی خاص} \\
\hline
\endfirsthead

آلمان & ۱۶ لَند & ۸۳ & همکارانه & بوندسرات (نمایندگی ایالات) & فدرالیسم مالی قوی \\
\hline

آمریکا & ۵۰ ایالت & ۳۳۱ & دوگانه & سنا (۲ نماینده هر ایالت) & حقوق ایالتی گسترده \\
\hline

هند & ۲۸ ایالت + ۸ منطقه & ۱,۴۰۰ & نامتقارن & راجیا سابها & تنوع زبانی بالا \\
\hline

سوئیس & ۲۶ کانتون & ۸.۷ & مشارکتی & شورای کانتون‌ها & دموکراسی مستقیم \\
\hline

کانادا & ۱۰ استان + ۳ منطقه & ۳۸ & نامتقارن & سنا (انتصابی) & دوزبانگی رسمی \\
\hline

استرالیا & ۶ ایالت + ۲ منطقه & ۲۶ & همکارانه & سنا (۱۲ از هر ایالت) & فدرالیسم مالی \\
\hline

بلژیک & ۳ منطقه + ۳ جامعه & ۱۱.۵ & زبانی & سنا & مبتنی بر زبان \\
\hline

اسپانیا & ۱۷ منطقه خودمختار & ۴۷ & نامتقارن & سنا & خودمختاری متفاوت \\
\hline

\rowcolor{yellow!15}
ایران (پیشنهادی) & ۵ منطقه + ۱۵ استان & ۸۶ & نامتقارن-همبسته & مجلس اقوام & فدرالیسم قومی \\
\hline

\end{longtable}
\end{small}
\end{center}

\subsection{جدول ۲۱: توزیع صلاحیت‌ها در نظام فدرال پیشنهادی}

\begin{center}
\begin{tikzpicture}
    % سه ستون
    \node[
        draw=DemocracyBlue,
        line width=2pt,
        fill=DemocracyBlue!15,
        rounded corners=10pt,
        minimum width=4.5cm,
        minimum height=9cm,
        align=center
    ] (fed) at (-5,0) {};
    
    \node[
        draw=SuccessGreen,
        line width=2pt,
        fill=SuccessGreen!15,
        rounded corners=10pt,
        minimum width=4.5cm,
        minimum height=9cm,
        align=center
    ] (shared) at (0,0) {};
    
    \node[
        draw=WisdomGold,
        line width=2pt,
        fill=WisdomGold!15,
        rounded corners=10pt,
        minimum width=4.5cm,
        minimum height=9cm,
        align=center
    ] (local) at (5,0) {};
    
    % عناوین
    \node[fill=DemocracyBlue, text=white, rounded corners=3pt, font=\small\bfseries] at (-5,4) {صلاحیت فدرال};
    \node[fill=SuccessGreen, text=white, rounded corners=3pt, font=\small\bfseries] at (0,4) {صلاحیت مشترک};
    \node[fill=WisdomGold, text=white, rounded corners=3pt, font=\small\bfseries] at (5,4) {صلاحیت منطقه‌ای};
    
    % محتوا
    \node[font=\scriptsize, align=right, text width=4cm] at (-5,1) {
        • دفاع ملی\\
        • سیاست خارجی\\
        • پول و بانک مرکزی\\
        • گمرک و تجارت خارجی\\
        • مهاجرت و تابعیت\\
        • ارتباطات ملی\\
        • انرژی هسته‌ای\\
        • حقوق اساسی
    };
    
    \node[font=\scriptsize, align=right, text width=4cm] at (0,1) {
        • آموزش عالی\\
        • بهداشت عمومی\\
        • محیط زیست\\
        • مدیریت آب\\
        • حمل‌ونقل بین‌استانی\\
        • انرژی\\
        • مالیات\\
        • امنیت داخلی
    };
    
    \node[font=\scriptsize, align=right, text width=4cm] at (5,1) {
        • آموزش پایه\\
        • فرهنگ و زبان\\
        • پلیس محلی\\
        • شهرسازی\\
        • خدمات اجتماعی\\
        • گردشگری\\
        • کشاورزی محلی\\
        • امور شهرداری‌ها
    };
\end{tikzpicture}
\end{center}

\subsection{جدول ۲۲: فرمول توزیع منابع صندوق توازن}

\begin{center}
\begin{small}
\begin{longtable}{|>{\columncolor{WisdomGold!10}}p{3cm}|r|p{5cm}|p{4cm}|}
\hline
\rowcolor{WisdomGold!30}
\textbf{معیار} & \textbf{وزن (٪)} & \textbf{شاخص سنجش} & \textbf{منطق} \\
\hline
\endfirsthead

جمعیت & ۵۰٪ & جمعیت رسمی سرشماری & عدالت توزیعی پایه \\
\hline

مساحت & ۲۰٪ & کیلومتر مربع & هزینه‌های زیرساختی \\
\hline

شاخص محرومیت & ۱۵٪ & ترکیب HDI، فقر، بیکاری & جبران عقب‌ماندگی \\
\hline

عملکرد & ۱۰٪ & شاخص‌های حکمرانی و کارایی & تشویق به بهبود \\
\hline

سهم تولیدکننده & ۵٪ & تولید منابع طبیعی & حقوق مناطق تولیدکننده \\
\hline

\end{longtable}
\end{small}
\end{center}

\subsubsection{نمونه محاسبه برای مناطق خودمختار}

\begin{center}
\begin{small}
\begin{longtable}{|>{\columncolor{gray!10}}p{2.5cm}|r|r|r|r|r|r|}
\hline
\rowcolor{gray!30}
\textbf{منطقه} & \textbf{جمعیت (٪)} & \textbf{مساحت (٪)} & \textbf{محرومیت} & \textbf{عملکرد} & \textbf{تولیدکننده} & \textbf{سهم نهایی (٪)} \\
\hline
\endfirsthead

آذربایجان & ۸.۵ & ۶.۵ & ۰.۷۵ & ۰.۸۰ & ۰.۵۰ & ۷.۸۵ \\
\hline

کردستان & ۴.۵ & ۵.۲ & ۱.۲۰ & ۰.۷۰ & ۰.۳۰ & ۵.۴۵ \\
\hline

خوزستان & ۵.۶ & ۳.۹ & ۱.۰۰ & ۰.۶۵ & ۳.۰۰ & ۶.۲۵ \\
\hline

بلوچستان & ۳.۴ & ۱۱.۰ & ۱.۸۰ & ۰.۵۰ & ۰.۲۰ & ۵.۹۰ \\
\hline

ترکمن‌صحرا & ۱.۰ & ۱.۲ & ۰.۹۰ & ۰.۶۰ & ۱.۵۰ & ۱.۴۵ \\
\hline

\rowcolor{yellow!20}
\textbf{جمع مناطق خودمختار} & \textbf{۲۳.۰} & \textbf{۲۷.۸} & — & — & — & \textbf{۲۶.۹۰} \\
\hline

\multicolumn{7}{l}{\scriptsize فرمول: سهم = (۰.۵ × جمعیت) + (۰.۲ × مساحت) + (۰.۱۵ × محرومیت) + (۰.۱ × عملکرد) + (۰.۰۵ × تولیدکننده)} \\

\end{longtable}
\end{small}
\end{center}

% ═══════════════════════════════════════════════════════════════════════════════
\section{بخش هشتم: شاخص‌های پایش و ارزیابی}
\label{sec:data-monitoring}
% ═══════════════════════════════════════════════════════════════════════════════

\subsection{جدول ۲۳: داشبورد ملی پایش گذار}

\begin{center}
\begin{tikzpicture}
    % کادر اصلی داشبورد
    \node[
        draw=DemocracyBlue,
        line width=2pt,
        fill=white,
        rounded corners=15pt,
        minimum width=15cm,
        minimum height=10cm
    ] at (0,0) {};
    
    % عنوان
    \node[fill=DemocracyBlue, text=white, rounded corners=5pt, font=\large\bfseries, minimum width=10cm] at (0,4.5) {داشبورد ملی پایش گذار دموکراتیک};
    
    % چهار بخش
    % بخش ۱: سیاسی
    \node[
        draw=WarningRed,
        fill=WarningRed!10,
        rounded corners=5pt,
        minimum width=6.5cm,
        minimum height=3.5cm
    ] at (-3.5,1.5) {};
    \node[fill=WarningRed, text=white, rounded corners=3pt, font=\small\bfseries] at (-3.5,3) {شاخص‌های سیاسی};
    \node[font=\tiny, align=right] at (-3.5,1.2) {
        دموکراسی: ۴.۵/۱۰ \quad {\color{orange}▲}\\
        آزادی مطبوعات: ۱۲۰/۱۸۰ \quad {\color{SuccessGreen}▲}\\
        مشارکت انتخاباتی: ۶۵٪ \quad {\color{SuccessGreen}▲}\\
        اعتماد به دولت: ۴۵٪ \quad {\color{orange}▲}
    };
    
    % بخش ۲: اقتصادی
    \node[
        draw=SuccessGreen,
        fill=SuccessGreen!10,
        rounded corners=5pt,
        minimum width=6.5cm,
        minimum height=3.5cm
    ] at (3.5,1.5) {};
    \node[fill=SuccessGreen, text=white, rounded corners=3pt, font=\small\bfseries] at (3.5,3) {شاخص‌های اقتصادی};
    \node[font=\tiny, align=right] at (3.5,1.2) {
        رشد GDP: ۵.۲٪ \quad {\color{SuccessGreen}▲}\\
        تورم: ۲۵٪ \quad {\color{orange}▼}\\
        بیکاری: ۸.۵٪ \quad {\color{SuccessGreen}▼}\\
        سرمایه‌گذاری خارجی: ۱۲B\$ \quad {\color{SuccessGreen}▲}
    };
    
    % بخش ۳: اجتماعی
    \node[
        draw=WisdomGold,
        fill=WisdomGold!10,
        rounded corners=5pt,
        minimum width=6.5cm,
        minimum height=3.5cm
    ] at (-3.5,-2.5) {};
    \node[fill=WisdomGold, text=white, rounded corners=3pt, font=\small\bfseries] at (-3.5,-1) {شاخص‌های اجتماعی};
    \node[font=\tiny, align=right] at (-3.5,-2.8) {
        HDI: ۰.۷۹ \quad {\color{SuccessGreen}▲}\\
        نرخ فقر: ۱۸٪ \quad {\color{orange}▼}\\
        نابرابری (ژینی): ۰.۳۸ \quad {\color{SuccessGreen}▼}\\
        دسترسی به بهداشت: ۸۵٪ \quad {\color{SuccessGreen}▲}
    };
    
    % بخش ۴: محیطی
    \node[
        draw=purple,
        fill=purple!10,
        rounded corners=5pt,
        minimum width=6.5cm,
        minimum height=3.5cm
    ] at (3.5,-2.5) {};
    \node[fill=purple, text=white, rounded corners=3pt, font=\small\bfseries] at (3.5,-1) {شاخص‌های محیطی};
    \node[font=\tiny, align=right] at (3.5,-2.8) {
        بیلان آبی: -۱۵B م³ \quad {\color{orange}▲}\\
        آلودگی هوا: ۱۲۰ AQI \quad {\color{WarningRed}—}\\
        انرژی تجدیدپذیر: ۸٪ \quad {\color{SuccessGreen}▲}\\
        جنگل: ۷.۵٪ \quad {\color{orange}▲}
    };
\end{tikzpicture}
\end{center}

\subsection{جدول ۲۴: ماتریس شاخص‌های KPI}

\begin{center}
\begin{small}
\begin{longtable}{|>{\columncolor{DemocracyBlue!10}}p{3cm}|p{2.5cm}|r|r|r|r|p{1.5cm}|}
\hline
\rowcolor{DemocracyBlue!30}
\textbf{شاخص} & \textbf{منبع داده} & \textbf{خط پایه} & \textbf{هدف سال ۵} & \textbf{هدف سال ۱۰} & \textbf{فعلی} & \textbf{وضعیت} \\
\hline
\endfirsthead

\multicolumn{7}{c}{\textbf{— حوزه سیاسی —}} \\
\hline

شاخص دموکراسی EIU & EIU & ۱.۹۵ & ۴.۵ & ۶.۵ & — & پایه \\
\hline

آزادی مطبوعات RSF & RSF & ۱۷۶ & ۱۲۰ & ۸۰ & — & پایه \\
\hline

مشارکت انتخاباتی & کمیسیون & — & ۶۵٪ & ۷۵٪ & — & — \\
\hline

زنان در پارلمان & IPU & ۵٪ & ۲۵٪ & ۳۵٪ & — & پایه \\
\hline

\multicolumn{7}{c}{\textbf{— حوزه اقتصادی —}} \\
\hline

GDP سرانه (\$) & WB & ۴,۶۵۰ & ۸,۰۰۰ & ۱۵,۰۰۰ & — & پایه \\
\hline

نرخ تورم (٪) & بانک مرکزی & ۴۵ & ۲۰ & ۱۰ & — & پایه \\
\hline

نرخ بیکاری (٪) & مرکز آمار & ۹.۱ & ۷ & ۵ & — & پایه \\
\hline

شاخص فساد TI & TI & ۲۵ & ۴۰ & ۵۵ & — & پایه \\
\hline

\multicolumn{7}{c}{\textbf{— حوزه اجتماعی —}} \\
\hline

شاخص توسعه انسانی & UNDP & ۰.۷۷ & ۰.۸۲ & ۰.۸۷ & — & پایه \\
\hline

نرخ فقر (٪) & WB & ۲۵ & ۱۵ & ۸ & — & پایه \\
\hline

ضریب جینی & WB & ۰.۴۲ & ۰.۳۸ & ۰.۳۲ & — & پایه \\
\hline

امید به زندگی (سال) & WHO & ۷۴.۵ & ۷۶ & ۷۸ & — & پایه \\
\hline

\multicolumn{7}{c}{\textbf{— حوزه محیط زیست —}} \\
\hline

کسری آب (میلیارد م³) & وزارت نیرو & ۱۸ & ۱۲ & ۵ & — & پایه \\
\hline

انرژی تجدیدپذیر (٪) & IEA & ۲ & ۱۰ & ۲۵ & — & پایه \\
\hline

انتشار CO2 سرانه (تن) & WB & ۸.۵ & ۷.۵ & ۶ & — & پایه \\
\hline

\end{longtable}
\end{small}
\end{center}

\subsection{جدول ۲۵: سیستم هشدار زودهنگام}

\begin{center}
\begin{small}
\begin{longtable}{|>{\columncolor{WarningRed!10}}p{3cm}|p{3.5cm}|p{2cm}|p{2cm}|p{3cm}|}
\hline
\rowcolor{WarningRed!30}
\textbf{ریسک} & \textbf{شاخص هشدار} & \textbf{آستانه زرد} & \textbf{آستانه قرمز} & \textbf{اقدام واکنشی} \\
\hline
\endfirsthead

بی‌ثباتی سیاسی & اعتراضات خشونت‌آمیز & ۵ مورد/ماه & ۱۵ مورد/ماه & گفتگوی ملی، امتیازات \\
\hline

تنش قومی & درگیری‌های قومی & ۳ مورد/ماه & ۱۰ مورد/ماه & شورای آشتی، میانجیگری \\
\hline

فروپاشی اقتصادی & کاهش GDP & -۳٪ & -۷٪ & بسته نجات، کمک بین‌المللی \\
\hline

تورم افسارگسیخته & نرخ تورم ماهانه & ۵٪ & ۱۰٪ & سیاست پولی انقباضی \\
\hline

بحران ارزی & کاهش ارزش ریال & ۳۰٪/سال & ۶۰٪/سال & مداخله ارزی، کنترل \\
\hline

کودتای نظامی & فعالیت‌های مشکوک & شایعات پایدار & حرکت نیروها & فعال‌سازی پروتکل ضدکودتا \\
\hline

مداخله خارجی & تحریکات مرزی & ۳ مورد/ماه & ۱۰ مورد/ماه & دیپلماسی فعال، هشدار UN \\
\hline

بحران انسانی & جابجایی داخلی & ۵۰,۰۰۰ نفر & ۲۰۰,۰۰۰ نفر & کمک اضطراری، UNHCR \\
\hline

\end{longtable}
\end{small}
\end{center}

% ═══════════════════════════════════════════════════════════════════════════════
\section{خلاصه و منابع داده‌ها}
\label{sec:data-summary}
% ═══════════════════════════════════════════════════════════════════════════════

\begin{kholasebox}
\textbf{خلاصه پیوست داده‌ها و جداول آماری}

\begin{center}
\begin{tabular}{r r}
\textbf{بخش} & \textbf{تعداد جداول} \\
\hline
داده‌های جمعیتی و قومی & ۳ جدول \\
داده‌های اقتصادی & ۴ جدول \\
داده‌های بحران آب & ۳ جدول \\
شاخص‌های حکمرانی & ۳ جدول \\
تجارب جهانی & ۳ جدول \\
جداول زمان‌بندی و بودجه & ۳ جدول \\
داده‌های فدرالیسم & ۳ جدول \\
شاخص‌های پایش & ۳ جدول \\
\hline
\textbf{مجموع} & \textbf{۲۵ جدول} \\
\end{tabular}
\end{center}
\end{kholasebox}

\subsection{منابع اصلی داده‌ها}

\begin{longtable}{|>{\columncolor{gray!10}}p{4cm}|p{5cm}|p{5cm}|}
\hline
\rowcolor{gray!30}
\textbf{سازمان} & \textbf{نوع داده} & \textbf{آدرس} \\
\hline
\endfirsthead

بانک جهانی (World Bank) & اقتصادی، توسعه، فقر & data.worldbank.org \\
\hline

صندوق بین‌المللی پول (IMF) & اقتصاد کلان، مالی & imf.org/data \\
\hline

سازمان ملل (UN) & جمعیت، HDI، محیط زیست & data.un.org \\
\hline

UNDP & توسعه انسانی & hdr.undp.org \\
\hline

Transparency International & فساد & transparency.org \\
\hline

Freedom House & آزادی سیاسی و مدنی & freedomhouse.org \\
\hline

Economist Intelligence Unit & شاخص دموکراسی & eiu.com \\
\hline

گزارشگران بدون مرز (RSF) & آزادی مطبوعات & rsf.org \\
\hline

مرکز آمار ایران & داده‌های داخلی & amar.org.ir \\
\hline

FAO & کشاورزی، آب، غذا & fao.org/faostat \\
\hline

IEA & انرژی & iea.org \\
\hline

WHO & بهداشت & who.int/data \\
\hline

\end{longtable}

\begin{enghelabbox}
\textbf{⚠️ تذکر مهم درباره داده‌ها}

\begin{itemize}[nosep]
    \item برخی داده‌های مربوط به ایران به دلیل محدودیت دسترسی، \textbf{تخمینی} هستند.
    \item داده‌های قومی بر اساس برآوردهای غیررسمی است زیرا سرشماری رسمی قومی وجود ندارد.
    \item شاخص‌های آینده بر اساس \textbf{سناریوهای خوش‌بینانه} با فرض اجرای کامل طرح است.
    \item داده‌ها باید پس از گذار با \textbf{سرشماری و آمارگیری جدید} به‌روزرسانی شوند.
    \item برای برنامه‌ریزی دقیق، همکاری با نهادهای بین‌المللی آماری ضروری است.
\end{itemize}
\end{enghelabbox}

% ═══════════════════════════════════════════════════════════════════════════════
% پایان پیوست ۴
% ═══════════════════════════════════════════════════════════════════════════════