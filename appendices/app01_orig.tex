% ═══════════════════════════════════════════════════════════════════════════════
% پیوست ۱: متن قانون اساسی پیشنهادی جمهوری فدرال ایران
% فایل: app01-constitution.tex
% ═══════════════════════════════════════════════════════════════════════════════

\chapter{متن قانون اساسی پیشنهادی}
\label{app:constitution}

\begin{kholasebox}
این پیوست متن کامل پیش‌نویس قانون اساسی جمهوری فدرال ایران را ارائه می‌دهد. این سند حقوقی بنیادین، حاصل تلفیق تجارب موفق جهانی با واقعیت‌های تاریخی-فرهنگی ایران است. ساختار این قانون اساسی بر سه ستون استوار است: \textbf{دموکراسی مشارکتی}، \textbf{فدرالیسم همبسته}، و \textbf{حقوق بنیادین تضمین‌شده}. متن شامل ۱۸۷ اصل در هفت فصل به همراه مقررات انتقالی است.
\end{kholasebox}

% ═══════════════════════════════════════════════════════════════════════════════
\section{دیباچه قانون اساسی}
% ═══════════════════════════════════════════════════════════════════════════════

\begin{center}
\begin{tikzpicture}
    % کادر اصلی دیباچه
    \node[
        draw=DemocracyBlue,
        line width=3pt,
        fill=DemocracyBlue!5,
        rounded corners=15pt,
        inner sep=20pt,
        text width=14cm,
        align=justify
    ] (preamble) {
        \begin{center}
            {\Huge\textbf{قانون اساسی}}\\[5pt]
            {\LARGE\textbf{جمهوری فدرال ایران}}\\[10pt]
            {\large مصوب مجلس مؤسسان | سال ۱۴۰۵ هجری شمسی}
        \end{center}
    };
    
    % نمادهای تزئینی
    \node[above=5pt of preamble, DemocracyBlue] {\Large ★ ★ ★};
\end{tikzpicture}
\end{center}

\vspace{10pt}

\begin{naghlbox}
\textbf{دیباچه}

ما، ملت ایران،

وارثان تمدنی کهن با پیشینه‌ای درخشان در تاریخ بشری؛

مردمانی متکثر از تبارها، زبان‌ها، فرهنگ‌ها و باورهای گوناگون که در طول هزاره‌ها در این سرزمین در کنار یکدیگر زیسته‌ایم؛

با یادآوری رنج‌های مشترک از استبداد، تبعیض و بی‌عدالتی؛

با درس‌آموزی از تجربه‌های تلخ و شیرین تاریخ معاصر؛

با ایمان به کرامت ذاتی انسان و برابری همه شهروندان؛

با تعهد به آزادی، عدالت، دموکراسی و حاکمیت قانون؛

با احترام به تنوع فرهنگی به‌مثابه ثروت ملی؛

با مسئولیت در برابر نسل‌های آینده و محیط زیست؛

با امید به ساختن آینده‌ای بهتر برای همه فرزندان این سرزمین؛

\textbf{این قانون اساسی را به‌عنوان میثاق ملی خود تصویب و اعلام می‌کنیم.}
\sourceline{مجلس مؤسسان جمهوری فدرال ایران}
\end{naghlbox}

% ═══════════════════════════════════════════════════════════════════════════════
\section{فصل اول: اصول کلی و هویت ملی}
\label{sec:const-ch1}
% ═══════════════════════════════════════════════════════════════════════════════

\subsection{بخش یکم: ماهیت و شکل حکومت}

\begin{center}
\begin{tikzpicture}
    % عنوان
    \node[
        fill=DemocracyBlue,
        text=white,
        font=\large\bfseries,
        rounded corners=5pt,
        inner sep=10pt
    ] at (0,4) {اصول بنیادین نظام (اصول ۱-۱۰)};
    
    % ستون اصول
    \foreach \i/\title/\content in {
        1/{اصل ۱: ماهیت حکومت}/{جمهوری فدرال دموکراتیک},
        2/{اصل ۲: منشأ حاکمیت}/{حاکمیت ملی متعلق به مردم},
        3/{اصل ۳: وحدت ملی}/{تمامیت ارضی و همبستگی ملی},
        4/{اصل ۴: تفکیک قوا}/{استقلال قوای سه‌گانه},
        5/{اصل ۵: حاکمیت قانون}/{برتری قانون اساسی}
    } {
        \node[
            draw=DemocracyBlue!60,
            fill=DemocracyBlue!10,
            rounded corners=5pt,
            minimum width=6.5cm,
            minimum height=1cm,
            align=center,
            font=\small\bfseries
        ] (left\i) at (-4, 3-\i*1.2) {\title};
        
        \node[
            draw=gray!40,
            fill=gray!5,
            rounded corners=5pt,
            minimum width=6.5cm,
            minimum height=1cm,
            align=center,
            font=\small
        ] (right\i) at (4, 3-\i*1.2) {\content};
        
        \draw[->, thick, DemocracyBlue!60] (left\i) -- (right\i);
    }
\end{tikzpicture}
\end{center}

\begin{longtable}{|>{\columncolor{DemocracyBlue!15}}r|p{12cm}|}
\hline
\rowcolor{DemocracyBlue!30}
\textbf{شماره اصل} & \textbf{متن اصل} \\
\hline
\endfirsthead
\hline
\rowcolor{DemocracyBlue!30}
\textbf{شماره اصل} & \textbf{متن اصل} \\
\hline
\endhead

\textbf{اصل ۱} & 
\textbf{ماهیت حکومت}

ایران کشوری است با نظام جمهوری فدرال دموکراتیک که بر پایه حاکمیت ملی، حقوق بشر، و حکومت قانون بنیان نهاده شده است. \\
\hline

\textbf{اصل ۲} &
\textbf{منشأ حاکمیت}

حاکمیت ملی به‌طور انحصاری متعلق به مردم ایران است. هیچ فرد، گروه، نهاد یا مقامی نمی‌تواند این حاکمیت را از آن خود سازد یا به‌نام خدا، تاریخ، نژاد، یا هر مفهوم دیگری اعمال کند مگر از طریق سازوکارهای دموکراتیک پیش‌بینی‌شده در این قانون اساسی. \\
\hline

\textbf{اصل ۳} &
\textbf{وحدت و تمامیت ارضی}

ایران سرزمینی واحد، تجزیه‌ناپذیر و دارای تمامیت ارضی است. تنوع قومی، زبانی، فرهنگی و مذهبی ملت ایران به رسمیت شناخته شده و محترم است و همبستگی ملی از طریق فدرالیسم دموکراتیک تضمین می‌شود. \\
\hline

\textbf{اصل ۴} &
\textbf{تفکیک قوا}

قدرت حاکمیت ملی به سه قوه مستقل و متوازن تفکیک می‌شود:

الف) قوه مقننه: مجلس ملی و مجلس اقوام

ب) قوه مجریه: رئیس‌جمهور و هیئت وزیران

ج) قوه قضائیه: دادگستری مستقل

این سه قوه تحت نظارت نهادهای مستقل و در چارچوب قانون اساسی عمل می‌کنند. \\
\hline

\textbf{اصل ۵} &
\textbf{برتری قانون اساسی}

قانون اساسی برترین قانون کشور است. هیچ قانون، مقرره، تصمیم یا اقدامی نمی‌تواند مغایر با آن باشد. دیوان عالی قانون اساسی مرجع نهایی تفسیر و پاسداری از قانون اساسی است. \\
\hline

\textbf{اصل ۶} &
\textbf{جدایی دین از دولت}

دولت نسبت به همه ادیان، مذاهب و باورها بی‌طرف است. هیچ دینی، دین رسمی دولت نیست. آزادی دین و وجدان تضمین می‌شود. دولت نمی‌تواند هیچ دینی را تحمیل یا تبلیغ کند و نمی‌تواند بر اساس دین تبعیض روا دارد. \\
\hline

\textbf{اصل ۷} &
\textbf{زبان‌های ملی}

الف) زبان فارسی زبان رسمی و مشترک سراسری جمهوری فدرال ایران است.

ب) زبان‌های آذری، کردی، عربی، بلوچی، ترکمنی، لری، گیلکی و مازندرانی زبان‌های ملی ایران هستند که در مناطق مربوطه رسمیت دارند.

ج) آموزش به زبان مادری در کنار زبان فارسی حق هر شهروند است. \\
\hline

\textbf{اصل ۸} &
\textbf{پرچم و نمادهای ملی}

پرچم ملی، سرود ملی، و نشان ملی جمهوری فدرال ایران به موجب قانون تعیین می‌شود. این نمادها باید بازتاب‌دهنده تنوع و وحدت ملت ایران باشند. \\
\hline

\textbf{اصل ۹} &
\textbf{پایتخت}

تهران پایتخت جمهوری فدرال ایران است. انتقال پایتخت مستلزم تصویب دوسوم هر دو مجلس است. \\
\hline

\textbf{اصل ۱۰} &
\textbf{سیاست خارجی}

سیاست خارجی جمهوری فدرال ایران بر اصول زیر استوار است:

الف) احترام به حقوق بین‌الملل و منشور ملل متحد

ب) همزیستی مسالمت‌آمیز و عدم مداخله در امور داخلی کشورها

ج) پیگیری منافع ملی از طریق دیپلماسی

د) همکاری منطقه‌ای و بین‌المللی برای صلح و توسعه پایدار \\
\hline

\end{longtable}

\subsection{بخش دوم: اصول غیرقابل تغییر}

\begin{enghelabbox}
\textbf{ (!)  اصول غیرقابل تغییر (اصل ۱۱)}

اصول زیر غیرقابل تغییر هستند و هیچ بازنگری در قانون اساسی نمی‌تواند آنها را نقض کند:

\begin{enumerate}[nosep, label=\alph*)]
    \item ماهیت جمهوری و دموکراتیک نظام
    \item حاکمیت ملی و انتخابی بودن حکومت
    \item حقوق بنیادین شهروندان مندرج در فصل دوم
    \item تفکیک و استقلال قوای سه‌گانه
    \item جدایی دین از دولت
    \item حقوق اقوام و ساختار فدرال
    \item استقلال و تمامیت ارضی کشور
\end{enumerate}
\end{enghelabbox}

% ═══════════════════════════════════════════════════════════════════════════════
\section{فصل دوم: حقوق بنیادین شهروندان}
\label{sec:const-ch2}
% ═══════════════════════════════════════════════════════════════════════════════

\begin{center}
\begin{tikzpicture}
    % عنوان مرکزی
    \node[
        draw=SuccessGreen,
        line width=2pt,
        fill=SuccessGreen!20,
        rounded corners=10pt,
        minimum width=4cm,
        minimum height=1.5cm,
        font=\large\bfseries
    ] (center) at (0,0) {حقوق بنیادین شهروندان};
    
    % دسته‌بندی حقوق
    \foreach \angle/\title/\color in {
        90/{حقوق مدنی}/DemocracyBlue,
        30/{حقوق سیاسی}/WisdomGold,
        -30/{حقوق اقتصادی}/SuccessGreen,
        -90/{حقوق اجتماعی}/WarningRed,
        -150/{حقوق فرهنگی}/purple,
        150/{حقوق قضایی}/orange
    } {
        \node[
            draw=\color,
            fill=\color!15,
            rounded corners=5pt,
            minimum width=3cm,
            minimum height=1cm,
            font=\small\bfseries
        ] (\angle) at (\angle:4cm) {\title};
        
        \draw[->, thick, \color] (center) -- (\angle);
    }
\end{tikzpicture}
\end{center}

\subsection{بخش یکم: حقوق مدنی و آزادی‌های فردی (اصول ۱۲-۳۵)}

\begin{longtable}{|>{\columncolor{DemocracyBlue!15}}r|p{12cm}|}
\hline
\rowcolor{DemocracyBlue!30}
\textbf{شماره اصل} & \textbf{متن اصل} \\
\hline
\endfirsthead
\hline
\rowcolor{DemocracyBlue!30}
\textbf{شماره اصل} & \textbf{متن اصل} \\
\hline
\endhead

\textbf{اصل ۱۲} &
\textbf{کرامت انسانی}

کرامت ذاتی انسان مصون از تعرض است. احترام به این کرامت و حمایت از آن وظیفه همه نهادهای حکومتی است. \\
\hline

\textbf{اصل ۱۳} &
\textbf{برابری در برابر قانون}

همه شهروندان در برابر قانون برابرند. هرگونه تبعیض بر اساس جنسیت، قومیت، نژاد، زبان، دین، مذهب، باور سیاسی، منشأ اجتماعی، وضعیت اقتصادی، معلولیت، سن، یا هر وضعیت دیگر ممنوع است. \\
\hline

\textbf{اصل ۱۴} &
\textbf{حق حیات}

حق حیات هر انسان از تولد تا مرگ طبیعی محترم و تضمین‌شده است. مجازات اعدام ملغی است. \\
\hline

\textbf{اصل ۱۵} &
\textbf{ممنوعیت شکنجه}

شکنجه، رفتار غیرانسانی، تحقیرآمیز یا بی‌رحمانه در هر شرایطی مطلقاً ممنوع است. هرگونه اعتراف یا مدرکی که از طریق شکنجه به دست آمده باشد فاقد اعتبار قانونی است. \\
\hline

\textbf{اصل ۱۶} &
\textbf{آزادی فردی و امنیت شخصی}

الف) هیچ‌کس را نمی‌توان بازداشت کرد مگر به موجب حکم مقام قضایی صالح و طبق قانون.

ب) هر بازداشت‌شده باید ظرف ۲۴ ساعت به دادگاه معرفی شود.

ج) هر کس حق دارد فوراً از دلایل بازداشت خود مطلع شود و به وکیل دسترسی داشته باشد.

د) بازداشت موقت استثنایی است و مدت آن نمی‌تواند از حدود قانونی تجاوز کند. \\
\hline

\textbf{اصل ۱۷} &
\textbf{حریم خصوصی}

حریم خصوصی افراد، مسکن، مکاتبات، ارتباطات و داده‌های شخصی مصون از تعرض است. تفتیش و نظارت تنها با حکم قضایی و در چارچوب قانون مجاز است. \\
\hline

\textbf{اصل ۱۸} &
\textbf{آزادی رفت‌وآمد}

هر شهروند حق دارد آزادانه در سراسر کشور رفت‌وآمد کند، محل سکونت خود را انتخاب نماید، از کشور خارج و به آن بازگردد. سلب یا محدودیت این حق تنها به موجب قانون و با حکم قضایی ممکن است. \\
\hline

\textbf{اصل ۱۹} &
\textbf{آزادی عقیده و بیان}

الف) هر کس حق دارد آزادانه عقاید خود را داشته و ابراز کند.

ب) آزادی بیان، نوشتار، هنر و رسانه تضمین می‌شود.

ج) سانسور ممنوع است. محدودیت آزادی بیان تنها برای حمایت از حقوق دیگران، امنیت ملی یا نظم عمومی و به موجب قانون مجاز است.

د) تحریک به خشونت، نفرت‌پراکنی و انکار جنایات علیه بشریت ممنوع است. \\
\hline

\textbf{اصل ۲۰} &
\textbf{آزادی دین و وجدان}

الف) هر کس حق دارد دین، مذهب یا باور خود را آزادانه انتخاب کند یا بدون دین باشد.

ب) هر کس حق دارد دین خود را به‌صورت فردی یا جمعی، در خلوت یا علن، عبادت و آموزش دهد.

ج) هیچ‌کس را نمی‌توان به پذیرش یا ترک دینی مجبور کرد.

د) تغییر دین حق هر فرد است و مجازات ندارد. \\
\hline

\textbf{اصل ۲۱} &
\textbf{آزادی تجمع و تشکل}

الف) شهروندان حق برگزاری تجمعات و راهپیمایی‌های مسالمت‌آمیز بدون سلاح را دارند.

ب) تجمعات عمومی نیازمند اطلاع‌رسانی قبلی هستند، نه مجوز.

ج) تشکیل احزاب، انجمن‌ها، سندیکاها و سازمان‌های مدنی آزاد است. \\
\hline

\textbf{اصل ۲۲} &
\textbf{حق مالکیت}

الف) مالکیت خصوصی به رسمیت شناخته شده و حمایت می‌شود.

ب) سلب مالکیت تنها برای مصالح عمومی، با پرداخت غرامت عادلانه و به موجب قانون مجاز است.

ج) مالکیت متضمن مسئولیت اجتماعی است. \\
\hline

\end{longtable}

\subsection{بخش دوم: حقوق سیاسی (اصول ۳۶-۴۵)}

\begin{longtable}{|>{\columncolor{WisdomGold!15}}r|p{12cm}|}
\hline
\rowcolor{WisdomGold!30}
\textbf{شماره اصل} & \textbf{متن اصل} \\
\hline
\endfirsthead
\hline
\rowcolor{WisdomGold!30}
\textbf{شماره اصل} & \textbf{متن اصل} \\
\hline
\endhead

\textbf{اصل ۳۶} &
\textbf{حق رأی}

الف) هر شهروند ایرانی که به سن ۱۸ سال رسیده باشد حق رأی دارد.

ب) رأی‌گیری آزاد، برابر، مستقیم و مخفی است.

ج) شرکت در انتخابات حق شهروندی است، نه تکلیف. \\
\hline

\textbf{اصل ۳۷} &
\textbf{حق انتخاب شدن}

الف) هر شهروندی که واجد شرایط قانونی باشد حق نامزد شدن برای مناصب انتخابی را دارد.

ب) هیچ نهادی حق رد صلاحیت بر اساس معیارهای غیرقانونی یا سیاسی را ندارد.

ج) احراز صلاحیت تنها بر اساس معیارهای عینی قانونی (سن، تابعیت، سابقه کیفری) و با حق اعتراض قضایی انجام می‌شود. \\
\hline

\textbf{اصل ۳۸} &
\textbf{آزادی احزاب}

الف) تشکیل و فعالیت احزاب سیاسی آزاد است.

ب) احزاب باید در ساختار و عملکرد خود دموکراتیک باشند.

ج) انحلال حزب تنها به حکم دیوان عالی قانون اساسی و به دلیل نقض اصول دموکراتیک ممکن است. \\
\hline

\textbf{اصل ۳۹} &
\textbf{حق همه‌پرسی}

الف) شهروندان می‌توانند از طریق همه‌پرسی در تصمیمات ملی مشارکت کنند.

ب) ابتکار عمل مردمی: یک میلیون امضا می‌تواند برگزاری همه‌پرسی را الزامی کند.

ج) همه‌پرسی نمی‌تواند ناقض حقوق بنیادین یا اصول غیرقابل تغییر باشد. \\
\hline

\textbf{اصل ۴۰} &
\textbf{حق دادخواهی}

هر شهروند حق دارد از نهادهای دولتی دادخواهی کند و پاسخ مکتوب دریافت نماید. \\
\hline

\textbf{اصل ۴۱} &
\textbf{دسترسی به اطلاعات}

الف) هر شهروند حق دسترسی به اطلاعات و اسناد دولتی را دارد.

ب) طبقه‌بندی اسناد باید محدود، موجه و زمان‌دار باشد.

ج) افشاگری فساد (سوت‌زنی) تحت حمایت قانون است. \\
\hline

\textbf{اصل ۴۲} &
\textbf{حق اعتراض و نافرمانی مدنی}

الف) اعتراض مسالمت‌آمیز حق شهروندی است.

ب) نافرمانی مدنی در برابر قوانین ناقض حقوق بنیادین مشروع است.

ج) حق مقاومت در برابر کودتا یا تلاش برای براندازی نظام دموکراتیک به رسمیت شناخته می‌شود. \\
\hline

\end{longtable}

\subsection{بخش سوم: حقوق اقتصادی و اجتماعی (اصول ۴۶-۶۵)}

\begin{longtable}{|>{\columncolor{SuccessGreen!15}}r|p{12cm}|}
\hline
\rowcolor{SuccessGreen!30}
\textbf{شماره اصل} & \textbf{متن اصل} \\
\hline
\endfirsthead
\hline
\rowcolor{SuccessGreen!30}
\textbf{شماره اصل} & \textbf{متن اصل} \\
\hline
\endhead

\textbf{اصل ۴۶} &
\textbf{حق کار}

الف) هر شهروند حق کار، انتخاب آزادانه شغل، و شرایط کار عادلانه را دارد.

ب) کار اجباری ممنوع است.

ج) دولت موظف به تلاش برای ایجاد اشتغال کامل است. \\
\hline

\textbf{اصل ۴۷} &
\textbf{حقوق کارگران}

الف) حق تشکیل و عضویت در سندیکا و اتحادیه‌های کارگری تضمین می‌شود.

ب) حق اعتصاب به رسمیت شناخته می‌شود.

ج) دستمزد باید عادلانه و تضمین‌کننده زندگی با کرامت باشد.

د) حداقل دستمزد به موجب قانون تعیین و سالانه بازنگری می‌شود. \\
\hline

\textbf{اصل ۴۸} &
\textbf{حق تأمین اجتماعی}

الف) هر شهروند حق بهره‌مندی از تأمین اجتماعی را دارد.

ب) بیمه بیکاری، بازنشستگی، ازکارافتادگی و بیمه درمانی همگانی تضمین می‌شود.

ج) حمایت ویژه از اقشار آسیب‌پذیر وظیفه دولت است. \\
\hline

\textbf{اصل ۴۹} &
\textbf{حق بهداشت و درمان}

الف) هر شهروند حق دسترسی به خدمات بهداشتی و درمانی را دارد.

ب) خدمات بهداشت اولیه رایگان است.

ج) بیمه درمانی پایه همگانی و اجباری است. \\
\hline

\textbf{اصل ۵۰} &
\textbf{حق مسکن}

دولت موظف است زمینه دسترسی همه شهروندان به مسکن مناسب را فراهم سازد. بی‌خانمانی باید ریشه‌کن شود. \\
\hline

\textbf{اصل ۵۱} &
\textbf{حق آموزش}

الف) آموزش رایگان و اجباری تا پایان دوره متوسطه است.

ب) آموزش عالی باید در دسترس همگان بر اساس شایستگی باشد.

ج) دولت موظف به حمایت از آموزش مادام‌العمر است. \\
\hline

\textbf{اصل ۵۲} &
\textbf{حق آب}

الف) دسترسی به آب سالم و کافی حق بنیادین هر شهروند است.

ب) دولت موظف به تأمین آب شرب سالم برای همه است.

ج) منابع آب ثروت ملی و متعلق به نسل‌های حال و آینده است. \\
\hline

\textbf{اصل ۵۳} &
\textbf{حقوق محیط زیست}

الف) هر شهروند حق زندگی در محیط زیست سالم را دارد.

ب) حمایت از محیط زیست وظیفه دولت و حق و تکلیف همه شهروندان است.

ج) آلوده‌سازی محیط زیست جرم است. \\
\hline

\end{longtable}

\subsection{بخش چهارم: حقوق فرهنگی و اقوام (اصول ۶۶-۷۵)}

\begin{longtable}{|>{\columncolor{purple!15}}r|p{12cm}|}
\hline
\rowcolor{purple!30}
\textbf{شماره اصل} & \textbf{متن اصل} \\
\hline
\endfirsthead
\hline
\rowcolor{purple!30}
\textbf{شماره اصل} & \textbf{متن اصل} \\
\hline
\endhead

\textbf{اصل ۶۶} &
\textbf{تنوع فرهنگی}

تنوع فرهنگی، قومی، زبانی و مذهبی ایران ثروت ملی است. دولت موظف به حمایت و حفاظت از این تنوع است. \\
\hline

\textbf{اصل ۶۷} &
\textbf{حقوق زبانی}

الف) هر شهروند حق استفاده از زبان مادری خود در عرصه‌های عمومی و خصوصی را دارد.

ب) آموزش به زبان مادری در کنار زبان فارسی حق هر کودک است.

ج) خدمات عمومی در مناطق دوزبانه به هر دو زبان ارائه می‌شود. \\
\hline

\textbf{اصل ۶۸} &
\textbf{حقوق قومی}

الف) هیچ قومی بر قوم دیگر برتری ندارد.

ب) هر قوم حق حفظ و توسعه هویت، فرهنگ، زبان و سنن خود را دارد.

ج) مشارکت عادلانه اقوام در قدرت سیاسی و اقتصادی تضمین می‌شود. \\
\hline

\textbf{اصل ۶۹} &
\textbf{میراث فرهنگی}

حفاظت از میراث فرهنگی، تاریخی و باستانی ایران وظیفه دولت است. غارت، تخریب یا قاچاق آثار تاریخی جرم سنگین است. \\
\hline

\textbf{اصل ۷۰} &
\textbf{آزادی علم و هنر}

تحقیق علمی و خلاقیت هنری آزاد است. دولت موظف به حمایت از علم، فناوری و هنر است. \\
\hline

\end{longtable}

\subsection{بخش پنجم: حقوق قضایی (اصول ۷۶-۸۵)}

\begin{longtable}{|>{\columncolor{orange!15}}r|p{12cm}|}
\hline
\rowcolor{orange!30}
\textbf{شماره اصل} & \textbf{متن اصل} \\
\hline
\endfirsthead
\hline
\rowcolor{orange!30}
\textbf{شماره اصل} & \textbf{متن اصل} \\
\hline
\endhead

\textbf{اصل ۷۶} &
\textbf{اصل برائت}

هر کس بی‌گناه است تا زمانی که جرم او در دادگاه صالح به اثبات برسد. بار اثبات بر عهده دادستان است. \\
\hline

\textbf{اصل ۷۷} &
\textbf{حق محاکمه عادلانه}

الف) هر کس حق دارد در مدت معقول توسط دادگاه مستقل و بی‌طرف محاکمه شود.

ب) محاکمات علنی است مگر در موارد استثنایی قانونی.

ج) هر کس حق دارد از اتهامات علیه خود مطلع شود و فرصت کافی برای دفاع داشته باشد. \\
\hline

\textbf{اصل ۷۸} &
\textbf{حق وکیل}

الف) هر کس در تمام مراحل رسیدگی حق داشتن وکیل را دارد.

ب) اگر متهم توانایی مالی نداشته باشد، دولت وکیل تسخیری تعیین می‌کند.

ج) ارتباط بین وکیل و موکل محرمانه است. \\
\hline

\textbf{اصل ۷۹} &
\textbf{اصل قانونی بودن جرم و مجازات}

الف) هیچ عملی جرم نیست مگر به موجب قانون.

ب) هیچ مجازاتی اعمال نمی‌شود مگر به موجب قانون.

ج) قوانین کیفری عطف به ماسبق نمی‌شوند مگر به نفع متهم. \\
\hline

\textbf{اصل ۸۰} &
\textbf{ممنوعیت محاکمه مجدد}

هیچ‌کس را نمی‌توان برای جرمی که قبلاً محاکمه و تبرئه یا محکوم شده دوباره محاکمه کرد. \\
\hline

\textbf{اصل ۸۱} &
\textbf{حق تجدیدنظر}

هر محکوم حق درخواست تجدیدنظر در حکم را دارد. \\
\hline

\textbf{اصل ۸۲} &
\textbf{حق جبران خسارت}

هر کس که به‌ناحق بازداشت یا محکوم شده باشد حق جبران خسارت مادی و معنوی را دارد. \\
\hline

\textbf{اصل ۸۳} &
\textbf{اصلاح مجرمان}

هدف نظام کیفری اصلاح و بازپروری است، نه انتقام. شرایط زندان‌ها باید انسانی و اصلاحی باشد. \\
\hline

\end{longtable}

% ═══════════════════════════════════════════════════════════════════════════════
\section{فصل سوم: ساختار حکومت}
\label{sec:const-ch3}
% ═══════════════════════════════════════════════════════════════════════════════

\begin{center}
\begin{tikzpicture}[
    branch/.style={
        draw=#1,
        line width=2pt,
        fill=#1!15,
        rounded corners=10pt,
        minimum width=4cm,
        minimum height=2.5cm,
        align=center,
        font=\bfseries
    },
    sub/.style={
        draw=#1!70,
        fill=#1!10,
        rounded corners=5pt,
        minimum width=2.8cm,
        minimum height=0.8cm,
        align=center,
        font=\small
    }
]
    % عنوان
    \node[
        fill=gray!20,
        draw=gray,
        line width=2pt,
        rounded corners=10pt,
        minimum width=6cm,
        minimum height=1.5cm,
        font=\Large\bfseries
    ] (title) at (0,5) {ساختار حکومت فدرال};
    
    % سه قوه
    \node[branch=DemocracyBlue] (leg) at (-5,2) {\shortstack{قوه مقننه\\(اصول ۸۶-۱۱۰)}};
    \node[branch=SuccessGreen] (exe) at (0,2) {\shortstack{قوه مجریه\\(اصول ۱۱۱-۱۳۵)}};
    \node[branch=WisdomGold] (jud) at (5,2) {\shortstack{قوه قضائیه\\(اصول ۱۳۶-۱۵۵)}};
    
    % اتصالات
    \draw[->, thick, gray] (title) -- (leg);
    \draw[->, thick, gray] (title) -- (exe);
    \draw[->, thick, gray] (title) -- (jud);
    
    % زیرمجموعه‌های قوه مقننه
    \node[sub=DemocracyBlue] (mn) at (-6.5,-0.5) {مجلس ملی};
    \node[sub=DemocracyBlue] (ma) at (-3.5,-0.5) {مجلس اقوام};
    \draw[->, DemocracyBlue!70] (leg) -- (mn);
    \draw[->, DemocracyBlue!70] (leg) -- (ma);
    
    % زیرمجموعه‌های قوه مجریه
    \node[sub=SuccessGreen] (pr) at (-1,-0.5) {رئیس‌جمهور};
    \node[sub=SuccessGreen] (cab) at (2,-0.5) {هیئت وزیران};
    \draw[->, SuccessGreen!70] (exe) -- (pr);
    \draw[->, SuccessGreen!70] (exe) -- (cab);
    
    % زیرمجموعه‌های قوه قضائیه
    \node[sub=WisdomGold] (sc) at (4,-0.5) {دیوان‌عالی کشور};
    \node[sub=WisdomGold] (cc) at (7,-0.5) {دیوان قانون اساسی};
    \draw[->, WisdomGold!70] (jud) -- (sc);
    \draw[->, WisdomGold!70] (jud) -- (cc);
    
    % نهادهای نظارتی
    \node[
        draw=WarningRed,
        line width=1.5pt,
        fill=WarningRed!10,
        rounded corners=8pt,
        minimum width=10cm,
        minimum height=1.2cm,
        font=\bfseries
    ] (oversight) at (0,-2.5) {نهادهای مستقل نظارتی (اصول ۱۵۶-۱۷۵)};
    
    \draw[->, thick, WarningRed!70, dashed] (oversight) -- (leg);
    \draw[->, thick, WarningRed!70, dashed] (oversight) -- (exe);
    \draw[->, thick, WarningRed!70, dashed] (oversight) -- (jud);
\end{tikzpicture}
\end{center}

\subsection{بخش یکم: قوه مقننه (اصول ۸۶-۱۱۰)}

\subsubsection{الف) ساختار دومجلسی}

\begin{longtable}{|>{\columncolor{DemocracyBlue!15}}r|p{12cm}|}
\hline
\rowcolor{DemocracyBlue!30}
\textbf{شماره اصل} & \textbf{متن اصل} \\
\hline
\endfirsthead
\hline
\rowcolor{DemocracyBlue!30}
\textbf{شماره اصل} & \textbf{متن اصل} \\
\hline
\endhead

\textbf{اصل ۸۶} &
\textbf{پارلمان فدرال}

قوه مقننه متشکل از دو مجلس است:

الف) \textbf{مجلس ملی} (مجلس اول): نمایندگی مستقیم مردم

ب) \textbf{مجلس اقوام} (مجلس دوم): نمایندگی مناطق و اقوام

هر دو مجلس در تصویب قوانین مشارکت دارند. \\
\hline

\textbf{اصل ۸۷} &
\textbf{مجلس ملی}

الف) مجلس ملی متشکل از ۳۵۰ نماینده است که با رأی مستقیم و مخفی مردم انتخاب می‌شوند.

ب) انتخابات به روش نسبی-فهرستی با آستانه ۳٪ برگزار می‌شود.

ج) دوره نمایندگی چهار سال است.

د) حداقل ۳۰٪ کاندیداها باید زن باشند. \\
\hline

\textbf{اصل ۸۸} &
\textbf{مجلس اقوام}

الف) مجلس اقوام متشکل از ۱۲۰ نماینده است:
\begin{itemize}[nosep]
    \item ۶۰ نماینده: ۴ نماینده از هر استان (انتخابی)
    \item ۳۰ نماینده: نمایندگان مناطق خودمختار
    \item ۲۰ نماینده: نمایندگان اقلیت‌های زبانی-فرهنگی
    \item ۱۰ نماینده: نمایندگان ایرانیان خارج از کشور
\end{itemize}

ب) دوره نمایندگی شش سال است و هر دو سال یک‌سوم اعضا تجدید می‌شوند.

ج) رئیس مجلس اقوام معاون اول رئیس‌جمهور در تشریفات است. \\
\hline

\textbf{اصل ۸۹} &
\textbf{شرایط نمایندگی}

الف) تابعیت ایرانی

ب) سن حداقل ۲۵ سال برای مجلس ملی و ۳۵ سال برای مجلس اقوام

ج) برخورداری از حقوق مدنی

د) نداشتن محکومیت کیفری مؤثر

هـ) حداقل مدرک کارشناسی \\
\hline

\textbf{اصل ۹۰} &
\textbf{مصونیت پارلمانی}

الف) نمایندگان به خاطر اظهارات و آرایشان در مجلس قابل تعقیب نیستند.

ب) بازداشت نماینده در حال انجام وظیفه مستلزم اجازه مجلس است، مگر در جرم مشهود.

ج) مصونیت شامل جرایم عادی خارج از وظایف نمایندگی نمی‌شود. \\
\hline

\textbf{اصل ۹۱} &
\textbf{تعارض منافع}

الف) نمایندگان نمی‌توانند همزمان شغل دولتی داشته باشند.

ب) نمایندگان موظف به اعلام دارایی‌ها و منافع مالی خود هستند.

ج) رأی‌دادن در موضوعاتی که نماینده ذی‌نفع است ممنوع است. \\
\hline

\end{longtable}

\subsubsection{ب) فرآیند قانون‌گذاری}

\begin{center}
\begin{tikzpicture}[
    step/.style={
        draw=DemocracyBlue,
        fill=DemocracyBlue!15,
        rounded corners=5pt,
        minimum width=2.5cm,
        minimum height=1.2cm,
        align=center,
        font=\small
    }
]
    % مراحل قانون‌گذاری
    \node[step] (s1) at (0,0) {\shortstack{پیشنهاد\\لایحه/طرح}};
    \node[step] (s2) at (3,0) {\shortstack{کمیسیون\\تخصصی}};
    \node[step] (s3) at (6,0) {\shortstack{بررسی در\\مجلس ملی}};
    \node[step] (s4) at (9,0) {\shortstack{رأی‌گیری\\مجلس ملی}};
    
    \node[step] (s5) at (9,-2) {\shortstack{ارسال به\\مجلس اقوام}};
    \node[step] (s6) at (6,-2) {\shortstack{بررسی در\\مجلس اقوام}};
    \node[step] (s7) at (3,-2) {\shortstack{رأی‌گیری\\مجلس اقوام}};
    \node[step] (s8) at (0,-2) {\shortstack{امضای\\رئیس‌جمهور}};
    
    \node[
        draw=SuccessGreen,
        fill=SuccessGreen!20,
        rounded corners=5pt,
        minimum width=2.5cm,
        minimum height=1.2cm,
        font=\small\bfseries
    ] (s9) at (0,-4) {\shortstack{انتشار و\\اجرا}};
    
    % فلش‌ها
    \draw[->, thick, DemocracyBlue] (s1) -- (s2);
    \draw[->, thick, DemocracyBlue] (s2) -- (s3);
    \draw[->, thick, DemocracyBlue] (s3) -- (s4);
    \draw[->, thick, DemocracyBlue] (s4) -- (s5);
    \draw[->, thick, DemocracyBlue] (s5) -- (s6);
    \draw[->, thick, DemocracyBlue] (s6) -- (s7);
    \draw[->, thick, DemocracyBlue] (s7) -- (s8);
    \draw[->, thick, SuccessGreen] (s8) -- (s9);
    
    % کمیته مشترک در صورت اختلاف
    \node[
        draw=WisdomGold,
        fill=WisdomGold!15,
        rounded corners=5pt,
        minimum width=2.5cm,
        minimum height=1cm,
        font=\scriptsize
    ] (joint) at (4.5,-3.5) {\shortstack{کمیته مشترک\\(در صورت اختلاف)}};
    
    \draw[->, dashed, WisdomGold] (s7) -- (joint);
    \draw[->, dashed, WisdomGold] (joint) -- (s4);
\end{tikzpicture}
\end{center}

\begin{longtable}{|>{\columncolor{DemocracyBlue!15}}r|p{12cm}|}
\hline
\rowcolor{DemocracyBlue!30}
\textbf{شماره اصل} & \textbf{متن اصل} \\
\hline
\endfirsthead
\hline
\rowcolor{DemocracyBlue!30}
\textbf{شماره اصل} & \textbf{متن اصل} \\
\hline
\endhead

\textbf{اصل ۹۲} &
\textbf{ابتکار قانون‌گذاری}

حق پیشنهاد قانون متعلق است به:

الف) هیئت وزیران (لایحه)

ب) حداقل ۱۵ نماینده مجلس ملی (طرح)

ج) حداقل ۱۰ نماینده مجلس اقوام (طرح)

د) ۵۰۰,۰۰۰ شهروند (ابتکار عمل مردمی) \\
\hline

\textbf{اصل ۹۳} &
\textbf{تصویب قوانین عادی}

الف) قوانین عادی با اکثریت ساده هر دو مجلس تصویب می‌شوند.

ب) در صورت اختلاف، کمیته مشترک تشکیل می‌شود.

ج) اگر کمیته مشترک به توافق نرسد، مجلس ملی با اکثریت مطلق تصمیم نهایی را می‌گیرد. \\
\hline

\textbf{اصل ۹۴} &
\textbf{قوانین اساسی و بنیادین}

قوانینی که به حقوق اقوام، ساختار فدرال، یا حقوق بنیادین مربوط می‌شوند، نیازمند تصویب دوسوم هر دو مجلس هستند. \\
\hline

\textbf{اصل ۹۵} &
\textbf{حق وتوی مجلس اقوام}

مجلس اقوام در موضوعات زیر حق وتوی مطلق دارد:

الف) قوانین مربوط به ساختار فدرال و تقسیمات کشوری

ب) قوانین مربوط به حقوق اقوام و زبان‌ها

ج) قوانین مربوط به توزیع منابع ملی بین مناطق

د) انتصاب قضات دیوان عالی و دیوان قانون اساسی \\
\hline

\textbf{اصل ۹۶} &
\textbf{حق وتوی رئیس‌جمهور}

الف) رئیس‌جمهور می‌تواند ظرف ۱۵ روز قانون مصوب را با ذکر دلایل به مجلس بازگرداند.

ب) مجلس می‌تواند با رأی دوسوم وتو را رد کند.

ج) رئیس‌جمهور موظف است قانون مصوب نهایی را ظرف ۱۰ روز امضا و ابلاغ کند. \\
\hline

\textbf{اصل ۹۷} &
\textbf{بودجه سالانه}

الف) دولت موظف است لایحه بودجه را تا پایان آبان‌ماه به مجلس تقدیم کند.

ب) مجلس ملی بودجه را بررسی و تصویب می‌کند.

ج) مجلس اقوام بر توزیع عادلانه بودجه بین مناطق نظارت دارد.

د) تصویب نهایی بودجه تا پایان اسفند الزامی است. \\
\hline

\textbf{اصل ۹۸} &
\textbf{نظارت پارلمانی}

الف) هیئت وزیران در برابر مجلس ملی مسئول است.

ب) نمایندگان حق سؤال و استیضاح وزرا را دارند.

ج) مجلس می‌تواند کمیسیون تحقیق تشکیل دهد.

د) رأی عدم اعتماد به هیئت وزیران با اکثریت مطلق ممکن است. \\
\hline

\end{longtable}

\subsection{بخش دوم: قوه مجریه (اصول ۱۱۱-۱۳۵)}

\subsubsection{الف) رئیس‌جمهور}

\begin{center}
\begin{tikzpicture}
    % کادر اصلی
    \node[
        draw=SuccessGreen,
        line width=2pt,
        fill=SuccessGreen!10,
        rounded corners=15pt,
        minimum width=14cm,
        minimum height=6cm
    ] (main) at (0,0) {};
    
    % عنوان
    \node[
        fill=SuccessGreen,
        text=white,
        rounded corners=5pt,
        font=\large\bfseries] at (0,2.5) {رئیس‌جمهور جمهوری فدرال ایران};
    
    % ویژگی‌های کلیدی
    \node[align=right, text width=6cm, font=\small] at (-3.5,0.5) {
        \textbf{انتخاب:} رأی مستقیم مردم\\[3pt]
        \textbf{دوره:} ۴ سال (حداکثر ۲ دوره)\\[3pt]
        \textbf{شرایط سنی:} حداقل ۴۰ سال\\[3pt]
        \textbf{تابعیت:} ایرانی‌الاصل
    };
    
    \node[align=right, text width=6cm, font=\small] at (3.5,0.5) {
        \textbf{نقش:} رئیس دولت و رئیس کشور\\[3pt]
        \textbf{مسئولیت:} در برابر ملت و مجلس\\[3pt]
        \textbf{اختیارات:} اجرایی و تشریفاتی\\[3pt]
        \textbf{انتصابات:} وزرا با تأیید مجلس
    };
    
    % نمودار وظایف
    \node[
        draw=SuccessGreen!70,
        fill=white,
        rounded corners=5pt,
        minimum width=12cm,
        minimum height=1.5cm,
        font=\small
    ] at (0,-1.8) {
        \textbf{وظایف اصلی:} اجرای قوانین | سیاست خارجی | فرماندهی کل قوا | انتصاب وزرا | پیشنهاد بودجه
    };
\end{tikzpicture}
\end{center}

\begin{longtable}{|>{\columncolor{SuccessGreen!15}}r|p{12cm}|}
\hline
\rowcolor{SuccessGreen!30}
\textbf{شماره اصل} & \textbf{متن اصل} \\
\hline
\endfirsthead
\hline
\rowcolor{SuccessGreen!30}
\textbf{شماره اصل} & \textbf{متن اصل} \\
\hline
\endhead

\textbf{اصل ۱۱۱} &
\textbf{جایگاه رئیس‌جمهور}

رئیس‌جمهور بالاترین مقام رسمی کشور پس از ملت است. رئیس‌جمهور رئیس قوه مجریه، رئیس دولت و نماینده رسمی جمهوری فدرال ایران در روابط بین‌المللی است. \\
\hline

\textbf{اصل ۱۱۲} &
\textbf{انتخاب رئیس‌جمهور}

الف) رئیس‌جمهور با رأی مستقیم، آزاد و مخفی مردم انتخاب می‌شود.

ب) کاندیدای برنده باید بیش از ۵۰٪ آرا را کسب کند.

ج) در صورت عدم کسب اکثریت مطلق، دور دوم بین دو نامزد اول برگزار می‌شود.

د) انتخابات در فاصله ۳۰ تا ۶۰ روز قبل از پایان دوره رئیس‌جمهور فعلی برگزار می‌شود. \\
\hline

\textbf{اصل ۱۱۳} &
\textbf{شرایط کاندیداتوری}

کاندیدای ریاست جمهوری باید:

الف) تابعیت ایرانی داشته و ایرانی‌الاصل باشد

ب) حداقل ۴۰ سال سن داشته باشد

ج) دارای حداقل مدرک کارشناسی ارشد باشد

د) فاقد محکومیت کیفری مؤثر باشد

هـ) حداقل ۱۰۰,۰۰۰ امضای حمایتی یا معرفی ۵۰ نماینده مجلس را داشته باشد \\
\hline

\textbf{اصل ۱۱۴} &
\textbf{دوره ریاست جمهوری}

الف) دوره ریاست جمهوری چهار سال است.

ب) هیچ‌کس نمی‌تواند بیش از دو دوره متوالی یا سه دوره در مجموع رئیس‌جمهور شود.

ج) دوره ریاست جمهوری از روز تحلیف آغاز می‌شود. \\
\hline

\textbf{اصل ۱۱۵} &
\textbf{سوگند رئیس‌جمهور}

رئیس‌جمهور منتخب در جلسه مشترک دو مجلس سوگند زیر را یاد می‌کند:

\textit{«من [نام]، در برابر ملت ایران سوگند یاد می‌کنم که از قانون اساسی پاسداری کنم، حقوق و آزادی‌های شهروندان را محترم بشمارم، استقلال و تمامیت ارضی کشور را حفظ کنم، و با تمام توان در خدمت مردم ایران باشم.»} \\
\hline

\textbf{اصل ۱۱۶} &
\textbf{وظایف و اختیارات رئیس‌جمهور}

رئیس‌جمهور:

الف) سیاست‌های کلی دولت را تعیین و بر اجرای آنها نظارت می‌کند

ب) هیئت وزیران را تشکیل داده و ریاست آن را بر عهده دارد

ج) وزرا را منصوب می‌کند (با تأیید مجلس ملی)

د) فرماندهی کل نیروهای مسلح را بر عهده دارد

هـ) معاهدات بین‌المللی را امضا می‌کند (با تصویب مجلس)

و) سفرا را منصوب و استوارنامه سفرای خارجی را می‌پذیرد

ز) عفو خصوصی اعطا می‌کند (به پیشنهاد وزیر دادگستری)

ح) قوانین مصوب را امضا و ابلاغ می‌کند \\
\hline

\textbf{اصل ۱۱۷} &
\textbf{مسئولیت رئیس‌جمهور}

الف) رئیس‌جمهور در برابر ملت، قانون اساسی و مجلس ملی مسئول است.

ب) رئیس‌جمهور سالانه گزارش عملکرد به مجلس ارائه می‌دهد.

ج) رئیس‌جمهور در دوران تصدی از مصونیت کیفری برخوردار است، مگر در موارد خیانت، فساد کلان، یا نقض فاحش قانون اساسی. \\
\hline

\textbf{اصل ۱۱۸} &
\textbf{استیضاح رئیس‌جمهور}

الف) یک‌سوم نمایندگان مجلس ملی می‌توانند طرح استیضاح رئیس‌جمهور را مطرح کنند.

ب) استیضاح باید متضمن دلایل مشخص باشد.

ج) رئیس‌جمهور حق دفاع در مجلس را دارد.

د) رأی عدم اعتماد با دوسوم آرای مجلس ملی و تأیید اکثریت مجلس اقوام تصویب می‌شود.

هـ) پس از رأی عدم اعتماد، رئیس‌جمهور برکنار و انتخابات جدید برگزار می‌شود. \\
\hline

\textbf{اصل ۱۱۹} &
\textbf{خلأ ریاست جمهوری}

در صورت فوت، استعفا، برکناری یا عجز دائم رئیس‌جمهور:

الف) معاون اول به‌طور موقت وظایف را بر عهده می‌گیرد

ب) شورای موقت متشکل از رؤسای دو مجلس و رئیس دیوان عالی، خلأ را تأیید می‌کند

ج) انتخابات جدید ظرف ۶۰ روز برگزار می‌شود

د) رئیس‌جمهور موقت حق انحلال مجلس یا تغییرات اساسی در کابینه را ندارد \\
\hline

\end{longtable}

\subsubsection{ب) هیئت وزیران}

\begin{longtable}{|>{\columncolor{SuccessGreen!15}}r|p{12cm}|}
\hline
\rowcolor{SuccessGreen!30}
\textbf{شماره اصل} & \textbf{متن اصل} \\
\hline
\endfirsthead
\hline
\rowcolor{SuccessGreen!30}
\textbf{شماره اصل} & \textbf{متن اصل} \\
\hline
\endhead

\textbf{اصل ۱۲۰} &
\textbf{ترکیب هیئت وزیران}

الف) هیئت وزیران متشکل است از رئیس‌جمهور، معاونان رئیس‌جمهور و وزرا.

ب) تعداد وزارتخانه‌ها نباید از ۲۰ وزارتخانه تجاوز کند.

ج) ترکیب کابینه باید بازتاب‌دهنده تنوع جغرافیایی، قومی و جنسیتی کشور باشد.

د) حداقل ۳۰٪ اعضای کابینه باید زن باشند. \\
\hline

\textbf{اصل ۱۲۱} &
\textbf{انتصاب وزرا}

الف) رئیس‌جمهور وزرای پیشنهادی را ظرف ۳۰ روز پس از تحلیف به مجلس ملی معرفی می‌کند.

ب) هر وزیر باید رأی اعتماد اکثریت نمایندگان را کسب کند.

ج) در صورت رد وزیر پیشنهادی، رئیس‌جمهور فرد دیگری را معرفی می‌کند.

د) اگر سه بار متوالی کابینه رأی اعتماد نگیرد، انتخابات زودهنگام برگزار می‌شود. \\
\hline

\textbf{اصل ۱۲۲} &
\textbf{وظایف هیئت وزیران}

هیئت وزیران:

الف) سیاست‌های دولت را تدوین و اجرا می‌کند

ب) لوایح قانونی را تهیه و به مجلس تقدیم می‌کند

ج) بودجه سالانه را تنظیم می‌کند

د) آیین‌نامه‌های اجرایی را تصویب می‌کند

هـ) بر عملکرد دستگاه‌های اجرایی نظارت می‌کند

و) امنیت ملی و نظم عمومی را تأمین می‌کند \\
\hline

\textbf{اصل ۱۲۳} &
\textbf{مسئولیت وزرا}

الف) هر وزیر در قبال وظایف وزارتخانه خود مسئول است.

ب) هیئت وزیران به‌طور جمعی در برابر مجلس ملی مسئول است.

ج) نمایندگان می‌توانند از وزرا سؤال کنند و پاسخ باید ظرف ۱۵ روز داده شود.

د) استیضاح وزیر با امضای ۱۵ نماینده مطرح می‌شود و رأی عدم اعتماد با اکثریت ساده کافی است. \\
\hline

\textbf{اصل ۱۲۴} &
\textbf{تعارض منافع وزرا}

الف) وزرا و معاونان نمی‌توانند شغل دیگری داشته باشند.

ب) وزرا موظف به اعلام کامل دارایی‌های خود در ابتدا و انتهای دوره هستند.

ج) وزرا نمی‌توانند در قراردادهای دولتی ذی‌نفع باشند.

د) محدودیت‌های پس از خدمت (۲ سال) برای ورود به بخش خصوصی مرتبط اعمال می‌شود. \\
\hline

\end{longtable}

\subsubsection{ج) نیروهای مسلح}

\begin{longtable}{|>{\columncolor{SuccessGreen!15}}r|p{12cm}|}
\hline
\rowcolor{SuccessGreen!30}
\textbf{شماره اصل} & \textbf{متن اصل} \\
\hline
\endfirsthead
\hline
\rowcolor{SuccessGreen!30}
\textbf{شماره اصل} & \textbf{متن اصل} \\
\hline
\endhead

\textbf{اصل ۱۲۵} &
\textbf{وظیفه نیروهای مسلح}

نیروهای مسلح جمهوری فدرال ایران موظف به:

الف) دفاع از استقلال و تمامیت ارضی کشور

ب) حمایت از نظام قانون اساسی و نظم دموکراتیک

ج) کمک در بلایای طبیعی و شرایط اضطراری

نیروهای مسلح ابزار سیاست خارجی تهاجمی نیستند. \\
\hline

\textbf{اصل ۱۲۶} &
\textbf{فرماندهی نیروهای مسلح}

الف) رئیس‌جمهور فرمانده کل نیروهای مسلح است.

ب) شورای عالی دفاع ملی به ریاست رئیس‌جمهور تشکیل می‌شود.

ج) اعلام جنگ و صلح مستلزم تصویب دوسوم مجلس ملی است.

د) اعزام نیرو به خارج از کشور نیازمند تصویب مجلس است. \\
\hline

\textbf{اصل ۱۲۷} &
\textbf{غیرسیاسی بودن ارتش}

الف) نیروهای مسلح از سیاست‌ورزی و فعالیت حزبی منع هستند.

ب) نظامیان در دوران خدمت حق رأی دارند اما حق کاندیداتوری ندارند.

ج) دخالت نظامیان در امور سیاسی جرم سنگین است.

د) نظارت پارلمانی بر نیروهای مسلح الزامی است. \\
\hline

\textbf{اصل ۱۲۸} &
\textbf{بودجه دفاعی}

الف) بودجه دفاعی باید به تصویب مجلس برسد.

ب) کمیسیون دفاع مجلس بر هزینه‌های نظامی نظارت دارد.

ج) گزارش سالانه عملکرد مالی نیروهای مسلح به مجلس ارائه می‌شود. \\
\hline

\textbf{اصل ۱۲۹} &
\textbf{ممنوعیت کودتا}

الف) هرگونه تلاش نظامی برای تغییر نظام قانون اساسی کودتا محسوب می‌شود.

ب) اطاعت از دستورات غیرقانونی و ضد قانون اساسی ممنوع است.

ج) شهروندان و نهادها حق مقاومت در برابر کودتا را دارند. \\
\hline

\end{longtable}

\subsection{بخش سوم: قوه قضائیه (اصول ۱۳۶-۱۵۵)}

\begin{center}
\begin{tikzpicture}[
    court/.style={
        draw=WisdomGold,
        line width=1.5pt,
        fill=WisdomGold!15,
        rounded corners=8pt,
        minimum width=4cm,
        minimum height=1.5cm,
        align=center,
        font=\small\bfseries
    }
]
    % عنوان
    \node[
        fill=WisdomGold,
        text=white,
        rounded corners=5pt,
        font=\large\bfseries,
        minimum width=8cm
    ] at (0,4) {ساختار قوه قضائیه};
    
    % دیوان عالی قانون اساسی
    \node[court, fill=WisdomGold!30] (cc) at (0,2.5) {\shortstack{دیوان عالی قانون اساسی\\(۱۵ قاضی)}};
    
    % دیوان عالی کشور
    \node[court] (sc) at (-4,0.5) {\shortstack{دیوان عالی کشور\\(رئیس + ۲۴ قاضی)}};
    
    % دادگاه عالی اداری
    \node[court] (ac) at (4,0.5) {\shortstack{دیوان عدالت اداری\\(۱۲ قاضی)}};
    
    % دادگاه‌های استیناف
    \node[court, fill=WisdomGold!10] (app) at (-4,-1.5) {\shortstack{دادگاه‌های استیناف\\(استانی)}};
    
    % دادگاه‌های بدوی
    \node[court, fill=WisdomGold!10] (first) at (4,-1.5) {\shortstack{دادگاه‌های بدوی\\(شهرستانی)}};
    
    % دادگاه‌های تخصصی
    \node[court, fill=gray!10, draw=gray] (spec) at (0,-1.5) {\shortstack{دادگاه‌های تخصصی\\(خانواده، کار، تجارت)}};
    
    % شورای عالی قضایی
    \node[
        draw=WarningRed,
        fill=WarningRed!10,
        rounded corners=5pt,
        minimum width=4cm,
        minimum height=1cm,
        font=\small\bfseries
    ] (council) at (0,-3.5) {شورای عالی قضایی};
    
    % اتصالات
    \draw[->, thick, WisdomGold] (cc) -- (sc);
    \draw[->, thick, WisdomGold] (cc) -- (ac);
    \draw[->, thick, WisdomGold!70] (sc) -- (app);
    \draw[->, thick, WisdomGold!70] (app) -- (first);
    \draw[->, thick, gray] (spec) -- (sc);
    \draw[->, thick, WarningRed, dashed] (council) -- (sc);
    \draw[->, thick, WarningRed, dashed] (council) -- (ac);
\end{tikzpicture}
\end{center}

\begin{longtable}{|>{\columncolor{WisdomGold!15}}r|p{12cm}|}
\hline
\rowcolor{WisdomGold!30}
\textbf{شماره اصل} & \textbf{متن اصل} \\
\hline
\endfirsthead
\hline
\rowcolor{WisdomGold!30}
\textbf{شماره اصل} & \textbf{متن اصل} \\
\hline
\endhead

\textbf{اصل ۱۳۶} &
\textbf{استقلال قضایی}

الف) قوه قضائیه مستقل است و هیچ نهادی حق دخالت در امور قضایی را ندارد.

ب) قضات در صدور رأی مستقل‌اند و تنها تابع قانون هستند.

ج) بودجه قضایی باید کافی و مستقل از قوه مجریه تأمین شود. \\
\hline

\textbf{اصل ۱۳۷} &
\textbf{شورای عالی قضایی}

الف) شورای عالی قضایی بالاترین نهاد اداری قوه قضائیه است.

ب) ترکیب شورا: رئیس دیوان عالی کشور، دادستان کل، ۵ قاضی منتخب قضات، ۳ حقوقدان منتخب مجلس

ج) وظایف: انتصاب و ترفیع قضات، نظارت بر عملکرد دادگاه‌ها، تدوین مقررات قضایی \\
\hline

\textbf{اصل ۱۳۸} &
\textbf{انتصاب قضات}

الف) قضات توسط شورای عالی قضایی بر اساس شایستگی و استقلال منصوب می‌شوند.

ب) قضات دیوان عالی کشور با پیشنهاد شورا و تأیید مجلس اقوام منصوب می‌شوند.

ج) قضات دیوان قانون اساسی با رأی دوسوم مجلس ملی و تأیید مجلس اقوام منصوب می‌شوند. \\
\hline

\textbf{اصل ۱۳۹} &
\textbf{تصدی قضاوت}

الف) قضات دارای تصدی مادام‌العمر تا سن بازنشستگی (۷۰ سال) هستند.

ب) عزل قاضی تنها به حکم دادگاه انتظامی قضات و در موارد قانونی ممکن است.

ج) انتقال قاضی بدون رضایت وی ممنوع است مگر در موارد قانونی. \\
\hline

\textbf{اصل ۱۴۰} &
\textbf{دیوان عالی کشور}

الف) دیوان عالی کشور بالاترین مرجع قضایی در امور مدنی و کیفری است.

ب) وظایف: نظارت بر حسن اجرای قوانین، ایجاد رویه قضایی واحد، رسیدگی به فرجام‌خواهی

ج) آرای دیوان در ایجاد رویه قضایی برای دادگاه‌های پایین‌تر الزام‌آور است. \\
\hline

\textbf{اصل ۱۴۱} &
\textbf{دیوان عالی قانون اساسی}

الف) دیوان عالی قانون اساسی پاسدار قانون اساسی و حقوق بنیادین است.

ب) ترکیب: ۱۵ قاضی با دوره ۱۲ ساله غیرقابل تمدید (هر ۴ سال، ۵ نفر تجدید)

ج) صلاحیت‌ها:
\begin{itemize}[nosep]
    \item بررسی انطباق قوانین با قانون اساسی
    \item رسیدگی به شکایات نقض حقوق بنیادین
    \item حل اختلاف بین قوا و بین مرکز و مناطق
    \item تفسیر قانون اساسی
    \item نظارت بر انتخابات ملی و همه‌پرسی
\end{itemize} \\
\hline

\textbf{اصل ۱۴۲} &
\textbf{دسترسی به دیوان قانون اساسی}

حق طرح شکایت در دیوان قانون اساسی متعلق است به:

الف) هر شهروند که حقوق بنیادینش نقض شده باشد

ب) رئیس‌جمهور، رئیس هر مجلس

ج) یک‌پنجم نمایندگان هر مجلس

د) دولت‌های محلی (در موضوعات فدرالی)

هـ) دادگاه‌ها هنگام شک در قانون اساسی بودن قانون \\
\hline

\textbf{اصل ۱۴۳} &
\textbf{دیوان عدالت اداری}

الف) دیوان عدالت اداری مرجع رسیدگی به شکایات از دستگاه‌های دولتی است.

ب) هر شهروند حق شکایت از تصمیمات اداری را دارد.

ج) دیوان می‌تواند تصمیمات غیرقانونی را ابطال و حکم به جبران خسارت دهد. \\
\hline

\textbf{اصل ۱۴۴} &
\textbf{دادستانی کل}

الف) دادستان کل مسئول تعقیب جرایم و نظارت بر اجرای قانون است.

ب) دادستان کل توسط شورای عالی قضایی پیشنهاد و با تأیید مجلس منصوب می‌شود.

ج) دادستانی مستقل است و تحت نظارت مجلس قرار دارد. \\
\hline

\textbf{اصل ۱۴۵} &
\textbf{دادرسی منصفانه}

الف) محاکمات باید علنی باشد مگر در موارد استثنایی قانونی.

ب) هر فرد حق داشتن وکیل و فرصت کافی برای دفاع را دارد.

ج) احکام باید مستدل و مکتوب باشند.

د) حق تجدیدنظر در همه احکام تضمین می‌شود. \\
\hline

\end{longtable}

% ═══════════════════════════════════════════════════════════════════════════════
\section{فصل چهارم: مجلس اقوام و ساختار فدرال}
\label{sec:const-ch4}
% ═══════════════════════════════════════════════════════════════════════════════

\begin{olgoobox}
\textbf{فدرالیسم همبسته: الگوی ایرانی}

ساختار فدرال جمهوری فدرال ایران بر اساس اصل «وحدت در کثرت» طراحی شده است. این ساختار:

\begin{itemize}[nosep]
    \item تنوع قومی-فرهنگی را به رسمیت می‌شناسد
    \item خودمختاری منطقه‌ای را تضمین می‌کند
    \item همبستگی و وحدت ملی را حفظ می‌کند
    \item از تجزیه‌طلبی جلوگیری می‌کند
\end{itemize}
\end{olgoobox}

\begin{center}
\begin{tikzpicture}[
    level/.style={
        draw=DemocracyBlue,
        line width=1.5pt,
        fill=DemocracyBlue!15,
        rounded corners=8pt,
        minimum width=12cm,
        minimum height=1.5cm,
        align=center,
        font=\bfseries
    }
]
    % سطوح حکومت
    \node[level, fill=DemocracyBlue!30] (fed) at (0,3) {سطح فدرال: حاکمیت ملی، دفاع، سیاست خارجی، اقتصاد کلان};
    
    \node[level, fill=SuccessGreen!20, draw=SuccessGreen] (reg) at (0,1) {سطح منطقه‌ای: ۵ منطقه خودمختار + ۱۵ استان};
    
    \node[level, fill=WisdomGold!20, draw=WisdomGold] (prov) at (0,-1) {سطح استانی: آموزش، بهداشت، عمران محلی};
    
    \node[level, fill=purple!15, draw=purple] (loc) at (0,-3) {سطح محلی: شهرداری‌ها و شوراهای روستایی};
    
    % فلش‌های دوطرفه
    \draw[<->, thick, gray] (fed) -- (reg);
    \draw[<->, thick, gray] (reg) -- (prov);
    \draw[<->, thick, gray] (prov) -- (loc);
    
    % برچسب‌ها
    \node[right=1cm of fed, font=\small, align=right, text width=3cm] {صلاحیت انحصاری فدرال};
    \node[right=1cm of reg, font=\small, align=right, text width=3cm] {صلاحیت مشترک};
    \node[right=1cm of prov, font=\small, align=right, text width=3cm] {صلاحیت محلی};
    \node[right=1cm of loc, font=\small, align=right, text width=3cm] {خدمات شهری};
\end{tikzpicture}
\end{center}

\begin{longtable}{|>{\columncolor{DemocracyBlue!15}}r|p{12cm}|}
\hline
\rowcolor{DemocracyBlue!30}
\textbf{شماره اصل} & \textbf{متن اصل} \\
\hline
\endfirsthead
\hline
\rowcolor{DemocracyBlue!30}
\textbf{شماره اصل} & \textbf{متن اصل} \\
\hline
\endhead

\textbf{اصل ۱۴۶} &
\textbf{تقسیمات کشوری}

جمهوری فدرال ایران متشکل است از:

الف) پنج منطقه خودمختار: آذربایجان، کردستان، بلوچستان، خوزستان (عربستان)، ترکمن‌صحرا

ب) پانزده استان: تهران، اصفهان، فارس، خراسان شمالی، خراسان رضوی، خراسان جنوبی، گیلان، مازندران، لرستان، کرمان، یزد، سمنان، قم، مرکزی، ساحلی جنوب

ج) شهرستان‌ها و بخش‌ها

د) تغییر مرزهای منطقه‌ای مستلزم تصویب دوسوم مجلس اقوام و همه‌پرسی منطقه‌ای است. \\
\hline

\textbf{اصل ۱۴۷} &
\textbf{مناطق خودمختار}

الف) مناطق خودمختار دارای پارلمان و دولت منطقه‌ای منتخب هستند.

ب) زبان قومی در منطقه خودمختار زبان رسمی دوم است.

ج) مناطق خودمختار در چارچوب قانون اساسی، اساسنامه خود را تصویب می‌کنند.

د) فرماندار منطقه توسط پارلمان منطقه‌ای انتخاب و توسط رئیس‌جمهور تأیید می‌شود. \\
\hline

\textbf{اصل ۱۴۸} &
\textbf{صلاحیت انحصاری فدرال}

صلاحیت‌های انحصاری دولت فدرال:

الف) دفاع ملی و نیروهای مسلح

ب) سیاست خارجی و معاهدات بین‌المللی

ج) سیاست پولی، بانک مرکزی و ارز

د) گمرک و تجارت خارجی

هـ) مهاجرت و تابعیت

و) ارتباطات و حمل‌ونقل بین‌استانی

ز) انرژی هسته‌ای و منابع استراتژیک \\
\hline

\textbf{اصل ۱۴۹} &
\textbf{صلاحیت مشترک}

صلاحیت‌های مشترک فدرال و منطقه‌ای:

الف) آموزش عالی و تحقیقات

ب) بهداشت عمومی

ج) حفاظت از محیط زیست

د) مدیریت منابع آب

هـ) توسعه اقتصادی و صنعتی

در موارد تعارض، قانون فدرال ارجحیت دارد مگر در حوزه‌های خاص منطقه‌ای. \\
\hline

\textbf{اصل ۱۵۰} &
\textbf{صلاحیت منطقه‌ای و محلی}

صلاحیت‌های انحصاری مناطق و استان‌ها:

الف) آموزش ابتدایی و متوسطه (در چارچوب استانداردهای ملی)

ب) فرهنگ و میراث محلی

ج) پلیس محلی و امنیت داخلی

د) عمران و شهرسازی

هـ) خدمات اجتماعی و رفاهی محلی

و) گردشگری منطقه‌ای \\
\hline

\textbf{اصل ۱۵۱} &
\textbf{توزیع منابع}

الف) درآمدهای نفت و گاز متعلق به کل ملت ایران است.

ب) صندوق توازن منطقه‌ای برای توزیع عادلانه منابع تشکیل می‌شود.

ج) فرمول توزیع: ۵۰٪ جمعیت، ۲۵٪ مساحت، ۱۵٪ محرومیت، ۱۰٪ عملکرد

د) مناطق تولیدکننده منابع طبیعی سهم اضافی (۵٪) دریافت می‌کنند.

هـ) شورای توازن مالی بر اجرای عادلانه نظارت دارد. \\
\hline

\textbf{اصل ۱۵۲} &
\textbf{همکاری بین‌منطقه‌ای}

الف) مناطق می‌توانند برای پروژه‌های مشترک همکاری کنند.

ب) شورای فرمانداران به ریاست رئیس‌جمهور برای هماهنگی تشکیل می‌شود.

ج) اختلافات بین‌منطقه‌ای توسط دیوان قانون اساسی حل‌وفصل می‌شود. \\
\hline

\end{longtable}

% ═══════════════════════════════════════════════════════════════════════════════
\section{فصل پنجم: نهادهای مستقل نظارتی}
\label{sec:const-ch5}
% ═══════════════════════════════════════════════════════════════════════════════

\begin{center}
\begin{tikzpicture}
    % عنوان مرکزی
    \node[
        draw=WarningRed,
        line width=2pt,
        fill=WarningRed!15,
        rounded corners=10pt,
        minimum width=5cm,
        minimum height=1.5cm,
        font=\large\bfseries
    ] (center) at (0,0) {نهادهای نظارتی مستقل};
    
    % نهادها در اطراف
    \foreach \angle/\name/\desc in {
        60/{کمیسیون انتخابات}/{برگزاری و نظارت},
        120/{بانک مرکزی}/{سیاست پولی مستقل},
        180/{سازمان حسابرسی}/{نظارت مالی},
        240/{کمیسیون حقوق بشر}/{حمایت از حقوق},
        300/{سازمان ضد فساد}/{مبارزه با فساد},
        0/{کمیسیون رسانه}/{آزادی رسانه}
    } {
        \node[
            draw=WarningRed!70,
            fill=white,
            rounded corners=5pt,
            minimum width=3cm,
            minimum height=1.2cm,
            align=center,
            font=\scriptsize
        ] (\angle) at (\angle:4cm) {\textbf{\name}\\\desc};
        
        \draw[->, thick, WarningRed!50] (center) -- (\angle);
    }
\end{tikzpicture}
\end{center}

\begin{longtable}{|>{\columncolor{WarningRed!15}}r|p{12cm}|}
\hline
\rowcolor{WarningRed!30}
\textbf{شماره اصل} & \textbf{متن اصل} \\
\hline
\endfirsthead
\hline
\rowcolor{WarningRed!30}
\textbf{شماره اصل} & \textbf{متن اصل} \\
\hline
\endhead

\textbf{اصل ۱۵۳} &
\textbf{اصل نهادهای مستقل}

نهادهای نظارتی مستقل رکن چهارم نظام حکومتی هستند. این نهادها:

الف) از قوای سه‌گانه مستقل‌اند

ب) رؤسای آنها با رأی دوسوم مجلس منصوب می‌شوند

ج) بودجه مستقل دارند

د) تنها در برابر مجلس پاسخگو هستند

هـ) از مداخله سیاسی مصون‌اند \\
\hline

\textbf{اصل ۱۵۴} &
\textbf{کمیسیون مستقل انتخابات}

الف) کمیسیون انتخابات مسئول برگزاری، نظارت و اعلام نتایج همه انتخابات است.

ب) ترکیب: ۱۱ عضو (۵ قاضی، ۳ حقوقدان، ۳ متخصص انتخابات)

ج) دوره عضویت: ۷ سال غیرقابل تمدید

د) اعضا نباید وابستگی حزبی داشته باشند

هـ) تصمیمات کمیسیون در دیوان قانون اساسی قابل اعتراض است \\
\hline

\textbf{اصل ۱۵۵} &
\textbf{بانک مرکزی مستقل}

الف) بانک مرکزی جمهوری فدرال ایران نهاد مستقل سیاست پولی است.

ب) هدف اصلی: ثبات قیمت‌ها و ارزش پول ملی

ج) رئیس بانک مرکزی با پیشنهاد رئیس‌جمهور و تأیید دوسوم مجلس منصوب می‌شود

د) دوره ریاست: ۶ سال (حداکثر ۲ دوره)

هـ) دولت حق استقراض مستقیم از بانک مرکزی را ندارد \\
\hline

\textbf{اصل ۱۵۶} &
\textbf{سازمان حسابرسی کل کشور}

الف) سازمان حسابرسی نهاد عالی نظارت مالی است.

ب) وظایف: حسابرسی همه نهادهای دولتی، شرکت‌های دولتی، و نهادهای عمومی

ج) گزارش سالانه به مجلس ارائه می‌شود

د) دسترسی نامحدود به همه اسناد مالی دارد \\
\hline

\textbf{اصل ۱۵۷} &
\textbf{کمیسیون ملی حقوق بشر}

الف) کمیسیون حقوق بشر نهاد حمایت و ارتقای حقوق بنیادین است.

ب) وظایف: دریافت شکایات، تحقیق، توصیه، آموزش حقوق بشر

ج) ترکیب: ۱۵ عضو از حقوقدانان، فعالان مدنی، و نمایندگان اقلیت‌ها

د) گزارش سالانه به مجلس و عموم \\
\hline

\textbf{اصل ۱۵۸} &
\textbf{سازمان ملی مبارزه با فساد}

الف) سازمان ضد فساد نهاد مستقل پیشگیری و مبارزه با فساد است.

ب) صلاحیت: تحقیق درباره فساد در همه سطوح، از جمله مقامات عالی

ج) حفاظت از افشاگران (سوت‌زنان)

د) همکاری با دادستانی برای تعقیب متخلفین

هـ) انتشار شاخص سالانه فساد \\
\hline

\textbf{اصل ۱۵۹} &
\textbf{شورای عالی رسانه}

الف) شورای رسانه نهاد تنظیم‌گر مستقل بخش رسانه است.

ب) وظایف: صدور مجوز، نظارت بر استانداردها، حفاظت از تکثرگرایی

ج) ممنوعیت سانسور دولتی

د) صدا و سیما باید بی‌طرف و متکثر باشد

هـ) شورا نمی‌تواند محتوا را سانسور کند، تنها نظارت پس از انتشار \\
\hline

\textbf{اصل ۱۶۰} &
\textbf{نهاد بازرسی کل کشور}

الف) بازرسی کل نهاد رسیدگی به شکایات از دستگاه‌های اجرایی است.

ب) هر شهروند می‌تواند بدون هزینه شکایت کند

ج) بازرس کل حق بازرسی از همه نهادها را دارد

د) توصیه‌های بازرسی لازم‌الاجرا است \\
\hline

\end{longtable}

% ═══════════════════════════════════════════════════════════════════════════════
\section{فصل ششم: حکومت محلی و شوراها}
\label{sec:const-ch6}
% ═══════════════════════════════════════════════════════════════════════════════

\begin{longtable}{|>{\columncolor{purple!15}}r|p{12cm}|}
\hline
\rowcolor{purple!30}
\textbf{شماره اصل} & \textbf{متن اصل} \\
\hline
\endfirsthead
\hline
\rowcolor{purple!30}
\textbf{شماره اصل} & \textbf{متن اصل} \\
\hline
\endhead

\textbf{اصل ۱۶۱} &
\textbf{اصل تمرکززدایی}

اداره امور محلی بر اساس اصل تمرکززدایی و واگذاری اختیارات به پایین‌ترین سطح ممکن (اصل تقرب) انجام می‌شود. \\
\hline

\textbf{اصل ۱۶۲} &
\textbf{شوراهای محلی}

الف) در هر شهر و روستا شورای منتخب تشکیل می‌شود.

ب) اعضای شورا با رأی مستقیم مردم انتخاب می‌شوند.

ج) دوره شوراها چهار سال است.

د) شوراها مسئول نظارت بر اداره محلی و تصویب بودجه محلی هستند. \\
\hline

\textbf{اصل ۱۶۳} &
\textbf{شهرداری‌ها}

الف) شهردار توسط شورای شهر انتخاب می‌شود.

ب) شهرداری‌ها استقلال مالی و اداری دارند.

ج) منابع شهرداری: مالیات محلی، عوارض، کمک‌های دولتی

د) شهرداری‌ها در برابر شورا و شهروندان پاسخگو هستند. \\
\hline

\textbf{اصل ۱۶۴} &
\textbf{استانداران}

الف) استاندار نماینده دولت مرکزی در استان است.

ب) استاندار توسط وزیر کشور پیشنهاد و با تأیید شورای استان منصوب می‌شود.

ج) وظایف: هماهنگی بین دولت مرکزی و محلی، نظارت بر اجرای قوانین ملی \\
\hline

\textbf{اصل ۱۶۵} &
\textbf{شوراهای استانی}

الف) شورای استان متشکل از نمایندگان شوراهای شهر و روستا است.

ب) وظایف: برنامه‌ریزی توسعه استانی، هماهنگی بین شهرستان‌ها

ج) شورای استان بودجه استانی را تصویب می‌کند. \\
\hline

\textbf{اصل ۱۶۶} &
\textbf{مشارکت مردمی}

الف) شهروندان حق مشارکت مستقیم در تصمیمات محلی را دارند.

ب) همه‌پرسی محلی برای موضوعات مهم اجباری است.

ج) جلسات شوراها علنی است و شهروندان حق حضور دارند.

د) بودجه مشارکتی در شهرهای بزرگ الزامی است. \\
\hline

\end{longtable}

% ═══════════════════════════════════════════════════════════════════════════════
\section{فصل هفتم: بازنگری در قانون اساسی}
\label{sec:const-ch7}
% ═══════════════════════════════════════════════════════════════════════════════

\begin{center}
\begin{tikzpicture}[
    step/.style={
        draw=DemocracyBlue,
        fill=DemocracyBlue!15,
        rounded corners=8pt,
        minimum width=3.5cm,
        minimum height=2cm,
        align=center,
        font=\small
    }
]
    % عنوان
    \node[
        fill=DemocracyBlue,
        text=white,
        rounded corners=5pt,
        font=\large\bfseries,
        minimum width=10cm
    ] at (0,4) {فرآیند بازنگری قانون اساسی};
    
    % مراحل
    \node[step] (s1) at (-5,1.5) {\textbf{مرحله ۱}\\پیشنهاد\\(دوسوم هر مجلس\\یا ۲ میلیون امضا)};
    
    \node[step] (s2) at (0,1.5) {\textbf{مرحله ۲}\\بررسی\\(کمیته ویژه\\بازنگری)};
    
    \node[step] (s3) at (5,1.5) {\textbf{مرحله ۳}\\تصویب\\(سه‌چهارم\\هر دو مجلس)};
    
    \node[step, fill=SuccessGreen!20, draw=SuccessGreen] (s4) at (0,-1.5) {\textbf{مرحله ۴}\\همه‌پرسی\\(اکثریت مطلق\\با مشارکت ۵۰٪)};
    
    % فلش‌ها
    \draw[->, thick, DemocracyBlue] (s1) -- (s2);
    \draw[->, thick, DemocracyBlue] (s2) -- (s3);
    \draw[->, thick, DemocracyBlue] (s3) -- (s4);
    
    % خط قرمز
    \node[
        draw=WarningRed,
        fill=WarningRed!10,
        rounded corners=5pt,
        minimum width=10cm,
        font=\small
    ] at (0,-3.5) {\textbf{ (!)  اصول غیرقابل تغییر (اصل ۱۱) قابل بازنگری نیستند}};
\end{tikzpicture}
\end{center}

\begin{longtable}{|>{\columncolor{DemocracyBlue!15}}r|p{12cm}|}
\hline
\rowcolor{DemocracyBlue!30}
\textbf{شماره اصل} & \textbf{متن اصل} \\
\hline
\endfirsthead
\hline
\rowcolor{DemocracyBlue!30}
\textbf{شماره اصل} & \textbf{متن اصل} \\
\hline
\endhead

\textbf{اصل ۱۶۷} &
\textbf{اصل بازنگری}

قانون اساسی سندی زنده است که می‌تواند با تغییر شرایط و نیازهای جامعه بازنگری شود، مشروط به رعایت اصول غیرقابل تغییر و تشریفات پیش‌بینی‌شده. \\
\hline

\textbf{اصل ۱۶۸} &
\textbf{پیشنهاد بازنگری}

حق پیشنهاد بازنگری متعلق است به:

الف) رئیس‌جمهور با موافقت هیئت وزیران

ب) دوسوم نمایندگان مجلس ملی

ج) دوسوم نمایندگان مجلس اقوام

د) دو میلیون شهروند واجد شرایط رأی \\
\hline

\textbf{اصل ۱۶۹} &
\textbf{کمیته بازنگری}

الف) پس از پذیرش پیشنهاد، کمیته ویژه بازنگری تشکیل می‌شود.

ب) ترکیب: ۲۰ نماینده مجلس ملی، ۱۰ نماینده مجلس اقوام، ۵ قاضی دیوان قانون اساسی، ۱۵ حقوقدان و صاحب‌نظر

ج) کمیته ظرف ۶ ماه متن پیشنهادی را تهیه می‌کند. \\
\hline

\textbf{اصل ۱۷۰} &
\textbf{تصویب در مجلس}

الف) متن پیشنهادی باید به تصویب سه‌چهارم نمایندگان هر دو مجلس برسد.

ب) رأی‌گیری علنی و اسمی است.

ج) فاصله دو قرائت حداقل ۶۰ روز است.

د) دیوان قانون اساسی انطباق با اصول غیرقابل تغییر را تأیید می‌کند. \\
\hline

\textbf{اصل ۱۷۱} &
\textbf{همه‌پرسی}

الف) هر بازنگری در قانون اساسی مستلزم تأیید در همه‌پرسی است.

ب) همه‌پرسی ظرف ۹۰ روز پس از تصویب مجلس برگزار می‌شود.

ج) تصویب نیازمند اکثریت مطلق آرای مأخوذه با مشارکت حداقل ۵۰٪ واجدین شرایط است.

د) اگر بازنگری به حقوق منطقه خاصی مربوط شود، اکثریت آن منطقه نیز لازم است. \\
\hline

\textbf{اصل ۱۷۲} &
\textbf{محدودیت‌های بازنگری}

الف) اصول غیرقابل تغییر (اصل ۱۱) به هیچ وجه قابل بازنگری نیستند.

ب) در دوران حکومت نظامی یا وضعیت اضطراری بازنگری ممنوع است.

ج) در سال آخر دوره ریاست جمهوری بازنگری ممنوع است.

د) هر اصل تنها یک‌بار در هر ۱۰ سال قابل بازنگری است (مگر برای اصلاح فنی). \\
\hline

\textbf{اصل ۱۷۳} &
\textbf{اعلام و اجرا}

الف) رئیس‌جمهور موظف است بازنگری مصوب را ظرف ۱۵ روز اعلام کند.

ب) تغییرات از تاریخ اعلام لازم‌الاجرا است مگر تاریخ دیگری تعیین شده باشد.

ج) متن جدید در روزنامه رسمی و رسانه‌های عمومی منتشر می‌شود. \\
\hline

\end{longtable}

% ═══════════════════════════════════════════════════════════════════════════════
\section{فصل هشتم: مقررات انتقالی}
\label{sec:const-ch8}
% ═══════════════════════════════════════════════════════════════════════════════

\begin{enghelabbox}
\textbf{ (!)  اهمیت مقررات انتقالی}

مقررات انتقالی پل ارتباطی بین نظام قدیم و نظام جدید هستند. این مقررات:
\begin{itemize}[nosep]
    \item چارچوب زمانی مشخص برای اجرای قانون اساسی تعیین می‌کنند
    \item از خلأ قانونی و هرج‌ومرج جلوگیری می‌کنند
    \item حقوق مکتسبه را در حد معقول حفظ می‌کنند
    \item پس از اتمام دوره انتقال، خودبه‌خود منقضی می‌شوند
\end{itemize}
\end{enghelabbox}

\begin{longtable}{|>{\columncolor{gray!15}}r|p{12cm}|}
\hline
\rowcolor{gray!30}
\textbf{شماره اصل} & \textbf{متن اصل} \\
\hline
\endfirsthead
\hline
\rowcolor{gray!30}
\textbf{شماره اصل} & \textbf{متن اصل} \\
\hline
\endhead

\textbf{اصل ۱۷۴} &
\textbf{لازم‌الاجرا شدن}

این قانون اساسی از تاریخ تصویب در همه‌پرسی لازم‌الاجرا می‌شود. دولت موقت موظف است ظرف ۳۰ روز تدابیر لازم برای اجرا را اتخاذ کند. \\
\hline

\textbf{اصل ۱۷۵} &
\textbf{دوره انتقالی}

الف) دوره انتقالی از تاریخ تصویب قانون اساسی تا برگزاری اولین انتخابات عمومی است.

ب) این دوره نباید از ۱۸ ماه تجاوز کند.

ج) دولت موقت تا تشکیل دولت منتخب به کار ادامه می‌دهد. \\
\hline

\textbf{اصل ۱۷۶} &
\textbf{اولین انتخابات}

الف) اولین انتخابات ریاست جمهوری و مجلس ظرف ۱۲ ماه پس از تصویب قانون اساسی برگزار می‌شود.

ب) کمیسیون موقت انتخابات توسط دولت موقت با مشورت نیروهای سیاسی تشکیل می‌شود.

ج) انتخابات تحت نظارت بین‌المللی برگزار می‌شود. \\
\hline

\textbf{اصل ۱۷۷} &
\textbf{تشکیل نهادها}

جدول زمانی تشکیل نهادهای قانون اساسی:

\begin{tabular}{r r}
\textbf{نهاد} & \textbf{مهلت تشکیل} \\
\hline
کمیسیون انتخابات & ۳ ماه \\
مجلس ملی و مجلس اقوام & ۱۲ ماه \\
دیوان عالی قانون اساسی & ۱۵ ماه \\
شورای عالی قضایی & ۱۵ ماه \\
پارلمان‌های منطقه‌ای & ۱۸ ماه \\
شوراهای محلی & ۲۴ ماه \\
\end{tabular} \\
\hline

\textbf{اصل ۱۷۸} &
\textbf{قوانین موجود}

الف) قوانین موجود تا زمانی که صریحاً نسخ نشده یا با این قانون اساسی مغایر نباشند، معتبر می‌مانند.

ب) مجلس جدید موظف است ظرف ۵ سال کلیه قوانین را با قانون اساسی تطبیق دهد.

ج) دیوان قانون اساسی می‌تواند قوانین مغایر را باطل اعلام کند. \\
\hline

\textbf{اصل ۱۷۹} &
\textbf{کارکنان دولت}

الف) کارکنان دولت که مرتکب نقض حقوق بشر نشده‌اند، در سمت خود باقی می‌مانند.

ب) کمیسیون بررسی صلاحیت برای مقامات ارشد تشکیل می‌شود.

ج) حقوق بازنشستگی کارکنان سابق محفوظ است.

د) افرادی که به دلیل باورهای سیاسی اخراج شده‌اند، حق بازگشت دارند. \\
\hline

\textbf{اصل ۱۸۰} &
\textbf{عدالت انتقالی}

الف) کمیسیون حقیقت و آشتی برای بررسی نقض حقوق بشر در گذشته تشکیل می‌شود.

ب) عاملان جنایات علیه بشریت تحت پیگرد قرار می‌گیرند.

ج) قربانیان حق جبران خسارت دارند.

د) عفو عمومی شامل جنایات علیه بشریت، شکنجه و کشتار نمی‌شود. \\
\hline

\textbf{اصل ۱۸۱} &
\textbf{نیروهای مسلح در دوره انتقال}

الف) نیروهای مسلح تحت فرماندهی دولت موقت قرار می‌گیرند.

ب) شورای انتقالی دفاع برای بازسازی نیروهای مسلح تشکیل می‌شود.

ج) نیروهای موازی و شبه‌نظامی منحل می‌شوند.

د) ادغام عناصر مناسب در نیروی مسلح واحد انجام می‌شود. \\
\hline

\textbf{اصل ۱۸۲} &
\textbf{اموال عمومی}

الف) کلیه اموال نهادهای حکومت سابق به دولت جدید منتقل می‌شود.

ب) اموال غارت‌شده باید به صاحبان اصلی یا خزانه عمومی بازگردد.

ج) کمیسیون ویژه برای شناسایی و استرداد اموال تشکیل می‌شود. \\
\hline

\textbf{اصل ۱۸۳} &
\textbf{تعهدات بین‌المللی}

الف) جمهوری فدرال ایران جانشین تعهدات بین‌المللی قانونی دولت‌های پیشین است.

ب) معاهدات مغایر با قانون اساسی بازنگری یا لغو می‌شوند.

ج) دولت جدید به کنوانسیون‌های بین‌المللی حقوق بشر می‌پیوندد. \\
\hline

\textbf{اصل ۱۸۴} &
\textbf{استقرار ساختار فدرال}

الف) تا تشکیل دولت‌های منطقه‌ای، استانداران موقت اداره امور را بر عهده دارند.

ب) انتخابات منطقه‌ای ظرف ۱۸ ماه برگزار می‌شود.

ج) انتقال اختیارات به مناطق به‌تدریج و طی ۵ سال انجام می‌شود. \\
\hline

\textbf{اصل ۱۸۵} &
\textbf{زبان‌ها در دوره انتقال}

الف) برنامه آموزش زبان‌های ملی ظرف ۲ سال آغاز می‌شود.

ب) اسناد رسمی به‌تدریج چندزبانه می‌شوند.

ج) آکادمی زبان‌های ایران برای استانداردسازی تشکیل می‌شود. \\
\hline

\textbf{اصل ۱۸۶} &
\textbf{انقضای مقررات انتقالی}

مقررات انتقالی (اصول ۱۷۴-۱۸۵) پس از ۵ سال از تاریخ لازم‌الاجرا شدن قانون اساسی یا با تحقق کامل موضوع آنها (هرکدام زودتر باشد) منقضی می‌شوند. \\
\hline

\textbf{اصل ۱۸۷} &
\textbf{اصل نهایی}

این قانون اساسی میثاق ملی مردم ایران برای زندگی مشترک در آزادی، برابری، عدالت و کرامت است. همه نهادها، مقامات و شهروندان موظف به احترام و پایبندی به آن هستند.

\begin{center}
\textbf{به نام ملت ایران}

مصوب مجلس مؤسسان

تأییدشده در همه‌پرسی ملی
\end{center} \\
\hline

\end{longtable}

% ═══════════════════════════════════════════════════════════════════════════════
\section{جداول تطبیقی با قوانین اساسی سایر کشورها}
\label{sec:const-comparative}
% ═══════════════════════════════════════════════════════════════════════════════

\subsection{مقایسه ساختار کلی}

\begin{center}
\begin{small}
\begin{longtable}{|>{\columncolor{DemocracyBlue!10}}p{2.5cm}|p{2.2cm}|p{2.2cm}|p{2.2cm}|p{2.2cm}|p{2.2cm}|}
\hline
\rowcolor{DemocracyBlue!30}
\textbf{ویژگی} & \textbf{ایران (پیشنهادی)} & \textbf{آلمان} & \textbf{هند} & \textbf{آفریقای جنوبی} & \textbf{اسپانیا} \\
\hline
\endfirsthead
\hline
\rowcolor{DemocracyBlue!30}
\textbf{ویژگی} & \textbf{ایران (پیشنهادی)} & \textbf{آلمان} & \textbf{هند} & \textbf{آفریقای جنوبی} & \textbf{اسپانیا} \\
\hline
\endhead

نوع نظام & جمهوری فدرال & جمهوری فدرال & جمهوری فدرال & جمهوری واحد & پادشاهی مشروطه \\
\hline

رئیس کشور & رئیس‌جمهور منتخب & رئیس‌جمهور (غیرمستقیم) & رئیس‌جمهور (غیرمستقیم) & رئیس‌جمهور (پارلمانی) & پادشاه \\
\hline

رئیس دولت & رئیس‌جمهور & صدراعظم & نخست‌وزیر & رئیس‌جمهور & نخست‌وزیر \\
\hline

پارلمان & دومجلسی & دومجلسی & دومجلسی & دومجلسی & دومجلسی \\
\hline

تعداد واحدهای فدرال & ۵ منطقه + ۱۵ استان & ۱۶ لَند & ۲۸ ایالت + ۸ منطقه & ۹ استان & ۱۷ منطقه خودمختار \\
\hline

دادگاه قانون اساسی & بله (۱۵ قاضی) & بله (۱۶ قاضی) & بله (دیوان عالی) & بله (۱۱ قاضی) & بله (۱۲ قاضی) \\
\hline

اصول غیرقابل تغییر & بله (۷ مورد) & بله (۲ مورد) & خیر (رویه قضایی) & بله (ارزش‌های بنیادین) & خیر \\
\hline

جدایی دین و دولت & کامل & بله & سکولار & بله & بله \\
\hline

\end{longtable}
\end{small}
\end{center}

\subsection{مقایسه حقوق بنیادین}

\begin{center}
\begin{small}
\begin{longtable}{|>{\columncolor{SuccessGreen!10}}p{3cm}|c|c|c|c|c|}
\hline
\rowcolor{SuccessGreen!30}
\textbf{حق بنیادین} & \textbf{ایران} & \textbf{آلمان} & \textbf{هند} & \textbf{آفریقا} & \textbf{کانادا} \\
\hline
\endfirsthead
\hline
\rowcolor{SuccessGreen!30}
\textbf{حق بنیادین} & \textbf{ایران} & \textbf{آلمان} & \textbf{هند} & \textbf{آفریقا} & \textbf{کانادا} \\
\hline
\endhead

لغو مجازات اعدام & ✓ & ✓ & ✗ & ✓ & ✓ \\
\hline

آزادی دین و تغییر دین & ✓ & ✓ & ✓ & ✓ & ✓ \\
\hline

برابری جنسیتی صریح & ✓ & ✓ & ✓ & ✓ & ✓ \\
\hline

حقوق زبانی اقلیت‌ها & ✓✓ & ✗ & ✓✓ & ✓✓ & ✓✓ \\
\hline

حق آب & ✓ & ✗ & ✗ & ✓ & ✗ \\
\hline

حق محیط زیست سالم & ✓ & ✓ & ✓ & ✓ & ✗ \\
\hline

حق مسکن & ✓ & ✗ & ✗ & ✓ & ✗ \\
\hline

ممنوعیت تبعیض گرایش جنسی & ✓ & ✓ & ✓ & ✓ & ✓ \\
\hline

حق دسترسی به اطلاعات & ✓ & ✓ & ✓ & ✓ & ✓ \\
\hline

حفاظت از افشاگران & ✓ & ✓ & ✗ & ✓ & ✓ \\
\hline

\multicolumn{6}{l}{\scriptsize ✓✓ = حمایت ویژه | ✓ = تضمین‌شده | ✗ = بدون تضمین صریح} \\

\end{longtable}
\end{small}
\end{center}

\subsection{مقایسه ساختار فدرال}

\begin{center}
\begin{tikzpicture}[scale=0.9]
    % محور عمودی - درجه تمرکززدایی
    \draw[->, thick] (0,0) -- (0,7) node[above, font=\small, align=center] {تمرکززدایی\\بیشتر};
    
    % محور افقی - تنوع قومی
    \draw[->, thick] (0,0) -- (12,0) node[right, font=\small, align=center] {تنوع قومی\\بیشتر};
    
    % کشورها
    \node[circle, draw=DemocracyBlue, fill=DemocracyBlue!30, minimum size=1.2cm, font=\tiny] at (8,5.5) {ایران};
    \node[circle, draw=SuccessGreen, fill=SuccessGreen!30, minimum size=1cm, font=\tiny] at (3,5) {آلمان};
    \node[circle, draw=WisdomGold, fill=WisdomGold!30, minimum size=1cm, font=\tiny] at (10,4.5) {هند};
    \node[circle, draw=purple, fill=purple!30, minimum size=1cm, font=\tiny] at (7,3.5) {اسپانیا};
    \node[circle, draw=orange, fill=orange!30, minimum size=1cm, font=\tiny] at (9,4) {آفریقا};
    \node[circle, draw=WarningRed, fill=WarningRed!30, minimum size=1cm, font=\tiny] at (5,6) {سوئیس};
    \node[circle, draw=gray, fill=gray!30, minimum size=1cm, font=\tiny] at (4,2) {فرانسه};
    
    % راهنما
    \node[font=\scriptsize, align=right] at (2,6.5) {فدرال};
    \node[font=\scriptsize, align=right] at (2,1) {متمرکز};
\end{tikzpicture}
\end{center}

\subsection{مقایسه فرآیند بازنگری قانون اساسی}

\begin{center}
\begin{small}
\begin{longtable}{|>{\columncolor{WisdomGold!10}}p{2.5cm}|p{3cm}|p{3cm}|p{3cm}|p{3cm}|}
\hline
\rowcolor{WisdomGold!30}
\textbf{کشور} & \textbf{پیشنهاددهنده} & \textbf{تصویب پارلمان} & \textbf{همه‌پرسی} & \textbf{اصول غیرقابل تغییر} \\
\hline
\endfirsthead
\hline
\rowcolor{WisdomGold!30}
\textbf{کشور} & \textbf{پیشنهاددهنده} & \textbf{تصویب پارلمان} & \textbf{همه‌پرسی} & \textbf{اصول غیرقابل تغییر} \\
\hline
\endhead

ایران (پیشنهادی) & دوسوم هر مجلس یا ۲ میلیون امضا & سه‌چهارم هر دو مجلس & اجباری & کرامت انسانی، دموکراسی، فدرالیسم، حقوق بنیادین \\
\hline

آلمان & دوسوم هر مجلس & دوسوم هر دو مجلس & خیر & کرامت انسانی، ساختار فدرال \\
\hline

فرانسه & رئیس‌جمهور یا پارلمان & سه‌پنجم کنگره & اختیاری & شکل جمهوری \\
\hline

آمریکا & دوسوم هر مجلس یا درخواست ایالات & سه‌چهارم ایالات & خیر & نمایندگی برابر سنا \\
\hline

سوئیس & پارلمان یا ۱۰۰,۰۰۰ امضا & اکثریت ساده & اجباری (اکثریت مردم + کانتون‌ها) & خیر \\
\hline

\end{longtable}
\end{small}
\end{center}

% ═══════════════════════════════════════════════════════════════════════════════
\section{نمودار کلی ساختار حکومت}
\label{sec:const-diagram}
% ═══════════════════════════════════════════════════════════════════════════════

\begin{center}
\begin{tikzpicture}[
    scale=0.85,
    transform shape,
    box/.style={
        draw=#1,
        line width=1.5pt,
        fill=#1!15,
        rounded corners=8pt,
        minimum width=2.8cm,
        minimum height=1.2cm,
        align=center,
        font=\small\bfseries
    },
    arrow/.style={->, thick, #1},
    label/.style={font=\tiny, #1}
]

    % === سطح ملت ===
    \node[
        draw=DemocracyBlue,
        line width=3pt,
        fill=DemocracyBlue!20,
        rounded corners=15pt,
        minimum width=14cm,
        minimum height=1.5cm,
        font=\Large\bfseries
    ] (nation) at (0,8) {ملت ایران — منشأ حاکمیت};

    % === قوه مقننه ===
    \node[box=DemocracyBlue] (nat) at (-5,5.5) {\shortstack{مجلس ملی\\۳۵۰ نماینده}};
    \node[box=DemocracyBlue] (sen) at (-1.5,5.5) {\shortstack{مجلس اقوام\\۱۲۰ نماینده}};
    
    % === قوه مجریه ===
    \node[box=SuccessGreen] (pres) at (2,5.5) {\shortstack{رئیس‌جمهور}};
    \node[box=lightSuccessGreen] (cab) at (2,3.5) {\shortstack{هیئت وزیران}};
    
    % === قوه قضائیه ===
    \node[box=WisdomGold] (cc) at (5.5,5.5) {\shortstack{دیوان\\قانون اساسی}};
    \node[box=lightWisdomGold] (sc) at (5.5,3.5) {\shortstack{دیوان عالی\\کشور}};
    
    % === نهادهای مستقل ===
    \node[box=WarningRed] (ind1) at (-5,1.5) {\shortstack{کمیسیون\\انتخابات}};
    \node[box=WarningRed] (ind2) at (-1.5,1.5) {\shortstack{بانک\\مرکزی}};
    \node[box=WarningRed] (ind3) at (2,1.5) {\shortstack{سازمان\\ضد فساد}};
    \node[box=WarningRed] (ind4) at (5.5,1.5) {\shortstack{کمیسیون\\حقوق بشر}};
    
    % === سطح منطقه‌ای ===
    \node[box=purple] (reg) at (0,-1) {\shortstack{دولت‌های منطقه‌ای\\(۵ منطقه خودمختار + ۱۵ استان)}};
    
    % === سطح محلی ===
    \node[box=gray] (loc) at (0,-3) {\shortstack{شوراها و شهرداری‌ها}};

    % === فلش‌های انتخاب ===
    \draw[arrow=DemocracyBlue] (nation) -- (nat) node[midway, left, label=black] {انتخاب};
    \draw[arrow=DemocracyBlue] (nation) -- (sen) node[midway, left, label=black] {};
    \draw[arrow=SuccessGreen] (nation) -- (pres) node[midway, right, label=black] {انتخاب};
    
    % === فلش‌های تأیید/انتصاب ===
    \draw[arrow=gray, dashed] (nat) -- (cab) node[midway, above, label=black] {رأی اعتماد};
    \draw[arrow=gray, dashed] (pres) -- (cab);
    \draw[arrow=WisdomGold, dashed] (nat) to[bend right=20] (cc);
    \draw[arrow=WisdomGold, dashed] (sen) to[bend left=10] (cc);
    
    % === فلش نظارت ===
    \draw[arrow=WarningRed, dotted] (ind1) -- (nat);
    \draw[arrow=WarningRed, dotted] (ind3) -- (cab);
    \draw[arrow=WisdomGold, dotted] (cc) -- (nat);
    \draw[arrow=WisdomGold, dotted] (cc) -- (pres);
    
    % === سطوح فدرال ===
    \draw[arrow=purple] (nation) to[bend right=60] (reg);
    \draw[arrow=gray] (reg) -- (loc);
    
    % === برچسب قوا ===
    \node[fill=DemocracyBlue!30, rounded corners=3pt, font=\scriptsize] at (-3.2,7) {قوه مقننه};
    \node[fill=SuccessGreen!30, rounded corners=3pt, font=\scriptsize] at (2,7) {قوه مجریه};
    \node[fill=WisdomGold!30, rounded corners=3pt, font=\scriptsize] at (5.5,7) {قوه قضائیه};
    \node[fill=WarningRed!30, rounded corners=3pt, font=\scriptsize] at (0,2.8) {نهادهای مستقل نظارتی};

\end{tikzpicture}
\end{center}

% ═══════════════════════════════════════════════════════════════════════════════
\section{خلاصه و جمع‌بندی}
\label{sec:const-summary}
% ═══════════════════════════════════════════════════════════════════════════════

\begin{kholasebox}
\textbf{خلاصه قانون اساسی پیشنهادی جمهوری فدرال ایران}

\begin{center}
\begin{tabular}{r r}
\textbf{شاخص} & \textbf{جزئیات} \\
\hline
تعداد کل اصول & ۱۸۷ اصل \\
تعداد فصول & ۸ فصل \\
اصول غیرقابل تغییر & ۷ اصل بنیادین \\
حقوق بنیادین & ۷۴ اصل (اصول ۱۲-۸۵) \\
ساختار حکومت & ۵۵ اصل \\
ساختار فدرال & ۷ اصل + ۱۵ اصل محلی \\
نهادهای مستقل & ۸ نهاد \\
مقررات انتقالی & ۱۴ اصل (خوداِنقضا) \\
\end{tabular}
\end{center}

\textbf{ویژگی‌های متمایز این قانون اساسی:}

\begin{enumerate}[nosep]
    \item \textbf{جدایی کامل دین از دولت} — برخلاف قانون اساسی فعلی
    \item \textbf{فدرالیسم همبسته} — پاسخ به مطالبات قومی بدون تجزیه
    \item \textbf{مجلس اقوام} — نهاد ویژه نمایندگی تنوع
    \item \textbf{لغو مجازات اعدام} — پیشرو در منطقه
    \item \textbf{حق آب} — پاسخ به بحران آب
    \item \textbf{نهادهای مستقل قوی} — پیشگیری از استبداد
    \item \textbf{اصول غیرقابل تغییر} — خطوط قرمز دموکراسی
    \item \textbf{عدالت انتقالی} — گذار مسالمت‌آمیز اما عادلانه
\end{enumerate}
\end{kholasebox}

\vspace{10pt}

\begin{olgoobox}
\textbf{منابع الهام‌بخش این قانون اساسی:}

\begin{itemize}[nosep]
    \item قانون اساسی آلمان (۱۹۴۹): ساختار فدرال، دادگاه قانون اساسی، اصول غیرقابل تغییر
    \item قانون اساسی آفریقای جنوبی (۱۹۹۶): منشور حقوق، عدالت انتقالی، حقوق اقتصادی-اجتماعی
    \item قانون اساسی هند (۱۹۵۰): مدیریت تنوع، فدرالیسم نامتقارن
    \item قانون اساسی اسپانیا (۱۹۷۸): مناطق خودمختار، گذار دموکراتیک
    \item قانون اساسی سوئیس (۱۹۹۹): دموکراسی مستقیم، فدرالیسم زبانی
    \item منشور حقوق بشر اروپا و میثاقین بین‌المللی حقوق بشر
\end{itemize}
\end{olgoobox}

% ═══════════════════════════════════════════════════════════════════════════════
% پایان پیوست ۱
% ═══════════════════════════════════════════════════════════════════════════════