%══════════════════════════════════════════════════════════════════════════════
% preamble.tex — تنظیمات و پکیج‌های کتاب
% از بحران تا بالندگی: طرح جامع تأسیس دموکراسی پایدار
% نویسنده: مهدی سالم | ریچموندهیل | ۱۴۰۴
%══════════════════════════════════════════════════════════════════════════════

%──────────────────────────────────────────────────────────────────────────────
% پکیج‌های اصلی
%──────────────────────────────────────────────────────────────────────────────
\usepackage{amsmath,amssymb}
\usepackage{fontspec}
\usepackage{geometry}
\usepackage{graphicx}
\usepackage{tikz}
\usepackage{pgfplots}
\usepackage{pgf-pie}           % برای نمودارهای دایره‌ای (\pie)
\usepackage{tcolorbox}
\usepackage{booktabs}
\usepackage{array}
\usepackage{colortbl}
\usepackage{multirow}
\usepackage{longtable}
\usepackage{tabularx}
\usepackage{enumitem}
\usepackage{fancyhdr}
\usepackage{pdflscape}
\usepackage{xcolor}
\usepackage{hyperref}
\usepackage{float}
\usepackage{caption}
\usepackage{subcaption}
\usepackage{setspace}
\usepackage{titlesec}
\usepackage{etoolbox}

%──────────────────────────────────────────────────────────────────────────────
% تنظیمات صفحه
%──────────────────────────────────────────────────────────────────────────────
\geometry{
    a4paper,
    top=2.5cm,
    bottom=2.5cm,
    left=2.5cm,
    right=3cm,
    headheight=15pt,
    bindingoffset=0.5cm
}

%──────────────────────────────────────────────────────────────────────────────
% کتابخانه‌های TikZ و pgfplots
%──────────────────────────────────────────────────────────────────────────────
\usetikzlibrary{
    shapes.geometric, 
    arrows.meta, 
    positioning, 
    calc, 
    decorations.pathreplacing, 
    backgrounds,
    fit,
    matrix,
    chains,
    scopes,
    shadows
}
\pgfplotsset{compat=1.18}
\usepgfplotslibrary{polar}     % برای نمودارهای راداری (polaraxis)

%──────────────────────────────────────────────────────────────────────────────
% تعریف رنگ‌ها — پالت اصلی
%──────────────────────────────────────────────────────────────────────────────
% رنگ‌های اصلی
\definecolor{bleurepublique}{RGB}{0, 85, 164}
\definecolor{rougerevolution}{RGB}{190, 30, 45}
\definecolor{orroyal}{RGB}{218, 165, 32}
\definecolor{orroyaldark}{RGB}{184, 134, 11}
\definecolor{vertnapoleon}{RGB}{0, 102, 51}
\definecolor{violetempire}{RGB}{102, 51, 153}
\definecolor{gris}{RGB}{128, 128, 128}

% رنگ‌های روشن (پس‌زمینه کادرها)
\definecolor{bleulight}{RGB}{230, 240, 250}
\definecolor{rougelight}{RGB}{255, 230, 230}
\definecolor{orroyallight}{RGB}{255, 248, 220}
\definecolor{vertlight}{RGB}{230, 250, 230}
\definecolor{violetlight}{RGB}{245, 235, 255}
\definecolor{grislight}{RGB}{245, 245, 245}

% رنگ‌های میانه (نمودارها)
\definecolor{bleumid}{RGB}{100, 149, 237}
\definecolor{rougemid}{RGB}{220, 100, 100}
\definecolor{vertmid}{RGB}{100, 180, 100}
\definecolor{violetmid}{RGB}{180, 140, 200}
\definecolor{orroyalmid}{RGB}{238, 201, 0}

% رنگ‌های فازها
\definecolor{phase1}{RGB}{70, 130, 180}
\definecolor{phase2}{RGB}{60, 179, 113}
\definecolor{phase3}{RGB}{255, 165, 0}
\definecolor{phase4}{RGB}{147, 112, 219}
\definecolor{phase5}{RGB}{220, 20, 60}

% ─── رنگ‌های معنایی پروژه (استفاده در پیوست‌ها و نمودارها) ──────────────────
\definecolor{DemocracyBlue}{RGB}{41, 128, 185}
\definecolor{SuccessGreen}{RGB}{39, 174, 96}
\definecolor{WisdomGold}{RGB}{241, 196, 15}
\definecolor{WarningRed}{RGB}{231, 76, 60}
% نسخه‌های روشن‌تر (برای زیرمجموعه‌ها در استایل‌های پارامتری)
\colorlet{lightSuccessGreen}{SuccessGreen!70}
\colorlet{lightWisdomGold}{WisdomGold!70}
% رنگ‌های ترکیبی از‌پیش‌ساخته (جلوگیری از خطای xcolor)
\colorlet{darkyellow}{yellow!80!black}
\colorlet{darkgreen}{green!60!black}

% رنگ‌های اضافی برای نمودارها
\definecolor{chart1}{RGB}{31, 119, 180}
\definecolor{chart2}{RGB}{255, 127, 14}
\definecolor{chart3}{RGB}{44, 160, 44}
\definecolor{chart4}{RGB}{214, 39, 40}
\definecolor{chart5}{RGB}{148, 103, 189}

%──────────────────────────────────────────────────────────────────────────────
% تنظیمات tcolorbox
%──────────────────────────────────────────────────────────────────────────────
\tcbuselibrary{skins, breakable}

% کادر خلاصه فصل (آبی)
\newtcolorbox{kholasebox}[1][]{
    colback=bleulight,
    colframe=bleurepublique,
    fonttitle=\bfseries\large,
    title={\hfill خلاصه فصل},
    breakable,
    enhanced,
    boxrule=1.5pt,
    arc=4pt,
    left=12pt, right=12pt, top=12pt, bottom=12pt,
    #1
}

% کادر نقل قول (طلایی)
\newtcolorbox{naghlbox}[1][]{
    colback=orroyallight,
    colframe=orroyaldark,
    fonttitle=\bfseries,
    breakable,
    enhanced,
    boxrule=1pt,
    arc=3pt,
    left=10pt, right=10pt, top=10pt, bottom=10pt,
    borderline west={4pt}{0pt}{orroyal},
    #1
}

% کادر الگو و درس (سبز)
\newtcolorbox{olgoobox}[1][]{
    colback=vertlight,
    colframe=vertnapoleon,
    fonttitle=\bfseries,
    title={\hfill \textbf{الگو و درس}},
    breakable,
    enhanced,
    boxrule=1pt,
    arc=3pt,
    left=10pt, right=10pt, top=10pt, bottom=10pt,
    #1
}

% کادر هشدار (قرمز)
\newtcolorbox{enghelabbox}[1][]{
    colback=rougelight,
    colframe=rougerevolution,
    fonttitle=\bfseries,
    title={\hfill \textbf{هشدار}},
    halign title=right,    
    breakable,
    enhanced,
    boxrule=1.5pt,
    arc=3pt,
    left=10pt, right=10pt, top=10pt, bottom=10pt,
    #1
}

% کادر تحلیل (بنفش)
\newtcolorbox{tahlilbox}[1][]{
    colback=violetlight,
    colframe=violetempire,
    fonttitle=\bfseries,
    title={\hfill \textbf{تحلیل}},
    breakable,
    enhanced,
    boxrule=1pt,
    arc=3pt,
    left=10pt, right=10pt, top=10pt, bottom=10pt,
    #1
}

% کادر نکته (خاکستری)
\newtcolorbox{noktebox}[1][]{
    colback=grislight,
    colframe=gris,
    fonttitle=\bfseries,
    breakable,
    enhanced,
    boxrule=1pt,
    arc=3pt,
    left=10pt, right=10pt, top=10pt, bottom=10pt,
    #1
}

% کادر مورد مطالعاتی (آبی روشن)
\newtcolorbox{casebox}[2][]{
    colback=bleulight,
    colframe=bleumid,
    fonttitle=\bfseries,
    title={\hfill \textbf{مورد مطالعاتی: #2}},
    breakable,
    enhanced,
    boxrule=1pt,
    arc=3pt,
    left=10pt, right=10pt, top=10pt, bottom=10pt,
    #1
}

% کادر داده‌های کلیدی
\newtcolorbox{databox}[1][]{
    colback=white,
    colframe=bleurepublique,
    fonttitle=\bfseries,
    title={\hfill \textbf{داده‌های کلیدی}},
    breakable,
    enhanced,
    boxrule=2pt,
    arc=0pt,
    left=10pt, right=10pt, top=10pt, bottom=10pt,
    #1
}

%──────────────────────────────────────────────────────────────────────────────
% دستورات سفارشی برای جداول
%──────────────────────────────────────────────────────────────────────────────
\newcommand{\tablemark}{\cellcolor{bleulight}}
\newcommand{\headmark}{\cellcolor{bleurepublique}\color{white}\bfseries}
\newcommand{\rowmark}{\rowcolor{bleulight}}
\newcolumntype{L}[1]{>{\raggedright\arraybackslash}p{#1}}
\newcolumntype{C}[1]{>{\centering\arraybackslash}p{#1}}
\newcolumntype{R}[1]{>{\raggedleft\arraybackslash}p{#1}}

%──────────────────────────────────────────────────────────────────────────────
% نمادهای سفارشی
%──────────────────────────────────────────────────────────────────────────────
\newcommand{\yes}{\textcolor{vertnapoleon}{\checkmark}}
\newcommand{\no}{\textcolor{rougerevolution}{$\times$}}
\newcommand{\somewhat}{\textcolor{orroyal}{$\circ$}}

% فازها
\newcommand{\phmark}[1]{\textcolor{phase#1}{\textbf{فاز #1}}}

%──────────────────────────────────────────────────────────────────────────────
% تنظیمات سربرگ و پاورقی
%──────────────────────────────────────────────────────────────────────────────
\pagestyle{fancy}
\fancyhf{}
\fancyhead[LE]{\small\nouppercase{\leftmark}}
\fancyhead[RO]{\small\nouppercase{\rightmark}}
\fancyfoot[C]{\thepage}
\renewcommand{\headrulewidth}{0.4pt}
\renewcommand{\footrulewidth}{0pt}

% صفحه اول فصل‌ها
\fancypagestyle{plain}{
    \fancyhf{}
    \fancyfoot[C]{\thepage}
    \renewcommand{\headrulewidth}{0pt}
}

%──────────────────────────────────────────────────────────────────────────────
% تنظیمات hyperref
%──────────────────────────────────────────────────────────────────────────────
\hypersetup{
    colorlinks=true,
    linkcolor=bleurepublique,
    citecolor=vertnapoleon,
    urlcolor=violetempire,
    bookmarksnumbered=true,
    pdfauthor={مهدی سالم},
    pdftitle={از بحران تا بالندگی},
    pdfsubject={طرح جامع تأسیس دموکراسی پایدار}
}

%──────────────────────────────────────────────────────────────────────────────
% تنظیمات فاصله خطوط
%──────────────────────────────────────────────────────────────────────────────
\onehalfspacing

%──────────────────────────────────────────────────────────────────────────────
% نماد تیک (✓) برای جداول — جایگزین \mark
%──────────────────────────────────────────────────────────────────────────────
\newcommand{\cmark}{\textcolor{vertnapoleon}{\checkmark}}

%──────────────────────────────────────────────────────────────────────────────
% دستور منبع نقل‌قول
%──────────────────────────────────────────────────────────────────────────────
\newcommand{\sourceline}[1]{%
    \par\vspace{4pt}%
    \begin{flushleft}%
    {\small\color{gris}— #1}%
    \end{flushleft}%
}

%──────────────────────────────────────────────────────────────────────────────
% محیط منابع فصل (جایگزین refsection بدون biblatex)
%──────────────────────────────────────────────────────────────────────────────
\newenvironment{refsection}{%
    \begin{quote}%
}{%
    \end{quote}%
}

%──────────────────────────────────────────────────────────────────────────────
% فونت و زبان فارسی (حتماً آخرین پکیج)
%──────────────────────────────────────────────────────────────────────────────
\usepackage{xepersian}
\settextfont{Vazirmatn}
\setdigitfont{Vazirmatn}
% اگر Linux Libertine O نصب نیست، از Times New Roman استفاده می‌شود
\IfFontExistsTF{Linux Libertine O}{%
    \setlatintextfont{Linux Libertine O}%
}{%
    \IfFontExistsTF{Times New Roman}{%
        \setlatintextfont{Times New Roman}%
    }{%
        \setlatintextfont{FreeSerif}%
    }%
}

%──────────────────────────────────────────────────────────────────────────────
% عنوان‌بندی فارسی
%──────────────────────────────────────────────────────────────────────────────
\renewcommand{\contentsname}{فهرست مطالب}
\renewcommand{\listfigurename}{فهرست شکل‌ها}
\renewcommand{\listtablename}{فهرست جدول‌ها}
\renewcommand{\bibname}{کتابشناسی}
\renewcommand{\indexname}{نمایه}
\renewcommand{\figurename}{شکل}
\renewcommand{\tablename}{جدول}
\renewcommand{\partname}{بخش}
\renewcommand{\chaptername}{فصل}
\renewcommand{\appendixname}{پیوست}