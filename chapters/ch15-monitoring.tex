% ch15-monitoring.tex
% فصل پانزدهم: پایش، ارزیابی و مدیریت ریسک
% نویسنده: مهدی سالم | ریچموندهیل | ۱۴۰۴

\chapter{پایش، ارزیابی و مدیریت ریسک}
\label{ch:monitoring}

\begin{kholasebox}
هر برنامه‌ای بدون سیستم پایش و ارزیابی محکوم به شکست است. این فصل چارچوب جامعی برای \textbf{پایش پیشرفت گذار دموکراتیک}، \textbf{ارزیابی دوره‌ای}، و \textbf{مدیریت ریسک‌ها} ارائه می‌دهد. سیستم پیشنهادی شامل: \textbf{داشبورد ملی} با ۵۰ شاخص کلیدی، \textbf{گزارش‌دهی شفاف} به مردم، \textbf{نظام هشدار زودهنگام} برای شناسایی انحرافات، و \textbf{مکانیزم اصلاح مسیر} است. اصل راهنما: \textbf{«آنچه اندازه‌گیری نشود، مدیریت نمی‌شود — و آنچه شفاف نباشد، فاسد می‌شود»}.
\end{kholasebox}

%═══════════════════════════════════════════════════════════════════════════════
\section{مقدمه: چرا پایش حیاتی است؟}
%═══════════════════════════════════════════════════════════════════════════════

\begin{naghlbox}
«برنامه‌ریزی بدون پایش مثل دویدن با چشم‌های بسته است. شاید بدانید کجا می‌خواهید بروید، اما نمی‌دانید کجا هستید و آیا در مسیر درست حرکت می‌کنید یا نه.»
\sourceline{پیتر دراکر، پدر مدیریت مدرن}
\end{naghlbox}

گذار دموکراتیک فرآیندی پیچیده، غیرخطی و پر از غافلگیری است. بهترین برنامه‌ها هم در برخورد با واقعیت تغییر می‌کنند. سیستم پایش و ارزیابی ابزاری است برای:

\begin{itemize}[nosep]
\item \textbf{فهم واقعیت}: کجا هستیم؟ چقدر پیشرفت کرده‌ایم؟
\item \textbf{شناسایی انحرافات}: آیا از مسیر خارج شده‌ایم؟
\item \textbf{پاسخگویی}: آیا مسئولان به تعهداتشان عمل کرده‌اند؟
\item \textbf{یادگیری}: چه چیزی کار کرد؟ چه چیزی نه؟
\item \textbf{اصلاح مسیر}: چه تغییراتی لازم است؟
\end{itemize}

\subsection{تفاوت پایش و ارزیابی}

\begin{table}[htbp]
\centering
\caption{تفاوت پایش و ارزیابی}
\label{tab:monitoring-vs-evaluation}
\begin{tabular}{>{\columncolor{blue!8}}l p{5cm} p{5cm}}
\toprule
\rowcolor{blue!25}
\textbf{بُعد} & \textbf{پایش (Monitoring)} & \textbf{ارزیابی (Evaluation)} \\
\midrule
زمان & مستمر و لحظه‌ای & دوره‌ای (سالانه، میان‌دوره، پایانی) \\
\rowcolor{gray!10}
سؤال اصلی & آیا در مسیر هستیم؟ & آیا به هدف رسیدیم؟ چرا؟ \\
تمرکز & فعالیت‌ها و خروجی‌ها & نتایج و تأثیرات \\
\rowcolor{gray!10}
مسئول & واحدهای اجرایی & ارزیابان مستقل \\
روش & داده‌های اداری، داشبورد & تحقیق، نظرسنجی، مصاحبه \\
\rowcolor{gray!10}
کاربرد & اصلاح فوری & یادگیری، سیاستگذاری \\
\bottomrule
\end{tabular}
\end{table}

%═══════════════════════════════════════════════════════════════════════════════
\section{چارچوب پایش: مدل منطقی}
\label{sec:logic-model}
%═══════════════════════════════════════════════════════════════════════════════

\begin{center}
\begin{tikzpicture}[
    node distance=1.5cm,
    box/.style={
        rectangle,
        rounded corners=5pt,
        draw=#1!70,
        fill=#1!15,
        thick,
        minimum width=2.5cm,
        minimum height=1.5cm,
        align=center,
        font=\small
    },
    arrow/.style={->, thick, >=stealth}
]
% زنجیره نتایج
\node[box=blue] (input) {\textbf{ورودی‌ها}\\ \scriptsize بودجه، نیروی انسانی،\\ \scriptsize قوانین};
\node[box=green, right=1.5cm of input] (activity) {\textbf{فعالیت‌ها}\\ \scriptsize اصلاحات، پروژه‌ها،\\ \scriptsize برنامه‌ها};
\node[box=orange, right=1.5cm of activity] (output) {\textbf{خروجی‌ها}\\ \scriptsize قوانین تصویب‌شده،\\ \scriptsize نهادهای ایجادشده};
\node[box=purple, right=1.5cm of output] (outcome) {\textbf{پیامدها}\\ \scriptsize رفتار تغییریافته،\\ \scriptsize دسترسی بهبودیافته};
\node[box=red, right=1.5cm of outcome] (impact) {\textbf{تأثیرات}\\ \scriptsize دموکراسی پایدار،\\ \scriptsize رفاه فراگیر};

% فلش‌ها
\draw[arrow] (input) -- (activity);
\draw[arrow] (activity) -- (output);
\draw[arrow] (output) -- (outcome);
\draw[arrow] (outcome) -- (impact);

% برچسب‌های پایش
\node[below=0.8cm of input, font=\tiny, align=center] {پایش منابع};
\node[below=0.8cm of activity, font=\tiny, align=center] {پایش اجرا};
\node[below=0.8cm of output, font=\tiny, align=center] {پایش خروجی};
\node[below=0.8cm of outcome, font=\tiny, align=center] {ارزیابی پیامد};
\node[below=0.8cm of impact, font=\tiny, align=center] {ارزیابی تأثیر};

% عنوان
\node[above=1.5cm of output, font=\large\bfseries] {زنجیره نتایج و سطوح پایش};

% کادر پایش vs ارزیابی
\draw[dashed, blue!50, thick] (-1,-1.5) rectangle (6.5,-2.5);
\node[blue!70, font=\scriptsize] at (2.75,-2) {پایش مستمر};
\draw[dashed, red!50, thick] (6.7,-1.5) rectangle (12.5,-2.5);
\node[red!70, font=\scriptsize] at (9.6,-2) {ارزیابی دوره‌ای};
\end{tikzpicture}
\end{center}

%═══════════════════════════════════════════════════════════════════════════════
\section{شاخص‌های کلیدی عملکرد (KPIs)}
\label{sec:kpis}
%═══════════════════════════════════════════════════════════════════════════════

\subsection{اصول انتخاب شاخص}

\begin{table}[htbp]
\centering
\caption{معیارهای SMART برای شاخص‌های خوب}
\label{tab:smart-criteria}
\begin{tabular}{>{\columncolor{green!8}}c l p{7cm}}
\toprule
\rowcolor{green!25}
\textbf{حرف} & \textbf{معیار} & \textbf{توضیح} \\
\midrule
S & Specific (مشخص) & دقیقاً چه چیزی اندازه‌گیری می‌شود \\
\rowcolor{gray!10}
M & Measurable (قابل اندازه‌گیری) & عدد و داده قابل جمع‌آوری \\
A & Achievable (دست‌یافتنی) & هدف واقع‌بینانه \\
\rowcolor{gray!10}
R & Relevant (مرتبط) & به هدف اصلی کمک می‌کند \\
T & Time-bound (زمان‌دار) & مهلت مشخص دارد \\
\bottomrule
\end{tabular}
\end{table}

\subsection{شاخص‌های حوزه سیاسی}

\begin{table}[htbp]
\centering
\caption{شاخص‌های کلیدی حوزه سیاسی و دموکراتیک}
\label{tab:political-kpis}
\small
\begin{tabular}{>{\columncolor{blue!8}}r p{3.5cm} c c c c c}
\toprule
\rowcolor{blue!25}
& \textbf{شاخص} & \textbf{مبدأ} & \textbf{س۲} & \textbf{س۵} & \textbf{س۱۰} & \textbf{س۱۵} \\
\midrule
۱ & شاخص دموکراسی (EIU, از ۱۰) & ۲.۲ & ۴.۵ & ۶.۰ & ۷.۵ & ۸.۰ \\
\rowcolor{gray!10}
۲ & آزادی مطبوعات (RSF, از ۱۸۰) & ۱۷۶ & ۱۲۰ & ۸۰ & ۵۰ & ۳۵ \\
۳ & مشارکت انتخاباتی (٪) & ۴۰ & ۵۵ & ۶۵ & ۷۵ & ۷۵ \\
\rowcolor{gray!10}
۴ & زنان در مجلس (٪) & ۶ & ۱۵ & ۲۵ & ۳۵ & ۴۵ \\
۵ & اعتماد به دولت (٪) & ۱۵ & ۳۰ & ۴۵ & ۵۵ & ۶۵ \\
\rowcolor{gray!10}
۶ & حاکمیت قانون (WJP, از ۱) & ۰.۴ & ۰.۵ & ۰.۶ & ۰.۷ & ۰.۸ \\
۷ & استقلال قضایی (از ۱۰) & ۳ & ۵ & ۶ & ۷ & ۸ \\
\rowcolor{gray!10}
۸ & تعداد NGOهای فعال (هزار) & ۲۰ & ۴۰ & ۷۰ & ۱۰۰ & ۱۵۰ \\
۹ & شاخص فساد (CPI, از ۱۰۰) & ۲۵ & ۳۵ & ۴۵ & ۵۵ & ۷۰ \\
\rowcolor{gray!10}
۱۰ & آزادی اینترنت (از ۱۰۰) & ۱۶ & ۴۰ & ۶۰ & ۷۵ & ۸۵ \\
\bottomrule
\end{tabular}
\end{table}

\subsection{شاخص‌های حوزه اقتصادی}

\begin{table}[htbp]
\centering
\caption{شاخص‌های کلیدی حوزه اقتصادی}
\label{tab:economic-kpis-monitoring}
\small
\begin{tabular}{>{\columncolor{green!8}}r p{3.5cm} c c c c c}
\toprule
\rowcolor{green!25}
& \textbf{شاخص} & \textbf{مبدأ} & \textbf{س۲} & \textbf{س۵} & \textbf{س۱۰} & \textbf{س۱۵} \\
\midrule
۱ & رشد GDP (٪) & ۲ & ۵ & ۷ & ۶ & ۵ \\
\rowcolor{gray!10}
۲ & تورم (٪) & ۵۰ & ۱۵ & ۷ & ۴ & ۲ \\
۳ & بیکاری (٪) & ۱۲ & ۱۰ & ۸ & ۶ & ۵ \\
\rowcolor{gray!10}
۴ & GDP سرانه (هزار \$ PPP) & ۱۵ & ۱۷ & ۲۲ & ۳۲ & ۴۵ \\
۵ & ضریب جینی & ۰.۴۲ & ۰.۴۰ & ۰.۳۸ & ۰.۳۶ & ۰.۳۵ \\
\rowcolor{gray!10}
۶ & نرخ فقر (٪) & ۳۰ & ۲۵ & ۱۸ & ۱۰ & ۵ \\
۷ & FDI سالانه (میلیارد \$) & ۱ & ۵ & ۲۵ & ۵۰ & ۶۰ \\
\rowcolor{gray!10}
۸ & سهم نفت از بودجه (٪) & ۳۵ & ۳۰ & ۲۲ & ۱۵ & ۱۰ \\
۹ & رتبه کسب‌وکار (از ۱۹۰) & ۱۲۷ & ۹۰ & ۵۰ & ۳۵ & ۲۵ \\
\rowcolor{gray!10}
۱۰ & نسبت مالیات/GDP (٪) & ۷ & ۹ & ۱۲ & ۱۶ & ۱۸ \\
\bottomrule
\end{tabular}
\end{table}

\subsection{شاخص‌های حوزه اجتماعی}

\begin{table}[htbp]
\centering
\caption{شاخص‌های کلیدی حوزه اجتماعی و انسانی}
\label{tab:social-kpis}
\small
\begin{tabular}{>{\columncolor{purple!8}}r p{3.5cm} c c c c c}
\toprule
\rowcolor{purple!25}
& \textbf{شاخص} & \textbf{مبدأ} & \textbf{س۲} & \textbf{س۵} & \textbf{س۱۰} & \textbf{س۱۵} \\
\midrule
۱ & شاخص توسعه انسانی (HDI) & ۰.۷۷ & ۰.۷۸ & ۰.۸۱ & ۰.۸۴ & ۰.۸۷ \\
\rowcolor{gray!10}
۲ & امید به زندگی (سال) & ۷۷ & ۷۷.۵ & ۷۹ & ۸۱ & ۸۳ \\
۳ & میانگین سال تحصیل & ۱۰ & ۱۰.۵ & ۱۱.۵ & ۱۲.۵ & ۱۴ \\
\rowcolor{gray!10}
۴ & شکاف جنسیتی (GGI, از ۱) & ۰.۵۷ & ۰.۶۲ & ۰.۷۰ & ۰.۷۵ & ۰.۸۰ \\
۵ & رضایت از زندگی (از ۱۰) & ۴.۵ & ۵.۵ & ۶.۵ & ۷.۰ & ۷.۵ \\
\rowcolor{gray!10}
۶ & اعتماد بین‌فردی (٪) & ۲۰ & ۲۵ & ۳۵ & ۴۵ & ۵۵ \\
۷ & پوشش بیمه سلامت (٪) & ۸۵ & ۹۰ & ۹۵ & ۹۸ & ۱۰۰ \\
\rowcolor{gray!10}
۸ & دسترسی به اینترنت (٪) & ۸۰ & ۸۵ & ۹۲ & ۹۷ & ۹۹ \\
۹ & مسکن مناسب (٪ خانوار) & ۷۰ & ۷۵ & ۸۲ & ۸۸ & ۹۵ \\
\rowcolor{gray!10}
۱۰ & رضایت اقوام از حقوق (٪) & ۳۰ & ۴۵ & ۶۰ & ۷۵ & ۸۵ \\
\bottomrule
\end{tabular}
\end{table}

\subsection{شاخص‌های حوزه محیط زیست}

\begin{table}[htbp]
\centering
\caption{شاخص‌های کلیدی حوزه محیط زیست}
\label{tab:environmental-kpis}
\small
\begin{tabular}{>{\columncolor{teal!8}}r p{3.5cm} c c c c c}
\toprule
\rowcolor{teal!25}
& \textbf{شاخص} & \textbf{مبدأ} & \textbf{س۲} & \textbf{س۵} & \textbf{س۱۰} & \textbf{س۱۵} \\
\midrule
۱ & کسری آب (میلیارد م۳) & ۱۸ & ۱۵ & ۱۰ & ۵ & ۰ \\
\rowcolor{gray!10}
۲ & PM2.5 تهران (میکروگرم) & ۳۵ & ۲۸ & ۲۰ & ۱۲ & ۸ \\
۳ & سهم انرژی تجدیدپذیر (٪) & ۸ & ۱۲ & ۲۵ & ۵۰ & ۷۰ \\
\rowcolor{gray!10}
۴ & انتشار CO2 (میلیون تن) & ۷۵۰ & ۷۲۰ & ۶۵۰ & ۵۰۰ & ۴۰۰ \\
۵ & حجم ارومیه (میلیارد م۳) & ۳ & ۴ & ۷ & ۱۲ & ۱۵ \\
\rowcolor{gray!10}
۶ & پوشش جنگلی (٪) & ۷ & ۷.۵ & ۸.۵ & ۱۰ & ۱۲ \\
۷ & بازیافت پسماند (٪) & ۸ & ۱۵ & ۳۰ & ۴۵ & ۶۰ \\
\rowcolor{gray!10}
۸ & مناطق حفاظت‌شده (٪) & ۱۰ & ۱۱ & ۱۳ & ۱۶ & ۱۸ \\
۹ & راندمان آبیاری (٪) & ۳۸ & ۴۵ & ۵۵ & ۷۰ & ۸۵ \\
\rowcolor{gray!10}
۱۰ & شاخص EPI (از ۱۰۰) & ۴۰ & ۴۵ & ۵۵ & ۶۵ & ۷۵ \\
\bottomrule
\end{tabular}
\end{table}

\subsection{شاخص‌های حوزه امنیت و بین‌الملل}

\begin{table}[htbp]
\centering
\caption{شاخص‌های کلیدی امنیت و روابط بین‌الملل}
\label{tab:security-kpis}
\small
\begin{tabular}{>{\columncolor{red!8}}r p{3.5cm} c c c c c}
\toprule
\rowcolor{red!25}
& \textbf{شاخص} & \textbf{مبدأ} & \textbf{س۲} & \textbf{س۵} & \textbf{س۱۰} & \textbf{س۱۵} \\
\midrule
۱ & شاخص صلح جهانی (GPI) & ۱۴۰ & ۱۱۰ & ۸۰ & ۵۵ & ۴۰ \\
\rowcolor{gray!10}
۲ & تعداد تحریم‌های فعال & ۳۸۰۰ & ۲۰۰۰ & ۵۰۰ & ۱۰۰ & ۰ \\
۳ & روابط دیپلماتیک عادی & ۸۰ & ۱۲۰ & ۱۶۰ & ۱۸۰ & ۱۹۰ \\
\rowcolor{gray!10}
۴ & عضویت در سازمان‌های بین‌المللی & ۵۰ & ۶۰ & ۸۰ & ۱۰۰ & ۱۲۰ \\
۵ & گردشگر ورودی (میلیون) & ۵ & ۸ & ۱۵ & ۲۵ & ۳۵ \\
\rowcolor{gray!10}
۶ & رتبه پاسپورت ایرانی & ۱۸۰+ & ۱۵۰ & ۱۰۰ & ۶۰ & ۴۰ \\
۷ & صادرات غیرنفتی (میلیارد \$) & ۳۵ & ۵۰ & ۸۰ & ۱۴۰ & ۲۲۰ \\
\rowcolor{gray!10}
۸ & نرخ جرم (قتل/۱۰۰ هزار) & ۳ & ۲.۵ & ۲ & ۱.۵ & ۱.۲ \\
\bottomrule
\end{tabular}
\end{table}

%═══════════════════════════════════════════════════════════════════════════════
\section{داشبورد ملی پیشرفت}
\label{sec:national-dashboard}
%═══════════════════════════════════════════════════════════════════════════════

\begin{center}
\begin{tikzpicture}[scale=0.95]
% کادر اصلی داشبورد
\draw[thick, rounded corners=15pt, fill=gray!5] (-7,-6) rectangle (7,5.5);
\node[font=\large\bfseries] at (0,5) {داشبورد ملی پیشرفت گذار دموکراتیک};

% ۹ کادر شاخص
% ردیف اول
\node[rectangle, rounded corners=5pt, draw=blue!70, fill=blue!15,
      minimum width=3.8cm, minimum height=1.8cm, align=center] at (-4.5,3) 
    {\textbf{دموکراسی}\\ \Large ۶.۵/۱۰\\ \scriptsize هدف: ۸.۰};
\node[rectangle, rounded corners=5pt, draw=green!70, fill=green!15,
      minimum width=3.8cm, minimum height=1.8cm, align=center] at (0,3) 
    {\textbf{توسعه انسانی}\\ \Large ۰.۸۲\\ \scriptsize هدف: ۰.۸۷};
\node[rectangle, rounded corners=5pt, draw=orange!70, fill=orange!15,
      minimum width=3.8cm, minimum height=1.8cm, align=center] at (4.5,3) 
    {\textbf{رشد اقتصادی}\\ \Large ۶٪\\ \scriptsize هدف: ۵٪+};

% ردیف دوم
\node[rectangle, rounded corners=5pt, draw=red!70, fill=red!15,
      minimum width=3.8cm, minimum height=1.8cm, align=center] at (-4.5,0.5) 
    {\textbf{تورم}\\ \Large ۱۲٪\\ \scriptsize هدف: ۵٪};
\node[rectangle, rounded corners=5pt, draw=purple!70, fill=purple!15,
      minimum width=3.8cm, minimum height=1.8cm, align=center] at (0,0.5) 
    {\textbf{بیکاری}\\ \Large ۸٪\\ \scriptsize هدف: ۵٪};
\node[rectangle, rounded corners=5pt, draw=teal!70, fill=teal!15,
      minimum width=3.8cm, minimum height=1.8cm, align=center] at (4.5,0.5) 
    {\textbf{فساد (CPI)}\\ \Large ۵۰/۱۰۰\\ \scriptsize هدف: ۷۰};

% ردیف سوم
\node[rectangle, rounded corners=5pt, draw=cyan!70, fill=cyan!15,
      minimum width=3.8cm, minimum height=1.8cm, align=center] at (-4.5,-2) 
    {\textbf{محیط زیست}\\ \Large ۵۵/۱۰۰\\ \scriptsize هدف: ۷۵};
\node[rectangle, rounded corners=5pt, draw=yellow!70!black, fill=yellow!15,
      minimum width=3.8cm, minimum height=1.8cm, align=center] at (0,-2) 
    {\textbf{برابری جنسیتی}\\ \Large ۰.۷۰\\ \scriptsize هدف: ۰.۸۰};
\node[rectangle, rounded corners=5pt, draw=pink!70!black, fill=pink!15,
      minimum width=3.8cm, minimum height=1.8cm, align=center] at (4.5,-2) 
    {\textbf{رضایت مردم}\\ \Large ۶.۵/۱۰\\ \scriptsize هدف: ۷.۵};

% نوار وضعیت کلی
\draw[thick, fill=green!30] (-6,-4) rectangle (6,-3.5);
\node[font=\small\bfseries] at (0,-3.75) {وضعیت کلی: در مسیر | پیشرفت: ۶۵٪ از اهداف سال ۱۰};

% نوار زمان
\draw[thick] (-6,-5) -- (6,-5);
\foreach \x/\label in {-6/۱۴۰۳, -3/س۵, 0/س۱۰, 3/س۱۵, 6/س۲۵} {
    \fill (\x,-5) circle (3pt);
    \node[below, font=\tiny] at (\x,-5) {\label};
}
\fill[red] (-1.5,-5) circle (5pt);
\node[above, font=\tiny\bfseries, red] at (-1.5,-4.8) {اکنون};
\end{tikzpicture}
\captionof{figure}{نمونه داشبورد ملی پیشرفت (فرضی — سال ۷)}
\label{fig:dashboard-sample}
\end{center}

\subsection{ویژگی‌های داشبورد}

\begin{table}[htbp]
\centering
\caption{ویژگی‌های داشبورد ملی پیشرفت}
\label{tab:dashboard-features}
\begin{tabular}{>{\columncolor{blue!8}}r p{4cm} p{5.5cm}}
\toprule
\rowcolor{blue!25}
\textbf{ویژگی} & \textbf{توضیح} & \textbf{اهمیت} \\
\midrule
دسترسی عمومی & آنلاین، رایگان، به فارسی و انگلیسی & شفافیت و پاسخگویی \\
\rowcolor{gray!10}
به‌روزرسانی & ماهانه برای اکثر شاخص‌ها & اطلاعات تازه \\
سطوح مختلف & ملی، استانی، شهری & جزئیات برای هر منطقه \\
\rowcolor{gray!10}
مقایسه بین‌المللی & جایگاه ایران در جهان & بنچ‌مارکینگ \\
روند تاریخی & نمودار تغییرات در زمان & دیدن پیشرفت \\
\rowcolor{gray!10}
دانلود داده & داده‌های خام قابل دانلود & تحقیق و تحلیل مستقل \\
API & برای توسعه‌دهندگان & اکوسیستم داده باز \\
\bottomrule
\end{tabular}
\end{table}

%═══════════════════════════════════════════════════════════════════════════════
\section{نظام ارزیابی دوره‌ای}
\label{sec:periodic-evaluation}
%═══════════════════════════════════════════════════════════════════════════════

\subsection{تقویم ارزیابی}

\begin{table}[htbp]
\centering
\caption{تقویم ارزیابی‌های دوره‌ای}
\label{tab:evaluation-calendar}
\begin{tabular}{>{\columncolor{orange!8}}l c p{4cm} p{3.5cm}}
\toprule
\rowcolor{orange!25}
\textbf{نوع ارزیابی} & \textbf{زمان} & \textbf{محتوا} & \textbf{مسئول} \\
\midrule
گزارش ماهانه & هر ماه & شاخص‌های کلیدی، انحرافات & دفتر رئیس‌جمهور \\
\rowcolor{gray!10}
گزارش فصلی & هر ۳ ماه & تحلیل عمیق‌تر، اصلاحات & دولت به مجلس \\
گزارش سالانه & هر سال & ارزیابی جامع، نظرسنجی & نهاد مستقل ارزیابی \\
\rowcolor{gray!10}
ارزیابی میان‌دوره‌ای & هر ۵ سال & ارزیابی استراتژیک & پانل مستقل داخلی-خارجی \\
ارزیابی فاز & پایان هر فاز & آیا به اهداف رسیدیم؟ & کمیته ملی ارزیابی \\
\rowcolor{gray!10}
ممیزی بین‌المللی & هر ۲ سال & توسط نهادهای بین‌المللی & UN, EU, IMF, WB \\
\bottomrule
\end{tabular}
\end{table}

\subsection{معیارهای ارزیابی موفقیت}

\begin{olgoobox}
\textbf{سه سؤال کلیدی ارزیابی}

هر ارزیابی باید به سه سؤال پاسخ دهد:
\begin{enumerate}[nosep]
\item \textbf{اثربخشی}: آیا به اهداف رسیدیم؟ چقدر؟
\item \textbf{کارایی}: با چه هزینه‌ای؟ آیا راه بهتری بود؟
\item \textbf{پایداری}: آیا دستاوردها ماندگارند؟
\end{enumerate}

\textbf{سؤالات تکمیلی}:
\begin{itemize}[nosep]
\item \textbf{مرتبط بودن}: آیا اهداف درست انتخاب شدند؟
\item \textbf{انسجام}: آیا سیاست‌ها هماهنگ بودند؟
\item \textbf{تأثیر}: چه تغییر واقعی در زندگی مردم ایجاد شد؟
\end{itemize}
\end{olgoobox}

%═══════════════════════════════════════════════════════════════════════════════
\section{مدیریت ریسک}
\label{sec:risk-management}
%═══════════════════════════════════════════════════════════════════════════════

\subsection{ماتریس ریسک‌های گذار}

\begin{table}[htbp]
\centering
\caption{ماتریس ریسک‌های اصلی گذار دموکراتیک}
\label{tab:risk-matrix}
\small
\begin{tabular}{>{\columncolor{red!8}}l c c p{4.5cm}}
\toprule
\rowcolor{red!25}
\textbf{ریسک} & \textbf{احتمال} & \textbf{تأثیر} & \textbf{استراتژی کاهش} \\
\midrule
بازگشت اقتدارگرایی & متوسط & بحرانی & نهادسازی قوی، جامعه مدنی \\
\rowcolor{gray!10}
جنگ داخلی/خشونت & پایین & فاجعه‌بار & عدالت انتقالی، دیالوگ ملی \\
شکست اقتصادی & متوسط & بالا & سیاست‌های احتیاطی، FDI \\
\rowcolor{gray!10}
تجزیه‌طلبی & پایین-متوسط & بحرانی & فدرالیسم، حقوق اقوام \\
مداخله خارجی & متوسط & بالا & دیپلماسی فعال، موازنه \\
\rowcolor{gray!10}
فساد سیستماتیک & بالا & بالا & شفافیت، نهاد ضدفساد مستقل \\
بحران آب/محیط زیست & بالا & بالا & اقدام فوری، سرمایه‌گذاری \\
\rowcolor{gray!10}
پوپولیسم & متوسط-بالا & متوسط & آموزش، رسانه مسئول \\
خستگی اصلاحات & متوسط & متوسط & دستاوردهای ملموس، ارتباط \\
\rowcolor{gray!10}
تنش با همسایگان & متوسط & متوسط-بالا & دیپلماسی منطقه‌ای \\
\bottomrule
\end{tabular}
\end{table}

\subsection{نقشه حرارتی ریسک}

\begin{center}
\begin{tikzpicture}
% محورها
\draw[thick, ->] (0,0) -- (10,0) node[right] {احتمال};
\draw[thick, ->] (0,0) -- (0,8) node[above] {تأثیر};

% برچسب‌های محورها
\node[below] at (2,0) {پایین};
\node[below] at (5,0) {متوسط};
\node[below] at (8,0) {بالا};
\node[left] at (0,2) {پایین};
\node[left] at (0,4) {متوسط};
\node[left] at (0,6) {بالا};
\node[left] at (0,7.5) {بحرانی};

% مناطق رنگی
\fill[green!30] (0,0) rectangle (3.3,4);
\fill[yellow!30] (3.3,0) rectangle (6.6,4);
\fill[yellow!30] (0,4) rectangle (3.3,6);
\fill[orange!30] (6.6,0) rectangle (10,4);
\fill[orange!30] (3.3,4) rectangle (6.6,6);
\fill[orange!30] (0,6) rectangle (3.3,8);
\fill[red!30] (6.6,4) rectangle (10,8);
\fill[red!30] (3.3,6) rectangle (6.6,8);

% خطوط شبکه
\draw[gray, dashed] (3.3,0) -- (3.3,8);
\draw[gray, dashed] (6.6,0) -- (6.6,8);
\draw[gray, dashed] (0,4) -- (10,4);
\draw[gray, dashed] (0,6) -- (10,6);

% ریسک‌ها
\node[circle, fill=black, minimum size=8pt, label=right:{\scriptsize جنگ داخلی}] at (2,7.5) {};
\node[circle, fill=black, minimum size=8pt, label=right:{\scriptsize تجزیه‌طلبی}] at (3.5,7) {};
\node[circle, fill=black, minimum size=8pt, label=left:{\scriptsize بازگشت اقتدارگرایی}] at (5,6.5) {};
\node[circle, fill=black, minimum size=8pt, label=right:{\scriptsize شکست اقتصادی}] at (5,5.5) {};
\node[circle, fill=black, minimum size=8pt, label=above:{\scriptsize بحران آب}] at (7.5,5.5) {};
\node[circle, fill=black, minimum size=8pt, label=right:{\scriptsize فساد}] at (7,4.5) {};
\node[circle, fill=black, minimum size=8pt, label=below:{\scriptsize مداخله خارجی}] at (5,4.5) {};
\node[circle, fill=black, minimum size=8pt, label=right:{\scriptsize پوپولیسم}] at (6,3) {};

% راهنما
\node[fill=green!30, minimum width=0.5cm, minimum height=0.3cm] at (8.5,7.5) {};
\node[right, font=\tiny] at (8.8,7.5) {کم‌خطر};
\node[fill=yellow!30, minimum width=0.5cm, minimum height=0.3cm] at (8.5,7) {};
\node[right, font=\tiny] at (8.8,7) {هشدار};
\node[fill=orange!30, minimum width=0.5cm, minimum height=0.3cm] at (8.5,6.5) {};
\node[right, font=\tiny] at (8.8,6.5) {جدی};
\node[fill=red!30, minimum width=0.5cm, minimum height=0.3cm] at (8.5,6) {};
\node[right, font=\tiny] at (8.8,6) {بحرانی};
\end{tikzpicture}
\captionof{figure}{نقشه حرارتی ریسک‌های گذار دموکراتیک}
\label{fig:risk-heatmap}
\end{center}

\subsection{سیستم هشدار زودهنگام}

\begin{table}[htbp]
\centering
\caption{شاخص‌های هشدار زودهنگام}
\label{tab:early-warning}
\begin{tabular}{>{\columncolor{orange!8}}l p{4.5cm} p{4.5cm}}
\toprule
\rowcolor{orange!25}
\textbf{حوزه} & \textbf{شاخص‌های هشدار} & \textbf{آستانه بحران} \\
\midrule
سیاسی & کاهش مشارکت، افزایش اعتراضات & مشارکت زیر ۴۰٪، اعتراض ۱۰۰K+ \\
\rowcolor{gray!10}
اقتصادی & تورم، بیکاری، فرار سرمایه & تورم بالای ۳۰٪، بیکاری بالای ۱۵٪ \\
قومی & تنش‌های قومی، خشونت محلی & بیش از ۱۰ حادثه/ماه \\
\rowcolor{gray!10}
اجتماعی & رضایت، اعتماد به نهادها & رضایت زیر ۳۰٪ \\
امنیتی & جرم، تروریسم، تنش مرزی & افزایش ۵۰٪+ \\
\rowcolor{gray!10}
رسانه‌ای & سانسور، تهدید خبرنگاران & بازگشت محدودیت‌ها \\
\bottomrule
\end{tabular}
\end{table}

\begin{enghelabbox}
\textbf{پروتکل پاسخ به هشدار}

\textbf{سطح زرد} (هشدار اولیه):
\begin{itemize}[nosep]
\item تشکیل کارگروه بررسی
\item گزارش به مقامات ارشد
\item تقویت پایش
\end{itemize}

\textbf{سطح نارنجی} (تنش جدی):
\begin{itemize}[nosep]
\item جلسه اضطراری کابینه
\item اقدام اصلاحی فوری
\item ارتباط عمومی شفاف
\end{itemize}

\textbf{سطح قرمز} (بحران):
\begin{itemize}[nosep]
\item فعال‌سازی پروتکل مدیریت بحران
\item گفتگوی ملی/مذاکره
\item درخواست کمک بین‌المللی در صورت نیاز
\end{itemize}
\end{enghelabbox}

%═══════════════════════════════════════════════════════════════════════════════
\section{ساختار نهادی پایش}
\label{sec:institutional-framework}
%═══════════════════════════════════════════════════════════════════════════════

\begin{center}
\begin{tikzpicture}[
    node distance=1.5cm,
    institution/.style={
        rectangle,
        rounded corners=5pt,
        draw=#1!70,
        fill=#1!15,
        thick,
        minimum width=4cm,
        minimum height=1.3cm,
        align=center,
        font=\small
    }
]
% نهادها
\node[institution=blue] (stats) at (0,4) {\textbf{مرکز ملی آمار}\\ جمع‌آوری داده، پایگاه‌ها};
\node[institution=green] (eval) at (0,2) {\textbf{سازمان ملی ارزیابی}\\ ارزیابی مستقل، گزارش‌دهی};
\node[institution=orange] (audit) at (0,0) {\textbf{دیوان محاسبات}\\ ممیزی مالی و عملکردی};
\node[institution=purple] (ombuds) at (0,-2) {\textbf{آمبودزمان ملی}\\ شکایات مردمی، نظارت};

% نهادهای ناظر بیرونی
\node[institution=teal, right=2.5cm of stats] (civil) {\textbf{جامعه مدنی}\\ رصدخانه‌های مردمی};
\node[institution=red, right=2.5cm of eval] (media) {\textbf{رسانه‌های مستقل}\\ روزنامه‌نگاری تحقیقی};
\node[institution=cyan, right=2.5cm of audit] (intl) {\textbf{نهادهای بین‌المللی}\\ UN, WB, IMF, EU};
\node[institution=darkyellow, right=2.5cm of ombuds] (acad) {\textbf{دانشگاه‌ها}\\ تحقیق و تحلیل};

% کادر دولتی و مستقل
\draw[dashed, blue!50, thick, rounded corners] (-2.5,-3) rectangle (2.5,5);
\node[blue!70, font=\scriptsize] at (0,5.3) {نهادهای رسمی};

\draw[dashed, green!50, thick, rounded corners] (4,-3) rectangle (9,5);
\node[green!70, font=\scriptsize] at (6.5,5.3) {نظارت مستقل};

% اتصالات
\draw[thick, <->] (stats) -- (civil);
\draw[thick, <->] (eval) -- (media);
\draw[thick, <->] (audit) -- (intl);
\draw[thick, <->] (ombuds) -- (acad);
\end{tikzpicture}
\captionof{figure}{ساختار نهادی پایش و ارزیابی}
\label{fig:monitoring-institutions}
\end{center}

\begin{table}[htbp]
\centering
\caption{وظایف نهادهای پایش و ارزیابی}
\label{tab:monitoring-institutions}
\begin{tabular}{>{\columncolor{blue!8}}l p{4.5cm} p{4.5cm}}
\toprule
\rowcolor{blue!25}
\textbf{نهاد} & \textbf{وظایف اصلی} & \textbf{ویژگی کلیدی} \\
\midrule
مرکز ملی آمار & جمع‌آوری داده، سرشماری، نظرسنجی & استقلال، متدولوژی استاندارد \\
\rowcolor{gray!10}
سازمان ملی ارزیابی & ارزیابی سیاست‌ها و برنامه‌ها & مستقل از دولت، گزارش به مجلس \\
دیوان محاسبات & ممیزی مالی، ارزیابی عملکرد & گزارش علنی، قدرت پیگیری \\
\rowcolor{gray!10}
آمبودزمان ملی & شکایات مردمی، نظارت بر حقوق & دسترسی آسان، قدرت تحقیق \\
کمیته پایش مجلس & نظارت بر قوه مجریه & جلسات علنی، استیضاح \\
\bottomrule
\end{tabular}
\end{table}

%═══════════════════════════════════════════════════════════════════════════════
\section{پایش مشارکتی: صدای مردم}
\label{sec:participatory-monitoring}
%═══════════════════════════════════════════════════════════════════════════════

\begin{naghlbox}
«دموکراسی فقط رأی دادن هر چند سال نیست. شهروندان باید بتوانند روزانه بر عملکرد دولت نظارت کنند و صدایشان شنیده شود.»
\sourceline{نویسنده}
\end{naghlbox}

\subsection{ابزارهای پایش مردمی}

\begin{table}[htbp]
\centering
\caption{ابزارهای پایش مشارکتی}
\label{tab:participatory-tools}
\begin{tabular}{>{\columncolor{green!8}}r p{3.5cm} p{5.5cm}}
\toprule
\rowcolor{green!25}
\textbf{ابزار} & \textbf{توضیح} & \textbf{کاربرد} \\
\midrule
نظرسنجی ملی & پیمایش سالانه ۵۰,۰۰۰ نفری & سنجش رضایت، اولویت‌ها \\
\rowcolor{gray!10}
اپلیکیشن شهروندی & گزارش مشکلات، امتیازدهی & فیدبک لحظه‌ای \\
کارت امتیاز اجتماعی & ارزیابی خدمات محلی توسط مردم & پاسخگویی محلی \\
\rowcolor{gray!10}
بودجه مشارکتی & تصمیم‌گیری مردم درباره بخشی از بودجه & مشارکت مستقیم \\
جلسات عمومی & حضور مقامات برای پاسخگویی & تعامل مستقیم \\
\rowcolor{gray!10}
رصدخانه‌های مدنی & NGOهای ناظر بر حوزه‌های خاص & نظارت تخصصی \\
پلتفرم شفافیت & دسترسی به اسناد و قراردادهای دولتی & مبارزه با فساد \\
\bottomrule
\end{tabular}
\end{table}

\begin{olgoobox}
\textbf{الگوی موفق: کره جنوبی — e-People}

سامانه e-People کره جنوبی نمونه‌ای از پایش مشارکتی است:
\begin{itemize}[nosep]
\item سامانه آنلاین ثبت شکایات و پیشنهادات شهروندی
\item الزام دولت به پاسخ در ۱۴ روز
\item پیگیری آنلاین توسط شهروند
\item انتشار آمار شکایات و نحوه رسیدگی
\item سالانه ۲ میلیون+ شکایت و پیشنهاد
\item \textbf{نتیجه}: افزایش پاسخگویی، کاهش فساد
\end{itemize}
\end{olgoobox}

\subsection{نظرسنجی ملی سالانه}

\begin{table}[htbp]
\centering
\caption{محتوای نظرسنجی ملی سالانه}
\label{tab:annual-survey}
\begin{tabular}{>{\columncolor{purple!8}}r p{4cm} p{5cm}}
\toprule
\rowcolor{purple!25}
\textbf{بخش} & \textbf{موضوعات} & \textbf{سؤالات نمونه} \\
\midrule
رضایت کلی & زندگی، آینده، کشور & آیا کشور در مسیر درست است؟ \\
\rowcolor{gray!10}
اعتماد به نهادها & دولت، مجلس، قضا، ارتش & چقدر به مجلس اعتماد دارید؟ \\
کیفیت خدمات & بهداشت، آموزش، حمل‌ونقل & کیفیت مدارس دولتی چطور است؟ \\
\rowcolor{gray!10}
آزادی‌ها & بیان، تجمع، رسانه & آیا می‌توانید آزادانه نظر دهید؟ \\
امنیت & جانی، مالی، اجتماعی & چقدر احساس امنیت می‌کنید؟ \\
\rowcolor{gray!10}
اقتصاد & وضعیت مالی، اشتغال & وضعیت مالی‌تان بهتر شده؟ \\
اقوام و هویت & حقوق قومی، تعلق & آیا به حقوق قومی‌تان احترام گذاشته می‌شود؟ \\
\rowcolor{gray!10}
محیط زیست & آب، هوا، آینده & وضعیت محیط زیست بهتر شده؟ \\
\bottomrule
\end{tabular}
\end{table}

%═══════════════════════════════════════════════════════════════════════════════
\section{مقایسه بین‌المللی (بنچ‌مارکینگ)}
\label{sec:benchmarking}
%═══════════════════════════════════════════════════════════════════════════════

\subsection{شاخص‌های بین‌المللی مرجع}

\begin{table}[htbp]
\centering
\caption{شاخص‌های بین‌المللی مرجع برای پایش}
\label{tab:international-indices}
\small
\begin{tabular}{>{\columncolor{cyan!8}}l l c c c}
\toprule
\rowcolor{cyan!25}
\textbf{شاخص} & \textbf{سازمان} & \textbf{رتبه فعلی ایران} & \textbf{هدف س۱۰} & \textbf{هدف س۱۵} \\
\midrule
شاخص دموکراسی & EIU & ۱۵۴ از ۱۶۷ & ۸۰ & ۵۰ \\
\rowcolor{gray!10}
آزادی جهان & Freedom House & Not Free & Partly Free & Free \\
شاخص فساد (CPI) & Transparency Intl & ۱۴۹ از ۱۸۰ & ۷۰ & ۴۵ \\
\rowcolor{gray!10}
حاکمیت قانون & WJP & ۱۱۸ از ۱۴۲ & ۶۰ & ۴۰ \\
آزادی مطبوعات & RSF & ۱۷۶ از ۱۸۰ & ۸۰ & ۴۰ \\
\rowcolor{gray!10}
توسعه انسانی (HDI) & UNDP & ۷۶ از ۱۹۱ & ۵۵ & ۴۰ \\
شکاف جنسیتی & WEF & ۱۴۳ از ۱۴۶ & ۸۰ & ۵۰ \\
\rowcolor{gray!10}
کسب‌وکار & World Bank & ۱۲۷ از ۱۹۰& ۳۵ & ۲۵ \\
شاخص صلح جهانی (GPI) & IEP & ۱۴۰ از ۱۶۳ & ۶۰ & ۴۰ \\
\rowcolor{gray!10}
عملکرد محیط زیستی (EPI) & Yale & ۱۲۰ از ۱۸۰ & ۶۰ & ۴۰ \\
نوآوری جهانی (GII) & WIPO & ۶۵ از ۱۳۲ & ۴۰ & ۲۵ \\
\bottomrule
\end{tabular}
\end{table}

\subsection{کشورهای مرجع برای مقایسه}

\begin{table}[htbp]
\centering
\caption{کشورهای مرجع برای بنچ‌مارکینگ}
\label{tab:benchmark-countries}
\begin{tabular}{>{\columncolor{teal!8}}l p{3.5cm} p{5.5cm}}
\toprule
\rowcolor{teal!25}
\textbf{دسته} & \textbf{کشورها} & \textbf{دلیل انتخاب} \\
\midrule
گذار موفق & کره جنوبی، تایوان، اسپانیا، شیلی & الگوی گذار از اقتدارگرایی \\
\rowcolor{gray!10}
مشابه منطقه‌ای & ترکیه، مصر، عربستان & مقایسه منطقه‌ای \\
مشابه اقتصادی & مالزی، مکزیک، آرژانتین & سطح درآمد مشابه \\
\rowcolor{gray!10}
چندقومی موفق & سوئیس، کانادا، هند & مدیریت تنوع \\
پیشرو محیط زیست & کاستاریکا، دانمارک & الگوی سبز \\
\rowcolor{gray!10}
هدف بلندمدت & پرتغال، چک، استونی & جایی که می‌خواهیم برسیم \\
\bottomrule
\end{tabular}
\end{table}

%═══════════════════════════════════════════════════════════════════════════════
\section{گزارش‌دهی و شفافیت}
\label{sec:reporting}
%═══════════════════════════════════════════════════════════════════════════════

\subsection{انواع گزارش‌ها}

\begin{table}[htbp]
\centering
\caption{نظام گزارش‌دهی پیشرفت گذار}
\label{tab:reporting-system}
\begin{tabular}{>{\columncolor{orange!8}}l c p{4cm} p{3cm}}
\toprule
\rowcolor{orange!25}
\textbf{نوع گزارش} & \textbf{دوره} & \textbf{محتوا} & \textbf{مخاطب} \\
\midrule
داشبورد آنلاین & لحظه‌ای & شاخص‌های کلیدی & عموم مردم \\
\rowcolor{gray!10}
خبرنامه ماهانه & ماهانه & خلاصه پیشرفت‌ها & عموم، رسانه‌ها \\
گزارش فصلی & فصلی & تحلیل عملکرد & مجلس، نخبگان \\
\rowcolor{gray!10}
گزارش سالانه & سالانه & ارزیابی جامع & همه ذی‌نفعان \\
گزارش تفصیلی بخشی & سالانه & هر حوزه جداگانه & متخصصان \\
\rowcolor{gray!10}
گزارش به مردم & سالانه & زبان ساده، تصویری & عموم مردم \\
\bottomrule
\end{tabular}
\end{table}

\subsection{اصول گزارش‌دهی شفاف}

\begin{olgoobox}
\textbf{هفت اصل گزارش‌دهی خوب}

\begin{enumerate}[nosep]
\item \textbf{صداقت}: هم موفقیت‌ها، هم شکست‌ها
\item \textbf{به‌موقع}: داده‌های تازه، نه کهنه
\item \textbf{قابل فهم}: زبان ساده برای عموم
\item \textbf{قابل دسترس}: آنلاین، رایگان، چندزبانه
\item \textbf{قابل مقایسه}: با گذشته و با دیگران
\item \textbf{قابل راستی‌آزمایی}: داده‌های خام قابل دانلود
\item \textbf{پاسخگو}: توضیح انحرافات و اقدامات اصلاحی
\end{enumerate}
\end{olgoobox}

%═══════════════════════════════════════════════════════════════════════════════
\section{مکانیزم اصلاح مسیر}
\label{sec:course-correction}
%═══════════════════════════════════════════════════════════════════════════════

\begin{naghlbox}
«هیچ برنامه‌ای از برخورد با واقعیت سالم بیرون نمی‌آید. مهم این نیست که برنامه اولیه بی‌نقص باشد؛ مهم این است که توانایی یادگیری و اصلاح داشته باشیم.»
\sourceline{دوایت آیزنهاور}
\end{naghlbox}

\subsection{فرآیند اصلاح مسیر}

\begin{center}
\begin{tikzpicture}[
    node distance=1.8cm,
    step/.style={
        rectangle,
        rounded corners=8pt,
        draw=blue!70,
        fill=blue!15,
        thick,
        minimum width=3cm,
        minimum height=1.2cm,
        align=center,
        font=\small
    },
    arrow/.style={->, thick, >=stealth, blue!60}
]
% مراحل
\node[step] (detect) {\textbf{۱. شناسایی}\\ انحراف از هدف};
\node[step, right=1.5cm of detect] (analyze) {\textbf{۲. تحلیل}\\ علت‌یابی};
\node[step, right=1.5cm of analyze] (design) {\textbf{۳. طراحی}\\ راه‌حل};
\node[step, below=1.5cm of design] (decide) {\textbf{۴. تصمیم}\\ تصویب اصلاح};
\node[step, left=1.5cm of decide] (implement) {\textbf{۵. اجرا}\\ پیاده‌سازی};
\node[step, left=1.5cm of implement] (monitor) {\textbf{۶. پایش}\\ ارزیابی اثر};

% فلش‌ها
\draw[arrow] (detect) -- (analyze);
\draw[arrow] (analyze) -- (design);
\draw[arrow] (design) -- (decide);
\draw[arrow] (decide) -- (implement);
\draw[arrow] (implement) -- (monitor);
\draw[arrow, dashed] (monitor) -- (detect);

% چرخه
\node[font=\scriptsize, gray] at (1.5,-2.5) {چرخه مستمر بهبود};
\end{tikzpicture}
\captionof{figure}{فرآیند اصلاح مسیر}
\label{fig:course-correction}
\end{center}

\subsection{سطوح اصلاح}

\begin{table}[htbp]
\centering
\caption{سطوح مختلف اصلاح مسیر}
\label{tab:correction-levels}
\begin{tabular}{>{\columncolor{purple!8}}l p{3.5cm} p{3.5cm} p{2.5cm}}
\toprule
\rowcolor{purple!25}
\textbf{سطح} & \textbf{نوع تغییر} & \textbf{مرجع تصمیم} & \textbf{زمان} \\
\midrule
عملیاتی & تنظیم فعالیت‌ها & مدیران اجرایی & فوری \\
\rowcolor{gray!10}
تاکتیکی & اصلاح برنامه‌ها & وزرا و معاونان & ۱-۳ ماه \\
استراتژیک & تغییر اولویت‌ها & کابینه & ۳-۶ ماه \\
\rowcolor{gray!10}
ساختاری & تغییر نهادها/قوانین & مجلس & ۶-۱۲ ماه \\
بنیادین & بازنگری اهداف کلان & مجلس + همه‌پرسی & ۱-۲ سال \\
\bottomrule
\end{tabular}
\end{table}

%═══════════════════════════════════════════════════════════════════════════════
\section{ارزیابی تأثیر (Impact Evaluation)}
\label{sec:impact-evaluation}
%═══════════════════════════════════════════════════════════════════════════════

\subsection{روش‌های ارزیابی تأثیر}

\begin{table}[htbp]
\centering
\caption{روش‌های ارزیابی تأثیر سیاست‌ها}
\label{tab:impact-methods}
\begin{tabular}{>{\columncolor{green!8}}l p{5cm} p{4cm}}
\toprule
\rowcolor{green!25}
\textbf{روش} & \textbf{توضیح} & \textbf{کاربرد} \\
\midrule
آزمایش تصادفی (RCT) & مقایسه گروه آزمایش و کنترل & برنامه‌های اجتماعی \\
\rowcolor{gray!10}
تفاوت در تفاوت & مقایسه قبل/بعد در دو گروه & سیاست‌های جدید \\
رگرسیون ناپیوستگی & استفاده از آستانه‌های سیاستی & برنامه‌های هدفمند \\
\rowcolor{gray!10}
تطبیق (Matching) & مقایسه موارد مشابه & وقتی RCT ممکن نیست \\
مطالعه موردی کیفی & تحلیل عمیق چند مورد & فهم مکانیزم‌ها \\
\rowcolor{gray!10}
نظرسنجی قبل/بعد & سنجش تغییر در نگرش‌ها & ارزیابی فرهنگی \\
\bottomrule
\end{tabular}
\end{table}

\subsection{سؤالات کلیدی ارزیابی تأثیر}

\begin{itemize}[nosep]
\item آیا زندگی مردم واقعاً بهتر شده است؟
\item کدام گروه‌ها بیشتر/کمتر بهره‌مند شده‌اند؟
\item آیا تغییرات پایدار خواهند بود؟
\item چه عوارض ناخواسته‌ای ایجاد شده؟
\item آیا با همین هزینه، نتیجه بهتری ممکن بود؟
\end{itemize}

%═══════════════════════════════════════════════════════════════════════════════
\section{یادگیری سازمانی}
\label{sec:organizational-learning}
%═══════════════════════════════════════════════════════════════════════════════

\begin{table}[htbp]
\centering
\caption{مکانیزم‌های یادگیری سازمانی}
\label{tab:learning-mechanisms}
\begin{tabular}{>{\columncolor{cyan!8}}r p{4.5cm} p{4.5cm}}
\toprule
\rowcolor{cyan!25}
\textbf{مکانیزم} & \textbf{توضیح} & \textbf{خروجی} \\
\midrule
جلسات درس‌آموخته & بازنگری پس از هر پروژه بزرگ & گزارش درس‌آموخته‌ها \\
\rowcolor{gray!10}
پایگاه دانش & ذخیره تجربیات و بهترین شیوه‌ها & ویکی داخلی، راهنماها \\
آموزش مستمر & دوره‌های توانمندسازی کارکنان & کارکنان ماهرتر \\
\rowcolor{gray!10}
تبادل بین‌المللی & یادگیری از کشورهای دیگر & بهترین شیوه‌های جهانی \\
تحقیق و توسعه سیاستی & آزمایش پایلوت قبل از گسترش & سیاست‌های اثبات‌شده \\
\rowcolor{gray!10}
بازخورد مردمی & گوش‌دادن به شکایات و پیشنهادات & بهبود خدمات \\
\bottomrule
\end{tabular}
\end{table}

%═══════════════════════════════════════════════════════════════════════════════
\section{چارچوب پاسخگویی}
\label{sec:accountability}
%═══════════════════════════════════════════════════════════════════════════════

\begin{center}
\begin{tikzpicture}[
    node distance=2cm,
    actor/.style={
        ellipse,
        draw=#1!70,
        fill=#1!20,
        thick,
        minimum width=2.5cm,
        minimum height=1.5cm,
        align=center,
        font=\small
    }
]
% مرکز - دولت
\node[actor=blue, minimum width=3cm] (gov) {\textbf{دولت}\\قوه مجریه};

% حلقه‌های پاسخگویی
\node[actor=green, above=1.5cm of gov] (parl) {\textbf{مجلس}\\نظارت قانونی};
\node[actor=purple, above right=1cm of gov] (court) {\textbf{قضا}\\نظارت قضایی};
\node[actor=orange, right=2cm of gov] (audit) {\textbf{دیوان محاسبات}\\ممیزی};
\node[actor=red, below right=1cm of gov] (media) {\textbf{رسانه}\\افکار عمومی};
\node[actor=teal, below=1.5cm of gov] (civil) {\textbf{جامعه مدنی}\\نظارت مردمی};
\node[actor=darkyellow, below left=1cm of gov] (intl) {\textbf{بین‌المللی}\\استانداردها};
\node[actor=pink, left=2cm of gov] (party) {\textbf{احزاب}\\رقابت سیاسی};
\node[actor=cyan, above left=1cm of gov] (ombuds) {\textbf{آمبودزمان}\\شکایات};

% فلش‌ها
\foreach \n in {parl, court, audit, media, civil, intl, party, ombuds} {
    \draw[thick, <->, gray!60] (gov) -- (\n);
}

% عنوان
\node[above=3cm of gov, font=\large\bfseries] {اکوسیستم پاسخگویی};
\end{tikzpicture}
\captionof{figure}{چارچوب چندلایه پاسخگویی}
\label{fig:accountability-framework}
\end{center}

\begin{table}[htbp]
\centering
\caption{ابزارهای پاسخگویی}
\label{tab:accountability-tools}
\begin{tabular}{>{\columncolor{blue!8}}l p{4cm} p{5cm}}
\toprule
\rowcolor{blue!25}
\textbf{نوع پاسخگویی} & \textbf{ابزار} & \textbf{ضمانت اجرا} \\
\midrule
قانونی & استیضاح، رأی عدم اعتماد & برکناری \\
\rowcolor{gray!10}
قضایی & محاکمه، جریمه & مجازات قانونی \\
مالی & ممیزی، شفافیت بودجه & جریمه، استرداد \\
\rowcolor{gray!10}
انتخاباتی & رأی مردم & عدم انتخاب مجدد \\
اجتماعی & افکار عمومی، رسانه & فشار سیاسی \\
\rowcolor{gray!10}
حرفه‌ای & استانداردهای اخلاقی & تعلیق، اخراج \\
\bottomrule
\end{tabular}
\end{table}

%═══════════════════════════════════════════════════════════════════════════════
\section{تقویم پایش و ارزیابی}
\label{sec:me-timeline}
%═══════════════════════════════════════════════════════════════════════════════

\begin{table}[htbp]
\centering
\caption{تقویم فعالیت‌های پایش و ارزیابی}
\label{tab:me-timeline}
\begin{tabular}{>{\columncolor{orange!8}}l p{5cm} p{4cm}}
\toprule
\rowcolor{orange!25}
\textbf{زمان} & \textbf{فعالیت} & \textbf{مسئول} \\
\midrule
ماه ۱ & راه‌اندازی داشبورد ملی (نسخه اولیه) & مرکز آمار \\
\rowcolor{gray!10}
ماه ۶ & اولین گزارش شش‌ماهه & دولت \\
سال ۱ & نظرسنجی ملی پایه (Baseline) & سازمان ارزیابی \\
\rowcolor{gray!10}
سال ۱ & تأسیس سازمان ملی ارزیابی & مجلس \\
سال ۲ & ارزیابی فاز گذار (فاز ۱) & پانل مستقل \\
\rowcolor{gray!10}
سال ۳ & اولین گزارش جامع به مردم & دولت + NGOها \\
سال ۵ & ارزیابی میان‌دوره‌ای بزرگ & پانل داخلی-خارجی \\
\rowcolor{gray!10}
سال ۵ & ارزیابی فاز نهادسازی (فاز ۲) & پانل مستقل \\
سال ۱۰ & ارزیابی دهه اول گذار & کمیته ملی + بین‌المللی \\
\rowcolor{gray!10}
سال ۱۵ & ارزیابی تکمیلی گذار & کمیته ملی \\
سال ۲۵ & ارزیابی نهایی و جشن موفقیت & ملی و بین‌المللی \\
\bottomrule
\end{tabular}
\end{table}

%═══════════════════════════════════════════════════════════════════════════════
\section{جمع‌بندی: پایش برای موفقیت}
\label{sec:monitoring-conclusion}
%═══════════════════════════════════════════════════════════════════════════════

\begin{olgoobox}
\textbf{پیام کلیدی فصل}

سیستم پایش و ارزیابی:
\begin{itemize}[nosep]
\item \textbf{چشم برنامه است}: بدون آن، کور حرکت می‌کنیم
\item \textbf{ابزار پاسخگویی است}: مردم باید بدانند چه می‌شود
\item \textbf{موتور یادگیری است}: از اشتباهات درس می‌گیریم
\item \textbf{مبنای اصلاح است}: بدون داده، اصلاح ممکن نیست
\item \textbf{پیش‌شرط شفافیت است}: آنچه پنهان است، فاسد می‌شود
\item \textbf{مشارکتی است}: مردم هم ناظرند، هم ارزیاب
\end{itemize}

\textbf{شعار}: «آنچه اندازه‌گیری نشود، مدیریت نمی‌شود — آنچه شفاف نباشد، فاسد می‌شود»
\end{olgoobox}

\begin{naghlbox}
«در نهایت، موفقیت گذار دموکراتیک با یک معیار ساده سنجیده می‌شود: آیا زندگی مردم عادی بهتر شده است؟ آیا آزادترند؟ آیا امنیت بیشتری دارند؟ آیا به آینده امیدوارترند؟ اگر پاسخ مثبت است، در مسیر درست هستیم.»
\sourceline{نویسنده}
\end{naghlbox}

%═══════════════════════════════════════════════════════════════════════════════
% منابع فصل
%═══════════════════════════════════════════════════════════════════════════════

\section*{منابع فصل پانزدهم}
\addcontentsline{toc}{section}{منابع فصل پانزدهم}

\begin{itemize}[nosep, font=\small]
\item Kusek, J. Z., \& Rist, R. C. (2004). \textit{Ten Steps to a Results-Based Monitoring and Evaluation System}. World Bank.
\item Gertler, P. J. et al. (2016). \textit{Impact Evaluation in Practice}. World Bank.
\item OECD. (2019). \textit{Governance at a Glance}. Paris.
\item World Bank. (2017). \textit{World Development Report: Governance and the Law}. Washington, DC.
\item Transparency International. (2023). \textit{Corruption Perceptions Index Methodology}.
\item Freedom House. (2023). \textit{Freedom in the World Methodology}.
\item Economist Intelligence Unit. (2023). \textit{Democracy Index Methodology}.
\item World Justice Project. (2023). \textit{Rule of Law Index}.
\item UNDP. (2022). \textit{Human Development Report}.
\item Institute for Economics and Peace. (2023). \textit{Global Peace Index}.
\item Reporters Without Borders. (2023). \textit{World Press Freedom Index}.
\item World Economic Forum. (2023). \textit{Global Gender Gap Report}.
\item Drucker, P. F. (1993). \textit{Management: Tasks, Responsibilities, Practices}. Harper Business.
\item Senge, P. M. (1990). \textit{The Fifth Discipline: The Art and Practice of the Learning Organization}. Doubleday.
\end{itemize}