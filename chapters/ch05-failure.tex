%══════════════════════════════════════════════════════════════════════════════
% فصل ۵: درس‌های گذارهای ناموفق
% از بحران تا بالندگی
%══════════════════════════════════════════════════════════════════════════════

\chapter{درس‌های گذارهای ناموفق}
\label{ch:failure}

%──────────────────────────────────────────────────────────────────────────────
% کادر خلاصه فصل
%──────────────────────────────────────────────────────────────────────────────
\begin{kholasebox}
این فصل پنج گذار ناموفق یا ناقص را بررسی می‌کند: عراق (۲۰۰۳-امروز)، لیبی (۲۰۱۱-امروز)، مصر (۲۰۱۱-۲۰۱۳)، یمن (۲۰۱۱-۲۰۱۵) و ونزوئلا (۱۹۹۹-امروز). هر یک به دلایل متفاوتی شکست خوردند: مداخله خارجی نادرست، فروپاشی نهادها، ناتوانی در ائتلاف‌سازی، پوپولیسم ویرانگر، یا ترکیبی از اینها. الگوهای مشترک شکست شامل: فقدان توافق نخبگان، تقسیمات عمیق اجتماعی، شکست اقتصادی، و مداخله خارجی مخرب است. مهم‌ترین درس: آنچه نباید کرد به اندازه آنچه باید کرد، آموزنده است.
\end{kholasebox}

%══════════════════════════════════════════════════════════════════════════════
\section{چرا شکست‌ها را مطالعه کنیم؟}
\label{sec:why-failures}
%══════════════════════════════════════════════════════════════════════════════

\begin{naghlbox}
«تاریخ نه‌تنها از موفقیت‌ها، بلکه از شکست‌ها نیز درس می‌دهد — شاید بیشتر. دانستن آنچه نباید کرد، گاه مهم‌تر از دانستن آنچه باید کرد است.»

\hfill --- جورج سانتایانا
\end{naghlbox}

\begin{table}[H]
\centering
\caption{مقایسه اجمالی پنج گذار ناموفق}
\label{tab:five-failures}
\begin{tabular}{L{2cm} C{2cm} L{3cm} L{3.5cm} C{2cm}}
\toprule
\headmark کشور & \headmark سال & \headmark نوع شکست & \headmark علت اصلی & \headmark وضعیت فعلی \\
\midrule
\rowcolor{rougelight}
عراق & ۲۰۰۳ & دولت شکننده & مداخله خارجی + فرقه‌گرایی & بی‌ثبات \\
\rowcolor{rougelight}
لیبی & ۲۰۱۱ & فروپاشی & فقدان نهاد + قبیله‌گرایی & جنگ داخلی \\
\rowcolor{orroyallight}
مصر & ۲۰۱۱-۱۳ & بازگشت & شکست ائتلاف + اقتصاد & اقتدارگرایی \\
\rowcolor{rougelight}
یمن & ۲۰۱۱ & فروپاشی & گفتگو بدون پشتوانه & جنگ داخلی \\
\rowcolor{orroyallight}
ونزوئلا & ۱۹۹۹ & تحلیل تدریجی & پوپولیسم + نفرین نفت & بحران \\
\bottomrule
\end{tabular}
\end{table}

%══════════════════════════════════════════════════════════════════════════════
\section{عراق: فروپاشی نهادها و فرقه‌گرایی}
\label{sec:iraq}
%══════════════════════════════════════════════════════════════════════════════

\subsection{اشتباهات کلیدی}

\begin{enghelabbox}[title={\hfill \textbf{اشتباه مرگبار: انحلال ارتش و De-Ba'athification}}]
پل برمر، حاکم آمریکایی عراق، دو تصمیم فاجعه‌بار گرفت:

\begin{enumerate}[nosep]
    \item \textbf{انحلال ارتش:} ۴۰۰,۰۰۰ نظامی مسلح و بیکار شدند — بسیاری به داعش پیوستند
    \item \textbf{پاکسازی بعثی‌ها:} ۵۰۰,۰۰۰ کارمند دولتی اخراج شدند — دولت فلج شد
\end{enumerate}

\textbf{درس:} نهادها را نابود نکنید، اصلاح کنید. حتی نهادهای رژیم قبلی بهتر از هیچ‌اند.
\end{enghelabbox}

\begin{figure}[H]
\centering
\begin{tikzpicture}[
    node distance=1.5cm,
    errorbox/.style={
        rectangle,
        rounded corners=3pt,
        minimum width=3.5cm,
        minimum height=1.3cm,
        text centered,
        font=\small,
        draw=rougerevolution,
        fill=rougelight,
        line width=1pt
    },
    resultbox/.style={
        rectangle,
        rounded corners=3pt,
        minimum width=3.5cm,
        minimum height=1.3cm,
        text centered,
        font=\small,
        draw=black,
        fill=grislight,
        line width=1pt
    },
    arrow/.style={->, >=Stealth, thick, color=rougerevolution}
]

% اشتباهات
\node[errorbox] (e1) at (0,3) {انحلال ارتش};
\node[errorbox] (e2) at (4.5,3) {پاکسازی بعثی‌ها};
\node[errorbox] (e3) at (9,3) {تقسیم قدرت فرقه‌ای};

% نتایج میانی
\node[resultbox] (r1) at (0,0) {نظامیان مسلح بیکار};
\node[resultbox] (r2) at (4.5,0) {فلج اداری};
\node[resultbox] (r3) at (9,0) {تعمیق شکاف‌ها};

% نتیجه نهایی
\node[errorbox, minimum width=10cm, minimum height=1.5cm] (final) at (4.5,-3) {
    \textbf{داعش + جنگ داخلی + دولت شکننده}
};

% فلش‌ها
\draw[arrow] (e1) -- (r1);
\draw[arrow] (e2) -- (r2);
\draw[arrow] (e3) -- (r3);
\draw[arrow] (r1) -- (final);
\draw[arrow] (r2) -- (final);
\draw[arrow] (r3) -- (final);

\end{tikzpicture}
\caption{زنجیره علّی شکست عراق}
\label{fig:iraq-chain}
\end{figure}

\subsection{مشکل محاصصه فرقه‌ای}

\begin{table}[H]
\centering
\caption{سیستم محاصصه در عراق}
\label{tab:iraq-muhasasa}
\begin{tabular}{L{3cm} L{2.5cm} L{7cm}}
\toprule
\headmark پست & \headmark سهم & \headmark مشکل \\
\midrule
\rowcolor{bleulight}
رئیس‌جمهور & کُرد & نقش تشریفاتی — کُردها از حکومت مرکزی فاصله می‌گیرند \\
نخست‌وزیر & شیعه & قدرت واقعی اینجاست — تنش دائمی با سُنی‌ها \\
\rowcolor{bleulight}
رئیس پارلمان & سُنی & سُنی‌ها احساس طرد و حاشیه‌نشینی می‌کنند \\
\bottomrule
\end{tabular}
\end{table}

\begin{naghlbox}
«محاصصه فرقه‌ای به جای حل مشکل، آن را نهادینه کرد. سیاستمداران انگیزه داشتند هویت فرقه‌ای را تقویت کنند چون کرسی‌شان به آن وابسته بود.»

\hfill --- تحلیل‌گر سیاسی عراقی
\end{naghlbox}

\subsection{درس‌های عراق برای ما}

\begin{table}[H]
\centering
\caption{چه نباید کرد: درس‌های عراق}
\label{tab:iraq-lessons}
\begin{tabular}{L{4cm} L{4.5cm} L{4cm}}
\toprule
\headmark اشتباه عراق & \headmark چرا اشتباه بود & \headmark جایگزین صحیح \\
\midrule
\rowcolor{rougelight}
انحلال ارتش & نظامیان مسلح بیکار = شورشی & اصلاح تدریجی، بازآموزی \\
\rowcolor{rougelight}
پاکسازی گسترده & فلج اداری & پاکسازی محدود سران \\
\rowcolor{rougelight}
محاصصه فرقه‌ای & نهادینه کردن شکاف‌ها & سهمیه موقت + احزاب فراقومی \\
\rowcolor{rougelight}
مداخله بدون برنامه & فقدان استراتژی خروج & برنامه‌ریزی بلندمدت \\
\bottomrule
\end{tabular}
\end{table}

%══════════════════════════════════════════════════════════════════════════════
\section{لیبی: دولت‌سازی بدون ملت‌سازی}
\label{sec:libya}
%══════════════════════════════════════════════════════════════════════════════

\subsection{میراث قذافی: فقدان نهاد}

\begin{enghelabbox}[title={\hfill \textbf{مشکل بنیادین: کشور بدون دولت}}]
معمر قذافی طی ۴۲ سال حکومت:
\begin{itemize}[nosep]
    \item هیچ نهاد دولتی واقعی نساخت — همه چیز شخصی بود
    \item ارتش را عمداً ضعیف نگه داشت (ترس از کودتا)
    \item جامعه مدنی را سرکوب کرد
    \item بر قبیله‌گرایی تکیه کرد
\end{itemize}

\textbf{نتیجه:} وقتی قذافی سقوط کرد، هیچ چیزی برای ساختن روی آن وجود نداشت.
\end{enghelabbox}

\subsection{تکه‌تکه شدن}

\begin{figure}[H]
\centering
\begin{tikzpicture}[
    node distance=1.5cm,
    fragbox/.style={
        rectangle,
        rounded corners=3pt,
        minimum width=3cm,
        minimum height=1.2cm,
        text centered,
        font=\small,
        line width=1pt
    }
]

% قبل
\node[fragbox, draw=gris, fill=grislight] (before) at (0,0) {
    \begin{tabular}{c}
    لیبی واحد\\
    (تحت قذافی)
    \end{tabular}
};

% بعد - تکه‌ها
\node[fragbox, draw=rougerevolution, fill=rougelight] (f1) at (6,2) {دولت طرابلس};
\node[fragbox, draw=rougerevolution, fill=rougelight] (f2) at (6,0) {دولت بنغازی};
\node[fragbox, draw=rougerevolution, fill=rougelight] (f3) at (6,-2) {میلیشیاهای محلی};
\node[fragbox, draw=orroyal, fill=orroyallight] (f4) at (10,1) {ژنرال حفتر};
\node[fragbox, draw=orroyal, fill=orroyallight] (f5) at (10,-1) {قبایل جنوب};

% فلش‌ها
\draw[->, >=Stealth, very thick, color=rougerevolution] (before) -- node[above, font=\small] {سقوط ۲۰۱۱} (3,0);
\draw[->, >=Stealth, thick, color=gris] (3,0) -- (f1);
\draw[->, >=Stealth, thick, color=gris] (3,0) -- (f2);
\draw[->, >=Stealth, thick, color=gris] (3,0) -- (f3);
\draw[->, >=Stealth, thick, color=gris] (f1) -- (f4);
\draw[->, >=Stealth, thick, color=gris] (f2) -- (f5);

\end{tikzpicture}
\caption{تکه‌تکه شدن لیبی پس از سقوط قذافی}
\label{fig:libya-fragmentation}
\end{figure}

\subsection{درس‌های لیبی}

\begin{table}[H]
\centering
\caption{درس‌های لیبی}
\label{tab:libya-lessons}
\begin{tabular}{L{4cm} L{8.5cm}}
\toprule
\headmark درس & \headmark توضیح \\
\midrule
\rowcolor{rougelight}
نهادها مهم‌اند & بدون نهاد، حتی سقوط دیکتاتور به هرج‌ومرج منجر می‌شود \\
\rowcolor{rougelight}
مداخله ناقص خطرناک است & ناتو قذافی را سرنگون کرد اما برنامه‌ای برای بعد نداشت \\
\rowcolor{rougelight}
سلاح را کنترل کنید & میلیشیاهای مسلح = جنگ داخلی \\
\rowcolor{rougelight}
قبیله‌گرایی مهار شود & هویت‌های پیشامدرن در خلأ نهادی غالب می‌شوند \\
\bottomrule
\end{tabular}
\end{table}

%══════════════════════════════════════════════════════════════════════════════
\section{مصر: شکست ائتلاف و بازگشت نظامیان}
\label{sec:egypt}
%══════════════════════════════════════════════════════════════════════════════

\subsection{چرا انقلاب ۲۰۱۱ شکست خورد؟}

\begin{figure}[H]
\centering
\begin{tikzpicture}[
    node distance=0.5cm,
    eventbox/.style={
        rectangle,
        rounded corners=3pt,
        minimum width=2.5cm,
        minimum height=1.2cm,
        text centered,
        font=\scriptsize,
        line width=1pt
    }
]

% خط زمان
\draw[line width=2pt, color=gris] (0,0) -- (14,0);

% رویدادها
\node[eventbox, draw=vertnapoleon, fill=vertlight, above] at (0,0.5) {
    \begin{tabular}{c}
    ژانویه ۲۰۱۱\\
    انقلاب
    \end{tabular}
};

\node[eventbox, draw=bleurepublique, fill=bleulight, above] at (3.5,0.5) {
    \begin{tabular}{c}
    فوریه ۲۰۱۱\\
    سقوط مبارک
    \end{tabular}
};

\node[eventbox, draw=orroyal, fill=orroyallight, above] at (7,0.5) {
    \begin{tabular}{c}
    ژوئن ۲۰۱۲\\
    مرسی رئیس‌جمهور
    \end{tabular}
};

\node[eventbox, draw=rougerevolution, fill=rougelight, above] at (10.5,0.5) {
    \begin{tabular}{c}
    ژوئیه ۲۰۱۳\\
    کودتای سیسی
    \end{tabular}
};

\node[eventbox, draw=rougerevolution, fill=rougelight, above] at (14,0.5) {
    \begin{tabular}{c}
    ۲۰۱۴-امروز\\
    اقتدارگرایی جدید
    \end{tabular}
};

% نقاط
\foreach \x in {0,3.5,7,10.5,14} {
    \fill[black] (\x,0) circle (4pt);
}

\end{tikzpicture}
\caption{خط زمانی شکست گذار مصر}
\label{fig:egypt-timeline}
\end{figure}

\begin{enghelabbox}[title={\hfill \textbf{سه دلیل اصلی شکست مصر}}]
\begin{enumerate}[nosep]
    \item \textbf{شکست ائتلاف:} اخوان‌المسلمین و لیبرال‌ها نتوانستند همکاری کنند
    \item \textbf{شکست اقتصادی:} معیشت مردم بدتر شد — صف نان طولانی‌تر
    \item \textbf{ناتوانی در حکمرانی:} مرسی تجربه اداره کشور نداشت
\end{enumerate}

\textbf{نتیجه:} وقتی سیسی کودتا کرد، بسیاری از مردم (حتی انقلابیون) استقبال کردند.
\end{enghelabbox}

\begin{naghlbox}
«ما فکر می‌کردیم دشمن ما فقط مبارک است. نفهمیدیم که دشمن واقعی، ناتوانی ما در ساختن ائتلاف و بهبود زندگی مردم بود.»

\hfill --- فعال مصری، ۲۰۱۵
\end{naghlbox}

\subsection{درس کلیدی مصر}

\begin{olgoobox}[title={\hfill \textbf{درس حیاتی برای ما}}]
\textbf{مصر ثابت کرد که سرنگونی دیکتاتور کافی نیست.} شما باید:
\begin{itemize}[nosep]
    \item ائتلاف گسترده بسازید — حتی با کسانی که دوست‌شان ندارید
    \item اقتصاد را مدیریت کنید — مردم نان می‌خواهند نه فقط آزادی
    \item ظرفیت حکمرانی داشته باشید — ایده کافی نیست، باید بتوانید اجرا کنید
    \item ارتش را مدیریت کنید — یا با شما باشد یا بی‌طرف
\end{itemize}
\end{olgoobox}

%══════════════════════════════════════════════════════════════════════════════
\section{یمن: گفتگوی ملی بدون پشتوانه}
\label{sec:yemen}
%══════════════════════════════════════════════════════════════════════════════

\subsection{چرا گفتگوی ملی یمن شکست خورد؟}

یمن پس از ۲۰۱۱ یک «گفتگوی ملی» بسیار جامع برگزار کرد (۲۰۱۳-۲۰۱۴) با:
\begin{itemize}[nosep]
    \item ۵۶۵ نماینده از همه گروه‌ها
    \item ۱۰ ماه مذاکره
    \item توافق بر فدرالیسم ۶ منطقه‌ای
\end{itemize}

\textbf{اما شکست خورد چون:}

\begin{table}[H]
\centering
\caption{دلایل شکست گفتگوی ملی یمن}
\label{tab:yemen-failure}
\begin{tabular}{C{1cm} L{4cm} L{7cm}}
\toprule
\headmark \# & \headmark دلیل & \headmark توضیح \\
\midrule
\rowcolor{rougelight}
۱ & حوثی‌ها بیرون ماندند & گروه مسلح قدرتمند در فرآیند نبود \\
\rowcolor{rougelight}
۲ & پشتوانه نظامی نبود & هیچ نیرویی توافق را تضمین نمی‌کرد \\
\rowcolor{rougelight}
۳ & اقتصاد فروپاشید & مردم گرسنه بودند \\
\rowcolor{rougelight}
۴ & مداخله خارجی & عربستان و ایران وارد شدند \\
\bottomrule
\end{tabular}
\end{table}

\begin{enghelabbox}[title={\hfill \textbf{درس یمن}}]
\textbf{گفتگوی ملی بدون پشتوانه اجرایی، فقط کاغذ است.} شما باید:
\begin{itemize}[nosep]
    \item همه گروه‌های مسلح را پای میز بیاورید
    \item مکانیزم اجرا داشته باشید
    \item مداخله خارجی را مدیریت کنید
    \item اقتصاد را فراموش نکنید
\end{itemize}
\end{enghelabbox}

%══════════════════════════════════════════════════════════════════════════════
\section{ونزوئلا: پوپولیسم و نفرین منابع}
\label{sec:venezuela}
%══════════════════════════════════════════════════════════════════════════════

\subsection{مسیر فروپاشی}

ونزوئلا یک دموکراسی نسبتاً پایدار بود که از درون فروپاشید:

\begin{figure}[H]
\centering
\begin{tikzpicture}[
    node distance=1.5cm,
    stagebox/.style={
        rectangle,
        rounded corners=3pt,
        minimum width=3.5cm,
        minimum height=1.5cm,
        text centered,
        font=\small,
        line width=1pt
    },
    arrow/.style={->, >=Stealth, thick}
]

% مراحل
\node[stagebox, draw=bleurepublique, fill=bleulight] (s1) at (0,0) {
    \begin{tabular}{c}
    ۱۹۵۸-۱۹۹۸\\
    دموکراسی نفتی
    \end{tabular}
};

\node[stagebox, draw=orroyal, fill=orroyallight] (s2) at (5,0) {
    \begin{tabular}{c}
    ۱۹۹۹-۲۰۱۳\\
    چاوز: پوپولیسم
    \end{tabular}
};

\node[stagebox, draw=rougerevolution, fill=rougelight] (s3) at (10,0) {
    \begin{tabular}{c}
    ۲۰۱۳-امروز\\
    مادورو: فروپاشی
    \end{tabular}
};

% فلش‌ها
\draw[arrow, color=orroyal] (s1) -- node[above, font=\scriptsize] {فساد + نابرابری} (s2);
\draw[arrow, color=rougerevolution] (s2) -- node[above, font=\scriptsize] {سقوط قیمت نفت} (s3);

% شاخص‌ها زیر
\node[font=\scriptsize, text width=3cm, align=center, below=0.5cm of s1] {
    GDP بالا\\
    نابرابری بالا
};
\node[font=\scriptsize, text width=3cm, align=center, below=0.5cm of s2] {
    کاهش فقر\\
    تخریب نهادها
};
\node[font=\scriptsize, text width=3cm, align=center, below=0.5cm of s3] {
    تورم ۱,۰۰۰,۰۰۰٪\\
    مهاجرت ۷ میلیون
};

\end{tikzpicture}
\caption{مسیر فروپاشی ونزوئلا}
\label{fig:venezuela-path}
\end{figure}

\begin{enghelabbox}[title={\hfill \textbf{دام ونزوئلایی}}]
چاوز با درآمد نفت:
\begin{itemize}[nosep]
    \item برنامه‌های اجتماعی گسترده اجرا کرد — فقر کاهش یافت
    \item اما نهادها را تخریب کرد — قوه قضائیه، بانک مرکزی، رسانه
    \item اقتصاد را متنوع نکرد — ۹۵٪ صادرات = نفت
    \item وقتی قیمت نفت سقوط کرد، همه چیز فروپاشید
\end{itemize}

\textbf{درس:} پوپولیسم نفتی می‌تواند کوتاه‌مدت محبوب باشد، اما بدون نهادسازی و تنوع اقتصادی، فاجعه‌بار است.
\end{enghelabbox}

%══════════════════════════════════════════════════════════════════════════════
\section{الگوهای مشترک شکست}
\label{sec:failure-patterns}
%══════════════════════════════════════════════════════════════════════════════

\begin{figure}[H]
\centering
\begin{tikzpicture}[
    node distance=1.5cm,
    factor/.style={
        rectangle,
        rounded corners=5pt,
        minimum width=3.5cm,
        minimum height=1.5cm,
        text centered,
        font=\small,
        draw=rougerevolution,
        fill=rougelight,
        line width=1.5pt
    },
    center/.style={
        circle,
        minimum size=3cm,
        text centered,
        font=\small\bfseries,
        draw=black,
        fill=grislight,
        line width=2pt
    }
]

% مرکز
\node[center] (c) at (0,0) {
    \begin{tabular}{c}
    گذار\\
    ناموفق
    \end{tabular}
};

% عوامل
\node[factor] (f1) at (90:4.5cm) {
    \begin{tabular}{c}
    فقدان توافق\\
    {\scriptsize نخبگان متفرق}
    \end{tabular}
};

\node[factor] (f2) at (30:4.5cm) {
    \begin{tabular}{c}
    شکست اقتصادی\\
    {\scriptsize معیشت بدتر}
    \end{tabular}
};

\node[factor] (f3) at (330:4.5cm) {
    \begin{tabular}{c}
    مداخله خارجی مخرب\\
    {\scriptsize بدون برنامه}
    \end{tabular}
};

\node[factor] (f4) at (270:4.5cm) {
    \begin{tabular}{c}
    فروپاشی نهادها\\
    {\scriptsize بدون جایگزین}
    \end{tabular}
};

\node[factor] (f5) at (210:4.5cm) {
    \begin{tabular}{c}
    تقسیمات عمیق\\
    {\scriptsize قومی/فرقه‌ای}
    \end{tabular}
};

\node[factor] (f6) at (150:4.5cm) {
    \begin{tabular}{c}
    گروه‌های مسلح\\
    {\scriptsize خارج از کنترل}
    \end{tabular}
};

% اتصالات
\foreach \f in {f1,f2,f3,f4,f5,f6} {
    \draw[->, >=Stealth, thick, color=rougerevolution] (\f) -- (c);
}

\end{tikzpicture}
\caption{شش عامل مشترک گذارهای ناموفق}
\label{fig:failure-factors}
\end{figure}

%══════════════════════════════════════════════════════════════════════════════
\section{چک‌لیست: آنچه نباید کرد}
\label{sec:donts-checklist}
%══════════════════════════════════════════════════════════════════════════════

\begin{table}[H]
\centering
\caption{چک‌لیست جامع اشتباهات قابل اجتناب}
\label{tab:donts-checklist}
\begin{tabular}{C{1cm} L{5cm} L{3.5cm} L{3.5cm}}
\toprule
\headmark \# & \headmark اشتباه & \headmark نمونه & \headmark جایگزین \\
\midrule
\rowcolor{rougelight}
۱ & انحلال نهادهای موجود & عراق & اصلاح تدریجی \\
\rowcolor{rougelight}
۲ & پاکسازی گسترده & عراق & پاکسازی محدود سران \\
\rowcolor{rougelight}
۳ & نهادینه کردن شکاف‌ها & عراق، لبنان & سهمیه موقت + فراگیری \\
\rowcolor{rougelight}
۴ & فراموش کردن اقتصاد & مصر، یمن & بهبود معیشت فوری \\
\rowcolor{rougelight}
۵ & طرد گروه‌های مسلح & یمن & آوردن همه پای میز \\
\rowcolor{rougelight}
۶ & مداخله بدون برنامه & لیبی & برنامه بلندمدت \\
\rowcolor{rougelight}
۷ & پوپولیسم نفتی & ونزوئلا & نهادسازی + تنوع اقتصادی \\
\rowcolor{rougelight}
۸ & تخریب نهادهای نظارتی & ونزوئلا & تقویت نظارت \\
\rowcolor{rougelight}
۹ & انتظار نتیجه فوری & همه & صبر استراتژیک \\
\rowcolor{rougelight}
۱۰ & تقلید کورکورانه & همه & بومی‌سازی خلاقانه \\
\bottomrule
\end{tabular}
\end{table}

%══════════════════════════════════════════════════════════════════════════════
\section*{منابع فصل}
%══════════════════════════════════════════════════════════════════════════════

\begin{enumerate}[nosep, label={[\arabic*]}]
    \item Dodge, T. (2012). \textit{Iraq: From War to a New Authoritarianism}. Routledge.
    
    \item Wehrey, F. (2018). \textit{The Burning Shores: Inside the Battle for the New Libya}. FSG.
    
    \item Kandil, H. (2015). \textit{Inside the Brotherhood}. Polity Press.
    
    \item Lackner, H. (2017). \textit{Yemen in Crisis}. Saqi Books.
    
    \item Corrales, J. \& Penfold, M. (2011). \textit{Dragon in the Tropics: Hugo Chavez}. Brookings.
    
    \item Diamond, L. (2015). "Facing Up to the Democratic Recession." \textit{Journal of Democracy}, 26(1).
    
    \item Brownlee, J. et al. (2015). \textit{The Arab Spring: Pathways of Repression and Reform}. Oxford.
    
    \item Brancati, D. (2016). \textit{Democracy Protests}. Cambridge University Press.
\end{enumerate}