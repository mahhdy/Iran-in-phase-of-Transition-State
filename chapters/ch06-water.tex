%══════════════════════════════════════════════════════════════════════════════
% فصل ۶: درس‌های مدیریت بحران آب
% از بحران تا بالندگی
%══════════════════════════════════════════════════════════════════════════════

\chapter{درس‌های مدیریت بحران آب}
\label{ch:water}

%──────────────────────────────────────────────────────────────────────────────
% کادر خلاصه فصل
%──────────────────────────────────────────────────────────────────────────────
\begin{kholasebox}
این فصل تجارب موفق جهانی در مدیریت بحران آب را بررسی می‌کند: اسرائیل (از کمبود به صادرات فناوری)، سنگاپور (چهار شیر ملی)، استرالیا (مدیریت خشکسالی هزاره)، و اسپانیا (انتقال آب و تنش‌ها). درس‌های کلیدی شامل: مدیریت تقاضا مهم‌تر از افزایش عرضه است؛ قیمت‌گذاری واقعی ضروری است؛ فناوری‌های جدید (شیرین‌سازی، بازچرخانی) تحول‌آفرین‌اند؛ و حکمرانی آب باید یکپارچه باشد. برای کشور ما که با فروپاشی سفره‌های زیرزمینی مواجه است، این تجارب راهگشایند.
\end{kholasebox}

%══════════════════════════════════════════════════════════════════════════════
\section{چرا مدیریت آب حیاتی است؟}
\label{sec:water-importance}
%══════════════════════════════════════════════════════════════════════════════

\begin{enghelabbox}[title={\hfill \textbf{یادآوری: بحران آب ما}}]
همان‌طور که در فصل ۲ دیدیم:
\begin{itemize}[nosep]
    \item ۷۰٪ سفره‌های زیرزمینی در وضعیت بحرانی یا ممنوعه
    \item برداشت مازاد سالانه: ۷+ میلیارد مترمکعب
    \item ۹۰٪ مصرف آب: کشاورزی (اغلب ناکارآمد)
    \item بدون اقدام: ۳۰-۵۰ میلیون آواره اقلیمی تا ۲۰۴۰
\end{itemize}

این فصل نشان می‌دهد که \textbf{راه‌حل وجود دارد} — اگر اراده سیاسی باشد.
\end{enghelabbox}

%══════════════════════════════════════════════════════════════════════════════
\section{اسرائیل: از کمبود به فراوانی}
\label{sec:israel-water}
%══════════════════════════════════════════════════════════════════════════════

\subsection{معجزه آب اسرائیل}

اسرائیل کشوری است که ۶۰٪ آن بیابان است، اما امروز:
\begin{itemize}[nosep]
    \item صادرکننده محصولات کشاورزی است
    \item صادرکننده فناوری آب به ۱۵۰ کشور است
    \item مازاد آب دارد و به همسایگان می‌فروشد
\end{itemize}

\begin{figure}[H]
\centering
\begin{tikzpicture}
\begin{axis}[
    width=13cm,
    height=7cm,
    xlabel={سال},
    ylabel={میلیون مترمکعب در سال},
    xmin=1960, xmax=2025,
    ymin=0, ymax=2500,
    legend style={at={(0.02,0.98)}, anchor=north west},
    grid=both,
    grid style={line width=0.2pt, draw=gray!30}
]

% منابع سنتی
\addplot[color=bleurepublique, very thick, mark=*] coordinates {
    (1960,1400) (1970,1500) (1980,1600) (1990,1650) (2000,1600) (2010,1500) (2020,1400)
};

% شیرین‌سازی
\addplot[color=vertnapoleon, very thick, mark=square*] coordinates {
    (1960,0) (1970,0) (1980,0) (1990,0) (2000,50) (2010,300) (2020,700)
};

% بازچرخانی
\addplot[color=orroyal, very thick, mark=triangle*] coordinates {
    (1960,0) (1970,50) (1980,100) (1990,200) (2000,300) (2010,450) (2020,600)
};

\legend{منابع طبیعی, شیرین‌سازی, بازچرخانی فاضلاب}

\end{axis}
\end{tikzpicture}
\caption{تحول ترکیب منابع آب اسرائیل}
\label{fig:israel-water-sources}
\end{figure}

\subsection{چهار ستون موفقیت اسرائیل}

\begin{olgoobox}[title={\hfill \textbf{الگوی اسرائیل}}]
\begin{enumerate}[nosep]
    \item \textbf{آبیاری قطره‌ای:} اختراع اسرائیلی — ۹۰٪ صرفه‌جویی در آب کشاورزی
    \item \textbf{شیرین‌سازی:} ۵ کارخانه بزرگ — ۷۰٪ آب شرب
    \item \textbf{بازچرخانی فاضلاب:} ۸۷٪ فاضلاب بازچرخانی می‌شود (رتبه اول جهان)
    \item \textbf{قیمت‌گذاری واقعی:} آب ارزان نیست — انگیزه صرفه‌جویی
\end{enumerate}
\end{olgoobox}

\begin{table}[H]
\centering
\caption{مقایسه شاخص‌های آب: اسرائیل و کشور ما}
\label{tab:israel-comparison}
\begin{tabular}{L{5cm} C{3cm} C{3cm}}
\toprule
\headmark شاخص & \headmark اسرائیل & \headmark کشور ما \\
\midrule
\rowcolor{vertlight}
بارش سالانه (میلیمتر) & ۵۰۰ & ۲۵۰ \\
\rowcolor{rougelight}
بازچرخانی فاضلاب & ۸۷٪ & کمتر از ۱۰٪ \\
\rowcolor{vertlight}
آبیاری قطره‌ای/بارانی & ۷۵٪ اراضی & کمتر از ۱۵٪ \\
\rowcolor{rougelight}
اتلاف در شبکه توزیع & ۱۰٪ & ۳۰٪+ \\
\rowcolor{vertlight}
شیرین‌سازی & ۷۰٪ آب شرب & نزدیک صفر \\
\bottomrule
\end{tabular}
\end{table}

\subsection{درس‌های اسرائیل}

\begin{table}[H]
\centering
\caption{درس‌های قابل انتقال از اسرائیل}
\label{tab:israel-lessons}
\begin{tabular}{L{3.5cm} L{4.5cm} L{4.5cm}}
\toprule
\headmark درس & \headmark کاربرد & \headmark چالش اجرا \\
\midrule
\rowcolor{vertlight}
آبیاری قطره‌ای & تغییر الگوی آبیاری & سرمایه اولیه بالا \\
\rowcolor{vertlight}
شیرین‌سازی & مناطق ساحلی جنوب & هزینه انرژی \\
\rowcolor{vertlight}
بازچرخانی فاضلاب & همه شهرهای بزرگ & زیرساخت + پذیرش فرهنگی \\
\rowcolor{bleulight}
قیمت‌گذاری واقعی & کاهش مصرف ناکارآمد & مقاومت سیاسی \\
\bottomrule
\end{tabular}
\end{table}

%══════════════════════════════════════════════════════════════════════════════
\section{سنگاپور: چهار شیر ملی}
\label{sec:singapore-water}
%══════════════════════════════════════════════════════════════════════════════

\subsection{چالش سنگاپور}

سنگاپور جزیره کوچکی است با:
\begin{itemize}[nosep]
    \item ۷۲۰ کیلومتر مربع مساحت
    \item ۵.۵ میلیون جمعیت
    \item بدون رودخانه یا دریاچه مهم
    \item تا ۱۹۶۵: ۱۰۰٪ وابسته به واردات آب از مالزی
\end{itemize}

\subsection{استراتژی چهار شیر ملی}

\begin{figure}[H]
\centering
\begin{tikzpicture}[
    node distance=2cm,
    tapbox/.style={
        rectangle,
        rounded corners=5pt,
        minimum width=3.5cm,
        minimum height=2cm,
        text centered,
        font=\small,
        line width=1.5pt
    }
]

% چهار شیر
\node[tapbox, draw=bleurepublique, fill=bleulight] (t1) at (0,0) {
    \begin{tabular}{c}
    \textbf{شیر ۱}\\[3pt]
    آب وارداتی\\
    {\scriptsize از مالزی}\\[3pt]
    {\footnotesize ۴۰٪}
    \end{tabular}
};

\node[tapbox, draw=vertnapoleon, fill=vertlight] (t2) at (5,0) {
    \begin{tabular}{c}
    \textbf{شیر ۲}\\[3pt]
    جمع‌آوری باران\\
    {\scriptsize ۱۷ مخزن}\\[3pt]
    {\footnotesize ۲۰٪}
    \end{tabular}
};

\node[tapbox, draw=violetempire, fill=violetlight] (t3) at (10,0) {
    \begin{tabular}{c}
    \textbf{شیر ۳}\\[3pt]
    NEWater\\
    {\scriptsize بازچرخانی}\\[3pt]
    {\footnotesize ۳۰٪}
    \end{tabular}
};

\node[tapbox, draw=orroyal, fill=orroyallight] (t4) at (15,0) {
    \begin{tabular}{c}
    \textbf{شیر ۴}\\[3pt]
    شیرین‌سازی\\
    {\scriptsize دریا}\\[3pt]
    {\footnotesize ۱۰٪}
    \end{tabular}
};

% هدف آینده
\node[font=\small, text width=12cm, align=center] at (7.5,-2.5) {
    \textbf{هدف ۲۰۶۰:} کاهش وابستگی به مالزی به صفر\\
    NEWater: ۵۵٪ | شیرین‌سازی: ۲۵٪ | جمع‌آوری باران: ۲۰٪
};

\end{tikzpicture}
\caption{استراتژی چهار شیر ملی سنگاپور}
\label{fig:singapore-taps}
\end{figure}

\begin{olgoobox}[title={\hfill \textbf{NEWater: انقلاب بازچرخانی}}]
NEWater فاضلاب تصفیه‌شده با فناوری پیشرفته است:
\begin{itemize}[nosep]
    \item سه مرحله تصفیه: میکروفیلتراسیون + اسمز معکوس + UV
    \item کیفیت بالاتر از استانداردهای آب آشامیدنی WHO
    \item مصرف: صنعت (wafer fabrication) + مخلوط با آب آشامیدنی
    \item پذیرش عمومی: کمپین آموزشی گسترده + بازدید از کارخانه
\end{itemize}

\textbf{درس:} با فناوری و آموزش، حتی بازچرخانی فاضلاب قابل پذیرش است.
\end{olgoobox}

\subsection{درس‌های سنگاپور}

\begin{table}[H]
\centering
\caption{درس‌های قابل انتقال از سنگاپور}
\label{tab:singapore-lessons}
\begin{tabular}{L{4cm} L{8.5cm}}
\toprule
\headmark درس & \headmark توضیح \\
\midrule
\rowcolor{vertlight}
تنوع منابع & وابستگی به یک منبع = آسیب‌پذیری \\
\rowcolor{vertlight}
بازچرخانی پیشرفته & با فناوری، فاضلاب به آب آشامیدنی تبدیل می‌شود \\
\rowcolor{vertlight}
آموزش عمومی & پذیرش فرهنگی به اندازه فناوری مهم است \\
\rowcolor{vertlight}
افق بلندمدت & برنامه‌ریزی ۵۰ ساله — نه فقط برای انتخابات بعدی \\
\bottomrule
\end{tabular}
\end{table}

%══════════════════════════════════════════════════════════════════════════════
\section{استرالیا: مدیریت خشکسالی هزاره}
\label{sec:australia-water}
%══════════════════════════════════════════════════════════════════════════════

\subsection{بحران Murray-Darling}

حوضه Murray-Darling سبد غذایی استرالیا است، اما خشکسالی ۱۹۹۷-۲۰۱۰ آن را به بحران کشاند:
\begin{itemize}[nosep]
    \item کاهش ۷۰٪ جریان رودخانه
    \item ورشکستگی هزاران کشاورز
    \item مرگ اکوسیستم‌های آبی
\end{itemize}

\subsection{راه‌حل: بازار آب}

\begin{olgoobox}[title={\hfill \textbf{الگوی استرالیا: حقوق قابل معامله آب}}]
استرالیا «حقوق آب» را از «زمین» جدا کرد:
\begin{itemize}[nosep]
    \item کشاورزان سهمی از آب دارند (نه آب نامحدود)
    \item می‌توانند این سهم را بفروشند یا بخرند
    \item قیمت بازاری تعیین می‌شود — کمیابی منعکس می‌شود
    \item کسانی که آب را کارآمدتر مصرف می‌کنند، سود می‌برند
\end{itemize}

\textbf{نتیجه:} آب به سمت مصارف با ارزش بالاتر رفت. صرفه‌جویی ۳۰٪.
\end{olgoobox}

\begin{figure}[H]
\centering
\begin{tikzpicture}[
    node distance=1.5cm,
    box/.style={
        rectangle,
        rounded corners=3pt,
        minimum width=4cm,
        minimum height=1.5cm,
        text centered,
        font=\small,
        line width=1pt
    },
    arrow/.style={->, >=Stealth, thick}
]

% قبل
\node[box, draw=rougerevolution, fill=rougelight] (b1) at (0,0) {
    \begin{tabular}{c}
    \textbf{قبل}\\
    آب = حق نامحدود\\
    هدررفت گسترده
    \end{tabular}
};

% بعد
\node[box, draw=vertnapoleon, fill=vertlight] (b2) at (8,0) {
    \begin{tabular}{c}
    \textbf{بعد}\\
    آب = دارایی قابل معامله\\
    مصرف کارآمد
    \end{tabular}
};

% فلش
\draw[arrow, very thick, color=bleurepublique] (b1) -- node[above, font=\small] {اصلاحات ۲۰۰۴-۲۰۰۷} (b2);

\end{tikzpicture}
\caption{تحول مدل حکمرانی آب در استرالیا}
\label{fig:australia-reform}
\end{figure}

\subsection{درس‌های استرالیا}

\begin{table}[H]
\centering
\caption{درس‌های قابل انتقال از استرالیا}
\label{tab:australia-lessons}
\begin{tabular}{L{4cm} L{4.5cm} L{4cm}}
\toprule
\headmark درس & \headmark کاربرد & \headmark چالش \\
\midrule
\rowcolor{vertlight}
بازار آب & تخصیص کارآمد & نیاز به حقوق مالکیت روشن \\
\rowcolor{vertlight}
سقف برداشت & جلوگیری از تخلیه بیش از حد & مقاومت کشاورزان \\
\rowcolor{bleulight}
جبران کشاورزان & خرید حقوق آب توسط دولت & هزینه بالا \\
\rowcolor{bleulight}
نظارت دقیق & کنتورهای هوشمند & زیرساخت \\
\bottomrule
\end{tabular}
\end{table}

%══════════════════════════════════════════════════════════════════════════════
\section{جمع‌بندی: اصول جهانی مدیریت پایدار آب}
\label{sec:water-principles}
%══════════════════════════════════════════════════════════════════════════════

\begin{figure}[H]
\centering
\begin{tikzpicture}[
    node distance=1.5cm,
    principle/.style={
        rectangle,
        rounded corners=5pt,
        minimum width=4cm,
        minimum height=1.8cm,
        text centered,
        font=\small,
        draw=bleurepublique,
        fill=bleulight,
        line width=1.5pt
    }
]

% اصول
\node[principle] (p1) at (0,3) {
    \begin{tabular}{c}
    \textbf{۱. مدیریت تقاضا}\\
    {\scriptsize مهم‌تر از افزایش عرضه}
    \end{tabular}
};

\node[principle] (p2) at (6,3) {
    \begin{tabular}{c}
    \textbf{۲. قیمت‌گذاری واقعی}\\
    {\scriptsize آب رایگان = هدررفت}
    \end{tabular}
};

\node[principle] (p3) at (12,3) {
    \begin{tabular}{c}
    \textbf{۳. تنوع منابع}\\
    {\scriptsize وابستگی به یک منبع = ریسک}
    \end{tabular}
};

\node[principle] (p4) at (0,0) {
    \begin{tabular}{c}
    \textbf{۴. فناوری}\\
    {\scriptsize شیرین‌سازی + بازچرخانی}
    \end{tabular}
};

\node[principle] (p5) at (6,0) {
    \begin{tabular}{c}
    \textbf{۵. حکمرانی یکپارچه}\\
    {\scriptsize یک نهاد مسئول}
    \end{tabular}
};

\node[principle] (p6) at (12,0) {
    \begin{tabular}{c}
    \textbf{۶. مشارکت محلی}\\
    {\scriptsize ذینفعان در تصمیم‌گیری}
    \end{tabular}
};

% مرکز
\node[circle, minimum size=2.5cm, draw=vertnapoleon, fill=vertlight, line width=2pt] at (6,-3) {
    \begin{tabular}{c}
    \textbf{آب}\\
    \textbf{پایدار}
    \end{tabular}
};

% اتصالات
\foreach \p in {p1,p2,p3,p4,p5,p6} {
    \draw[->, >=Stealth, thick, color=vertnapoleon] (\p) -- (6,-1.5);
}

\end{tikzpicture}
\caption{شش اصل جهانی مدیریت پایدار آب}
\label{fig:water-principles}
\end{figure}

%══════════════════════════════════════════════════════════════════════════════
\section{کاربرد برای کشور ما}
\label{sec:water-application}
%══════════════════════════════════════════════════════════════════════════════

\begin{table}[H]
\centering
\caption{برنامه پیشنهادی مدیریت آب بر اساس تجارب جهانی}
\label{tab:water-plan}
\small
\begin{tabular}{L{3cm} L{4cm} L{3.5cm} C{2cm}}
\toprule
\headmark اقدام & \headmark الگوی جهانی & \headmark اولویت مکانی & \headmark فاز \\
\midrule
\rowcolor{bleulight}
کاهش اتلاف شبکه & همه کشورها & شهرهای بزرگ & ۱ \\
تغییر الگوی کشت & اسرائیل، استرالیا & مناطق بحرانی & ۱-۲ \\
\rowcolor{bleulight}
آبیاری قطره‌ای & اسرائیل & دشت‌های مرکزی & ۱-۳ \\
بازچرخانی فاضلاب & سنگاپور، اسرائیل & شهرهای بزرگ & ۲-۳ \\
\rowcolor{bleulight}
شیرین‌سازی & اسرائیل، سنگاپور & ساحل جنوب & ۲-۴ \\
بازار حقوق آب & استرالیا & حوضه‌های بحرانی & ۳-۴ \\
\bottomrule
\end{tabular}
\end{table}

\begin{tahlilbox}[title={\hfill \textbf{نتیجه‌گیری}}]
بحران آب ما قابل حل است — اگر:
\begin{enumerate}[nosep]
    \item \textbf{اراده سیاسی} باشد: تصمیمات سخت گرفته شود
    \item \textbf{مدیریت تقاضا} اولویت باشد: نه فقط سد و انتقال
    \item \textbf{قیمت‌گذاری} واقعی شود: یارانه هدفمند به فقرا
    \item \textbf{فناوری} وارد شود: بازچرخانی و شیرین‌سازی
    \item \textbf{کشاورزی} متحول شود: تغییر الگوی کشت و آبیاری
    \item \textbf{حکمرانی} یکپارچه باشد: یک نهاد مسئول
\end{enumerate}

کشورهایی با شرایط بدتر از ما (اسرائیل، سنگاپور) موفق شدند. ما هم می‌توانیم.
\end{tahlilbox}

%══════════════════════════════════════════════════════════════════════════════
\section*{منابع فصل}
%══════════════════════════════════════════════════════════════════════════════

\begin{enumerate}[nosep, label={[\arabic*]}]
    \item Siegel, S. (2015). \textit{Let There Be Water: Israel's Solution for a Water-Starved World}. Thomas Dunne.
    
    \item Tortajada, C. et al. (2013). \textit{The Singapore Water Story}. Routledge.
    
    \item Grafton, R.Q. et al. (2011). "Determinants of Residential Water Consumption." \textit{Land Economics}, 87(4).
    
    \item World Bank. (2017). \textit{Water Scarce Cities: Thriving in a Finite World}. WB Publications.
    
    \item FAO. (2020). \textit{The State of Food and Agriculture: Water}. Rome: FAO.
    
    \item Gleick, P. (2014). \textit{The World's Water Volume 8}. Island Press.
    
    \item OECD. (2015). \textit{Water and Cities: Ensuring Sustainable Futures}. OECD Publishing.
    
    \item PUB Singapore. (2023). \textit{Our Water, Our Future}. pub.gov.sg.
\end{enumerate}